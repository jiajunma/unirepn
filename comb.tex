\documentclass[ssunip]{subfiles}

\begin{document}
 \section{The descents of painted bipartitions}\label{sec:comb}

As before, let  $\star\in \{ B, C,  D, \widetilde{C},  C^*, D^*\}$ and let $\check \CO$ be a Young diagram that has $\star$-good parity. Put
\begin{equation}\label{lstarco}
  l:=l_{\star, \check \CO}:=\begin{cases}
 \frac{\bfrr_1(\ckcO)}{2}; & \quad \textrm{if } \star\in \{B, \widetilde C\};\smallskip\\
 \frac{\bfrr_1(\ckcO)-1}{2}, &\quad \textrm{if } \star\in \{C, C^* \};\smallskip\\
 \frac{\bfrr_1(\ckcO)+1}{2}, &\quad \textrm{if } \star\in \{ D, D^*\}.\\
\end{cases}
\end{equation}
This is the length of the leading column of every element of $\mathrm{PBP}_\star(\check \CO)$. 

 
 In various context, we use $\emptyset$ to denote the empty set, the empty Young diagram or the painted Young diagram whose underlying Young diagram is empty. For every Young diagram $\imath$, its descent, which is denoted by $\nabla(\jmath)$, is defined to be the Young diagram obtained from $\jmath$ by removing the first column. By convention, $\nabla(\emptyset)=\emptyset$. 
 
 In the rest of this section, we assume that $\check \CO\neq \emptyset$, and write $\check \CO'$ for its dual descent. Write $\star'$ for the Howe dual of $\star$ so that $\check \CO'$ has $\star'$-good parity. Put
\[
l':=l_{\star', \check \CO'}
\]
    
 \subsection{Naive descents of painted bipartitions }
\def\bipartl{\mathrm{bi\cP_L}}
\def\bipartr{\mathrm{bi\cP_R}}
\def\dsdiagl{\mathrm{DS_L}}
\def\dsdiagr{\mathrm{DS_R}}
\def\DDl{\eDD_\mathrm{L}}
\def\DDr{\eDD_\mathrm{R}}


In this subsection, let $\tau=(\imath,\cP)\times (\jmath,\cQ)\times \alpha$ be a  painted bipartition such that $\star_\tau=\star$. Write $\star'$ for the Howe dual of $\star$ and put
  \begin{equation} \label{eq:def.alphap}
    \alpha'=\begin{cases} B^+,
  & \textrm{if $\alpha=\widetilde{C}$ and $c$ does not occur in the leading column of $\tau$}; \smallskip \\
  B^-,
  & \textrm{if $\alpha=\widetilde{C}$ and  $c$ occurs in the leading column of $\tau$}; \smallskip \\
  \star', & \textrm{if $\alpha\neq \widetilde C$}. 
  \end{cases}
  \end{equation}
\begin{lem}\label{lemDDn1}
  If $\star \in \set{B,C,C^*}$, then there is a unique painted bipartition of the form $\tau'= (\imath',\cP')\times (\jmath',\cQ')\times \alpha'$ with the following properties:
  \begin{itemize}
        \item $
   (\imath',\jmath')= (\imath,\DD(\jmath)); \smallskip
   $
   \item for all $(i,j)\in \BOX(\imath')$,
   \[
     \cP'(i,j)=\begin{cases}   
    \bullet \textrm{ or } s,&\textrm{ if  $\ \cP(i,j)\in \{\bullet, s\}$;} \smallskip \\
  \cP(i,j),& \textrm{ if $\ \cP(i,j)\notin \{\bullet, s\}$};\end{cases}
   \]
   \item for all $(i,j)\in \BOX(\jmath')$,
   \[
     \cQ'(i,j)=\begin{cases}   
    \bullet \textrm{ or } s,&\textrm{ if  $\ \cQ(i,j+1)\in \{\bullet, s\}$;} \smallskip \\
  \cQ(i,j+1), & \textrm{ if $\ \cQ(i,j+1)\notin \{\bullet, s\}$}.  \end{cases}
   \]
    \end{itemize} 
    \end{lem}
    
    


   \begin{proof}
    First assume that the images of $\cP$ and $\cQ$ are both contained in $\{\bullet, s\}$. Then  the image of $\cP$  is in fact contained in $\{\bullet\}$, and $(\imath, \jmath)$ is  right interlaced in the sense that 
 \[
 \mathbf{c}_1(\jmath)\geq \mathbf{c}_1(\imath)\geq \mathbf{c}_2(\jmath)\geq \mathbf{c}_2(\imath)\geq \mathbf{c}_3(\jmath)\geq \mathbf{c}_3(\imath) \geq \cdots.
 \]
 Hence $ (\imath',\jmath'):= (\imath,\DD(\jmath))$ is left interlaced in the sense that 
 \[
 \mathbf{c}_1(\imath')\geq \mathbf{c}_1(\jmath')\geq \mathbf{c}_2(\imath')\geq \mathbf{c}_2(\jmath')\geq \mathbf{c}_3(\imath')\geq \mathbf{c}_3(\jmath') \geq \cdots.
 \]
 Then it is clear that there is  unique painted bipartition of the form  $\tau'=(\imath',\cP')\times (\jmath',\cQ')\times \alpha'$ such that images of $\cP'$ and $\cQ'$ are both contained in $\{\bullet, s\}$. This proves the lemma in the special case when the images of $\cP$ and $\cQ$ are both contained in $\{\bullet, s\}$. 
 
 The proof of the lemma in the general case is easily reduced to this special case. 
   \end{proof}
    \begin{lem}\label{lemDDn2}
    If $\star \in \set{ \widetilde C, D,D^*}$, then there is a unique painted bipartition of the form $\tau'= (\imath',\cP')\times (\jmath',\cQ')\times \alpha'$ with the following properties:
  \begin{itemize}
        \item $
   (\imath',\jmath')= (\DD(\imath),\jmath); \smallskip
   $
   \item for all $(i,j)\in \BOX(\imath')$,
   \[
     \cP'(i,j)=\begin{cases}   
    \bullet \textrm{ or } s,&\textrm{ if  $\ \cP(i,j+1)\in \{\bullet, s\}$;} \smallskip \\
  \cP(i,j+1),& \textrm{ if $\ \cP(i,j+1)\notin \{\bullet, s\}$};\end{cases}
   \]
   \item for all $(i,j)\in \BOX(\jmath')$,
   \[
     \cQ'(i,j)=\begin{cases}   
    \bullet \textrm{ or } s,&\textrm{ if  $\ \cP(i,j)\in \{\bullet, s\}$;} \smallskip \\
  \cQ(i,j), & \textrm{ if $\ \cQ(i,j)\notin \{\bullet, s\}$}.  \end{cases}
   \]
  
    \end{itemize}
\end{lem}
\begin{proof}
  The proof is similar to that of \Cref{lemDDn1}. 
  
\end{proof}

 In the notation of \Cref{lemDDn1,lemDDn2}, we call $\tau'$ the naive descent of $\tau$, to be denoted by $\DDn(\tau)$.  
 

  

  
 \begin{Example} If
    \[
     \tau = \ytb{\bullet\bullet\bullet {c},\bullet {s} {c},{s},{c}}
    \times \ytb{\bullet\bullet\bullet ,\bullet {r} {d},{d}{d}, \none}
    \times \widetilde C, \]
   then 
   \[
    \nabla_{\mathrm{naive}}(\tau) =\ytb{\bullet\bullet{c} ,\bullet{c},\none }
    \times  \ytb{\bullet\bullet {s} ,\bullet {r} {d},{d}{d}}\times B^-.
    \]
    
\end{Example}
 
  \subsection{Descents of painted bipartitions}\label{sec:desc}
 

Suppose that 
$
\tau=(\imath,\cP)\times(\jmath,\cQ)\times \alpha \in  \mathrm{PBP}_\star(\check \CO)
$
and write 
\[
  \tau'_{\mathrm{naive}}=(\imath', \cP'_{\mathrm{naive}})\times (\jmath', \cQ'_{\mathrm{naive}})\times \alpha'
\]
for the naive descent of $\tau$. This is clearly an element of $  \mathrm{PBP}_{\star'}(\check \CO')$. 
%Put
%\begin{equation}\label{lstarco}
%  l:=l_{\star, \check \CO}:=\begin{cases}
% \frac{\bfrr_2(\ckcO)}{2}; & \quad \textrm{if } \star\in \{B, \widetilde C\};\\
% \frac{\bfrr_2(\ckcO)+1}{2}, &\quad \textrm{if } \star\in \{C, C^* \};\\
% \frac{\bfrr_2(\ckcO)-1}{2}, &\quad \textrm{if } \star\in \{ D, D^*\}.\\
%\end{cases}
%\end{equation}

The following two lemmas are easily verified and we omit the proofs. We will give an example for each of them. 
\delete{
\begin{lem}\label{descb}
Suppose that 
\[ 
\begin{cases}
\alpha = B^+; & \\
(2,3)\in \wp;\quad  &\\
\cQ(l',1)\in \set{r,d}.
\end{cases}
\]
Then there is a unique element in $\mathrm{PBP}_{\star'}(\check \CO',\wp')$ of the form
  \[
      \tau'=(\imath', \cP')\times (\jmath', \cQ')\times \alpha'
  \]
such that 
     $
     \cP' = \cP'_{\mathrm{naive}}
     $
     and 
     for all $(i,j)\in \BOX(\jmath')$, 
\[
\cQ'(i,j) = \begin{cases}
  r, & \ \text{ if  $(i,j) = (l',1)$;}\\
  \cQ'_{\mathrm{naive}}(i,j), & \ \text{ otherwise}.
\end{cases}
\]
\end{lem}


\begin{Example}
 If 
 \[
 \tau= \ytb{\bullet\bullet,\none} \times \ytb{\bullet \bullet, dd}\times 
  B^+,
 \]
 then 
\[
 \tau'_{\mathrm{naive}}= \ytb{\bullet s,\none} \times \ytb{\bullet, d}\times 
  \widetilde C\qquad\textrm{and}\qquad \tau'= \ytb{\bullet s,\none} \times \ytb{\bullet, r}\times 
  \widetilde C.
 \]
 Note that in this case, the nonzero row lengths of $\check \CO$ are $4,4,2,2$, $\wp=\{(2,3)\}$ and $l'=2$.
\end{Example}
\delete{\begin{proof}
 We only need to check that the triple $\tau'$ defined in the lemma is a painted bipartition. 
 
 Note that 
 \[
  \bar \Lambda_{l-1,2}(\imath', \cP')=\bar \Lambda_{l-1,2}(\imath'_{\mathrm{naive}}, \cP'_{\mathrm{naive}})
 \]
 and 
 \[
 \begin{array}{ccc}
 
      \Lambda_{l-1,1}(\cP_\tau)\times \Lambda_{l-1,2}(\cQ_\tau)
     &  &
        \Lambda_{l-1,1}(\cP_{\tau'})\times \Lambda_{l-1,2}(\cQ_{\tau'})\\
     \hline 
     \hspace{1em}\\
       \emptyset
      \times
      \ytb{ {x_{1}}{x_0},{\enon{\vdots}},{\enon{\vdots}},{x_{n}}}
      &
        \mapsto  &
        \emptyset 
        \times
      \ytb{ {\none}{r},{\none},{\none},\none}
      \end{array}
    \]
  
\end{proof}

Lemma \ref{descb} is easy to check and we omit the details. Note that $(\frac{\bfrr_2(\ckcO)}{2},1) \in \BOX(\jmath')$ under the first two conditions  of Lemma \ref{descb}. Similarly, we also have the following three lemmas. 
}
 }
\begin{lem}\label{descb2}
  Suppose that 
\[  \begin{cases}
 \alpha = B^+; & \\
 \bfrr_2(\ckcO)>0; & \\
 \cQ(l,1)\in \set{r,d}.
\end{cases}
\]
 Then there is a unique element in $\mathrm{PBP}_{\star'}(\check \CO')$ of the form
  \[
      \tau'=(\imath', \cP')\times (\jmath', \cQ')\times \alpha'
  \]
 such that 
     $
     \cQ' = \cQ'_{\mathrm{naive}}
     $
     and
     for all $(i,j)\in \BOX(\imath')$, 
\[
\cP'(i,j) = \begin{cases}
  s, & \ \text{ if $(i,j) = (l',1)$;}\\
  \cP'_{\mathrm{naive}}(i,j), & \ \text{ otherwise}.
\end{cases}
\]
\end{lem}

\begin{Example}
 If 
 \[
 \tau= \ytb{\bullet c, c} \times \ytb{\bullet r, rd}\times 
  B^+,
 \]
 then 
\[
 \tau'_{\mathrm{naive}}= \ytb{s c, c} \times \ytb{r, d}\times 
  \widetilde C\qquad\textrm{and}\qquad \tau'= \ytb{s c, s} \times \ytb{r, d}\times 
  \widetilde C.
 \]
 Note that in this case, the nonzero row lengths of $\check \CO$ are $4,4,4,2$, and $l'=2$.
\end{Example}

\delete{
\begin{lem}\label{descd1}
  Suppose that 
  \[  \begin{cases}
 \alpha = D; & \\
 (2,3)\in \wp;\quad  &\\
 \cP(l',1) \in \set{r,c}.
\end{cases}
\]
 Then there is a unique element in $\mathrm{PBP}_{\star'}(\check \CO',\wp')$ of the form
  \[
      \tau'=(\imath', \cP')\times (\jmath', \cQ')\times \alpha'
  \]
  such that $\cQ'=\cQ'_{\mathrm{naive}}$ and  for all $(i,j)\in \BOX(\imath')$, 
  \[
\cP'(i,j) = \begin{cases}
  r, & \ \text{ if } (i,j) = (l',1); \\
  \cP(l',1), &\  \text{ if } (i,j) = (l'+1,1);\\
  \cP'_{\mathrm{naive}}(i,j), & \ \text{ otherwise}.
\end{cases}
\]
   
\end{lem}




\begin{Example}
 If 
 \[
 \tau= \ytb{\bullet s,  c c, d d} \times \ytb{\bullet,\none, \none }\times 
  D,
 \]
 then 
\[
 \tau'_{\mathrm{naive}}=  \ytb{\bullet,  c,  d}  \times  \ytb{\bullet,\none, \none }\times 
  C,\qquad\textrm{and}\qquad \tau'= \ytb{\bullet, r, c}  \times  \ytb{\bullet,\none, \none }\times
  C.
 \]
 Note that in this case, the nonzero row lengths of $\check \CO$ are $5,5,3,1$,  $\wp=\{(2,3)\}$ and $l'=2$.
\end{Example}
}
\begin{lem}\label{descd2}
  Suppose that 
  \[  \begin{cases}
 \alpha = D; & \\
\mathbf r_2(\check \CO)=\mathbf r_3(\check \CO)>0;  &\\
\cP(l'+1,1)=r; &\\
\cP(l'+1,2)=c; &\\
 \cP(l,1)\in \set{r,d}.
\end{cases}
\]
 Then there is a unique element in $\mathrm{PBP}_{\star'}(\check \CO')$ of the form
  \[
      \tau'=(\imath', \cP')\times (\jmath', \cQ')\times \alpha'
  \]
  such that $\cQ'=\cQ'_{\mathrm{naive}}$ and  for all $(i,j)\in \BOX(\imath')$, 
  \[
\cP'(i,j) = \begin{cases}
  r, & \ \text{ if } (i,j) = (l'+1,1); \\
  \cP'_{\mathrm{naive}}(i,j), & \ \text{ otherwise}.
\end{cases}
\]
   
\end{lem}


\begin{Example}
 If 
 \[
 \tau= \ytb{\bullet\bullet, \bullet s, \bullet s, r c} \times \ytb{\bullet\bullet,\bullet,\bullet, \none }\times 
  D,
 \]
 then 
\[
 \tau'_{\mathrm{naive}}=\ytb{\bullet, \bullet , \bullet ,  c} \times \ytb{\bullet s,\bullet,\bullet, \none } \times 
  C,\qquad\textrm{and}\qquad \tau'=\ytb{\bullet, \bullet , \bullet ,  r} \times \ytb{\bullet s,\bullet,\bullet, \none } \times
  C.
 \]
 Note that in this case, the nonzero row lengths of $\check \CO$ are $7,7,7,3$,   and $l'=3$.
\end{Example}

Finally, we define the descent of $\tau$ to be 
\[
  \nabla(\tau):= \begin{cases}
  \tau', & \ \text{ if the condition of Lemma \ref{descb2}  or \ref{descd2} holds}; \\
  \nabla_{\mathrm{naive}}( \tau), & \ \text{ otherwise},
\end{cases}
\]
which is an element of $  \mathrm{PBP}_{\star'}(\check \CO')$. 
Here $\tau'$ is as in Lemmas  \ref{descb2} and \ref{descd2}. In conclusion, we have defined the descent map
\[
\nabla: \mathrm{PBP}_{\star}(\check \CO)\rightarrow \mathrm{PBP}_{\star'}(\check \CO').
\]


\subsection{Tails of painted bipartitions}
Because of the following proposition, we assume in the rest of this paper that $\check \CO$ is quasi-distinghuished  when $\star\in \{C^*, D^*\}$. 
 

\begin{prop}
  Suppose that $\star\in \{C^*, D^*\}$. If the set $\mathrm{PBP}_\star(\check \CO)$ is nonempty, then $\check \CO$ is quasi-distinguished.  
\end{prop}
\begin{proof}
  Suppose that $\tau=(\imath,\cP)\times(\jmath,\cQ)\times \alpha \in  \mathrm{PBP}_\star(\check \CO)$. If  $\star=C^*$, then  the definition of painted bipartitions implies that 
 \[
 \bfcc_i(\imath)\leq \bfcc_i(\jmath) \qquad \textrm{for all } i=1,2,3, \cdots.
 \]
This forces that $\check \CO$ is quasi-distinguished. 
 
 If  $\star=D^*$, then  the definition of painted bipartitions implies that 
 \[
 \bfcc_{i+1}(\imath)\leq \bfcc_i(\jmath) \qquad \textrm{for all } i=1,2,3, \cdots.
 \]
This  also forces that   $\check \CO$ is quasi-distinguished.
 \end{proof}

In the rest of this subsection, we assume that $\star\in\{B, D, C^*\}$. Note that $l\geq l'$ if $\star\in \{B,C^*\}$,   and $l\geq l'+1$ if $\star=D$.
Put
\[
  \star_{\mathbf t}:= \begin{cases}
  D, & \ \text{ if $\star\in \{B,D\}$}; \\
C^*, &\  \text{ if $\star=C^*$}.
 \end{cases}
\]
Let $\ckcO_{\bftt}$ be the following Young diagram that is 
determined by the pair $(\star, \ckcO)$.
\begin{itemize}
    \item If $\star =B$,
then $\ckcO_{\bftt}$  consists of two rows with lengths $2(l-l')+1$ and $1$.
\item
If $\star =D$,
then $\ckcO_{\bftt}$  consists of two rows with lengths $2(l-l')-1$
and $1$.
\item 
If $\star =C^*$, then $\ckcO_{\bftt}$ consists of one row
with length  $2(l-l')+1$.
\end{itemize}
Note that  in all these three cases
 $\check \CO_{\mathbf t}$ has $\star_{\mathbf t}$-good parity and every element in $\PBP_{\star_\bftt}(\ckcO_\bftt)$ has the form 
 \be\label{tail0}
  \ytb{{x_1} , {x_2} , {\enon\vdots},{\enon{\vdots}},{x_k}  } \times \emptyset \times 
  D,\qquad \qquad  \ytb{{x_1} , {x_2} , {\enon\vdots},{\enon{\vdots}},{x_k}  } \times \emptyset \times 
  D\qquad\textrm{or}\qquad \emptyset \times  \ytb{{x_1} , {x_2} , {\enon\vdots},{\enon{\vdots}},{x_k}  } \times 
 C^*,
  \ee
  respectively if $\star=B, D$ or $C^*$. Here $k=l-l'+1, l-l'$ or $l-l'$ respectively. 

%\subsubsection{The case when $\star = B$}
Let 
$
\tau=(\imath,\cP)\times(\jmath,\cQ)\times \alpha \in  \mathrm{PBP}_\star(\check \CO)
$ be as before. 
 
 
\noindent {\bf The case when $\star = B$.}
In this case, we define the tail $\tau_\bftt$ of $\tau$ to be the first painted bipartition in \eqref{tail0} such that the multiset $\{x_1, x_2, \cdots, x_k\}$ is the 
union of the multiset 
\[
\set{\cQ(l'+1,1),\cQ(l'+2,1),\cdots, \cQ(l,1)}
\]
with the set
\[
  \begin{cases}
 \set{c}, &
 \qquad 
  \text{if $\alpha = B^+$, and either $l'=0$ or $\cQ(l',1)\in \set{\bullet,s}$};  \\ 
 \set{s},&
  \qquad \text{if $\alpha = B^-$, and either $l'=0$ or $\cQ(l',1)\in \set{\bullet,s}$}; \\
%  \qquad\text{when } \alpha_\tau = B^-, \text{ and, } l'=0 \textrm{ or } \cQ_\tau(l',1)\in \set{\bullet,s},  \\ 
\set{\cQ(l',1)},&
\qquad  \text{if $l'>0$ and $\cQ(l',1)\in \{r,d\}$.}
\end{cases}
\]


\smallskip

 \smallskip




 
 
\noindent {\bf The case when $\star = D$.}
In this case, we define the tail $\tau_\bftt$ of $\tau$ to be the second painted bipartition in \eqref{tail0} such that the multiset $\{x_1, x_2, \cdots, x_k\}$ is the 
union of the multiset 
\[
\set{\cP(l'+2,1),\cP(l'+3,1),\cdots, \cP(l,1)}
\]
with the set
\[
  \begin{cases}
 \set{c}, &
 \, 
  \text{if $\bfrr_2(\ckcO)=\bfrr_3(\ckcO)$, 
  $\cP(l'+1,1) = r$, $\cP(l'+1,2) = c$ and $\cP(l,1)\in \set{r,d}$};  \\ 
\set{\cP_\tau(l'+1,1)},&
\,   \text{otherwise.}
\end{cases}
\]
 
 \smallskip
 
 \smallskip
 
\noindent {\bf The case $\star = C^*$.}
In this case, we define the tail $\tau_\bftt$ of $\tau$ to be the third painted bipartition in \eqref{tail0} such that 
\[
  (x_1, x_2, \cdots, x_k)= (\cQ(l'+1,1),\cQ(l'+2,1),\cdots, \cQ(l,1)).
\]
 
 
 When $\star \in \set{B,D}$, the symbol in the last box of the tail $\tau_\bftt\in \PBP_{\star_\bftt}(\ckcO_\bftt)$ will be impotent for us. We write $x_\tau$ for it, namely
\[
x_\tau := \cP_{\tau_\bftt}(\bfcc_1(\imath_{\tau_\bftt}),1).
\]
 The following lemma is easy to check.
 
\begin{lem}\label{tailtip}
If $\star=B$, then
\[
x_\tau=s\Longleftrightarrow
\begin{cases}
  \alpha=B^-;\\ 
  \cQ(l,1) \in\{\bullet, s\}, 
  \end{cases}
%\quad \textrm{if and only if}\quad \alpha=B^- \ \textrm{ and }\  \cQ(l,1) = s, 
\]
and 
\[
x_\tau=d \Longleftrightarrow
%\quad \textrm{if and only if}\quad
\cQ(l,1) =d. 
\]
If $\star=D$, then
\[
x_\tau=s\Longleftrightarrow \cP(l,1) = s, 
\]
and 
\[
x_\tau=d\Longleftrightarrow \cP(l,1) =d. 
\]

\end{lem}







\subsection{Some properties of the descent maps}



%We state the key properties of the descent map in this section and the proofs will be given in \Cref{sec:DD.proof}.


The key properties of the descent map when $\star\in \set{C,\wtC,D^*}$ are summarized in the following proposition.   

\begin{prop}\label{lem:DD.bij}
Suppose that $\star \in \set{C,\wtC,D^*}$ and cosider the
descent map
\begin{equation}\label{eq:DD.CC}
\nabla: \PBP_\star(\ckcO)\longrightarrow  \PBP_{\star'}(\ckcOp). 
\end{equation} 

\noindent (a) If 
$\star=D^*$ or $\bfrr_1(\ckcO)>\bfrr_2(\ckcO)$, then 
the map \eqref{eq:DD.CC}  is bijective.
 
 \noindent (b) If  $\star\in \{C,\widetilde C\}$ and $\bfrr_1(\ckcO)=\bfrr_2(\ckcO)$, then the  map \eqref{eq:DD.CC} is injective and its image equals 
\[
\Set{\tau'\in \PBP_{\star'}(\ckcOp)| x_{\tau'}\neq s}.
\]

\end{prop}

\begin{proof}
We assume that $\star = \wtC$ and $\bfrr_1(\ckcO)=\bfrr_2(\ckcO)$. The proofs in the other  cases are similar and are left to the reader. 

Note that the map \eqref{eq:DD.CC} induces a map
\begin{equation}\label{eq:DD.CC1}
\nabla: \Set{\tau\in \PBP_{\star}(\ckcO)| \cP_\tau(l,1)\neq c}\rightarrow \Set{\tau'\in \PBP_{\star'}(\ckcOp)|  \alpha_{\tau'}=B^+}. 
\end{equation}
Suppose that $\tau'$ is an element  in the codomain of the map \eqref{eq:DD.CC1}. Similar to the proof of Lemma \ref{lemDDn1}, there is a unique element in $\tau:=\nabla^{-1}(\tau')\in \PBP_{\star}(\ckcO)$ such that
for all $i=1,2, \cdots,l$, 
\[
  \cP_{\tau}(i,1)\in \{\bullet, s\},
\]
and
for all $(i,j)\in \BOX(\tau')$, 
\begin{equation}
     \cP_\tau(i,j+1)=\begin{cases}   
    \bullet \textrm{ or } s,&\textrm{ if  $\ \cP_{\tau'}(i,j)\in \{\bullet, s\}$;} \smallskip \\
  \cP_{\tau'}(i,j),& \textrm{ if $\ \cP_{\tau'}(i,j)\notin \{\bullet, s\}$},\end{cases}
   \end{equation}
 and
   \begin{equation}
     \cQ_\tau(i,j)=\begin{cases}   
    \bullet \textrm{ or } s,&\textrm{ if  $\ \cQ_{\tau'}(i,j)\in \{\bullet, s\}$;} \smallskip \\
  \cQ_{\tau'}(i,j), & \textrm{ if $\ \cQ_{\tau'}(i,j)\notin \{\bullet, s\}$}.  \end{cases}
   \end{equation}
Note that $\tau$ is in the domain of \eqref{eq:DD.CC1}. It is then routine to check that the map
\[
  \nabla^{-1}: \Set{\tau'\in \PBP_{\star'}(\ckcOp)|  \alpha_{\tau'}=B^+} \rightarrow  \Set{\tau\in \PBP_{\star}(\ckcO)| \cP_\tau(l,1)\neq c}
\]
and the map \eqref{eq:DD.CC1} are inverse to each other. Hence the map \eqref{eq:DD.CC1} is bijective. 


Similarly, the map
\[
\nabla: \Set{\tau\in \PBP_{\star}(\ckcO)| \cP_\tau(l,1)= c}\rightarrow \Set{\tau'\in \PBP_{\star'}(\ckcOp)| \alpha_{\tau'}=B^-, \cQ_{\tau'}(l,1)\in\{r,d\}} 
\]
is well-defined, and we show that it is bijective by explicitly  constructing its inverse. In view of Lemma \ref{tailtip}, this proves the proposition in the case we are considering. 


\end{proof}

\medskip

The key properties of the descent map when $\star\in \set{D,B,C^*}$ are summarized in the following two propositions.   

\begin{prop}\label{prop:DD.BD}
Suppose that $\star \in \set{D,B,C^*}$.
Then the map 
\begin{equation}\label{eq:DD.BD}
   \PBP_\star(\ckcO)\longrightarrow
   \PBP_{\star'}(\ckcOp)\times \PBP_{\star_\bftt}(\ckcO_\bftt)
   \qquad \tau \mapsto (\DD(\tau), \tau_\bftt)
\end{equation}
is injective. Moreover, \eqref{eq:DD.BD} is bijective 
unless $\star\in \set{B,D}$ and $\bfrr_2(\ckcO)=\bfrr_3(\ckcO)>0$. 
% when $\star = C^*$ 
\end{prop}
\begin{proof}
 
\end{proof}





%The following equation of signatures will be crucial in our computation of the local system in the next section. 


For every painted bipartition $\tau$, write
\[
  \ssign(\tau):=(p_\tau, q_\tau).
\]
When $\bfrr_2(\check \CO)>0$, the double descent $\nabla^2(\tau):=\nabla(\nabla(\tau))$ is well-defined whenever $\tau\in \mathrm{PBP}_\star(\check \CO)$. 
As in the Introduction, $\CO$ denotes the Barbasch-Vogan dual of $\check \CO$. We also consider it as a Young diagram. 

\begin{prop}\label{prop:delta}
Assume that $\star \in \set{D,B,C^*}$ and $\bfrr_2(\ckcO)>0$. Write $\ckcOpp := \ckDD(\ckcO')$ and consider the map 
\begin{equation}\label{eq:delta}
    \colon \PBP_\star(\ckcO)\longrightarrow 
    \PBP_\star(\ckcOpp)\times \PBP_{\star_\bftt}(\ckcO_\bftt),
    \qquad \tau \mapsto (\DD^2(\tau),\tau_\bftt).
\end{equation}

\noindent (a) Suppose that 
$\star = C^*$ or $\bfrr_2(\ckcO)>\bfrr_3(\ckcO)$. Then the map \eqref{eq:delta} is a bijective, and for every $\tau\in  \PBP_\star(\ckcO) $, 
    % We have the following equation of signatures.
\[
\ssign(\tau)
=(\bfcc_2(\cO),\bfcc_2(\cO))+\ssign(\DD^2(\tau))+\ssign(\tau_\bftt).
\]

\noindent (b) Suppose that  $\star \in \set{B,D}$ and $\bfrr_2(\ckcO)=\bfrr_3(\ckcO)$. Then the map \eqref{eq:delta} is  injection and its  image equals 
    \[
    \Set{ (\tau'',\tau_0)  \in \PBP_\star(\ckcOpp)\times \PBP_D(\ckcO_\bftt)  | 
    x_{\tau''} = d\, \text{ or } \,
    \cP_{\tau_0}^{-1}(\set{s,c})\neq \emptyset }.
    \]
    Moreover,  for every $\tau\in  \PBP_\star(\ckcO) $, 
\[
\ssign(\tau)
=(\bfcc_2(\cO)-1,\bfcc_2(\cO)-1)+\ssign(\DD^2(\tau))+\ssign(\tau_\bftt).
\]
\end{prop}

\begin{proof}
 
\end{proof}



\section{Marked Young diagrams}




\subsection{Operations on  painted Young diagrams } 
  
Let $(\imath,\cP)$ be a painted Young diagram in this subsection. 

\begin{defn}
For all $a,b\in\bN$, the truncation 
$(\imath_0, \cP_0):=\Lambda_{a,b}(\imath,\cP)$ is the painted Young diagram such that 
%\[
%\begin{array}{rcl}
%\Lambda_{a,b} \colon 
%\set{\text{painted Young diagram}}  & \rightarrow &
%\set{\text{painted Young diagram}},\smallskip\\
%v$ & \mapsto & (\imath_0, \cP_0):=\Lambda_{a,b}(\imath,\cP) 
%\end{array}
%\]
%of  painted Young diagrams by requiring that
\begin{itemize}
    \item $\BOX(\imath_0) = \set{(i,j)\in \bN^+\times \bN^+| (i+a,j)\in \BOX(\imath), j\leq b}$, and
    \item $\cP_0(i,j) = \cP(i+a,j)$ for all $(i,j)\in \BOX(\imath_0)$.
\end{itemize}
\end{defn}
Graphically,  $\Lambda_{a,b}(\imath,\cP)$ is obtained form $(\imath,\cP)$ by removing the first $a$ rows and  all but the first $b$ columns. 
We define $\barLambda_{a,b}(\imath,\cP)$ to be the pair $(\imath_1,\cP_1)$ where 
\begin{itemize}
\item $\imath_1 := \set{(i,j)\in \BOX(\imath)| i\leq a \text{ or } j>b }$, and
\item $\cP_1 = \cP|_{\imath_1}: \imath_1\rightarrow \set{\bullet,s,r,c,d}$. 
\end{itemize}
Graphically, this is the complement of $\Lambda_{a,b}(\imath,\cP)$ in $(\imath, \cP)$.

\begin{defn}
The descent of $(\imath, \cP)$, which is denoted by $\nabla(\imath, \cP)$, is the painted Young diagram obtained from $(\imath, \cP)$ by removing the first column. Similarly, the descent of a Young diagram $\jmath$, which is denoted by $\nabla(\jmath)$, is defined to be the Young diagram obtained from $\jmath$ by removing the first column.
 \end{defn}
 By convention, the descents of the empty Young diagram and the empty painted Young diagram are empty. 
 
\begin{Example}
If 
\[
 (\imath, \cP)= \ytb{{\bullet}{\bullet}{\bullet},{r}{c}{d},{r},{c} },
\]
then 
\[
  \Lambda_{1,2}(\imath, \cP)= \ytb{{r}{c},{r},{c} }\qquad\textrm{and}\qquad \nabla(\imath, \cP)=
  \ytb{{\bullet}{\bullet},{c}{d} }\, .
\]
  \end{Example}
  
\begin{defn} A painted Young diagram $(\imath, \cP)$ is represented by    a pair $(\tilde \imath, \tilde \cP)$, where  $\tilde \imath$ is a Young diagram and $\tilde \cP: \mathrm{Box}(\tilde \imath)\rightarrow \{\bullet, s,r,c,d, \emptyset \}$ is a map, if there exist $a,b\in \bN$ such that
      \[
      \begin{cases} 
    \tilde \cP^{-1}(\{\bullet, s, r,c,d\})=\mathrm{Box}(\imath)+(a,b); \quad \textrm{and} &\hbox{} \smallskip\\
  \cP(i,j)=\tilde \cP(i+a,j+b) \quad \textrm{for all  $(i,j)\in \mathrm{Box}(\imath)$.} &\hbox{} \\
  \end{cases} 
          \]
          \end{defn}
  Graphically, the painted Young diagram $(\imath, \cP)$   is obtained from $(\tilde \imath, \tilde \cP)$ by removing all the boxes with label $\emptyset$.  We say that a pair  $(\imath, \cP)\times (\jmath, \cQ)$ of painted Young diagrams is represented by a pair  $(\widetilde \imath, \widetilde \cP)\times (\widetilde \jmath, \widetilde \cQ)$ if $( \imath, \cP)$ is represented by $(\widetilde \imath, \widetilde \cP)$ and $( \jmath, \cQ )$ is represented by $(\widetilde \jmath, \widetilde \cQ )$. 
  
   \begin{Example}
The painted Young diagram
\[
  \ytb{{r}{d},{r},{c} }
\qquad \textrm{is represented by}\qquad 
  \ytb{{\eee}{\eee}{\eee},{r}{d}{\eee},{r},{c},{\eee} }.
\]

  \end{Example}
  
 
%It is obvious  that $\pbpsns(\ckcO)$ is  empty if either 
%\[\star\in \set{C,\wtC}\quad\textrm{ and }\quad  %(1,2)\notin \DP(\ckcO);
%\]
%or 
%\[
%\star \in \set{B,D}\quad\textrm{ and }\quad(2,3)\notin \DP(\ckcO).
%\]



%\begin{enumerate}
%\item[(a)] $\star\in \set{C^*,D^*}$;
%\item[(b)] $\star\in \set{C,\wtC}$ and %$(1,2)\notin \DP(\ckcO)$;
%\item[(c)]  $\star \in \set{B,D}$ and 5$(2,3)\notin \DP(\ckcO)$.
%\end{enumerate}
%




\delete{
\subsection{$e$-painted Young diagrams}
  For the convenience of describing some painted Young diagrams,
  we introduce one more symbol $e$
  and make the following definition. 
  
  \begin{defn}\label{defepaint} An $e$-painted Young diagram is   a pair $(\tilde \imath, \tilde \cP)$ where  $\tilde \imath$ is a Young diagram and $\tilde \cP: \mathrm{Box}(\tilde \imath)\rightarrow \{\bullet, s,r,c,d, e\}$ is a map, subject to  the following condition: 
           there exist $t\in \bN$ and a painted Young diagram $(\imath, \cP)$ such that 
      \[
      \begin{cases} 
    \tilde \cP^{-1}(\{\bullet, s, r,c,d\})=\mathrm{Box}(\imath)+(t,0); \quad \textrm{and} &\hbox{} \smallskip\\
  \cP(i,j)=\tilde \cP(i+t,j) \quad \textrm{for all  $(i,j)\in \mathrm{Box}(\imath)$.} &\hbox{} \\
  \end{cases} 
          \]
  \end{defn}
  Graphically, the painted Young diagram $(\imath, \cP)$ in Definition \ref{defepaint}  is obtained from $(\tilde \imath, \tilde \cP)$ by removing all the boxes with label e.  We say that the $e$-painted Young diagram $(\tilde \imath, \tilde \cP)$ represents the painted Young diagram $(\imath, \cP)$. 
 
  \begin{Example}
The $e$-painted Young diagram 
\[
  \ytb{{e}{e}{e},{r}{d}{e},{r},{c},{e} }
\]
represents the painted Young diagram
\[
  \ytb{{r}{d},{r},{c} }.
\]
  \end{Example}
   We say that a painted bipartition $\tau=(\imath, \cP)\times (\jmath, \cQ)\times \alpha$ is represented by a triple $\tilde \tau=(\tilde \imath, \tilde \cP)\times (\tilde \jmath, \tilde \cQ)\times \alpha$ if $(\tilde \imath, \tilde \cP)$ is an $e$-painted Young diagram that represents  $(\imath, \cP) $ and $(\tilde \jmath, \tilde \cQ)$ is an $e$-painted Young diagram that represents  $(\jmath, \cQ) $. 
 }
 
 \delete{
 For the convenience of describing some painted Young diagrams, 
  
  
%\subsection{Convention}
To simplify the discussion below, we adopt the following convention:
When we draw a diagram to represent the painted Young diagram, the ``$\emptyset$'' label in a box means the box is not contained the corresponding painted Young diagram. 

  \begin{Example}
The diagram
\[
  \ytb{{\eee}{\eee}{\eee},{r}{d}{\eee},{r},{c},{\eee} }
\]
represents the painted Young diagram
\[
  \ytb{{r}{d},{r},{c} }.
\]
  \end{Example}


}
   


   
\subsection{Switching painted bipartitions}

\delete{ A painted bipartition $\tau\in  \mathrm{PBP}_\star(\check \CO) $ is said to be pre-special if
  \[
  \begin{cases} 
  %\textrm{the trivial representation $\C$}, &\hbox{if $\abs{\check \CO_\tau}\leq 1$}; \medskip\\
  (1,2)\notin\wp_\tau, &\hbox{when  $\star \in \{C, \widetilde C, C^* \}$;} \\
  (2,3)\notin\wp_\tau, &\hbox{when $\star\in \{ B, D, D^*\}$.} \\
  \end{cases} 
  \]
  
  Write
 \[
   \pbp_\star(\check \CO)=\pbpssp(\check \CO)\sqcup
   \pbpsns(\check \CO),
 \]
 where $\pbpssp(\check \CO)$ is the set of pre-special elements in $\pbpst(\check \CO)$, and  $\pbpsns(\check \CO)$ is the complementary set of $\pbpssp(\ckcO)$.
}
  In  this subsection, we assume that 
  \[
  \star\in \{C, \wtC\}\qquad \textrm{and}\qquad  (1,2)\in \wp.
  \]
  Set $\bar \wp:=\wp\setminus\{(1,2)\}$. 
  Our goal is to define a bijective map
 \[
 \mathrm{PBP}_\star(\ckcO, \wp )\longrightarrow \mathrm{PBP}_\star(\ckcO, \bar\wp), \quad
 \tau \mapsto \bartau. 
 \]
% Put 
% \[
% m:=m_{\star,\ckcO}:=
%       \frac{\bfrr_1(\ckcO)-\bfrr_2(\ckcO)-2}{2}\in \bN.
%       \]

  
\subsubsection{The case of $\star=C$}


Suppose that $\star=C$. Then $\bfrr_1(\ckcO)>\bfrr_2(\ckcO)\geq \bfrr_3(\ckcO)\geq 1$.
%Write 
%\[
%l:=l_{\star, \ckcO} :=\begin{cases} \frac{\bfrr_2(\ckcO)-3}{2},
%  & \textrm{if $\bfrr_2(\ckcO)\geq 3$}; \smallskip \\
%  0, & \textrm{if $\bfrr_2(\ckcO)=1$}. 
%  \end{cases}
%\]

For every $\tau= (\imath,\cP)\times(\jmath,\cQ)\times \star\in \pbpst(\ckcO)$, 
 its  leg is defined to be  the  pair
\[
\LEG(\tau) := \Lambda_{\max(l'-2,0),2}(\imath,\cP)\times \Lambda_{\max(l'-2,0),2}(\jmath,\cQ),
\]
and its body is defined to be the pair 
\[
\BODY(\tau) := 
\barLambda_{\max(l'-2,0),2}(\imath,\cP)\times \barLambda_{\max(l'-2,0),2}(\jmath,\cQ).
\]
  

Note that if $\tau\in \mathrm{PBP}_\star(\ckcO,\wp ) $,  then $\LEG(\tau)$ is represented by the first pair  in \eqref{legtau1} %of the form
where  
\[
x_1\neq \eee, \qquad x_2\neq \eee,\qquad x_0=\emptyset\Leftrightarrow x_5=\emptyset \Leftrightarrow \mathbf{r}_2(\check \CO)=1,
\]
and  the grey part consisting of $(l-l')$ boxes with label $r$. It is clear that $x_1\in \{r,c\}$. 

\begin{equation}\label{legtau1}
 \LEG(\tau):\ \  \ytb{{x_{0}}{x_3},{*(srcol)r}{x_{4}},{\enon[srcol]\vdots},{\enon[srcol]{\vdots}},
  {*(srcol)r},{x_{1}},{x_{2}} }
    \times\ytb{{x_5}{x_6},\none,\none ,\none ,\none ,\none,\none}
    \hspace{8em}
    \LEG(\bartau):\ \ 
 \ytb{\none,{y_{0}}{y_3},{y_{2}}{y_{4}},
    \none,\none,\none,\none,\none,\none}\times  \ytb{\none,{y_5}{y_6},{y_1},{*(srcol)s},
     {\enon[srcol]\vdots}, {\enon[srcol]\vdots},{*(srcol)s},\none,\none}
\end{equation}

Likewise, for every $\bartau\in \mathrm{PBP}_\star(\ckcO,\bar \wp ) $,
 $\LEG(\bartau)$ is represented by the second pair of \eqref{legtau1}  
where  \[
y_1\neq \eee, \qquad y_2\neq \eee,\qquad y_0=\emptyset\Leftrightarrow 
y_5=\emptyset \Leftrightarrow \mathbf{r}_2(\check \CO)=1,
\]
and  the grey part consisting of $(l-l')$ boxes with label $s$. 
It is clear that $y_1\in \{\bullet, s\}$.

\begin{prop}\label{propswithc}
Suppose that $\star=C$ and $(1,2)\in \wp$. 
Let $\tau\in \mathrm{PBP}_\star(\ckcO,\wp ) $ such that $\LEG(\tau)$ 
is represented by the first pair in
\eqref{legtau1}. Then there is a unique element 
$\bartau\in \mathrm{PBP}_\star(\ckcO,\bar \wp )$
with the following properties: 
\begin{itemize}
    \item $\BODY(\bartau) = \BODY(\tau)$;
    \item $\LEG(\bartau)$ is represented by the second pair in 
\eqref{legtau1} such that 
\[
(y_3,y_5,y_6)=(x_3,x_5,x_6)
\]
and
\[
  (y_0,y_1, y_2, y_4)=\begin{cases}
   (x_0, \bullet, \bullet, r),
  & \textrm{if $x_1=r$ and $(x_2, x_4)=(d,c)$}; \smallskip \\
  (x_0, s, x_2, x_4),
  & \textrm{if $x_1=r$ and $(x_2, x_4)\neq (d,c)$}; \smallskip \\
   (c, s, d, x_4),
  & \textrm{if $x_1=c$ and $x_0=r$}; \smallskip \\
   (x_0, \bullet, \bullet, x_4),
  & \textrm{if $x_1=c$ and $x_0\neq r$}. \\
  \end{cases}
\]
\end{itemize}
Moreover, the map 
\be\label{switchc1}
 \mathrm{PBP}_\star(\ckcO,\wp)\rightarrow \mathrm{PBP}_\star(\ckcO,\bar \wp), \qquad
 \tau \mapsto \bartau
 \ee
 is bijective.
\end{prop}


\begin{proof}
The uniqueness assertion is obvious. It is routine to verify case by case  that the conditions  of the proposition define an element $\bartau\in \mathrm{PBP}_\star(\ckcO,\bar \wp ) $. Thus the map \eqref{switchc1} is well-defined. 

Similarly, given an element $\bar \tau\in \mathrm{PBP}_\star(\ckcO,\bar \wp)$ such that  $\LEG(\bar \tau)$ is represented by the second pair in 
\eqref{legtau1}, there is a unique element $\tau\in  \mathrm{PBP}_\star(\ckcO,\wp) $ satisfying the following conditions: 
\begin{itemize}
    \item $\BODY(\tau) = \BODY(\bar \tau)$;
    \item $\LEG(\tau)$ is represented by the first pair in 
\eqref{legtau1} such that 
\[
(x_3,x_5,x_6)=(y_3,y_5,y_6)
\]
and
\[
  (x_0,x_1, x_2, x_4)=\begin{cases}
   (y_0, r,d,c),
  & \textrm{if $y_1=\bullet$ and $y_4=r$}; \smallskip \\
  (y_0, r,y_2, y_4),
  & \textrm{if $y_1=s$ and $y_0\neq c$}; \smallskip \\
   (r, c, d, y_4),
  & \textrm{if $y_1=s$ and $y_0= c$}; \smallskip \\
   (y_0, c,d, y_4),
  & \textrm{if $y_1=\bullet$ and $y_4\neq r$}. \\
  \end{cases}
\]
\end{itemize}
Thus we also have a well-defined map
\be\label{switchc2}
 \mathrm{PBP}_\star(\ckcO, \bar \wp)\rightarrow \mathrm{PBP}_\star(\ckcO, \wp), \qquad
 \bar \tau \mapsto \tau.
 \ee

The proposition then follows by verifying that the maps \eqref{switchc1} and \eqref{switchc2} are inverse to each other. 

%This is routine to check and we omit the details.  
  
\end{proof}
\delete{
\begin{table}[h!]
\[
\begin{array}{c|c|c}
\hline
\hline
    & \LEG(\tau) & \LEG(\bartau) \\
  \hline
\begin{cases}y_1=r \\(y_2,y_4)=(d,c)\end{cases} &
  \ytb{\none,{y_{0}}{y_3},
  {*(srcol)r}{c},{\enon[srcol]{\vdots}},{\enon[srcol]{\vdots}},
  {*(srcol)r},{r},{d},\none}
    \times\ytb{\none,{y_5}{y_6},\none,\none,
    \none,\none ,\none,\none,\none }
    & \ytb{\none,{y_{0}}{y_3},{\bullet}{r},
    \none,\none,\none,\none,\none,\none }
    \times
    \ytb{\none,{y_5}{y_6},{\bullet},{*(srcol)s},
    {\enon[srcol]\vdots},{\enon[srcol]\vdots},{*(srcol)s},\none,\none,}\\
    \hline
\begin{cases}y_1=r \\(y_2,y_4)\neq (d,c)\end{cases} 
&
  \ytb{\none,{y_{0}}{y_3},
  {*(srcol)r}{y_{4}}, {\enon[srcol]\vdots},{\enon[srcol]\vdots},
  {*(srcol)r},{r},{y_{2}},\none}
    \times\ytb{\none,{y_5}{y_6},\none,\none,
    \none,\none ,\none,\none,\none}
    &\ytb{\none,{y_{0}}{y_3},{y_{2}}{y_{4}},
    \none,\none,\none,\none,\none,\none}
    \times
    \ytb{\none,{y_5}{y_6},{s},{*(srcol)s},
     {\enon[srcol]\vdots}, {\enon[srcol]\vdots},{*(srcol)s},\none,\none}\\
    \hline
 \begin{cases} y_1=c\\y_0=r\end{cases} &
  \ytb{\none,{r}{y_3},
  {*(srcol)r}{y_4}, {\enon[srcol]\vdots},{\enon[srcol]\vdots},
  {*(srcol)r},{c},{d},\none}
    \times\ytb{\none,{s}{y_6},\none,\none,
    \none,\none ,\none,\none,\none}
    &\ytb{\none,{c}{y_3},{d}{y_4},
    \none,\none,\none,\none,\none\none,\none}
    \times
    \ytb{\none,{s}{y_6},{s},{*(srcol)s},
     {\enon[srcol]\vdots}, {\enon[srcol]\vdots},
     {*(srcol)s},\none,\none}\\
    \hline
 \begin{cases} y_1=c\\y_0 \neq r \end{cases}  &
  \ytb{\none,{y_{0}}{y_3},
  {*(srcol)r}{y_{4}}, {\enon[srcol]\vdots}, {\enon[srcol]\vdots},
  {*(srcol)r},{c},{d},\none,}
    \times\ytb{\none,{y_5}{y_6},\none,\none,
    \none,\none ,\none,\none,\none}
    &\ytb{\none,{y_{0}}{y_3},{\bullet}{y_{4}},
    \none,\none,\none,\none,\none,\none}
    \times
    \ytb{\none,{y_5}{y_6},{\bullet},{*(srcol)s},
     {\enon[srcol]\vdots}, {\enon[srcol]\vdots},{*(srcol)s},\none,\none}\\
  \hline
  \hline
\end{array}
\]
\caption{}%{``special-non-special'' switch}
\label{tab:nonsp.C}
\end{table}
}

%For every painted bipartition $\tau=(\imath, \CP_{\mathrm l})\times (\jmath, \CP_{\mathrm r})\times \alpha$, we call the pained Young diagram $(\imath, \CP_{\mathrm l})$ the left painting of $\tau$, and call  $(\jmath, \CP_{\mathrm r})$ the right painting of $\tau$.


% \[
  %\mathrm{PBP}^{\mathrm{ps}}_\star(\check \CO):= \{\tau\in \mathrm{PBP}_\star(\check \CO) \mid \textrm{ $\tau$ is pre-special}\}.
   %\]

\subsubsection{The case of $\star=\wtC$}
Suppose that $\star=\wtC$.
Then $\bfrr_1(\ckcO)>\bfrr_2(\ckcO) \geq 0$. 
%Write
%\[
%       l:= l_{\star,\ckcO} :=  \begin{cases} \frac{\bfrr_2(\ckcO)-2}{2},
%  & \textrm{if $\bfrr_2(\ckcO)\geq 2$}; \smallskip \\
%  0, & \textrm{if $\bfrr_2(\ckcO)=0$}. 
%  \end{cases}
%  \]


For every  $\tau= (\imath,\cP)\times(\jmath,\cQ)\times \star\in \pbpst(\ckcO)$, its leg is defined to be the  pair
\[
\LEG(\tau) := \Lambda_{\max(l'-1,0),1}(\imath,\cP)\times \Lambda_{\max(l'-1,0),1}(\jmath,\cQ),
\]
and its  body  is defined to be the pair 
\[\BODY(\tau) := 
\barLambda_{\max(l'-1,0),1}(\imath,\cP)\times \barLambda_{\max(l'-1,0),1}(\jmath,\cQ).\]
  
 Note that if $\tau\in \mathrm{PBP}_\star(\ckcO,\wp ) $,  then $\LEG(\tau)$ is represented by the first pair  in \eqref{legtau11} where  
\[
x_1\neq \eee,\qquad x_0=\emptyset\Leftrightarrow x_2=\emptyset\Leftrightarrow \mathbf{r}_2(\check \CO)=0,
\]
and the grey part consisting of $l-l'-1$ boxes with label $r$.
 
\begin{equation}\label{legtau11}
 \LEG(\tau):\ \   \ytb{{x_2},\none,\none ,\none ,\none ,\none}
    \times
  \ytb{{x_{0}},{*(srcol)r},{\enon[srcol]\vdots},{\enon[srcol]{\vdots}},
  {*(srcol)r},{x_{1}} }
    \hspace{8em}
    \LEG(\bartau):\ \ 
 \ytb{{y_{0}},{*(srcol)s},{\enon[srcol]\vdots},{\enon[srcol]{\vdots}},
  {*(srcol)s},{y_{1}} } 
    \times
    \ytb{{y_2},\none,\none ,\none ,\none ,\none},
\end{equation}


 Likewise, for every $\bartau\in \mathrm{PBP}_\star(\ckcO,\bar \wp ) $,  $\LEG(\bartau)$ is represented by the second pair of \eqref{legtau11}  
where
\[
y_1\neq \eee,\qquad y_0=\emptyset\Leftrightarrow y_2=\emptyset\Leftrightarrow \mathbf{r}_2(\check \CO)=0,
\]
and the grey part consisting of $l-l'-1$ boxes with label $s$.
 
 \delete{
 If $\tau\in \mathrm{PBP}_\star(\ckcO,\bar \wp ) $,  then $\LEG(\tau)$ is represented by a pair of the form
\be\label{tildecps}
  \ytb{{x_{0}},{*(srcol)r},{\enon[srcol]\vdots},{\enon[srcol]{\vdots}},
  {*(srcol)r},{x_{1}} } 
    \times
    \ytb{{x_2},\none,\none ,\none ,\none ,\none},
\ee
where
\[
x_1\neq \eee,\qquad x_0=\emptyset\Leftrightarrow x_2=\emptyset\Leftrightarrow \mathbf{r}_2(\check \CO)=0,
\]
and the grey part consisting of $m$ boxes with label $r$.
}

The following proposition is much easier to check than Proposition \ref{propswithc}. We omit the details. 
\begin{prop}\label{propswithc2}
Suppose that $\star=\widetilde C$ and $(1,2)\in \wp$. For every $\tau\in \mathrm{PBP}_\star(\ckcO,\wp ) $ such that $\LEG(\tau)$ is represented by the first pair  in \eqref{legtau11}, there is a unique element $\bartau\in \mathrm{PBP}_\star(\ckcO,\bar \wp ) $ such that 
 \[
    \BODY(\bartau) = \BODY(\tau) \qquad 
\]
and $\LEG(\bartau)$ is represented by the second pair  in \eqref{legtau11} with
\[
y_i = \bullet,r,s,d,c, \text{ or }\eee\qquad (i=0,1,2),
\]
respectively if 
\[
x_i = \bullet,s,r,c,d, \text{ or }\eee.  %\text{ respectively} 
\]
Moreover, the map 
\[
 \mathrm{PBP}_\star(\ckcO,\wp)\rightarrow \mathrm{PBP}_\star(\ckcO,\bar \wp), \qquad
 \tau \mapsto \bartau. 
 \]
 is bijective.
\end{prop}



%\subsection{dot-$s$ bipartitions} % $s$-dot   bipartitions and  


%In what follows we will define case by case a painted bipartition 
%\be\label{taudes2}
%\tau':=\nabla(\tau)=(\imath',\cP')\times(\jmath',\cQ')\times \alpha'\in  \mathrm{PBP}_{\star'}(\check \CO', \wp'),
%\ee
%to be called the descent of $\tau$.we define a descent map of the painted bipartitions  by modifying the naive descent.
\delete{\subsubsection{The cases when $\star = C,\wtC, C^*, D^*$} 
 \begin{defn}
 Supose $\star = C,\wtC, C^*, D^*$. 
 Let $\tau\in \PBP_\star(\ckcO)$. The descent $\DD(\tau)$ of $\tau$ is defined by
 \[
   \DD(\tau) := \begin{cases}
     %\DDn(\tau) & \text{ if } (1,2)\notin \wp_\tau \\%\star_\tau = C^*, D^*\\
     \DDn(\bartau) & \text{ if } (1,2) \in \wp_{\tau} \\
     \DDn(\tau) & \text{ otherwise.} 
   \end{cases} 
 \]
 \end{defn}

\subsubsection{The case when $\star=D$}

Let $\tau = (\imath,\cP)\times (\jmath,\cQ)\times \alpha \in \PBP_\star(\ckcO)$. 
We define 
\[
    \taupna:= \DDn(\tau) 
\]
Let $(l,m) := (\frac{\bfrr_2(\ckcO)+1}{2},\frac{\bfrr_1(\ckcO)+1}{2})$.

We define $\cP'\colon \BOX(\imath_\taupna) \rightarrow \dsrcd$ by the following recipe.  
\begin{enumC}
\item
Suppose $\bfrr_3(\ckcO)>0$, $(2,3)\notin \PP(\ckcO)$, $\cP(l,2)=c$, and 
$(\cP(l,1),\cP(m,1)) = (r,r)$ or $(r,d)$. (Note that, we have $\bfrr_2(\ckcO)=\bfrr_3(\ckcO)\geq 1$ and so $(l,2)\in \BOX(\imath)$.)
We define
\[
\cP'(i,j) := \begin{cases}
  r & \text{ if } (i,j) = (l,1),\\
  \cP_\taupna(i,j) & \text{ otherwise}.
\end{cases}
\]
\item
Suppose $(2,3) \in \wp_\tau$ and $\cP(l-1,1) \in  \set{r,c}$. 
(Note that, we have $\bfrr_2(\ckcO)>\bfrr_3(\ckcO)\geq 1$ and so $l\geq 2$.)
We define
\[
\cP'(i,j) := \begin{cases}
  r & \text{ if } (i,j) = (l-1,1),\\
  \cP(l-1,1) & \text{ if } (i,j) = (l,1),\\
  \cP_\taupna(i,j) & \text{ otherwise}.
\end{cases}
\]
\item Suppose we are not in the above two cases.
We define $\cP' := \cP_\taupna$.
\end{enumC}

We define  the descent of $\tau$ by 
\[
    \tau' :=\DD(\tau):= (\imath_\taupna,\cP')\times (\jmath_\taupna, \cQ_\taupna)\times \alpha_\taupna.
\]

\begin{Example}
 \[
 \text{ By rule (a),  }\quad
 \ytb{rc,r,d}\times \emptyset\times D \quad\text{ descents to }\quad
 \ytb{r, \none,\none} \times \emptyset \times C
 \]
 \[
 \text{ By rule (b),  }\quad
 \ytb{cc,dd}\times \emptyset\times D\quad \text{ descents to }\quad
 \ytb{r, c} \times \emptyset \times C
 \]
\end{Example}

\subsubsection{The case when $\star=B$}

Let $\tau = (\imath,\cP)\times (\jmath,\cQ)\times \alpha \in \PBP_\star(\ckcO)$. 
We define 
\[
    \taupna:= \DDn(\tau) 
\]
Let $(l,m) := (\frac{\bfrr_2(\ckcO)}{2},\frac{\bfrr_1(\ckcO)}{2})$.

We define $\cP'\colon \BOX(\imath_\taupna) \rightarrow \dsrcd$  and 
$\cQ'\colon \BOX(\imath_\taupna) \rightarrow \dsrcd$ by the following recipe.  

\begin{enumC}
\item
Suppose $\alpha = B^+$, $\bfrr_2(\ckcO)>0$, $(2,3)\notin \wp_\tau$, and $\cQ(l,1)\in \set{r,d}$. 
\[
\cP'(i,j) := \begin{cases}
  s & \text{ if } (i,j) = (l,1),\\
  \cP_\taupna(i,j) & \text{ otherwise}.
\end{cases}
\]
and $\cQ':= \cQ_\taupna$.
\item
Suppose $\alpha=B^+$, $(2,3) \in \wp_\tau$ and $\cQ(l,1) \in  \set{r,d}$. 
(Note that, we  have $\bfrr_2(\ckcO)\geq 2$ and so $(l,1)\in \BOX(\jmath)$.)
We define
$\cP' := \cP_\taupna$ and 
\[
\cQ'(i,j) := \begin{cases}
  r & \text{ if } (i,j) = (l,1),\\
  \cQ_\taupna(i,j) & \text{ otherwise}.
\end{cases}
\]
\item Suppose we are not in the above two cases.
We define $\cP' = \cP_\taupna$ and $\cQ' = \cQ_\taupna$.
\end{enumC}

We define  the descent of $\tau$ by 
\[
    \tau' :=\DD(\tau):= (\imath_\taupna,\cP')\times (\jmath_\taupna, \cQ')\times \alpha_\taupna.
\]

 
\begin{Example}
 \[
 \begin{array}{llcl}
 \text{ By rule (a),  } & 
 \ytb{c}\times \ytb{d\none}\times B^+  & \text{ descents to } &
 \ytb{s} \times \emptyset \times \wtC\\[1em]
  \text{ By rule (b),  } & \ytb{{\none[\emptyset]}} \times \ytb{dd}\times 
  B^+ &  \text{ descents to } &
 \ytb{r} \times \emptyset \times \wtC
 \end{array}
 \]

\end{Example}
}
% 

%%\subsubsection{The quaternionic case}%{Descent from $C^*$ to $D^*$}
%%
%%
%%We begin with the simplest case when  $\star=C^*$.
%%\begin{prop}
%%Suppose that $\star=C^*$. There is a unique element $\tau'$ as in \eqref{taudes2}   such that 
%%  for all $(i,j)\in \BOX(\jmath')$, 
%%  \[ \cQ'(i,j) = \begin{cases}
%%   \bullet,\  & \text{ if } \ \cQ(i+1,j)\in\{\bullet,s\};\\
%%      r,\   & \text{ if }\   \cQ(i+1,j)=r.
%%    \end{cases}
%%    \]
%%\end{prop}
%%
%%
%%
%%Suppose that $\tau\in \pbp_{\star}(\ckcO)$. Its  descent 
%%is the unique element 
%%\[
%%\tau':= \DD(\tau) =(\imath',\cP')\times(\jmath',\cQ')\times \star'\in \pbp_{\star'}(\ckcO')
%%\]
%%such that
%%\begin{itemize}
%%    \item $\imath' = \imath$;
%%    \item $\jmath' = \DD(\jmath) $; $\ $ and 
%%    \item for all $(i,j)\in \BOX(\jmath')$, \[ \cQ'(i,j) = \begin{cases}
%%      r,  & \text{ if } \cQ(i+1,j)=r,\\
%%      \bullet, & \text{ otherwise}.
%%    \end{cases}
%%    \]
%%    \end{itemize}
%%We remark that $\cP'$ is determined by the above conditions. 
%%\begin{Example}
%%    \[
%%    \text{If } \quad \tau = \ytb{\bullet\bullet\bullet,\bullet\bullet,\bullet\bullet,\bullet,\bullet,\none}
%%    \times \ytb{\bullet\bullet\bullet,\bullet\bullet{s},\bullet\bullet,\bullet{r},\bullet,s}\times C^*, \qquad
%%    \text{ then }\quad
%%     \tau' = \DD(\tau) = \ytb{\bullet\bullet\bullet,\bullet\bullet,\bullet s,s,s,\none}
%%    \times \ytb{\none\bullet\bullet,\none\bullet{\bullet},\none\bullet,\none{r},\none,\none}
%%     \times D^*.
%%    \]
%%    Note that the nonzero row lengths of $\ckcO_\tau$ are 
%%    $13,9,9,5,5,1,1$.
%%\end{Example}
%%
%%\subsubsection{Descent from $D^*$ to $C^*$}
%%Now assume $(\star,\star')=(D^*,C^*)$.
%%Suppose that $\tau\in \pbp_{\star}(\ckcO)$ the descent of $\tau$
%%is the unique element 
%%\[
%%\tau':=\DD(\tau) =(\imath',\cP')\times(\jmath',\cQ')\times \star'\in \pbp_{\star'}(\ckcO')
%%\]
%%such that
%%\begin{itemize}
%%    \item $\imath' = \DD(\imath)$
%%    and $\jmath' = \jmath$,
%%    \item $\cQ'(i,j) = \begin{cases}
%%      r & \text{if } \cQ(i+1,j)=r,\\
%%      \bullet & \text{if} \cQ(i+1,j)\neq r \text{ and } (i,j)\in \Box(\imath'),\\
%%      s & \text{otherwise.}
%%    \end{cases}\quad
%%    \text{for each } \ (i,j)\in \BOX(\jmath').
%%    $
%%\end{itemize}
%%We remark that $\cP'$ is determined by the above conditions. 
%%\begin{lem}
%%Suppose that $\tau$
%%
%%\end{lem}
%%\noindent
%%$\star=C$. In this case,
%%
%% is nonempty
%%Put
%% \[
%%  \mathrm{PBP}^{\mathrm{ps}}_\star(\check \CO):= \{\tau\in \mathrm{PBP}_\star(\check \CO) \mid \textrm{ $\tau$ is pre-special}\}.
%%   \]
%%
%%  and the map \eqref{destau0}  has been defined.
%% In the rest of this subsection  we assume that $\mathrm{PBP}_\star(\check \CO)\neq \mathrm{PBP}^{\mathrm{ps}}_\star(\check \CO)$. Then
%%  either \[
%%  \textrm{$\star\in \{B, D\}\quad $ and $\quad (2,3)\in \mathrm{DP}_\star(\check \CO)$}
%%  \]
%%  or
%%  \[
%%   \textrm{$\star\in \{C, \widetilde C\}\quad $ and $\quad (1,2)\in \mathrm{DP}_\star(\check \CO)$. }
%% \]
%% We define the map  \eqref{destau0} case by case.
%%
%% \smallskip
%%
%% \smallskip
%%
%%
%% The goal of this subsection is to define a map
%%  \be \label{destau0}
%%   \mathrm{PBP}_\star(\check \CO) \rightarrow  \mathrm{PBP}^{\mathrm{ps}}_\star(\check \CO) , \quad \tau\mapsto \tau^{\mathrm{ps}}.
%% \ee
%% Firstly, for every  $\tau \in \mathrm{PBP}^{\mathrm{ps}}_\star(\check \CO)$, we define $\tau^{\mathrm{ps}}:=\tau$. In particular, the map \eqref{destau0} has been defined when
%% $\mathrm{PBP}_\star(\check \CO)= \mathrm{PBP}^{\mathrm{ps}}_\star(\check \CO)$.
%%
%%   \noindent $\star=B$:
%%
%%
%%   \noindent $\star=C$:
%%
%%
%%
%%
%%
%%   \noindent $\star=C$:  keep the left painted Young diagram unchanged.
%%
%%    \noindent $\star=C$:  removing the first column of the left painted Young diagram.
%%
%%
%%
%%
%% \begin{lem}\label{wptauscd}
%% Suppose that $\star\in\{C^*,D^*\}$. Then $\wp_\tau=\emptyset$ for all $\tau\in  \mathrm{PBP}_\star(\check \CO)$.
%% \end{lem}
%%
%%
%%By Lemma \ref{wptauscd}, $\mathrm{PBP}_\star(\check \CO)= \mathrm{PBP}^{\mathrm{ps}}_\star(\check \CO)$   when  $\star\in\{C^*,D^*\}$.
%%
%%



\smallskip

 \smallskip


 
 
\delete{
We define $\tau_\bftt$ to be the painted bipartition in $\PBP_{\star_\bftt}(\ckcO_\bftt)$ such that 
\[
\tau_\bftt = \emptyset \times \Lambda_{l',1}(\jmath_\tau,\cQ_\tau) \times  C^*
\] 
 
\noindent The case when $\star = D$: 
In this case, we define the tail $\tau_\bftt$ of $\tau$ to be the first painted bipartition in \eqref{tail0} such that the multiset $\{x_1, x_2, \cdots, x_k\}$ is the 
union of the multiset 
\[
\set{\cQ_\tau(l'+1,1),\cQ_\tau(l'+2,1),\cdots, \cQ_\tau(l,1)}
\]
with the set
\[
  \begin{cases}
 \set{c}, &
 \qquad 
  \text{if $\alpha_\tau = B^+$, and either $l'=0$ or $\cQ_{\tau}(l',1)\in \set{\bullet,s}$};  \\ 
 \set{s},&
  \qquad \text{if $\alpha_\tau = B^-$, and either $l'=0$ or $\cQ_{\tau}(l',1)\in \set{\bullet,s}$}; \\
%  \qquad\text{when } \alpha_\tau = B^-, \text{ and, } l'=0 \textrm{ or } \cQ_\tau(l',1)\in \set{\bullet,s},  \\ 
\set{\cQ_\tau(l',1)},&
\qquad  \text{otherwise.}
\end{cases}
\]





unique element in $\PBP_{\star_\bftt}(\ckcO_\bftt)$ in the first triple of 


such that the multiset of the labels appearing in the painted Young diagram $(\imath_{\tau_\bftt}, \cP_{\tau_\bftt})$
equals 
\begin{eqnarray*}
  \{\cP_{\tau_\bftt}(i,1)\mid 1\leq i\leq l-l'+1\}\smallskip
  =\begin{cases}
 \set{c,\cQ_\tau(l'+1,1),\cdots, \cQ_\tau(l,1)}, &\\
 \qquad 
  \text{if } \alpha_\tau = B^+, 
    (l',1)\in \cQ_{\tau}^{-1}(\set{\bullet,s})\cup\set{(0,1)} \\
 % \text{ and, } l'=0 \textrm{ or } \cQ_\tau(l',1)\in \set{\bullet,s},  \\ 
 \set{s,\cQ_\tau(l'+1,1),\cdots, \cQ_\tau(l,1)},\\
  \qquad \text{if } \alpha_\tau = B^-, 
    (l',1)\in \cQ_{\tau}^{-1}(\set{\bullet,s})\cup\set{(0,1)} \\
%  \qquad\text{when } \alpha_\tau = B^-, \text{ and, } l'=0 \textrm{ or } \cQ_\tau(l',1)\in \set{\bullet,s},  \\ 
\set{\cQ_\tau(l',1),\cdots, \cQ_\tau(l,1)}, \\
\qquad  \text{otherwise}
\end{cases}

\begin{eqnarray*}
  \{\cP_{\tau_\bftt}(i,1)\mid 1\leq i\leq l-l'+1\}\smallskip
  =\begin{cases}
 \set{c,\cQ_\tau(l'+1,1),\cdots, \cQ_\tau(l,1)}, &\\
 \qquad 
  \text{if } \alpha_\tau = B^+, 
    (l',1)\in \cQ_{\tau}^{-1}(\set{\bullet,s})\cup\set{(0,1)} \\
 % \text{ and, } l'=0 \textrm{ or } \cQ_\tau(l',1)\in \set{\bullet,s},  \\ 
 \set{s,\cQ_\tau(l'+1,1),\cdots, \cQ_\tau(l,1)},\\
  \qquad \text{if } \alpha_\tau = B^-, 
    (l',1)\in \cQ_{\tau}^{-1}(\set{\bullet,s})\cup\set{(0,1)} \\
%  \qquad\text{when } \alpha_\tau = B^-, \text{ and, } l'=0 \textrm{ or } \cQ_\tau(l',1)\in \set{\bullet,s},  \\ 
\set{\cQ_\tau(l',1),\cdots, \cQ_\tau(l,1)}, \\
\qquad  \text{otherwise}
\end{cases}
\end{eqnarray*}
as a multiset.  

\[
\tau_\bftt = \cP_{\tau_\bftt}\times \emptyset \times D
\]
where
\[
\cP_{\tau_\bftt}  
\begin{cases}
  \text{consists of } \set{c,\cQ_\tau(l'+1,1),\cdots \cQ_\tau(l,1)} &
  \text{if } \cQ_\tau(l',1)\in \set{\bullet,s}, \text{ and } \alpha_\tau = B^+\\ 
  \text{consists of } \set{s,\cQ_\tau(l'+1,1),\cdots \cQ_\tau(l,1)} &
  \text{if } \cQ_\tau(l',1)\in \set{\bullet,s}, \text{ and } \alpha_\tau = B^-\\ 
 = \Lambda_{l'-1,1}(\imath_\tau,\cQ_\tau) & \text{otherwise}
\end{cases}
\]
\begin{Example}
If 
\[
\tau = \ytb{rc,r,r} \times \emptyset \times D 
\quad\text{then} \quad 
\tau_\bftt = \ytb{r,r,c}\times \emptyset\times D .
\]
If 
\[
\tau = \ytb{\bullet,\none,\none}\times \ytb{\bullet c,r,d}\times B^+
\quad\text{then} \quad 
\tau_\bftt = \ytb{r,c,d}\times \emptyset\times D .
\]
\end{Example}
%\delete{
%\[
%(\bfrr_1(\ckcO_{\bftt}) , \bfrr_2(\ckcO_{\bftt})) = 
%\begin{cases}
%\bfrr_1(\ckcO)-\bfrr_2(\ckcO)+1,1 & \text{if } \star \in \set{B,D}\\
%\bfrr_1(\ckcO)-\bfrr_2(\ckcO)-1,0 & \text{if } \star = C^* 
%\text{ and } \bfrr_1(\ckcO)>\bfrr_2(\ckcO)>0\\
%\bfrr_1(\ckcO),0 & \text{if } \star = C^* \text{ and } \bfrr_2(\ckcO)=0\\
%1,0 & \text{if } \star = C^* \text{ and } \bfrr_1(\ckcO)=\bfrr_2(\ckcO)>0.\\
%\end{cases}
%\]
%}

\subsubsection{The case $\star=D$}
%Recall the definition of $l_{\star,\ckcO}$ in \eqref{lstarco}.
%Suppose $(2,3)\in \PP_\star(\ckcO)$. 
%We define
%\[
%\tau_\bftt := \Lambda_{l_{\star,\ckcO},1}  \times \emptyset \times \star_\bftt.
%\]
%Suppose $\bfrr_2(\ckcO)=\bfrr_3(\ckcO)>0$.
%(In particular, $(2,3)\notin \PP_\star(\ckcO)$.)
We define the tail $\tau_\bftt$ of $\tau$ to be the unique element in $\PBP_{\star_\bftt}(\ckcO_\bftt)$ such that 
\[
\tau_\bftt = \cP_{\tau_\bftt}\times \emptyset \times D
\]
where
\[
\cP_{\tau_\bftt}  
\begin{cases}
  \text{consists of } \set{r,\cdots, r,c, \cP_\tau(l,1)} &
  \begin{minipage}{12em}if $\bfrr_2(\ckcO)=\bfrr_3(\ckcO)>0$ and \\
  $\cP_\tau(l',1) = r$, $\cP_\tau(l',2) = c$,\\ $\cP_\tau(l,1)\in \set{r,d}$
  \end{minipage}
  \\
 = \Lambda_{l',1}(\imath_\tau,\cP_\tau) & \text{otherwise}
\end{cases}
\]

\subsubsection{The case $\star = C^*$}
We define $\tau_\bftt$ to be the painted bipartition in $\PBP_{\star_\bftt}(\ckcO_\bftt)$ such that 
\[
\tau_\bftt = \emptyset \times \Lambda_{l',1}(\jmath_\tau,\cQ_\tau) \times  C^*
\]
}







%$\delta_\wp$ is bijective. 
%Moreover, for each $\tau\in \PBP_\star(\ckcO,\wp)$,
%The following equation of signatures holds. 
%\[
%\ssign(\tau)
%=(\bfcc_2(\cO),\bfcc_2(\cO))+\ssign(\DD^2(\tau))+\ssign(\tau_\bftt).
%\]

%\begin{prop}\label{prop:PP.BD}
%Suppose $\star \in \set{D,B,C^*}$ and $(2,3)\in\PP(\ckcO)$. 
%Then the map $\delta_\wp$ is bijective. 
%Moreover, for each $\tau\in \PBP_\star(\ckcO,\wp)$,
%The following equation of signatures holds. 
%\[
%\ssign(\tau)
%=(\bfcc_2(\cO),\bfcc_2(\cO))+\ssign(\DD^2(\tau))+\ssign(\tau_\bftt).
%\]
%\end{prop}

%
%\begin{prop}\label{prop:PP.BD}
%Suppose $\star \in \set{D,B,C^*}$ and
%$\bfrr_2(\ckcO) = \bfrr_3(\ckcO)>0$
%(So $(2,3)\notin\PP(\ckcO)$). 
%Then the map $\delta_\wp$ is injective with image 
%\[
%\Set{
%(\tau'',\tau_0)  \in \PBP_\star(\ckcO,\wp'')\times \PBP_D(\ckcO_\bftt,\emptyset) 
%| 
%x_{\tau''} = d, \text{ or }
%\cP_{\tau_0}^{-1}(\set{s,c})\neq \emptyset
%%\begin{minipage}{15em}
%%$x_{\tau''} = d$ , or \\
%%$x_{\tau''} \in \set{r,c}$ and 
%%$\cP_{\tau_\bftt}^{-1}(\set{s,c})\neq \emptyset$.
%%\end{minipage}
%}
%\]
%Moreover, for each $\tau\in \PBP_\star(\ckcO,\wp)$,
%The following equation of signatures holds. 
%\[
%\ssign(\tau)
%=(\bfcc_2(\cO)-1,\bfcc_2(\cO)-1)+\ssign(\DD^2(\tau))+\ssign(\tau_\bftt).
%\]
%\end{prop}



\subsubsection{A key property in the descent case}
We retain the notation in \Cref{def:sp-nsp.D.sp}, where
\[
\cO=(C_{2k+1}=2c_{2k+1},C_{2k}=2c_{2k},C_{2k-1}=2c_{2k-1},\cdots, C_{0}) \text{
  such that } k\geq 1.
\]
Let $\cOp=\eDD(\cO)$ and $\cOpp=\eDD(\cOp)$.
The following lemma is the key property satisfied by our definition
\begin{lem}\label{lem:sp-nsp.D}
  Let $\tau^{s}$ and $\tau^{ns}$ are two representations attached to $\cO$ as in
  \Cref{def:sp-nsp.D.sp}.
  Then
  \begin{enumS}
    \item \label{lem:sp-nsp.D.1} For every $\uptau\in \drc(\tau^{s})\sqcup \drc(\tau^{ns})$, the shape
    of $\eDD^{2}(\uptau)$ is
    \[
      \taupp = (c_{2k-1},\tau_{2k-3},\tau_{1})\times (\tau_{2k-2},\cdots, \tau_{0})
    \]
    \item \label{lem:sp-nsp.D.2} Let $\cO_{1}= (2(c_{2k+1}-c_{2k}),)\in \Nil(D)$ be the trivial orbit
    of $\rO(2(c_{2k-1}-c_{2k}),\bC)$.
    We define $\delta\colon \drc(\cO)\rightarrow \drc(\cOpp)\times \drc(\cO_{1})$ by $\delta(\uptau) = (\eDD^{2}(\uptau),\bfxx_{\uptau})$.
    The following maps are bijections
    \[
      \begin{tikzcd}[row sep=0em]
        \drc(\tau^{s})\ar[r,"\delta"] & \drc(\taupp)\times \drc(\cO_{1}) &\ar[l,"\delta"'] \drc(\tau^{ns})\\
        \uptau^{s}\ar[r,maps to] & (\eDD^{2}(\uptau^{s}), \bfxx_{\uptau^{s}})&\\
        & (\eDD^{2}(\uptau^{ns}), \bfxx_{\uptau^{ns}})& \uptau^{ns}\ar[l,maps to]\\
      \end{tikzcd}
    \]
    In particular, we obtain an one-one correspondence
    $\uptau^{s}\leftrightarrow \uptau^{ns}$ such that $\delta(\uptau^{s})=\delta(\uptau^{ns})$.
    \item\label{lem:sp-nsp.D.3}
    Suppose $\uptau^{s}$ and $\uptau^{ns}$ correspond as the above such that
    $\delta(\uptau^{s})=\delta(\uptau^{ns})=(\uptaupp,\bfxx)$. Then
    \begin{equation} \label{eq:sp-nsp-sig}
      \ssign(\uptau^{s})=\ssign(\uptau^{ns})=(C_{2k},C_{2k})+\ssign(\uptaupp)+\ssign(\bfxx).
    \end{equation}
  \end{enumS}
\end{lem}
\begin{proof}
  Suppose $\uptau^{s}$ has special shape.
  the the behavior of $\uptau^{s}$ under the descent map $\eDD$ is illustrated
  as the following:
  \[\tiny
    \uptau^{s}: \hspace{1em}
    \ytb{
      {*(srcol)\cdots}\cdots\cdots,
      {*(srcol)\ast}{\ast}{\cdots},
      {*(srcol)\bullet},
      {*(srcol)\vdots},{*(srcol)\bullet},{x_1},\vdots,{x_n}}
    \times
    \ytb{{*(srcol)\cdots}\cdots\cdots,
      {*(srcol)\ast}{\ast}{\cdots},{*(srcol)\bullet},{*(srcol)\vdots},{*(srcol)\bullet},\none,\none,\none}
    \mapsto
    \uptau'^{s}
     \ytb{\none\cdots\cdots,\none{\ast}{\cdots},\none,\none,\none,\none,\none,\none}
    \times \ytb{{*(srcol)\cdots}\cdots\cdots,{*(srcol)\ast}{\ast}{\cdots},{*(srcol)s},{*(srcol)\vdots},{*(srcol)s},\none,\none,\none}
    \mapsto
    \uptaupp:
    \ytb{\none\cdots\cdots,\none{\ast}{\cdots},\none,\none,\none,\none,\none,\none}
    \times \ytb{\none\cdots\cdots,{\none}{\ast}{\cdots},\none,\none,\none,\none,\none,\none}
  \]
  Note that $\uptau^{s}$ is obtained from $\taupp$ by attaching the grey part
  consisting totally $2c_{2k}$ dots and $\bfxx$. Now the claims for $\uptau^{s}$ is
  clear.


  Now consider the descent of a non-special diagram $\uptau^{ns}$:
  \[\tiny
    \uptau^{ns}:\ytb{{*(srcol)\cdots}\cdots\cdots\cdots,{*(srcol)\ast}{y_{0}}{\ast}{\cdots},{*(srcol)s}{*(srcol)r}{y_{2}}\cdots,
      {*(srcol)\vdots}{*(srcol)\vdots},{*(srcol)s}{*(srcol)r},{x_{0}}{y_{1}},{x_{1}}{y_{3}},{\vdots},{x_{n}} }
    \times\ytb{{*(srcol)\cdots}\cdots\cdots,{*(srcol)u_{0}}{\cdots},\ ,\ ,\ ,\ ,\ ,\ ,\ }
    \mapsto
    \uptau'^{ns}:\ytb{\cdots\cdots\cdots,{y'_{0}}{\ast}{\cdots},{*(srcol)r}{y'_{2}}\cdots,
      {*(srcol)\vdots},{*(srcol)r},{y'_{1}},{y'_{3}},\ ,\ }
    \times\ytb{{*(srcol)\cdots}\cdots\cdots,{*(srcol)w_{0}}{\cdots},\ ,\ ,\ ,\ ,\ ,\ ,\ }
    \mapsto
    \uptaupp:\hspace{1em}
            \ytb{{\cdots}{\cdots}{\cdots},{x'_{0}}{\ast}{\cdots},{x'_{1}}{x'_{2}}\cdots,\none,\none,\none,\none,\none,\none}
            \times
            \ytb{{\cdots}{\cdots},{\cdots}\none,\none,\none,\none,\none,\none,\none,\none}
            % \ytb{\emptyset,\none,\none,\none,\none,\none}
  \]
  The bijection follows from the observation that 1. $\uptau^{ns}$ is obtained by
  attaching the grey part $x_{0},\cdots,x_{n}$ and $y_{0},y_{1},y_{2},y_{3}$
  to the $\ast/\cdots$ part of $\uptaupp$;
  2. the value of $(y_{0},y_{1},y_{2},y_{3})$ is completely determined by
  $(x'_{0},x'_{1},x'_{2})$ and $\bfxx=x_{1}\cdots x_{n}$.

  We leave it to the reader for checking case by case that the grey part of $\uptau^{ns}$ has signature
  $(C_{2k}-2,C_{2k}-2)$ and
  \[\ssign(x_{0}y_{0}y_{1}y_{2}y_{3})-\ssign(x'_{0}x'_{1}x'_{2})=(2,2).
  \]
 Therefore \eqref{eq:sp-nsp-sig} follows.

 \trivial[]{
   First assume that $C_{2k}=2$. Note that we can not/(or now?) assume $k=1$, consider the
   orbit $\cOpp=(2,1,1,1,1)$.
   \begin{enumPF}
     \item
     $x'_{1}=s$, now $y'_{0} = \emptyset/\bullet$ when $x'_{0}=\emptyset/s$
     \begin{enumPF}
       \item Suppose $x'_{2}\neq r$. Then $(y'_{2},y'_{1},y'_{3}) = (x'_{2},r,d)$,
       $(x_{0},y_{0},y_{2},y_{1},y_{3}) = (s,x'_{0}, x'_{2},c,d)$.
       Therefore, the sign difference is $\ssign(s,c,d)-\ssign(s)=(2,2)$.
       \item Suppose $x'_{2} = r$. Then $(y'_{2},y'_{1},y'_{3}) = (c,r,d)$.
       $(x_{0},y_{0},y_{2},y_{1},y_{3}) = (s,x'_{0},c,r,d)$.
       Therefore, the sign difference is $\ssign(s,c,r,d)-\ssign(s,r)=(2,2)$.
     \end{enumPF}
     \item $x'_{1}\neq s$.
     \begin{enumPF}
       \item Suppose $x'_{0}\neq c$.
       Then $x'_{0}=\emptyset/s/r$, $x'_{1}=r/c/d$, $(y'_{0}, y'_{2},y'_{1},y'_{3}) = (\emptyset/\bullet/r,x'_{2},r,x'_{1})$.
       \begin{enumPF}
         \item $x'_{1}=r$. Then
         $(x_{0},y_{0},y_{2},y_{1},y_{3}) = (r,x'_{0},x'_{2},c,d)$.
         Therefore, the sign difference is $\ssign(r,c,d)-\ssign(r)=(2,2)$.
         \item $x'_{1}=c$. Then  $x_{1}=d$.
         $(x_{0},y_{0},y_{2},y_{1},y_{3}) = (c,x'_{0},x'_{2},c,d)$.
         Therefore, the sign difference is $\ssign(c,c,d)-\ssign(c)=(2,2)$.
         \item $x'_{1}=d$. Then
         $(x_{0},y_{0},y_{2},y_{1},y_{3}) = (s,x'_{0},y'_{2},y'_{1},y'_{3})=(s,x'_{0},x'_{2},r,d)$.
         Therefore, the sign difference is $\ssign(s,r,d)-\ssign(d)=(2,2)$.
       \end{enumPF}
       $(x_{0},y_{0},y_{2},y_{1},y_{3}) = (s,r,x'_{2},c,d)$.
       Therefore, the sign difference is $\ssign(s,r,x'_{2},c,d)-\ssign(c,d,x'_{2})=(2,2)$.
       \item Suppose $x'_{0} = c$.
       Then $x'_{1}=d$, $(y'_{0}, y'_{2},y'_{1},y'_{3}) = (r,x'_{2},c,d)$.
       $(x_{0},y_{0},y_{2},y_{1},y_{3}) = (s,r,x'_{2},c,d)$.
       Therefore, the sign difference is $\ssign(s,r,x'_{2},c,d)-\ssign(c,d,x'_{2})=(2,2)$.
     \end{enumPF}
   \end{enumPF}
   }
 \end{proof}

\subsubsection{A key proposition in the generalized descent case}\label{sec:gd2.CD}

We now assume that $k\geq 1$ and $C_{2k}$ is odd, i.e. $\cOp\leadsto \cOpp$ is a
generalized descent.

Without loss of generality, we can assume the dot-r-c diagrams have the
following shapes with $n = (C_{2k+1}-C_{2k}+1)/2$:
\begin{equation}\label{eq:gd2.drc}
  \uptau:\tytb{
    {*(srcol)\cdots}\cdots\cdots\cdots,
    {*(srcol)\ast}{\ast}{\ast}{\ast},
    {x_{1}}{x_{0}}{\cdots}\cdots,
    \vdots,
    {x_{n}}
  }
  \times
\tytb{{*(srcol)\cdots}\cdots\cdots,
    {*(srcol)\ast}{\ast}{\cdots},
    \none,\none,\none}
  \xmapsto{\hspace{1em}\eDDo\hspace{1em}}
  \uptaup:\tytb{
    \cdots\cdots\cdots,
    {\ast}{\ast}{\ast},
    {x_{\uptaupp}}{\cdots}\cdots,
    \none,
    \none
  }
  \times
\tytb{{*(srcol)\cdots}\cdots\cdots,
    {*(srcol)\ast}{\ast}{\cdots},
    \none,\none,\none}
  \xmapsto{\hspace{1em}\eDDo\hspace{1em}}
  \uptaupp:\tytb{
    \cdots\cdots\cdots,
    {\ast}{\ast}{\ast},
    {x_{\uptaupp}}{\cdots}\cdots,
    \none,
    \none
  }
  \times
\tytb{\cdots\cdots,
    {\ast}{\cdots},
    \none,\none,\none}
\end{equation}

The diagram $\uptau$ is obtained from the diagram of $\uptau$ by adding
$\bullet$ at the
grey parts, changing $x_{\uptaupp}$ to $x_{0}$ and attaching $x_{1}\cdots x_{n}$.


We define
\[
  \bfpp_{\uptau}:=\tytb{{x_{1}}{x_{0}},\vdots,{x_{n}}}
\]
and call $\bfpp_{\uptau}$ the peduncle part of $\uptau$. Let
$\cO_{1} = (C_{2k+1}-C_{2k}+1,)\in \dpeNil(D)$. We define
$\bfuu_{\uptau}\in \drc(\cO_{1})$ by the following formula:
\begin{equation}\label{eq:def.u}
  \bfuu_{\uptau}:=u_{1} \cdots u_{n} =
  \begin{cases}
    r\cdots r c, & \text{when } \bfpp_{\uptau} = \tytb{rc,\vdots,r}\\
    r\cdots c d, & \text{when } \bfpp_{\uptau} = \tytb{rc,\vdots,d}\\
    x_{1}\cdots, x_{n} & \text{otherwise}.
  \end{cases}
\end{equation}


Let $\tcO = (C_{2k+1}-C_{2k}+1,1,1,)\in \dpeNil(D)$ and
$\tcOpp = \eDD^{2}(\tcO) = (2)\in \dpeNil(D)$. Then
$\bfpp_{\uptau}\in \drc(\tcO)$ and $x_{\uptau}\in\dpeNil(D)$.
There is some restriction on the possibilities  of $\bfuu_{\uptau}$.

\begin{lem}\label{lem:u}
  Let
  \[
    \uptau = \tytb{{x_{1}}{x_{0}},\vdots, {x_{n}}}\in \drc(\tcO)
    \quad \text{and}\quad \uptaupp:=\eDDo^{2}(\uptau).
  \]
  Then we have the following properties which is easy to verify.
  \begin{enumS}
    \item  If $\uptaupp = r$, then $s \in \bfuu_{\uptau}$ or
    $c \in \bfuu_{\uptau}$. In particular, $\ssign(\bfuu_{\uptau})\succ (0,1)$.
    If $\ssign(\bfuu_{\uptau})\nsucc (0,2)$, we have
    \[
      \uptau = \tytb{rc, \vdots,r} \quad \text{and}\quad  \bfuu_{\uptau} = \tytb{r,\vdots,c}.
    \]
    \item  If $\uptaupp = c$, then $s \in \bfuu_{\uptau}$ or
    $c \in \bfuu_{\uptau}$. In particular, $\ssign(\bfuu_{\uptau})\succeq (0,1)$.
    \item If $\uptaupp = d$, there is no restriction on $\bfuu_{\uptau}$.
    \item  We have $x_{n}=d$ if and only $d\in \bfuu_{\uptau}$.
    \item If $x_{n}=d$ and $\uptaupp =r/c$,  we have $n\geq 2$ and
    $\ssign(\bfuu_{\uptau})\succeq (0,2)$.
    \item If $\bfuu_{\uptau} = r\cdots r$ or $r\cdots rd$, we have $\uptaupp=d$.\qedhere
  \end{enumS}

\end{lem}


\begin{lem}\label{lem:gd.inj}
  The map $\delta\colon  \drc(\cO)\rightarrow \drc(\cOpp) \times \drc(\cO_{1})$
  given by $\uptau\mapsto (\eDDo^{2}(\uptau),\bfuu_{\uptau})$ is injective.
  Moreover,
  \[
  \ssign(\uptau) =\ssign(\uptaupp) + (C_{2k}-1, C_{2k}-1)+\ssign(\bfuu_{\uptau}).
  \]
  The map $\tdelta\colon \drc(\cO)\rightarrow \drc(\cOpp) \times \bN^{2}\times \bZ/2\bZ$
  given by $\uptau\mapsto (\eDDo^{2}(\uptau), \ssign(\uptau),\upepsilon_{\uptau})$ is injective.
\end{lem}
\begin{proof}
  By our algorithm,
  \[
    (x_{\uptaupp},\bfuu_{\uptau}) = (x_{\eDDo^{2}(\bfpp_{\uptau})}, \bfuu_{\bfpp_{\uptau}}).
  \]
  The the injectivity of $\delta$ and the siginature formula follows directly
  from the definition of $\bfuu_{\uptau}$.
  Now the injectivity of $\tdelta$ follows from the injectivity of
  $\bfuu_{\uptau}\mapsto (\ssign(\bfuu_{\uptau}), \upepsilon)$ by
  \Cref{c:init.CD}.
  % Let $\uptaupp:=\eDDo^{2}(\uptau)$ and
  % Then $\uptau\mapsto (\uptaupp,\bfuu_{\uptau})$ is injective and
  % \[
  % %\ssign(\uptau) =\ssign(\uptaupp) + 2(n_1,n_{1})+\ssign(\bfuu_{\uptau}) \text{ where }2n_{1}+1 = C_{2k}.
  % \ssign(\uptau) =\ssign(\uptaupp) + (C_{2k}-1,C_{2k}-1)+\ssign(\bfuu_{\uptau}) \text{ where }2n_{1}+1 = C_{2k}.
  % \]
  % Now the lemma follows from
  % $\bfuu_{\uptau}\mapsto (\ssign(\bfuu_{\uptau}), \upepsilon)$ is injective by \Cref{c:init.CD}.
\end{proof}


\section{Type B/M case}

\subsection{Definition of $\eDD$ for type B and M}
\def\taulf{\uptau_{L,1}}
\def\taurf{\uptau_{R,1}}
\subsubsection{The defintion of $\upepsilon$.} \label{sec:upepsilon.BM}
\begin{enumerate}[label=(\arabic*).,series=alg2]
  \item When $\uptau\in \drc(B)$, $\upepsilon$ is determined by the ``basal
  disk'' of $\uptau$ (see \eqref{eq:x.uptau.BM} for the definition of $x_{\uptau}$):
  \[
    \upepsilon_{\uptau}:=
    \begin{cases}
      0, & \text{if $x_{\uptau}=d$;} \\
      1, & \text{otherwise.}
    \end{cases}
  \]
  \item When $\uptau\in \drc(M)$, the  $\upepsilon$  is determined by
  the lengths of $\taulf$ and $\taurf$:
  \[
    \upepsilon_{\uptau} :=
    \begin{cases}
      0, & \text{if }\abs{\taulf} - \abs{\taurf} \geq  0;\\
      1, & \text{if }\abs{\taulf} - \abs{\taurf}<0.
    \end{cases}
  \]
\end{enumerate}


\subsubsection{Initial cases}

\begin{enumerate}[resume*=alg2]
  \item Suppose $\cO = (2c_{1}+1)$ has only one column. Then
        $\cOp_{0}:=\eDD(\cO)$ is the trivial orbit of $\Mp(0,\bR)$. $\drc(\cO)$
        consists of diagrams of shape
        $\tau_{L}\times \tau_{R} =\emptyset\times (c_{1},)$.
        The set $\drc(\cOp_{0})$ is a singleton $\set{\uptaup_{0}=\emptyset\times \emptyset}$, and every element
        $\uptau \in \drc(\cO)$ maps to $\uptau_{0}$ (see \eqref{eq:uptau0}):
        \[\tiny
          \drc(\cO)\ni \uptau:= \hspace{1em} \ytb{\emptyset,\none,\none,\none}
          \times \ytb{{x_{1}},\vdots,{x_{n}},{m_{\uptau}}}
          \mapsto \uptaup_{0}:=\emptyset\times \emptyset
        \]
        Here $m_{\uptau}=a$ or $b$ is the mark of $\uptau$.
        We define
        \[
          \bfpp_{\uptau}:=\bfxx_{\uptau} := x_{1}\cdots x_{n} m_{\uptau}.%  \text{ and }x_{\uptau}:=x_{n}
        \]
        We call $\bfpp_{\uptau}=\bfxx_{\uptau}$ the ``peduncle'' part of
        $\uptau$.
\end{enumerate}

\subsubsection{The descent from M to B}
The descent of a special shape diagram is simple, the descent of a non-special shape
reduces to the corresponding special one:
\begin{enumerate}[resume*=alg2]
  \item Suppose $\abs{\taulf} \geq \abs{\taurf} $, i.e. $\uptau$ has special
        shape. Keep $r,d$ unchanged, delete $\taulf$ and fill the remaining part
        with ``$\bullet$'' and ``$s$'' by $\DDr(\uptau^{\bullet})$ using the
        dot-s switching algorithm. The mark $m_{\uptaup}$ of
        $\uptaup:=\eDDo(\uptau)$ is given by the following formula:
        % \footnote{Note that $\taulf$ ends with $s$ or $c$ since
        % $\abs{\taulf}>\abs{\taurf}$}
        \[
        m_{\uptaup}:= \begin{cases}
          a & \text{if  $\taulf$ ends with $s$ or $\bullet$},\\
          b & \text{if $\taulf$ ends with $c$}.
        \end{cases}
        \]

  \item Suppose $\abs{\taulf} < \abs{\taurf}$, i.e. $\uptau$ has non-special
        shape. We define
        \[\eDDo(\uptau):=\eDDo(\uptau^{s})\]
        where $\uptau^{s}$ is the special diagram corresponding to $\uptau$
        defined in \Cref{def:sp-nsp.M}.
\end{enumerate}

\begin{lem}\label{lem:ds.BM}
  Suppose $\cO = (C_{2k},C_{2k-1}, \cdots, C_{0})$ with $k\geq 1$ and $C_{2k}$
  is odd.
  Let $\cOp := \eDD(\cO)=(C_{2k-1},\cdots, C_{0})$.
  Then
  \[
    \begin{split}
      \drcs(\cO) &\xrightarrow{\hspace{2em}\eDDo\hspace{2em}} \drc(\cOp) \\
      \drcns(\cO)&
      \xrightarrow{\hspace{2em}\eDDo\hspace{2em}} \drc(\cOp)
    \end{split}
  \]
  are a bijections.
\end{lem}
\begin{proof}
  The claim for $\drcs(\cO)$ is clear by the definition of descent.
  The claim for $\drcns(\cO)$ reduces to that of $\drcs(\cO)$ using \Cref{def:sp-nsp.M}.
\end{proof}

\begin{lem}\label{lem:gd.BM}
  Suppose $\cO = (C_{2k},C_{2k-1}, \cdots, C_{0})$ with $k\geq 1$ and $C_{2k}$ even.
  Let $\cOp := \eDD(\cO)=(C_{2k-1}+1,\cdots, C_{0})$.
  Then the following map is a bijection
  \[
    \drc(\cO) \xrightarrow{\hspace{2em}\eDDo\hspace{2em}}
    \Set{\uptaup\in \drc(\cOp)| \begin{minipage}{11em}
        $m_{\uptaup}\neq b$ and\\
         $\uptaup_{R,0}$ dose not ends with $s$.
      \end{minipage}
    }. %\subsetneq \drc(\cOp).
  \]
\end{lem}
\begin{proof}
  This is clear by the descent algorithm.
\end{proof}


\subsubsection{Descent from $B$ to $M$}
Now we define the general case of the descent from type $B$ to type $M$.
We assume that $k\geq 1$ i.e. $\cO$ has at least 3 columns.
First note that the shape of $\uptau' = \eDD(\uptau)$ is the shape of $\uptau$ deleting the
most left column on the right diagram.

% We assume $\uptau$ and $\uptau'$ has the following shape where $(y_{1},y_{2})$
% could be empty and $w_{0}$ is non-empty if $\uptau_{R}\neq \emptyset$.
The definition splits in cases below. In all these cases we define
\begin{equation}\label{eq:x.uptau.BM}
 % \bfpp_{\uptau}:=
 \tbfxx_{\uptau} := x_{1}\cdots x_{n}m_{\uptau} \text{ and } \txx_{\uptau} := x_{n}
\end{equation}
which is marked by $s/r/c/d$.
For the part marked by $*/\cdots$ , $\eDD$ keeps $r,c,d$ and maps the rest part consisting of $\bullet$ and $s$ by dot-s switching algorithm.
\begin{enumerate}[resume*=alg2]
  \item When $C_{2k}$ is odd and $\uptau$ has special shape, the descent is given by the following diagram
      \[
        \uptau: \hspace{1em}
        \tytb{
        \cdots\cdots\cdots,
        {\ast}{\cdots}{\cdots},
        {*(srcol)\bullet},
        {*(srcol)\vdots},{*(srcol)\bullet},{y_0},\none,\none,\none}
      \times
      \tytb{\cdots\cdots\cdots,
        {\ast}{\ast}{\cdots},
        {*(srcol)\bullet},
        {*(srcol)\vdots},
        {*(srcol)\bullet},
        {x_{1}},{\vdots},{x_{n}},{m_{\uptau}}}
        \mapsto
       \uptaup: \tytb{\cdots\cdots\cdots,{\ast}{\cdots}{\cdots},{*(srcol)s},{*(srcol)\vdots},{*(srcol)s},{y'_{0}},\none,\none,\none}
        \times \tytb{\none\cdots\cdots,\none{\ast}{\cdots},\none,\none,\none,\none,\none,\none,\none}
      \]
      Here the grey columns has length $(C_{2k}-C_{2k-1})/2$ and $n = (C_{2k+1}-C_{2k})/2$.
      The  ``$\ast/\cdots$'' part of $\uptaup$ is given by keeping the corresponding
      entries marked by $r/d/c$ in
       $\uptau$ unchange and filling $s/\bullet$ accordingly in the rest of the
       entries.
       The entry $y'_{0}$ of $\uptaup$ is given by the following formula:
       % \footnote{When $x_{1}\cdots x_{n}=r\cdots r$, we have $y_{0}=c$.
       %   Otherwise $x_{1}=s$ or $x_{1}\cdots x_{n} = r\cdots r d$.}
       % \[
       %   y'_{0} := \begin{cases}
       %     s & \text{if $x_{1}=\bullet$ or
       %       $x_{1}\cdots x_{n} m_{\uptau}= r\cdots ra$ or $r\cdots rda$}\\
       %     c & \text{if $x_{1}=s$ or
       %       $x_{1}\cdots x_{n} m_{\uptau}= r\cdots rb$ or $r\cdots rdb$}.
       %   \end{cases}
       % \]
       \[
         y'_{0} := \begin{cases}
           s & \text{if $x_{1}=\bullet$ or
             $(x_{1},m_{\uptau})=(r/d, a)$}\\
           c & \text{if $x_{1}=s$ or
             $(x_{1},m_{\uptau})=(r/d, b)$}.
         \end{cases}
       \]
       % We define $\tbfxx_{\uptau}$ be the diagram by replacing $\bullet$ to $s$
       % in $\bfxx_{\uptau}$.
       %  We define $\tfbfxx_{\uptau}\in \DRC()$
  \item When $C_{2k}$ is odd and $\uptau$ has non-special shape, then
        \[
        \uptau: \hspace{1em}
        \tytb{
        {\cdots}\cdots\cdots,
        {\ast}{\cdots}{\cdots},
        \none,\none,\none,\none,\none,\none,\none}
        \times
        \tytb{\cdots\cdots\cdots,
        {\ast}{\ast}{\cdots},
        {*(srcol)s}{*(srcol)r},
        {*(srcol)\vdots}{*(srcol)\vdots},
        {*(srcol)s}{*(srcol)r},
        {x_{1}}{x_{0}},{\vdots},{x_{n}},{m_{\uptau}}}
        \mapsto
        \uptaup: \tytb{\cdots\cdots\cdots,{\ast}{\cdots}{\cdots},\none,\none,\none,\none,\none,\none,\none}
        \times \tytb{
        \none\cdots\cdots,\none{\ast}{\cdots},
        \none{*(srcol)r},
        \none{*(srcol)\vdots},
        \none{*(srcol)r},
        \none{x'_{0}},\none,\none,\none}
        \]
        Here the grey columns has length $(C_{2k}-C_{2k-1})/2$ and $n = (C_{2k+1}-C_{2k})/2$.
        The  ``$\ast/\cdots$'' part of $\uptaup$ is given by keeping the corresponding
        entries marked by $r/d/c$ in
        $\uptau$ unchange and filling $s/\bullet$ accordingly in the rest of the
        entries.
        The entry $x'_{0}$ of $\uptaup$ is given by the following formula:
        % \footnote{When $x_{1}\cdots x_{n}=r\cdots r$, we have $x_{0}=d$.}
        \[
        x'_{0} := \begin{cases}
          r & \text{if $(x_{1},m_{\uptau})=(r/d, a)$,}\\
          d & \text{if $(x_{1},m_{\uptau})=(r/d, b)$,}\\
          x_{0} & \text{if $x_{1}=s$.}
        \end{cases}
        \]
        Note that
        % \[
        %   x'_{0} := \begin{cases}
        %     r & \text{if $x_{1}\cdots x_{n}= r\cdots r$}\\
        %     x_{0} & \text{otherwise}
        %   \end{cases}
        % \]
        % We define $\tbfxx_{\uptau}:=\bfxx_{\uptau}$.
        %
  \item When $C_{2k}=C_{2k-1}$ is even, we could assume $\uptau$ and $\uptaup$ hase
        the following forms with $n = (C_{2k+1}-C_{2k}+1)/2$.
      \[
        \uptau: \hspace{1em}
        \tytb{
        {\cdots}\cdots\cdots,
        {\ast}{\cdots}{\cdots},
        {y_{0}}\cdots,\none,\none,\none}
      \times
      \tytb{\cdots\cdots\cdots,
        {\ast}{\ast}{\cdots},
        {x_{1}}{x_{0}},{\vdots},{x_{n}},{m_{\uptau}}}
        \mapsto
       \uptaup: \tytb{\cdots\cdots\cdots,{\ast}{\cdots}{\cdots},{y'_{0}}\cdots,\none,\none,\none}
        \times \tytb{\none\cdots\cdots,\none{\ast}{\cdots},\none{x'_{0}},\none,\none,\none}
      \]
      In the most of the case, $\uptau'$ is obtained from $\uptau$ by deleting
      the first column of $\uptau_{R}$, keeping $c/r/d$ and filling $s/\bullet$ accordingly.
      The exceptional cases are when $x_{1}=r/d$. In the exceptional case we
      always have $x_{0}=d$.  We define
      \[
        \begin{split}
          x'_{0} & := d, \\
          y'_{0} & := \begin{cases}
            s & \text{if $m_{\uptau}=a$},\\
            c & \text{if $m_{\uptau}=b$}.
          \end{cases}
        \end{split}
      \]
      Note that this definition is similar to that of the special
      shape diagrams in the descent case.
      We define the ``peduncle'' of $\uptau$ to be
      \[
        \bfpp_{\uptau} = \tytb{{x_{1}}{x_{0}}, \vdots, {x_{n}},{m_{\uptau}}}.
      \]
\end{enumerate}

In all the cases, let
$\cO_{1} = (2n,)$ is the trivial nilpotent of type D.
For each $\uptau\in \drc(\cO)$, we define $\bfxx_{\uptau} \in \drc(\cO_{1})$ by
require the following identity of multisets: %the entires of $\bfxx_{\uptau}$ forms the following multiset:
\[
\set{\text{entry of }\bfxx_{\uptau}} =
\begin{cases}
  \set{c, x_{2},\cdots, x_{n}} & \text{if $(x_1,m_{\uptau}) = (\bullet/s,a)$},\\
  \set{s, x_{2},\cdots, x_{n}} & \text{if $(x_1,m_{\uptau}) = (\bullet/s,b)$},\\
  \set{x_{1}, x_{2},\cdots, x_{n}} & \text{otherwise ($x_{1}=r/d$)}.
\end{cases}
\]
In the above definition we include the initial case where $\cO$ is a trivial
orbit of type B. We define $\bfxx_{\uptau}=\emptyset$ when $\cO$ is the trivial
orbit of $\rO(1,\bC)$.

We define the  ``basal disk'' of $\uptau$ as the following:
\[
  x_{\uptau} :=\begin{cases}
    % \emptyset & \text{if $\cO = (1)$}\\
    \bfxx_{\uptau} & \text{if $C_{2k+1}=C_{2k}+1$}.\\
    c & \text{if $x_{n}m_{\uptau} =sa$ or $\bullet a$}\\
    x_{n} & \text{otherwise}. % \text{if $C_{2k+1}-C_{2k}\geq 1$ and $x_{n}m_{\uptau}\neq sa$}\\
  \end{cases}
\]


\subsubsection{A key proposition for descent case}
Now we assume $\cO$ is given by \eqref{eq:B.orb.ds}
Let $\cOp=\eDD(\cO)$ and $\cOpp=\eDD(\cOp)$.



The following lemma is the key property satisfied by our definition
\begin{lem}\label{lem:sp-nsp.B}
  Suppose $k\geq 1$ and
  $\tau^{s}$ and $\tau^{ns}$ are two representations attached to $\cO$ as in
  \Cref{def:sp-nsp.B.sp}.
  Then
  \begin{enumS}
    \item \label{lem:sp-nsp.B.1} For every $\uptau\in \drc(\tau^{s})\sqcup \drc(\tau^{ns})$, the shape
    of $\eDD^{2}(\uptau)$ is
    \[
      \taupp = (\tau_{2k-2},\cdots, \tau_{2})\times  (c_{2k-1},\tau_{2k-3},\tau_{1})
    \]
    \item \label{lem:sp-nsp.B.2} % Let $\cO_{1}= (C_{2k+1}-C_{2k}+1,)\in \Nil(B)$ be the trivial orbit
    % of $\rO(C_{2k+1}-C_{2k}+1,\bC)$.
    We define $\delta\colon \drc(\cO)\rightarrow \drc(\cOpp)\times\drc(\cO_{1})$
    by $\delta(\uptau) = (\eDD^{2}(\uptau),\bfxx_{\uptau})$.
    The following maps are bijective%with the same image
    \[
      \begin{tikzcd}[row sep=0em]
        \drc(\tau^{s})\ar[r,"\delta"] & \drc(\taupp)\times \drc(\cO_{1}) &\ar[l,"\delta"'] \drc(\tau^{ns})\\
        \uptau^{s}\ar[r,maps to] & (\eDD^{2}(\uptau^{s}), \bfxx_{\uptau^{s}})&\\
        & (\eDD^{2}(\uptau^{ns}), \bfxx_{\uptau^{ns}})& \uptau^{ns}\ar[l,maps to]\\
      \end{tikzcd}
    \]
    In particular, we obtain an one-one correspondence
    $\uptau^{s}\leftrightarrow \uptau^{ns}$ such that $\delta(\uptau^{s})=\delta(\uptau^{ns})$.
    \item\label{lem:sp-nsp.B.3}
    Suppose $\uptau^{s}$ and $\uptau^{ns}$ correspond as the above such that
    $\delta(\uptau^{s})=\delta(\uptau^{ns})=(\uptaupp,\bfxx)$. Then
    \begin{equation} \label{eq:sp-nsp-sig.B}
      \ssign(\uptau^{s})=\ssign(\uptau^{ns})=(C_{2k},C_{2k})+\ssign(\uptaupp)+\ssign(\bfxx).
    \end{equation}
    % with
    % \[
    %   \tsign(\tbfxx) = \begin{cases}
    %     \ssign(x_{1}\cdots x_{n})  & \text{if $x_{1}=r/d$}\\
    %     \ssign(\tbfxx) - (0,1) & \text{otherwise.}\\
    %   \end{cases}
    % \]
    Note that $\ssign(\bfxx)\in \bN\times \bN$.
  \end{enumS}
\end{lem}
\begin{proof}
  By our assumption, we have $c_{2k+1}\geq c_{2k}>c_{2k-1}$ and
  \[
    \tau^{s} = (c_{2k},\tau_{2k-2},\cdots, \tau_{2})\times (c_{2k+1}, c_{2k-1}).
  \]
  The the behavior of $\uptau^{s}$ under the descent map $\eDD$ is illostrated
  as the following:
  \begin{equation}\label{eq:ds2.B.sp}
    \uptau^{s}: \hspace{1em}
    \tytb{
      {*(srcol)\cdots}\cdots\cdots,
      {*(srcol)\ast}{\ast}{\cdots},
      {*(srcol)\bullet},
      {*(srcol)\vdots},{*(srcol)\bullet},{x_0},\none,\none,\none}
    \times
    \tytb{{*(srcol)\cdots}\cdots\cdots,
      {*(srcol)\ast}{\ast}{\cdots},{*(srcol)\bullet},{*(srcol)\vdots},{*(srcol)\bullet},{x_{1}},\vdots,{x_{n}},{m_{\uptau}}}
    \mapsto
    \uptau'^{s}: \hspace{1em}
    \tytb{{*(srcol)\cdots}\cdots\cdots,{*(srcol)\ast}{\ast}{\cdots},{*(srcol)s},{*(srcol)\vdots},{*(srcol)s},{x'_{0}},
      \none,\none,\none}
    \times \tytb{\none\cdots\cdots,\none{\ast}{\cdots},\none,\none,\none,\none,\none,\none,\none}
    \mapsto
    \uptaupp:
    \tytb{\none\cdots\cdots,\none{\ast}{\cdots},\none,\none,\none,\none,\none,\none,\none}
    \times \tytb{\none\cdots\cdots,{\none}{\ast}{\cdots},\none{\phantom{m}\mathclap{m_{\uptaupp}}},\none,\none,\none,\none,\none,\none}
  \end{equation}
  The grey part consists of totally $2(c_{2k}-1)=C_{2k}-1$ dots.
  Hence the total signature of the gray part is $(C_{2k}-1,C_{2k}-1)$.
  Note that $\uptau^{s}$ is obtained from the ``$\ast\cdots$'' part of $\taupp$ by attaching the grey part
  and $x_{0},\cdots, x_{n},m_{\uptau}$.

  The descent of a non-special diagram $\uptau^{ns}$:
  \begin{equation}\label{eq:ds2.B.nsp}
        \uptau^{ns}: \hspace{1em}
        \tytb{
        {*(srcol)\cdots}\cdots\cdots,
        {*(srcol)\ast}{\ast}{\cdots},
        \none,\none,\none,\none,\none,\none,\none}
      \times
      \tytb{{*(srcol)\cdots}\cdots\cdots,
        {*(srcol)\ast}{\ast}{\cdots},
        {*(srcol)s}{*(srcol)r},
        {*(srcol)\vdots}{*(srcol)\vdots},
        {*(srcol)s}{*(srcol)r},
        {x_{1}}{x_{0}},{\vdots},{x_{n}},{m_{\uptau}}}
        \mapsto
       {\uptaup}^{ns}: \tytb{{*(srcol)\cdots}\cdots\cdots,{*(srcol)\ast}{\ast}{\cdots},\none,\none,\none,\none,\none,\none,\none}
       \times
       \tytb{\none\cdots\cdots,\none{\ast}{\cdots},
         \none{*(srcol)r},
         \none{*(srcol)\vdots},
         \none{*(srcol)r},
         \none{x'_{0}},\none,\none,\none}
       \mapsto
       \uptaupp:
       \tytb{\none\cdots\cdots,\none{\ast}{\cdots},\none,\none,\none,\none,\none,\none,\none}
       \times \tytb{\none\cdots\cdots,{\none}{\ast}{\cdots},\none{\phantom{m}\mathclap{m_{\uptaupp}}},\none,\none,\none,\none,\none,\none}
     \end{equation}
     Here the gray parts in $\uptau$ consists of $C_{2k}-1$ marks of $\bullet$, $s$, or
     $r$ and $s$ and $r$ occur with the same multiplicity. Hence the total
     signature of the gray part is $(C_{2k}-1,C_{2k}-1)$.

     To prove the lemma, it suffice to verify the following claims which we leave to the reader:
     \begin{itemize}
       \item In both the special shape and non-special shape cases, the
       following map is bijective:
       \[
       x_{0}x_{1}\cdots x_{n}m_{\uptau^{s/ns}}\mapsto (\bfxx_{\uptau^{s/ns}},m_{\uptaupp}).
       \]
       \item The following equation of signatures holds
       \[
         \ssign( x_{0}x_{1}\cdots x_{n}m_{\uptau} ) - \ssign(m_{\uptaupp})  =\ssign(\bfxx_{\uptau}).
       \]
     \end{itemize}
     In the verification, one can use the fact that $m_{\uptau}=m_{\uptaupp}$
     when $x_{1} = r/d$.

     \trivial[]{
       Suppose $x_{1}=r/d$. Then $x_{0} = c/d$ and
       \[\ssign(x_{0}\cdots x_{n}m_{\uptau})-\ssign(m_{\uptaupp}) = (1,1)+\ssign(x_{1}\cdots x_{n}).\]

       Now we consider the generic cases, i.e. $x_{1}=\bullet/s$:
       In this case, we clain that $\ssign(x_{0}x_{1}) - \ssign(m_{\uptaupp}) = (1,2)$.
       Now
       \[
         \begin{split}
           & \ssign(x_{0}x_{1}\cdots x_{n}m_{\uptau})-\ssign(m_{\uptaupp}) \\
           =& (1,1)+\ssign(x_{2}\cdots x_{n}m_{\uptau}) +(0,1) \\
           = &\begin{cases} (1,1) +\ssign(x_{2}\cdots x_{n}) + \ssign(c)
             & \text{if } m_{\uptau} = a\\
             (1,1) +\ssign(x_{2}\cdots x_{n}) + \ssign(s) & \text{if } m_{\uptau} = b\\
           \end{cases}\\
         \end{split}
       \]

       First consider the special shape diagram $\uptau=\uptau^{s}$.
       \begin{enumPF}
         \item Suppose $m_{\uptaupp}=a$. Then $x_{0}\times x_{1}=\bullet\times \bullet$.
           \item Suppose $m_{\uptaupp}= b$. Then $x_{0}\times x_{1}=c\times s$.
       \end{enumPF}
       Now consider the non-special shape diagram $\uptau=\uptau^{ns}$.
       \begin{enumPF}
         \item Suppose $m_{\uptaupp}=a$. Then
         $\emptyset \times x_{1}x_{0}=\emptyset \times sr$.
         \item Suppose $m_{\uptaupp}= b$. Then
         $\emptyset \times x_{1}x_{0}=\emptyset \times sd$.
       \end{enumPF}
       The macthing between $\uptau^{s}$ and $\uptau^{ns}$ is also clear by the
       above listing of cases.
     }
 \end{proof}



\subsubsection{A key proposition for the generalized descent case}

Now assume $C_{2k}$ is even. By our assumption, we have
$c_{2k+1}\geq c_{2k}=c_{2k-1}$.

\begin{equation}\label{eq:gd.drc.B}
  \uptau: \hspace{1em}
  \tytb{
    {*(srcol)\cdots}\cdots\cdots,
    {y_{0}}{\cdots}{\cdots},
    \none,\none,\none,\none}
  \times
  \tytb{{*(srcol)\cdots}\cdots\cdots,
    {x_{1}}{x_{0}}{\cdots},
    {x_{2}},{\vdots},{x_{n}},{m_{\uptau}}}
  \mapsto
  \uptaup: \tytb{{*(srcol)\cdots}\cdots\cdots,{y'_{0}}{\cdots}{\cdots},\none,\none,\none,\none}
  \times \tytb{\none\cdots\cdots,\none{x'_{0}}{\cdots},\none,\none,\none,\none}
  \mapsto
  \uptaupp: \tytb{\none\cdots\cdots,{\none}{\cdots}{\cdots},\none,\none,\none,\none}
  \times \tytb{\none\cdots\cdots,\none{x''}{\cdots},\none{\phantom{m}\mathclap{m_{\uptaupp}}},\none,\none,\none}
\end{equation}
Here the gray parts in $\uptau$ are two columns of $\bullet$ of length
$C_{2k}/2-1$.

\begin{lem}\label{lem:gd.inj.M}
  In \eqref{eq:gd.drc.B}, $x''=x_{0}$.
  The map $\delta\colon  \drc(\cO)\rightarrow \drc(\cOpp) \times \drc(\cO_{1})$
  given by $\uptau\mapsto (\eDDo^{2}(\uptau),\bfxx_{\uptau})$ is injective.
  Moreover,
  \[
  \ssign(\uptau) =\ssign(\uptaupp) + (C_{2k}-1, C_{2k}-1)+\ssign(\bfxx_{\uptau}).
  \]
  The map $\tdelta\colon \drc(\cO)\rightarrow \drc(\cOpp) \times \bN^{2}\times \bZ/2\bZ$
  given by $\uptau\mapsto (\eDDo^{2}(\uptau), \ssign(\uptau),\upepsilon_{\uptau})$ is injective.
\end{lem}
\begin{proof}
  The claim $x''=x_{0}$, the injectivity of $\delta$ and the siginature formula follows directly from our algorithm.

  Now the injectivity of $\tdelta$ follows from
  $\bfxx_{\uptau}\mapsto (\ssign(\bfxx_{\uptau}), \upepsilon)$ is injective by
  \Cref{c:init.CD}.
  \trivial{
    We can fully recover $\uptau$ from $\bfxx_{\uptau}$ and $\uptaupp$.

    Suppose $\bfxx_{\uptau}$ dose not contains $s$ or $c$.
    Then $x_{1}=r/d$,  $m_{\uptau}=m_{\uptaupp}$, $y_{0}=c$.
    Hence the claim holds.
    \[
      \ssign(\uptau) =\ssign(\uptaupp) + (C_{2k}-2,C_{2k}-2) + \ssign(c) +\ssign(\bfxx_{\uptau})
    \]

    Now assume $\bfxx_{\uptau}$ contain at least one $s$ or $c$.
    Now
    \[
      m_{\uptau} = \begin{cases}
        a & \text{if $\bfxx_{\uptau}$ contains $c$}\\
        b & \text{otherwise}
      \end{cases}
    \]
    Therefore, $\ssign(x_{2}\cdots x_{n}m_{\uptau}) = \ssign(\bfxx_{\uptau}) - (0,1)$.


    Clearly, we can recover $x_{2}\cdots x_{n}$ from $\bfxx_{\uptau}$ by
    deleting the $c$ (if it exists) or a $s$.
    On the other hand, $(y_{0}, x_{1})$ is completely determined by
    $m_{\uptaupp}$:
    \[
      (y_{0},x_{1}) = \begin{cases}
        (\bullet, \bullet) & \text{if $m_{\uptaupp}=a$}\\
        (c, s) & \text{if $m_{\uptaupp}=b$}
      \end{cases}
    \]
    Hence $\ssign(y_{0}x_{1}) = \ssign(m_{\uptaupp}) + (1,2)$.

    Now
    $\ssign(y_{0}x_{0}x_{1}\cdots x_{n}m_{\uptau})  =  \ssign(\bfxx_{\uptau}) +\ssign(x'' m_{\uptaupp}) + (1,2)-(0,1)$.
    This yields the signature identity.
  }
\end{proof}


\subsection{Proof of the type BM case}
We will reduce the proof to the type CD case.

In this section, $k\geq 1$ and
$\cO = (C_{2k+1}, C_{2k}, \cdots,C_{1}, C_{0}=0)\in \dpeNil(B)$.
$\cOp:=\eDD(\cO)$ and $\cOpp := \eDD(\cOp)$.

Take $\uptau\in \drc(\cO)$, we marks the entries in $\uptau$ as in  \cref{eq:ds2.B.sp,eq:ds2.B.nsp,eq:gd.drc.B}.
\subsubsection{Usual decent case}
Thanks to \Cref{lem:sp-nsp.B}, the proof in the descent case is exactly the same
as that in \Cref{sec:pf.ds.CD}.

\subsubsection{Reduction to type CD case in the general descent case.}

Suppose $C_{2k}$ is even.
We define $\tcO = (C_{2k+1}-C_{2k}+1,1,1)\in\dpeNil(D)$ and
$\tcOpp := \eDDo(\tcO)= (2)$.

The key proposition of reduction.
\begin{prop}
 We have  the following properties:
  \begin{enumS}
    \item When $\uptaupp = \eDDo^{2}(\uptau)$ for certain $\uptau\in \drc(\cO)$,
    we have $x_{\uptaupp}\neq s$;
    \item For each $\uptaupp$ such that $x_{\uptaupp}\neq s$, we have a
    bijection
    \[
      \begin{tikzcd}[row sep=0em]
       \bdelta \colon  \set{\uptau\in \drc(\cO)| \eDDo^{2}(\uptau)=\uptaupp} \ar[r] &
        \set{(x_{\tuptau},\bfuu_{\tuptau})| \tuptau\in \drc(\tcO) \text{ s.t.
          } \eDDo^{2}(\tuptau)=x_{\uptaupp}}\\
        \uptau \ar[r,mapsto] & (x_{\uptaupp}, \bfxx_{\uptau})
      \end{tikzcd}
  \]
  where $\bfuu_{\tuptau}$ is defined in \eqref{eq:def.u}.
  \item
  We have the following properties:
  \begin{enumS}
    \item $x_{\tuptau}=s$ if and only if $x_{n}m_{\uptau} = sb$ or $\bullet b$.
    \item $x_{\tuptau}=d$ if and only if $x_{n}=d$. \qedhere
  \end{enumS}
  \end{enumS}
\end{prop}


The situation could be summarized in the following commutative diagram:
  \[
    \begin{tikzcd}
     & \uptau \ar[r,"\delta",mapsto] \ar[d,mapsto]& (\uptaupp, \bfxx_{\uptau})\ar[d,mapsto]\ar[r,mapsto] & \uptaupp\ar[d,mapsto]\\
     x_{\uptau} \ar[d,leftrightarrow,dashed] &\ar[l,mapsto]  \bfpp_{\uptau}\ar[r,mapsto]\ar[d,leftrightarrow]
     & (x_{\uptaupp}, \bfxx_{\uptau}) \ar[d,equal]\ar[r,mapsto]& x_{\uptaupp} \ar[d,equal]\\
    x_{\tuptau} & \ar[l,mapsto] \tuptau \ar[r,"\delta",mapsto]& (\tuptaupp,\bfuu_{\tuptau})\ar[r,mapsto]& \tuptaupp
    \end{tikzcd}
  \]
  We remark that $x_{\uptau}$ and $x_{\tuptau}$ are equal in the most of the
  case, especially when $n=1$. But there are exceptional cases.

  \begin{proof}
    This follows from a case by case verification according to our algorithm.
    We list all possible cases below:

    When $n=1$, see \Cref{tb:rd1}.% We have the following table:
    \begin{table}[p]
      \[
        \begin{array}{c|c|c|c|c}
          \hline
          \hline
          x_{\uptau} & \tytb{{\txx_{\uptaupp}},{m_{\uptaupp}}} & \bfpp_{\uptau} & (x_{\uptaupp},\bfxx_{\uptau}) & \tuptau_{L} \\
          \hline
          r & \tytb{r,a} &  \tytb{\bullet r,b} & (r,s) & \tytb{sr} \\
          \cline{3-5}
                     &            &  \tytb{\bullet r,a} & (r,c) & \tytb{rc} \\
          \hline
          r & \tytb{r,b} &  \tytb{sr,b} & (r,s) & \tytb{sr} \\
          \cline{3-5}
                     &            &  \tytb{sr,a} & (r,c) & \tytb{rc} \\
          \hline
          c & \tytb{s,a} &  \tytb{\bullet s,b} & (c,s) & \tytb{sc} \\
          \cline{3-5}
                     &            &  \tytb{\bullet s,a} & (c,c) & \tytb{cc} \\
          \hline
          d & \tytb{d,a} & \tytb{\bullet d,b} & (d,s) & \tytb{sd} \\
          \cline{3-5}
                     &            & \tytb{\bullet d,a} & (d,c) & \tytb{cd} \\
          \cline{3-5}
                     &            & \tytb{r d,a} & (d,r) & \tytb{rd} \\
          \cline{3-5}
                     &            & \tytb{d d,a} & (d,d) & \tytb{dd} \\
          \cline{2-5}
                     & \tytb{d,b} & \tytb{s d,b} & (d,s) & \tytb{sd} \\
          \cline{3-5}
                     &            & \tytb{s d,a} & (d,c) & \tytb{cd} \\
          \cline{3-5}
                     &            & \tytb{r d,b} & (d,r) & \tytb{rd} \\
          \cline{3-5}
                     &            & \tytb{d d,b} & (d,d) & \tytb{dd} \\
          \hline
          \hline
        \end{array}
      \]
      \caption{Reduction when $n=1$}
      \label{tb:rd1}
    \end{table}

    Now assume $n\geq 1$.
    We let $\bfxx = x_{1}x_{2}\cdots x_{n}$ and define
    \[
      \bfxx' = \begin{cases}
        \bfxx \text{ deleteing ``$c$''}  & \text{if } c\in \bfxx\\
        \bfxx \text{ deleteing ``$s$''}  & \text{if } c\notin \bfxx, s\in \bfxx\\
        \text{undefined} & \text{otherwise.}
      \end{cases}
    \]
    Now all the cases are listed in \Cref{tb:rd2}.
    % If $c\in \bfxx$, let $\bfxx'$ be the string obtained by deleting  $c$ from
    % $\bfxx$.
    % In the following table, if $c$ occures in $\bfxx$ or
    % $\bfxx'$, we  will view $c$ as $s$ and
    % arrange the order of entries when place them in $\bfpp_{\uptau}$.

    \begin{table}[p]
      \[
        \begin{array}{c|c|c|c|c|c:c}
          \hline
          \hline
          x_{\uptau} & \tytb{{\txx_{\uptaupp}},{m_{\uptaupp}}} & \bfpp_{\uptau} & x_{\uptaupp}, \bfxx_{\uptau}
          & \tuptau_{L} & \multicolumn{2}{ c}{\text{conditions}}\\
          \hline
          r & \tytb{r,a} &  \tytb{\bullet r,{\bfxx'},a} & \tytb{r{,}{\bfxx}} & \tytb{{\bfxx}r,}& x_{1}\neq r& c\in \bfxx  \\
          \cline{5-6}
                     & & & &\tytb{rc,{\bfxx'}} & x_{1}=r& \\
          \cline{3-3}\cline{4-7}
                     &            &  \tytb{\bullet r,{\bfxx'},b} & \tytb{r{,}{\bfxx}} & \tytb{{\bfxx}r}
                        & \multicolumn{2}{c}{c\notin \bfxx,s\in \bfxx} \\
          \hline
          r & \tytb{r,b} &  \tytb{sr,{\bfxx'},a} & \tytb{r{,}{\bfxx}} & \tytb{{\bfxx}r,}& x_{1}\neq r& c\in \bfxx  \\
          \cline{5-6}
                     & & & &\tytb{rc,{\bfxx'}} & x_{1}=r& \\
          \cline{3-3}\cline{4-7}
                     &            &  \tytb{sr,{\bfxx'},b} & \tytb{r{,}{\bfxx}} & \tytb{{\bfxx}r}
                        & \multicolumn{2}{c}{c\notin \bfxx,s\in \bfxx} \\
          \hline
          c & \tytb{s,a} &\tytb{\bullet s,{\bfxx'},a} & \tytb{c{,}{\bfxx}} & \tytb{{\bfxx}c}
                        & c\in \bfxx   & s\in \bfxx \text{ is automatic} \\%\multicolumn{2}{c}{c\in \bfxx}\\
          \cline{3-3}\cline{6-6}
                     &            &\tytb{\bullet s,{\bfxx'},b} &                    &
                        & c\notin \bfxx   &\\%\multicolumn{2}{c}{c\in \bfxx}\\
          \hline
          d & \tytb{d,a} &\tytb{\bullet d,{\bfxx'},a} & \tytb{d{,}{\bfxx}} & \tytb{{\bfxx}d}& \multicolumn{2}{c}{c\in \bfxx}\\
          \cline{3-3}\cline{6-7}
                     &            &\tytb{\bullet d,{\bfxx'},b} &                    &
                        & x_{1}=s & c\notin \bfxx   \\%\multicolumn{2}{c}{c\in \bfxx}\\
          \cline{3-3}\cline{6-7}
                     &            &\tytb{{\bfxx}d,a} &                    &
                        & x_{1}\neq s & c\notin \bfxx   \\%\multicolumn{2}{c}{c\in \bfxx}\\
          \hline
          d & \tytb{d,b} &\tytb{s d,{\bfxx'},a} & \tytb{d{,}{\bfxx}} & \tytb{{\bfxx}d}
                        & \multicolumn{2}{c}{c\in \bfxx}\\
          \cline{3-3}\cline{6-7}
                     &            &\tytb{s d,{\bfxx'},b} &                    &
                        & x_{1}=s & c\notin \bfxx   \\%\multicolumn{2}{c}{c\in \bfxx}\\
          \cline{3-3}\cline{6-7}
                     &            &\tytb{{\bfxx}d,b} &                    &
                        & x_{1}\neq s & c\notin \bfxx   \\%\multicolumn{2}{c}{c\in \bfxx}\\
          \hline
          \hline
        \end{array}
      \]
      \caption{Reduction when $n\geq 1$}
      \label{tb:rd2}
    \end{table}
  \end{proof}

  Thanks to $\bdelta$ defined in  the above lemma, we can use \eqref{eq:gd.ls}.
  The rest of the proof is similar to that in \Cref{sec:pf.gd.CD}.
  We leave the details to the reader.

% \begin{prop}
%   Inductively, we can establish the following properties:
%   \begin{enumS}
%     \item  The local system $\cL_{\uptau}$ has the factorization according to
%     $\txx_{\uptau}$.
%   \end{enumS}
% \end{prop}


\end{document}
