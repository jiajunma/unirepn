\documentclass[ssunip]{subfiles}

\begin{document}
 \section{The descents of painted bipartitions}\label{sec:comb}

As before, let  $\star\in \{ B, C,  D, \widetilde{C},  C^*, D^*\}$ and let $\check \CO$ be a Young diagram that has $\star$-good parity. Put
\begin{equation}\label{lstarco}
  l:=l_{\star, \check \CO}:=\begin{cases}
 \frac{\bfrr_1(\ckcO)}{2}; & \quad \textrm{if } \star\in \{B, \widetilde C\};\smallskip\\
 \frac{\bfrr_1(\ckcO)-1}{2}, &\quad \textrm{if } \star\in \{C, C^* \};\smallskip\\
 \frac{\bfrr_1(\ckcO)+1}{2}, &\quad \textrm{if } \star\in \{ D, D^*\}.\\
\end{cases}
\end{equation}
This is the length of the leading column of every element of $\mathrm{PBP}_\star(\check \CO)$. 

 
 In various context, we use $\emptyset$ to denote the empty set, the empty Young diagram or the painted Young diagram whose underlying Young diagram is empty. For every Young diagram $\imath$, its descent, which is denoted by $\nabla(\jmath)$, is defined to be the Young diagram obtained from $\jmath$ by removing the first column. By convention, $\nabla(\emptyset)=\emptyset$. 
 
 In the rest of this section, we assume that $\check \CO\neq \emptyset$, and write $\check \CO'$ for its dual descent. Write $\star'$ for the Howe dual of $\star$ so that $\check \CO'$ has $\star'$-good parity. Put
\[
l':=l_{\star', \check \CO'}
\]
    
 \subsection{Naive descents of painted bipartitions }
\def\bipartl{\mathrm{bi\cP_L}}
\def\bipartr{\mathrm{bi\cP_R}}
\def\dsdiagl{\mathrm{DS_L}}
\def\dsdiagr{\mathrm{DS_R}}
\def\DDl{\eDD_\mathrm{L}}
\def\DDr{\eDD_\mathrm{R}}


In this subsection, let $\tau=(\imath,\cP)\times (\jmath,\cQ)\times \alpha$ be a  painted bipartition such that $\star_\tau=\star$. Write $\star'$ for the Howe dual of $\star$ and put
\begin{equation} \label{eq:def.alphap}
\alpha'=\begin{cases} B^+,
& \textrm{if $\alpha = \wtC$ and $\cP_\tau(l_{\star,\ckcO},1),1) \neq c$;}\\
B^-,
& \textrm{if $\alpha = \wtC$ and $\cP_\tau(l_{\star,\ckcO},1),1)  = c$;}\\
\star', & \textrm{if $\alpha\neq \widetilde C$}. 
\end{cases}
\end{equation}
\trivial{
  \begin{equation} \label{eq:def.alphap}
    \alpha'=\begin{cases} B^+,
  & \textrm{if $\alpha=\widetilde{C}$ and $c$ does not occur in the leading column of $\tau$}; \smallskip \\
  B^-,
  & \textrm{if $\alpha=\widetilde{C}$ and  $c$ occurs in the leading column of $\tau$}; \smallskip \\
  \star', & \textrm{if $\alpha\neq \widetilde C$}. 
  \end{cases}
  \end{equation}
  }
\begin{lem}\label{lemDDn1}
  If $\star \in \set{B,C,C^*}$, then there is a unique painted bipartition of the form $\tau'= (\imath',\cP')\times (\jmath',\cQ')\times \alpha'$ with the following properties:
  \begin{itemize}
        \item $
   (\imath',\jmath')= (\imath,\DD(\jmath)); \smallskip
   $
   \item for all $(i,j)\in \BOX(\imath')$,
   \[
     \cP'(i,j)=\begin{cases}   
    \bullet \textrm{ or } s,&\textrm{ if  $\ \cP(i,j)\in \{\bullet, s\}$;} \smallskip \\
  \cP(i,j),& \textrm{ if $\ \cP(i,j)\notin \{\bullet, s\}$};\end{cases}
   \]
   \item for all $(i,j)\in \BOX(\jmath')$,
   \[
     \cQ'(i,j)=\begin{cases}   
    \bullet \textrm{ or } s,&\textrm{ if  $\ \cQ(i,j+1)\in \{\bullet, s\}$;} \smallskip \\
  \cQ(i,j+1), & \textrm{ if $\ \cQ(i,j+1)\notin \{\bullet, s\}$}.  \end{cases}
   \]
    \end{itemize} 
    \end{lem}
    
    


   \begin{proof}
    First assume that the images of $\cP$ and $\cQ$ are both contained in $\{\bullet, s\}$. Then  the image of $\cP$  is in fact contained in $\{\bullet\}$, and $(\imath, \jmath)$ is  right interlaced in the sense that 
 \[
 \mathbf{c}_1(\jmath)\geq \mathbf{c}_1(\imath)\geq \mathbf{c}_2(\jmath)\geq \mathbf{c}_2(\imath)\geq \mathbf{c}_3(\jmath)\geq \mathbf{c}_3(\imath) \geq \cdots.
 \]
 Hence $ (\imath',\jmath'):= (\imath,\DD(\jmath))$ is left interlaced in the sense that 
 \[
 \mathbf{c}_1(\imath')\geq \mathbf{c}_1(\jmath')\geq \mathbf{c}_2(\imath')\geq \mathbf{c}_2(\jmath')\geq \mathbf{c}_3(\imath')\geq \mathbf{c}_3(\jmath') \geq \cdots.
 \]
 Then it is clear that there is  unique painted bipartition of the form  $\tau'=(\imath',\cP')\times (\jmath',\cQ')\times \alpha'$ such that images of $\cP'$ and $\cQ'$ are both contained in $\{\bullet, s\}$. This proves the lemma in the special case when the images of $\cP$ and $\cQ$ are both contained in $\{\bullet, s\}$. 
 
 The proof of the lemma in the general case is easily reduced to this special case. 
   \end{proof}
    \begin{lem}\label{lemDDn2}
    If $\star \in \set{ \widetilde C, D,D^*}$, then there is a unique painted bipartition of the form $\tau'= (\imath',\cP')\times (\jmath',\cQ')\times \alpha'$ with the following properties:
  \begin{itemize}
        \item $
   (\imath',\jmath')= (\DD(\imath),\jmath); \smallskip
   $
   \item for all $(i,j)\in \BOX(\imath')$,
   \[
     \cP'(i,j)=\begin{cases}   
    \bullet \textrm{ or } s,&\textrm{ if  $\ \cP(i,j+1)\in \{\bullet, s\}$;} \smallskip \\
  \cP(i,j+1),& \textrm{ if $\ \cP(i,j+1)\notin \{\bullet, s\}$};\end{cases}
   \]
   \item for all $(i,j)\in \BOX(\jmath')$,
   \[
     \cQ'(i,j)=\begin{cases}   
    \bullet \textrm{ or } s,&\textrm{ if  $\ \cP(i,j)\in \{\bullet, s\}$;} \smallskip \\
  \cQ(i,j), & \textrm{ if $\ \cQ(i,j)\notin \{\bullet, s\}$}.  \end{cases}
   \]
  
    \end{itemize}
\end{lem}
\begin{proof}
  The proof is similar to that of \Cref{lemDDn1}. 
  
\end{proof}

\begin{defn}
 In the notation of \Cref{lemDDn1,lemDDn2}, we call $\tau'$ the naive descent of $\tau$, to be denoted by $\DDn(\tau)$.  
\end{defn} 

  

  
 \begin{Example} If
    \[
     \tau = \ytb{\bullet\bullet\bullet {c},\bullet {s} {c},{s},{c}}
    \times \ytb{\bullet\bullet\bullet ,\bullet {r} {d},{d}{d}, \none}
    \times \widetilde C, \]
   then 
   \[
    \nabla_{\mathrm{naive}}(\tau) =\ytb{\bullet\bullet{c} ,\bullet{c},\none }
    \times  \ytb{\bullet\bullet {s} ,\bullet {r} {d},{d}{d}}\times B^-.
    \]
    
\end{Example}
 
  \subsection{Descents of painted bipartitions}\label{sec:desc}
 

Suppose that 
$
\tau=(\imath,\cP)\times(\jmath,\cQ)\times \alpha \in  \mathrm{PBP}_\star(\check \CO)
$
and write 
\[
  \tau'_{\mathrm{naive}}=(\imath', \cP'_{\mathrm{naive}})\times (\jmath', \cQ'_{\mathrm{naive}})\times \alpha'
\]
for the naive descent of $\tau$. This is clearly an element of $  \mathrm{PBP}_{\star'}(\check \CO')$. 
%Put
%\begin{equation}\label{lstarco}
%  l:=l_{\star, \check \CO}:=\begin{cases}
% \frac{\bfrr_2(\ckcO)}{2}; & \quad \textrm{if } \star\in \{B, \widetilde C\};\\
% \frac{\bfrr_2(\ckcO)+1}{2}, &\quad \textrm{if } \star\in \{C, C^* \};\\
% \frac{\bfrr_2(\ckcO)-1}{2}, &\quad \textrm{if } \star\in \{ D, D^*\}.\\
%\end{cases}
%\end{equation}

The following two lemmas are easily verified and we omit the proofs. We will give an example for each of them. 
\delete{
\begin{lem}\label{descb}
Suppose that 
\[ 
\begin{cases}
\alpha = B^+; & \\
(2,3)\in \wp;\quad  &\\
\cQ(l',1)\in \set{r,d}.
\end{cases}
\]
Then there is a unique element in $\mathrm{PBP}_{\star'}(\check \CO',\wp')$ of the form
  \[
      \tau'=(\imath', \cP')\times (\jmath', \cQ')\times \alpha'
  \]
such that 
     $
     \cP' = \cP'_{\mathrm{naive}}
     $
     and 
     for all $(i,j)\in \BOX(\jmath')$, 
\[
\cQ'(i,j) = \begin{cases}
  r, & \ \text{ if  $(i,j) = (l',1)$;}\\
  \cQ'_{\mathrm{naive}}(i,j), & \ \text{ otherwise}.
\end{cases}
\]
\end{lem}


\begin{Example}
 If 
 \[
 \tau= \ytb{\bullet\bullet,\none} \times \ytb{\bullet \bullet, dd}\times 
  B^+,
 \]
 then 
\[
 \tau'_{\mathrm{naive}}= \ytb{\bullet s,\none} \times \ytb{\bullet, d}\times 
  \widetilde C\qquad\textrm{and}\qquad \tau'= \ytb{\bullet s,\none} \times \ytb{\bullet, r}\times 
  \widetilde C.
 \]
 Note that in this case, the nonzero row lengths of $\check \CO$ are $4,4,2,2$, $\wp=\{(2,3)\}$ and $l'=2$.
\end{Example}
\delete{\begin{proof}
 We only need to check that the triple $\tau'$ defined in the lemma is a painted bipartition. 
 
 Note that 
 \[
  \bar \Lambda_{l-1,2}(\imath', \cP')=\bar \Lambda_{l-1,2}(\imath'_{\mathrm{naive}}, \cP'_{\mathrm{naive}})
 \]
 and 
 \[
 \begin{array}{ccc}
 
      \Lambda_{l-1,1}(\cP_\tau)\times \Lambda_{l-1,2}(\cQ_\tau)
     &  &
        \Lambda_{l-1,1}(\cP_{\tau'})\times \Lambda_{l-1,2}(\cQ_{\tau'})\\
     \hline 
     \hspace{1em}\\
       \emptyset
      \times
      \ytb{ {x_{1}}{x_0},{\enon{\vdots}},{\enon{\vdots}},{x_{n}}}
      &
        \mapsto  &
        \emptyset 
        \times
      \ytb{ {\none}{r},{\none},{\none},\none}
      \end{array}
    \]
  
\end{proof}

Lemma \ref{descb} is easy to check and we omit the details. Note that $(\frac{\bfrr_2(\ckcO)}{2},1) \in \BOX(\jmath')$ under the first two conditions  of Lemma \ref{descb}. Similarly, we also have the following three lemmas. 
}
 }
\begin{lem}\label{descb2}
  Suppose that 
\[  \begin{cases}
 \alpha = B^+; & \\
 \bfrr_2(\ckcO)>0; & \\
 \cQ(l,1)\in \set{r,d}.
\end{cases}
\]
 Then there is a unique element in $\mathrm{PBP}_{\star'}(\check \CO')$ of the form
  \[
      \tau'=(\imath', \cP')\times (\jmath', \cQ')\times \alpha'
  \]
 such that 
     $
     \cQ' = \cQ'_{\mathrm{naive}}
     $
     and
     for all $(i,j)\in \BOX(\imath')$, 
\[
\cP'(i,j) = \begin{cases}
  s, & \ \text{ if $(i,j) = (l',1)$;}\\
  \cP'_{\mathrm{naive}}(i,j), & \ \text{ otherwise}.
\end{cases}
\]
\end{lem}

\begin{Example}
 If 
 \[
 \tau= \ytb{\bullet c, c} \times \ytb{\bullet r, rd}\times 
  B^+,
 \]
 then 
\[
 \tau'_{\mathrm{naive}}= \ytb{s c, c} \times \ytb{r, d}\times 
  \widetilde C\qquad\textrm{and}\qquad \tau'= \ytb{s c, s} \times \ytb{r, d}\times 
  \widetilde C.
 \]
 Note that in this case, the nonzero row lengths of $\check \CO$ are $4,4,4,2$, and $l'=2$.
\end{Example}

\delete{
\begin{lem}\label{descd1}
  Suppose that 
  \[  \begin{cases}
 \alpha = D; & \\
 (2,3)\in \wp;\quad  &\\
 \cP(l',1) \in \set{r,c}.
\end{cases}
\]
 Then there is a unique element in $\mathrm{PBP}_{\star'}(\check \CO',\wp')$ of the form
  \[
      \tau'=(\imath', \cP')\times (\jmath', \cQ')\times \alpha'
  \]
  such that $\cQ'=\cQ'_{\mathrm{naive}}$ and  for all $(i,j)\in \BOX(\imath')$, 
  \[
\cP'(i,j) = \begin{cases}
  r, & \ \text{ if } (i,j) = (l',1); \\
  \cP(l',1), &\  \text{ if } (i,j) = (l'+1,1);\\
  \cP'_{\mathrm{naive}}(i,j), & \ \text{ otherwise}.
\end{cases}
\]
   
\end{lem}




\begin{Example}
 If 
 \[
 \tau= \ytb{\bullet s,  c c, d d} \times \ytb{\bullet,\none, \none }\times 
  D,
 \]
 then 
\[
 \tau'_{\mathrm{naive}}=  \ytb{\bullet,  c,  d}  \times  \ytb{\bullet,\none, \none }\times 
  C,\qquad\textrm{and}\qquad \tau'= \ytb{\bullet, r, c}  \times  \ytb{\bullet,\none, \none }\times
  C.
 \]
 Note that in this case, the nonzero row lengths of $\check \CO$ are $5,5,3,1$,  $\wp=\{(2,3)\}$ and $l'=2$.
\end{Example}
}
\begin{lem}\label{descd2}
  Suppose that 
  \[  \begin{cases}
 \alpha = D; & \\
\mathbf r_2(\check \CO)=\mathbf r_3(\check \CO)>0;  &\\
\cP(l'+1,1)=r; &\\
\cP(l'+1,2)=c; &\\
 \cP(l,1)\in \set{r,d}.
\end{cases}
\]
 Then there is a unique element in $\mathrm{PBP}_{\star'}(\check \CO')$ of the form
  \[
      \tau'=(\imath', \cP')\times (\jmath', \cQ')\times \alpha'
  \]
  such that $\cQ'=\cQ'_{\mathrm{naive}}$ and  for all $(i,j)\in \BOX(\imath')$, 
  \[
\cP'(i,j) = \begin{cases}
  r, & \ \text{ if } (i,j) = (l'+1,1); \\
  \cP'_{\mathrm{naive}}(i,j), & \ \text{ otherwise}.
\end{cases}
\]
   
\end{lem}


\begin{Example}
 If 
 \[
 \tau= \ytb{\bullet\bullet, \bullet s, \bullet s, r c} \times \ytb{\bullet\bullet,\bullet,\bullet, \none }\times 
  D,
 \]
 then 
\[
 \tau'_{\mathrm{naive}}=\ytb{\bullet, \bullet , \bullet ,  c} \times \ytb{\bullet s,\bullet,\bullet, \none } \times 
  C,\qquad\textrm{and}\qquad \tau'=\ytb{\bullet, \bullet , \bullet ,  r} \times \ytb{\bullet s,\bullet,\bullet, \none } \times
  C.
 \]
 Note that in this case, the nonzero row lengths of $\check \CO$ are $7,7,7,3$,   and $l'=3$.
\end{Example}

\begin{defn}
We define the descent of $\tau$ to be 
\[
  \nabla(\tau):= \begin{cases}
  \tau', & \ \text{ if the condition of Lemma \ref{descb2}  or \ref{descd2} holds}; \\
  \nabla_{\mathrm{naive}}( \tau), & \ \text{ otherwise},
\end{cases}
\]
which is an element of $  \mathrm{PBP}_{\star'}(\check \CO')$. 
Here $\tau'$ is as in Lemmas  \ref{descb2} and \ref{descd2}. 
\end{defn}
In conclusion, we have defined the descent map
\[
\nabla: \mathrm{PBP}_{\star}(\check \CO)\rightarrow \mathrm{PBP}_{\star'}(\check \CO').
\]


\subsection{Tails of painted bipartitions}
Because of the following proposition, we assume in the rest of this paper that $\check \CO$ is quasi-distinghuished  when $\star\in \{C^*, D^*\}$. 
 

\begin{prop}
  Suppose that $\star\in \{C^*, D^*\}$. If the set $\mathrm{PBP}_\star(\check \CO)$ is nonempty, then $\check \CO$ is quasi-distinguished.  
\end{prop}
\begin{proof}
  Suppose that $\tau=(\imath,\cP)\times(\jmath,\cQ)\times \alpha \in  \mathrm{PBP}_\star(\check \CO)$. If  $\star=C^*$, then  the definition of painted bipartitions implies that 
 \[
 \bfcc_i(\imath)\leq \bfcc_i(\jmath) \qquad \textrm{for all } i=1,2,3, \cdots.
 \]
This forces that $\check \CO$ is quasi-distinguished. 
 
 If  $\star=D^*$, then  the definition of painted bipartitions implies that 
 \[
 \bfcc_{i+1}(\imath)\leq \bfcc_i(\jmath) \qquad \textrm{for all } i=1,2,3, \cdots.
 \]
This  also forces that   $\check \CO$ is quasi-distinguished.
 \end{proof}

In the rest of this subsection, we assume that $\star\in\{B, D, C^*\}$. Note that $l\geq l'$ if $\star\in \{B,C^*\}$,   and $l\geq l'+1$ if $\star=D$.
Put
\[
  \star_{\mathbf t}:= \begin{cases}
  D, & \ \text{ if $\star\in \{B,D\}$}; \\
C^*, &\  \text{ if $\star=C^*$}.
 \end{cases}
\]
Let $\ckcO_{\bftt}$ be the following Young diagram that is 
determined by the pair $(\star, \ckcO)$.
\begin{itemize}
    \item If $\star =B$,
then $\ckcO_{\bftt}$  consists of two rows with lengths $2(l-l')+1$ and $1$.
\item
If $\star =D$,
then $\ckcO_{\bftt}$  consists of two rows with lengths $2(l-l')-1$
and $1$.
\item 
If $\star =C^*$, then $\ckcO_{\bftt}$ consists of one row
with length  $2(l-l')+1$.
\end{itemize}
Note that  in all these three cases
 $\check \CO_{\mathbf t}$ has $\star_{\mathbf t}$-good parity and every element in $\PBP_{\star_\bftt}(\ckcO_\bftt)$ has the form 
 \be\label{tail0}
  \ytb{{x_1} , {x_2} , {\enon\vdots},{\enon{\vdots}},{x_k}  } \times \emptyset \times 
  D,\qquad \qquad  \ytb{{x_1} , {x_2} , {\enon\vdots},{\enon{\vdots}},{x_k}  } \times \emptyset \times 
  D\qquad\textrm{or}\qquad \emptyset \times  \ytb{{x_1} , {x_2} , {\enon\vdots},{\enon{\vdots}},{x_k}  } \times 
 C^*,
  \ee
  respectively if $\star=B, D$ or $C^*$. Here $k=l-l'+1, l-l'$ or $l-l'$ respectively. 

%\subsubsection{The case when $\star = B$}
Let 
$
\tau=(\imath,\cP)\times(\jmath,\cQ)\times \alpha \in  \mathrm{PBP}_\star(\check \CO)
$ be as before. 
 
 
\noindent {\bf The case when $\star = B$.}
In this case, we define the tail $\tau_\bftt$ of $\tau$ to be the first painted bipartition in \eqref{tail0} such that the multiset $\{x_1, x_2, \cdots, x_k\}$ is the 
union of the multiset 
\[
\set{\cQ(l'+1,1),\cQ(l'+2,1),\cdots, \cQ(l,1)}
\]
with the set
\[
  \begin{cases}
 \set{c}, &
 \qquad 
  \text{if $\alpha = B^+$, and either $l'=0$ or $\cQ(l',1)\in \set{\bullet,s}$};  \\ 
 \set{s},&
  \qquad \text{if $\alpha = B^-$, and either $l'=0$ or $\cQ(l',1)\in \set{\bullet,s}$}; \\
%  \qquad\text{when } \alpha_\tau = B^-, \text{ and, } l'=0 \textrm{ or } \cQ_\tau(l',1)\in \set{\bullet,s},  \\ 
\set{\cQ(l',1)},&
\qquad  \text{if $l'>0$ and $\cQ(l',1)\in \{r,d\}$.}
\end{cases}
\]


\smallskip

 \smallskip




 
 
\noindent {\bf The case when $\star = D$.}
In this case, we define the tail $\tau_\bftt$ of $\tau$ to be the second painted bipartition in \eqref{tail0} such that the multiset $\{x_1, x_2, \cdots, x_k\}$ is the 
union of the multiset 
\[
\set{\cP(l'+2,1),\cP(l'+3,1),\cdots, \cP(l,1)}
\]
with the set
\[
  \begin{cases}
 \set{c}, &
 \, 
  \text{if $\bfrr_2(\ckcO)=\bfrr_3(\ckcO)$, 
  $\cP(l'+1,1) = r$, $\cP(l'+1,2) = c$ and $\cP(l,1)\in \set{r,d}$};  \\ 
\set{\cP_\tau(l'+1,1)},&
\,   \text{otherwise.}
\end{cases}
\]
 
 \smallskip
 
 \smallskip
 
\noindent {\bf The case $\star = C^*$.}
In this case, we define the tail $\tau_\bftt$ of $\tau$ to be the third painted bipartition in \eqref{tail0} such that 
\[
  (x_1, x_2, \cdots, x_k)= (\cQ(l'+1,1),\cQ(l'+2,1),\cdots, \cQ(l,1)).
\]
 
 
 When $\star \in \set{B,D}$, the symbol in the last box of the tail $\tau_\bftt\in \PBP_{\star_\bftt}(\ckcO_\bftt)$ will be impotent for us. We write $x_\tau$ for it, namely
\[
x_\tau := \cP_{\tau_\bftt}(\bfcc_1(\imath_{\tau_\bftt}),1).
\]
 The following lemma is easy to check.
 
\begin{lem}\label{tailtip}
If $\star=B$, then
\[
x_\tau=s\Longleftrightarrow
\begin{cases}
  \alpha=B^-;\\ 
  \cQ(l,1) \in\{\bullet, s\}, 
  \end{cases}
%\quad \textrm{if and only if}\quad \alpha=B^- \ \textrm{ and }\  \cQ(l,1) = s, 
\]
and 
\[
x_\tau=d \Longleftrightarrow
%\quad \textrm{if and only if}\quad
\cQ(l,1) =d. 
\]
If $\star=D$, then
\[
x_\tau=s\Longleftrightarrow \cP(l,1) = s, 
\]
and 
\[
x_\tau=d\Longleftrightarrow \cP(l,1) =d. 
\]
\qed
\end{lem}







\subsection{Some properties of the descent maps}



%We state the key properties of the descent map in this section and the proofs will be given in \Cref{sec:DD.proof}.


The key properties of the descent map when $\star\in \set{C,\wtC,D^*}$ are summarized in the following proposition.   

\begin{prop}\label{prop:CC.bij}
Suppose that $\star \in \set{C,\wtC,D^*}$ and cosider the
descent map
\begin{equation}\label{eq:DD.CC}
\nabla: \PBP_\star(\ckcO)\longrightarrow  \PBP_{\star'}(\ckcOp). 
\end{equation} 

\noindent (a) If 
$\star=D^*$ or $\bfrr_1(\ckcO)>\bfrr_2(\ckcO)$, then 
the map \eqref{eq:DD.CC}  is bijective.
 
 \noindent (b) If  $\star\in \{C,\widetilde C\}$ and $\bfrr_1(\ckcO)=\bfrr_2(\ckcO)$, then the  map \eqref{eq:DD.CC} is injective and its image equals 
\[
\Set{\tau'\in \PBP_{\star'}(\ckcOp)| x_{\tau'}\neq s}.
\]

\end{prop}

\begin{proof}
We assume that $\star = \wtC$ and $\bfrr_1(\ckcO)=\bfrr_2(\ckcO)$. The proofs in the other  cases are similar and are left to the reader. 

Note that the map \eqref{eq:DD.CC} induces a map
\begin{equation}\label{eq:DD.CC1}
\nabla: \Set{\tau\in \PBP_{\star}(\ckcO)| \cP_\tau(l,1)\neq c}\rightarrow \Set{\tau'\in \PBP_{\star'}(\ckcOp)|  \alpha_{\tau'}=B^+}. 
\end{equation}
Suppose that $\tau'$ is an element  in the codomain of the map \eqref{eq:DD.CC1}. Similar to the proof of Lemma \ref{lemDDn1}, there is a unique element in $\tau:=\nabla^{-1}(\tau')\in \PBP_{\star}(\ckcO)$ such that
for all $i=1,2, \cdots,l$, 
\[
  \cP_{\tau}(i,1)\in \{\bullet, s\},
\]
and
for all $(i,j)\in \BOX(\tau')$, 
\begin{equation}
     \cP_\tau(i,j+1)=\begin{cases}   
    \bullet \textrm{ or } s,&\textrm{ if  $\ \cP_{\tau'}(i,j)\in \{\bullet, s\}$;} \smallskip \\
  \cP_{\tau'}(i,j),& \textrm{ if $\ \cP_{\tau'}(i,j)\notin \{\bullet, s\}$},\end{cases}
   \end{equation}
 and
   \begin{equation}
     \cQ_\tau(i,j)=\begin{cases}   
    \bullet \textrm{ or } s,&\textrm{ if  $\ \cQ_{\tau'}(i,j)\in \{\bullet, s\}$;} \smallskip \\
  \cQ_{\tau'}(i,j), & \textrm{ if $\ \cQ_{\tau'}(i,j)\notin \{\bullet, s\}$}.  \end{cases}
   \end{equation}
Note that $\tau$ is in the domain of \eqref{eq:DD.CC1}. It is then routine to check that the map
\[
  \nabla^{-1}: \Set{\tau'\in \PBP_{\star'}(\ckcOp)|  \alpha_{\tau'}=B^+} \rightarrow  \Set{\tau\in \PBP_{\star}(\ckcO)| \cP_\tau(l,1)\neq c}
\]
and the map \eqref{eq:DD.CC1} are inverse to each other. Hence the map \eqref{eq:DD.CC1} is bijective. 


Similarly, the map
\[
\nabla: \Set{\tau\in \PBP_{\star}(\ckcO)| \cP_\tau(l,1)= c}\rightarrow \Set{\tau'\in \PBP_{\star'}(\ckcOp)| \alpha_{\tau'}=B^-, \cQ_{\tau'}(l,1)\in\{r,d\}} 
\]
is well-defined, and we show that it is bijective by explicitly  constructing its inverse. In view of Lemma \ref{tailtip}, this proves the proposition in the case we are considering. 


\end{proof}

\medskip

The key properties of the descent map when $\star\in \set{D,B,C^*}$ are summarized in the following two propositions.   

%% The formulation of the following proposition is not correct!


%The following equation of signatures will be crucial in our computation of the local system in the next section. 


For every painted bipartition $\tau$, write
\[
  \ssign(\tau):=(p_\tau, q_\tau).
\]
When $\bfrr_2(\check \CO)>0$, the double descent $\nabla^2(\tau):=\nabla(\nabla(\tau))$ is well-defined whenever $\tau\in \mathrm{PBP}_\star(\check \CO)$. 
As in the Introduction, $\CO$ denotes the Barbasch-Vogan dual of $\check \CO$. We also consider it as a Young diagram. 

\begin{prop}\label{prop:delta}
Assume that $\star \in \set{D,B,C^*}$ and $\bfrr_2(\ckcO)>0$. Write $\ckcOpp := \ckDD(\ckcO')$ and consider the map 
\begin{equation}\label{eq:delta}
  \delta  \colon \PBP_\star(\ckcO)\longrightarrow 
    \PBP_\star(\ckcOpp)\times \PBP_{\star_\bftt}(\ckcO_\bftt),
    \qquad \tau \mapsto (\DD^2(\tau),\tau_\bftt).
\end{equation}

\noindent (a) Suppose that 
$\star = C^*$ or $\bfrr_2(\ckcO)>\bfrr_3(\ckcO)$. Then the map \eqref{eq:delta} is a bijective, and for every $\tau\in  \PBP_\star(\ckcO) $, 
    % We have the following equation of signatures.
\begin{equation}\label{eq:sign.D}
\ssign(\tau)
=(\bfcc_2(\cO),\bfcc_2(\cO))+\ssign(\DD^2(\tau))+\ssign(\tau_\bftt).
\end{equation}

\noindent (b) Suppose that  $\star \in \set{B,D}$ and $\bfrr_2(\ckcO)=\bfrr_3(\ckcO)$. Then the map \eqref{eq:delta} is  injection and its  image equals 
    \[
    \Set{ (\tau'',\tau_0)  \in \PBP_\star(\ckcOpp)\times \PBP_D(\ckcO_\bftt)  |
      \begin{minipage}{15em}
    $x_{\tau''} = d$ or \\
    $x_{\tau''}\in \set{r,c}$  and
    $\cP_{\tau_0}^{-1}(\set{s,c})\neq \emptyset$
  \end{minipage}
}.
    \]
    Moreover,  for every $\tau\in  \PBP_\star(\ckcO) $,
\begin{equation}\label{eq:sign.GD}
\ssign(\tau)
=(\bfcc_2(\cO)-1,\bfcc_2(\cO)-1)+\ssign(\DD^2(\tau))+\ssign(\tau_\bftt).
\end{equation}
\end{prop}

\begin{proof}
This follows from case by case check of the descent algorithm.
\end{proof}


\begin{prop}\label{cor:D.inj1}
Suppose that $\star \in \set{D,B,C^*}$ and $\bfrr_{2}(\ckcO)>0$.
Then the map
\begin{equation}\label{eq:D.BD}
  \delta' \colon \PBPes(\ckcO)\longrightarrow
   \PBPesp(\ckcOp)\times \PBP_{\star_\bftt}(\ckcO_\bftt)
   \qquad \tau \mapsto (\DD(\tau), \tau_\bftt)
\end{equation}
is injective. Moreover, \eqref{eq:D.BD} is bijective
unless $\star\in \set{B,D}$ and $\bfrr_2(\ckcO)=\bfrr_3(\ckcO)>0$.
% when $\star = C^*$
\end{prop}
\begin{proof}
  Suppose $\uptau_{1}, \uptau_{2} \in \PBPes(\ckcO)$ such that
  $ \delta' (\uptau_{1}) = \delta'(\uptau_{2})$. By the definition of descent,
  $\wp_{\uptau_{1}} = \wp_{\uptau_{2}}$. Now $\tau_{1} = \tau_{2}$ follows from
  the injectivity results in \Cref{prop:delta} and \Cref{prop:CC.bij}.

\end{proof}


\end{document}

%%% Local Variables:
%%% coding: utf-8
%%% mode: latex
%%% TeX-master: "ssunip.tex"
%%% TeX-engine: xetex
%%% ispell-local-dictionary: "en_US"
%%% End:
