\documentclass[ssunip]{subfiles}

\begin{document}
 \section{The descents of painted bipartitions}\label{sec:comb}

  Let  $\star\in \{ B, C,  D, \widetilde{C},  C^*, D^*\}$, and let $\check \CO$ be a Young diagrams that has $\star$-good parity. A painted bipartition $\tau\in  \mathrm{PBP}_\star(\check \CO) $ is said to be pre-special if
 \[
  \left\{
     \begin{array}{ll}
          %\textrm{the trivial representation $\C$}, &\hbox{if $\abs{\check \CO_\tau}\leq 1$}; \medskip\\
        (1,2)\notin\wp_\tau, &\hbox{when  $\star \in \{C, \widetilde C, C^* \}$;} \\
        (2,3)\notin\wp_\tau, &\hbox{when $\star\in \{ B, D, D^*\}$.} \\
            \end{array}
   \right.
\]

 Write
 \[
   \pbp_\star(\check \CO)=\pbpssp(\check \CO)\sqcup
   \pbpsns(\check \CO),
 \]
 where $\pbpssp(\check \CO)$ is the set of pre-special elements in $\pbpst(\check \CO)$, and  $\pbpsns(\check \CO)$ is the complementary set of $\pbpssp(\ckcO)$.
We remark that $\pbpsns(\ckcO)$ is  empty in the following cases:
\begin{enumerate}
\item[(a)] $\star\in \set{C^*,D^*}$;
\item[(b)] $\star\in \set{C,\wtC}$ and $(1,2)\notin \DP(\ckcO)$;
\item[(c)]  $\star \in \set{B,D}$ and $(2,3)\notin \DP(\ckcO)$.
\end{enumerate}

Let 
\[
    m_{\ckcO} := \begin{cases}
      \frac{\bfrr_1(\ckcO)-\bfrr_2(\ckcO)-2}{2} & \text{when } \star \in \set{C,\wtC,C^*}\\
      \frac{\bfrr_1(\ckcO)-\bfrr_2(\ckcO)+2}{2} & \text{when } \star \in \set{B,D,D^*}
    \end{cases} 
\]

 \subsection{Switching painted bipartitions}
  In this subsection, we assume that $\star\in \{C, \widetilde C\}$ and $(1,2)\in \mathrm{DP}_\star(\check \CO)$.
  We will define a bijective map
 \[
 \pbpsns(\ckcO)\longrightarrow \pbpssp(\ckcO), \quad
 \tau \mapsto \bartau
 \]
  in what follows.

  
\subsubsection{The case of $\star=C$}

Suppose that $\star=C$ and write $m:=m_{\ckcO}$. 

\begin{defn}[Body and tail]
\begin{enumC}
Suppose $\tau\in\pbpssp(\ckcO)$, define the tail of $\tau$ to be the  
Wfine the tail of $\tau$ to be the first
The $\tau$ has the following shape
\end{enumC}
\end{defn}
\[
  \ytb{{y_{0}}{\cdots},{*(srcol)r}{y_{2}},{*(srcol)\vdots},{*(srcol)r},{y_{1}},{y_{3}} }
    \times\ytb{{\ast}\cdots,\none,\none ,\none ,\none ,\none }\\
    \]

  Let $\wp\in \mathrm{DP}_\star(\check \CO)$. Put
 \[
   \mathrm{PBP}_\star(\check \CO, \wp)=\{\tau\in \mathrm{PBP}_\star(\check \CO)\mid \wp_\tau=\wp \}.
 \]
 We say that a painted bipartition $\tau$ is pre-special if $\wp_\tau$ has no intersection with the set $\{(1,2), (2,3)\}$. Write
 \[
   \mathrm{PBP}_\star(\check \CO)=\mathrm{PBP}^{\mathrm{ps}}_\star(\check \CO)\sqcup \mathrm{PBP}^{\mathrm{ns}}_\star(\check \CO),
 \]
 where $ \mathrm{PBP}^{\mathrm{ps}}_\star(\check \CO)$ is the set of pre-special elements in $ \mathrm{PBP}^{\mathrm{ps}}_\star(\check \CO)$, and  $\mathrm{PBP}^{\mathrm{ns}}_\star(\check \CO)$ is the complementary set.

For every painted bipartition $\tau=(\imath, \CP_{\mathrm l})\times (\jmath, \CP_{\mathrm r})\times \alpha$, we call the pained Young diagram $(\imath, \CP_{\mathrm l})$ the left painting of $\tau$, and call  $(\jmath, \CP_{\mathrm r})$ the right painting of $\tau$.


% \[
  %\mathrm{PBP}^{\mathrm{ps}}_\star(\check \CO):= \{\tau\in \mathrm{PBP}_\star(\check \CO) \mid \textrm{ $\tau$ is pre-special}\}.
   %\]



 \subsection{Descending painted bipartitions}\label{sec:desc}
Assume that $\abs{\check \CO}>0$ and let $\check \CO'$ be its descent as defined in the Introduction. Let $\tau\in  \mathrm{PBP}_\star(\check \CO)$.  In what follows we will define case by case a painted bipartition $\tau':=\nabla(\tau)\in  \mathrm{PBP}_{\star'}(\check \CO')$, to be called the descent of $\tau$.


\begin{lem}
Suppose that $\tau$

\end{lem}
\noindent
$\star=C$. In this case,

 is nonempty
Put
 \[
  \mathrm{PBP}^{\mathrm{ps}}_\star(\check \CO):= \{\tau\in \mathrm{PBP}_\star(\check \CO) \mid \textrm{ $\tau$ is pre-special}\}.
   \]





  and the map \eqref{destau0}  has been defined.
 In the rest of this subsection  we assume that $\mathrm{PBP}_\star(\check \CO)\neq \mathrm{PBP}^{\mathrm{ps}}_\star(\check \CO)$. Then
  either \[
  \textrm{$\star\in \{B, D\}\quad $ and $\quad (2,3)\in \mathrm{DP}_\star(\check \CO)$}
  \]
  or
  \[
   \textrm{$\star\in \{C, \widetilde C\}\quad $ and $\quad (1,2)\in \mathrm{DP}_\star(\check \CO)$. }
 \]
 We define the map  \eqref{destau0} case by case.

 \smallskip

 \smallskip


 The goal of this subsection is to define a map
  \be \label{destau0}
   \mathrm{PBP}_\star(\check \CO) \rightarrow  \mathrm{PBP}^{\mathrm{ps}}_\star(\check \CO) , \quad \tau\mapsto \tau^{\mathrm{ps}}.
 \ee
 Firstly, for every  $\tau \in \mathrm{PBP}^{\mathrm{ps}}_\star(\check \CO)$, we define $\tau^{\mathrm{ps}}:=\tau$. In particular, the map \eqref{destau0} has been defined when
 $\mathrm{PBP}_\star(\check \CO)= \mathrm{PBP}^{\mathrm{ps}}_\star(\check \CO)$.

   \noindent $\star=B$:


   \noindent $\star=C$:





   \noindent $\star=C$:  keep the left painted Young diagram unchanged.

    \noindent $\star=C$:  removing the first column of the left painted Young diagram.




 \begin{lem}\label{wptauscd}
 Suppose that $\star\in\{C^*,D^*\}$. Then $\wp_\tau=\emptyset$ for all $\tau\in  \mathrm{PBP}_\star(\check \CO)$.
 \end{lem}


By Lemma \ref{wptauscd}, $\mathrm{PBP}_\star(\check \CO)= \mathrm{PBP}^{\mathrm{ps}}_\star(\check \CO)$   when  $\star\in\{C^*,D^*\}$.


\section{The definition of $\eDD$ }
\def\taur{\uptau_{R}}
\def\taul{\uptau_{L}}
\def\taulf{\uptau_{L,0}}
\def\tauls{\uptau_{L,1}}
\def\taurf{\uptau_{R,0}}
\def\taurs{\uptau_{R,1}}

\def\tauplf{\uptau'_{L,0}}
\def\taupls{\uptau'_{L,1}}
\def\tauprf{\uptau'_{R,0}}
\def\tauprs{\uptau'_{R,1}}
\def\tail{\mathrm{tail}}




From now on, we assume
\[
  \uptau = (\taulf,\tauls,\cdots)\times(\taurf,\taurs,\cdots).
\]

% The definition of sign $\upepsilon$
% \begin{enumS}
%   \item Suppose $\uptau$ is type D or B, $\upepsilon=0$ if and only if
%   $\# d (\uptau)> \#d(\uptau')$. This means:
%   \begin{enumS}
%     \item When $\uptau$ is type D,  $\tail(\taulf) = d$.
%     \item When $\uptau$ is type B,   $\tail(\taurf) = d$ or
%     $(\tail(\taulf),\tail(\taurf),\tail(\taurs)) = (c,r,d)$.
%   \end{enumS}
%   \item Suppose $\uptau$ is type C or M, $\upepsilon=0$ if and only if
%   $\abs{\taulf}\geq \abs{\taurf}$.
% \end{enumS}

For any $\uptau$ let $\uptau^{\bullet}$ be the subdiagram consisting of
$\bullet$ and $s$ entries only.


\subsection{dot-s switching algorithm}
\def\bipartl{\mathrm{bi\cP_L}}
\def\bipartr{\mathrm{bi\cP_R}}
\def\dsdiagl{\mathrm{DS_L}}
\def\dsdiagr{\mathrm{DS_R}}
\def\DDl{\eDD_\mathrm{L}}
\def\DDr{\eDD_\mathrm{R}}

We say a bipartition
$\tau = \tau_{L}\times \tau_{R}$
is interlaced with lager left part if  $\tau$ has columns
$(\tau_{L,0}\geq \tau_{L,1}, \cdots, )\times (\tau_{R,0},\tau_{R,1}, \cdots,)$
such that
\[
\tau_{L,0}\geq \tau_{R,0}\geq \tau_{L,1}\geq \tau_{R,1}\geq \tau_{L,2} \geq \cdots.
\]
$\bipartl$ denote the set of all interlaced bipartitions with larger left parts.
Similarly, define $\bipartr$ be the set of bipartitoins with larger right parts:
\[
  \bipartr = \set{\tau=(\tau_{L,0}\geq \tau_{L,1}, \cdots, )\times (\tau_{R,0},\tau_{R,1}, \cdots,)|
\tau_{L,0}\geq \tau_{R,0}\geq \tau_{L,1}\geq \tau_{R,1}\geq \tau_{L,2} \geq \cdots}
\]

Let $\dsdiagl$ (resp. $\dsdiagr$) be the set of of filled diagrams with only
``$s$'' entries on the left (resp. right) diagram and the rest are all ``$\bullet$''.

Note that for each $\uptau\in \dsdiagl$, its shape $\tau$ is in $\bipartl$.
On the other hand, for each $\tau\in \bipartl$, we can fill ``$\bullet$'' in
all entries on the right diagram and corresponding entries on the left, and than fill ``$s$''
in the rest entries on the left. This procedure yields a valid dot-s diagram.
In summary, $\bipartl$ and $\dsdiagl$ are naturally bijective to each other.

Let $\DDl\colon \bipartl\rightarrow \bipartr $ (resp.
$\DDr\colon \bipartr\rightarrow \bipartl$) be the operation
of deleting the longest column on the left (resp. on the right).
Then the operation gives a well defined maps between $\dsdiagl$ and $\dsdiagr$ making
the following diagrams commute (the vertical maps are the natural identification
discussed above):
\[
\begin{tikzcd}
  \bipartl \ar[r,"\DDl"] \ar[d,equal]& \bipartr \ar[d,equal] &
  \bipartr \ar[r,"\DDr"] \ar[d,equal]& \bipartl \ar[d,equal] \\
  \dsdiagl\ar[r,dashed,"\DDl"] & \dsdiagr & \dsdiagr\ar[r,dashed,"\DDr"] & \dsdiagl\\
\end{tikzcd}
\]
By abuse of notation, we also call the dashed arrow above $\DDl$ and $\DDr$.


\subsection{Definition of $\eDD$ of Type D and C} \label{sec:alg.CD}

\subsubsection{Bijection between ``special'' and ``non-special'' diagrams for type C}
We define a bijection between ``special'' diagram $\uptau^{s}$ and
``non-special'' diagrams $\uptau$.

\begin{defn}\label{def:sp-nsp.C.sp}
  Let $\cO = (C_{2k},C_{2k-1}, \cdots, C_{1},C_{0}, C_{-1}=0)\in \dpeNil(C)$
  with $k\geq 0$.
  Let
  \[
  \tau = (\tau_{2k-1}, \cdots, \tau_{1},\tau_{-1}=0) \times (\tau_{2k}, \cdots, \tau_{0}).
  \]
  be a Weyl group representation attached to $\cO\in \dpeNil(C)$.
  We say $\tau$ has
  \begin{itemize}
    \item \idxemph{special shape} if $\tau_{2k-1}\leq \tau_{2k}+1$;
    \item \idxemph{non-special shape} if $\tau_{2k-1}> \tau_{2k}+1$.
  \end{itemize}
  Clearly, this gives a partition of the set of Weyl group representations
  attached to $\cO$.

  Suppose $k\geq 1$, $C_{2k}=2c_{2k}$ and $C_{2k-1}=2c_{2k-1}$.
  The special shape representation $\tau^{s}$
  \[
    \tau^{s} = \tau_{L}^{s}\times \tau_{R}^{s}=(c_{2k-1},\tau_{2k-3},\cdots, \tau_{1} )
    \times (c_{2k},\tau_{2k-2},\cdots, \tau_{0})
  \]
  is paired with the non-special shape representation
  \[
    \tau^{ns} = \tau_{L}^{ns}\times \tau_{R}^{ns} = (c_{2k}+1,\tau_{2k-3},\cdots, \tau_{1} )
    \times (c_{2k-1}-1,\tau_{2k-2},\cdots, \tau_{0})
  \]
  In the paring $\tau^{s}\leftrightarrow \tau^{ns}$, $\tau_{j}$  is unchanged for $j\leq 2k-2$.
  We call $\tau^{s}$ the \emph{special shape} and $\tau^{ns}$ the
  corresponding \emph{non-special shape}
\end{defn}



\begin{defn}\label{def:sp-nsp.C}
  Suppose $C_{2k}$ is even.
We retain the notation in \Cref{def:sp-nsp.C.sp} where the shapes $\tau^{s}$ and $\tau^{ns}$
correspond with each other.
We define a bijection between $\drc(\tau^{s})$ and $\drc(\tau^{ns})$:
\[
  \begin{tikzcd}[row sep=0em]
    \drc(\tau^{s}) \ar[rr,<->]&\hspace{2em} & \drc(\tau^{ns})\\
    \uptau^{s} \ar[rr,<->]& & \uptau^{ns}
  \end{tikzcd}
\]

Without loss of generality, we can assume
that $\uptau^{s}\in \drc(\tau^{s})$ and $\uptau^{ns}\in \drc(\tau^{ns})$ have the following shapes:
\begin{equation}\label{eq:sp-nsp.C}
  \uptau^{s}:\tytb{\cdots\cdots\cdots,{x_{0}}{\ast}{\cdots},{x_{1}}{x_{2}}{\cdots},\ ,\ ,\ ,\ }
  \times\tytb{\cdots\cdots\cdots,{v_{0}}{\cdots},{x_{3}},{*(srcol)s},{*(srcol)\vdots},{*(srcol)s},\ } \longleftrightarrow \hspace{2em}
  \uptau^{ns}:\tytb{\cdots\cdots\cdots,{y_{0}}{\ast}{\cdots},{*(srcol)r}{y_{2}}\cdots,
    {*(srcol)\vdots},{*(srcol)r},{y_{1}},{y_{3}} }
  \times\tytb{\cdots\cdots\cdots,{w_{0}}{\cdots},\ ,\ ,\ ,\ ,\ }
\end{equation}
(Here the grey parts have length $(C_{2k}-C_{2k-1})/2$ (could be zero length), the row contains
$x_{0}$ and $y_{0}$ may be empty, and $*$/$\cdots$ are
arbitrary entries. We remark that the value of $v_{0}$ and $w_{0}$ are
completely determined by $x_{0}$ and $y_{0}$. In particular, $v_{0}$ and $w_{0}$
are non-empty if and only if
$x_{0}/y_{0}\neq \emptyset$, i.e. $C_{2k-1}\geq 4$.)

% We marks the entries of $\uptau^{s}$ and $\uptau^{ns}$ as in
% \eqref{eq:sp-nsp.C}.
The correspondence between $(x_{0},x_{1}, x_{2}, x_3)$, with
$(y_{0},y_{1},y_{2},y_{3})$ is given by \Cref{tab:nonsp.C}.
The rest part of the diagrams are unchanged.

We define
\[
  \drcs(\cO) := \bigsqcup_{\tau^{s}} \drc(\tau^{s}) \quad \text{ and }\quad
  \drcns(\cO) := \bigsqcup_{\tau^{ns}} \drc(\tau^{ns})
\]
where $\tau^{s}$ (resp. $\tau^{ns}$) runs over all specail (resp. non-special) shape representations attached to $\cO$.
Clearly $\drc(\cO) = \drcs(\cO)\sqcup \drcns(\cO)$.
\end{defn}
It is easy to check the bijectivity in the above definition, which we leave it
to the reader.



\subsubsection{Special and non-special shapes of type D}
Now we define the notion of special and non-special shape of type D.
\begin{defn}\label{def:sp-nsp.D.sp}
  Let
  \[
  \tau = (\tau_{2k+1},\tau_{2k-1}, \cdots, \tau_{1})\times (\tau_{2k}, \cdots, \tau_{0})
  \]
  be a Weyl group representation attached to $\cO\in \dpeNil(D)$.
  We say $\tau$ has
  \begin{itemize}
    \item \idxemph{special shape} if $\tau_{2k-1}\leq \tau_{2k}+1$;
    \item \idxemph{non-special shape} if $\tau_{2k-1}> \tau_{2k}+1$.
  \end{itemize}

  Suppose
  $\cO=(C_{2k+1}=2c_{2k+1},C_{2k}=2c_{2k},C_{2k-1}=2c_{2k-1},\cdots,C_{0}\geq 0)$
  with $k\geq 1$.\footnote{Note that $C_{2k}$ and $C_{2k-1}$ are non-zero
    even integers.}
  A representation $\tau^{s}$ attached to $\cO$ has the shape
  \[
    \tau^{s} = \tau_{L}^{s}\times \tau_{R}^{s}=(c_{2k+1},c_{2k-1},\tau_{2k-3},\cdots, \tau_{1} )
    \times (c_{2k},\tau_{2k-2},\cdots, \tau_{0})
  \]
  is paired with
  \[
    \tau^{ns} = \tau_{L}^{ns}\times \tau_{R}^{ns} = (c_{2k+1},c_{2k}+1,\tau_{2k-3},\cdots, \tau_{1} )
    \times (c_{2k-1}-1,\tau_{2k-2},\cdots, \tau_{0}).
  \]
  In the paring $\tau^{s}\leftrightarrow \tau^{ns}$, $\tau_{j}$  is unchanged for $j\leq 2k-2$.
  % We call $\tau^{s}$ the \emph{special shape} and $\tau^{ns}$ the
  % corresponding \emph{non-special shape}
\end{defn}

\begin{table}[hpb]
\[\tiny
\begin{array}{c|c|c:c|c}
  & \uptaup & \uptau^{s} & \uptau^{ns} \\
  \hline
           & \ytb{{x'_{0}}{\ast},{x'_{1}}{x'_{2}},\none,\none,\none,\none}
             \times \ytb{\ast,\none,\none,\none,\none,\none}
           &\ytb{{x_{0}}{\ast},{x_{1}}{x_{2}},\ ,\ ,\ ,\ }
    \times\ytb{{\ast}{\ast},{x_{3}},{*(srcol)s},{*(srcol)\vdots},{*(srcol)s},\ }&
  \ytb{{y_{0}}{\ast},{*(srcol)r}{y_{2}},{*(srcol)\vdots},{*(srcol)r},{y_{1}},{y_{3}} }
    \times\ytb{{\ast}\ast,\ ,\ ,\ ,\ ,\ }\\
  \hline
  \ytb{{x'_{1}}{=}{s},{\text{then}},{z_{0}}{=}{\emptyset/}{s},{x_{0}}{=}{\emptyset/}{\bullet},{x_{1}}{=}{\bullet},
  {x_{2}}{=}{x'_{2}}}
  &
            \ytb{{x'_{0}}{\ast},{s}{x_{2}},\none,\none,\none,\none}
             \times \ytb{\ast,\none,\none,\none,\none,\none}
                &
                \ytb{{x_{0}}{\ast},{\bullet}{x_{2}},\ ,\ ,\ ,\ }
    \times\ytb{{\ast}\ast,{\bullet},{*(srcol)s},{*(srcol)\vdots},{*(srcol)s},\ }&
  \ytb{{x_{0}}{\ast},{*(srcol)r}{x_{2}},{*(srcol)\vdots},{*(srcol)r},{c},{d} }
                          \times\ytb{{\ast}{\ast},\ ,\ ,\ ,\ ,\ }
          &x_{2}\neq r
  \\
  \cline{3-5} & &
                  \ytb{{x_{0}}{\ast},{\bullet}{r},\ ,\ ,\ ,\ }
    \times\ytb{{\ast}\ast,{\bullet},{*(srcol)s},{*(srcol)\vdots},{*(srcol)s},\ }&
  \ytb{{x_{0}}{\ast},{*(srcol)r}{c},{*(srcol)\vdots},{*(srcol)r},{r},{d} }
    \times\ytb{{\ast}\ast,\ ,\ ,\ ,\ ,\ } & x_{2}=r \\
  \hline
\ytb{{x'_{1}}{\neq}{s},{\text{then}},{x_{1}}{=}{x'_{1}},{x_{2}}{=}{x'_{2}}}
  &
            \ytb{{x'_{0}}{\ast},{x'_{1}}{x_{2}},\none,\none,\none,\none}
             \times \ytb{\ast,\none,\none,\none,\none,\none}
   &
    \ytb{{x_{0}}{\ast},{x_{1}}{x_{2}},\ ,\ ,\ ,\ }
    \times\ytb{{\ast}\ast,{s},{*(srcol)s},{*(srcol)\vdots},{*(srcol)s},\ }&
  \ytb{{x_{0}}{\ast},{*(srcol)r}{x_{2}},{*(srcol)\vdots},{*(srcol)r},{r},{x_{1}} }
       \times\ytb{{\ast}\ast,\ ,\ ,\ ,\ ,\ } &
     \ytb{{x'_{0}}{\neq}{c},{\text{then}},{x'_{0}}{=}{\emptyset/}{s/}{r},{x_{0}}{=}{\emptyset/}{\bullet/}{r}}
  \\
  %\hline
  \cline{3-5}
   & &
    \ytb{{c}{\ast},{d}{x_{2}},\ ,\ ,\ ,\ }
    \times\ytb{{\ast}\ast,{s},{*(srcol)s},{*(srcol)\vdots},{*(srcol)s},\ }&
  \ytb{{r}{\ast},{*(srcol)r}{x_{2}},{*(srcol)\vdots},{*(srcol)r},{c},{d} }
  \times\ytb{{\ast}\ast,\ ,\ ,\ ,\ ,\ }&
  \ytb{{x'_{0}}{=}{c}\none\none,{\text{then}},{x_{0}}{=}{x'_{0}}{=}{c},{x_{1}}{=}{x'_{1}}{=}{d},{x_{2}}{=}{x'_{2}}{=}{\emptyset/d}}
  \\
  \hline
  \hline
\end{array}
\]
\caption{``special-non-special'' switch}
\label{tab:nonsp.C}
\end{table}


\subsubsection{Convention}
The induction starts with type C: For the trivial orbit $\cO$ of
$\wtG = \Sp(2c_{0},\bR)$, $\drc(\cO) = \set{\uptau_{\cO}}$ and
$\uppi_{\uptau_{\cO}}=\bfone$ is the trivial representation, see \eqref{eq:uptau0}.

\subsubsection{The defintion of $\upepsilon$.} \label{sec:upepsilon}
\begin{enumerate}[label=(\arabic*).,series=alg1]
  \item When $\uptau\in \drc(D)$, $\upepsilon$ is determined by the ``basal
  disk'' of $\uptau$ (see \eqref{eq:x.uptau} for the definition of $x_{\uptau}$):%escent from type D to type C,
  \[
    \upepsilon_{\uptau}:=
    \begin{cases}
      0, & \text{if $x_{\uptau}=d$;} \\
      1, & \text{otherwise.}
    \end{cases}
  \]
  \item When $\uptau\in \drc(C)$, the  $\upepsilon$  is determined by
  the lengths of $\taulf$ and $\taurf$:
  \[
    \upepsilon_{\uptau} :=
    \begin{cases}
      0, & \text{if $\uptau$ has special shape, i.e. } \abs{\taulf}-\abs{\taurf} \leq  1;\\
      1, & \text{if $\uptau$ has non-special shape, i.e. }\abs{\taulf} - \abs{\taurf}> 1.
    \end{cases}
  \]
\end{enumerate}
%To indicate the relation with $\uptau$, we will also w

\medskip

%{\bf Definition of the descent of diagram $\uptaup$.}

\subsubsection{Initial cases}
%For type D, %when $\cO$ must have odd number of column.
Let $\cO$ be a nilpotent orbit of type D with at most 2 columns.
\begin{enumerate}[resume*=alg1]
  \item Suppose $\cO = (2c_{1},2c_{0})$.
        Then
        $\cOp:=\eDD(\cO)$ is the trivial orbit of $\Sp(2c_{0},\bR)$. $\drc(\cO)$
        consists of diagrams of shape
        $\tau_{L}\times \tau_{R} =(c_{1},)\times (c_{0},)$.
        The set $\drc(\cOp)$ is a singleton $\set{\uptau_{\cOp}}$, and every element
        $\uptau \in \drc(\cO)$ maps to $\uptau_{\cOp}$ (see \eqref{eq:uptau0}):
        \[\tiny
          \drc(\cO)\ni \uptau: \hspace{1em} \ytb{\bullet,\vdots,\bullet,{x_{1}},\vdots,{x_{n}}}
          \times \ytb{\bullet,\vdots,\bullet,\none,\none,\none}
          \mapsto \uptau_{\cOp}:
          \ytb{\emptyset,\none,\none,\none,\none,\none}
          \times \ytb{s,\vdots,s,\none,\none,\none}
        \]
        We define
        \[
          \bfpp_{\uptau}:=\bfxx_{\uptau} := \tytb{{x_{1}},\vdots,{x_{n}}} \text{ and }x_{\uptau}:=x_{n}
        \]
        We call $\bfpp_{\uptau}=\bfxx_{\uptau}$ the ``peduncle'' part of $\uptau$ and
        $x_{\uptau}$ the
        ``basal disk'' of $\uptau$.
        Note that $\bfxx_{\uptau}$ consists of entries marked by $s/r/c/d$.
\end{enumerate}


\subsubsection{The descent from C to D}
The descent of a special shape diagram is simple, the descent of a non-special shape
reduces to the corresponding special one:
\begin{enumerate}[resume*=alg1]
  \item Suppose $\abs{\taulf} - \abs{\taurf} < 2$, i.e. $\uptau$ has special
        shape. Keep $r,c,d$ unchanged, delete $\taurf$ and fill the remaining
        part with ``$\bullet$'' and ``$s$'' by $\DDr(\uptau^{\bullet})$ using
        the dot-s switching algorithm.
  \item Suppose $\abs{\taulf} - \abs{\taurf} \geq 2$. We define
        \[\uptaup=\eDDo(\uptau):=\eDDo(\uptau^{s})\] where $\uptau^{s}$ is the
        special diagram corresponding to $\uptau$ defined in
        \Cref{def:sp-nsp.C}.
\end{enumerate}

\begin{lem}\label{lem:ds.CD}
  Suppose $\cO = (C_{2k},C_{2k-1}, \cdots, C_{0})$ with $k\geq 1$ and $C_{2k}$ even.
  Let $\cOp := \eDD(\cO)=(C_{2k-1},\cdots, C_{0})$.
  Then
  \[
    \begin{split}
      \drcs(\cO) &\xrightarrow{\hspace{2em}\eDDo\hspace{2em}} \drc(\cOp) \\
      \drcns(\cO)&
      \xrightarrow{\hspace{2em}\eDDo\hspace{2em}} \drc(\cOp)
    \end{split}
  \]
  are a bijections.
\end{lem}
\begin{proof}
  The claim for $\drcs(\cO)$ is clear by the definition of descent.
  The claim for $\drcns(\cO)$ reduces to that of $\drcs(\cO)$ using \Cref{def:sp-nsp.C}.
\end{proof}

\begin{lem}\label{lem:gd.CD}
  Suppose $\cO = (C_{2k},C_{2k-1}, \cdots, C_{0})$ with $k\geq 1$ and $C_{2k}$ odd.
  Let $\cOp := \eDD(\cO)=(C_{2k-1}+1,\cdots, C_{0})$.
  Then the following map is a bijection
  \[
      \drc(\cO) \xrightarrow{\hspace{2em}\eDDo\hspace{2em}} \set{\uptaup\in \drc(\cOp)| x_{\uptau}\neq s}. %\subsetneq \drc(\cOp).
  \]
\end{lem}
\begin{proof}
  This is clear by the descent algorithm.
\end{proof}
\subsubsection{Descent from $D$ to $C$} Now we consider the general case of the
descent from type $D$ to type $C$. So we assume $\cO$ has at least 3 columns.
First note that the shape of $\uptau' = \eDD(\uptau)$ is the shape of $\uptau$ deleting the
longest column on the left.

% We assume $\uptau$ and $\uptau'$ has the following shape where $(y_{1},y_{2})$
% could be empty and $w_{0}$ is non-empty if $\uptau_{R}\neq \emptyset$.
The definition splits in cases below. In all these cases we define
\begin{equation}\label{eq:x.uptau}
\bfxx_{\uptau} := x_{1}\cdots x_{n} \text{ and } x_{\uptau} := x_{n}
\end{equation}
which is marked by $s/r/c/d$.
For the part marked by $*/\cdots$ , $\eDD$ keeps $r,c,d$ and maps the rest part consisting of $\bullet$ and $s$ by dot-s switching algorithm.
\begin{enumerate}[resume*=alg1]
  \item When $C_{2k}=C_{2k-1}$ is odd, we could assume $\uptau$ and $\uptaup$ hase
        the following forms with $n = (C_{2k+1}-C_{2k}+1)/2$.
      \[
        \uptau: \hspace{1em} \tytb{\cdots\cdots\cdots,{\ast}{\ast}{\cdots},{x_{1}}{y_{1}}\cdots,{\vdots},{x_{n}}}
        \times \tytb{\cdots\cdots\cdots,{w_{0}}{\ast}{\cdots},\none,\none,\none}
        \mapsto
        \uptaup:  \tytb{\none\cdots\cdots,\none{\ast}{\cdots},\none{z_{1}}\cdots,\none,\none}
        \times \tytb{\cdots\cdots\cdots,{w_{0}}{\ast}{\cdots},\none,\none,\none}
      \]
        Suppose $y_{1} = c$ and $(x_{1}, \cdots, x_{n}) = (r, \cdots, r)$ or
        $(r, \cdots, r,d)$, then let $z_{1}= r$; otherwise, set $z_{1}:= y_{1}$.
        We define
        \[
        \bfpp_{\uptau}:=\tytb{{x_{1}}{y_{1}},\vdots,{x_{n}}}.
        \]
  \item When $C_{2k}$ is even and $\uptau$ has special shape, the descent is given by the following diagram
      \[\tiny
        \uptau: \hspace{1em}
        \ytb{
        \cdots\cdots\cdots,
        {\ast}{\ast}{\cdots},
        {*(srcol)\bullet},
        {*(srcol)\vdots},{*(srcol)\bullet},{x_1},\vdots,{x_n}}
        \times \ytb{\cdots\cdots\cdots,{\ast}{\ast}{\cdots},{*(srcol)\bullet},{*(srcol)\vdots},{*(srcol)\bullet},\none,\none,\none}
        \mapsto
       \uptaup: \ytb{\none\cdots\cdots,\none{\ast}{\cdots},\none,\none,\none,\none,\none,\none}
        \times \ytb{\cdots\cdots\cdots,{\ast}{\ast}{\cdots},{*(srcol)s},{*(srcol)\vdots},{*(srcol)s},\none,\none,\none}
      \]
  \item When $C_{2k}$ is even and $\uptau$ has non-special shape, then
        \[
        \uptau:\tytb{{\cdots}\cdots\cdots\cdots,{\ast}{\ast}{\cdots}{\cdots},{*(srcol)s}{*(srcol)r}{\cdots}\cdots,
        {*(srcol)\vdots}{*(srcol)\vdots},{*(srcol)s}{*(srcol)r},{x_{0}}{y_{1}},{x_{1}}{y_{2}},{\vdots},{x_{n}}}
        \times\tytb{{\cdots}\cdots\cdots,{\ast}{\cdots}\cdots,\ ,\ ,\ ,\ ,\ ,\ ,\ }
        \mapsto \uptau':
        \tytb{\cdots\cdots\cdots,{\ast}{\ast}{\cdots},{*(srcol)r}{\ast}\cdots,{*(srcol)\vdots},{*(srcol)r},{z_{1}},{z_{2}},\ ,\ }
        \times\tytb{{\cdots}\cdots\cdots,{\ast}{\cdots}\cdots,\ ,\ ,\ ,\ ,\ ,\ ,\ }
        \]
      where $y_{1},y_{2}$ are non empty.
      In most of the case, we just delete the longest column on the left and set
      $(z_{1},z_{2})=(y_{1},y_{2})$. The exceptional cases are listed below:
      \begin{itemize}
        \item When
 $x_{0}=r$, we have $(y_{1},y_{2})=(c,d)$. We let
      \[
          \tytb{{x_{0}}{y_{1}}\none{r}{c},{x_{1}}{y_{2}}{=}{x_{1}}{d},\vdots\none\none\vdots,{x_{n}}\none\none{x_{n}}}
          \mapsto \tytb{{z_{1}}\none{r},{z_{2}}{:=}r,\none,\none}
        \]

        \item When $x_{0}=c$, we have $n=1$ and $(x_{0},x_{1})=(y_{1},y_{2})=(c,d)$. We let
        \[
          \tytb{{x_{0}}{y_{1}},{x_{1}}{y_{2}}} = \tytb{cc,dd}
          \mapsto \tytb{{z_{1}},{z_{2}}}:=\tytb{r,c}.
      \]
      \item
      We remark that $x_{0}$ never equals to $d$.
  \end{itemize}
\end{enumerate}

\subsubsection{A key property in the descent case}
We retain the notation in \Cref{def:sp-nsp.D.sp}, where
\[
\cO=(C_{2k+1}=2c_{2k+1},C_{2k}=2c_{2k},C_{2k-1}=2c_{2k-1},\cdots, C_{0}) \text{
  such that } k\geq 1.
\]
Let $\cOp=\eDD(\cO)$ and $\cOpp=\eDD(\cOp)$.
The following lemma is the key property satisfied by our definition
\begin{lem}\label{lem:sp-nsp.D}
  Let $\tau^{s}$ and $\tau^{ns}$ are two representations attached to $\cO$ as in
  \Cref{def:sp-nsp.D.sp}.
  Then
  \begin{enumS}
    \item \label{lem:sp-nsp.D.1} For every $\uptau\in \drc(\tau^{s})\sqcup \drc(\tau^{ns})$, the shape
    of $\eDD^{2}(\uptau)$ is
    \[
      \taupp = (c_{2k-1},\tau_{2k-3},\tau_{1})\times (\tau_{2k-2},\cdots, \tau_{0})
    \]
    \item \label{lem:sp-nsp.D.2} Let $\cO_{1}= (2(c_{2k+1}-c_{2k}),)\in \Nil(D)$ be the trivial orbit
    of $\rO(2(c_{2k-1}-c_{2k}),\bC)$.
    We define $\delta\colon \drc(\cO)\rightarrow \drc(\cOpp)\times \drc(\cO_{1})$ by $\delta(\uptau) = (\eDD^{2}(\uptau),\bfxx_{\uptau})$.
    The following maps are bijections
    \[
      \begin{tikzcd}[row sep=0em]
        \drc(\tau^{s})\ar[r,"\delta"] & \drc(\taupp)\times \drc(\cO_{1}) &\ar[l,"\delta"'] \drc(\tau^{ns})\\
        \uptau^{s}\ar[r,maps to] & (\eDD^{2}(\uptau^{s}), \bfxx_{\uptau^{s}})&\\
        & (\eDD^{2}(\uptau^{ns}), \bfxx_{\uptau^{ns}})& \uptau^{ns}\ar[l,maps to]\\
      \end{tikzcd}
    \]
    In particular, we obtain an one-one correspondence
    $\uptau^{s}\leftrightarrow \uptau^{ns}$ such that $\delta(\uptau^{s})=\delta(\uptau^{ns})$.
    \item\label{lem:sp-nsp.D.3}
    Suppose $\uptau^{s}$ and $\uptau^{ns}$ correspond as the above such that
    $\delta(\uptau^{s})=\delta(\uptau^{ns})=(\uptaupp,\bfxx)$. Then
    \begin{equation} \label{eq:sp-nsp-sig}
      \ssign(\uptau^{s})=\ssign(\uptau^{ns})=(C_{2k},C_{2k})+\ssign(\uptaupp)+\ssign(\bfxx).
    \end{equation}
  \end{enumS}
\end{lem}
\begin{proof}
  Suppose $\uptau^{s}$ has special shape.
  the the behavior of $\uptau^{s}$ under the descent map $\eDD$ is illostrated
  as the following:
  \[\tiny
    \uptau^{s}: \hspace{1em}
    \ytb{
      {*(srcol)\cdots}\cdots\cdots,
      {*(srcol)\ast}{\ast}{\cdots},
      {*(srcol)\bullet},
      {*(srcol)\vdots},{*(srcol)\bullet},{x_1},\vdots,{x_n}}
    \times
    \ytb{{*(srcol)\cdots}\cdots\cdots,
      {*(srcol)\ast}{\ast}{\cdots},{*(srcol)\bullet},{*(srcol)\vdots},{*(srcol)\bullet},\none,\none,\none}
    \mapsto
    \uptau'^{s}
     \ytb{\none\cdots\cdots,\none{\ast}{\cdots},\none,\none,\none,\none,\none,\none}
    \times \ytb{{*(srcol)\cdots}\cdots\cdots,{*(srcol)\ast}{\ast}{\cdots},{*(srcol)s},{*(srcol)\vdots},{*(srcol)s},\none,\none,\none}
    \mapsto
    \uptaupp:
    \ytb{\none\cdots\cdots,\none{\ast}{\cdots},\none,\none,\none,\none,\none,\none}
    \times \ytb{\none\cdots\cdots,{\none}{\ast}{\cdots},\none,\none,\none,\none,\none,\none}
  \]
  Note that $\uptau^{s}$ is obtained from $\taupp$ by attaching the grey part
  consisting totally $2c_{2k}$ dots and $\bfxx$. Now the claims for $\uptau^{s}$ is
  clear.


  Now consider the descent of a non-special diagram $\uptau^{ns}$:
  \[\tiny
    \uptau^{ns}:\ytb{{*(srcol)\cdots}\cdots\cdots\cdots,{*(srcol)\ast}{y_{0}}{\ast}{\cdots},{*(srcol)s}{*(srcol)r}{y_{2}}\cdots,
      {*(srcol)\vdots}{*(srcol)\vdots},{*(srcol)s}{*(srcol)r},{x_{0}}{y_{1}},{x_{1}}{y_{3}},{\vdots},{x_{n}} }
    \times\ytb{{*(srcol)\cdots}\cdots\cdots,{*(srcol)u_{0}}{\cdots},\ ,\ ,\ ,\ ,\ ,\ ,\ }
    \mapsto
    \uptau'^{ns}:\ytb{\cdots\cdots\cdots,{y'_{0}}{\ast}{\cdots},{*(srcol)r}{y'_{2}}\cdots,
      {*(srcol)\vdots},{*(srcol)r},{y'_{1}},{y'_{3}},\ ,\ }
    \times\ytb{{*(srcol)\cdots}\cdots\cdots,{*(srcol)w_{0}}{\cdots},\ ,\ ,\ ,\ ,\ ,\ ,\ }
    \mapsto
    \uptaupp:\hspace{1em}
            \ytb{{\cdots}{\cdots}{\cdots},{x'_{0}}{\ast}{\cdots},{x'_{1}}{x'_{2}}\cdots,\none,\none,\none,\none,\none,\none}
            \times
            \ytb{{\cdots}{\cdots},{\cdots}\none,\none,\none,\none,\none,\none,\none,\none}
            % \ytb{\emptyset,\none,\none,\none,\none,\none}
  \]
  The bijection follows from the observation that 1. $\uptau^{ns}$ is obtained by
  attaching the grey part $x_{0},\cdots,x_{n}$ and $y_{0},y_{1},y_{2},y_{3}$
  to the $\ast/\cdots$ part of $\uptaupp$;
  2. the value of $(y_{0},y_{1},y_{2},y_{3})$ is completely determined by
  $(x'_{0},x'_{1},x'_{2})$ and $\bfxx=x_{1}\cdots x_{n}$.

  We leave it to the reader for checking case by case that the grey part of $\uptau^{ns}$ has signature
  $(C_{2k}-2,C_{2k}-2)$ and
  \[\ssign(x_{0}y_{0}y_{1}y_{2}y_{3})-\ssign(x'_{0}x'_{1}x'_{2})=(2,2).
  \]
 Therefore \eqref{eq:sp-nsp-sig} follows.

 \trivial[]{
   First assume that $C_{2k}=2$. Note that we can not/(or now?) assume $k=1$, consider the
   orbit $\cOpp=(2,1,1,1,1)$.
   \begin{enumPF}
     \item
     $x'_{1}=s$, now $y'_{0} = \emptyset/\bullet$ when $x'_{0}=\emptyset/s$
     \begin{enumPF}
       \item Suppose $x'_{2}\neq r$. Then $(y'_{2},y'_{1},y'_{3}) = (x'_{2},r,d)$,
       $(x_{0},y_{0},y_{2},y_{1},y_{3}) = (s,x'_{0}, x'_{2},c,d)$.
       Therefore, the sign difference is $\ssign(s,c,d)-\ssign(s)=(2,2)$.
       \item Suppose $x'_{2} = r$. Then $(y'_{2},y'_{1},y'_{3}) = (c,r,d)$.
       $(x_{0},y_{0},y_{2},y_{1},y_{3}) = (s,x'_{0},c,r,d)$.
       Therefore, the sign difference is $\ssign(s,c,r,d)-\ssign(s,r)=(2,2)$.
     \end{enumPF}
     \item $x'_{1}\neq s$.
     \begin{enumPF}
       \item Suppose $x'_{0}\neq c$.
       Then $x'_{0}=\emptyset/s/r$, $x'_{1}=r/c/d$, $(y'_{0}, y'_{2},y'_{1},y'_{3}) = (\emptyset/\bullet/r,x'_{2},r,x'_{1})$.
       \begin{enumPF}
         \item $x'_{1}=r$. Then
         $(x_{0},y_{0},y_{2},y_{1},y_{3}) = (r,x'_{0},x'_{2},c,d)$.
         Therefore, the sign difference is $\ssign(r,c,d)-\ssign(r)=(2,2)$.
         \item $x'_{1}=c$. Then  $x_{1}=d$.
         $(x_{0},y_{0},y_{2},y_{1},y_{3}) = (c,x'_{0},x'_{2},c,d)$.
         Therefore, the sign difference is $\ssign(c,c,d)-\ssign(c)=(2,2)$.
         \item $x'_{1}=d$. Then
         $(x_{0},y_{0},y_{2},y_{1},y_{3}) = (s,x'_{0},y'_{2},y'_{1},y'_{3})=(s,x'_{0},x'_{2},r,d)$.
         Therefore, the sign difference is $\ssign(s,r,d)-\ssign(d)=(2,2)$.
       \end{enumPF}
       $(x_{0},y_{0},y_{2},y_{1},y_{3}) = (s,r,x'_{2},c,d)$.
       Therefore, the sign difference is $\ssign(s,r,x'_{2},c,d)-\ssign(c,d,x'_{2})=(2,2)$.
       \item Suppose $x'_{0} = c$.
       Then $x'_{1}=d$, $(y'_{0}, y'_{2},y'_{1},y'_{3}) = (r,x'_{2},c,d)$.
       $(x_{0},y_{0},y_{2},y_{1},y_{3}) = (s,r,x'_{2},c,d)$.
       Therefore, the sign difference is $\ssign(s,r,x'_{2},c,d)-\ssign(c,d,x'_{2})=(2,2)$.
     \end{enumPF}
   \end{enumPF}
   }
 \end{proof}

\subsubsection{A key proposition in the generalized descent case}\label{sec:gd2.CD}

We now assume that $k\geq 1$ and $C_{2k}$ is odd, i.e. $\cOp\leadsto \cOpp$ is a
generalized descent.

Without loss of generality, we can assume the dot-r-c diagrams have the
following shapes with $n = (C_{2k+1}-C_{2k}+1)/2$:
\begin{equation}\label{eq:gd2.drc}
  \uptau:\tytb{
    {*(srcol)\cdots}\cdots\cdots\cdots,
    {*(srcol)\ast}{\ast}{\ast}{\ast},
    {x_{1}}{x_{0}}{\cdots}\cdots,
    \vdots,
    {x_{n}}
  }
  \times
\tytb{{*(srcol)\cdots}\cdots\cdots,
    {*(srcol)\ast}{\ast}{\cdots},
    \none,\none,\none}
  \xmapsto{\hspace{1em}\eDDo\hspace{1em}}
  \uptaup:\tytb{
    \cdots\cdots\cdots,
    {\ast}{\ast}{\ast},
    {x_{\uptaupp}}{\cdots}\cdots,
    \none,
    \none
  }
  \times
\tytb{{*(srcol)\cdots}\cdots\cdots,
    {*(srcol)\ast}{\ast}{\cdots},
    \none,\none,\none}
  \xmapsto{\hspace{1em}\eDDo\hspace{1em}}
  \uptaupp:\tytb{
    \cdots\cdots\cdots,
    {\ast}{\ast}{\ast},
    {x_{\uptaupp}}{\cdots}\cdots,
    \none,
    \none
  }
  \times
\tytb{\cdots\cdots,
    {\ast}{\cdots},
    \none,\none,\none}
\end{equation}

The diagram $\uptau$ is obtained from the diagram of $\uptau$ by adding
$\bullet$ at the
grey parts, changing $x_{\uptaupp}$ to $x_{0}$ and attaching $x_{1}\cdots x_{n}$.


We define
\[
  \bfpp_{\uptau}:=\tytb{{x_{1}}{x_{0}},\vdots,{x_{n}}}
\]
and call $\bfpp_{\uptau}$ the peduncle part of $\uptau$. Let
$\cO_{1} = (C_{2k+1}-C_{2k}+1,)\in \dpeNil(D)$. We define
$\bfuu_{\uptau}\in \drc(\cO_{1})$ by the following formula:
\begin{equation}\label{eq:def.u}
  \bfuu_{\uptau}:=u_{1} \cdots u_{n} =
  \begin{cases}
    r\cdots r c, & \text{when } \bfpp_{\uptau} = \tytb{rc,\vdots,r}\\
    r\cdots c d, & \text{when } \bfpp_{\uptau} = \tytb{rc,\vdots,d}\\
    x_{1}\cdots, x_{n} & \text{otherwise}.
  \end{cases}
\end{equation}


Let $\tcO = (C_{2k+1}-C_{2k}+1,1,1,)\in \dpeNil(D)$ and
$\tcOpp = \eDD^{2}(\tcO) = (2)\in \dpeNil(D)$. Then
$\bfpp_{\uptau}\in \drc(\tcO)$ and $x_{\uptau}\in\dpeNil(D)$.
There is some restriction on the possibilities  of $\bfuu_{\uptau}$.

\begin{lem}\label{lem:u}
  Let
  \[
    \uptau = \tytb{{x_{1}}{x_{0}},\vdots, {x_{n}}}\in \drc(\tcO)
    \quad \text{and}\quad \uptaupp:=\eDDo^{2}(\uptau).
  \]
  Then we have the following properties which is easy to verify.
  \begin{enumS}
    \item  If $\uptaupp = r$, then $s \in \bfuu_{\uptau}$ or
    $c \in \bfuu_{\uptau}$. In particular, $\ssign(\bfuu_{\uptau})\succ (0,1)$.
    If $\ssign(\bfuu_{\uptau})\nsucc (0,2)$, we have
    \[
      \uptau = \tytb{rc, \vdots,r} \quad \text{and}\quad  \bfuu_{\uptau} = \tytb{r,\vdots,c}.
    \]
    \item  If $\uptaupp = c$, then $s \in \bfuu_{\uptau}$ or
    $c \in \bfuu_{\uptau}$. In particular, $\ssign(\bfuu_{\uptau})\succeq (0,1)$.
    \item If $\uptaupp = d$, there is no restriction on $\bfuu_{\uptau}$.
    \item  We have $x_{n}=d$ if and only $d\in \bfuu_{\uptau}$.
    \item If $x_{n}=d$ and $\uptaupp =r/c$,  we have $n\geq 2$ and
    $\ssign(\bfuu_{\uptau})\succeq (0,2)$.
    \item If $\bfuu_{\uptau} = r\cdots r$ or $r\cdots rd$, we have $\uptaupp=d$.\qedhere
  \end{enumS}

\end{lem}


\begin{lem}\label{lem:gd.inj}
  The map $\delta\colon  \drc(\cO)\rightarrow \drc(\cOpp) \times \drc(\cO_{1})$
  given by $\uptau\mapsto (\eDDo^{2}(\uptau),\bfuu_{\uptau})$ is injective.
  Moreover,
  \[
  \ssign(\uptau) =\ssign(\uptaupp) + (C_{2k}-1, C_{2k}-1)+\ssign(\bfuu_{\uptau}).
  \]
  The map $\tdelta\colon \drc(\cO)\rightarrow \drc(\cOpp) \times \bN^{2}\times \bZ/2\bZ$
  given by $\uptau\mapsto (\eDDo^{2}(\uptau), \ssign(\uptau),\upepsilon_{\uptau})$ is injective.
\end{lem}
\begin{proof}
  By our algorithm,
  \[
    (x_{\uptaupp},\bfuu_{\uptau}) = (x_{\eDDo^{2}(\bfpp_{\uptau})}, \bfuu_{\bfpp_{\uptau}}).
  \]
  The the injectivity of $\delta$ and the siginature formula follows directly
  from the definition of $\bfuu_{\uptau}$.
  Now the injectivity of $\tdelta$ follows from the injectivity of
  $\bfuu_{\uptau}\mapsto (\ssign(\bfuu_{\uptau}), \upepsilon)$ by
  \Cref{c:init.CD}.
  % Let $\uptaupp:=\eDDo^{2}(\uptau)$ and
  % Then $\uptau\mapsto (\uptaupp,\bfuu_{\uptau})$ is injective and
  % \[
  % %\ssign(\uptau) =\ssign(\uptaupp) + 2(n_1,n_{1})+\ssign(\bfuu_{\uptau}) \text{ where }2n_{1}+1 = C_{2k}.
  % \ssign(\uptau) =\ssign(\uptaupp) + (C_{2k}-1,C_{2k}-1)+\ssign(\bfuu_{\uptau}) \text{ where }2n_{1}+1 = C_{2k}.
  % \]
  % Now the lemma follows from
  % $\bfuu_{\uptau}\mapsto (\ssign(\bfuu_{\uptau}), \upepsilon)$ is injective by \Cref{c:init.CD}.
\end{proof}


\section{Type B/M case}


\subsection{``Special'' and ``non-special'' diagrams}
\subsubsection{type M}

\begin{defn}\label{def:sp-nsp.M.sp}
  Let $\cO = (C_{2k},C_{2k-1}, \cdots, C_{1},C_{0}=0)\in \dpeNil(M)$
  with $k\geq 1$.
  Let
  \[
  \tau =  (\tau_{2k}, \cdots, \tau_{2})\times  (\tau_{2k-1}, \cdots, \tau_{1}).
  \]
  be a Weyl group representation attached to $\cO\in \dpeNil(M)$.
  We say $\tau$ has
  \begin{itemize}
    \item \idxemph{special shape} if $\tau_{2k}\geq \tau_{2k-1}$;
    \item \idxemph{non-special shape} if $\tau_{2k}< \tau_{2k-1}$.
  \end{itemize}
  Clearly, this gives a partition of the set of Weyl group representations
  attached to $\cO$.

  Suppose $k\geq 1$, $C_{2k}=2c_{2k}-1$ and $C_{2k-1}=2c_{2k-1}+1$.
  The special shape representation $\tau^{s}$
  \[
    \tau^{s} = \tau_{L}^{s}\times \tau_{R}^{s}
    = (c_{2k},\tau_{2k-2}, \cdots, \tau_{2})\times (c_{2k-1},\tau_{2k-3} \cdots, \tau_{1}).
  \]
  is paired with the non-special shape representation
  \[
    \tau^{ns} = \tau_{L}^{ns}\times \tau_{R}^{ns} =
     (c_{2k-1},\tau_{2k-2},\cdots, \tau_{2})\times (c_{2k},\tau_{2k-3},\cdots, \tau_{1} )
  \]
  In the paring $\tau^{s}\leftrightarrow \tau^{ns}$, $\tau_{j}$  is unchanged for $j\leq 2k-2$.
  We call $\tau^{s}$ the \emph{special shape} and $\tau^{ns}$ the
  corresponding \emph{non-special shape}
\end{defn}
% We define a bijection between ``special'' diagram $\uptau^{s}$ and
% ``non-special'' diagrams $\uptau^{ns}$.

% \begin{defn}\label{def:sp-nsp.M.sp}

%   Let $\cO\in \Nil(M)$ such that $C_{2k}=2c_{2k}-1$ and $C_{2k-1}=2c_{2k-1}+1$.
%   A representation $\tau^{s}$ attached to $\cO$ has the shape
%   \[
%     \tau^{s} = \tau_{L}^{s}\times \tau_{R}^{s}=
%     (c_{2k},\tau_{2k-2},\cdots, \tau_{2}) \times (c_{2k-1},\tau_{2k-3},\cdots, \tau_{1})
%   \]
%   is paired with
%   \[
%     \tau^{ns} = \tau_{L}^{ns}\times \tau_{R}^{ns} =
%     (c_{2k-1},\tau_{2k-2},\cdots, \tau_{2}) \times (c_{2k},\tau_{2k-3},\cdots, \tau_{1})
%   \]
%   where the $c_{j}$ are unchanged for $j\leq 2k-2$.
%   We call $\tau^{s}$ the \emph{special shape} and $\tau^{ns}$ the
%   corresponding \emph{non-special shape}
% \end{defn}

% Note that the shape
% $\tau = (\tau_{2k},\cdots, \tau_{2},0)\times (\tau_{2k-1},\cdots, \tau_{1})$
% is a special shape if and only if $\tau_{2k}>\tau_{2k-1}$.



\begin{defn}\label{def:sp-nsp.M}
  Suppose $C_{2k}$ is odd.
We retain the notation in \Cref{def:sp-nsp.M.sp} where the shapes $\tau^{s}$ and $\tau^{ns}$
correspond with each other.
We define a bijection between $\drc(\tau^{s})$ and $\drc(\tau^{ns})$:
\[
  \begin{tikzcd}[row sep=0em]
    \drc(\tau^{s}) \ar[rr,<->]&\hspace{2em} & \drc(\tau^{ns})\\
    \uptau^{s} \ar[rr,<->]& & \uptau^{ns}
  \end{tikzcd}
\]
Without loss of generality, we can assume
that $\uptau^{s}\in \drc(\tau^{s})$ and $\uptau^{ns}\in \drc(\tau^{ns})$ have the following shapes:
\begin{equation}\label{eq:sp-nsp.M}
  \uptau^{s}:
  \tytb{\cdots\cdots\cdots,{y_{1}}\cdots\cdots,{*(srcol)s},{*(srcol)\vdots},{*(srcol)s},{y_{2}}}
  \times
  \tytb{\cdots\cdots\cdots,{x_{0}}{\cdots}{\cdots},\ ,\ ,\ ,\ }
  \longleftrightarrow \hspace{2em}
  \uptau^{ns}:
  \tytb{\cdots\cdots\cdots,{y_{0}}\cdots\cdots,\ ,\ ,\ ,\ }
  \times
  \tytb{\cdots\cdots\cdots,{x_{1}}{\cdots}\cdots,{*(srcol)r},
    {*(srcol)\vdots},{*(srcol)r},{x_{2}} }
\end{equation}
Here the grey parts have length $(C_{2k}-C_{2k-1})/2$ (could be zero length).
The entries $x_{0}$, $x_{1}$, $y_{0}$ and $y_{1}$ are either all empty or all
non-empty.
% and the rows contains them may be empty. % We remark that the
% value of $v_{0}$ and $w_{0}$ are completely determined by $x_{0}$ and $y_{0}$.
% In particular, $v_{0}$ and $w_{0}$ are non-empty if and only if
% $x_{0}/y_{0}\neq \emptyset$, i.e. $C_{2k-1}\geq 4$.
% We marks the entries of $\uptau^{s}$ and $\uptau^{ns}$ as in
% \eqref{eq:sp-nsp.C}.
The correspondence between $(x_{0},x_{1}, x_{2})$, with
$(y_{0},y_{1},y_{2})$ is given by switching $r$ and $s$,  $d$ and $c$. i.e.
\[
  \begin{cases}
    y_{i} = s \Leftrightarrow x_{i} = r\\
    y_{i} = c \Leftrightarrow x_{i} = d\\
  \end{cases}
  \quad \text{for } i=0,1,2.
\]

We define
\[
  \drcs(\cO) := \bigsqcup_{\tau^{s}} \drc(\tau^{s}) \quad \text{ and }\quad
  \drcns(\cO) := \bigsqcup_{\tau^{ns}} \drc(\tau^{ns})
\]
where $\tau^{s}$ (resp. $\tau^{ns}$)runs over all specail (resp. non-special) shape representations attached to $\cO$.
Clearly $\drc(\cO) = \drcs(\cO)\sqcup \drcns(\cO)$.
\end{defn}
It is easy to check the bijectivity in the above definition, which we leave it
to the reader.



\subsubsection{Special and non-special shapes of type B}
Now we define the notion of special and non-special shape of type B.
\begin{defn}\label{def:sp-nsp.B.sp}
  Let
  \[
    \tau = (\tau_{2k}, \cdots, \tau_{2}) \times (\tau_{2k+1},\tau_{2k-1}, \cdots, \tau_{1})
  \]
  be a Weyl group representation attached to $\cO\in \dpeNil(B)$. We say $\tau$
  has
  \begin{itemize}
    \item \idxemph{special shape} if $\tau_{2k}\geq \tau_{2k-1}$;
    \item \idxemph{non-special shape} if $\tau_{2k} < \tau_{2k-1}$.
  \end{itemize}

  Suppose $k\geq 1$ and
  \begin{equation} \label{eq:B.orb.ds}
    \cO=(C_{2k+1}=2c_{2k+1}+1,C_{2k}=2c_{2k}-1,C_{2k-1}=2c_{2k-1}+1,\cdots,C_{1}>0, C_{0}=0).
  \end{equation}
  A representation $\tau^{s}$ attached to $\cO$ has the shape
  \[
    \tau^{s} = \tau_{L}^{s}\times \tau_{R}^{s}= (c_{2k},\tau_{2k-2},\cdots, \tau_{2}) \times (c_{2k+1},c_{2k-1},\tau_{2k-3},\cdots, \tau_{1})
  \]
  is paired with
  \[
    \tau^{ns} = \tau_{L}^{ns}\times \tau_{R}^{ns} = (c_{2k-1},\tau_{2k-2},\cdots, \tau_{2}) \times (c_{2k+1},c_{2k},\tau_{2k-3},\cdots, \tau_{1}).
  \]
  In the paring $\tau^{s}\leftrightarrow \tau^{ns}$ described as the above,
  $\tau_{j}$ are unchanged for $j\leq 2k-2$.
\end{defn}




\subsection{Definition of $\eDD$ for type B and M}


\subsubsection{The defintion of $\upepsilon$.} \label{sec:upepsilon.BM}
\begin{enumerate}[label=(\arabic*).,series=alg2]
  \item When $\uptau\in \drc(B)$, $\upepsilon$ is determined by the ``basal
  disk'' of $\uptau$ (see \eqref{eq:x.uptau.BM} for the definition of $x_{\uptau}$):
  \[
    \upepsilon_{\uptau}:=
    \begin{cases}
      0, & \text{if $x_{\uptau}=d$;} \\
      1, & \text{otherwise.}
    \end{cases}
  \]
  \item When $\uptau\in \drc(M)$, the  $\upepsilon$  is determined by
  the lengths of $\taulf$ and $\taurf$:
  \[
    \upepsilon_{\uptau} :=
    \begin{cases}
      0, & \text{if }\abs{\taulf} - \abs{\taurf} \geq  0;\\
      1, & \text{if }\abs{\taulf} - \abs{\taurf}<0.
    \end{cases}
  \]
\end{enumerate}


\subsubsection{Initial cases}

\begin{enumerate}[resume*=alg2]
  \item Suppose $\cO = (2c_{1}+1)$ has only one column. Then
        $\cOp_{0}:=\eDD(\cO)$ is the trivial orbit of $\Mp(0,\bR)$. $\drc(\cO)$
        consists of diagrams of shape
        $\tau_{L}\times \tau_{R} =\emptyset\times (c_{1},)$.
        The set $\drc(\cOp_{0})$ is a singleton $\set{\uptaup_{0}=\emptyset\times \emptyset}$, and every element
        $\uptau \in \drc(\cO)$ maps to $\uptau_{0}$ (see \eqref{eq:uptau0}):
        \[\tiny
          \drc(\cO)\ni \uptau:= \hspace{1em} \ytb{\emptyset,\none,\none,\none}
          \times \ytb{{x_{1}},\vdots,{x_{n}},{m_{\uptau}}}
          \mapsto \uptaup_{0}:=\emptyset\times \emptyset
        \]
        Here $m_{\uptau}=a$ or $b$ is the mark of $\uptau$.
        We define
        \[
          \bfpp_{\uptau}:=\bfxx_{\uptau} := x_{1}\cdots x_{n} m_{\uptau}.%  \text{ and }x_{\uptau}:=x_{n}
        \]
        We call $\bfpp_{\uptau}=\bfxx_{\uptau}$ the ``peduncle'' part of
        $\uptau$.
\end{enumerate}

\subsubsection{The descent from M to B}
The descent of a special shape diagram is simple, the descent of a non-special shape
reduces to the corresponding special one:
\begin{enumerate}[resume*=alg2]
  \item Suppose $\abs{\taulf} \geq \abs{\taurf} $, i.e. $\uptau$ has special
        shape. Keep $r,d$ unchanged, delete $\taulf$ and fill the remaining part
        with ``$\bullet$'' and ``$s$'' by $\DDr(\uptau^{\bullet})$ using the
        dot-s switching algorithm. The mark $m_{\uptaup}$ of
        $\uptaup:=\eDDo(\uptau)$ is given by the following formula:
        % \footnote{Note that $\taulf$ ends with $s$ or $c$ since
        % $\abs{\taulf}>\abs{\taurf}$}
        \[
        m_{\uptaup}:= \begin{cases}
          a & \text{if  $\taulf$ ends with $s$ or $\bullet$},\\
          b & \text{if $\taulf$ ends with $c$}.
        \end{cases}
        \]

  \item Suppose $\abs{\taulf} < \abs{\taurf}$, i.e. $\uptau$ has non-special
        shape. We define
        \[\eDDo(\uptau):=\eDDo(\uptau^{s})\]
        where $\uptau^{s}$ is the special diagram corresponding to $\uptau$
        defined in \Cref{def:sp-nsp.M}.
\end{enumerate}

\begin{lem}\label{lem:ds.BM}
  Suppose $\cO = (C_{2k},C_{2k-1}, \cdots, C_{0})$ with $k\geq 1$ and $C_{2k}$
  is odd.
  Let $\cOp := \eDD(\cO)=(C_{2k-1},\cdots, C_{0})$.
  Then
  \[
    \begin{split}
      \drcs(\cO) &\xrightarrow{\hspace{2em}\eDDo\hspace{2em}} \drc(\cOp) \\
      \drcns(\cO)&
      \xrightarrow{\hspace{2em}\eDDo\hspace{2em}} \drc(\cOp)
    \end{split}
  \]
  are a bijections.
\end{lem}
\begin{proof}
  The claim for $\drcs(\cO)$ is clear by the definition of descent.
  The claim for $\drcns(\cO)$ reduces to that of $\drcs(\cO)$ using \Cref{def:sp-nsp.M}.
\end{proof}

\begin{lem}\label{lem:gd.BM}
  Suppose $\cO = (C_{2k},C_{2k-1}, \cdots, C_{0})$ with $k\geq 1$ and $C_{2k}$ even.
  Let $\cOp := \eDD(\cO)=(C_{2k-1}+1,\cdots, C_{0})$.
  Then the following map is a bijection
  \[
    \drc(\cO) \xrightarrow{\hspace{2em}\eDDo\hspace{2em}}
    \Set{\uptaup\in \drc(\cOp)| \begin{minipage}{11em}
        $m_{\uptaup}\neq b$ and\\
         $\uptaup_{R,0}$ dose not ends with $s$.
      \end{minipage}
    }. %\subsetneq \drc(\cOp).
  \]
\end{lem}
\begin{proof}
  This is clear by the descent algorithm.
\end{proof}


\subsubsection{Descent from $B$ to $M$}
Now we define the general case of the descent from type $B$ to type $M$.
We assume that $k\geq 1$ i.e. $\cO$ has at least 3 columns.
First note that the shape of $\uptau' = \eDD(\uptau)$ is the shape of $\uptau$ deleting the
most left column on the right diagram.

% We assume $\uptau$ and $\uptau'$ has the following shape where $(y_{1},y_{2})$
% could be empty and $w_{0}$ is non-empty if $\uptau_{R}\neq \emptyset$.
The definition splits in cases below. In all these cases we define
\begin{equation}\label{eq:x.uptau.BM}
 % \bfpp_{\uptau}:=
 \tbfxx_{\uptau} := x_{1}\cdots x_{n}m_{\uptau} \text{ and } \txx_{\uptau} := x_{n}
\end{equation}
which is marked by $s/r/c/d$.
For the part marked by $*/\cdots$ , $\eDD$ keeps $r,c,d$ and maps the rest part consisting of $\bullet$ and $s$ by dot-s switching algorithm.
\begin{enumerate}[resume*=alg2]
  \item When $C_{2k}$ is odd and $\uptau$ has special shape, the descent is given by the following diagram
      \[
        \uptau: \hspace{1em}
        \tytb{
        \cdots\cdots\cdots,
        {\ast}{\cdots}{\cdots},
        {*(srcol)\bullet},
        {*(srcol)\vdots},{*(srcol)\bullet},{y_0},\none,\none,\none}
      \times
      \tytb{\cdots\cdots\cdots,
        {\ast}{\ast}{\cdots},
        {*(srcol)\bullet},
        {*(srcol)\vdots},
        {*(srcol)\bullet},
        {x_{1}},{\vdots},{x_{n}},{m_{\uptau}}}
        \mapsto
       \uptaup: \tytb{\cdots\cdots\cdots,{\ast}{\cdots}{\cdots},{*(srcol)s},{*(srcol)\vdots},{*(srcol)s},{y'_{0}},\none,\none,\none}
        \times \tytb{\none\cdots\cdots,\none{\ast}{\cdots},\none,\none,\none,\none,\none,\none,\none}
      \]
      Here the grey columns has length $(C_{2k}-C_{2k-1})/2$ and $n = (C_{2k+1}-C_{2k})/2$.
      The  ``$\ast/\cdots$'' part of $\uptaup$ is given by keeping the corresponding
      entries marked by $r/d/c$ in
       $\uptau$ unchange and filling $s/\bullet$ accordingly in the rest of the
       entries.
       The entry $y'_{0}$ of $\uptaup$ is given by the following formula:
       % \footnote{When $x_{1}\cdots x_{n}=r\cdots r$, we have $y_{0}=c$.
       %   Otherwise $x_{1}=s$ or $x_{1}\cdots x_{n} = r\cdots r d$.}
       % \[
       %   y'_{0} := \begin{cases}
       %     s & \text{if $x_{1}=\bullet$ or
       %       $x_{1}\cdots x_{n} m_{\uptau}= r\cdots ra$ or $r\cdots rda$}\\
       %     c & \text{if $x_{1}=s$ or
       %       $x_{1}\cdots x_{n} m_{\uptau}= r\cdots rb$ or $r\cdots rdb$}.
       %   \end{cases}
       % \]
       \[
         y'_{0} := \begin{cases}
           s & \text{if $x_{1}=\bullet$ or
             $(x_{1},m_{\uptau})=(r/d, a)$}\\
           c & \text{if $x_{1}=s$ or
             $(x_{1},m_{\uptau})=(r/d, b)$}.
         \end{cases}
       \]
       % We define $\tbfxx_{\uptau}$ be the diagram by replacing $\bullet$ to $s$
       % in $\bfxx_{\uptau}$.
       %  We define $\tfbfxx_{\uptau}\in \DRC()$
  \item When $C_{2k}$ is odd and $\uptau$ has non-special shape, then
        \[
        \uptau: \hspace{1em}
        \tytb{
        {\cdots}\cdots\cdots,
        {\ast}{\cdots}{\cdots},
        \none,\none,\none,\none,\none,\none,\none}
        \times
        \tytb{\cdots\cdots\cdots,
        {\ast}{\ast}{\cdots},
        {*(srcol)s}{*(srcol)r},
        {*(srcol)\vdots}{*(srcol)\vdots},
        {*(srcol)s}{*(srcol)r},
        {x_{1}}{x_{0}},{\vdots},{x_{n}},{m_{\uptau}}}
        \mapsto
        \uptaup: \tytb{\cdots\cdots\cdots,{\ast}{\cdots}{\cdots},\none,\none,\none,\none,\none,\none,\none}
        \times \tytb{
        \none\cdots\cdots,\none{\ast}{\cdots},
        \none{*(srcol)r},
        \none{*(srcol)\vdots},
        \none{*(srcol)r},
        \none{x'_{0}},\none,\none,\none}
        \]
        Here the grey columns has length $(C_{2k}-C_{2k-1})/2$ and $n = (C_{2k+1}-C_{2k})/2$.
        The  ``$\ast/\cdots$'' part of $\uptaup$ is given by keeping the corresponding
        entries marked by $r/d/c$ in
        $\uptau$ unchange and filling $s/\bullet$ accordingly in the rest of the
        entries.
        The entry $x'_{0}$ of $\uptaup$ is given by the following formula:
        % \footnote{When $x_{1}\cdots x_{n}=r\cdots r$, we have $x_{0}=d$.}
        \[
        x'_{0} := \begin{cases}
          r & \text{if $(x_{1},m_{\uptau})=(r/d, a)$,}\\
          d & \text{if $(x_{1},m_{\uptau})=(r/d, b)$,}\\
          x_{0} & \text{if $x_{1}=s$.}
        \end{cases}
        \]
        Note that
        % \[
        %   x'_{0} := \begin{cases}
        %     r & \text{if $x_{1}\cdots x_{n}= r\cdots r$}\\
        %     x_{0} & \text{otherwise}
        %   \end{cases}
        % \]
        % We define $\tbfxx_{\uptau}:=\bfxx_{\uptau}$.
        %
  \item When $C_{2k}=C_{2k-1}$ is even, we could assume $\uptau$ and $\uptaup$ hase
        the following forms with $n = (C_{2k+1}-C_{2k}+1)/2$.
      \[
        \uptau: \hspace{1em}
        \tytb{
        {\cdots}\cdots\cdots,
        {\ast}{\cdots}{\cdots},
        {y_{0}}\cdots,\none,\none,\none}
      \times
      \tytb{\cdots\cdots\cdots,
        {\ast}{\ast}{\cdots},
        {x_{1}}{x_{0}},{\vdots},{x_{n}},{m_{\uptau}}}
        \mapsto
       \uptaup: \tytb{\cdots\cdots\cdots,{\ast}{\cdots}{\cdots},{y'_{0}}\cdots,\none,\none,\none}
        \times \tytb{\none\cdots\cdots,\none{\ast}{\cdots},\none{x'_{0}},\none,\none,\none}
      \]
      In the most of the case, $\uptau'$ is obtained from $\uptau$ by deleting
      the first column of $\uptau_{R}$, keeping $c/r/d$ and filling $s/\bullet$ accordingly.
      The exceptional cases are when $x_{1}=r/d$. In the exceptional case we
      always have $x_{0}=d$.  We define
      \[
        \begin{split}
          x'_{0} & := d, \\
          y'_{0} & := \begin{cases}
            s & \text{if $m_{\uptau}=a$},\\
            c & \text{if $m_{\uptau}=b$}.
          \end{cases}
        \end{split}
      \]
      Note that this definition is similar to that of the special
      shape diagrams in the descent case.
      We define the ``peduncle'' of $\uptau$ to be
      \[
        \bfpp_{\uptau} = \tytb{{x_{1}}{x_{0}}, \vdots, {x_{n}},{m_{\uptau}}}.
      \]
\end{enumerate}

In all the cases, let
$\cO_{1} = (2n,)$ is the trivial nilpotent of type D.
For each $\uptau\in \drc(\cO)$, we define $\bfxx_{\uptau} \in \drc(\cO_{1})$ by
require the following identity of multisets: %the entires of $\bfxx_{\uptau}$ forms the following multiset:
\[
\set{\text{entry of }\bfxx_{\uptau}} =
\begin{cases}
  \set{c, x_{2},\cdots, x_{n}} & \text{if $(x_1,m_{\uptau}) = (\bullet/s,a)$},\\
  \set{s, x_{2},\cdots, x_{n}} & \text{if $(x_1,m_{\uptau}) = (\bullet/s,b)$},\\
  \set{x_{1}, x_{2},\cdots, x_{n}} & \text{otherwise ($x_{1}=r/d$)}.
\end{cases}
\]
In the above definition we include the initial case where $\cO$ is a trivial
orbit of type B. We define $\bfxx_{\uptau}=\emptyset$ when $\cO$ is the trivial
orbit of $\rO(1,\bC)$.

We define the  ``basal disk'' of $\uptau$ as the following:
\[
  x_{\uptau} :=\begin{cases}
    % \emptyset & \text{if $\cO = (1)$}\\
    \bfxx_{\uptau} & \text{if $C_{2k+1}=C_{2k}+1$}.\\
    c & \text{if $x_{n}m_{\uptau} =sa$ or $\bullet a$}\\
    x_{n} & \text{otherwise}. % \text{if $C_{2k+1}-C_{2k}\geq 1$ and $x_{n}m_{\uptau}\neq sa$}\\
  \end{cases}
\]


\subsubsection{A key proposition for descent case}
Now we assume $\cO$ is given by \eqref{eq:B.orb.ds}
Let $\cOp=\eDD(\cO)$ and $\cOpp=\eDD(\cOp)$.



The following lemma is the key property satisfied by our definition
\begin{lem}\label{lem:sp-nsp.B}
  Suppose $k\geq 1$ and
  $\tau^{s}$ and $\tau^{ns}$ are two representations attached to $\cO$ as in
  \Cref{def:sp-nsp.B.sp}.
  Then
  \begin{enumS}
    \item \label{lem:sp-nsp.B.1} For every $\uptau\in \drc(\tau^{s})\sqcup \drc(\tau^{ns})$, the shape
    of $\eDD^{2}(\uptau)$ is
    \[
      \taupp = (\tau_{2k-2},\cdots, \tau_{2})\times  (c_{2k-1},\tau_{2k-3},\tau_{1})
    \]
    \item \label{lem:sp-nsp.B.2} % Let $\cO_{1}= (C_{2k+1}-C_{2k}+1,)\in \Nil(B)$ be the trivial orbit
    % of $\rO(C_{2k+1}-C_{2k}+1,\bC)$.
    We define $\delta\colon \drc(\cO)\rightarrow \drc(\cOpp)\times\drc(\cO_{1})$
    by $\delta(\uptau) = (\eDD^{2}(\uptau),\bfxx_{\uptau})$.
    The following maps are bijective%with the same image
    \[
      \begin{tikzcd}[row sep=0em]
        \drc(\tau^{s})\ar[r,"\delta"] & \drc(\taupp)\times \drc(\cO_{1}) &\ar[l,"\delta"'] \drc(\tau^{ns})\\
        \uptau^{s}\ar[r,maps to] & (\eDD^{2}(\uptau^{s}), \bfxx_{\uptau^{s}})&\\
        & (\eDD^{2}(\uptau^{ns}), \bfxx_{\uptau^{ns}})& \uptau^{ns}\ar[l,maps to]\\
      \end{tikzcd}
    \]
    In particular, we obtain an one-one correspondence
    $\uptau^{s}\leftrightarrow \uptau^{ns}$ such that $\delta(\uptau^{s})=\delta(\uptau^{ns})$.
    \item\label{lem:sp-nsp.B.3}
    Suppose $\uptau^{s}$ and $\uptau^{ns}$ correspond as the above such that
    $\delta(\uptau^{s})=\delta(\uptau^{ns})=(\uptaupp,\bfxx)$. Then
    \begin{equation} \label{eq:sp-nsp-sig.B}
      \ssign(\uptau^{s})=\ssign(\uptau^{ns})=(C_{2k},C_{2k})+\ssign(\uptaupp)+\ssign(\bfxx).
    \end{equation}
    % with
    % \[
    %   \tsign(\tbfxx) = \begin{cases}
    %     \ssign(x_{1}\cdots x_{n})  & \text{if $x_{1}=r/d$}\\
    %     \ssign(\tbfxx) - (0,1) & \text{otherwise.}\\
    %   \end{cases}
    % \]
    Note that $\ssign(\bfxx)\in \bN\times \bN$.
  \end{enumS}
\end{lem}
\begin{proof}
  By our assumption, we have $c_{2k+1}\geq c_{2k}>c_{2k-1}$ and
  \[
    \tau^{s} = (c_{2k},\tau_{2k-2},\cdots, \tau_{2})\times (c_{2k+1}, c_{2k-1}).
  \]
  The the behavior of $\uptau^{s}$ under the descent map $\eDD$ is illostrated
  as the following:
  \begin{equation}\label{eq:ds2.B.sp}
    \uptau^{s}: \hspace{1em}
    \tytb{
      {*(srcol)\cdots}\cdots\cdots,
      {*(srcol)\ast}{\ast}{\cdots},
      {*(srcol)\bullet},
      {*(srcol)\vdots},{*(srcol)\bullet},{x_0},\none,\none,\none}
    \times
    \tytb{{*(srcol)\cdots}\cdots\cdots,
      {*(srcol)\ast}{\ast}{\cdots},{*(srcol)\bullet},{*(srcol)\vdots},{*(srcol)\bullet},{x_{1}},\vdots,{x_{n}},{m_{\uptau}}}
    \mapsto
    \uptau'^{s}: \hspace{1em}
    \tytb{{*(srcol)\cdots}\cdots\cdots,{*(srcol)\ast}{\ast}{\cdots},{*(srcol)s},{*(srcol)\vdots},{*(srcol)s},{x'_{0}},
      \none,\none,\none}
    \times \tytb{\none\cdots\cdots,\none{\ast}{\cdots},\none,\none,\none,\none,\none,\none,\none}
    \mapsto
    \uptaupp:
    \tytb{\none\cdots\cdots,\none{\ast}{\cdots},\none,\none,\none,\none,\none,\none,\none}
    \times \tytb{\none\cdots\cdots,{\none}{\ast}{\cdots},\none{\phantom{m}\mathclap{m_{\uptaupp}}},\none,\none,\none,\none,\none,\none}
  \end{equation}
  The grey part consists of totally $2(c_{2k}-1)=C_{2k}-1$ dots.
  Hence the total signature of the gray part is $(C_{2k}-1,C_{2k}-1)$.
  Note that $\uptau^{s}$ is obtained from the ``$\ast\cdots$'' part of $\taupp$ by attaching the grey part
  and $x_{0},\cdots, x_{n},m_{\uptau}$.

  The descent of a non-special diagram $\uptau^{ns}$:
  \begin{equation}\label{eq:ds2.B.nsp}
        \uptau^{ns}: \hspace{1em}
        \tytb{
        {*(srcol)\cdots}\cdots\cdots,
        {*(srcol)\ast}{\ast}{\cdots},
        \none,\none,\none,\none,\none,\none,\none}
      \times
      \tytb{{*(srcol)\cdots}\cdots\cdots,
        {*(srcol)\ast}{\ast}{\cdots},
        {*(srcol)s}{*(srcol)r},
        {*(srcol)\vdots}{*(srcol)\vdots},
        {*(srcol)s}{*(srcol)r},
        {x_{1}}{x_{0}},{\vdots},{x_{n}},{m_{\uptau}}}
        \mapsto
       {\uptaup}^{ns}: \tytb{{*(srcol)\cdots}\cdots\cdots,{*(srcol)\ast}{\ast}{\cdots},\none,\none,\none,\none,\none,\none,\none}
       \times
       \tytb{\none\cdots\cdots,\none{\ast}{\cdots},
         \none{*(srcol)r},
         \none{*(srcol)\vdots},
         \none{*(srcol)r},
         \none{x'_{0}},\none,\none,\none}
       \mapsto
       \uptaupp:
       \tytb{\none\cdots\cdots,\none{\ast}{\cdots},\none,\none,\none,\none,\none,\none,\none}
       \times \tytb{\none\cdots\cdots,{\none}{\ast}{\cdots},\none{\phantom{m}\mathclap{m_{\uptaupp}}},\none,\none,\none,\none,\none,\none}
     \end{equation}
     Here the gray parts in $\uptau$ consists of $C_{2k}-1$ marks of $\bullet$, $s$, or
     $r$ and $s$ and $r$ occur with the same multiplicity. Hence the total
     signature of the gray part is $(C_{2k}-1,C_{2k}-1)$.

     To prove the lemma, it suffice to verify the following claims which we leave to the reader:
     \begin{itemize}
       \item In both the special shape and non-special shape cases, the
       following map is bijective:
       \[
       x_{0}x_{1}\cdots x_{n}m_{\uptau^{s/ns}}\mapsto (\bfxx_{\uptau^{s/ns}},m_{\uptaupp}).
       \]
       \item The following equation of signatures holds
       \[
         \ssign( x_{0}x_{1}\cdots x_{n}m_{\uptau} ) - \ssign(m_{\uptaupp})  =\ssign(\bfxx_{\uptau}).
       \]
     \end{itemize}
     In the verification, one can use the fact that $m_{\uptau}=m_{\uptaupp}$
     when $x_{1} = r/d$.

     \trivial[]{
       Suppose $x_{1}=r/d$. Then $x_{0} = c/d$ and
       \[\ssign(x_{0}\cdots x_{n}m_{\uptau})-\ssign(m_{\uptaupp}) = (1,1)+\ssign(x_{1}\cdots x_{n}).\]

       Now we consider the generic cases, i.e. $x_{1}=\bullet/s$:
       In this case, we clain that $\ssign(x_{0}x_{1}) - \ssign(m_{\uptaupp}) = (1,2)$.
       Now
       \[
         \begin{split}
           & \ssign(x_{0}x_{1}\cdots x_{n}m_{\uptau})-\ssign(m_{\uptaupp}) \\
           =& (1,1)+\ssign(x_{2}\cdots x_{n}m_{\uptau}) +(0,1) \\
           = &\begin{cases} (1,1) +\ssign(x_{2}\cdots x_{n}) + \ssign(c)
             & \text{if } m_{\uptau} = a\\
             (1,1) +\ssign(x_{2}\cdots x_{n}) + \ssign(s) & \text{if } m_{\uptau} = b\\
           \end{cases}\\
         \end{split}
       \]

       First consider the special shape diagram $\uptau=\uptau^{s}$.
       \begin{enumPF}
         \item Suppose $m_{\uptaupp}=a$. Then $x_{0}\times x_{1}=\bullet\times \bullet$.
           \item Suppose $m_{\uptaupp}= b$. Then $x_{0}\times x_{1}=c\times s$.
       \end{enumPF}
       Now consider the non-special shape diagram $\uptau=\uptau^{ns}$.
       \begin{enumPF}
         \item Suppose $m_{\uptaupp}=a$. Then
         $\emptyset \times x_{1}x_{0}=\emptyset \times sr$.
         \item Suppose $m_{\uptaupp}= b$. Then
         $\emptyset \times x_{1}x_{0}=\emptyset \times sd$.
       \end{enumPF}
       The macthing between $\uptau^{s}$ and $\uptau^{ns}$ is also clear by the
       above listing of cases.
     }
 \end{proof}



\subsubsection{A key proposition for the generalized descent case}

Now assume $C_{2k}$ is even. By our assumption, we have
$c_{2k+1}\geq c_{2k}=c_{2k-1}$.

\begin{equation}\label{eq:gd.drc.B}
  \uptau: \hspace{1em}
  \tytb{
    {*(srcol)\cdots}\cdots\cdots,
    {y_{0}}{\cdots}{\cdots},
    \none,\none,\none,\none}
  \times
  \tytb{{*(srcol)\cdots}\cdots\cdots,
    {x_{1}}{x_{0}}{\cdots},
    {x_{2}},{\vdots},{x_{n}},{m_{\uptau}}}
  \mapsto
  \uptaup: \tytb{{*(srcol)\cdots}\cdots\cdots,{y'_{0}}{\cdots}{\cdots},\none,\none,\none,\none}
  \times \tytb{\none\cdots\cdots,\none{x'_{0}}{\cdots},\none,\none,\none,\none}
  \mapsto
  \uptaupp: \tytb{\none\cdots\cdots,{\none}{\cdots}{\cdots},\none,\none,\none,\none}
  \times \tytb{\none\cdots\cdots,\none{x''}{\cdots},\none{\phantom{m}\mathclap{m_{\uptaupp}}},\none,\none,\none}
\end{equation}
Here the gray parts in $\uptau$ are two columns of $\bullet$ of length
$C_{2k}/2-1$.

\begin{lem}\label{lem:gd.inj.M}
  In \eqref{eq:gd.drc.B}, $x''=x_{0}$.
  The map $\delta\colon  \drc(\cO)\rightarrow \drc(\cOpp) \times \drc(\cO_{1})$
  given by $\uptau\mapsto (\eDDo^{2}(\uptau),\bfxx_{\uptau})$ is injective.
  Moreover,
  \[
  \ssign(\uptau) =\ssign(\uptaupp) + (C_{2k}-1, C_{2k}-1)+\ssign(\bfxx_{\uptau}).
  \]
  The map $\tdelta\colon \drc(\cO)\rightarrow \drc(\cOpp) \times \bN^{2}\times \bZ/2\bZ$
  given by $\uptau\mapsto (\eDDo^{2}(\uptau), \ssign(\uptau),\upepsilon_{\uptau})$ is injective.
\end{lem}
\begin{proof}
  The claim $x''=x_{0}$, the injectivity of $\delta$ and the siginature formula follows directly from our algorithm.

  Now the injectivity of $\tdelta$ follows from
  $\bfxx_{\uptau}\mapsto (\ssign(\bfxx_{\uptau}), \upepsilon)$ is injective by
  \Cref{c:init.CD}.
  \trivial{
    We can fully recover $\uptau$ from $\bfxx_{\uptau}$ and $\uptaupp$.

    Suppose $\bfxx_{\uptau}$ dose not contains $s$ or $c$.
    Then $x_{1}=r/d$,  $m_{\uptau}=m_{\uptaupp}$, $y_{0}=c$.
    Hence the claim holds.
    \[
      \ssign(\uptau) =\ssign(\uptaupp) + (C_{2k}-2,C_{2k}-2) + \ssign(c) +\ssign(\bfxx_{\uptau})
    \]

    Now assume $\bfxx_{\uptau}$ contain at least one $s$ or $c$.
    Now
    \[
      m_{\uptau} = \begin{cases}
        a & \text{if $\bfxx_{\uptau}$ contains $c$}\\
        b & \text{otherwise}
      \end{cases}
    \]
    Therefore, $\ssign(x_{2}\cdots x_{n}m_{\uptau}) = \ssign(\bfxx_{\uptau}) - (0,1)$.


    Clearly, we can recover $x_{2}\cdots x_{n}$ from $\bfxx_{\uptau}$ by
    deleting the $c$ (if it exists) or a $s$.
    On the other hand, $(y_{0}, x_{1})$ is completely determined by
    $m_{\uptaupp}$:
    \[
      (y_{0},x_{1}) = \begin{cases}
        (\bullet, \bullet) & \text{if $m_{\uptaupp}=a$}\\
        (c, s) & \text{if $m_{\uptaupp}=b$}
      \end{cases}
    \]
    Hence $\ssign(y_{0}x_{1}) = \ssign(m_{\uptaupp}) + (1,2)$.

    Now
    $\ssign(y_{0}x_{0}x_{1}\cdots x_{n}m_{\uptau})  =  \ssign(\bfxx_{\uptau}) +\ssign(x'' m_{\uptaupp}) + (1,2)-(0,1)$.
    This yields the signature identity.
  }
\end{proof}


\subsection{Proof of the type BM case}
We will reduce the proof to the type CD case.

In this section, $k\geq 1$ and
$\cO = (C_{2k+1}, C_{2k}, \cdots,C_{1}, C_{0}=0)\in \dpeNil(B)$.
$\cOp:=\eDD(\cO)$ and $\cOpp := \eDD(\cOp)$.

Take $\uptau\in \drc(\cO)$, we marks the entries in $\uptau$ as in  \cref{eq:ds2.B.sp,eq:ds2.B.nsp,eq:gd.drc.B}.
\subsubsection{Usual decent case}
Thanks to \Cref{lem:sp-nsp.B}, the proof in the descent case is exactly the same
as that in \Cref{sec:pf.ds.CD}.

\subsubsection{Reduction to type CD case in the general descent case.}

Suppose $C_{2k}$ is even.
We define $\tcO = (C_{2k+1}-C_{2k}+1,1,1)\in\dpeNil(D)$ and
$\tcOpp := \eDDo(\tcO)= (2)$.

The key proposition of reduction.
\begin{prop}
 We have  the following properties:
  \begin{enumS}
    \item When $\uptaupp = \eDDo^{2}(\uptau)$ for certain $\uptau\in \drc(\cO)$,
    we have $x_{\uptaupp}\neq s$;
    \item For each $\uptaupp$ such that $x_{\uptaupp}\neq s$, we have a
    bijection
    \[
      \begin{tikzcd}[row sep=0em]
       \bdelta \colon  \set{\uptau\in \drc(\cO)| \eDDo^{2}(\uptau)=\uptaupp} \ar[r] &
        \set{(x_{\tuptau},\bfuu_{\tuptau})| \tuptau\in \drc(\tcO) \text{ s.t.
          } \eDDo^{2}(\tuptau)=x_{\uptaupp}}\\
        \uptau \ar[r,mapsto] & (x_{\uptaupp}, \bfxx_{\uptau})
      \end{tikzcd}
  \]
  where $\bfuu_{\tuptau}$ is defined in \eqref{eq:def.u}.
  \item
  We have the following properties:
  \begin{enumS}
    \item $x_{\tuptau}=s$ if and only if $x_{n}m_{\uptau} = sb$ or $\bullet b$.
    \item $x_{\tuptau}=d$ if and only if $x_{n}=d$. \qedhere
  \end{enumS}
  \end{enumS}
\end{prop}


The situation could be summarized in the following commutative diagram:
  \[
    \begin{tikzcd}
     & \uptau \ar[r,"\delta",mapsto] \ar[d,mapsto]& (\uptaupp, \bfxx_{\uptau})\ar[d,mapsto]\ar[r,mapsto] & \uptaupp\ar[d,mapsto]\\
     x_{\uptau} \ar[d,leftrightarrow,dashed] &\ar[l,mapsto]  \bfpp_{\uptau}\ar[r,mapsto]\ar[d,leftrightarrow]
     & (x_{\uptaupp}, \bfxx_{\uptau}) \ar[d,equal]\ar[r,mapsto]& x_{\uptaupp} \ar[d,equal]\\
    x_{\tuptau} & \ar[l,mapsto] \tuptau \ar[r,"\delta",mapsto]& (\tuptaupp,\bfuu_{\tuptau})\ar[r,mapsto]& \tuptaupp
    \end{tikzcd}
  \]
  We remark that $x_{\uptau}$ and $x_{\tuptau}$ are equal in the most of the
  case, especially when $n=1$. But there are exceptional cases.

  \begin{proof}
    This follows from a case by case verification according to our algorithm.
    We list all possible cases below:

    When $n=1$, see \Cref{tb:rd1}.% We have the following table:
    \begin{table}[p]
      \[
        \begin{array}{c|c|c|c|c}
          \hline
          \hline
          x_{\uptau} & \tytb{{\txx_{\uptaupp}},{m_{\uptaupp}}} & \bfpp_{\uptau} & (x_{\uptaupp},\bfxx_{\uptau}) & \tuptau_{L} \\
          \hline
          r & \tytb{r,a} &  \tytb{\bullet r,b} & (r,s) & \tytb{sr} \\
          \cline{3-5}
                     &            &  \tytb{\bullet r,a} & (r,c) & \tytb{rc} \\
          \hline
          r & \tytb{r,b} &  \tytb{sr,b} & (r,s) & \tytb{sr} \\
          \cline{3-5}
                     &            &  \tytb{sr,a} & (r,c) & \tytb{rc} \\
          \hline
          c & \tytb{s,a} &  \tytb{\bullet s,b} & (c,s) & \tytb{sc} \\
          \cline{3-5}
                     &            &  \tytb{\bullet s,a} & (c,c) & \tytb{cc} \\
          \hline
          d & \tytb{d,a} & \tytb{\bullet d,b} & (d,s) & \tytb{sd} \\
          \cline{3-5}
                     &            & \tytb{\bullet d,a} & (d,c) & \tytb{cd} \\
          \cline{3-5}
                     &            & \tytb{r d,a} & (d,r) & \tytb{rd} \\
          \cline{3-5}
                     &            & \tytb{d d,a} & (d,d) & \tytb{dd} \\
          \cline{2-5}
                     & \tytb{d,b} & \tytb{s d,b} & (d,s) & \tytb{sd} \\
          \cline{3-5}
                     &            & \tytb{s d,a} & (d,c) & \tytb{cd} \\
          \cline{3-5}
                     &            & \tytb{r d,b} & (d,r) & \tytb{rd} \\
          \cline{3-5}
                     &            & \tytb{d d,b} & (d,d) & \tytb{dd} \\
          \hline
          \hline
        \end{array}
      \]
      \caption{Reduction when $n=1$}
      \label{tb:rd1}
    \end{table}

    Now assume $n\geq 1$.
    We let $\bfxx = x_{1}x_{2}\cdots x_{n}$ and define
    \[
      \bfxx' = \begin{cases}
        \bfxx \text{ deleteing ``$c$''}  & \text{if } c\in \bfxx\\
        \bfxx \text{ deleteing ``$s$''}  & \text{if } c\notin \bfxx, s\in \bfxx\\
        \text{undefined} & \text{otherwise.}
      \end{cases}
    \]
    Now all the cases are listed in \Cref{tb:rd2}.
    % If $c\in \bfxx$, let $\bfxx'$ be the string obtained by deleting  $c$ from
    % $\bfxx$.
    % In the following table, if $c$ occures in $\bfxx$ or
    % $\bfxx'$, we  will view $c$ as $s$ and
    % arrange the order of entries when place them in $\bfpp_{\uptau}$.

    \begin{table}[p]
      \[
        \begin{array}{c|c|c|c|c|c:c}
          \hline
          \hline
          x_{\uptau} & \tytb{{\txx_{\uptaupp}},{m_{\uptaupp}}} & \bfpp_{\uptau} & x_{\uptaupp}, \bfxx_{\uptau}
          & \tuptau_{L} & \multicolumn{2}{ c}{\text{conditions}}\\
          \hline
          r & \tytb{r,a} &  \tytb{\bullet r,{\bfxx'},a} & \tytb{r{,}{\bfxx}} & \tytb{{\bfxx}r,}& x_{1}\neq r& c\in \bfxx  \\
          \cline{5-6}
                     & & & &\tytb{rc,{\bfxx'}} & x_{1}=r& \\
          \cline{3-3}\cline{4-7}
                     &            &  \tytb{\bullet r,{\bfxx'},b} & \tytb{r{,}{\bfxx}} & \tytb{{\bfxx}r}
                        & \multicolumn{2}{c}{c\notin \bfxx,s\in \bfxx} \\
          \hline
          r & \tytb{r,b} &  \tytb{sr,{\bfxx'},a} & \tytb{r{,}{\bfxx}} & \tytb{{\bfxx}r,}& x_{1}\neq r& c\in \bfxx  \\
          \cline{5-6}
                     & & & &\tytb{rc,{\bfxx'}} & x_{1}=r& \\
          \cline{3-3}\cline{4-7}
                     &            &  \tytb{sr,{\bfxx'},b} & \tytb{r{,}{\bfxx}} & \tytb{{\bfxx}r}
                        & \multicolumn{2}{c}{c\notin \bfxx,s\in \bfxx} \\
          \hline
          c & \tytb{s,a} &\tytb{\bullet s,{\bfxx'},a} & \tytb{c{,}{\bfxx}} & \tytb{{\bfxx}c}
                        & c\in \bfxx   & s\in \bfxx \text{ is automatic} \\%\multicolumn{2}{c}{c\in \bfxx}\\
          \cline{3-3}\cline{6-6}
                     &            &\tytb{\bullet s,{\bfxx'},b} &                    &
                        & c\notin \bfxx   &\\%\multicolumn{2}{c}{c\in \bfxx}\\
          \hline
          d & \tytb{d,a} &\tytb{\bullet d,{\bfxx'},a} & \tytb{d{,}{\bfxx}} & \tytb{{\bfxx}d}& \multicolumn{2}{c}{c\in \bfxx}\\
          \cline{3-3}\cline{6-7}
                     &            &\tytb{\bullet d,{\bfxx'},b} &                    &
                        & x_{1}=s & c\notin \bfxx   \\%\multicolumn{2}{c}{c\in \bfxx}\\
          \cline{3-3}\cline{6-7}
                     &            &\tytb{{\bfxx}d,a} &                    &
                        & x_{1}\neq s & c\notin \bfxx   \\%\multicolumn{2}{c}{c\in \bfxx}\\
          \hline
          d & \tytb{d,b} &\tytb{s d,{\bfxx'},a} & \tytb{d{,}{\bfxx}} & \tytb{{\bfxx}d}
                        & \multicolumn{2}{c}{c\in \bfxx}\\
          \cline{3-3}\cline{6-7}
                     &            &\tytb{s d,{\bfxx'},b} &                    &
                        & x_{1}=s & c\notin \bfxx   \\%\multicolumn{2}{c}{c\in \bfxx}\\
          \cline{3-3}\cline{6-7}
                     &            &\tytb{{\bfxx}d,b} &                    &
                        & x_{1}\neq s & c\notin \bfxx   \\%\multicolumn{2}{c}{c\in \bfxx}\\
          \hline
          \hline
        \end{array}
      \]
      \caption{Reduction when $n\geq 1$}
      \label{tb:rd2}
    \end{table}
  \end{proof}

  Thanks to $\bdelta$ defined in  the above lemma, we can use \eqref{eq:gd.ls}.
  The rest of the proof is similar to that in \Cref{sec:pf.gd.CD}.
  We leave the details to the reader.

% \begin{prop}
%   Inductively, we can establish the following properties:
%   \begin{enumS}
%     \item  The local system $\cL_{\uptau}$ has the factorization according to
%     $\txx_{\uptau}$.
%   \end{enumS}
% \end{prop}


\end{document}
