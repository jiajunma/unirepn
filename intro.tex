\documentclass[ssunip.tex]{subfiles}

\begin{document}

\section{Introduction and the main results}\label{sec:intro}

\subsection{Unitary representations and the orbit method}
A fundamental problem in representation theory is to determine the unitary dual
of a given Lie group $G$, namely the set of equivalent classes of irreducible
unitary representations of $G$. A principal idea, due to Kirillov and Kostant,
is that there is a close connection between irreducible unitary representations
of $G$ and the orbits of $G$ on the dual of its Lie algebra \cite{Ki62,Ko70}.
This is known as orbit method (or the method of coadjoint orbits). Due to its
resemblance with the process of attaching a quantum mechanical system to a
classical mechanical system, the process of attaching a unitary representation
to a coadjoint orbit is also referred to as quantization in the representation
theory literature.

As it is well-known, the orbit method has achieved tremendous success in the
context of nilpotent and solvable Lie groups \cite{Ki62,AK}. For more general
Lie groups, work of Mackey and Duflo \cite{Ma,Du82} suggest that one should
focus attention on reductive Lie groups. As expounded by Vogan in his writings
(see for example \cite{VoBook,Vo98,Vo00}), the problem finally is to quantize
nilpotent coadjoint orbits in reductive Lie groups. The ``corresponding''
unitary representations are called unipotent representations.

Significant developments on the problem of unipotent representations occurred in
the 1980's. We mention two. Motivated by Arthur's conjectures on unipotent
representations in the context of automorphic forms \cite{ArPro,ArUni}, Adams,
Barbasch and Vogan established some important local consequences for the unitary
representation theory of the group $G$ of real points of a connected reductive
algebraic group defined over $\R$. See \cite{ABV}. The problem of finding
(integral) special unipotent representations for complex semisimple groups (as
well as their distribution characters) was solved earlier by Barbasch and Vogan
\cite{BVUni} and the unitarity of these representations was established by
Barbasch for complex classical groups \cite{B.Class}. In a similar vein,
Barbasch outlined his proof of the unitarity of special unipotent
representations for real classical groups in his 1990 ICM talk \cite{B.Uni}. The
second major development is Vogan's theory of associated varieties \cite{Vo89}
in which Vogan pursues the method of coadjoint orbits by investigating the
relationship between a Harish-Chandra module and its associate variety. Roughly
speaking, the Harish-Chandra module of a representation attached to a nilpotent
coadjoint orbit should have a simple structure after taking the ``classical
limit'', and it should have a specified support dictated by the nilpotent
coadjoint orbit via the Kostant-Sekiguchi correspondence.

Simultaneously but in an entirely different direction, there were significant
developments in Howe's theory of (local) theta lifting and it was clear by the
end of 1980's that the theory has much relevance for unitary representations of
classical groups. The relevant works include the notion of rank by Howe
\cite{HoweRank}, the description of discrete spectrum by Adams \cite{Ad83} and
the preservation of unitarity in stable range theta lifting by Li \cite{Li89}.
Therefore it was natural, and there were many attempts, to link the orbit method
with Howe's theory, and in particular to construct unipotent representations in
this formalism. See for example \cite{Sa,Pz,HZ,HL,Br,He,Tr,PT,B17}. Particularly
worth mentioning were the work of Przebinda \cite{Pz} in which a double
fiberation of moment maps made its appearance in the context of theta lifting,
and the work of He \cite{He} in which an innovative technique called quantum
induction was devised to show the non-vanishing of the lifted representations.
More recently the double fiberation of moment maps was successfully used by a
number of authors to understand refined (nilpotent) invariants of
representations such as associated cycles and generalized Whittaker models
\cite{NOTYK, NZ, GZ, LM}, which among other things demonstrate the tight link
between the orbit method and Howe's theory.


In the present article we will demonstrate that the orbit method and Howe's
theory in fact have perfect synergy when it comes to unipotent representations.
(Barbasch, M{\oe}glin, He and Trapa pursued a similar theme. See
\cite{B17,Mo17,He,Tr}.) We will restrict our attention to a real classical group
$G$ of orthogonal or symplectic type and we will construct all unipotent
representations of $\wt{G}$ attached to $\CO$ (in the sense of Barbasch and
Vogan) via the method of theta lifting. Here $\wtG$ is either $G$, or the
metaplectic cover of $G$ if $G$ is a real symplectic group, and $\CO$ is a
member of our preferred set of complex nilpotent orbits, large enough to include
all those which are rigid special. Here we wish to emphasize that there have
been extensive investigations of unipotent representations for real reductive
groups by Vogan and his collaborators (see e.g. \cite{VoBook,Vo89,ABV}),
%in particular on the related problems of classification of the primitive ideals and Fourier inversion of unipotent orbit integrals \cite{BVPri1, BVPri2},
only for unitary groups complete results are known (\emph{cf}. \cite{BV83, Tr}), in which case all such representations may also be described in terms of cohomological induction.




\subsection{Special unipotent representations of classical groups of type $B$, $C$ or $D$}\label{secsu}
In this article, we are aimed to understand special unipotent representations of classical groups of type $B$, $C$ or $D$. As the cases of complex orthogonal groups  and complex symplectic groups are well-understood (see
\cite{BVUni} and \cite{B17}), we will focus on the following groups:
\be\label{typebcd}
  \oO(p,q), \Sp_{2n}(\R), \  \widetilde \Sp_{2n}(\R), \ \Sp(p,q), \  \oO^*(2n),
  \ee
  where $p,q, n\in \BN:=\{0,1,2, \cdots\}$. Here $ \widetilde \Sp_{2n}(\R)$ denotes the real metaplectic group, namely the double cover of the symplectic group  $\Sp_{2n}(\R)$ that does not split unless $n=0$.

Let $G$ be one the groups in   \eqref{typebcd}.  As usual, we view it as  a real form of $G_\C$, or a double cover of a real form of $G_\C$ in the metaplecitic case, where
\[
  G_\C:=
  \left\{
    \begin{array}{ll}
      \oO_{p+q}(\C) , & \hbox{if $G=\oO(p,q)$;} \smallskip\\
    \Sp_{2n}(\C) , & \hbox{if $G=\Sp_{2n}(\R)$ or $\widetilde \Sp_{2n}(\R)$;} \smallskip \\
  \Sp_{2p+2q}(\C) , & \hbox{if $G=\Sp(p,q)$;} \smallskip \\
 \oO_{2n}(\C) , & \hbox{if $G=\oO^*(2n)$.} \\
    \end{array}
  \right.
\]
Respectively write $\fgg_\R$ and $\g$ for the Lie algebras of $G$ and $G_\C$. Then $\g$ is viewed as a complexification of $\g_\R$.

Let $r_\g$ denote the rank of $\fgg$. Let $W_{r_\g}$ denote the subgroup of $\GL_{r_\g}(\C)$ generated
by the permutation matrices and the diagonal matrices with diagonal entries
$\pm 1$.
 %We identify a Cartan subalgebra of $\g_V$ with $\C^{r_V}$, using the standard coordinates.
 Then as usual, Harish-Chandra
isomorphism yields an identification
\be\label{ugz}
  \oU(\g)^{G_\C}=\left(\oS(\C^{r_\g})\right)^{W_{r_\g}}.
\ee
Here and henceforth,  ``$\oU$'' indicates the universal enveloping algebra of a Lie algebra,  a superscript group indicate the
space of invariant vectors under the group action, and  ``$\oS$'' indicates
the symmetric algebra. Unless $G_\C$ is an even orthogonal group,
$\oU(\g)^{G_\C}$ equals the center $\oZ(\g)$ of $\oU(\g)$.
By \eqref{ugz}, we have the following parameterization of  characters of $\oU(\g)^{G_\C}$:
\[
  \Hom_{\mathrm{alg}}(\oU(\g)^{G_\C}, \C)=W_{r_\g} \backslash (\C^{r_\g})^*= W_{r_\g}\backslash \C^{r_\g}
\]
Here ``$ \Hom_{\mathrm{alg}}$" indicates the set of $\C$-algebra homomorphisms, and a superscript ``$\,^*\,$" over a vector space indicates the dual space.


   We define the Langlands dual of $G$ to be the complex group
\[
  \check G:=
  \left\{
    \begin{array}{ll}
      \Sp_{p+q-1}(\C) , & \hbox{if $G=\oO(p,q)$ and $p+q$ is odd;} \smallskip\\
    \oO_{p+q}(\C) , & \hbox{if $G=\oO(p,q)$ and $p+q$ is even;} \smallskip \\
    \oO_{2n+1}(\C) , & \hbox{if $G=\Sp_{2n}(\R)$;} \smallskip \\
   \Sp_{2n}(\C) , & \hbox{if $G=\widetilde \Sp_{2n}(\R)$;} \smallskip \\
 \oO_{2p+2q+1}(\C) , & \hbox{if $G=\Sp(p,q)$;} \smallskip \\
 \oO_{2n}(\C) , & \hbox{if $G=\oO^*(2n)$.} \\
    \end{array}
  \right.
\]
 Write $\check \fgg$ for the Lie algebra of $\check G$.

Denote by $\Nil(\check \fgg)$ the set of nilpotent  $\check G$-orbits in $\check \fgg$, namely the set of all  orbits of nilpotent matrices under the adjoint action of $\check G$ in $\check \g$.
When no confusion is possible, we will not distinguish a nilpotent orbit in $\GL_n(\C)$, $\oO_n(\C)$ or $\Sp_{2n}(\C)$ with its corresponding Young diagram. In particular, the zero orbit is also regarded as the Young  with  one nonempty column at most.

Let  $\check \CO \in \Nil(\check \fgg) $.  It determines a character $\chi(\check \CO): \oU(\g)^{G_\C}\rightarrow \C$ as in what follows. For every  integer $a\geq 0$, write
\[
  \rho(a):=\left\{ \begin{array}{ll}
                  (1, 2, \cdots, \frac{a-1}{2}), \quad &\textrm{if $a$ is odd;}\\
                    (\frac{1}{2}, \frac{3}{2}, \cdots, \frac{a-1}{2}), \quad &\textrm{if $a$ is even;}\\
                    \end{array}
                 \right.
\]
By convention, $\rho(1)$ and $\rho(0)$ are  the empty sequence.
Write $a_1\geq  a_2\geq \cdots\geq a_s>0$ ($s\geq 0$)  for the row lengths of  $\check \CO$. Define
\be\label{chico}
 \chi(\check \CO):= (\rho( a_1), \rho(a_2),  \cdots, \rho(a_s), 0, 0, \cdots, 0 ),
\ee
to be viewed as a character $\chi(\check \CO): \oU(\g)^{G_\C}\rightarrow \C$.
Here the number of $0$'s is
\[
 \left\lfloor\frac{\textrm{the number of odd rows of the Young diagram of $\check \CO$}}{2}\right\rfloor.
\]


Recall the following well-known result of Dixmier (\cite[Section 3]{Bor}): for every algebraic character $\chi$ of $\oZ(\g)$, there exists a unique maximal ideal of $\oU(\g)$ that contains the kernel of $\chi$. %, to be called the maximal ideal of $\oU(\g)$ with infinitesimal character $\chi$.
As an easy consequence, we know that there is a unique maximal $G_\C$-stable ideal of $\oU(\g)$ that contains the kernel of $\chi(\check \CO)$. Write $I_{\check \CO}$ for this ideal.

Recall that a  smooth Fr\'echet representation of  moderate growth  of a real reductive group is called a Casselman-Wallach representation (\cite{Ca89,Wa2}) if its Harish-Chandra module has  finite length. When $G=\widetilde \Sp_{2n}(\R)$ is a metaplectic group, write $\varepsilon_G$ for the non-trivial element in the kernel of the covering map $G\rightarrow \Sp_{2n}(\R)$. Then a representation of $G$ is said to be genuine if $\varepsilon_G$ acts on it through the scalar multiplication by $-1$.
Following Barbasch and Vogan (\cite{ABV,BVUni}), we make the following definition.

\begin{defn}
%Let $\check \CO\in \Nil(\check \g)$.
Let $\check \CO\in \Nil(\check \g)$. An irreducible Casselman-Wallach representation $\pi$ of $G$  is attached to $\check \CO$ if
\begin{itemize}
\item  $I_{\check \CO}$ annihilates $\pi$; and
\item $V$ is genuine if $G$ is a metaplectic group.
\end{itemize}
\end{defn}

Write  $\Unip_{\check \cO}(G)$ for the  set of all isomorphism classes of irreducible Casselman-Wallach  representations of $G$ that are attached to $\check \CO$.
We say that an irreducible Casselman-Wallach representation  of $G$ is special unipotent if it is attached to $\check \CO$, for some $\check \CO\in \Nil(\check \g)$.
We will construct all the special unipotent representations, and show that they are all unitarizable as predicted by  the Arthur-Barbasch-Vogan conjecture \cite[Introduction]{ABV}.



\subsection{Combinatorics :  bipartitions}\label{secbip}

For every Young diagram $\CO$, write
\[
 \mathbf r_1(\CO)\geq \mathbf r_2(\CO)\geq \mathbf r_3(\CO)\geq \cdots
\]
for its row lengths, and similarly,
write
\[
 \mathbf c_1(\CO)\geq \mathbf c_2(\CO)\geq \mathbf c_3(\CO)\geq \cdots
\]
for its column lengths.
Denote by $\abs{\CO}:=\sum_{i=1}^\infty \mathbf r_i(\CO)$ the total size of $\CO$.

We introduce six symbols $B$, $C$, $D$, $\widetilde {C}$, $C^*$ and $D^*$ to indicate the types of the groups that we are considering as in \eqref{typebcd}, namely odd real orthogonal groups, real symplectic groups, even real orthogonal groups, real metaplectic groups, quaternionic symplectic groups and quaternionic orthogonal groups, respectively.
  Let $\star\in \{B,C,D,\widetilde {C}, C^*, D^*\}$, and suppose that $G$ has type $\star$.
  %Then $\check G$ is a complex symplectic group if $\star=B$ or $\widetilde{C}$, and is a complex orthogonal group otherwise.

  Following \cite[Definition 4.1]{MR},  We say that
    $\check \CO\in \Nil(\check \g)$ has  $\star$-good parity if
\[
  \left\{ \begin{array}{l}
               \textrm{all nonzero row lengths of $\check \CO$ are even if $\check G$ is a complex symplectic group; and }\\
                     \textrm{all nonzero row lengths of $\check \CO$ are odd if $\check G$ is a complex orthogonal group.}
                       % \textrm{the total size $\abs{\check \CO}$ is  odd if and only if $\star\in \{C, C^*\}$}.
                          \end{array}
                 \right.
\]
%We will also not distinguish a Young diagram with the corresponding partition of a natural number.
%Let $\mathrm{YD}$ denote the set of all Young diagrams.
The study of the special unipotent representations in general case reduced to the case when $\check \CO$ has $\star$-good parity. See Appendix \ref{secapp}. Now we suppose that  $\check \CO$ has  $\star$-good parity. Equivalently but in a completely combinatorial setting, we consider $\check \CO$ as a Young diagram that has  $\star$-good parity in the following sense:
\[
  \left\{ \begin{array}{l}
               \textrm{all nonzero row lengths of $\check \CO$ are even if $\star\in \{B, \widetilde{C}\}$;}\\
                     \textrm{all nonzero row lengths of $\check \CO$ are odd if $\star\in \{C, D, C^*, D^*\}$; and}\\
                        \textrm{the total size $\abs{\check \CO}$ is  odd if and only if $\star\in \{C, C^*\}$}.                   \end{array}
                 \right.
\]
%In particular, the empty Young diagram $\emptyset$ is $\star$-even.

%We call an ordered pair of Young diagrams a  bipartition. In what follows we will define a  set   $\mathrm{BP}_\star(\check \CO)$ of bipartitions.

\begin{defn}
 A $\star$-pair is a pair  $(i,i+1)$ of successive positive integers such that
\[
   \left\{
     \begin{array}{ll}
      i\textrm{ is odd}, \quad &\textrm{if $\star\in\{C, \widetilde{C}, C^*\}$};  \\
      i \textrm{ is even}, \quad &\textrm{if $\star\in\{B, D, D^*\}$}. \\
       \end{array}
   \right.
\]
A $\star$-pair   $(i,i+1)$ is said to be
\begin{itemize}
\item
vacan in $\check \CO$, if $\mathbf r_i(\check \CO)=\mathbf r_{i+1}(\check \CO)=0$;
\item
balanced in $\check \CO$,  if  $\mathbf r_i(\check \CO)=\mathbf r_{i+1}(\check \CO)>0$;
\item
tailed in $\check \CO$,  if $\star\in \{C, D, C^*, D^*\}$ and $\mathbf r_i(\check \CO)>\mathbf r_{i+1}(\check \CO)=0$;
\item
primitive in $\check \CO$, if  it is not empty, balanced or tailed in $\check \CO$.
\end{itemize}

\end{defn}

Let $\mathrm{PP}_\star(\check \CO)$ denote the  set of all $\star$-pairs that are primitive in $\check \CO$. For every subset $\wp\subset \mathrm{PP}_\star(\check \CO)$, we define a pair \[
(\imath_\wp, \jmath_\wp):=(\imath_\star(\check \CO, \wp), \jmath_\star(\check \CO, \wp))
\]
 of Young diagrams  as in what follows.

If $\star=B$, then
 \[
   \mathbf c_{1}(\jmath_\wp)=\frac{\mathbf r_1(\check \CO)}{2},
\]
and for all $i\geq 1$,
\[
(\mathbf c_{i}(\imath_\wp), \mathbf c_{i+1}(\jmath_\wp))=
   \left\{
     \begin{array}{ll}
           (\frac{\mathbf r_{2i+1}(\check \CO)}{2},  \frac{\mathbf r_{2i}(\check \CO)}{2}), &\hbox{if $(2i, 2i+1)\in \wp$}; \smallskip\\
            (\frac{\mathbf r_{2i}(\check \CO)}{2},  \frac{\mathbf r_{2i+1}(\check \CO)}{2}), &\hbox{otherwise}.\\
            \end{array}
   \right.
\]


If $\star=\widetilde{C}$, then for all $i\geq 1$,
\[
(\mathbf c_{i}(\imath_\wp), \mathbf c_{i}(\jmath_\wp))=
   \left\{
     \begin{array}{ll}
           (\frac{\mathbf r_{2i}(\check \CO)}{2},  \frac{\mathbf r_{2i-1}(\check \CO)}{2}), &\hbox{if $(2i-1, 2i)\in \wp$}; \smallskip\\
            (\frac{\mathbf r_{2i-1}(\check \CO)}{2},  \frac{\mathbf r_{2i}(\check \CO)}{2}), &\hbox{otherwise}.\\
            \end{array}
   \right.
\]


If $\star=D$ or $D^*$, then
 \[
   \mathbf c_{1}(\imath_\wp)= \left\{
     \begin{array}{ll}
            \frac{\mathbf r_1(\check \CO)+1}{2},   &\hbox{if $\mathbf r_1(\check \CO)>0$}; \smallskip\\
       0,  &\hbox{if $\mathbf r_1(\check \CO)=0$},\\
            \end{array}
   \right.
 \]
and for all $i\geq 1$,
\[
(\mathbf c_{i}(\jmath_\wp), \mathbf c_{i+1}(\imath_\wp))=
   \left\{
     \begin{array}{ll}
            (\frac{\mathbf r_{2i+1}(\check \CO)-1}{2},  \frac{\mathbf r_{2i}(\check \CO)+1}{2}), &\hbox{if $(2i, 2i+1)\in \wp$}; \smallskip\\
        (\frac{\mathbf r_{2i}(\check \CO)-1}{2},  0), & \hbox{if $(2i, 2i+1)$ is tailed in $\check \CO$};\smallskip\\
         (0,  0), &\hbox{if $(2i, 2i+1)$ is vacant in $\check \CO$};\\
         (\frac{\mathbf r_{2i}(\check \CO)-1}{2},  \frac{\mathbf r_{2i+1}(\check \CO)+1}{2}), &\hbox{otherwise}.\\
            \end{array}
   \right.
\]


If $\star=C$ or $C^*$, then for all $i\geq 1$,
\[
(\mathbf c_{i}(\jmath_\wp), \mathbf c_{i}(\imath_\wp))=
   \left\{
     \begin{array}{ll}
            (\frac{\mathbf r_{2i}(\check \CO)-1}{2},  \frac{\mathbf r_{2i-1}(\check \CO)+1}{2}), &\hbox{if $(2i-1, 2i)\in \wp$}; \smallskip\\
        (\frac{\mathbf r_{2i-1}(\check \CO)-1}{2},  0), & \hbox{if $(2i-1, 2i)$ is tailed in $\check \CO$};\smallskip\\
         (0,  0), &\hbox{if $(2i-1, 2i)$ is vacant  in $\check \CO$};\\
         (\frac{\mathbf r_{2i-1}(\check \CO)-1}{2},  \frac{\mathbf r_{2i}(\check \CO)+1}{2}), &\hbox{otherwise}.\\
            \end{array}
   \right.
\]


In all cases, it is known that (see \cite[Section13.2]{Carter})
\be\label{eqbp}
(\imath_\star(\check \CO, \wp), \jmath_\star(\check \CO, \wp))=(\imath_\star(\check \CO', \wp'), \jmath_\star(\check \CO', \wp'))\quad \textrm{if and only if}\quad
 \left\{
     \begin{array}{l}
           \check \CO=\check \CO';\\
           \wp=\wp',
            \end{array}
   \right.
\ee
where $\check \CO'$ is another  Young diagram that has $\star$-good parity, and $\wp'\subset \mathrm{DP}_\star(\check \CO')$.
%Recall that  the pair $\mathrm{BP}_\star(\check \CO)$ parametrizes certain Weyl group representations (\cite[Section 11.4]{Carter}).


\begin{Example} Suppose that $\star=C$, and $\check \CO$ is the following Young diagram which has $\star$-good parity.
\begin{equation*}\label{eq:sp-nsp.C}
  \tytb{\ \ \ \ \  , \ \ \  , \ \ \ , \ \ \  , \ \ \ , \  ,\  }
   \end{equation*}
   Then
\[
  \mathrm{DP}_\star(\check \CO)=\{(1,2), (5,6)\}.
\]
and $(\imath_\wp, \jmath_\wp)$ %\in \mathrm{BP}_\star(\check \CO)$
has the  following form.

\begin{equation*}\label{eq:sp-nsp.C}
\begin{array}{rclcrcl}
  \wp=\emptyset & : & \tytb{\ \ \ ,\ \  } \times \tytb{\ \ \ , \  }  & \qquad \quad &  \wp=\{(1,2)\}& : & \tytb{\ \ \  , \ \ , \   } \times \tytb{\ \ \  } \medskip \medskip \medskip \\
    \wp=\{(5,6)\} & : & \tytb{\ \ \ ,\ \ \ } \times \tytb{\ \ , \   }  & \qquad \quad &  \wp=\{(1,2), (5,6)\}  & : & \tytb{\ \ \  , \ \ \ ,  \ } \times \tytb{\ \   } \\
  \end{array}
  \end{equation*}

\end{Example}


Here and henceforth, when no confusion is possible, we write $\alpha\times \beta$ for a pair $(\alpha, \beta)$.  We will also write $\alpha\times \beta\times \gamma$ for a triple $(\alpha, \beta, \gamma)$.


\subsection{Combinatorics : painted bipartitions}

For every Young diagram $\imath$, we view the set $\BOX(\imath)$ of boxes of $\imath$ as the following subset 
of $\bN^+\times \bN^+$ ($\bN^+$ denotes the set of positive integers):
\begin{equation}\label{eq:BOX}
\BOX(\imath):=\Set{(i,j)\in\bN^+\times \bN^+| j\leq \bfrr_i(\imath)}.
\end{equation}
%We will also call a subset of $\bN^+\times \bN^+$  of the form \eqref{eq:BOX} a Young diagram.

%We say that a Young diagram $\imath'$ is contained
%in $\imath$ (and write $\imath'\subset \imath$) if
%\[
%  \mathbf r_i(\imath')\leq \mathbf r_i(\imath)\qquad \textrm{for all } i=1,2, 3, \cdots.
%\]
%When  this is the case, $\mathrm{Box}(\imath')$ is viewed as a subset of $\mathrm{Box}(\imath)$ concentrating on the upper-left corner.
%We say that a subset of $\mathrm{Box}(\imath)$ is a Young subdiagram if it equals $\mathrm{Box}(\imath')$ for a Young diagram $\imath'\subset \imath$.
%  In this case, we call $\imath'$ the Young diagram corresponding to this Young subdiagram.

We introduce five symbols $\bullet$, $s$, $r$, $c$ and $d$.
\begin{defn}
A painting on a Young diagram $\imath$ is a map
\[
  \CP: \mathrm{Box}(\imath) \rightarrow \{\bullet, s, r, c, d \}
\]
with the following properties:
\begin{itemize}
\item
 $\CP^{-1}(S)$ is the set of boxes of a Young diagram when $S=\{\bullet\}, \{\bullet, s \}, \{\bullet, s, r\}$ or $\{\bullet, s, r, c \} $;
 \item
 every row of $\imath$ has at most one  box in $\CP^{-1}(S)$ when $S=\{s\}$ or $ \{r\}$;
   \item
 every column of $\imath$ has at most one  box in $\CP^{-1}(S)$ when $S=\{c\}$ or $ \{d \}$.
 \end{itemize}
A painted Young diagram is a pair $(\imath, \CP)$ consisting of a Young diagram and a painting on it.

\end{defn}


\begin{Example} Suppose that $\tau=\tytb{\ \ ,\  }$, then there are $25+12+6+2=45$ paintings on $\tau$ in total as listed below.
\begin{equation*}\label{eq:sp-nsp.C}
\begin{array}{ll}
   \tytb{\bullet \alpha ,\beta } \quad \alpha, \beta\in \{\bullet, s,r,c,d\} \qquad \qquad  \qquad  & \tytb{s \alpha ,\beta } \quad \alpha\in \{r,c,d\}, \beta\in \{s,r,c,d\} \medskip \medskip \\
     \tytb{r \alpha ,\beta } \quad \alpha\in \{c,d\}, \beta\in \{r,c,d\} \qquad \qquad  \qquad
   &  \tytb{c \alpha , d } \quad  \alpha\in \{c,d\}
     \end{array}
  \end{equation*}

  \end{Example}


 We introduce two more symbols $B^+$ and $B^-$, and make the following definition.
 \begin{defn}\label{defpbp0}
 A painted bipartition is a triple $\tau=(\imath, \CP)\times (\jmath, \CQ)\times \alpha$, where $(\imath, \CP)$ and $ (\jmath, \CQ)$ are painted Young diagrams, and $\alpha\in \{B^+,B^-, C,D,\widetilde {C}, C^*, D^*\}$, subject to the following conditions:
 \begin{itemize}
 \item
 $\CP^{-1}(\bullet)=\CQ^{-1}(\bullet)$;
 \item
 the image of $\CP$ is contained in
 \[
 \left\{
     \begin{array}{ll}
         \{\bullet, c\}, &\hbox{if $\alpha=B^+$ or $B^-$}; \smallskip\\
            \{\bullet,  r, c,d\}, &\hbox{if $\alpha=C$}; \smallskip\\
          \{\bullet, s, r, c,d\}, &\hbox{if $\alpha=D$}; \smallskip\\
            \{\bullet, s, c\}, &\hbox{if $\alpha=\widetilde{ C}$}; \smallskip \\
        \{\bullet\}, &\hbox{if $\alpha=C^*$}; \smallskip \\
          \{\bullet, s\}, &\hbox{if $\alpha=D^*$},\\
            \end{array}
   \right.
 \]
 \item
 the image of $\CQ$ is contained in
 \[
 \left\{
     \begin{array}{ll}
         \{\bullet, s, r, d\}, &\hbox{if $\alpha=B^+$ or $B^-$}; \smallskip\\
           \{\bullet, s\}, &\hbox{if $\alpha=C$}; \smallskip\\
           \{\bullet\}, &\hbox{if $\alpha=D$}; \smallskip\\
             \{\bullet, r, d\}, &\hbox{if $\alpha=\widetilde{ C}$}; \smallskip\\
        \{\bullet, s,r\}, &\hbox{if $\alpha=C^*$}; \smallskip \\
          \{\bullet, r\}, &\hbox{if $\alpha=D^*$}.
            \end{array}
   \right.
 \]


 \end{itemize}


 \end{defn}
 With the notation of Definition \ref{defpbp0}, we define in what follows some objects attached to the painted bipartition $\tau$:
 \[
   \star_\tau, \ \check \CO_\tau, \  \wp_\tau, \  \abs{\tau}, \  \varepsilon_\tau, \  p_\tau, \ q_\tau, \ G_\tau\, \ \dim \tau.
 \]

 \smallskip


 \smallskip




\noindent $\star_\tau$:  This is the symbol given by
   \[
  \star_\tau:= \left\{
     \begin{array}{ll}
         B, &\hbox{if $\alpha=B^+$ or $B^-$}; \smallskip\\
            \alpha, & \hbox{otherwise}.           \end{array}
   \right.
   \]

   \noindent $\check \CO_\tau$ and $ \wp_\tau$: This is the unique pair such that
   \begin{itemize}
   \item
   $\check \CO_\tau$ is a Young diagram that has $\star_\tau$-good parity;
   \item $\wp_\tau\subset \mathrm{DP}_{\star_\tau}(\check \CO_\tau)$; and
   \item
   $
   (\imath, \jmath)=(\imath_{\star_\tau}(\check \CO_\tau, \wp_\tau), \jmath_{\star_\tau}(\check \CO_\tau, \wp_\tau)).
$
\end{itemize}
The existence of such a pair is a consequence of the explicit description of the coherent continuation representations. See   \cite[Theorems 6, 7, 10, 11]{Mc} and 
\cite[Proposition 6.9]{RT} 


 \smallskip

 \smallskip

 \noindent $\abs{\tau}$: This is the natural number \[
  \abs{\tau}:=\abs{\imath}+\abs{\jmath}.
\]
Note that
\[
 \abs{\tau}:= \left\{
     \begin{array}{ll}
        \frac{\abs{\check \CO_\tau}-1}{2}, &\hbox{if $\star_\tau=C$ or $C^*$}; \smallskip\\
          \frac{\abs{\check \CO_\tau}}{2}, &\hbox{otherwise}. \smallskip\\
                      \end{array}
   \right.
\]



 \smallskip

 \smallskip

 \noindent $\varepsilon_{\tau}$: This is an element of $\Z/2\Z$. It equals $0$ in the following cases:
   \begin{itemize}
   \item[(a)]
   $\star_\tau=B$, $\abs{\tau}>0$ and  the symbol $d$ occurs in the first column of  the painted Young diagram $(\jmath, \CQ)$;
    \item[(b)]
   $\star_\tau=D$, $\abs{\tau}>0$ and  the symbol $d$ occurs in the first column of  the painted Young diagram $(\imath, \CP)$;
   \item[(c)]
   $\star_\tau\in \{C, \widetilde C\}$ and  $(1,2)\not\in \wp_\tau$.
  \end{itemize}
In all other case, $\varepsilon_{\tau}=1$.

 \smallskip


 \smallskip



  \noindent $p_{\tau}$ and $q_\tau$: This is a pair of natural numbers given by counting  the various symbols appearing in $(\imath, \CP)$, $(\jmath, \CQ)$ and $\{\alpha\}$ :
  \[
  \left\{
     \begin{array}{l}
    p_\tau := \# \bullet+ 2 \# r + \# c + \# d + \# B^+;\smallskip\\
    q_\tau := \# \bullet+ 2 \# s + \# c + \# d + \# B^-.\\
    \end{array}
    \right.
\]

\smallskip


 \smallskip


  \noindent $G_{\tau}$: This is a classical group given by
  \[
 G_\tau:= \left\{
     \begin{array}{ll}
         \oO(p_\tau, q_\tau), &\hbox{if $\star_\tau=B$ or $D$}; \smallskip\\
            \Sp_{2\abs{\tau}}(\R), &\hbox{if $\star_\tau=C$}; \smallskip\\
           \widetilde{\Sp}_{2\abs{\tau}}(\R), &\hbox{if $\star_\tau=\widetilde{ C}$}; \smallskip \\
        \Sp(\frac{p_\tau}{2}, \frac{q_\tau}{2}), &\hbox{if $\star_\tau=C^*$}; \smallskip \\
          \oO^*(2\abs{\tau}), &\hbox{if $\star_\tau=D^*$}.\\
            \end{array}
   \right.
\]

\smallskip


 \smallskip

  \noindent $\dim \tau$:
This is the dimension of the standard representation of the complexification of $G_\tau$, equvalently,
 \[
 \dim \tau:= \left\{
     \begin{array}{ll}
          2\abs{ \tau}+1, &\hbox{if $\star_\tau=B$}; \medskip\\
         2 \abs{\tau}, &\hbox{otherwise}.
            \end{array}
   \right.
 \]


 \smallskip

\begin{Example} Suppose that
\[
\tau= \tytb{\bullet \bullet ,\bullet , c } \times \tytb{\bullet \bullet r d ,\bullet , d }\times B^+.
\]
Then
\[
\star_\tau=B, \qquad \check \CO_\tau =\tytb{\ \ \ \ \ \ , \ \ \ \ \ \ , \ \ , \ \ , \ \ , \ \  },  \qquad \wp_\tau=\{(6,7)\}, \quad \abs{\tau}=4+6=10, \qquad \varepsilon_\tau=0,
\]
and
\[
 p_\tau=6+2+1+2+1=12, \qquad q_\tau=6+0+1+2+0=9, \qquad G_\tau=\oO(12,9), \qquad \dim \tau=21.
 \]
\end{Example}

Write $\mathrm{PBP}$ for the set of all painted bipartitions. Then
\[
   \mathrm{PBP}=\bigsqcup_{(\star, \check \CO)}  \mathrm{PBP}_\star(\check \CO),
\]
where $\star$ runs over the set $\{ B,\mathrm  C,  D, \widetilde{C},  C^*, D^*\}$, $\check \CO$ runs over all Young diagrams that has $\star$-good parity, and
 \[
  \mathrm{PBP}_\star(\check \CO):=\{\tau\in \mathrm{PBP} \mid \star_\tau=\star, \ \check \CO_\tau=\check \CO\}.
   \]


\subsection{Descents of painted bipartitions and theta lifts}

Let $\star$ and $\check \CO$ be as before. For every $\tau\in \mathrm{PBP}_\star(\check \CO)$, in what follows we will construct a representation $\pi_\tau$ of $G_\tau$ by theta lift.
For the starting case when $\check \CO$ is the empty Young diagram, define
 \[
 \pi_\tau:= \left\{
     \begin{array}{ll}
         \textrm{the one dimensional genuine representation} , &\hbox{ if $\star=\widetilde C$ so that $G_\tau=\widetilde{\Sp}_0(\R)$;} \smallskip\\
 \textrm{the one dimensional trivial representation} , &\hbox{  if $\star\neq \widetilde C$.}            \end{array}
   \right.
\]



Define a symbol
\[
\star':=\widetilde{C}, \ D, \  C, \ B, \ D^*\  \textrm{ or } \ C^*
\]
respectively if
\[
\star=B,\  C, \ D, \ \widetilde{C}, \ C^* \ \textrm{ or }\  D^*.
\]
We call $\star'$
 the Howe dual of $\star$. Now we assume that $\check \CO$ is nonempty, and define its decent to be the Young diagram
 \[
   \check \CO':=\textrm{the one obtained from $\check \CO$ by removing the first row.}
         \]
% Here $\emptyset$ stands for the empty Young diagram, and $\tytb{\ }$ stands for the Young diagram of total size $1$.
 Note that $\check \CO'$ has $\star'$-good parity.
 In section \ref{sec:desc}, we will define the descent map
 \[
   \nabla:  \mathrm{PBP}_\star(\check \CO)\rightarrow  \mathrm{PBP}_{\star'}(\check \CO').
 \]

 Let $\tau\in  \mathrm{PBP}_\star(\check \CO)$.  Put $\tau':=\nabla(\tau)$.
   Let $(W_{\tau, \tau'}, \la \,\cdot\,,\,\cdot\,\ra_{\tau, \tau'})$ be a real symplectic space of  dimension $\dim \tau\cdot \dim \tau'$. As usual, there are continuous homomorphisms $G_\tau\rightarrow \Sp(W_{\tau, \tau'})$ and $G_{\tau'}\rightarrow \Sp(W_{\tau, \tau'})$ whose images form a reductive dual pair in $\Sp(W_{\tau, \tau'})$. We form the semidirect prduct
   \[
   J_{\tau, \tau'}:=(G_\tau\times G_{\tau'})\ltimes \oH(W_{\tau, \tau'}),
   \]
   where
  \[
  \oH(W_{\tau, \tau'}):=W_{\tau, \tau'}\times \R
  \]
  is the Heisenberg group with group multiplication
  \[
  (w,t)(w',t'):=(w+w', t+t'+\la w,w'\ra_{\tau, \tau'}), \quad
  w,w'\in W_{\tau, \tau'},\,\,t,t'\in \R.
 \]

 Let $\omega_{\tau, \tau'}$ be a suitably normalized smooth oscillator representation of $J_{\tau, \tau'}$ such that every $t\in \R\subset J_{\tau, \tau'}$ acts on it through the scalar multiplication by $e^{2\pi \sqrt{-1}\, t}$ (the letter $\pi$  often denotes a representation, but here it stands for the circumference ratio).  See Section \ref{secosc0} for details.

 For every  Casselman-Wallach representation $\pi'$ of $G_{\tau'}$, write
 \[
   \check \Theta_{\tau'}^{\tau}(\pi'):=(\omega_{\tau, \tau'}\widehat \otimes \pi')_{G_{\tau'}} \qquad (\textrm{the Hausdorff coinvariant space}),
 \]
 where $\widehat \otimes$ indicates the complete projective tensor product. This is a Casselman-Wallach representation of $G_\tau$.
 Now we define the representation $\pi_\tau$ of $G_\tau$ by induction on $\mathbf c_1(\check \CO)$:
 \[
   \pi_\tau:=\left\{
     \begin{array}{ll}
          %\textrm{the trivial representation $\C$}, &\hbox{if $\abs{\check \CO_\tau}\leq 1$}; \medskip\\
         \check \Theta_{\tau'}^{\tau}(\pi_{\tau'})\otimes (1_{p_\tau, q_\tau}^{+,-})^{\varepsilon_\tau}, &\hbox{if  $\star_\tau=B$ or $D$}; \\
         \check \Theta_{\tau'}^{\tau}(\pi_{\tau'}\otimes \det^{\varepsilon_\tau}), &\hbox{if $\star_\tau=C$ or $\widetilde C$}; \\
              \check \Theta_{\tau'}^{\tau}(\pi_{\tau'}), &\hbox{if $\star_\tau=C^*$ or $D^*$}. \\
            \end{array}
   \right.
 \]
 Here $1_{p_\tau, q_\tau}^{+,-}$ denotes the character of $\oO(p_\tau, q_\tau)$ whose restriction to $\oO(p_\tau)\times \oO(q_\tau)$ equals $1\otimes \det$ ($1$ stands for the trivial character).


 Put
\[
  \mathrm{Unip}_{\star}(\check \CO):=\left\{
     \begin{array}{ll}
         \bigsqcup_{p,q\in \BN, p+q=\abs{\check \CO}+1} \Unip_{\check \cO}(\oO(p,q)), &\hbox{if $\star=B$}; \medskip\\
           \Unip_{\check \CO}(\Sp_{\abs{\check \CO}-1}(\R)), &\hbox{if $\star=C$}; \medskip\\
           \bigsqcup_{p,q\in \BN, p+q=\abs{\check \CO}} \Unip_{\check \cO}(\oO(p,q)), &\hbox{if $\star=D$}; \medskip\\
          \Unip_{\check \CO}(\widetilde \Sp_{\abs{\check \CO}}(\R)), &\hbox{if $\star=\widetilde{ C}$}; \medskip \\
     \bigsqcup_{p,q\in \BN, 2p+2q=\abs{\check \CO}-1} \Unip_{\check \cO}(\Sp(p,q)), &\hbox{if $\star=C^*$}; \medskip \\
          \Unip_{\check \CO} \oO^*(\abs{\check \CO}), &\hbox{if $\star=D^*$}.\\
            \end{array}
   \right.
\]

We are now ready to formulate our first main theorem.
\begin{thm}\label{thm1}
Let $\star\in \{B, C,D,\widetilde {C}, C^*, D^*\}$, and let $\check \CO$ be a Young diagram that has $\star$-good parity.

\noindent (a) For every $\tau\in \mathrm{PBP}_\star(\check \CO)$, the representation $\pi_\tau$ of $G_\tau$ is irreducible and attached to $\check \CO$.

\noindent  (b) If $\star\in \{B,D\}$ and $\abs{\check \CO}>0$, then the map
\[
\begin{array}{rcl}
\mathrm{PBP}_\star(\check \CO)\times \Z/2\Z&\rightarrow &\mathrm{Unip}_{\star}(\check \CO),\\
  (\tau, \epsilon)&\mapsto& \pi_\tau\otimes \det^\epsilon
  \end{array}
\]
is bijective.

\noindent
(c) In all other cases, the map
\[
\begin{array}{rcl}
\mathrm{PBP}_\star(\check \CO)&\rightarrow &\mathrm{Unip}_{\star}(\check \CO),\\
  \tau&\mapsto& \pi_\tau
  \end{array}
\]
is bijective.
\end{thm}

By the above theorem (and the reduction in Appendix \ref{secapp}), we have explicitly constructed all special unipotent representations of the classical groups in \eqref{typebcd}. By this construction and the method of matrix coefficient integrals, we are able to prove the unitarity of these representations.

\begin{thm}
All special unipotent representations of the classical groups in \eqref{typebcd} are unitarizable.
\end{thm}

\subsection{Associated cycles}
Let $G,  G_\C, \g_\R, \g, \check \g$ and $\check \CO \in \mathrm{Nil}(\check \g)$ be as in Section \ref{secsu}. Fix a maximal compact subgroup $K$ of $G$ whose Lie algebra is denoted by $\frak k_\R$. Then we have an orthogonal decomposition
\[
  \g=\frak k\oplus \p,
\]
where $\frak k$ is the complexification of $\frak k_\R$, and $\g$ is equipped with the trace form. By taking the dual spaces, we have a
decomposition
\[
  \g^*=\frak k^*\oplus \p^*,
\]
Write $K_\C$ for the complexification of the complex group $K$. It is a complex algebraic group with an obvious algebraic action on $\p^*$.


As before, suppose that $G$ has type $\star$ and $\check \CO$ has $\star$-good parity. Write $\CO\in \mathrm{Nil}(\g^*)$ for the Barbarsch-Vogan dual of $\check \CO$ so that its Zariski closure in $\g^*$ equals the associated variety of the ideal $I_{\check \CO}\subset \oU(\g)$.
The algebraic variety
$
  \CO\cap \p^*
$
is a finite union of $K_\C$-orbits.  Given such an orbit $\sO$, write
$\cK_{\sO}(K_\C)$ for the  Grothendieck group of the category of $K_\C$-equivariant algebraic vector bundles on $\sO$. Put
\[
\cK_{\CO}(K_\C):=\bigoplus_{\sO\textrm{ is a $K_\C$-orbit in
      $\CO\cap \p^*$}} \cK_{\sO}(K_\C).
\]

%Recall the group $\cK^{\mathbb p}_{\cO}(\wtbfK)$ from  \Cref{sec:LVB}.


We say that a Casselman-Wallach representation of $G$ is  $\CO$-bounded  if
the associated variety  of its annihilator ideal
is contained in the Zariski closure of $\CO$. Note that all representations in $\mathrm{Unip}_{\check \CO}(G)$ are $\CO$-bounded. Write $\cK(G)_{\CO-\textrm{bounded}}$ for the  Grothendieck group of the category of all such representations.
From \cite[Theorem 2.13]{Vo89},  we have a canonical homomorphism
\[
\xymatrix{
  \mathrm{AC}_\cO\colon   \cK(G)_{\CO-\textrm{bounded}} \ar[r]& \cK_{\CO}(K_\C).
}
\]
We call $ \mathrm{AC}_\cO(\pi)$ the associated cycle of $\pi$, where $\pi$ is an $\CO$-bounded Casselman-Wallach representation of $G$. This is a very important invariant attached to $\pi$.


Following Vogan \cite[Section 8]{Vo89}, we make the following definition.

\begin{defn}\label{defaod}
  Let $\sO$ be a $K_\C$-orbit in $\CO\cap \p^*$. An admissible orbit datum over
  $\sO$ is an irreducible $K_\C$-equivariant algebraic vector bundle $\CE$
  on $\sO$ such that
  \begin{itemize}
    \item $\CE_X$ is isomorphic to a multiple of
    $(\bigwedge^{\mathrm{top}} \fkk_X )^{\frac{1}{2}}$ as a representation of
    $\fkk_X$;
    \item if $\star=\widetilde C$, then $\varepsilon_G$ acts on $\CE$ by the scalar multiplication by $-1$.
  \end{itemize}
  Here $X\in \sO$, $\CE_X$ is the fibre of $\cE$ at $X$, $\fkk_X$
  denotes the Lie algebra of the stabilizer of $X$ in $K_\C$, and
  $(\bigwedge^{\mathrm{top}} \fkk_X)^{\frac{1}{2}}$ is a one-dimensional
  representation of $\fkk_X$ whose tensor square is the top degree wedge
  product $\bigwedge^{\mathrm{top}} \fkk_X$.
\end{defn}

Note that in the situation of the classical groups we consider here, all admissible
orbit data are line bundles.  Denote by $\mathrm{AOD}_{\sO}(K_\C)$ the
set of isomorphism classes of admissible orbit data over $\sO$, to be viewed as
a subset of $\CK_{\CO}(K_\C)$. Put
\[
  \mathrm{AOD}_{\CO}(K_\C):=\bigsqcup_{\sO\textrm{ is a $K_\C$-orbit in
      $\CO\cap \p^*$}} \mathrm{AOD}_{\sO}(K_\C)\subset
  \CK_{\CO}(K_\C).
\]


Recall that a nilpotent orbit in $\check \g$ is said to be distinguished if it is has no intersection with every  proper Levi subalgebra of $\check \g$. Combinatorially, this is equivalent to saying that no pair of rows of the Young diagram  has equal nonzero length. Note that all  distinguished nilpotent orbits in $\check \g$ has $\star$-good parity.

\begin{defn}
\noindent
(a) The orbit $\check \CO$ (which has $\star$-good parity) is said to be quasi-distinguished if there is no $\star$-pair that is balanced in $\check \CO$.

\noindent
(b) If  $\star\in \{B, D, D^*\}$, then $\check \CO$ is said to be  be pseudo-distinguished if there is no positive even integer $i$ such that $\mathbf{r}_i(\check \CO)=\mathbf{r}_{i+1}(\check \CO)= \mathbf{r}_{i+2}(\check \CO)=\mathbf{r}_{i+3}(\check \CO)>0$.


\noindent
(c) If  $\star\in \{C, \widetilde C, C^*\}$ so that its Howe dual $\star'\in  \{B, D, D^*\}$, then $\check \CO$ is said to be  be pseudo-distinguished if either it is the empty Young diagram or it is nonempty and its descent $\check \CO'$ (which is a Young diagram that has $\star'$-good parity) is   pseudo-distinguished.

\end{defn}





We will calculate the associated character of $\pi_\tau$ for every  bipartition $\tau$. The calculation is a key ingredient in the proof of Theorem \ref{thm1}.
Especially, we have the following theorem concerning the associated characters of the special unipotent representations.

\begin{thm}
Let $G$ be a group in \eqref{typebcd} that has type $\star\in \{B, C,D,\widetilde {C}, C^*, D^*\}$, and suppose that $\check \CO\in \mathrm{Nil}(\check \g)$  has $\star$-good parity.

\noindent (a) For every $\pi\in \mathrm{Unip}_{\check \CO}(G)$, the associated cycle $\mathrm{AC}_{\CO}(\pi)\in \cK_{\CO}(K_\C) $ is a nonzero  sum of  pairwise distinct elements of $\mathrm{AOD}_{\CO}(K_\C)$.

\noindent  (b) If $\check \CO$ is pseudo-distinguished, then the map
\[
\mathrm{AC}_\CO: \mathrm{Unip}_{\check \CO}(G)\rightarrow  \cK_{\CO}(K_\C)
\]
is injective.

\noindent  (c) If $\check \CO$ is quasi-distinguished, then the map $\mathrm{AC}_{\check \CO}$ induces a bijection
\[
\mathrm{Unip}_{\check \CO}(G)\rightarrow  \mathrm{AOD}_{\CO}(K_\C).
\]

\end{thm}

\begin{remark}
Suppose that $\star\in \{C^*, D^*\}$ so that $G$ is quaternionic.
Then there is  precisely one admissible orbit datum over  $\sO$ for each $K_\C$-orbit $\sO\subset \CO\cap \p^*$.
Thus
\[
 \mathrm{AOD}_{\sO}(K_\C)=K_\C\backslash  (\CO\cap \p^*).
\]
If $\check \CO$ is not quasi-distinghuished, then
$\CO\cap \p^*$ is empty (see \cite[Theorems 9.3.4 and 9.3.5]{CM}), and hence $\mathrm{Unip}_{\check \CO}(G)$ is also empty.


\end{remark}




\end{document}
