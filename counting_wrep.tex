\documentclass[counting_main.tex]{subfiles}
\begin{document}

\section{Combinatorics of Weyl group representations in the classical types}

\subsection{The $j$-induction}
If $\mu$ and $\nu$ are two partitions representing two symmetric groups
representations.
Then
\[
  j_{S_{\abs{\mu}}\times S_{\abs{\nu}}}^{S_{\abs{\mu}+\abs{\nu}}}
  \mu\boxtimes \nu
  = \mu\cup \nu,
\]
where $\mu\cup \nu$ is the partition such that
\[
  \set{\bfcc_{i}(\mu\cup\nu) | i\in \bN^{+}} =
  \set{\bfcc_{i}(\mu) | i\in \bN^{+}}
  \cup
  \set{\bfcc_{i}(\nu) | i\in \bN^{+}}
\]
as multisets.

\trivial{
Use the inductive by stage of $j$-induction
\[
  \begin{split}
  &j_{S_{\abs{\mu}}\times S_{\abs{\nu}}}^{S_{\abs{\mu}+\abs{\nu}}}
  \mu\boxtimes \nu\\
  &=
j_{S_{\abs{\mu}}\times S_{\abs{\nu}}}^{S_{\abs{\mu}+\abs{\nu}}}
j^{S_{\abs{\mu}}\times S_{\abs{\nu}}}_{\prod_{i}S_{\bfcc_{i}(\mu)}\times
  \prod_{j} S_{\bfcc_{j}(\nu)}}\sgn\\
 &=
j^{S_{\abs{\mu}+\abs{\nu}}}_{\prod_{i}S_{\bfcc_{i}(\mu\cup \nu)}}\sgn\\
  &= \mu\cup \nu.
  \end{split}
\]
}


We have
\[
  j_{S_{n}}^{W_{n}} \sgn = \begin{cases}
    (\cboxs{k},\cboxs{k}) & \text{if $n=2k$ is even,}\\
    (\cboxs{k+1},\cboxs{k}) & \text{if $n=2k+1$ is odd.}\\
    \end{cases}
\]
\trivial{
  The symbol of trivial of trivial group is
  \[
    \symb{0, 1, \cdots, k}{0,\cdots, k-1} .
  \]
  Apply the formula \cite{Lu}*{4.5.4},
  When $n$ is even, the symbol of the induce is
  \[
    \symb{0,2, \cdots, k+1}{1,\cdots, k}.
  \]
  corresponds to $(\cboxs{k}, \cboxs{k})$.

  When $n$ is even, the symbol of the induce is
  \[
    \symb{1,2, \cdots, k+1}{1,\cdots, k}.
  \]
  corresponds to $(\cboxs{k+1}, \cboxs{k})$.
}

If $\tau = (\tau_{L}, \tau_{R})$ and $\sigma = (\sigma_{L}, \sigma_{R})$
be two bipartition. Then
\[
  j_{W_{\abs{\tau}}\times W_{\abs{\sigma}}}^{W_{\abs{\tau}+\abs{\sigma}}}
  \tau \boxtimes \sigma = (\tau_{L}\cup \sigma_{L}, \tau_{R}\cup \sigma_{R})
\]


\delete{
\subsection{Parameterize of Unipotent representations}
We fix an abstract complex Cartan subgroup $\bfH_a$ and $\fhh_a$ in $\bfG$ and a
set of simple roots $\Pi_a$.  Let $\cP(\bfG)$ be the set of all Langlands
parameters of $G$-modules with character $\rho$ (i.e. the infinitesimal
character of the trivial representation). For $\gamma\in \cP(\bfG)$, let
$\cL(\gamma)$, $\cS(\gamma)$ and $\Phi_\gamma$ be the corresponding Langlands
quotient, standard module and coherent family such that
$\Phi_\gamma(\rho) = \cL(\gamma)$. Let $\cM(\bfG)$ be the span of $\cL(\gamma)$.
Let $\set{\bB}$ be the set of all blocks. Then $\cP(\bfG) = \bigsqcup_\cB \cB$.
The Weyl group $W = W(G)$ acts on $\cM(\bfG)$ by coherent continuation.  Let
$\cM_{\cB}$ be the submodule of $\cM(\bfG)$ spanned by $\gamma\in\cB$, then
\[
  \cM(\bfG) = \bigoplus_\cB \cM_{\cB}
\]
Let $\tau(\gamma)\subset \Pi_a$ be the $\tau$-invariant of $\gamma$.

Let $\ckcO$ be even orbit. $\lambda= \half \ckhh$.  Define
\[
  S(\lambda) = \set{\alpha\in \Pi_a| \inn{\alpha}{\lambda}=0}.
\]
Let $\cP_{\lambda}(\bfG)$ be the set of all Langlands parameters with
infinitesimal character $\lambda$. Let $T_{\lambda,\rho}$ be the translation
functor.  Let
\[
  \cB(S) = \set{\gamma\in \cB|S\cap \tau(\gamma)=\emptyset}
\]
and
\[
  \cP(\bfG,S) = \bigsqcup_{\cB} \cB(S)
\]


Then
\[
  \begin{tikzcd}[row sep=0em]
    \cP(\bfG,S) \ar[r] & \cP_{\lambda}(\bfG)\\
    \gamma \ar[r, maps to]& T_{\lambda, \rho}(\gamma)
  \end{tikzcd}
\]

Let $\cO$ be a complex nilpotent orbit in $\fgg$.  Let
\[
  \cB(S,\cO) = \set{\gamma\in \cB(S)|\AVC(\cL(\gamma))\subset \bcO}
\]

Let
\[
  \begin{aligned}
    m_S(\sigma) &= [\sigma: \Ind_{W(S)}^{W}\bfone]\\
    m_\cB(\sigma)& = [\sigma: \cM_\cB]
  \end{aligned}
\]


Barbasch \cite{B10}*{Theorem~9.1} established the following theorem.
\begin{thm}
  \[
    |\cB(S,\cO)| = \sum_{\sigma} m_\cB(\sigma)m_S(\sigma)
  \]
  Here $\sigma\times \sigma$ running over the $W\times W$ appears in the double
  cell $\cC(\cO)$.
\end{thm}
\begin{proof}
  We need to take the graded module of $\cM(\bfG)$ with respect to the
  $\LRleq$. By abuse of notation, we identify the basis $\cP(\bfG)$ with its
  image in the graded module.  Note that $S\cap \tau(\lambda)=\emptyset$ if and
  only if $W(S)$ acts on $\gamma$ trivially by \cite[Lemma~14.7]{V4}.  On the
  other hand, by \cite[Theorem~14.10, and page 58]{V4},
  $\AVC(\cL(\gamma))\subset \bcO$ only if $\gamma$ generate a $W$-module in the
  double cell of $\cO$.
\end{proof}

Now assume $S=S(\lambda)$. By \cite[Cor~5.30 b) and c)]{BVUni},
$[\sigma: \Ind_{W(S)}^{W}\bfone]=[\bfone|_{W(S)}:\sigma]\leq 1$.

}



% \section{Remarks on the Counting theorem of unipotent representations}



% \subsection{Coherent family}
% For each finite dimensional $\fgg$-module or $\Gc$-module $F$, let
% $F^*$ be its contragredient representation and let
% $\WT{F}\subseteq \aX$ denote the multi-set of weights in $F$.

% Let $\PiGlfin$ be the set of irreducible finite dimensional representations of $\Gc$
% with extreme weight in $\Lambda_0$ and $\Glfin$ be the subgroup generated by $\PiGlfin$.
% Let
% \[
% \aP  := \set{\mu \in \aX| \text{$\mu$ is a $\hha$-weight of an $F\in \PiGfin$}}.
% \]
% Via the highest weight theory,
% every $W$-orbit $W\cdot \mu$ in $\aP$ corresponds with the irreducible finite dimensional representation
% $F\in \PiGfin$ with extremal weight $\mu$.

% Now the Grothendieck group $\Gfin$ of finite dimensional representation of $\Gc$
% is identified with $\bZ[\aP/W]$. In fact $\Gfin$ is a $\bZ$-algebra under the
% tensor product and equipped with the involution $F\mapsto F^*$.

% Fix a $W$-invariant sub-lattice $\Lambda_0\subset \aX$ containing $\aQ$.

% %  Let $\Pi$
% % $\Glfin$ be the $\star$-invariant subalgebra of $\Gfin$ generated by irreducible
% % representations corresponds to $\Lambda_0/W$.


% For any $\lambda\in \hha^{*}$, we define
% \begin{equation}
%   \label{eq:wlam}
%   \begin{split}
%   [\lambda ]  &:= \lambda  +  \aQ,\\
%   R_{[\lambda]} &:= \Set{\alpha\in \aR| \inn{\lambda}{\ckalpha}\in \bZ},\\
%   W_{[\lambda]} &:= \braket{s_\alpha|\alpha\in \Rlam} \subseteq W,\\
%   R_{\lambda} &:= \Set{\alpha\in \aR| \inn{\lambda}{\ckalpha}=0}, \AND\\
%   W_{\lambda} &:= \braket{s_\alpha|\alpha\in R_{\lambda}} = \braket{w\in W|w\cdot \lambda = \lambda} \subseteq W.
%   \end{split}
% \end{equation}

% For any  lattice  $\Lambda = \lambda + \Lambda_0 \in \fhh^*/\Lambda_{0}$ with $\lambda \in \bCon$,
% we define
% \[
%   W_{\Lambda} := \set{w\in W | w\cdot \Lambda  = \Lambda}.
% \]
% Clearly, we have
% \[
%   W_{[\lambda]} < W_{\Lambda}, \quad \forall \ \lambda \in \Lambda.
% \]



% \begin{defn}
% Suppose $\cM$ is an abelian group with $\Glfin$-action
% \[
%   \Glfin\times \cM \ni(F,m)\mapsto F\otimes m.
% \]
% In addition,  we fix a subgroup $\cM_{\barmu}$ of $\cM$ for each
%  for each $W_{\Lambda}$-orbit $\barmu = W_{\Lambda} \cdot \mu\in \Lambda/W_{\Lambda}$.

% A function $f\colon \Lambda \rightarrow \cM$ is called
%   a coherent family based on $\Lambda$ if it satisfies
%   $f(\mu)\in \cM_\mu$ and
%   \[
%   F\otimes f(\mu)  = \sum_{\nu \in \WT{F}} f(\mu+\nu) \qquad \forall \mu\in \Lambda, F\in \PiGlfin.
%   \]
% Let $\Cohlm$ be the abelian group of all coherent families based on $\Lambda$ and value in $\cM$.
% \end{defn}

% \def\Grt{\cG}

% In this paper, we will consider the following cases.

% \begin{eg}
% Suppose $\cM = \bQ$ and $F\otimes m = \dim(F)\cdot m$ for $F\in \PiGlfin$ and $m\in \cM$.
% We let $\cM_{\barmu} = \cM$ for every $\mu\in \Lambda$.
% When $\Lambda = \Lambda_0$, the set of $W$-harmonic polynomials on $\hha^*$ is naturally
% identified with $\Cohlm$ via restriction (Vogan's result)

% \end{eg}

% \begin{eg}
% Let $\Grt(\fgg,K)$ be the Grothendieck group of finite length $(\fgg,K)$-modules
% and $\Grt_{\chi}(\fgg,K)$ be the subgroup of $\Grt(\fgg,K)$ generated by the
% set of irreducible $(\fgg,K)$-modules with infinitesimal character $\chi$.

% Then $\Coh_\Lambda(\cG(\fgg,K))$ is the group of coherent families of Harish-Chandra modules.
% The space $\Coh_\Lambda(\cG(\fgg,K))$ is equipped with a $\WLam$-action by
%   \[
%     w\cdot f(\mu) =  f(w^{-1} \mu) \qquad \forall \mu\in \Lambda, w\in \WLam,
%     f \in \Coh_{\Lambda}(\Grt(\fgg,K)).
%   \]
% \end{eg}

% \begin{eg}
%   Fix a $\Gc$-invariant closed subset $\cZ$ in the nilpotent cone of $\fgg$. Let
%   $\Grt_{\cZ}(\fgg,K)$ be the Grothendieck group of $(\fgg,K)$-modules
%   whose complex associated varieties are contained in $\cZ$.
%   We define
%   \[
%     \Grt_{\chi,\cZ}(\fgg,K) := \Grt_{\chi}(\fgg,K)\cap \Grt_{\cZ}(\fgg,k).
%   \]

%   Now $\Coh_{\Lambda}(\Grt_{\cZ}(\fgg,K))$ is also a $\WLam$-submodule of
%   $\Coh_\Lambda(\cG(\fgg,K))$.
% \end{eg}

% \begin{eg}
%   Fixing a Borel subalgebra $\fbb = \fhh\oplus \fnn \subset \fgg$, let
%   $\cG(\fgg,\fhh,\fnn)$ be the Grothendieck group of the category $\cO$.
%   The space $\cG_{\cZ}(\fgg,\fhh,\fnn)$ is defined similarly.
%   The
%   space $\Coh_\Lambda(\cG(\fgg,\fhh,\fnn))$ and $\Coh_{\Lambda(\cG)}$defined
%   similarly.

% %Note that the lattice $\Lambda$ is stable under the $\Wlam$ action.
% We can define $W_{\Lambda}$ action on $\Coh_\Lambda(\cM)$ by
% \[
%    w\cdot f(\mu) =  f(w^{-1} \mu) \qquad \forall \mu\in \Lambda, w\in \WLam.
% \]
% \end{eg}

% \begin{eg}
% For each infinitesimal character $\chi$ and
% a close $G$-invariant set $\cZ\in \cN_{\fgg}$.
% Let $\Grt_{\chi,\cZ}(\fgg,K)$ be the Grothendieck group of $(\fgg,K)$-module
% with infinitesimal character $\chi$ and complex associated variety contained
% $\cZ$.
% Similarly, let $\Grt_{\chi,\cZ}()$
% \end{eg}


% \def\Parm{\mathrm{Parm}}
% \def\cof{\Theta}
% \subsection*{Translation principal assumption}
% Recall that we have fixed a set of simple roots of $W_\Lambda$.

% We make the following assumption for $\Coh_\Lambda(\cM)$.
% \begin{itemize}
% \item There is a basis $\set{\cof_\gamma|\gamma\in \Parm}$ of
% $\Coh_\Lambda(\cM)$ where $\Parm$ is a parameter set;
% \item For every $\mu\in \Lambda$, the evaluation at $\mu$ is surjective;
%   \[
%     \ev{\mu}\colon \Coh_\Lambda(\cM)\rightarrow \cM_{\barmu} \qquad f\mapsto f(\mu)
%   \]
%   is surjective;
% \item a subset $\tau(\gamma)$ of the simple roots of $W_\Lambda$ is attached
% to each $\gamma\in \Parm$ such that $s\cdot \cof_\gamma = - \cof_\gamma$;
% \item $\cof_\gamma(\mu) =0$ if and only if $\tau(\gamma)\cap R_\mu \neq \emptyset$.
% \item $\set{\cof_\gamma(\mu)| \tau(\gamma)\cap R_\mu = \emptyset}$ form a basis of
% $\cM_{\bargamma}$.
% \end{itemize}

% The translation principle assumption implies
% \begin{equation}\label{eq:kevmu}
%   \Ker\ev{\mu} = \sspan\set{{\cof}_{\gamma}|\tau(\gamma)\cap R_\mu \neq \emptyset } %= \Ker \ev{\mu}.
% \end{equation}


% \def\cohm{\Coh_\Lambda(\cM)}
% Now we have the following counting lemma.
% \begin{lem}
%  For each $\mu$, we have
%  \[
%     \dim \cM_{\barmu}  = \dim (\cohm)_{W_\mu} = [\cohm, 1_{W_\mu}].
%  \]
% \end{lem}
% \begin{proof}
%   Clearly
%   \[
%   \sspan\set{\cof_\gamma - w\cof_\gamma | \gamma\in \Parm, w\in W_\Lambda} \subseteq \ker \ev{\mu}
%   \]
%   since $w\cdot \cof_\gamma(\mu) = \cof_\gamma(w^{-1}\cdot \mu)=\cof_\gamma(\mu)$ for $w\in W_\mu$.
%   Combine this with \eqref{eq:kevmu}, we conclude that
%   \[
%    \sspan\set{\cof_\gamma - w\cof_\gamma | \gamma\in \Parm, w\in W_\Lambda} = \ker \ev{\mu}.
%   \]
%   Therefore, $\ev{\mu}$ induces an isomorphism $(\cohm)_{W_\mu}\rightarrow \cM_{\barmu}$.
%   Now the dimension equality follows.
% \end{proof}

% \begin{eg}
%   For the case of category $\cO$. We can take $\Parm  = W$.
%   Let $\cof_w(\mu) = L(w\mu)$ for each $w\in W_\Lambda$ with
%   $\tau(w) = \set{s_\alpha |\alpha\in \Delta^+, w\alpha \not\in R^+ }$.
% \end{eg}

% \begin{eg}
%  In the Harish-Chandra module case,
%  $\Parm$ consists of certain irreducible $K$-equivariant local systems on a $K$-orbit of the flag variety of $G$.
%  $\tau(\gamma)$ is the $\tau$-invariants of the parameter $\gamma$.
% \end{eg}
%   % Clearly the evaluation map factor through
%   % \[
%   % (\cohm)_{W_\mu}  = \cohm / \braket{\cof_\gamma - w\cof_\gamma | \gamma\in \Parm, w\in W_\Lambda}.
%   % \]
%   %



% \subsection{Primitive ideals and left cells}
% In this section, we recall some results about the primitive ideals and cells developed by Barbasch Vogan, Joseph and Lusztig etc.

% \subsection{Highest weight module}
% Let $\fgg$ be a reductive Lie algebra, $\fbb = \fhh \oplus \fnn$ is a fixed Borel.
% %Fix $\lambda \in \fhh^*$ dominant i.e. $\inn{\lambda}{\ckalpha} \geq 0 $
% Let
% \[
%   M(\lambda)  := \cU(\fgg)\otimes_{\cU(\fbb)} \bC_{\lambda-\rho}
% \]
% and $L(\lambda) $ be the unique irreducible quotient of $M(\lambda)$.

% In the rest of this section, we fix a lattice $\Lambda  \in   \fhh^*/ X^*$.
% Let $\RLam$ and $\RLamp$ be the set of integral root system and the set of positive integral roots.
% Write $\WLam$ for the integral Weyl group.


% For $\mu\in \Lambda$, % is called if
% \[
%   \begin{array}{ccccc}
%   \mu \succeq 0&\Leftrightarrow &\mu \text{ is dominant} &\Leftrightarrow&  \inn{\lambda}{\ckalpha}\geq 0 \quad \forall \alpha \in \RLamp \\
%   \mu \not\sim 0&\Leftrightarrow& \mu \text{ is regular} &\Leftrightarrow&   \inn{\lambda}{\ckalpha}\neq 0 \quad \forall \alpha \in \RLamp
%   \end{array}
% \] %for all $\alpha \in R^+_{\Lambda}$and is called
% regular if $\inn{\lambda}{\ckalpha}\neq 0$.

% For each $w\in W$, it give a coherent family such that
% \[
% M_w(\mu) = M(w\mu) \quad \forall \mu \in \Lambda %\text{ dominant}
% \]
% Let $L_w$ be the unique coherent family such that $L_w(\mu) = L(w\mu)$ for any
% regular dominant $\mu$ in $\Lambda$.
% \trivial{Note that the Grothendieck group
%   $\cK(\cO)$ of category $\cO$ is naturally embedded in the space of formal
%   character. $M_w$ is a coherent family: the formal character $\ch M_w (\mu)$ of
%   $M_w(\mu)$ is
%   \[
%     \ch M_{w}(\mu)=\frac{e^{w\mu-\rho}}{\prod_{\alpha\in R^+} (1-e^{-\alpha})}
%     =\frac{e^{w\mu}}{\prod_{\alpha\in R^+} (e^{\alpha/2}-e^{-\alpha/2})}
%   \]
%   It is clear
%   that $\ch M_w$ satisfies the condition for coherent continuation.

%   From now on, we fix a regular dominant weight $\lambda \in \fhh^{*}$.
%   Then $w[\lambda] = [w\cdot \lambda]$ for any $w\in W$.

%  Now $W_{w[\lambda]} = w\, W_{[\lambda]}\, w^{-1}$.

%  As $W_{[\lambda]}$-module, we have the following decomposition
%  \[
%    \begin{split}
%      \Coh_{[\lambda]} &= \bigoplus_{r\in W/W_{[\lambda]}}
%      \Coh_{r}\quad \text{with}\\
%      \Coh_{r} &=\Coh_{r\lambda} := \set{\cof\in \Coh_{[\lambda]}| \cof(\lambda)\in \cO'_{[r\cdot \lambda]}}
%    \end{split}
%  \]

%  Here $\cO'_{S}$ is the set of highest weight module whose $\fhh$-weights are in
%  $S\subset \fhh^{*}$.

%  % Here
%  % \[
%  %   \begin{split}
%  %     \Coh_{r}
%  %    % &:= \sspan_{\bZ}\set{M_{w}| w \in r\, W_{[\lambda]}}\\
%  %     &= \set{\cof\in \Coh_{[\lambda]}| \cof(\lambda)\in \cO'_{[r\cdot \lambda]}}
%  %   \end{split}
%  % \]

%  Note that the following map
%  \[
%    \begin{array}{ccccc}
%      \bC[W] & \longrightarrow & \Coh_{[\lambda]} &\longrightarrow & \bC[S]\\
%      w & \mapsto & (\mu\mapsto M_{w}(\mu)) & \mapsto & w\cdot \lambda
%    \end{array}
%  \]
%  is $W_{[\lambda]}$-equivariant
%  where $W_{[\lambda]}$ acts by right translation on $\bC[W]$ and $S = W\cdot \lambda$.
%  The action of $W_{[\lambda]}$ on $S$ is by transport of structure and so
%  $(a, w\,\lambda)\mapsto wa^{-1}\,  \lambda$.

%  Now $\bC[r W_{[\lambda]}] \subset \bC[W]$ and
%  $\bC[r W_{[\lambda]}\cdot \lambda]$
%  are identified with $\Coh_{r}$.


% }



% % Fix $\lambda \in \fhh^{*}$ and regular dominant.

% % As $W_{\lambda}$ module, $\Coh_{[\lambda]}$ can be


% For each $w\in \WLam$, the function
% \[
%   \wtpp_w(\mu) := \rank (\cU(\fgg)/\Ann (L(w \mu))) \qquad \forall \mu \in \Lambda \text{ dominant}.
% \]
% extends to a Harmonic polynomial on $\fhh^*$.
% In particular, $\wtpp_w\in P(\fhh^*) = S(\fhh)$.
% \trivial[]{
%   Let $\Lambda^+$ be the set of dominant weights in $\Lambda$.
% Assume $\lambda \succ 0$, i.e. dominant regular. Then
% $\R^0_\lambda = \empty$, $w^{w\lambda} = w^{-1}$. The set
% \[
%  \hat F_{w\lambda} = \set{\mu \in \Lambda |w^{-1}\mu \succeq 0 }
% =  w \Lambda^+.
% \]
% Joseph's $p_w(\mu) = \rank \cU(\fgg)/(\Ann L(w\mu))$ is defined on $\hat F_{w\lambda} = w\Lambda^+$.
% Now his $\wtpp_w(\mu) = w^{-1} p_w$ is defined on $\Lambda^+$ and given by
% $\wtpp_w(\mu) = \rank \cU(\fgg)/(\Ann L(w\mu))$.
% }
% \def\PIP#1{\cP^{#1}}

% There is a unique function $\aLam \colon \WLam \times \WLam\rightarrow \bQ$ such that
% \[
% L_w  = \sum_{w'\in \WLam} \aLam (w,w') M_{w'}
% \]
% \trivial[h]{
% In fact, we have, for any $\mu \succ 0$,
% \[
% L(w\mu) = \sum_{} \aLam(w,w') M(w'\mu)  %\qquad \forall \mu \succ 0.
% \]
% The coefficient is independent of $\mu$. } Suppose $V$ is a module in $\cO$ whose
% Gelfand-Kirillov dimension $\leq d$ Let $c(V)$ be the Bernstein degree. Let
% $\Cint{\Lambda}^d$ be the submodule of $\Cint{\Lambda}$ generated by $L_w$ such
% that the $\dim L(w\mu)\leq d$. Then the map
% \[
% \PIP{d}: \CLam^d\rightarrow S(\fhh) \qquad \Theta \mapsto (\mu \mapsto c(\Theta(\mu)) )
% \]
% is $\WLam$-equivariant. Let $c_w := c\circ (L_w(\mu))\in S(\fhh)$.

% The following result of Joseph implies that the set of primitive ideals can be
% parameterized by Goldie rank polynomials.
% \begin{thm}[Joseph] %, Barbasch-Vogan]
%   \label{thm:GR}
%   {\color{red} correct formulation?}
%   For $\mu$ dominant, the primitive ideal $\Ann L(w\mu) = \Ann L(w'\mu)$ if and only if $p_w = p_w'$. (\cite[ref?]{J1})
% %   Let $r = \dim \fnn$.
% %   The following holds:
% %  \begin{enumT}
% %   \item \label{it:j.1}
% %   For $\mu$ dominant, the primitive ideal $\Ann L(w\mu) = \Ann L(w'\mu)$ if and only if $p_w = p_w'$. (\cite[ref?]{J12})
% %  \end{enumT}
% \end{thm}

% \begin{thm}
%  Let $r = \dim \fnn$.  Suppose $\dim L(w\nu) = d$.
%  \begin{enumT}
%   \item Let $x\in \fhh$ such that $\alpha(x) = 1$ for every simple roots $\alpha$.
%   Then there are non-zero rational numbers $a,a'$ such that  \cite{J12}*{Theorem~5.1 and 5.7}
%   \[
%    a \, \wtpp_w(\mu)  = a' \, c_w(\mu) = \sum_{w'\in \WLam} \aLam (w,w') \inn{\mu}{w'^{-1} x}^{r-d}.
%   \]
%   \item The submodule $\bC \WLam \wtpp_w$ in $S(\fhh)$ is
%   a special  irreducible $\WLam$-representation.
%   Moreover all special representations of $\WLam$ occurs as a certain
%   $\sigma(w)$.
%   See \cite{BV1}*{Theorem~D}, \cite{BV2}*{Theorem~1.1}.
%   \item Let $\sigma(w)$ be the submodule of $S(\fhh)$ generated by
%   $\wtpp_w$. Then
%   $\sigma(w)$ is irreducible and it
%   occurs in $S^{r-d}(\fhh)$ with multiplicity $1$, see \cite{J.hw}*{5.3},
%   % occurs in
%   % $\wtpp_{w^{-1}}$ also generate $\sigma(w)$ and the module
%  \end{enumT}
% \end{thm}
% By \Cref{thm:GR}, the module $\sigma(w)$ only depends on the primitive ideal $\cJ :=\Ann L(w\mu)$. So we can also write
% $\sigma(\cJ):= \sigma(w)$.

% \trivial[h]{
% A representation $\sigma \in \widehat{W}$ is called univalent, if $\sigma$ occur in
% $S^{n(\sigma)}(\fhh)$ with multiplicity one. Here $n(\sigma)$ is the minimal degree such that $\sigma$ occur in $S(\fhh)$.
% }

% From the above theorem, we see that the map $\PIP{d}$ factor through
% $\CLam^{d}/\CLam^{d-1}$ and its image is contained in the space $\cH^{d}$ of degree $d$-Harmonic
% polynomials.

% Note that the associated variety of a primitive ideal $\cJ$ is the closure of
% a nilpotent orbit in $\fgg$.
% %$\Ann L_w (\mu)$ for $\mu\succ 0$
% Now the associated variety of $\cJ$  is computed by the following result of Joseph (and include Hotta ?).
% \begin{thm}
%   %Suppose $J = \Ann L(w\mu)$
%   Let $\cJ$ be a primitive ideal in $\cU(\fgg)$.
%   Then $\sigma(\cJ)$ is in the image of Springer correspondence.
%   Moreover, the associated variety of $\cJ$ is the orbit $\cO$
%   corresponding to $\sigma(\cJ)$.
% See \cite{J.av}*{Proposition~2.10}.
% \end{thm}


% \begin{lem}
%   Let $\mu\in \Lambda$, $W_{\mu} = \set{w\in W_{\Lambda}| w\cdot \mu = \mu}$ and
%   \[
%   a_{\mu} = \max\set{a(\sigma)|  \widehat{\WLam} \ni\sigma \text{ occurs in } \Ind_{W_{\mu}}^{\WLam} 1}.
%   \]
%   Suppose $W_{\mu}$ is a parabolic subgroup of $\WLam$.
%   % Then there is a unique special representation $\sigma_{\mu}$ such that
%   % $a(\sigma_{\mu}) = a_{\mu}$.
%   Then the set of all irreducible representations $\sigma$ occurs in
%   $\Ind_{W_{\mu}}^{\WLam} 1$ such that $a(\sigma) = a_{\mu}$ forms a left cell
%   given by
%   \[
%   \left(J_{W_{\mu}}^{\WLam} \sgn\right) \otimes \sgn.
%   \]
% \end{lem}


% Now the maximal primitive ideal having infinitesimal character $\mu$
% has the associated variety $\cO$.

% \def\Wb{W_{b}}
% \def\Wg{W_{g}}
% \def\WcOb{W_{\ckcO_b}}
% \def\WcOg{W_{\ckcO_g}}

% Recall that $\WLam = \Wb\times \Wg$.
% Let
% \[
% \begin{split}
%   \sigma_{b} &= (j_{\WcOb}^{\Wb}\sgn)\otimes \sgn\\
%   \sigma_{g} &= (j_{\WcOg}^{\Wg}\sgn)\otimes \sgn\\
% \end{split}
% \]
% Then $\sigma_{b}\otimes \sigma_{g}$ is a special representation of $\WLam$.
% Moreover $\cO$ corresponds to the representation
% \[
%  \sigma := j_{\Wb\times \Wg}^{W} \sigma_{b}\otimes \sigma_{g}
% \]

% % Then $W_{[\lambda_{\ckcO}]} = W'_n$ and the left cell is given by
% % \[
% % (J_{W_\ckcO}^{W'_n} \sgn) \otimes \sgn \quad \text{ with } \quad W_\ckcO = \prod_{i\in \bN^+} S_{\bfcc_{2i}(\ckcO)}.
% % \]
% % Here $W_\ckcO$ is an subgroup of $S_n$ and the embedding of $S_n$ in $W'_n$ is fixed.
% % When $n$ is even,we the symbol of $J_{S_n}^{W'_n} \sgn$ is degenerate and we label it by ``$I$''.

% \trivial{
%   We recall some facts about the $j$-induction and $J$-induction.
%   If $W'$ is a parabolic subgroup in $W$ then $j_{W'}^{W}$ maps special
%   representation to special representation.
%   It maps irreducible representation to irreducible ones.
%   The $j$ induction has induction by stage \cite{Carter}*{11.2.4}.

%   For classical group, the representations satisfies property B.
%   When $W'$ is parabolic, the orbit $\cO$ corresponds to $j_{W'}^{W}$ is the orbit
%   $\Ind_{L}^{G} \cO$.

%   The $J$-induction maps left cell to left cell, which is only defined for
%   parabolic subgroups.


%   The $j$-induction for classical group is in \cite{Carter}*{Section 11.4}.
%   Basically, $W_{n}$ has $D_{c_{i}}$, $B/C_{c_{i}}$ factors,
%   \[
%     j_{\prod_{i}D_{c_{i}}\times \prod_{j} B/C_{c_{j}}} \sgn  =
%       ((c_{i}), (c_{j})).
%   \]
%   Here $(c_{i})$ denote the Young diagram with columns $c_{i}$ etc.

%   Moreover, we have the following formula for $j$-induction.
%   \[
%     \ind_{A_{2m}}^{W_{2m}} = ((m),(m)) \qquad
%     \ind_{A_{2m+1}}^{W_{2m+1}} = ((m),(m+1)).
%   \]
% }

\end{document}
