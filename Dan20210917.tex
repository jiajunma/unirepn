
\documentclass[11pt ,reqno]{amsart}
%\TagsOnRight
%%%%Definitions
\usepackage{amssymb,amscd,verbatim, amsthm,tikz-cd,todonotes}
%\usepackage[active]{srcltx}
\usepackage{xcolor}
%\usepackage[small,nohug,heads=vee]{diagram}
%\diagramstyle[labelstyle=\scriptstyle]
\synctex=1
\begin{document}

\today\hfill



\bigskip

\title {The Associated Variety of a Unipotent Representation}
%\author{Dan Barbasch}
%       \address[D. Barbasch]{Dept. of Mathematics\\
%               Cornell University\\Ithaca, NY 14850}
%        \email{barbasch@math.cornell.edu}

\maketitle


\newfont{\sans}{cmss10}
\newfont{\sansm}{cmss8}
\newfont{\biig}{cmbx12}

\newcommand\clrr{\color{red}}
\newcommand\clrblu{\color{blue}}

\newcommand\lr{\overset{LR}{<}}
\newcommand\elr{\overset{LR}{\approx}}

\newcommand\vh{{\vspace{0.3in}}}
\newcommand\ovl{\overline}
\newcommand\uset{\underset}
\newcommand\ub{\underbrace}
\newcommand\unl{\underline}


\newcommand \ov[1]{{\overline{#1}}}
\newcommand \fk[1]{{\mathfrak{#1}}}
\newcommand\unb[2]{{\underbrace{#2}_{#1}}}
\newcommand\und[1]{{\underline{#1}}}
\newcommand\ul[1]{{\underline{#1}}}
\newcommand\ma[2]{{\begin{matrix} #1\\#2\end{matrix}}}
\newcommand\maf[4]{{\begin{matrix} #1\\#2\\#3\\#4\end{matrix}}}
\newcommand \C[1]{{\mathcal #1}}
\newcommand \ch[1]{{\check{#1}}}
\newcommand \bb[1]{{\mathbb #1}}
\newcommand \wti[1]{{\widetilde #1}}

\newcommand\mat{\begin{matrix}}
\newcommand\emat{\end{matrix}}
\newcommand\pmat{\begin{pmatrix}}
\newcommand\epmat{\end{pmatrix}}

\newcommand\msp[1]{&\multispan#1\hrulefill}
\newcommand\tableau[1]{
 \vbox{\offinterlineskip \hrule
 \halign{&\vrule##&\strut\ \hfil##\hfil\ &\vrule##&\strut\ \hfil##\hfil\ \cr #1\cr}}}

\newcommand\ph[1]{\phantom{#1}}


\newcommand\AV{{\mathcal V}}
\newcommand\Ac{{\mathcal V_c}}
\newcommand \CC{\mathcal C}
\newcommand \CF{\mathcal F}
\newcommand \CK{\mathcal K}
\newcommand \CN{\mathcal N}
\newcommand \CP{\mathcal P}
\newcommand \CQ{\mathcal Q}
\newcommand \CR{\mathcal R}

\newcommand \CO{\mathcal O}
\newcommand \COc{\mathcal O_c}
\newcommand\vO{{\check \CO}}
\newcommand\vOc{{\check \CO_c}}

\newcommand\chp{{\check\pi}}
\newcommand\chg{\check{\g}}
\newcommand\cchi{\check{\chi}}
\newcommand\eset{{\emptyset}}

\newcommand\g{{\mathfrak g}}
\newcommand\kf{{\mathfrak k}}
\newcommand\lf{{\mathfrak l}}
\newcommand\q{{\mathfrak q}}
\newcommand\s{{\mathfrak s}}
\newcommand\uf{{\mathfrak u}}
\newcommand\ie{\textit{i.e. }~}
\newcommand\cf{\textit{c.f. }~}

\newcommand \CSGs{\textit{Cartan subgroups}~}
\newcommand \CSG{\textit{Cartan subgroup}~}

\newcommand\al{{\alpha}}
\newcommand\la{{\lambda}}
\newcommand\Sig{{\Sigma}}
\newcommand\sig{{\sigma}}
\newcommand\ep{{\epsilon}}


\newcommand{\ba}{{\mathbf a}}
\newcommand{\bg}{{\mathbf g}}
\newcommand{\bk}{{\mathbf k}}
\newcommand{\bl}{{\mathbf l}}
\newcommand{\bm}{{\mathbf m}}
\newcommand{\bp}{{\mathbf p}}

\newcommand\vg{{\check{\fk g}}}
\newcommand\vl{{\check{\fk l}}}
\newcommand\vm{{\check{\fk m}}}
\newcommand\vq{{\check{\fk q}}}
\newcommand\vu{{\check{\fk u}}}




\newcommand{\bA}{{\mathbb A}}
\newcommand{\bC}{{\mathbb C}}
\newcommand{\bF}{{\mathbb F}}
\newcommand{\bN}{{\mathbb N}}
\newcommand{\bQ}{{\mathbb Q}}
\newcommand{\bR}{{\mathbb R}}
\newcommand{\bZ}{{\mathbb Z}}
\newcommand{\Hc}{H^{*}(}
%\newcommand{\CP}{\C P}
\newcommand{\hG}{\hat{G}}

\newcommand\kbar{\ovl{\bk}}
\newcommand\Ad{{\operatorname{Ad}}}
\newcommand\Hom{{\operatorname{Hom}}}
\newcommand\ad{{\operatorname{ad}}}
\newcommand\Ind{\operatorname{Ind}}
\newcommand\im{{\operatorname{im}}}
\newcommand\diag{{\operatorname{diag}}}
\newcommand\tr{{\operatorname{tr}}}
\newcommand\Ext{{\operatorname{Ext}}}
\newcommand\vol{{\operatorname{vol}}}
\newcommand\Ho{{\operatorname{H}}}


\newtheorem{theorem}{Theorem}
\newtheorem{conjecture}{Conjecture}
\newtheorem{corollary}{Corollary}
\newtheorem*{corollary*}{Corollary}
\newtheorem{definition}{Definition}
\newtheorem{example}{Example}
\newtheorem*{example*}{Example}
\newtheorem{lemma}{Lemma}
\newtheorem{proposition}{Proposition}
\newtheorem*{proposition*}{Proposition}
\newtheorem{remark}{Remark}
\newtheorem*{remark*}{Remark}


\numberwithin{equation}{subsection}


\section{Assumptions and Notation}\label{1}

\bigskip
The groups under consideration, are the real points of a classical
group, $So(a,b)$ or $Sp(2n,\bR)$, and $Mp(2n,\bR)$. The
complexification of the Cartan
decomposition is denoted $\g=\kf + \s.$

\medskip 
\subsection{Weyl groups}\label{1.1} We will denote the Weyl group of type $B_a$
by $W_a.$  The Weyl 
group of type $D$ will be written as $W'_{a}.$ The parametrization of
representations is as in [L1]. Thus the representations of $W_a$ are
parametrized by pairs of partitions $\tau=\tau_L\times\tau_R$, and
similarly for $W'_a.$ For $W',$ if $\tau_1\ne \tau_2,$ then
$\tau_1\times\tau_2$ and $\tau_2\times\tau_1$ parametrize the same
representation; we choose the one where the harmonic degree of the
realization in [L1] matches the lowest degree occurence of $\tau$ in
the space of harmonic polynomials on the \CSG. If $\tau_1=\tau_2,$
then there are two inequivalent representations $(\tau_I$ and
$\tau_{II}).$

We will use the {\it obvious} generalization of the Littlewood-Richardson
rule to $W_a.$ We only need it when one of the representations is trivial or sign.

\medskip
\subsection{Langlands parameters}\label{1.2} The infinitesimal character is a
translate of $\rho$ by the root lattice.  The representations are all {\it special
unipotent}. We will use the usual realizations of the \CSGs and their root systems
for the classical groups. A Langlands parameter is then a sequence of
numbers, integers in type B,\ D and half--integers in type B. The real
part of the parameter is denoted by underlining with superscript $\pm$
to denote the character on the compact part of the centralizer. The
imaginary part of the parameter may also have superscripts of $\pm$ in
types B,\ C. For  $Mp,$ the infinitesimal character will be formed
of half integers only.

\medskip
We will use [V] freely.
\subsection{Nilpotent Orbits}\label{1.3} The nilpotent orbits are parametrized by
partitions/tableaus with alternating signs on each row. Rows of the
same size are interchangeable. For types B,\ D, even rows
occur an even number of times; for type C, rows of odd size occur an
even number of times. They have to occur in pairs one
starting with $-$ one starting with $+.$ The same rules apply for type
C and odd sized rows.

\medskip
Let $\lf=\lf_0 + \lf_1$ be a Levi component of a maximal parabolic
subgroup $\q=\lf +\uf$ which is stable under conjugation. Here $\lf_0$
refers to a subalgebra of the same type as $\g$, and $\lf_1$ is type A. Let $\CO$ be a
nilpotent orbit in $\lf_0\cap \s.$ We are interested in the nilpotent
orbit which is induced from $\CO\times triv$ on $\lf_0\times\lf_1.$ If $\q$ is
$\theta$--stable, then $\lf_1=u(p,q).$ In types B,\ D, the induced
nilpotent is obtained by adding $p$ pluses and $q$ minuses one to the
beginning (right) of  each row and the same to the end (left), as high as possible so
that the result is again a signed tableau. If $\q$ is real, then
$\lf_1=gl(m).$ In this case, add 2 to each row (as high as possible) to
form a signed tableau. In this case we get a union of nilpotent orbits.

In type C the same rules apply, but for $\theta$-stable parabolics,
add $q$ pluses and $p$ minuses to the end of the rows.

\medskip
Given a special orbit $\CO,$ we denote by $\vO$ the corresponding dual orbit.


\medskip
The parameter $\la_{\C O}$, which is the infinitesimal character, is listed case by case. 

\bigskip
\noindent\textbf{Type B.\ } The column sizes are $C_{2a}\ge
C_{2a-1}\ge \dots \ge C_1\ge C_0$ satisfying $C_{2i+1}\equiv C_{2i}\
(2)$, and $C_{2a}\equiv 1\ (2).$ 
Special orbits satisfy $C_{2i+1}=C_{2i}$ whenever the columns are even. We
will reduce to the case when  $C_{2i-1}=C_{2i-2}$; this is equivalent to
the fact that $\C O$ has no even sized rows. {The complex nilpotent
orbit is induced from a Levi component $L=SO(2n+1-C_{2i-1})\times
GL(C_{2i-1})$ in this case}. Write $C_{2a}=2c_{2a}+1$,
$C_{2i+1}=2c_{2i+1}-\ep_{i}$ and  $C_{2i}=2c_{2i}+\ep_{i},$ with
$\ep_i=0,1$. The (special) Weyl
group representation associated to a special nilpotent orbit $\C O$
has columns
\begin{equation}
  \label{eq:tableaub}
\tau_L\times\tau_R=(c_{2a-1},\dots ,c_1)\times (c_{2a},c_{2a-2},\dots ,c_0).  
\end{equation}
The other representations in the primitive ideal cell are given by
interchanging 
$$
(c_{2i+1}, c_{2i})\longleftrightarrow
(c_{2i},c_{2i+1}).
$$  
This corresponds to the Springer representation
(associated to the trivial local system) for another nilpotent orbit.
The other representations in the primitive ideal double cell do not
occur at all in the coherent continuation representation. This is
clear from the computation in the next section and also from [McG].  

The infinitesimal character $\la_{\C O}$ is obtained by concatenating
$$
c_{i}\longleftrightarrow (c_{i}+1/2,\dots ,1/2).
$$
\begin{remark*}
When all but the largest odd row occurs an even number of times, there
is a 1-1 correspondence between local systems on real orbits $\CO$ and
unipotent representations. 
  
Adding a column of size greater than or equal to $C_{2a}$ on the left
with the obvious choice of signs gives a
nilpotent orbit in type C, and implements the moment map corresponding
to the $\Theta-$correspondence $Mp()\times O(odd).$   
\end{remark*}
\noindent\textbf{Type C.\ } Label the columns $C_{2a}\ge C_{2a-1}\ge
\dots \ge C_1\ge C_0\ge 0,$ satisfying $C_{2i}\equiv C_{2i-1}\ (2)$
and $C_0\equiv 0\ (2).$   Special orbits satisfy $C_{2i}=C_{2i-1}$ whenever
they are odd.  
We will reduce to the case when all $C_{2i}=C_{2i-1}$; this
is equivalent to the fact that $\C O$ has no odd sized rows. 
{The complex nilpotent orbit is induced from a
  Levi component 
  $L=Sp(2n-C_{2i})\times GL(C_{2i})$ in this case}. Write
$C_0=2c_0$, $C_{2i}=2c_{2i}+\ep_i$ and $C_{2i-1}=2c_{2i-1}-\ep_i.$
The (special) Weyl group representation associated to the special $\C O$
has columns
\begin{equation}
  \label{eq:tableauc}
\tau_L\times\tau_R=(c_{2a-1},\dots , c_1)\times (c_{2a},\dots ,c_0).
\end{equation}
The other representations in the primitive ideal cell are given by
interchanging 
\begin{equation}
  \label{eq:wccell}
 (c_{2i}, c_{2i-1})\longleftrightarrow (c_{2i-1}{\clrr -1},c_{2i}{\clrr +1}).  
\end{equation}
These  correspond to the Springer representation associated to the
trivial local system for other nilpotent orbits.
 
 The other representations in the primitive ideal double cell do not
 occur at all in the coherent continuation representation. This is
 clear from the computation in the next section and also from [McG].  

The infinitesimal character $\la_{\C O}$ is a concatenation of coordinates,
\begin{equation}
\begin{aligned}
    \label{eq:inflc}
&C_{2i}&&\longleftrightarrow &&(c_{2i},\dots ,1)\\
&C_{2i-1}&&\longleftrightarrow &&(c_{2i-1}-1,\dots ,0).
\end{aligned}
\end{equation}

\begin{remark*}
When $\CO_c$ has every even size row occuring an even number of times,
there is a 1-1 correspondence between local systems on real   orbits
$\CO$ and unipotent representations.

Adding a column of size greater than or equal to $C_{2a}$ gives a
nilpotent orbit of type $D$, and implements the
$\Theta-$correspondence. 
\end{remark*}
\noindent{

\noindent\textbf{Type $\mathbf{\wti C}$.\ } We consider the case when
the infinitesimal character $\chi$ is formed of half-integers
only. The integral system is type D.
The dual group is $G=Mp(2n,\bR)$ and the nilpotent orbits $\vO$  are
those with even rows only. 
The computations for the coherent continuation representation
are in \cite{RT}. At the corresponding regular infinitesimal character, there are two
copies of $D_n$, which combine together to give a representation of
the Weyl group of type $C.$    

Label the rows of $\vO$ paired as 
$(R_{2m+1}=2r_{2m+1}\ge R_{2m}=2r_{2m})\ge  \dots \ge (R_1=2r_1\ge
R_0=2r_0).$  The columns occur in pairs, and are labelled
$C_1,C_1\ge\dots \ge C_r,C_r$. The infinitesimal character associated
to $\vO$ is  
$$
\chi=\chi(\vO)=(\underset{C_1}{\underbrace{1/2,\dots ,1/2}},\dots , \underset{C_r}{\underbrace{r-1/2,\dots ,r-1/2}}).
$$
 The associated variety of the corresponding unipotent representations
 is (a union of real forms of) the complex nilpotent orbit with 
$$
\begin{aligned}
&{rows:\ } &(C_i,C_i) &\quad\text{ if }\quad &C_i \text{ is even},\\
           &&(C_{i}+1,C_i-1) &\quad\text{ if }\quad &C_i \text{ is odd},\\
&&\\
&{columns:\ } &(R_{2i+1,}R_{2i}) &\quad\text{ if }\quad &R_{2i+1}=R_{2i},\\
&&(R_{2i+1}-1,R_{2i}+1) &\quad\text{ if }\quad &R_{2i+1}>R_{2i}.  
\end{aligned}
$$
If a size of an odd  column is repeated, say
$C_i=C_i=C_{i+1}=C_{i+1}=2c_i+1$, $\CO$ gets only one pair of rows
$(2c_i+2,2c_i)$, the others become $(2c_i+1,2c_i+1).$


The Weyl group representation corresponding to $\CO$ has pairs of rows 
$c_i\times c_i$ for $c_i$ even, and $c_i+1\times c_i$ for $c_i$ odd.

\begin{example*}
  

A special case is when all row sizes of $\CO$ occur an even number of
times, except for the largest 
one, which is even and occurs an odd number of times. This is
equivalent to  $\vO$ having the property that each row occurs an even
number of times (corresponding to $\vO_I$ and $\vO_{II}$ in type
$D$). The associated variety is a single orbit, and there are exactly two
representations for each nilpotent orbit $\CO,$ corresponding to the
characters of the component group of the centralizer of an element in
$\CO.$ A more general case  is described later.
\end{example*}
\begin{remark*}
The nilpotent orbit $\CO$ listed is in $sp(2n,\bC),$   and is in
general not special. Considerations of primitive ideals take place in
type D.

Adding an odd sized column of length at least  $2c_{2a}+1$ gives a
nilpotent orbit of type B, and implements the $\Theta-$correspondence. 
\end{remark*}

}


\medskip
\noindent\textbf{Type D.\ } Label the columns $C_{2a-1}\ge\dots \ge
C_0\ge 0$ satisfying $C_{2i}\equiv C_{2i-1}\ (2),$ and $C_{2a-1}\equiv
C_0\equiv 0\ (2).$ As before, write
$C_{2i}=2c_{2i}+\ep_i$, $C_{2i-1}=2c_{2i-1}-\ep_i$ and $C_0=2c_0,\
C_{2a-1}=2c_{2a-1}.$  Special nilpotent orbits satisfy
$C_{2i}=C_{2i-1}$ whenever  they are odd. We will reduce to the case
$C_{2i}=C_{2i-1}$ which is the case of $\CO$ having no even
sized rows.
The special Weyl group representation associated to $\C O$ has columns
\begin{equation}
  \label{eq:drep}
\tau_L\times\tau_R=(c_{2a-1},\dots ,c_1)\times (c_{2a-2},\dots
,c_0)\sim
(c_{2a-2},\dots
,c_0)\times (c_{2a-1},\dots ,c_1).  
\end{equation}
The other representations in the primitive ideal cell are obtained by
interchanging pairs coming from even columns, 
$$
(c_{2i},c_{2i-1})\longleftrightarrow (c_{2i-1}+1,c_{2i}-1).
$$
This is again as in  [McG].  
The infinitesimal character $\la_{\C O}$ is as in type C, \eqref{eq:inflc}.
\begin{remark*}
Adding an even sized column of length at least $2c_{2a-1}$ gives a
nilpotent orbit of type C, and implements the $\Theta-$correspondence. 

In the case when all rows of odd size occur an even number of times,
the corresponding unipotent representations can be parametrized by the
local systems on the \textit{real} orbits.

\end{remark*}


  

%%%%%%%%%%%%%%%%%%%%%%%%%%%%%%%%%%%%

\section{The coherent continuation representation}\label{2}
For the linear groups we follow \cite{V4}. For $Mp(2n,\bR),$ we
follow \cite{RT}.

\begin{definition}
Fix a block $\C B$ with infinitesimal character
$\rho.$ The Hecke algebra $\C H(W)$ over $\bZ[u,u^{-1}]$ acts as
follows. Let $T_s$ be the operator corresponding to the simple
reflection $s.$ The action on a basis of standard modules $\{\gamma\}$
in the Grothendieck group $\C M(\C B),$ the action is 
\begin{description}
\item[(a) ] $\al$ compact imaginary ($\ep(\al)=0$)
  $T_s\gamma=u\gamma.$
\item[(a*) ] $\al$ real, no parity condition ($\check{\delta}(\al)=0$)
  $T_s\gamma=-\gamma$.
\item[(b) ] $\al$ complex $\theta\al\in R^+(\gamma)$,
$T_s\gamma=(s\times\gamma)$.
\item[(b*) ] $\al$ complex $\theta\al\notin R^+(\gamma)$,
$T_s\gamma=u(s\times\gamma) +(u-1)\gamma$.
\item[(c) ] $\al$ noncompact imaginary (type I)
  $T_s\gamma=s\times\gamma+ c^\al(\gamma).$
\item[(c*) ] $\al$ real  (type II) parity condition
  $T_s\gamma=(u-1)\gamma-s\times\gamma+ (u-1)c_\al(\gamma).$
\item[(d) ] $\al$ noncompact imaginary ($\ep(\gamma)=1$ type II),
$T_s\gamma=\gamma +\gamma^\al_++\gamma^\al_-.$ 
\item[(d*) ] $\al$ real ($\check\delta(\gamma)=1$ type I). Then
  $T_s\gamma=(u-2)\gamma +(u-1)(\gamma_\al^++\gamma_\al^-).$
\end{description}
The coherent continuation representation is obtained by changing $T_s$
to $-T_s,$ rescaling $\gamma$ by $(-1)^{\ell(\gamma)}$, and setting  $u=1.$ 
\begin{description}
\item[(a) ] $\al$ compact imaginary $\ep(\al)=0$)
  $t(s)\gamma=-\gamma.$
\item[(a*) ] $\al$ real, no parity condition ($\check{\delta}(\al)=0$)
  $t(s)\gamma=\gamma$.
\item[(b) ] $\al$ complex $\theta\al\in R^+(\gamma)$,
$t(s)\gamma=(s\times\gamma)$.
\item[(b*) ] $\al$ complex $\theta\al\notin R^+(\gamma)$,
$t(s)\gamma=s\times\gamma$.
\item[(c) ] $\al$ noncompact imaginary (type I)
  $t(s)\gamma=-s\times\gamma+ c^\al(\gamma).$
\item[(c*) ] $\al$ real  (type II) parity condition
  $t(s)\gamma=s\times\gamma.$
\item[(d) ] $\al$ noncompact imaginary ($\ep(\gamma)=1$ type II),\newline
$t(s)\gamma=-\gamma +\gamma^\al_++\gamma^\al_-.$ 
\item[(d*) ] $\al$ real ($\check\delta(\gamma)=1$ type I). Then
  $t(s)\gamma=\gamma.$
\end{description}
\end{definition}

The duality is obtained by tensoring with $sgn.$ Types $B$ and $C$ are
interchanged, while type $D$ maps to itself. The modification for $Mp$
is in \cite{RT}. 

\subsection{} 
{
Let $\chi$ be a singular infinitesimal character which
differs from a regular one $\chi_{reg}$ by the root lattice. Let
$\C I(\chi)$ be the maximal primitive ideal for the
integral system determined by $\chi,$  $\C
C(\C I(\chi))$ the corresponding cell, and $W(\chi)$ the centralizer in the
Weyl group for the coresponding integral system.

This section is devoted to proving the following multiplicity formula. The argument follows \cite{McG}. 
\begin{theorem}\label{t:mult}
The number of irreducible representations with annihilator $\C
I(\chi)$ in a block $\C B$ is given by the sum of
multiplicities
$$
\sum_{\sig\in\C C(\chi)}[\sig\ :\ \C C(\chi)]\cdot[\sig:\C B].
$$
For special unipotent representations, the cells $\C I(\chi)$ have
 multiplicity \newline $[\sig,\Ind_{W(\chi)}^W triv_\chi]$ with
 $\sigma$ in the double cell corresponding to $\C I(\chi).$ They satisfy
 $\le 1.$  
\end{theorem}
}

\subsubsection{}
Assume the infinitesimal character is regular integral. The following
definitions and properties are from Section 14 of  \cite{V4}.


Fix a block $\C B$ of representations, and denote by $\bZ(\C B)$ the corresponding
Grothendieck group. Then Definition 14.4 in \cite{V4} defines a
representation $t(w)$ of the (complex, abstract) Weyl group $W^a$ on $\bZ(\C B).$
The formulas are before Section 2.1 above.
\begin{definition}[14.6 in \cite{V4}]
 Let  $\overset{LR}{<}$ on $\C B$ be the smallest preorder relation with the following
property.  Fix  w$\in W, \gamma\in \C B$, and write
$$
t(w)\pi(\gamma)=\sum a_\phi\pi(\phi)
$$
Then 
$$
a_\phi\ne 0 \Longrightarrow \gamma \overset{LR}{<}\phi.
$$
The set of parameters $\phi\lr \gamma$ is denoted $\ovl{\C C}^{LR}(\gamma).$
\end{definition}
This amounts to saying that there is a finite dimensional
representation $F$ so that $\pi(\phi)$ occurs as a composition factor
in $\pi(\gamma)\otimes F.$
The associated representation of $W$ in the subspace of $\bZ(\C B)$ is
denoted $\ovl{\C V}^{LR}(\gamma)$.

The relation $\lr$ generates an equivalence relation 
$$
\gamma\elr \phi \Longleftrightarrow \gamma\lr\phi \text{ and } \phi \lr\gamma
$$
The equivalence clasees are called HC-cells, and we write $\C C^{LR}(\gamma)$ for the equivalence class of $\gamma$. If $(\pi_1,V_1)$ and $(\pi_2,V_2)$ are irreducible $(\fk
  g,K)-$modules,  $\pi_1$ and $\pi_2$ belong to the same
  Harish-Chandra cell, if there exist finite dimensional
  representations $F_1, F_2$ such that $\pi_2$ is a composition factor
  of $\pi_1\otimes F_1$ and $\pi_1$ is a composition factor of
  $\pi_2\otimes F_2.$ 

  The corresponding $W-$representations are defined as follows. Let
  $$
  \C C_+^{LR}:=\ovl{\C C}^{LR}(\gamma)\backslash \C C^{LR}(\gamma).
$$
The subspace in $\bZ(\C B)$ denoted $\C V_+(\gamma)$ is generated by the elements
in $\C C_+^{LR}(\gamma)$. This  is a representation of $W,$ and the HC-cell
of $\gamma$ as a representation is  
$$
\C V^{LR}(\gamma):=\ov{\C V}^{LR}(\gamma)/\C V_+^{LR}(\gamma).
$$1

\begin{proposition}[Corollary 14.9, \cite{V4}]
  Let $\C B^\vee$ be the dual block. Then
  \begin{description} 
  \item[(a)] $\gamma\lr \phi \Leftrightarrow \phi^\vee\lr
    \gamma^\vee$. In particular,
    $$
    { \C C^{LR}(\gamma)=\C C^{LR}(\gamma^\vee)}
    $$
  \item[(b)] $\langle t(w)m,n\rangle =(-1)^{\ell(w)}\langle m,t(w)n\rangle$.
\item[(c)] { $\C V^{LR}(\gamma)\cong \C V(\gamma^\vee)^*\otimes (sgn).$}    
  \end{description}  
\end{proposition}

\begin{theorem}[Theorem 14.10, Corollary 14.11
  \cite{V4}]\label{t:goldie} There is a natural map, called character
  polynomial 
  $$Ch:\C V^{LR}(\gamma)\longrightarrow S(\fk h^a)\quad (\fk h\ \text{ the
    Cartan subalgebra) }$$   so that
  \begin{description}
  \item[(a) ] $Ch (\pi(\gamma))$ is a homogeneous $W^a-$harmonic polynomial.
  \item[(b)] $Ch(\pi(\gamma)) $ is up to a nonzero constant Joseph's
    Goldie rank polynomial of  the primitive ideal
    $I(\pi(\gamma))\subset U(\fk g)$, the annihilator of
    $\pi(\gamma).$ 
    \item[(c)] $Ch(t(w)\pi(\gamma))=w\cdot ch(\pi(\gamma)).$ The
      HC-cell of $\pi(\gamma)$ contains the Goldie rank representation
      associated to $I(\pi(\gamma))$. 
  \end{description}
  
\end{theorem}
\subsubsection{} We refine the results in the previous section,
especially Theorem \ref{t:goldie}. Note the difference that \cite{K}
deals with just the most split Cartan subalgebra while \cite{C} deal
with all Cartan subalgebras.

The character of an admissible module $\pi$ (lifted to a neighborhood of $0\subset \fk g$)
$\theta(\pi)$ has a Taylor expansion. For $f\in C_c^\infty (\fk g),$
let $f_t(X):=t^{-\dim \fk g}f(X/t).$ 
Embed the character in a coherent family. 
By theorem 3.4 in \cite{C}, the Taylor expansion in $t$ satisfies 
$$
\theta(\pi(\mu))(f_t)=t^{-d}p(\mu)_dE_{d}+ \text{ higher order terms }.
$$
where $p(\mu)$ is a harmonic polynomial of degree equal to the
Gelfand-Kirillov dimension of $\pi.$ Furthermore, from \cite{BV1},
$AV(\pi)\subset V(I(\pi))$.



Choose a real $\theta-$stable Cartan subalgebra $\fk h$. and a
positive system $\Delta_\bR^+$ of the positive roots.  Let $\fk n$ be
any nilradical of a Borel subalgebra containing $\fk h,$ and such that
$\Delta^+_\bR\subset \Delta(\fk n).$ Let $H^-_{reg}=\{h\in H_{reg}\ :\
|e^\al(h)|<1\}.$  

\begin{theorem}[Theorem 3.1, \cite{C}] Let $M$ be a Harish-Chandra
  module. Then $H^*(\fk n,M)$ is finite dimensional, and
  $$
\theta(M)\mid_{H^-_{reg}}=\frac{\sum (-1)^q\tr_H H^q(\fk
  n,M)}{\prod_{\al\in\Delta(n)}(1-e_{-\al})} 
$$
\end{theorem}
Denote by $Ch(\fk n,M):=\sum (-1)^q H^q(\fk n,M).$  
\begin{proposition}[Proposition 2.11, \cite{C}] There are functors
  $\Gamma_{\fk n}^j$ from admissible modules to category ${\C O}'$ so that
  \begin{description}
  \item[(a)] $I(\pi)\subset I(\Gamma^j(\pi))$.
  \item[(b)] $Ch(\fk n,M)=\sum (-1)^q\Gamma_{\fk n} ^q(M).$
    \item[(c)] By embedding into coherent families,
      $\Gamma^j_{\fk n}(t(w)\pi)=t(w)\Gamma^j_{\fk n}(\pi)$. 
  \end{description}  
\end{proposition}
 
Primitive ideals are grouped according to  the complex double
cells which are parametrized by special nilpotnet orbits. The notion
of double cone corresponds to inclusions of closures of special
orbits. In particular, $I\subset I'$ implies the double cell of $I'$
is in the double cone of the double cell of $I.$ From the previous
results, we conclude that the representations occuring in $\ovl{\C
  C}^{LR}$ must occur in the double cone determined by the $I(\pi)$ of
the representations in the HC-cell. The representations occuring in
$\C C^{LR}_+$ must have strictly lower Gelfand-Kirillov dimensions, so
have to occur in the double cones for complex orbits of strictly lower
dimension.
\begin{corollary}[Theorem 3.4,\ \cite{C}]
Let $M$ be an irreducible Harish-Chandra module with coherent family
$\theta(\mu)$. There exista a distribution $E_d$ so that the leading
term of the Taylor expansion is $D_{-d}(\mu)=p(\mu)E_{-d}$ where
$p(\mu)$ is a multiple of the Goldie rank polynomial of $I(\pi)$.
\end{corollary}
This is essentially Theorem 2 above. The polynomial is made more
precise; it is a multiple of the Goldie rank polynomial. 


\subsubsection{} The fact that the HC-cell consists of
$W-$representations  of the double cell corresponding to the primitive
ideal, does not follow directly from the results above. We don't know
a proof in this generality. We review some of the steps in \cite{McG}.   

Let $\C C_0$ be the double cell containing the special representation
$\sig_0$ which is the Goldie rank representation attached to $I(\gamma),$
The ideals $I(\pi)$ for $\pi\in
\C C^{LR}(\gamma)$ all have Goldie rank polynomials belonging to the
special representation $\sig_0.$ The corresponding double cell
determine a \textit{lower} double cone of $W-$representations belonging to smaller
nilpotent orbits, and an \textit{upper} double cone corresponding to
larger nilpotent orbits. The Weyl group representations  are related
by tensoring with $Sgn$ and the duality in \cite{BV2}. 

The results in \cite{V4}
imply that $\C
V_+^{LR}(\gamma^\vee)$ belongs to the \textit{upper} double cone
determined by the primitive ideal $I(\gamma^\vee).$
This implies that $\C V^{LR}(\gamma)$ consists of representations which belong
to an upper and a lower double cone, More precisely, 
the Goldie rank polynomial representation of $\theta_{\gamma^\vee}$
belong to the special representation of the  double cell attached to
$I(\gamma^\vee)$, and the representation $\sig_0\otimes Sgn$ belongs
to this upper cone. If we knew that $\sig_0\otimes Sgn$ is the Goldie
rank representation of $I(\gamma^\vee)$ the proof in McGovern would
suffice. 


\medskip
Instead, the next paragraph claims Theorem 1 without the above fact. It is
counting the unipotent representations  in terms of the coherent
continuation representation. The main point is that these
representations have maximal primitive ideal at a particular singular
infinitesimal character. Representations at this particular
infinitesimal character cannot give rise to representations which
belong to the strictly lower double cone. 

\bigskip
Let $\chi$ be an infinitesimal character of unipotent
representations. Consider a HC-cell that contains  special unipotent
parameters for $\chi.$  Here $\gamma$ has regular
integral infinitesimal character $\chi_{reg}$, and the actual unipotent parameter
is $T_\chi(\gamma)$ with $T_\chi$ the appropriate translation functor.
It  makes the simple roots $(\chi,\al)=0$ singular. These roots are not in the $\tau-$invariant of the
$\pi_i.$ Dualizing, each of the $\pi_i^\vee$ has these
simple coroots in the $\tau-$invariant. This is from earlier, combining \cite{V4} 
with \cite{ABV} for special unipotent parameters. Then
$\sum c_i\pi_i^\vee$ transforms according to $Sgn$ of $W(\chi).$ 
So the expression $\sum c_i\pi_i$ transforms according to $Triv$ of $W(\chi).$
A  combination $\sum c_i\pi_i$  decomposes into
Weyl group representation components. All of them are in the double
cone below the double cell determined by the Goldie rank/Primitive
ideal. This was proved earlier using \cite{V4} and \cite{C}.
It is a fact about these unipotent primitive ideals that the
representations in the striclty lower double cone cannot contain
$Triv$ of $W(\chi)$ as explained in \cite{BV2}. The lower double
cone divided by the strictly lower double cone is the double cell. 

A key fact is that the translation functor
from regular to singular infinitesimal character is exact, and the
kernel is given precisely in terms of the $\tau-$invariant. One 
(maybe best) way to see this is looking at all the $\pi^\vee$ who
have to transform by the $Sgn$ representation of $W$. The
``\textit{kernel}'' of the translation functor to infinitesimal
character $\chi^\vee$ satisfying $(\chi^\vee,\al^\vee)=0$ is precisely the set of
representations which have these $\al^\vee$ in their
$\tau-$invariant. Any combination $\sum c_i\pi_i^\vee$ from a double
cell transforms according to $Sgn$ of $W(\chi)$ precisely if each of the $\pi_i^\vee$ do; i.e. the
roots are in the $\tau-$invariant. This follows from the formula for
the $T_\al$ and arguing with the Bruhat order. 


This provides a proof of Theorem \ref{t:mult}, which
is sufficient for our purposes. The tableaus in the next section count
the multiplicities in Theorem \ref{t:mult}. The matchup of the representations
constructed via $\Theta$ with the tableaus shows that all special
unipotent representations are accounted for. 

\begin{conjecture}
The HC-cells in the cases of linear groups considered are all primitive
ideal   cells for the special unipotent primitive ideals. Each cell is
indexed by the associated variety together with the block that the
representations belong to. 
\end{conjecture}
This is sharper than the counting argument. 
\newpage










%%%%%%%%%%%%%%%%%%%%%%%%%%%%%%%%%%%%%%%%
\subsection{Multiplicity Formulas}
\subsubsection{Type B}\label{2.1} The Cartan subgroups are parametrized by
four integers $(p,q,2s,t),$ satisfying $p+q+2s+t=n.$ The corresponding
representation is 
\begin{equation}
\sum_{\sig\in\widehat{W}_{2s}}Ind_{W_p\times W_q\times W_{2s}\times S_t}^{W_n}
[sgn\otimes sgn\otimes\sig\otimes triv].
\end{equation}
The sum is over the $\sig=\tau\times\tau$ where $\tau$ is a partition
of $s.$ 
The representation $\sig$ is labelled by dots, sign by $r$ or $r',$ and $triv$
by $c.$ Recall also the well known formula
\begin{equation}
Ind_{S_n}^{W_n}(triv)=\sum_{a+b=n} (a)\times(b)
\label{2.1.1}\end{equation}
To induce we add $r$ and $r'$ at most one to each row to $\tau_R$, and
$c$ at most one to each column to both $\tau_L$ and $\tau_R$ the total
number being $r$.

{
  \begin{remark*}
This only counts  $So(a,b)$ with $a>b.$ The answer matches type $C$
tensored with $sgn$.  
  \end{remark*}
}


\subsubsection{Type C}\label{2.2} The Cartan subgroups are parametrized by
three integers $(a,2s,b)$ where $a+2s+b=n.$ For the coherent
continuation representation we need four integers $(t,2s,p,q),$
satisfying $t+2s+p+q=n.$ The corresponding representation is 
\begin{equation}
\sum_{\sig\in\widehat{W}_{2s}} \Ind_{S_t\times W_{2s}\times W_p\times W_q}
^{W_n}[sgn\otimes\sig\otimes triv\otimes triv].
\label{2.2.1}\end{equation}
The sum is over the $\sig=\tau\times\tau$ where $\tau$ is a partition
of $s.$ The notation is set up to take the duality in [V1] of types $B$ and
$C$ into account. So we write $r$ for the sign representation of
$S_t,$ and $c$ and $c'$ for the trivial representation of $W_p,\ W_q.$
We label the rows of $\tau_L$ as $r_0,r_2,\dots ,r_{2m}$ and the rows of
$\tau_R$ as $r_1,r_3,\dots ,r_{2m-1}$ to conform to the notation of the
special symbol
\begin{equation}
\begin{pmatrix} r_0&   &r_2+1&     &      &\dots       &            &r_{2m}+m\\
            &r_1&     &r_3+1&\dots &            &r_{2m-1}+m-1&       \end{pmatrix}
          \label{2.2.2}\end{equation}

\subsubsection{Type $\mathbf{\wti{C}}$}\label{ctilde}
{The coherent continuation representation  consists of
  two blocks $\C B_\pm$  from type D with infinitesimal
  character  regular formed of half-integers  $\chi_\pm=(n-\frac12,\dots ,\frac32,\pm \frac12)$. 
The sum $\C B:=\C B_++\C B_-$ admits a representation of $W(C_n).$ The
analogue of Proposition \ref{t:mult} holds for the larger
representation $\C B$. A representation $\tau_L\times \tau_R$ with
$\tau_L\ne \tau_R$ restricts irreducibly. The representations $\tau_L\times \tau_R$ and
$\tau_R\times\tau_L$ have the same rstriction. A representation with
$\tau_L=\tau_R$ restricts to a sum
$\tau\times\tau_I+\tau\times\tau_{II}.$ We will not use the finer
results about $\C B_\pm$ for $W(D_n)$, just $\C B$ for $W(C_n).$  

The description of $\C B$ is as follows. The Cartan subgroups are
parametrized by $(a,2s,b)$ with $a+2s+b=n.$ The coherent continuation representation is
$$
\sum_{\tau\times\tau}\Ind_{S_a\times C_{2s}\times S_b}[sgn\otimes \sig\otimes triv]. 
$$
The algorithm for the multiplicities is the same as in the other
cases. Use $r'$s for the imaginary part, $\bullet$ for the complex part,
and $c$ for the real part; and use the Littlewood-Richardson rule.
For an orbit $\vO$ with columns $C_{2a+1},C_{2a},\dots , C_2,C_1$ and
infinitesimal character $\chi(\vO),$  the relevant representations in
the Harish-Chandra cell are obtained by exchanging $(c_{2i+1}+1)\times c_{2i}$
with $(c_{2i}+2)\times (c_{2i+1}-1)$ for the pairs of odd sized columns
satisfying $C_{2i+1}=2c_{2i+1}+1>C_{2i}=2c_{2i}+1.$ Recall that
$\tau_L\times\tau_R\cong \tau_R\times\tau_L.$ 
\begin{example*}[$Mp(4,\bb R)$]
  



In
$sp(4,\bb R),$ the orbit $\C O=(2,2)$ admits representations with
infinitesimal character $h^\vee/2=(1/2,1/2)$ for the same
orbit viewed as $\vO$. 

\medskip
\begin{tabular}{|l|l|l|l|l|}
  \hline
  &&&&\\
  Parameter&$\CO$&LKT&S-Parameter&LKT\\
  &&&&\\
  \hline
  &&&&\\
  $(3/2,1/2)$&$(2,2)$&$(5/2,5/2)$&$-$&\\
  &&&&\\ %\hline\\ 
  $(3/2,-1/2)$&$(4)$&$(5/2,-1/2)$&$(1/2,-1/2)$ &$(3/2,-1/2)$\\
  &&&&\\ %\hline\\ 
$(-3/2,1/2)$&$(4)$&$(1/2,-5/2)$&$(1/2,-1/2)$ &$(1/2,-3/2)$\\
  &&&&\\ %\hline\\ 
  $(-3/2,-1/2)$&$(2,2)$&$(-5/2,-5/2)$&$-$&\\
  &&&&\\ %\hline\\
  $(\ul{3/2,1/2})$&$(2,2)$&$(3/2,-1/2),(1/2,-3/2)$&$-$ &\\
  &&&&\\ %\hline\\ 
  $(\ul{3/2,-1/2})$&$(2,2)$&$(3/2,-3/2)$&$-$ &\\
   &&&&\\ %\hline\\
  $(3/2,\ul{1/2})$&$(4)$&$(5/2,3/2)$&$(1/2,\ul{-1/2})$ &$(3/2,1/2)$\\
  &&&&\\ %\hline\\ 
  $(3/2,\ul{-1/2})$&$(2,2)$&$(5/2,1/2)$&$(1/2,\ul{1/2})$ &$(3/2,3/2)$\\
  &&&&\\ %\hline\\ 
  $({-3/2},\ul{1/2})$&$(2,2)$&$(-1/2,-5/2)$&$(-1/2,\ul{-1/2})$ &$(-3/2,-3/2)$\\
  &&&&\\ %\hline\\ 
  $(-3/2,\ul{-1/2})$&$(4)$&$(-3/2,-5/2)$&$(-1/2,\ul{1/2})$ &$(-1/2,-3/2)$\\
  &&&&\\ %\hline\\ 
  $(\ul{3/2},1/2)$&$(2,2)$&$(3/2,3/2)$&$(\ul{1/2},\ul{1/2})$ &$(1/2,1/2)$\\
  &&&&\\ %\hline\\ 
  $(\ul{3/2},-1/2)$&$(2,1,1)$&$(-1/2,-3/2)$&$-$ &\\
  &&&&\\ %\hline\\ 
  $(\ul{-3/2},1/2)$&$(2,1,1)$&$(3/2,1/2)$&$-$ &\\
  &&&&\\ %\hline\\ 
  $(\ul{-3/2},-1/2)$&$(2,2)$&$(-3/2,-3/2)$&$(\ul{-1/2},\ul{-1/2})$ &$(-1/2,-1/2)$\\
  &&&&\\ %\hline 
  \ $(\ul{3/2},\ul{1/2})$&$(2,1,1)$&$(1/2,1/2)$&$-$ &\\
  &&&&\\  %\hline\\ 
  $(\ul{3/2},\ul{-1/2})$&$(2,2)$&$(1/2,-1/2)$&$(\ul{1/2},\ul{-1/2})$ &$(1/2,-1/2)$ same\\
  &&&&\\ %\hline\\ 
  $(\ul{-3/2},\ul{1/2})$&$(2,2)$&$(1/2,-1/2)$&$(\ul{1/2},\ul{-1/2})$&$(1/2,-1/2)$ same\\
  &&&&\\ %\hline\\
  $(\ul{-3/2},\ul{-1/2})$&$(2,1,1)$&$(-1/2,-1/2)$&$-$ &\\
  &&&&\\
  \hline
\end{tabular}
\end{example*}
}
\newpage

\subsubsection{Type D}\label{2.3}\
\begin{comment}
{\clrr This seems to count for $O(a,b)$, as well as for $SO(a,b).$ Some Cartan subgroups become conjugate under $O(a,b)$; so fewer parameters. On the other hand, representations extend in two different ways}. 
\end{comment}
The Cartan subgroups are parametrized by
integers 
$$
(p,q,2s,t,u),\qquad p+q+2s+t+u=n.
$$
There are
actually two \CSGs for each $s>0$ whenever $p=q=0$. The representation $\sig$ is
denoted $\sig_I$ or $\sig_{II}$ according to the choice of Cartan
subgroup. The choice of subscript depends on a choice of the roots in
the ``\textit{fork}'' of the diagram. The representations are $\C J_{W(A_{n-1})}^{W'}(sgn)$ (in
the sense of Lusztig-Spaltenstein) from the
respective maximal subalgebras of type A.

The corresponding coherent continuation representation is 
\begin{equation}
\sum_{\sig\in\widehat{W}'_{2s}}
Ind_{W'\cap W_p\times W_q\times W'_{2s}\times W_t\times W_u}^{W'_n}
[sgn\otimes sgn \otimes\sig\otimes triv\otimes triv].
\end{equation}
The sum is over the $\sig=\tau\times\tau$ where $\tau$ is a partition
of $s,$ and both $\sig_{I,II}$ occur.   We label the $\sig$ by $\bullet'$s, trivial representations
by $c$ and $c'$ and the 
$sgn$ representations by $r$ and $r'.$ These are all added to $\tau_L$ 
when inducing. {\clrr Add the $r'$ before the $r.$}
The $sgn$ representation of $W'_n$ is denoted
$(1^n)\times\emptyset\cong \emptyset\times (1^n)$ and the trivial
representation is $(n)\times\emptyset\cong \emptyset\times (n).$ 

\bigskip
If the nilpotent orbit has rows $R_0\le R_1\le\dots\le R_{2m-1}$
satisfying $R_{2i}\cong R_{2i+1}\mod{2},$ write
$R_{2i}=2r_{2i}+1$ and $R_{2i+1}=2r_{2i+1}-1$ when they are odd,
$R_{2i}=2r_{2i}$ and $R_{2i+1}=2r_{2i+1}$ when they are even. The pair
$\tau_L\times\tau_R$ has rows 
\begin{equation}
  \label{eq:weyld}
(r_0,r_2,\dots ,r_{2m-2})\times   (r_1,r_3,\dots ,r_{2m-1}).
\end{equation}
This conforms to the special
symbol notation
\begin{equation}\label{eq:symbold}
\begin{pmatrix} 
r_0&r_2+1&\dots &r_{2m-2}+m-1\\
r_1&r_3+1&\dots &r_{2m-1}+m-1
\end{pmatrix}
\sim
\begin{pmatrix} 
  r_1&r_3+1&\dots &r_{2m-1}+m-1\\
  r_0&r_2+1&\dots &r_{2m-2}+m-1
\end{pmatrix}.
\end{equation}

%%%%%%%%%%%%%%%%%%%%%%%%%%%%%%%%%
\section{Reduction Theorems}\label{sec:reduction}
\subsection{Classical Cases}\

{\clrr We give details for Type C}. The orbit $\CO$ is assumed special.
Consider the multiplicity of the special representation. We refer to
\eqref{eq:tableauc} for $\tau_L\times \tau_R.$ 

\textbf{Assume $\mathbf{c_0>0}$}. Then the largest rows of $\CO$ must
be odd sized, and occur 
an even number of times; there is only one choice of signature. The column for
$c_0$ must be filled with $r$'s. The remainder of the rows
corresponding to the column of size $c_0$  in both $\tau_L$ and $\tau_R$ must
be filled with $\bullet'$s. Note that the rows in $\tau_L$ must be of size
one less than those of $\tau_R.$ The multiplicity is not affected if
they are removed from both $\tau_L$ and $\tau_R.$ The computation of
multiplicities is reduced to the case $c_0=0.$ 


\textbf{Assume $\mathbf{c_0=0}$ but $\mathbf{c_2>c_1}$}. The largest rows of $\CO$ are even. The
next largest ones are odd. 
The first $c_1$ rows in the column for
$c_2$ must be filled by $\bullet'$s or $r's$. The remaining rows in
$c_2$ are filled with $r'$s. The remainder of rows $c_1+1$ to $c_2$ in both $\tau_L$ and
$\tau_R$ (columns $c_3, c_4\dots$) must be filled with
$\bullet'$s. They  correspond to the next largest odd rows of $\CO,$ and
do not affect the multiplicities.

Alternatively, consider the case $C_{2a}>C_{2a-1}.$ 
There are $C_{2a}-C_{2a-1}$ rows of size 1. 
The bottom $c_{2a}-c_{2a-1}$ entries in column $c_{2a}$ are filled with $r'$s. 
They do not affect the multiplicity, and can be removed.

Furthermore, suppose still that $c_{2a}>c_{2a-1},$ and consider the other
nonspecial representations in the $HC-$cell. The argument is the same
for those with pairs $(c_{2a-1},c_{2a}).$  For those with
$(c_{2a}+1,c_{2a-1}-1)$, there must be $c_{2a}-c_{2a-1}$ entries
in the largest column of $\tau_L$ labelled $r$. The multiplicity is
not affected by removing the rows of size $1$. 

In general, $c_{2i}>c_{2i-1}$ corresponds to $2c_{2i}-2c_{2i-1}$ odd
sized rows in $\CO$. The corresponding rows in $\tau_L$ are one less
than those in $\tau_R.$ So the rows of $\tau_L$ must be filled with
$\bullet'$s while the rows of $\tau_R$ with $\bullet'$s and one $r.$
Similarly for the nonspecial representations with $c_{2i},c_{2i-1})$
and those with $c_{2i-1}+1,c_{2i}-1.$ 

In conclusion,  the multiplicity is not affected if the odd sized rows
of $CO$ are removed. 
\begin{theorem}[{Type C}]\label{t:1}
The multiplicities of a $\tau_L\times\tau_R$ coincide with the case
when $C_0=0$ and $c_{2i}=c_{2i-1}$ when $C_{2i},C_{2i-1}$ are both even, and
$c_{2i}=c_{2i-1}+1$ in case $C_{2i}C_{2i-1}$ are both odd. 

\end{theorem}
\begin{proof}
See above. 
\end{proof}
For the other types, the reduction is to the case when 
\begin{itemize}
\item $\CO$ has no even row sizes in types B,D,
\item $\CO$ has no odd row sizes in type C.
\end{itemize}
\begin{remark*}
The dual parameters are obtained by $\theta-$induction.  
\end{remark*}


\medskip
{\clrr Back to Type C}. 
Assume that $C_{2i+1}>C_{2i}=C_{2i-1}=\dots =C_{2i-2k+2}=C_{2i-2k+1}>C_{2i-2k}$
with $k>1.$ There are two cases:
\begin{itemize}
\item[(a)] all even, so $c_{2i}=\dots =c_{2i-2k-1}=c$,
\item[(b)] all odd, so $c_{2i}=c_{2i-1}+1=\dots = c_{2i-2k}=c_{2i-2k-1}+1=c.$ 
\end{itemize}
{
\noindent Case (a) \textbf{does not} admit special unipotent
representations; the dual orbit $\vO$ in type B is not even. However $\CO$
can be even, so serves as $\vO$ for type B. It fits with the previous
reduction for type B which was omitted. 

\noindent Case (b) \textbf{does} admit integral special unipotent representations.}


\medskip
In Case (a),  there are two types of Weyl group representations in
HC. The tableaus $\tau_L$ and $\tau_R$ both have $k$ columns of size
$c$, or $\tau_L$ has $k$ columns of size $c+1$, and $\tau_R$ has
$k$ columns of size $c-1.$  
\begin{theorem}[Type C]\label{t:2}
In Case (a), (analogue of Theorem \ref{t:1} for type B) the multiplicity of $\tau_L\times\tau_R$ is identical to the case when
$k=1.$ 

\medskip
In Case (b), there is a reduction to $k=1$. 
\end{theorem}
\begin{proof}
Consider Case (a). The columns for $c_{2i}$ to $c_{2i-2k+2}$ and
$c_{2i-1}$ to $c_{2i-2k+1}$ all must be
labelled $\bullet.$ They do not contribute to the multiplicity. 
So the multiplcity of the special representations are the same for
$k>1$ and $k=1.$ 


In Case (b), the pattern for $\tau$ for the bottom of columns
$c_{2i-1}$ to $c_{2i-2k-1}$ is
$$
\begin{matrix}
\bullet&\bullet &  \dots &\bullet &x_1\\
x_2      &y       &  \dots &z       &w \\
x_3      &        &        &        &
\end{matrix}
\quad\times\quad
\begin{matrix}
\bullet&\bullet &  \dots &\bullet &y_1\\
y_2      &\emptyset      &\emptyset  \dots &\emptyset      &\emptyset \\  
y_3      &        &        &        &
\end{matrix}
$$
where 
\begin{enumerate}
\item $x_1$ can be $\bullet, r,c$, while $y_1$ must $\bullet$ if
  $x_1=\bullet,$ and $r$ otherwise. It is completely determined by
  $x-1.$ 
\item  $x_2$ can be $\bullet,r,c,c'$,
\item $y,z, w$ must be $c$ or $c',$
\item $x_3, y_3$ may or may not occur.
\end{enumerate}
We compare the case of $k>1$ with $k=1.$ Every tableau with $k>1$ can
be obtained from one with $k=1$ by adding columns ending in $y,\dots ,z.$
 
\begin{enumerate}
\item $y=r,c$ corresponds to $\theta-$induction. There is only one
  choice of a factor $u(p,q).$ 
\item Subsequent
  columns ending in $c$ correspond to $\theta$ or $r-$induction, and a
  column ending in $c'$ corresponds to    $r-$induction.  
\end{enumerate}
For $\vO,$  the $r,c$ correspond to adding a
  pair of equal rows with a $
  \begin{matrix}
    +\\ -
  \end{matrix}
$. A column ending in $c'$ corresponds
  to adding a pair of rows with $
  \begin{matrix}
    +\\ +
  \end{matrix}
.$
\end{proof}
The following also holds.
\begin{theorem}  [Type C]\label{t:reduction}

Assume that $C_{2i+1}-C_{2i}>2$ with $C_{2i+1}=2c_{2i+1}$ and
$C_{2i}=2c_{2i}.$ The tableaus have $c_{2i+1}-c_{2i}$  pairs of equal
rows; at least one such pair  ends in $r\times r$, $c\times r$ or
$\bullet\times\bullet.$ The nilpotent orbit $\CO$ has a pair of equal rows with signs $
\begin{matrix}
  +\\ +
\end{matrix}
$ or $\begin{matrix}
  -\\ -
\end{matrix}$
in the $r,c$ cases, $\begin{matrix}
  +\\ -
\end{matrix}$ in the $\bullet$ case. The dual orbit $\vO$
is real induced in the first two cases, $\theta-$induced in the
$\bullet$ case. 
\end{theorem}






  \subsection{Metaplectic Case}
  \subsection{Relation to type B}
Because of the reduction theorem in
the previous section, we only need to consider the case when both
$\CO$ and $\vO$ are even. The $\Theta-$correspondence provides a
relation between unipotent representations of type B and type $\wti
C.$ Since $O(p,q)\cong SO(p,q)\times\bZ_2,$ there is a factor of 2
between the two groups. Furthermore, the $\Theta-$correspondence has
domain all of $O(p,q)$ not just $p>q.$ This is another factor of 2
accounted for by the fact that there are two blocks in type $\wti C,$
encoded by counting $\tau_L\times\tau_R$ and $\tau_R\times\tau_L$
separately. 

\bigskip
An even nilpotent orbit $\CO_{\wti C}$ in type
$\wti C$ (same as type C of course) will have columns
$$
C_{2a+1}=C_{2a}>\dots >C_2=C_1
$$
with $C_i$ even or odd. Such an orbit comes (via $\Theta$) from
$\CO_B$ which has columns
$$
C_{2a}>C_{2a-1}=C_{2a-2}>\dots >C_2=C_1.
$$
A pair of even $C_{2i+1}=C_{2i}=2c_i$ contributes a pair of 
columns  $c_i\times c_i$ to $\tau_L^{\wti C}\times\tau_R^{\wti C}.$ A pair of odd
columns $C_{2i+1}=C_{2i}=2c_i+1$ contributes $(c_{i}+1)\times
c_{i}$. 
The relation between $\CO_{\wti C}$ and $\CO_{B}$ is as follows. If
$C_{2a+1}=2c_a+1$ is odd, remove the largest column, and let the rows
start with the same signs as for $\wti C.$ If $C_{2a+1}=2c_a$ is even,
remove it, and increase $C_{2a}$ by one. In all cases remove the
largest column of $\tau_R.$ An unmatched $\bullet$ in $\tau_L$ is
replaced by an $r'$. There should be no $r$ in $\tau_L^B$; replace
with $\bullet.$ This produces a tableau dual to those of type
C in the following sense. Consider the tableau of type C as described
in an earlier section. Tensor by sign which interchanges $\tau_L$
with $\tau_R$ and transposes. Replace $r\longleftrightarrow c$
and $c'\leftrightarrow r'.$ Rearrange to have the $r'$ earlier than  the
others.

\subsection{Nonspecial Representations}
\begin{proposition}[hccells]
  \label{p:hc}
The Weyl group representations in the Harish-Chandra  cells accociated
to the same orbit all occur with  the same multiplicity. 
\end{proposition}
\begin{proof}
The shapes $\tau_L\times\tau_R$ look like
$$
\begin{matrix}
  &\dots &&&c_i+1&\dots\\
  &\dots &\dots&\dots&\dots&\dots\\
  \star&\dots &\star&\star&\star&\dots\\
  \star&\dots &\star&\star&\star
\end{matrix}
\quad\times\quad
\begin{matrix}
  &\dots     &&c_i&&\dots\\
  &\dots &\dots&\dots&\dots&\\
  \star&\dots &\star&\star&\\
  \star&\dots &\star&
\end{matrix}.
$$
The rows above are of greater or
equal length, and the rows below of strictly smaller
length. We must move one block from the column of length $c_i+1$ in $\tau_L$ to the column of length $c_i$ in $\tau_r.$ The column in $\tau_L$ must end in $c$ or $r.$ If it ends in $r$, then the block to the left must be a $\bullet$. The corresponding block in $\tau_R$ must also be a $\bullet.$ In this case, remove one $r$ from the column $c_i+1$ and add it as high as possible to the column $c_i$ of $\tau_L.$ The other case is if the block is $c.$ If there is an $r$ above it, move one $r$ from $\tau_L$ to $\tau_R$ as before. If not, either $c_i=0$ or else there are only $\bullet$ above in the column. In these cases, move the $c$ from column $c_i+1$ in $\tau_L$ to the column $c_i$ in $\tau_R;$ The $c$ in the column for $\tau_L$ moves up one block.

The following are some examples.
$$
\begin{aligned}
&\begin{matrix}
    \bullet&\bullet&\al\\
  \bullet&\bullet&r\\
\end{matrix}
\quad\times\quad
\begin{matrix}
\bullet&\bullet&c\\
\bullet&\bullet&\\
\end{matrix}
\quad\mapsto\quad
\begin{matrix}
    \bullet&\bullet&\al\\
  \bullet&\bullet&
\end{matrix}
\quad\times\quad
\begin{matrix}
\bullet&\bullet&r\\
\bullet&\bullet&c\\ 
\end{matrix}
\\
&
\mat
\bullet&\bullet&r\\
\beta_1&\beta_2&c
\emat
\quad\times\quad
\mat
\bullet&\bullet&c\\
y_1&y_2&
\emat
\quad\mapsto\quad
\mat
\bullet&\bullet&c\\
\beta_1&\beta_2&
\emat
\quad\times\quad
\mat
\bullet&\bullet&r\\
\bullet&\bullet&c
\emat
\end{aligned}
$$
\end{proof}
\begin{proposition}[reduction, odd rows]
  \label{p:oddrows}
Suppose an odd columns size of $\vO$ repeats (\ie
$C_i=C_i=C_{i+1}=C_{i+1}$). The mumber of unipotent representations
for $\vO$ and $\vO'$ with one pair of columns removed is the same. For
$\CO,$ a pair of odd size rows is removed.
\end{proposition}
\begin{proof}
Consider the special representation, \ie the one with corresponding
rows all $c_i+1\times c_i$. The shape (with the first row indicating
the column position) 
$$
\begin{matrix}
  1&\dots &c_i-1&c_i&c_i+1\\
  &\dots &\dots&\dots&\dots\\
  \star&\dots &\star&\star&\star\\
  \star&\dots &\star&\star&\star
\end{matrix}
\quad\times\quad
\begin{matrix}
  1&\dots     &c_i-1&c_i\\
  &\dots &\dots&\dots\\
  \star&\dots &\star&\star\\
  \star&\dots &\star&\star
\end{matrix}
$$
occurs in $\tau_L\times\tau_R$, with the rows above of greater or
equal length, and the rows below of strictly smaller
length. The earlier row in $\tau_L$ must have $\bullet$ up
to entry $c_i,$ and therefore the same must hold for $\tau_R.$ Column $c_i+1$ 
in $\tau_L$ must end in  $
\begin{matrix}
r\\\star  
\end{matrix}
$. So the tableaus are
$$
\begin{matrix}
    1&\dots &c_i-1&c_i&c_i+1\\
  &\dots &\dots&\dots&\dots\\
  \bullet&\dots &\bullet&\bullet&r\\
  \star&\dots &\star&\star&\star
\end{matrix}
\quad\times\quad
\begin{matrix}
  1&\dots &c_i-1&c_i\\
  &\dots &\dots&\dots\\
  \bullet&\dots &\bullet&\bullet\\
  \star&\dots &\star&\star
\end{matrix}
$$
The claim follows from the fact that the higher pair of rows is always the same.
\end{proof}
Again,   counting is reduced to the case when both $\CO$ and $\vO$ have
even sized rows only.

\subsubsection{The orbits $\C O$ and  $\vO$}
Consider the orbit with columns
$C_i$ such that $C_{2i}\equiv C_{2i-1}\ (mod\ 2)$ and
$C_{2i}=C_{2i-1}$ if $C_{2i}\equiv 0\ (mod\ 2)$, so that it
represents a nilpotent orbit of type $C$. In addition, assume that
$C_{2i+1}>C_{2i}$; so this is after the reduction in theorem \ref{t:1}. Associate to it the dual orbit $\vO$ with rows
$C_{2i}+1,C_{2i-1}-1$ whenever the pair is odd length as before, and
$C_{2i},C_{2i-1}$ whenever  the length is even. The condition
$C_{2i+1}>C_{2i}$ insures that this is a nilpotent
orbit (even). 

\begin{theorem}\label{t:seesaw}
For the nilpotent orbits $\C O$ satisfying $C_{2i+1}>C_{2i}$ all the
representations are obtained by the $\Theta-$correspondence via the
algorithm using the columns. Pairs of even colummns $C_{2i}=C_{2i-1}$
are replaced by  $C_{2i}-1,C_{2i-1}+1$ to insure that the seesaw pairs
are all of the form $SO(2m+1)\times Mp(2n).$ This is where unions of
nilpotent orbits show up in $AV(\pi)$.

For cases when $C_{2i}=C_{2i-1}\equiv 0\ (mod\ 2),$ the unipotent
representations fit into induced modules from unipotent
representations on Levi components.
\end{theorem}

\begin{proposition*} When $C_{2i+1}>C_{2i},$ the number of special unipotent representations with associated
variety $\C O$ equals the sum over the conjugacy classes in the
component groups of the real forms of $\C O$. All special unipotent
representations as above have asymptotic cycles single real forms. 
  
\end{proposition*}


\begin{example}
Let $\C O=2^{2n}.$ The infinitesimal character is 
$$
(n-1/2,n-1/2,\dots ,1/2,1/2).
$$
There are $4n+1$ unipotent representations. They are all obtained by
the $\Theta-$correspondence from the pair $Sp(4n,\bb R)\times O(2n+1).$
There are $2n+1$ representations which are derived modules and $2n$
representations which occur in induced modules from characters of
$\wti{GL}(2n,\bb R).$ The associated cycle is a sum of two
  \textit{adjacent} nilpotent orbits. Each $O(p,q)$ with $p+q=2n+1$
  contributes, one representation if $p$ or $q=0,$ and three
  otherwise. There are coincidences between $O(p,q)$ and $O(p-1,q+1)$
  and $O(p,q)$ and $O(p+1,q-1).$ 
\end{example}
\begin{example}
 Let $\C O=2^{2n+1}.$ The infinitesimal character is
 $(n+1/2,n-1/2,n-1/2,\dots ,1/2,1/2).$ There are $8n$ representations, each
 with associated cycle one of $2n+1$ real forms. They correspond to
 the $O(p,q)$ with $p+q=2n+1;$ $2n-1$ get four representations each,
 and $2$ of them get two each. 

\end{example}




\section{Algorithms for the Associated Variety}\label{3}

Denote by $\pi$ an irreducible Harish--Chandra module, and by $\chp$ the
dual. Let $\AV(\pi)$ be its associated variety and $\Ac(\pi)$ the primitive ideal variety. We describe $\AV(\pi)$ and $\AV(\chp)$ for
special unipotent representations.
The representations for which all
the components in $\AV$ belong to the same complex orbit $\C O_c$ will 


\medskip
We parametrize everything in terms of the even
nilpotent $\vOc.$ The cases when $\vOc$ has only odd sized rows in
type C and only even sized rows in type B,\ D are omitted since they
are easy. It is well known that $\Ac(\pi)$ is the closure of a single
orbit, so $\AV(\pi)$ (and $\AV(\chp)$) is a union of real forms of a
single orbit $\COc.$ The {\it signature} of a column of size $x$ of a
real nilpotent orbit denotes the number of $+,-$ in the first column. 

\subsection{}\label{sec:3.1} The following will be a consequence of the
algorithms. 
\begin{enumerate}
\item In type B,D, the signature on an odd length column of any
  nilpotent orbit in  $\AV(\pi)$ is the
  same, say  $(a,b).$  On a column of even length, the signature can
  only be $(a,b+1)$ or $(a+1,b).$ 
\item In type C, the signature of an odd length column in any
  nilpotent orbit in $\AV(\pi)$ can only be $(a,b+1)$ or $(a+1,b).$
  On an even length column, the signature is always $(a,b).$ 
\end{enumerate}
This has to be true for $\AV(\chp)$ as well.

\subsection{Type  C/B}\label{3.2} The convention is that $\CO$ is type
C and $\vO$ is type B. Let $\COc$ be an even special nilpotent orbit (in
$\g_c$ of type C) so that the orbit has columns $(C_{2a}=C_{2a-1}> C_{2a-2}=C_{2a-3}> \dots >C_2= C_{1}>C_0,$  conforming to the reduction in the previous section. 

The dual orbit $\vOc$ is even with rows  
\begin{equation}
  \begin{cases}
    C_{2i}+1,C_{2i-1}-1 &\text{ if } C_{2i}=C_{2i-1} \text{ is even,}\\
   C_{2i},C_{2i-1} &\text{ if } C_{2i}=C_{2i-1} \text{ is odd,}\\
   C_0+1 & \text{ if } i=0.
  \end{cases}
\label{3.2.1}\end{equation}
%The Lusztig symbol is $(r_0,r_1)(r_2+1,r_3+1),\dots %(r_{2m-2}+m-1,r_{2m-1}+m-1)(r_{2m}+m).$ 
The Weyl group representation is
\begin{equation}
(c_{2a-1},c_{2a-3},\dots, c_{1})\times (c_{2a},c_{2a-2},\dots ,c_{0})
\end{equation}
as in \eqref{eq:tableauc}.
The labelling of $\tau_R$ is completely determined by
its shape and the labelling of $\tau_L,$ so we often only give
$\tau_L$. Recall that each row in $\tau_R$ is filled by $\bullet$
except possibly the last entry.


\bigskip
We  build $\AV(\pi)$ in stages. Start with the first pair of columns. The $\bullet'$s add pairs of rows with signs $\begin{matrix}  +\\-\end{matrix}$. The remainder of $r,c$ adds
\begin{equation}
\begin{matrix} r\\ \vdots\\ r\end{matrix} \times \begin{matrix} r\\ \vdots\\ x\end{matrix}
\leftrightarrow \begin{matrix} +-\\ \vdots \\ +-\end{matrix}\qquad
\begin{matrix} r\\ \vdots\\ c\end{matrix} \times \begin{matrix} r\\ \vdots\\ x\end{matrix}
\leftrightarrow \begin{matrix} -+\\ \vdots \\ -+\end{matrix}.
\label{3.2.5}\end{equation}
The number of rows in the nilpotent orbit is odd if $x=\emptyset$ and
even if $x=r.$  

A pair $c'\times r$ adds a pair of rows $\mat +-\\-+\emat$ while
$c'\times \emptyset$ adds a $\mat +-\emat\cup\mat -+\emat$ to make a
union of nilpotent orbits.

For example, in case $\CO=(222),$ we get the following
7 tableaus and their corresponding $\AV(\pi)$:
\begin{alignat}{2}
&\begin{matrix} r\\ r\end{matrix}\times \begin{matrix} r\\ \ \end{matrix},\quad 
\begin{matrix} r\\ c\end{matrix}\times \begin{matrix} r\\ \ \end{matrix}
&&\qquad\longleftrightarrow\qquad
\begin{matrix} +&-\\ +&-\\ +&-\end{matrix},\quad \begin{matrix} -&+\\ -&+ \\
-&+\end{matrix},\label{3.2.6a}\\
\noalign{\bigskip}
&\begin{matrix} \bullet\\ r\end{matrix}\times \begin{matrix} \bullet\\ \ \end{matrix},\quad
\begin{matrix} \bullet\\ c\end{matrix}\times \begin{matrix}\bullet\\ \ \end{matrix}
&&\qquad\longleftrightarrow\qquad
\begin{matrix} +&-\\ -&+\\ +&-\end{matrix},\quad \begin{matrix} +&-\\ -&+\\ -&+\end{matrix},
\label{3.2.6b}\\
\noalign{\bigskip}
&\begin{matrix} r\\ c'\end{matrix}\times \begin{matrix} r\\ \ \end{matrix},\quad \begin{matrix}
c\\ c'\end{matrix}\times \begin{matrix} r\\ \ \end{matrix} 
&&\qquad\longleftrightarrow\qquad 
\begin{matrix} +&-\\ +&-\\ +&-\end{matrix}\cup \begin{matrix} +&-\\+&-\\ -&+\end{matrix},\quad
\begin{matrix} -&+\\ -&+\\ -&+\end{matrix}\cup \begin{matrix} -&+\\ -&+\\ +&-\end{matrix},
\label{3.2.6c}\\ 
\noalign{\bigskip}
&\begin{matrix} \bullet\\ c'\end{matrix}\times \begin{matrix}\bullet\\ \ \end{matrix}
&&\qquad\longleftrightarrow\qquad 
\begin{matrix} +&-\\ -&+\\ +&-\end{matrix}\cup \begin{matrix} +&-\\ -&+\\ -&+\end{matrix}.
\label{3.2.6d}
\end{alignat}

\bigskip
\noindent Now we describe how to add the next pair of columns. Assume
the column in $\tau_L$ does not end in $c'.$ A 
typical pair of columns in $\tau_L\times \tau_R$ is of the form
$$
\begin{matrix}
  \bullet\\ \vdots \\\bullet\\ r\\ \vdots \\r \\y
\end{matrix}
\times
\begin{matrix}
  \bullet\\ \vdots \\\bullet\\ r\\ \vdots \\r \\ x
\end{matrix}
$$
The previous columns must have $\bullet$ next to the $\bullet$'s and $r'$s.
The pair $y\times x$ can be $r\times r,$ $r\times\emptyset,$ $c\times r$
or $c\times\emptyset.$ If $x\ne \emptyset,$ the previous column must have
a $\bullet.$ Otherwise the pattern in the last row is $cc\times r,$ $rc\times r$ or
$\bullet c\times r.$ 

\bigskip
\noindent\textbf{(I)} A pair $\bullet\times\bullet$ can only be added
to a pair of rows which end in $\bullet\times\bullet.$ Add two to the
row size, and preserve the signs.  

The remaining rows end in  (II) or (III).

\noindent\textbf{(II)} 
\begin{equation}
\begin{matrix} r\\ \vdots\\ r\end{matrix}\times\begin{matrix} r\\ \vdots\\ x\end{matrix}
\label{3.2.7}\end{equation}
can only be adjoined to a column with $\bullet;$ add 2 to the
corresponding rows and change all the signs to $+.$ If $x=\emptyset,$
all the rows change to $+.$ 

\noindent\textbf{(III)} A pair 
\begin{equation}
\begin{matrix} r\\ \vdots\\ r\\ c\end{matrix}\times\begin{matrix} r\\
  \vdots\\ r\\ x\end{matrix}
\label{3.2.8}\end{equation}
adds 2 to the corresponding rows. When $x=r,$ the previous columns in
both $\tau_L$ and $\tau_R$ are filled with $\bullet,$ so they have
pairs of rows with $\pm;$  change them all to $-$.
When $x=\emptyset,$ the previous columns have
$\bullet$ except for the entry next to $c$ which may be $\bullet,r,c.$
The patterns are 
{\clrr
$$
\begin{aligned}
&cc\times r, &&\text{changes the rows with - to +, except the last one gets a -},\\
&rc\times r, &&\text{changes the rows with + to -, except the last one stays a +},\\
&\bullet c\times r, &&\text{changes the rows to -.}
\end{aligned}
$$
}
%The first changes the rows starting with $-$ to $+$, except for the last one %which gets a
%$-$.  The second changes all the rows staring with $+$ to $-$, except for the %last one which stays $+$. The third changes all the rows to $-$.

\bigskip
\noindent We now add the $c'.$ 

\noindent\textbf{(IV)} A pair $c'\times r$ can only be added to a pair
of rows ending in $\bullet\times\bullet$ (due to the restriction of $\vO$ having to be even). It increases the size of the
corresponding pair of rows in $\vO$ by 2 and leaves their signs $\pm$ unchanged.

\noindent\textbf{(V)} A pair $c'\times \emptyset$ adds 2 to one of the rows of the appropriate size, and leaves the sign unchanged. If there are two rows of the same
size with opposite signs, add two nilpotent orbits to $\AV(\pi),$ one
for each choice of sign. 
\subsection{Type D}\label{3.3}  
Let $\CO$ be a special nilpotent orbit so that the dual orbit $\vO$ is
even. The special Weyl group representation is, as in \ref{eq:drep};
we use
\begin{equation*}
\tau_L\times\tau_R=(c_{2a-1},\dots ,c_1)\times (c_{2a-2},\dots
,c_0).  
\end{equation*}
The associated variety is obtained in the same way as for type C;
$r'\times \emptyset$ is treated as $\bullet$. Recall it is added first.
If there no earlier column it contributes a pair of rows
$\mat+\\-\emat$. If there is an earlier column,  the pattern is
$\mat\bullet r'\emat\times \mat\bullet\emptyset\emat$. The
$\bullet\times\bullet$ contributes a pair of even sized equal rows of opposite
signs; augment them both by  one. Subsequent columns and $r,c,c'$ are added as in
type C.

\begin{remark*}
The following provides an \textit{``embedding''} of type D into type
C.  Add 1 to each row in $\tau_R,$ ($m-1$ of them including the empty
ones). Change $r'$ to $\bullet$ in $\tau_L$  and put $\bullet$ in the corresponding entry in
$\tau_R.$ The remaining new entries in $\tau_R$ get filled with $r$'s.

This embeds  the $AV(\pi)$ corresponding to an $\CO_C$
in type D, into those for the $\CO_c$ for type C where each row has been
augmented by one. It implements the $\Theta-$correspondence. 
\end{remark*}
\begin{remark*}
In the case when $\CO$ and $\vO$ are both special, the $HC-$cells are
related by tensoring with $sgn.$ The main case is when they are both
even. The procedure for computing the
$AV(\pi)$ from the multiplicity of the special representation in the
coherent continuation representation is \textbf{not} compatible with
tensoring with $sgn$.  
\end{remark*}

\newpage
\subsection{TYPE B/C} The coherent continuation representation of type
$B$ is obtained from type $C$ by tensoring with $sgn.$ In this case
$\vO$ is type B and $\CO$ is even type C. As a result, the columns of
$\tau_L\tau_R$ are labelled $c_{2i}\ge \dots c_0\ge 0.$ 
Tensoring with $sgn$ means that the multiplicities are obtained from
type C by interchanging and taking transposes of $\tau_L\times\tau_R.$
The correspondence is $c,c'\longleftrightarrow r',r,$ and
$r\longleftrightarrow c.$  We add $\bullet$ first, then $r'$ then $r$
and last $c$. 

The labelling of $\tau_L$ is completely determined by its shape and
the labelling of $\tau_R,$ so we often only give $\tau_L$. 

\bigskip
Recall that the coherent continuation representation formulas are for
$SO(a,b)$ with $a>b.$ For the $\Theta-$correspondence we will need
$a<b$ as well (duplicate the answers by changing all the signs) and
O(a,b). The passage for this last one is more complicated; some
reprewsentations extend to irreducible ones, others in two different ways.
The algorithms give nilpotent orbits in $so(a,b)$ such that some times
$a>b$ and other times $b<a.$ 

\subsubsection{Case $c_{2j-1}>c_{2j}$} This is the case when the
number of represwentations matches the number of local systems on the
real forms of $\CO.$

We start with the pair $c_{2i}\ge c_{2i=1}.$ Start with the entries up
to $c_{2i}$ in both of them. The matchup is according to the entries
at the end
$$
\bullet\times\bullet\longleftrightarrow
\mat
-&+&-\\
+&-& \\
-&+ & \\
\vdots&\vdots &\\
+&-& \\
-&+ & \\
+ &&\\
+ &&\\
\emat
\qquad
c\times c, c\times r\longleftrightarrow
\mat
+&-&+\\
+&-& \\
-&+ & \\
\vdots&\vdots &\\
+&-& \\
-&+ & \\
+ &&\\
- &&\\
\emat
\qquad
c\times r'\longleftrightarrow
\mat
+&-&+\\
+&-& \\
-&+ & \\
\vdots&\vdots &\\
+&-& \\
-&+ & \\
+ &&\\
+ &&\\
\emat
$$
Each element in the tail in column $c_{2i}$ adds a pair
$$
r'\longleftrightarrow \mat +\\+\emat\qquad r,c\longrightarrow \mat +\\-\emat.
$$
We now add $c_{2i-2}>c_{2i-3}.$ The symbols $\bullet,r',r,c$ get added
next to $\bullet'$s in $\tau_R$. The same rules apply. for the
$c_{2i-3}$ pairs of entries, 
\begin{description}
\item[(a)] ending in $\bullet\times\bullet$ apply $\theta-induction$ so that the
  signs alternate on the odd size rows.
\item[(b)] ending in $c\times r'$ apply $\theta-$induction so that the
  last two even sized rows go up by one, and get the same sign as the
  next pair of odd sized rows
\item[(c)] endiong in $c\times r,c$ apply $\rho-$induction.  
\end{description}
The tail increases the last even pair of rows to a pair larger by one according to
\begin{description}
\item[d] $c\times r'$ changes the pair to the same sign as the pair of
  rows larger by 1, $c\times r,c$ changes it to apir with $\mat +\\-\emat$.
\end{description}

\subsubsection{Case $c_{2j-1}=c_{2j}$} Form the nilpotent with the
$c_{2j}-1.$ The remaining slot is occupied by $\emptyset\times r',
\emptyset\times r,c.$ For $r'$  apply $\theta$induction increasing one
row by two with the same sign as the existing rows. For
$\emptyset\times r,c$ use $\rho-induction.$ proceed from the smallest
pair upwards. An exception is the case
$$
\mat \bullet&r\\c& \emat,
\qquad
\mat\bullet &c\\r &\emat.
$$
In this case, increase row by two so as to form a pair
$$
\mat +&\dots
&+\\ - &\dots & -\emat
$$




\begin{comment}
  

\bigskip
We start with a pair of rows of size 2. The tableau

\begin{alignat}{3}
&\text{Tableau }\qquad     &\qquad\qquad&\CO\subset
Sp\qquad&\qquad\qquad&\vO\subset So\notag\\
\noalign{\smallskip}
&c\times r            &&\mat -+\\ -+\emat&&\mat +&-&+\\ +&&\\ -&&\emat\notag\\ 
\noalign{\smallskip}
&r\times r            &&\mat +-\\ +-\emat&&\mat +&-&+\\ +&&\\ -&&\emat\label{eq:11}\\ 
\noalign{\smallskip}
&c'\times r           &&\mat +-\\ -+\emat&&\mat +&-&+\\ +&&\\ +&&\emat\notag\\
\noalign{\smallskip}
&\bullet\times \bullet&&\mat +-\\ -+\emat&&\mat -&+&-\\ +&&\\ +&&\emat\notag
\end{alignat}

The pairs $c\times r$ and $r\times r$ correspond to $\rho-$induction of
the nilpotent in $O(1)$ with a  single $-$ by a $gl(2,\bR).$ The pairs $c'\times r$ and
$\bullet\times \bullet$ correspond to $\theta-$induction; the first with $u(2,0)$,
the second with $u(1,1).$ 

If there are more pairs of rows of size 1, apply $\rho/\theta$ induction for
each added pair starting at the bottom. A  $c'\times r$ is at the
bottom, and so the nilpotent orbit starts as in \eqref{eq:11}.     
Then apply  $\rho-$induction for each subsequent $r\times r$ or $c\times
r.$  The Levi component on the $\vO-$side is a $gl(2,\bR).$  For the
$\bullet\times\bullet'$s use $\theta-$induction with $u(2,0)$ or
$u(0,2)$; there is only one choice. If there is no $c'\times r$ present,
start with the nilpotent orbit corresponding to
\eqref{eq:11}. The $r\times r$ and $c\times r$ are added by
$\rho-$induction; $\AV$ is a union of two orbits. The
$\bullet\times\bullet$ are added by $\theta-$induction with Levi
component $u(1,1).$

\bigskip
Next consider the triangular case, $\tau_L=\tau_R$, and each size row
occurs exactly once in $\tau_L=\tau_R,$ say $(m,m-1,\dots ,1).$ The
corresponding nilpotent orbit has rows, $(2m+1,2m-1,2m-1,\dots
,3,3,1,1).$ The largest row of $\vO$ gets sign $+.$ Alternate the
signs of the pairs of rows of $\vO$ going downward. Then change the
signs as follows: 
$$
\begin{aligned}
  &\bullet\times \bullet\qquad &&\text{ keep
  the signs }&&\quad \mat +\\+\emat\text{ or }\mat -\\-\emat\\
  &c'\times r\qquad &&\text{ flip
    the signs }&&\quad \mat +\\+\emat \longleftrightarrow \mat -\\-\emat\\
  &r\times r,\ c\times r\qquad  &&\text{ change the signs to
    alternate }&&\quad \mat +\\-\emat.
\end{aligned}
$$
Another way of saying this is that each pair of rows corresponds to
$\theta-$induction when the pair ends in $\bullet\times\bullet$ or
$c'\times r,$ and $\rho-$induction when the pair ends in $r\times r$
or $c\times r$. In the case of $\theta-$induction, there are two
choices. For $\bullet\times\bullet$ choose the one where the result
has $p-q$ in $SO(p,q)$ smaller.
Repeated pairs of (equal) rows give the cases when $\tau_L=\tau_R.$ These are added
by the same rules as adding pairs of rows of size 1.

\bigskip
In the case there are rows of unequal length, consider the
semi-triangular case when the rows of $\tau_L$ and $\tau_R$ are
$$
(m,m-1,\dots ,1)\times (m-1,\dots ,0).
$$
The nilpotent orbit $\vO$
has rows
$$
\vO=(2m-1,2m-1,2m-3,2m-3,\dots , 3,3,1,1,1).
$$
The edges of $\tau_L$ end
in $r,c,c'.$ Strip them to get a triangular $\tau_L\times\tau_R$, and
form the nilpotent orbit. Then add the
edges starting with the largest row. An $r,c$ adds a pair of rows
$\mat +\\-\emat$ at
the bottom. A $c'$ adds a $\mat -\\-\emat$ if next to a $\bullet$, a
$\mat +\\+\emat$ if next to an $r.$

For the entry in the next lower row in $\tau_L,$ increase one row by two following
$\rho/\theta$-induction for $r/c$ and $c'.$ 
\begin{itemize}
\item Adding $r/c$ follows $\rho-induction$ increasing a row by 2,
  except for
  $$
\mat\bullet& r\\c&\emat \qquad \mat\bullet& c\\ r&\emat,
$$
when only a row with a $-$ is increased.
\item  Adding $c'$ adds two and flips a $-$ if below $\bullet,$ adds
  two and flips a $+$ if below $r.$ 
\end{itemize}
Examples \ref{3.5} and \ref{3.6} illustrate these rules.


Continue this way down the rows of $\tau_L$.

When sizes of pairs of rows are repeated, form the nilpotent orbit
from $\tau_L\times \tau_R$ with the duplicate pairs of rows
ending in $\bullet$ and $r/c$ removed, and then add the pairs of equal
rows as in the case $\tau_L=\tau_R$.

\end{comment}

\newpage
\subsection{Example}\label{3.5} Consider the infinitesimal character $(1,0,0)$
corresponding to $\CO=(3,1,1,1,1)$ and $\vO=(4,2).$ Only half the
size of each row of the nilpotent
orbit in $Sp$ is listed with its sign. {\clrr For type C, double the sizes of
the rows to get the actual partition of the nilpotent orbit}. We get
$\pmat  x&x\epmat\times \mat  x&\emat:$ 

\bigskip
\hrule
\medskip
$\mat \bullet&r\emat$\qquad $\mat -&+&-\\+&&\\+&&\\+&&\\-&&\emat$\qquad
\qquad $\mat+&-\\+&\emat$\qquad $(\ul{1}^-,\ul{0}^-,\ul{0}^-),\ (10\ul{0}^+)$
\hfill
\medskip
\hrule
\medskip
$\mat \bullet&c\emat$\qquad $\mat -&+&-\\+&&\\+&&\\+&&\\-&&\emat$\qquad
\qquad $\mat-&+\\-&\emat$\qquad $(\ul{1}^-,\ul{0}^-,\ul{0}^-),\ (0-1\ul{0}^+)$
\hfill
\medskip
\hrule
\medskip
$\mat \bullet&c'\emat$\qquad $\mat -&+&-\\+&&\\+&&\\+&&\\+&&\emat$\qquad
\qquad $\mat-&+\\+&\emat\cup\mat+&-\\-&\emat $
\qquad $(\ul{1}^+,\ul{0}^+,\ul{0}^+),\ (\ul{10}\ul{0}^+)$
\hfill
\medskip
\hrule
\medskip
$\mat c'&c'\emat$\qquad $\mat +&-&+\\+&&\\+&&\\+&&\\+&&\emat$\qquad
\qquad $\mat-&+\\+&\emat\cup\mat+&-\\-&\emat $
\qquad $(\ul{1}^+,\ul{0}^-,\ul{0}^+)$
\hfill
\medskip
\hrule
\medskip
$\mat r&c\emat$\qquad $\mat +&-&+\\+&&\\-&&\\+&&\\-&&\emat$\qquad
\qquad $\mat-&+\\+&\emat$\qquad $(\ul{1}^+,\ul{0}^+,\ul{0}^-)$
\hfill
\medskip
\hrule
\medskip
$\mat c&c\emat$\qquad $\mat +&-&+\\+&&\\-&&\\+&&\\-&&\emat$\qquad
\qquad $\mat+&-\\-&\emat$\qquad $(\ul{1}^+,\ul{0}^+,\ul{0}^-)$
\hfill
\medskip
\hrule
\medskip
$\mat r&c'\emat$\qquad $\mat +&-&+\\+&&\\-&&\\+&&\\+&&\emat$\qquad
\qquad $\mat+&-\\+&\emat$\qquad $(\ul{1}^-,\ul{0}^-,\ul{0}^+),\ (10\ul{0}^-)$
\hfill
\medskip
\hrule
\medskip
$\mat c&c'\emat$\qquad $\mat +&-&+\\+&&\\-&&\\+&&\\+&&\emat$\qquad
\qquad $\mat-&+\\-&\emat$\qquad $(\ul{1}^-,\ul{0}^-,\ul{0}^+),\ (0-1\ul{0}^-)$

\hfill
\medskip

\subsection{Example}\label{3.6} Consider the infinitesimal character
$(1,1,0,0)$ corresponding to the nilpotent $(4,2,2)$ in type $C$ dual to
$(3,3,1,1,1)$ in type $B.$ The parameters, $\AV(\pi)$ and $\AV(\chp)$  correspond to
$\mat x&x\\x&\emat\times\mat x&\\&\emat,$  and are as follows.

\bigskip
\hrule
\medskip
$\mat \bullet&r\\ c&\emat$\qquad  $\mat +&-&+\\ -&+&-\\+&&\\+&&\\-&&\emat$
\qquad $\mat +&-\\-&\\+\emat$\qquad $\ul{1}^+\ul{1}^-\ul{0}^+\ul{0}^-$
\hfill
\medskip
\hrule
\medskip
$\mat r&c\\ c&\emat$\qquad  $\mat +&-&+\\-&+&-\\+&&\\+&&\\-&&\emat
\cup  \mat +&-&+\\+&-&+\\-&&\\+&&\\-&&\emat$
\qquad $\mat +&-\\-&\\-\emat$\qquad $\ul{10}\ul{1}^-\ul{0}^-$\hfill
\medskip
\hrule
\medskip
$\mat \bullet&c'\\ r&\emat$\qquad  $\mat +&-&+\\ -&+&-\\+&&\\+&&\\+&&\emat$
\qquad $\mat +&-\\-&\\+\emat\cup\mat -&+\\+&\\+\emat $
\qquad $\ul{1}^+\ul{1}^-\ul{0}^-\ul{0}^-$
\hfill
\medskip
\hrule
\medskip
$\mat r&c\\ r&\emat$\qquad  $\mat +&-&+\\-&+&-\\+&&\\+&&\\-&&\emat
\cup  \mat +&-&+\\+&-&+\\-&&\\+&&\\-&&\emat$
\qquad $\mat -&+\\+&\\+\emat$\qquad $\ul{10}\ul{1}^-\ul{0}^-$\hfill
\medskip
\hrule
\medskip
$\mat \bullet&c\\ r&\emat$\qquad  $\mat +&-&+\\ -&+&-\\+&&\\+&&\\-&&\emat$
\qquad $\mat -&+\\-&\\+\emat$\qquad $\ul{1}^+\ul{1}^-\ul{0}^+\ul{0}^-$\hfill
\medskip
\hrule
\medskip
$\mat \bullet&c'\\ c&\emat$\qquad  $\mat +&-&+\\ -&+&-\\+&&\\+&&\\+&&\emat$
\qquad $\mat -&+\\+&\\-\emat\cup\mat +&-\\-&\\-\emat $
\qquad $\ul{1}^+\ul{1}^-\ul{0}^-\ul{0}^-$
\hfill
\medskip
\hrule
\medskip
$\mat c&c\\ c'&\emat$\qquad  $\mat -&+&-\\ -&+&-\\+&&\\-&&\\-&&\emat$
\qquad $\mat +&-\\-&\\-\emat\cup\mat +&-\\-&\\+\emat $
\qquad $\ul{1}^+\ul{1}^+\ul{0}^+\ul{0}^-$\hfill
\medskip
\hrule
\medskip
$\mat r&c\\ c'&\emat$\qquad  $\mat -&+&-\\ -&+&-\\+&&\\-&&\\-&&\emat$
\qquad $\mat -&+\\+&\\+\emat\cup\mat -&+\\+&\\-\emat $
\qquad $\ul{1}^+\ul{1}^+\ul{0}^+\ul{0}^-$\hfill
\medskip
\hrule
\medskip
$\mat \bullet&c'\\ c'&\emat$\qquad  $\mat -&+&-\\ -&+&-\\+&&\\+&&\\+&&\emat$
\qquad $\mat -&+\\+&\\+\emat\cup\mat -&+\\+&\\-\emat\cup 
\mat +&-\\-&\\-\emat\cup\mat +&-\\-&\\+\emat$
\qquad $\ul{1}^+\ul{1}^+\ul{0}^+\ul{0}^+$\hfill
\medskip
\hrule
\medskip
$\mat \bullet&r\\ r&\emat$\qquad  $\mat +&-&+\\+&-&+\\-&&\\-&&\\-&&\emat
\cup  \mat +&-&+\\-&+&-\\+&&\\-&&\\-&&\emat$
\qquad $\mat +&-\\+&\\+\emat$\qquad ${10}\ul{1}^-\ul{0}^-$\hfill
\medskip
\hrule
\medskip
$\mat \bullet&c\\ c&\emat$\qquad  $\mat +&-&+\\+&-&+\\-&&\\-&&\\-&&\emat
\cup  \mat +&-&+\\-&+&-\\+&&\\-&&\\-&&\emat$
\qquad $\mat -&+\\-&\\-\emat$\qquad $0-1\ul{1}^-\ul{0}^-$
\hfill
\medskip
\hrule
\medskip
$\mat r&c'\\ r&\emat$\qquad  $\mat +&-&+\\+&-&+\\-&&\\+&&\\+&&\emat
\cup  \mat +&-&+\\-&+&-\\+&&\\+&&\\+&&\emat$
\qquad $\mat +&-\\+&\\+\emat$\qquad ${10}\ul{1}^-\ul{0}^+$\hfill
\medskip
\hrule
\medskip
$\mat r&c'\\ c&\emat$\qquad  $\mat +&-&+\\+&-&+\\-&&\\+&&\\+&&\emat
\cup  \mat +&-&+\\-&+&-\\+&&\\+&&\\+&&\emat$
\qquad $\mat -&+\\-&\\-\emat$\qquad $0-1\ul{1}^-\ul{0}^+$\hfill
\medskip
\hrule
\medskip
$\mat \bullet&r\\ c'&\emat$\qquad  $\mat +&-&+\\ +&-&+\\-&&\\+&&\\-&&\emat$
\qquad $\mat +&-\\+&\\-\emat\cup\mat +&-\\+&\\+\emat $
\qquad $\ul{1}^-\ul{1}^-\ul{0}^-\ul{0}^-$
\hfill
\medskip
\hrule
\medskip
$\mat \bullet&c\\ c'&\emat$\qquad  $\mat +&-&+\\ +&-&+\\-&&\\+&&\\-&&\emat$
\qquad $\mat -&+\\-&\\-\emat\cup\mat -&+\\-&\\+\emat $
\qquad $\ul{1}^-\ul{1}^-\ul{0}^-\ul{0}^-$
\hfill
\medskip
\hrule
\medskip
$\mat r&c'\\ c'&\emat$\qquad  $\mat +&-&+\\ +&-&+\\+&&\\+&&\\+&&\emat$
\qquad $\mat +&-\\+&\\+\emat\cup\mat +&-\\+&\\-\emat\cup \mat-&+\\+&\\+&\emat $
\qquad $\ul{1}^-\ul{1}^-\ul{0}^-\ul{0}^+$
\hfill
\medskip
\hfill
\medskip
\hrule
\medskip
$\mat c&c'\\ c'&\emat$\qquad  $\mat +&-&+\\ +&-&+\\+&&\\+&&\\+&&\emat$
\qquad $\mat -&+\\-&\\-\emat\cup\mat -&+\\-&\\+\emat\cup\mat +&-\\-&\\-\emat  $
\qquad $\ul{1}^-\ul{1}^-\ul{0}^-\ul{0}^+$
\hfill
\medskip
\hrule


\subsection{Example}\label{3.7} Consider the infinitesimal character
$(2,1,1,1,0,0)$ corresponding to the nilpotent $(4,4,4,2)$ in type $C$ dual to
$(5,3,3,3,1)$ in type $B.$ The parameters and supports correspond to
$\pmat x&x\\x&x&\epmat\times\pmat x&x\\x&\epmat$ are as follows.

\bigskip
\hrule
\medskip
$\mat \bullet&\bullet\\ \bullet&r\emat$\qquad  
$\mat +&-&+&-&+\\ +&-&+&&\\-&+&-&&\\-&+&-&&\\+&&&&\emat$
\qquad $\mat +&-\\-&+\\+&-\\+&&\emat$
\qquad $\ul{2}^-\ul{1}^+\ul{1}^-\ul{0}^-\ul{1}^-\ul{1}^-\ul{0}^-$
\hfill
\medskip
\hrule
\medskip
$\mat \bullet&\bullet\\ \bullet&c\emat$\qquad  
$\mat +&-&+&-&+\\ +&-&+&&\\-&+&-&&\\-&+&-&&\\+&&&&\emat$
\qquad $\mat +&-\\-&+\\-&+\\-&&\emat$
\qquad $\ul{2}^-\ul{1}^+\ul{1}^-\ul{0}^-\ul{1}^-\ul{1}^-\ul{0}^-$
\hfill
\medskip
\hrule
\medskip
$\mat \bullet&\bullet\\ \bullet&c'\emat$\qquad  
$\mat +&-&+&-&+\\ -&+&-&&\\-&+&-&&\\-&+&-&&\\+&&&&\emat$
\qquad $\mat +&-\\-&+\\+&-\\-&&\emat\cup\mat +&-\\-&+\\-&+\\+&&\emat$
\qquad 
\qquad $\mat \ul{2}^+\ul{1}^+\ul{1}^+\ul{1}^+\ul{1}^+\ul{0}^+\ul{0}^+\\
\ul{2-1}\ \ul{10}\ \ul{1}^+\ul{1}^+\ul{0}^+\emat$
\hfill
\medskip
\hrule
\medskip
$\mat \bullet&\bullet\\ r&c\emat$\qquad  
$\mat -&+&-&+&-\\ +&-&+&&\\+&-&+&&\\-&+&-&&\\+&&&&\emat$
\qquad $\mat +&-\\-&+\\-&+\\+&&\emat$
\qquad 
\qquad $\mat\ul{2}^+\ul{1}^+\ul{1}^+\ul{0}^+\ul{1}^+\ul{1}^-\ul{0}^-,\\
\ul{2-1}\ \ul{10}\ \ul{1}^+\ul{1}^-\ul{0}^-\emat$
\hfill
\medskip
\hrule
\medskip
$\mat \bullet&\bullet\\ c&c\emat$\qquad  
$\mat -&+&-&+&-\\ +&-&+&&\\+&-&+&&\\-&+&-&&\\+&&&&\emat$
\qquad $\mat +&-\\-&+\\+&-\\-&&\emat$
\qquad 
\qquad $\mat\ul{2}^+\ul{1}^+\ul{1}^+\ul{0}^+\ul{1}^+\ul{1}^-\ul{0}^-,\\
\ul{2-1}\ \ul{10}\ \ul{1}^+\ul{1}^-\ul{0}^-\emat$
\hfill
\medskip
\hrule
\medskip
$\mat \bullet&\bullet\\ r&c'\emat$\qquad  
$\mat -&+&-&+&-\\ +&-&+&&\\-&+&-&&\\-&+&-&&\\+&&&&\emat$
\qquad $\mat +&-\\-&+\\+&-\\+&&\emat$
\qquad 
\qquad $\mat\ul{2}^-\ul{1}^-\ul{1}^-\ul{0}^-\ul{1}^+\ul{1}^+\ul{0}^+,\\
\ul{2}^-\ \ul{10}\ \ul{0}^-\ul{1}^+\ul{1}^+\ul{0}^+\emat$
\hfill
\medskip
\hrule
\medskip
$\mat \bullet&\bullet\\ c&c'\emat$\qquad  
$\mat -&+&-&+&-\\ +&-&+&&\\-&+&-&&\\-&+&-&&\\+&&&&\emat$
\qquad $\mat +&-\\-&+\\-&+\\-&&\emat$
\qquad 
\qquad $\mat\ul{2}^-\ul{1}^-\ul{1}^-\ul{0}^-\ul{1}^+\ul{1}^+\ul{0}^+,\\
\ul{2}^-\ \ul{10}\ \ul{0}^-\ul{1}^+\ul{1}^+\ul{0}^+\emat$
\hfill
\medskip
\hrule
\medskip
$\mat \bullet&\bullet\\ c'&c'\emat$\qquad  
$\mat -&+&-&+&-\\ -&+&-&&\\-&+&-&&\\-&+&-&&\\+&&&&\emat$
\qquad $\mat +&-\\-&+\\-&+\\+&&\emat\cup\mat +&-\\-&+\\+&-\\-&&\emat $
\qquad 
\qquad $\ul{2}^+\ul{1}^+\ul{1}^-\ul{0}^-\ul{1}^-\ul{1}^-\ul{0}^-$
\hfill
\medskip
\hrule
%\newpage
\hrule
\medskip
$\mat \bullet&r\\ \bullet&r\emat$\qquad  
$\mat -&+&-&+&-\\ +&-&+&&\\+&-&+&&\\+&-&+&&\\-&&&&\emat\cup
\mat -&+&-&+&-\\ +&-&+&&\\+&-&+&&\\-&+&-&&\\+&&&&\emat$
\qquad $\mat +&-\\+&-\\+&-\\+&&\emat$
\qquad $\mat 21\ \ul{1}^-\ul{0}^-\ul{1}^+\ul{1}^+\ul{0}^+,\\ 1\
\ul{2}^-\ul{1}^-\ul{0}^-\ul{1}^+\ul{1}^+\ul{0}^+\emat$ 
\hfill
\medskip
\hrule
\medskip
$\mat \bullet&r\\ \bullet&c\emat$\qquad  
$\mat -&+&-&+&-\\ +&-&+&&\\+&-&+&&\\+&-&+&&\\-&&&&\emat\cup
\mat -&+&-&+&-\\ +&-&+&&\\+&-&+&&\\-&+&-&&\\+&&&&\emat$
\qquad $\mat -&+\\-&+\\-&+\\-&&\emat$
\qquad $\mat -1-2\ \ul{1}^-\ul{0}^-\ul{1}^+\ul{1}^+\ul{0}^+,\\ -1\
\ul{2}^-\ul{1}^-\ul{0}^-\ul{1}^+\ul{1}^+\ul{0}^+\emat$
\hfill
\medskip
\hrule
\medskip
$\mat \bullet&r\\ \bullet&c'\emat$\qquad  
$\mat -&+&-&+&-\\ +&-&+&&\\+&-&+&&\\+&-&+&&\\+&&&&\emat$
\qquad $\mat +&-\\+&-\\-&+\\+&&\emat\cup\mat +&-\\+&-\\+&-\\-&&\emat$
\qquad $\ul{2}^-\ul{1}^-\ul{1}^-\ul{0}^-\ul{1}^-\ul{1}^-\ul{0}^-$
\hfill
\medskip
\hrule
\medskip
$\mat \bullet&c\\ \bullet&c'\emat$\qquad  
$\mat -&+&-&+&-\\ +&-&+&&\\+&-&+&&\\+&-&+&&\\+&&&&\emat$
\qquad $\mat -&+\\-&+\\-&+\\+&&\emat\cup\mat -&+\\-&+\\+&-\\-&&\emat$
\qquad $\ul{2}^-\ul{1}^-\ul{1}^-\ul{0}^-\ul{1}^-\ul{1}^-\ul{0}^-$
\hfill
\medskip
\hrule
\medskip
$\mat \bullet&r\\ r&c\emat$\qquad  
$\mat +&-&+&-&+\\ +&-&+&&\\-&+&-&&\\-&+&-&&\\+&&&&\emat\cup
\mat +&-&+&-&+\\ +&-&+&&\\-&+&-&&\\+&-&+&&\\-&&&&\emat$
\qquad $\mat -&+\\-&+\\-&+\\+&&\emat$
\qquad $1\ul{2-1}\ \ul{10}\ul{1}^-\ul{0}^-$
\hfill
\medskip
\hrule
\medskip
$\mat \bullet&r\\ c&c\emat$\qquad  
$\mat +&-&+&-&+\\ +&-&+&&\\-&+&-&&\\-&+&-&&\\+&&&&\emat\cup
\mat +&-&+&-&+\\ +&-&+&&\\-&+&-&&\\+&-&+&&\\-&&&&\emat$
\qquad $\mat +&-\\+&-\\+&-\\-&&\emat$
\qquad $-1\ul{2-1}\ \ul{10}\ul{1}^-\ul{0}^-$ 
\hfill
\medskip
\hrule
\medskip
$\mat \bullet&c\\ c&c'\emat$\qquad  
$\mat +&-&+&-&+\\ +&-&+&&\\+&-&+&&\\-&+&-&&\\+&&&&\emat\cup
\mat +&-&+&-&+\\ +&-&+&&\\+&-&+&&\\+&-&+&&\\-&&&&\emat$
\qquad $\mat -&+\\-&+\\-&+\\-&&\emat$
\qquad $\mat{-1}{-2}\ \ul{1}^-\ul{0}^-\ul{1}^-\ul{1}^-\ul{0}^-,\\
{-1}\ \ul{2}^-\ \ul{1}^-\ul{0}^-\ul{1}^-\ul{1}^-\ul{0}^-\emat$
\hfill
\medskip
\hrule
\medskip
$\mat \bullet&r\\ r&c'\emat$\qquad  
$\mat +&-&+&-&+\\ +&-&+&&\\+&-&+&&\\-&+&-&&\\+&&&&\emat\cup
\mat +&-&+&-&+\\ +&-&+&&\\+&-&+&&\\+&-&+&&\\-&&&&\emat$
\qquad $\mat +&-\\+&-\\+&-\\+&&\emat$
\qquad $\mat{2}{1}\ \ul{1}^-\ul{0}^-\ul{1}^-\ul{1}^-\ul{0}^-,\\
{1\ }\ul{2}^-\ \ul{1}^-\ul{0}^-\ul{1}^-\ul{1}^-\ul{0}^-\emat$
\hfill
\medskip
\hrule
\medskip
$\mat \bullet&c\\ r&c'\emat$\qquad  
$\mat -&+&-&+&-\\ -&+&-&&\\-&+&-&&\\+&-&+&&\\+&&&&\emat$
\qquad $\mat -&+\\-&+\\+&-\\+&&\emat$
\qquad $\mat\ul{2}^+\ul{1}^+\ul{1}^+\ul{0}^+\ul{1}^-\ul{1}^-\ul{0}^-,\\
\ul{2-1}\ \ul{10}\ \ul{1}^-\ul{1}^-\ul{0}^-\emat$
\hfill
\medskip
\hrule
\medskip
$\mat \bullet&r\\ c&c'\emat$\qquad  
$\mat -&+&-&+&-\\ -&+&-&&\\-&+&-&&\\+&-&+&&\\+&&&&\emat$
\qquad $\mat +&-\\+&-\\-&+\\-&&\emat$
\qquad $\mat\ul{2}^+\ul{1}^+\ul{1}^+\ul{0}^+\ul{1}^-\ul{1}^-\ul{0}^-,\\
\ul{2-1}\ \ul{10}\ \ul{1}^-\ul{1}^-\ul{0}^-\emat$
\hfill
\medskip
\hrule

\medskip
$\mat \bullet&c\\ c'&c'\emat$\qquad  
$\mat +&-&+&-&+\\ +&-&+&&\\+&-&+&&\\+&-&+&&\\+&&&&\emat$
\qquad $\mat -&+\\-&+\\+&-\\-&&\emat\cup\mat -&+\\-&+\\-&+\\+&&\emat$
\qquad $\mat\ul{2}^-\ul{1}^+\ul{1}^-\ul{0}^-\ul{1}^+\ul{1}^+\ul{0}^+,\\
-1\ \ul{2}^+\ul{1}^-\ul{0}^-\ul{1}^+\ul{1}^+\ul{0}^+\emat$
\hfill
\medskip
\hrule
\medskip
$\mat \bullet&r\\ c'&c'\emat$\qquad  
$\mat +&-&+&-&+\\ +&-&+&&\\+&-&+&&\\+&-&+&&\\+&&&&\emat$
\qquad $\mat +&-\\+&-\\+&-\\-&&\emat\cup\mat +&-\\+&-\\-&+\\+&&\emat$
\qquad $\mat\ul{2}^-\ul{1}^+\ul{1}^-\ul{0}^-\ul{1}^+\ul{1}^+\ul{0}^+\\ 
1\ \ul{2}^+\ul{1}^-\ul{0}^-\ul{1}^+\ul{1}^+\ul{0}^+\emat$
\hfill
\medskip
\hrule
\newpage
\subsection{Type $\wti{C}$}

Consider the case when all pairs of columns of $\tau_L$ and $\tau_R$
differ by one. Successive pairs are strictly smaller so that the
corresponding dual nilpotent orbit is even also. Otherwise $\AV$ is
obtained by $\rho-$induction or $\theta-$induction. The conventon is
\begin{enumerate}
\item $\rho-$induction means adding $2$ to the length of a  column
  leaving the sign unchanged
\item $\theta-$induction means adding $2$ to the end of the column,
  and changing the sign from $-$ to $+$ for $r$ and $+$ to $-$ for $c.$
\end{enumerate}
Due to the previous section, we only deal with special tableaus. We
add pairs of columns  starting with the  largest pair. The largest
pair of columns is 
$$
\begin{aligned}
&\mat
\bullet &\times &\bullet\\
\vdots & &\vdots\\
\bullet & &\bullet\\
r&   &r\\
\vdots & &\vdots\\
r& &r\\
r& &x\\
a & &
\emat &\qquad\text{ or } \qquad
\mat
\bullet &\times &\bullet\\
\vdots & &\vdots\\
\bullet & &\bullet\\
r&   &r\\
\vdots & &\vdots\\
r& &r\\
r& &r\\
a && x
\emat
\end{aligned}
$$
For unequal columns, the nilpotent orbit gets
\begin{enumerate}
\item pairs of rows $\mat +-\\ -+\emat$ for $\bullet\times\bullet$
\item pairs of rows $\mat +-\\+-\emat$ for $r\times r$ and $x=r$, and $\mat -+\\-+\emat$,   for $x=c$,
\item a row $\mat +-\emat$ for $a=r$, and $\mat -+\emat$ for $a=c$.
\end{enumerate}
For equal columns, use
\begin{enumerate}
\item pairs of rows $\mat+-\\-+\emat$ for $\bullet\times\bullet$
\item pairs of rows $\mat +-\\+-\emat$ for $a=r$, and $\mat-+\\-+\emat$  for $a=c$ for all $r\times r$.
\item one last pair of rows with sign as before for $a=x$, and a union of nilpotent orbits, one with $\mat +-\\+-\emat\cup\mat +-\\-+\emat$ for $a=r, x=c$, and $\mat +-\\-+\emat\cup\mat -+\\-+\emat$ for $a=c, x=r.$
\end{enumerate}
To describe the process of adding the next pair of columns, we
describe the last two pairs of columns of $\tau_L\times \tau_R.$  
$$
\begin{aligned}
&(a)&&(b)&&(c)&&(d)\\
&\mat
\vdots &\vdots&\times &\vdots &\vdots\\
\bullet &r& &\bullet&r\\
\bullet &\al & &\bullet&x\\
\bullet& & &\bullet&\\
\vdots && &\vdots&\\
\bullet&  & &\bullet&\\
r&&&r\\
\vdots && &\vdots&\\
r&&&r&\\
r&&&y&\\
\gamma&&&&
\emat
\quad
&&\mat
\vdots &\vdots&\times &\vdots &\vdots\\
\bullet &r& &\bullet&r\\
r&c & &r&c\\
r&&&r\\
\vdots && &\vdots&\\
\vdots && &\vdots&\\
\vdots && &\vdots&\\
r&&&r&\\
r&&&y&\\
\gamma&&&&
\emat
\quad
&&\mat
\vdots &\vdots&\times &\vdots &\vdots\\
\bullet &r& &\bullet&x\\
\bullet &\al & &\bullet&\\
\bullet& & &\bullet&\\
\vdots && &\vdots&\\
\bullet&  & &\bullet&\\
r&&&r\\
\vdots && &\vdots&\\
r&&&r&\\
r&&&r&\\
\gamma&&&y&
\emat
\quad
&&\mat
\vdots &\vdots&\times &\vdots &\vdots\\
\bullet &r& &\bullet&x\\
r&c & &r&\\
r&&&r\\
\vdots && &\vdots&\\
\vdots && &\vdots&\\
\vdots && &\vdots&\\
r&&&r&\\
r&&&r&\\
\gamma&&&y&
\emat
\end{aligned}
$$
and four more,
$$
\begin{aligned}
&(e)&&(f)&&(g)&&(h)\\  
&\mat
\vdots &\vdots&\times &\vdots &\vdots\\
\bullet &r& &\bullet&r\\
\bullet &\al & &\bullet&x\\
\bullet& & &\bullet&\\
\vdots && &\vdots&\\
\bullet&  & &\bullet&\\
r&&&r\\
\vdots && &\vdots&\\
r&&&r&\\
r&&&r&\\
\gamma&&&y&
\emat
\quad
&&\mat
\vdots &\vdots&\times &\vdots &\vdots\\
\bullet &r& &\bullet&r\\
r&c & &r&c\\
r&&&r\\
\vdots && &\vdots&\\
\vdots && &\vdots&\\
\vdots && &\vdots&\\
r&&&r&\\
r&&&r&\\
\gamma&&&y&
\emat
\quad
&&\mat
\vdots &\vdots&\times &\vdots &\vdots\\
\bullet &r& &\bullet&x\\
\bullet &\al & &\bullet&\\
\bullet& & &\bullet&\\
\vdots && &\vdots&\\
\bullet&  & &\bullet&\\
r&&&r\\
\vdots && &\vdots&\\
r&&&r&\\
r&&&y&\\
\gamma&&&&
\emat
\quad
&&\mat
\vdots &\vdots&\times &\vdots &\vdots\\
\bullet &r& &\bullet&x\\
r&c & &r&\\
r&&&r\\
\vdots && &\vdots&\\
\vdots && &\vdots&\\
\vdots && &\vdots&\\
r&&&r&\\
r&&&y&\\
\gamma&&&&
\emat
\end{aligned}  
$$
The vertical dots can be $\bullet\bullet\times\bullet\bullet$ or $\bullet
r\times\bullet r$. 
The cases are grouped by the pairs of columns (larger, smaller)
\begin{description}
\item[(a) and (b)] (unequal, equal)
\item[(c) and (d)] (equal, unequal)
\item[(e) and (f)] (equal,equal)
\item[(g) and (h)] (unequal, unequal)  
\end{description}
The appropriate rows are increased by two, and the signs are changed as follows.
Denote by the \textit{tail} the portion including and below 
$$
\begin{aligned}
&(a)(e)&&(b)(f)&&(c)(g)&&(d)(h)\\   
&\mat\bullet&\al\\&\emat\times\mat\bullet&x\\ &\emat,\qquad
&& \mat r&c\\&\emat\times\mat r&c\\&\emat,\qquad
&&\mat\bullet& r\\\bullet&\al\emat
\times\mat\bullet& x\\\bullet &\emat,\qquad
&&\mat\bullet&r\\ r&c\emat\times\mat\bullet&x\\ r& \emat,\qquad
%\mat\bullet&r\\\bullet&\al\emat \times \mat\bullet&x\\\bullet &\emat,\qquad
%\mat \bullet&r\\ r&c\emat\times \mat\bullet&x\\r&\emat.
  \end{aligned}
$$ 
We construct the signs for the tail. The remainder will be determined;
the remaining pairs of rows are  $\bullet\bullet\times\bullet\bullet$ and
$\bullet r\times\bullet r$.
\begin{enumerate}
\item $\bullet\bullet \times\bullet\bullet$
changes $\mat +\\-\emat$ to $\mat +-\\-+\emat.$   
\item In cases (a)(e) and (c)(g), the rows of size 2 follow the signs
  for a single pair of columns for the tail. The rows of size 4 follow
  the signs of the pair of smaller columns.
\item In cases (b), (d) and (h) construct the signs for the pair of columns
  below $rc\times r$ as for a single pair of columns. The extra $c$ in
  $\tau_L$ increases one row to $+-+-$ or $-+-+$ according to $y$.
  The rows above, $\bullet r\times\bullet r$ are increased by $2,$ and
  follow the sign of $rc\times r$.
\item For case (f), for the tail below $rc\times rc$, construct the
  signs as in types (b), (d), (h) without the $c$ in $\tau_L.$  The omitted
  $c$ in $tau_L$ is added by $\rho-$induction. The pairs of rows
  $\bullet r\times\bullet r$ above, acquire the sign from  $\gamma;$
  $+-+-$ for $\gamma=r,$ and $-+-+$ for $\gamma=c.$    
\end{enumerate}

\bigskip
The following examples are templates for the algorithms.
\begin{example} $\CO_c=(4442)$. We record half the sizes of the rows
  to save space. We start with the first pair of columns:
  $$
  \begin{aligned}
    &\mat\bullet\\\bullet\emat\times\mat\bullet\\\bullet\emat
    &&\mat\bullet\\ r\emat\times\mat\bullet\\ r\emat
    &&\mat\bullet\\ r\emat\times\mat\bullet\\ c\emat
    &&\mat\bullet\\ c\emat\times\mat\bullet\\ r\emat
    &&\mat\bullet\\ c\emat\times\mat\bullet\\ c\emat\\
    &\mat+\\-\\+\\-\emat
    &&\mat+\\-\\+\\+\emat
    &&\mat+\\-\\+\\+\emat\cup\mat +\\-\\+\\-\emat
    &&\mat+\\-\\+\\+\emat\cup\mat -\\-\\-\\+\emat
    &&\mat+\\-\\-\\-\emat\\
     &\mat r\\ r\emat\times\mat r\\ r\emat
    &&\mat r \\ r\emat\times\mat r\\ c\emat
    &&\mat r\\ c\emat\times\mat r\\ r\emat
    &&\mat r\\ c\emat\times\mat r\\ c\emat \\
    &\mat+\\+\\+\\+\emat
    &&\mat+\\+\\+\\+\emat\cup\mat +\\+\\+\\-\emat
    &&\mat-\\-\\-\\+\emat\cup\mat -\\-\\-\\-\emat
    &&\mat-\\-\\-\\-\emat
  \end{aligned}
  $$
$$
\begin{aligned}
&\mat\bullet&\bullet\\\bullet&r\emat\times\mat\bullet&\bullet\\\bullet&\emat\qquad
&&\mat\bullet&\bullet\\\bullet&c\emat\times\mat\bullet&\bullet\\\bullet &\emat
\\
&\mat+-\\-+\\+-\\+\quad\emat
&&\mat+-\\-+\\-+\\-\quad\emat
\end{aligned}
$$
$\line(1,0){360}$
$$
  \begin{aligned}
&  
\mat \bullet &\bullet\\r&c\emat \times\mat\bullet &\bullet\\{r}& \emat&&\quad\mat \bullet&
\bullet\\r&c\emat \times\mat\bullet& \bullet\\ {c}&\emat&&\quad\mat \bullet&
\bullet\\c&c\emat\times\mat\bullet& \bullet\\{r}& \emat&&\quad\mat \bullet&
\bullet\\c&c\emat \times\mat\bullet&\bullet\\{c}&\emat\\
&\mat +-\\-+\\+-\\{+}\quad\emat
&&\quad
\mat +-\\-+\\+-\\{+}\quad\emat\cup\mat +-\\-+\\+-\\ {+}\quad\emat
&&\quad
\mat +-\\-+\\-+\\{-}\quad\emat\cup\mat +-\\-+\\-+\\ {+}\quad\emat
&&\quad
\mat +-\\-+\\-+\\ {-}\quad\emat
\end{aligned}
$$
$\line(1,0){360}$
$$
\begin{aligned}
  &\mat \bullet &r\\\bullet&r\emat \times\mat\bullet &r\\\bullet& \emat
  &&\quad\mat \bullet&r\\\bullet&r\emat \times\mat\bullet& c\\\bullet&\emat
  &&\quad\mat \bullet&r\\\bullet &c\emat\times\mat\bullet& r\\\bullet&\emat
  &&\quad\mat \bullet&r\\\bullet&c\emat \times\mat\bullet&c\\\bullet&\emat\\
&\mat +-\\+-\\+-\\ {+}\quad\emat
&&\quad
\mat -+\\-+\\+-\\ {+}\quad\emat
&&\quad
\mat +-\\+-\\-+\\ {-}\quad\emat
&&\quad
\mat -+\\-+\\-+\\ {-}\quad\emat
\end{aligned}
$$
$$
  \begin{aligned}
&\mat \bullet &r\\r&c\emat \times\mat\bullet &r\\{r}& \emat
&&\quad\mat \bullet&r\\r&c\emat \times\mat\bullet& r\\{c}&\emat
&&\quad\mat \bullet&r\\r&c\emat\times\mat\bullet& c\\{r}& \emat
&&\quad\mat \bullet&r\\r&c\emat \times\mat\bullet&c\\{c}&\emat\\
&\mat +-\\+-\\+-\\ {+}\quad \emat
&&\quad\mat +-\\+-\\+-\\{+}\quad\emat\cup\mat+-\\+-\\+-\\{-}\quad\emat
&&\quad\mat -+\\-+\\+-\\{+}\quad\emat
&&\quad\mat -+\\-+\\+-\\{-}\quad\emat\cup\mat-+\\-+\\+-\\{+}\quad\emat
\end{aligned}
$$

$\line(1,0){360}$

$$
  \begin{aligned}
&  
\mat \bullet &r\\c&c\emat \times\mat\bullet &r\\{r}&& \emat&&\quad\mat \bullet&
r\\c&c\emat \times\mat\bullet& r\\ {c}&&\emat&&\quad\mat \bullet&
r\\c&c\emat\times\mat\bullet& c\\{r}& \emat&&\quad\mat \bullet&
r\\c&c\emat \times\mat\bullet&c\\{c}&\emat\\
&\mat +-\\+-\\-+\\ {-}\quad\emat\cup\mat +-\\+-\\-+\\{+}\quad\emat
&&\quad
\mat +-\\+-\\-+\\ {-}\quad\emat
&&\quad
\mat -+\\-+\\-+\\ {-}\quad\emat\cup\mat -+\\-+\\-+\\ {+}\quad\emat
&&\quad
\mat -+\\-+\\-+\\ {-}\quad\emat
\end{aligned}
$$
\end{example}

\begin{example}
  Let $\CO_c=(2{2})$.  The sizes of rows are not doubled.
  $$
  \begin{aligned}
 &\bullet\times\bullet &&{r}\times {r}&&{r}\times
 {c}&&{c}\times {r}&&c\times {c}\\
 &\mat +-\\-+\emat\qquad &&\mat +-\\+-\emat\qquad&&\mat +-\\-+\emat\cup\mat
 +-\\-+\emat\qquad&&\mat-+\\+-\emat\cup\mat-+\\-+\emat\qquad&&\mat -+\\-+\emat 
  \end{aligned}
  $$
\end{example}
\begin{example}
Let $\CO_c=(442{2})$. We record only the cases when the number of
$+$'s is at least as large as the number or $-'$s in the rows $44.$
For the others flip the signs. Only half the sizes of the rows are listed
$$
\begin{aligned}
&\mat\bullet&\bullet\\\bullet\emat\times\mat\bullet&\bullet\\\bullet\emat\quad
&&\mat\bullet&r\\\bullet\emat\times\mat\bullet&r\\\bullet\emat\quad
&&\mat\bullet&r\\\bullet\emat\times\mat\bullet&c\\\bullet\emat\quad\\
%&&\mat\bullet&c\\\bullet\emat\times\mat\bullet&r\\\bullet\emat\quad
%&&\mat\bullet&c\\\bullet\emat\times\mat\bullet&r\\\bullet\emat
&\mat +-\\-+\\+\quad\\-\quad\emat
&&\mat +-\\+-\\+\quad\\-\quad\emat
&&\mat +-\\+-\\+\quad\\-\quad\emat\cup\mat +-\\-+\\+\quad\\-\quad\emat
%&&\mat +-\\-+\\+\quad\\-\quad\emat\cup\mat -+\\-+\\+\quad\\-\quad\emat
%&&\mat -+\\-+\\+\quad\\-\quad\emat
\end{aligned}
$$
$$
\begin{aligned}
&\mat\bullet&\bullet\\ r\emat\times\mat\bullet&\bullet\\ r\emat\qquad
&&\mat\bullet&\bullet\\ r\emat\times\mat\bullet&\bullet\\ c\emat\qquad
&&\mat\bullet&\bullet\\ c\emat\times\mat\bullet&\bullet\\ r\emat\qquad
&&\mat\bullet&\bullet\\ c\emat\times\mat\bullet&\bullet \\ c\emat\qquad\\
&\mat +-\\-+\\+\quad\\+\quad\emat
&&\mat +-\\-+\\+\quad\\+\quad\emat\cup\mat +-\\-+\\+\quad\\-\quad\emat
&&\mat +-\\-+\\+\quad\\-\quad\emat\cup\mat +-\\-+\\-\quad\\-\quad\emat
&&\mat +-\\-+\\-\quad\\-\quad\emat
\end{aligned}
$$
$$
\begin{aligned}
&\mat\bullet&r\\ r\emat\times\mat\bullet&r\\ r\emat\quad
&&\mat\bullet&r\\ r\emat\times\mat\bullet&c\\ r\emat\quad
&&\mat\bullet&r\\ r\emat\times\mat\bullet&r\\ c\emat\quad
&&\mat\bullet&r\\ r\emat\times\mat\bullet&c \\ c\emat\quad\\
&\mat +-\\+-\\+\quad\\+\quad\emat
&&\mat +-\\-+\\+\quad\\+\quad\emat\cup\mat +-\\+-\\+\quad\\-\quad\emat
&&\mat +-\\+-\\+\quad\\+\quad\emat\cup\mat +-\\+-\\+\quad\\-\quad\emat
&&\mat +-\\-+\\+\quad\\+\quad\emat\cup\mat +-\\+-\\+\quad\\-\quad\emat
\cup\mat +-\\-+\\+\quad\\-\quad\emat\cup\mat +-\\+-\\-\quad\\-\quad\emat
\end{aligned}
$$
$$
\begin{aligned}
&\mat\bullet&r\\ c\emat\times\mat\bullet&r\\ r\emat\qquad
&&\mat\bullet&r\\ c\emat\times\mat\bullet&c\\ r\emat\qquad
&&\mat\bullet&r\\ c\emat\times\mat\bullet&r\\ c\emat\qquad
&&\mat\bullet&r\\ c\emat\times\mat\bullet&c \\ c\emat\qquad\\
&\mat +-\\+-\\-\quad\\-\quad\emat\cup\mat +-\\+-\\-\quad\\+\quad\emat
&&\mat +-\\-+\\-\quad\\-\quad\emat\cup\mat +-\\-+\\-\quad\\+\quad\emat\cup
\mat +-\\+-\\-\quad\\-\quad\emat
&&\mat +-\\+-\\-\quad\\-\quad\emat
&&\mat +-\\-+\\-\quad\\-\quad\emat
\end{aligned}
$$
$$
\begin{aligned}
&\mat r&c\\ r\emat\times\mat  r&c\\ r\emat\qquad
&&\mat r&c\\ r\emat\times\mat r&c\\ c\emat\qquad\\
%&&\mat r&c\\ c\emat\times\mat r&c\\ r\emat\qquad
%&&\mat r&c\\ c\emat\times\mat r&c \\ c\emat\qquad\\
&\mat +-\\+-\\+\quad\\+\quad\emat
&&\mat +-\\+-\\+\quad\\+\quad\emat\cup\mat+-\\+-\\+\quad\\-\quad\emat\cup
\mat+-\\-+\\+\quad\\+\quad\emat
%&&\mat -+\\-+\\+\quad\\-\quad\emat\cup\mat -+\\-+\\-\quad\\-\quad\emat
%&&\mat -+\\-+\\-\quad\\-\quad\emat
\end{aligned}
$$
\end{example}




\subsubsection{} We return to the case when all pairs of columns are odd size.
The largest size row occurs an odd number of tiumes while all others
occur an even number of times. We reduce the computation of $\AV(\pi)$
to that when the largest pair of columns is removed. The edges of 
$\tau_L\times\tau_R$ are
$$
\mat
  \vdots &\vdots\\
  \bullet &\beta\\
  \vdots &      \\
  \star &\\
  \gamma\\
\emat
  \qquad\times\qquad
\mat
  \vdots &\vdots\\
  \bullet &\\
  \vdots &      \\
  y&\\
  &
\emat
\qquad\text{ or }\qquad
\mat
  \vdots &\vdots\\
  r&c\\
  \vdots &      \\
  r&\\
  \gamma\\
\emat
  \qquad\times\qquad
\mat
  \vdots &\vdots\\
  r &\\
  \vdots &      \\
  y&\\
  &
\emat
$$
$\AV(\pi)$ for a tableau on the left comes from the tableau with the
largest columns removed. $\AV(\pi)$ of the  one on the right comes from
the tableau where the largest pair of columns is removed, and the $c$
in the second column of $\tau_L$ is replaced by $y.$ 

More precisely, let $\tau_L'\times \tau_R'$ be a pair of tableaus with
sizes same as $\tau_L\times\tau_R$ with the largest pair removed. Add
pairs of largest columns in all possible ways, reversing the
proicedure just outlined. In particular, move the last element
in $\tau_L'$ to the last element in $\tau_R$ when adding a row
$rc$. In particular, each row of $\AV(\pi')$ is extended by 2. Pairs
of rows of size two are added with signs all possible ways. Whenever
$\mat +-\\-+\emat$ occurs, there are two $\tau_L\times\tau_R$  giving the same
$\AV.$  
\begin{theorem}
When all pairs of columns of $\CO_c$ are odd, there is a one-to-one
correspondence between the local systems of the real nilpotent orbits
and the $\AV(\pi).$

In this theorem, in addition to the factor of 2 coming from the $\mat
+-\\-+\emat$, there is another factor coming from the non-special
tableaus, which also account for parameters.
\end{theorem}
 
\newpage
\section{The $\Theta-$correspondence, $C/D$ and $\wti{C}/B$}
\subsection{Local Systems on Nilpotent Orbits} The nilpotent orbits in
type C all have pairs of equal columns; by assumption, no size occurs
more than twice. In terms of rows, all rows are of even size, and
every even size occurs.  


The orbits in type D all have pairs of equal columns, except for the
smallest and the largest, which each occur once, and are even size. In
terms of rows, all are odd sized, and every odd size occurs.   

For type C, the Jordan blocks are represented by 
$$
\dots \longrightarrow e_i\longrightarrow e_{i+1}\longrightarrow\dots
\longrightarrow \pm f_{i+1}\longrightarrow \pm f_i\longrightarrow
\dots 
$$
The reductive part of the centralizer is a product of  $O(p_i)\times O(q_i)$ where $p_i,q_i$ are the number of rows of size $i$ starting with an $e$ or an $f$ respectively.

For type $B/D,$ the Jordan blocks are all of odd size, and every odd size occurs,
$$
\dots \longrightarrow v_i^+\longrightarrow
w_i^+\longrightarrow\dots\longrightarrow v_0/w_0\longrightarrow\dots\longrightarrow \pm v_i^-\longrightarrow \pm w^-_i\longrightarrow\dots 
$$
The elements $v_0,w_0$ satisfy $|v_0|=|w_0|= 1.$ So there are two types of rows,
those with  $v_0$ in the middle, call them $+$, and those with $w_0$
in the middle, call them $-$.
The component group is as in the case of type $C.$ 

The $\Theta-$correspondence is implemented by \textit{interlacing} the two Jordan forms corresponding to the variety $\C Z\subset\Hom[\C L,V]\oplus\Hom[\C L^\perp, W].$  The component groups match via the centralizers of the joint Jordan forms.


The $\Theta-$correspondence from $\CO_2$ to $\CO_1$ is
\begin{itemize}
\item from type C to a larger type B/D, an odd/even column of size  larger
  than the largest column of $\CO_2$  is added,
\item from type B/D to a larger type C, a column of size larger than or
  equal to  two less than the largest column of $\CO_2$ is added. 
\end{itemize}
 





Write $C(\CO_1,\CO_2)$ and $C(\C O_i)$ for the respective
centralizers. There is a diagram
\begin{equation}
  \label{eq:kernel}
\begin{tikzcd}
&\arrow[dl,"\pi_1" ']C(\CO_1,\CO_2)\arrow[rd,"\pi_2"]&\\
%\arrow[dl,"\pi_1"]&&\arrow[dr,"\pi_2"]\\
C(\CO_1)&& C(\CO_2)
\end{tikzcd}  
\end{equation}
\begin{itemize}
\item $\pi_1$ is always \textit{onto}.
\item The local systems are obtained by pulling back along $\pi_2,$
  and pushing forward along $\pi_1.$ 
  \textit{This MEANS} that for the system on $\CO_2$ to occur,
the pullback along $\pi_2$ must be trivial on the kernel of $\pi_1.$ 
\item The kernel of $\pi_1$ is trivial whenever adding a column of equal or larger size.
\item When the added column is of size two less, the
  $\Theta-$correspondence is from type B/D to type $\wti{C}/C.$ The kernel is
  nontrivial. Assume this is the case; $\C O_2$ has
  $p_1:=p_1(\CO_2)$ rows with $+$ and   $q_1:=q_1(\CO_2)$ rows (of size 1) 
  with $-$. The (reductive part of the) centralizer for the rows of
  size 1 is $O(p_1)\times O(q_1).$ When   $p_1/q_1=0$
(i.e. $p_1$ or $q_1=0$),  there is only one choice for $\C O_1;$ it will have
  $(p_1-1)/(q_1-1)$  rows of size 2 starting with $+/-.$  The local
  system must be trivial on $O(p_1)/O(q_1)$ in order to lift.

  When the
  rows of size 1 have both $\pm,$ say $(p_1,q_1)$ for the rows
  with $(+,-),$ there are two possible $\C O_1$. On the rows of size
  two of $\CO_1$, one has signature   $(p_1,q_1-1)$, call it $\C O_1^+,$ the other $(p_1-1,q_1),$ call it
  $\C O_1^-.$ In other words, extend all rows of $\CO_2$ of size $\ge 2$ by one
  to make a Jordan form, preserving the signs. For the rows of size one, remove a $-$ and
  extend the rest by one preserving the signs to form $\CO_1^+$. For $\CO_1^-$ remove a $+$ and extend
  the rest by one to form a Jordan block of type C.
  In order to lift, the local system has to be trivial on the $O(q_1)$ for
  $\C O_1^+,$  trivial on $O(p_1)$  for $\C O_1^-$ respectively.
  In particular, a representation containing $(\C O_2,\chi_2)$ in its $\AV,$
  will lift to one containing $(\C O_1^+,\chi^+)\sqcup
  (\CO_1^-,\chi^-)$ if and only if the local system $\chi_2$ is
  trivial on  $O(p_1)\times   O(q_1).$ The $\chi^\pm$ are trivial,
  subject to the tensoring rule in the next item.
\item {\clrblu When $p+q$ is odd, there is a \textit{tensoring with
      $Sgn$}. From type C to type D, the change is  \textit{before}
    lifting the local systems. From type D to type C, the change is
    \textit{after} lifting the local system. For type B, $p+q$ is
    always odd.} 
  
\item Tensoring with characters of $O(p,q)$ is done the standard
  way. A Jordan row is centered at a $v_0$ or  $w_0$; the characters
  on the corresponding $O(p_i)$ and $O(q_i)$ get tensored with the
  corresponding characters on $O(p)$ or $O(q)$ respectively. Each
  representation gives rise to ``\textit{four}'' new ones if $p_i\cdot q_i\ne 0$ 
  for some $i,$ ``\textit{two}'' if $p_i\cdot q_i=0$ for all $i.$ 

\end{itemize}
 



\subsection{Even sized columns only, type D} This is the  ``\textit{easy}''
case. The nilpotent orbits are parametrized by the columns of the
Jordan forms.
\begin{enumerate}
\item  See-saw pairs starting with the trivial orbit cover all
  cases.
\item The correspondence follows the Kraft-Procesi model
\item At most one repetition of a column size is allowed. Otherwise
  the moment map allows for representations that have $\CO_c$ strictly
  larger than the minimal orbit mandated by the infinitesimal
  character. The''\textit{first}'' example is $\CO_c=(4,2,2)$ ``\textit{as
  columns}'' ($(3,3,1,1)$ as rows). The infinitesimal character is
  $\chi=(2,1,1,0).$ The unipotent representation has
  ``\textit{columns}'' $(4,4).$
\item $\AV(\pi)$ consists of a single real form of the corresponding
  complex nilpotent orbit.   
\item The number of unipotent representations equals the number of
  local system on the real forms of $\CO_c$. There are twice as many
  representations in $O(a,b)$ than in $SO(a,b)$. Two representations
  of $O(a,b)$ that restrict to the same representation of $SO(a,b)$
  are related by tensoring with $det.$  Using $O(a,b)$ accounts for
  the extra nonspecial representations in the $HC-$cell of type
  $C$ via the $\Theta-$correspondence. The first example is $Sp(4)\times O(4)$ matching
  $\CO_c=(2,2)$ in $sp(4,\bb R)$ with $\CO_c=(2)$ in $O(a,b)$ with   $a+b=2.$ 

\item The algorithm for $\AV(\pi)$ in type D, (unlike type B for
  $So(p,q)$, computes for
  representations of $SO(a,b)$ or $O(a,b)$ with all $(a,b).$ 
\end{enumerate}
\subsubsection{Type D to Type C} Let $\CO_D$ be an orbit in type D, with even sized columns only. The case 
$$
D_{2i-1}>D_{2i-2}=D_{2i-3}>\dots >D_{2}=D_1>D_0=0
$$ 
is the main one.  The $\Theta-$correspondence on the level of orbits (coming from the moment map) is obtained by adding a column of size larger than the largest column of $\CO_D;$ the orbit $\CO_C=\Theta(\CO_D)$  has columns 
$$
\begin{aligned}
 C_{2i}&\ge (C_{2i-1}=D_{2i-1})>(C_{2i-2}=D_{2i-2})=(D_{2i-3}=C_{2i-3})>\dots >\\
&>(C_2=D_2)=(D_1=C_1)>C_0=D_0=0. 
\end{aligned}
$$ 
It is enough to consider the case when   $C_{2i}=D_{2i-1}$. Each
tableau for type $D$ embeds into one of type C by changing the $r'$ in
$\tau_L$ to $\bullet,$ adding a column of size $c_{2i}$ to $\tau_R,$
and adjusting the entries according to type C. This accounts for half
the unipotent representations of $Sp.$ The other half is accounted for
by the fact that the HC-cells have twice as many representations in
type C as in type D; $\AV(\Theta(\pi))$ for these representations
stays the same, but there are twice as many local systems in $O(a,b)$
than in $SO(a,b).$
\begin{example}
Let $\CO_D=(2),$ in type D match $\CO_C=(22)$ in type C.  The HC-cells
are
$$
\{1\times 0\}\mapsto \{1\times 1, 11\times 0\}.
$$
The matchup is
$$
r\times 0\mapsto r\times r,\quad c\times 0\mapsto c\times r,\qquad
c'\times 0\mapsto c'\times r,\qquad r'\times 0\mapsto \bullet\times \bullet,
$$
add an $r$ to the row in $\tau_L$; except the $r'$ is replaced by
$\bullet,$ and the column in $\tau_L$ acquires a $\bullet$ in the
appropriate place.

We group the tableaus in type C to match the HC-cells:
$$
\begin{aligned}
&r\times r\longleftrightarrow \mat r\\r\emat\times \mat\emptyset\\ \emat
\qquad&\qquad
c\times
  r\longleftrightarrow\mat r\\c\emat\times\mat\emptyset\\\emat &\\
 &\bullet\times \bullet\longleftrightarrow
  \mat r\\ c'\emat\times\mat\emptyset\\\emat\qquad &\qquad c'\times r\longleftrightarrow
  \mat c\\ c'\emat\times\mat\emptyset\\\emat&
\end{aligned}
$$
The general rule in the first row is to remove an $r$ from the column
of $\tau_R$ and add it to the column of $\tau_L.$ This does not give a
valid tableau in the second row, so it is modified accordingly. 
\end{example}
\begin{example}
Let $\CO_D=(4220)$ in type D match $(44220)$ in type $C$. The rules in
the previous example are extended to this case. The matchup of HC-cells is 
$$
\{ 21\times 1,\ 22\times \emptyset\}\longrightarrow \{21\times 21,\ 22\times 11,\ 211\times 2,\ 221\times 1\}
$$
There are 16 cells. We pair the tableaus for type D as follows.
$$
\begin{aligned}
&\left\{\mat\bullet&r'\\r'&\emat,\quad \mat
r'&r\\r'&c'\emat\right\}&&\mapsto\left\{\mat\bullet&\bullet\\\bullet
&\emat,\quad\mat\bullet&\bullet\\r&\\c'\emat,\quad\mat\bullet&r\\\bullet&c'\emat,\quad \mat\bullet&r\\r&c'\\c'&\emat\right\}\\
&\left\{\mat\bullet&r'\\r&\emat,\quad \mat r'&r\\r&c'\emat\right\}
&&\mapsto\left\{\mat\bullet&\bullet\\ r&\emat,\quad
\mat\bullet&\bullet\\r&\\ r&\emat,\quad\mat\bullet&r\\ r&c'\emat,\quad \mat\bullet&r\\r&c'\\r&\emat\right\}\\
&\left\{\mat\bullet&r'\\c&\emat,\quad \mat r'&r\\c&c'\emat\right\}
&&\mapsto\left\{\mat\bullet&\bullet\\ c&&\emat,
\quad\mat\bullet&\bullet\\r&\\c'&\emat,\quad\mat\bullet&r\\ c&c'&\emat,\quad \mat\bullet&r\\r&c'\\c&\emat\right\}\\
&\left\{\mat\bullet&r'\\ c'&\emat,\quad \mat r'&r\\c'&c'\emat\right\}
&&\mapsto\left\{\mat\bullet&\bullet\\c'
&\emat,\quad\mat\bullet&\bullet\\r&\\c'\emat,\quad\mat\bullet&r\\c'&c'\emat,\quad
\mat\bullet&r\\r&c'\\c'&\emat\right\}\\
&\dots &&
\end{aligned}
$$
with the remainder following the rules from Example 6 for each pair of
equal columns. The $\bullet$'s in type C removed from $\tau_R$ are replaced by
$r'$ in $\tau_L,$ and only the first and second tableaus for type C
match tableaus in type D.\hfill \qed

\end{example}

We write down the general rule for how to pair special and nonspecial
tableaus in type C.  For a pair of equal columns of size $c_i\times c_i$, remove
an $r$ from the column in $\tau_R,$ and insert it in the column  of
$\tau_L$. The cases $\bullet\times\bullet$ and $c'\times r$ are
exceptions and follow Example 6. Further exceptions when inserting an
$r$ and moving the end of the row down does not result in a tableau are
$$
\begin{aligned}
  &\mat
  \bullet&r\\r&\emat\longleftrightarrow\mat r&c\\ r&c'\emat\qquad &&\qquad  \mat \bullet&r\\ c&\emat\longleftrightarrow\mat r&c\\ c&c'\emat\\
&\mat\bullet&r\\ c'&\emat\longleftrightarrow \mat r& c\\ c'& c'\emat
\qquad&&\qquad
\mat\bullet&c\\ c'& \emat\longleftrightarrow\mat c&c\\ c'&c'\emat
\end{aligned}
$$
{For these exceptions, going from the special to the
  nonspecial tableau to the right,   replace the last $\bullet$ in the
  top row by the element to its   right, and add  $c$ at the end.  Add
  a $c'$ to the row below. The ``\textit{definition}'' of   the
  exception is that ``\textit{extending}'' the second column by an $r$
  does not result in a valid tableau; the exisitng entry being bumped
  one row down. Going the   other
  way, replace the corner by $\bullet,$ and move the entry in the
  corner, which is $r,c$, to the right. Remove the $c'$ below}.  
\subsubsection{Type C to Type D} Let $\CO_C$ be an orbit in type C, with even sized columns only. The case 
$$
C_{2i}=C_{2i-1}>C_{2i-2}=C_{2i-3}>\dots >C_{2}=C_1>C_0=0
$$
 is the main one.  The $\Theta-$correspondence on the level of orbits (coming from the moment map) is obtained by adding a column $D_{2i+1}$ of size larger than the largest column of $\CO_C;$ the orbit $\CO_D=\Theta(\CO_C)$  has columns 
$$
\begin{aligned}
 D_{2i+1}&>D_{2i}=C_{2i}=C_{2i-1}=D_{2i-1}>D_{2i-2}=C_{2i-2}>\dots >\\
&>D_2=C_2=C_1=D_1>D_0=C_0=0. 
\end{aligned}
$$ 
It is enough to consider the case when   $D_{2i+1}=C_{2i}+2$.
The algorithm for $\AV(\pi)$ to $\AV(\Theta(\pi))$ from $Sp(2n,\bR)$ to
$O(a,b)$ is as follows. For the cases when the largest columns of
$\tau$ of type C are $c_{2i-1}\times c_{2i}$, add
$\bullet$ to each row of $\tau_L,$ and a $r',r,c,c'$ for a row of one
at the bottom, depending on $O(a,b)$ that is being lifted to:
$$
\begin{aligned}
  &r\mapsto \mat +\\+\emat,\qquad  &&c\mapsto \mat -\\-\emat \\
  &r'\mapsto \mat +\\-\emat,\qquad  &&c'\mapsto \mat +\\-\emat 
\end{aligned}
$$
Change $\bullet$ to $r'$ where
necessary,  and similarly change the entries in $\tau_R$ to $\bullet$
to form a $\tau$  of type D. For the cases $c_{2i-1}+1\times c_{2i}-1$
follow the rules in blue from the previous section applying
$r'\mapsto\bullet$ where necessary to match non-special with special
tableaus and $\AV(\pi)'$s.



\subsection{Local Systems, even sized columns only}
We assign local systems to the tableaus in type C. Type D follows from
the algorithms and the $\Theta-$correspondence.

Nilpotent orbits were already assigned to the special tableaus.  
{\clrblu Recall that we assume NO columns repeat}.

The general pattern for a  pair of equal columns $c_j\times c_{j}$ and
the corresponding
$c_{j}+1\times c_{j}-1$ is schematically recorded as
$$
\begin{aligned}
&(-)\quad\longleftrightarrow &&r\times r \qquad &&
(+)\quad\longleftrightarrow&&\begin{matrix}
 r\\r
\end{matrix}\times\emptyset\\ 
&(-)\quad\longleftrightarrow &&c\times r\qquad &&(+)\quad\longleftrightarrow&&\begin{matrix}
r\\c
\end{matrix}\times\emptyset  && \\
&\left(\mat -\\ -\emat\right)\quad \longleftrightarrow&&\bullet\times\bullet\qquad &&
\left(\mat -\\ +\emat\right)\quad\longleftrightarrow&&\begin{matrix}
r\\c'
\end{matrix}\times\emptyset\\
&\left(\mat +\\ +\emat\right)\quad\longleftrightarrow &&c'\times r&&
\left(\mat +\\ -\emat\right)\quad\longleftrightarrow &&\begin{matrix}
c\\c'
\end{matrix}\times\emptyset
\end{aligned}
$$
This is the case of $\CO=(22).$ 
In the last two rows, the top sign refers to the row with a $+$; and
$\pm$ refers to trivial/sign.


The column may have a number of $\bullet$ and $r$ which were
omitted. The reductive part of the centralizer for each row size is $O(p_i)\times
O(q_i)$, with $p_i$ is the number of rows with a $+$ and $q_i$ the
number of rows with a $-.$ Recall that a $c'\times r$ gives a pair of
rows $\mat +-\dots\\-+\dots\emat.$ The rules are as follows.

In the cases $c_{j}\times c_{j}$
\begin{itemize}
\item no $c'$ present, only $\bullet$, $r$ and $c$ in the column for
  $\tau_L:$ $\left(\mat-\\-\emat\right).$ 
\item $c'$ present, same as before for the rest of the column:
  $\left(\mat +\\+\emat\right)$.
\end{itemize}

In the cases $c_{j}+1\times c_{j}-1$,
\begin{itemize}
\item no $c,c'$ present: $\left(\mat-\\+\emat\right)$
\item no $c',$ but $c$ present: $\left(\mat +\\-\emat\right)$
\item $c'$ present, no $c$:   $\left(\mat-\\ +\emat\right)$ 
\item $c'$ and $c$ present: $\left(\mat +\\-\emat\right)$  
\end{itemize}
{Tableaus which give rise to the same $\AV$ have
  $\bullet\times\bullet$ replaced by $c'\times r.$}
This provides a 1-1 correspondence between tableaus and local systems.

\newpage


\subsection{Odd sized columns present}
The odd sized columns give rise to pairs $c_j+1\times c_j.$
\begin{enumerate}
\item The same restriction about the size of a column occuring at most
  two times holds.  
\item There is no longer a one-to-one correspondence between local
  systems on real orbits and unipotent representations; some unipotent
  representations have $\AV(\pi)$ a union of real forms.
\item The number of representations does \textit{not double}.
\end{enumerate}
\subsection{Streamlining the Algorithm} Construct the nilpotent orbit
for the special tableau where all the $c'$ coming from odd columns where removed. Then add the $c'$ for the odd sized columns starting from the bottom. If there is a choice of adding to rows starting with different signs, form a union.

\bigskip
We write down an algorithm for the local system by
adding a largest  column.
\subsubsection{Type D to type C} When adding a column
$C=2c$, two less than the largest column of $\CO$ of type D, add
$2c+2$, the size of the largest column instead. 
The representations of $O(a,b)$ correspond to the representations of
\newline $Sp(a+b+2c+2)$ in a one-to-one fashion.
We use $Sp$ to parametrize $O(a,b)$ (different
from $So(a,b)$). We keep only half the tableaus as illustrated in Examples 6 and 7. 
We will  show how to remove a row of size $2$ to match all tableaus
with (some of the) local systems.

\medskip
Assume that the local systems are known for $\pi$ of
type D. Then the local systems for $\Theta(\pi)$ lift in one-to-one
fashion for $Sp(a+b+2c+2)$; the kernel in \eqref{eq:kernel} is trivial
and $\pi_1,\pi_2$ are isomorphisms.

If all the rows of size two have the same sign, there is only one way to
remove one. The local systems on $\CO_D$  trivial on $O(p_1-1)$
or $O(q_1-1)$ respectively,  lift. The tableaus are obtained by adding
a column of $\bullet$ with $r'$s or $r'$ and a $c$ to $\tau_L.$
Tensoring, one gets double the parameters; they match the missing
nonspecial tableaus. The
system trivial on $O(p_1-1)/O(q_1-1)$ also lifts to $Sp(a+b+2c).$
These are the tableaus with one less $r$ added in the largest column
of $\tau_L.$ .

If $\C O_C$  has both signatures in the rows of size 1, there are
three local systems $\left(\mat +\\+\emat\right),$ $\left(\mat
  +\\-\emat\right),$ and $\left(\mat -\\+\emat\right)$ that lift to
$Sp(a+b+2c).$ Take the corresponding lifted $\AV'$s for $Sp(2c+2)$ where the
\textit{tail} has hte \textit{extra} $c'\times r.$ The tableaus with
one $r$ removed from the largest column of $\tau_L$ correspond to
taking out a $+-$, $-+$ or making the union of the two when there is a
$c'$ present.

This accounts for all tableaus with $\AV$  real forms of $\C
O_C;$ they all come by the $\Theta$ correspondence. 

\subsubsection{Type C to type D} We add a column of even size larger
than the largest column of the orbits in $\AV(\pi)$ for $Sp(2n,\bR)$, to get nilpotent
orbits in $\AV(\Theta(\pi))$ for $O(a',b').$  All the local systems
lift; the kernoel of $\pi_1$ is trivial. If $\AV(\pi)$ is a
union of orbits all with the same signature $(a,b)$, then $\AV(\Theta(\pi))$
has rows of size greater than three with total signature $(a,b)$, and
the rows of size one have signature the difference to  $(a',b').$
This is the case when the largest column has even size, but also some
cases with odd largest column. When the
largest column of $\CO_C$ has odd size, then the total signatures on
the nilpotent orbits in the associated variety can be a mix of
$(a+1,b)$ and $(a,b+1).$ Assume this is the case.
If $a'\ge b+1$ and $b'\ge a+1,$ all nilpotent
orbits with total signature $(a',b')$ obtained by the algorithm of
extending each row by one occur. In the cases  when $a'=b,\ b'\ge a$ and  
$a'\ge b,\ b'=a,$ the nilpotent orbits with signature $(a,b-1)$ and
$(a+1,b)$ respectively drop out. The resulting local systems carry over for the
rows of size greater than 1, and are trivial on the component groups
for the rows of size 1. The tableaus of the representations
$\Theta(\pi_i)$ satisfying
$\AV(\Theta(\pi_1))=\AV(\Theta(\pi_2))$ must have $\tau_L$ of the form
$$
\tau_L(\Theta(\pi_1)):\ \mat
\bullet&r\\
\vdots&\vdots \\
\bullet&r,c\\
r&c'\\
r&\\
\vdots& \\
r&\\
r,c&\\
\emat
\qquad\longleftrightarrow\qquad
\tau_L(\Theta(\pi_2)):\ \mat
\bullet&r\\
\vdots&\vdots \\
\bullet&r\\
\bullet&r,c\\
r&\\
\vdots& \\
r&\\
r,c&\\
\emat
$$
with matching pairs $r,c$ in the corresponding columns. The $r,c$
means that $r$ or $c$ can occur in that position. The second
columns in $\tau_R$ are of length one less than those in $\tau_L.$

$\AV(\pi_1)$ contains $\AV(\pi_2)$ strictly. The local systems on the nilpotent orbits are
known/assigned  by induction. The two representations are related by
tensoring with characters of $O(a',b').$ 

{\clrblu The local systems have to be ``shifted'' according to the
  parity $(a'+b')/2$   every time}.

This shows that any unipotent representation of $O(a',b')$ with $\AV$
a union of real forms of $\C O_D$ is obtained by $\Theta-$lifts and
tensoring with characters.

\begin{example}
  Let $\CO_c=(1,1)$ in $Sp(2,\bb R).$ There are three representations,
  corresponding to $r\times\emptyset$, $c\times\emptyset$ and
  $c'\times\emptyset.$ $\AV(\pi)$ are $+-,$ $-+$ and $+-\cup-+.$
  Applying $\Theta$ from  $Sp(2,\bb R)$ to $O(6)$  gives 13
  representations corresponding to $\CO_c=(3,1,1,1).$
  There are however 20 representations for the real forms of $So(6),$
  and 40 for $O(6).$  The representations for $O(6)$ match those for
  $Sp(10,\bb C)$ for $\CO_c=(4,2,2,2).$ We need to tensor with
  characters of $O(p,q)$ to get to 40. 
\end{example}
\begin{example}
  Let $\CO_c=(4,1,1)$ in $O(6).$ Computing $\Theta$ from $O(6)$ to
  $Sp(8)$  gives 17 representations only. The tableaus
  $$
  \mat r'&r\\r'\emat\times \mat \emptyset\\ \emat\qquad
\mat r'&c\\r'\emat\times \mat \emptyset\\ \emat\qquad
\mat r'&c'\\r'\emat\times \mat \emptyset\\ \emat
  $$
  cannot be embedded into $\mat x&x\\ x\emat\times \mat x&\\ \ &
  \emat$ for $Sp(8).$ They do show up for $\Theta$ going to $Sp(2n,\bb
  R)$ for $2n\ge 10.$  In
  that case all 40 representations appear; this matches the fact that
  $\CO_c=(4,2,2,2)$ has $HC-$cells formed of  a special and a nonspecial
  representation each occuring with multiplicity one.
\end{example}
\subsubsection{} Consider the case when a pair of equal
$c_{i}\times c_{i}$ is added to a tableau with last columns
$c_{i-1}\times c_{i-1}+1$ going to type D,
and then type C which also parametrizes the orthogonal
group in type D. By induction, it is assumed that local
systems were attached to the smaller case. All rows larger
than one inherit the local systems from the smaller orbit; increase
all rows of the smaller nilpotent orbit by one to get a nilpotent
orbit of type $D.$  Denote the $+'s$ by $v$ and the $-'$s by $w.$
The last row of size one of $\tau_L$ is 
$$
\begin{aligned}[l]
&\mat r' r\emat\times\mat\emptyset \emat,\quad \mat rc'\emat\times\mat\emptyset \emat
\quad&\longleftrightarrow\quad &\mat v&w&v&\\ v&\emat,\\
&\mat r' c\emat\times\mat\emptyset \emat,\quad \mat cc'\emat\times\mat\emptyset \emat
\quad&\longleftrightarrow\quad &\mat v&w&v\\ v\emat,\\
&\mat r' c'\emat\times\mat\emptyset \emat,\quad \mat c'c'\emat\times\mat\emptyset \emat
\quad&\longleftrightarrow\quad &\mat v&w&v\\ w\emat\sqcup \mat w&v&w\\v\emat,\\
&\mat rc\emat\times\mat\emptyset \emat\quad&\longleftrightarrow\quad &\mat w&v&w\\ v\emat,\\
&\mat cc\emat\times\mat\emptyset \emat\quad&\longleftrightarrow\quad &\mat v&w&v\\ w\emat.
\end{aligned}
$$
Everything is  \textit{modulo the parity shift}.

 There are three pairs of tableaus that contain $c'$,
 paired up with another three pairs with the same $\AV.$ The tableaus
 with $c'$ in the first two rows and $c'c'$ in the third row get $(+)$
 on the rows of size one, and inherit the local  system on the larger
 rows from the representation they came from. 

The last two cases have  tableaus that do not have $c'.$ Assign
$(+)=triv$  to the row of size 1.  
  This is the lift from type C to type D. The missing representations
  for the orthogonal group are obtained by 
tensoring with $\ep\rtimes\eta$. The representations are now
parametrized by the $Sp$ where we added another column equal to the
largest column to $\tau_R$. The pair of largest columns doubles each occurence to
a \textit{``special''} and \textit{``nonspecial''} representation
``\textit{on the row of size 2}''. 
The first three cases get four each, the last two, only two. Tensoring changes
the local system according to the effect $\ep,\eta$ has on the
middle element of each row.  The associated tableaus are according to
$\ep\times\eta=(\pm)(\pm)$:
\begin{itemize}
\item $(+)(+)$ same,
\item $(-)(-)$ switches $\text{special}\longleftrightarrow \text{nonspecial}$,
\item $(+)(-)$ $r'\longleftrightarrow c'$,
\item $(-)(+)$ $r'\longleftrightarrow  c'$ and
      $\text{special}\longleftrightarrow \text{nonspecial}$.
    \end{itemize}
So for the first three cases each tableau gives rise to four, two for the last two cases.
For the last two tableaus, only two  representations are created by
tensoring. 
This accounts for all occurences of unipotent representations in the
$\Theta-$correspondence.
\begin{remark}
In the first two cases, we chose to  lift from the tableaus
$r\times\emptyset$ and $c\times\emptyset$ to 
$r'r\times\emptyset$ and $r'c\times\emptyset$ rather than  $c'\times
0$.This is  because the latter has $\AV$ a union of nilpotents, and some get
``\textit{killed}''  in the lifting. There are coincidences arising
from lifting and tensoring, and we made a specific choice.
\end{remark}


Going up to type C
replaces the $r'$ by $\bullet,$ and adjusts $\tau_R$ accordingly. The
rows of the ensuing nilpotent is obtained by increasing the sizes of all rows in
the nilpotent orbits by one, to form a tableau of type C. The local
systems get lifted. 
\subsubsection{} Assume that $c_{2i-1}\times c_{2i}$ with
$c_{2i}=c_{2i-1}>c_{2i-3}=c_{2i_2}+1.$  There is more than one row of
size one in $\tau_L$. The same rules as before are used, except the
first column of $\tau_L$ is longer, and the nilpotent orbit will have
several $v,w,$ in the row of ones.  As before,
sorting by $\AV,$ there will be cases where tensoring will create four
representations, and cases with only two.

\subsubsection{} Consider the case when $c_{2i-1}-1=c_{2i}\ge
c_{2i-3}-1=c_{2i-2}.$ Use the tableaus of type C with $c_{2i}+1$ to parametrize the
representations in type D, with the local systems already assigned.
$\AV(\pi)$ and the local system is obtained as follows.

\begin{itemize}
\item On the level of $\AV$, subtract a row of size two of the sign for
  which the local system is $(+)$; make a union if both have $(+)$,
  and assign the \textit{restriction} of the local system on the
  remaining rows of size two.
\item On the level of tableaus, those ending in $\bullet$ \textit{vanish}; the
  assignment is designed to attach $\left(\mat -\\-\emat\right)$ to
  the rows of size one. For the remainder of tableaus, only one of the paired
  special/nonspecial tableaus occurs; $\AV$ is obtained from the
  tableau of $Sp(2n+2,\bb C)$  by removing an $r$ from the largest
  column of $\tau_R,$ and  using the algorithm for the smaller $Sp$.  
\end{itemize}




\newpage
\section{The $\Theta-$correspondence, $B/\wti{C}$}
{\clrr Needs more work}.

Some times the
$\Theta-$correspondence has coincidences. Other times it does not seem
to produce enough representations. 

\begin{example}
The orbit $\C O_2=(3,1)$ (columns) in $sp(4,\bb R)$ is obtained from $\C
O_1=(1)$ in $O(1,0)$ and $O(0,1)$ by the $\Theta-$correspondence. There
are exactly four characters and exactly four metaplectic representations.  
\end{example}
\begin{example}
Let $\C O_2=(3,3,1)$ in $so(7)$. It comes from $\C O_1=(3,1)$ in $sp(4).$ There
are two representations for $\C O_2$ and four for $\C O_1.$   
\end{example}
\begin{example}
Let $\C O_2=(5,3,1)$ in $so(9).$ It comes from $\C O_1=(3,1)$. This
time I count $16=2\cdot 8$ for $\C O_2$ ($so(a,b)$ for all $a,b$ not
just $a>b$). There are only four available from $sp(4).$   
\end{example}

\begin{example}
  Let $\C O_2=(3,3)$ in $sp(6)$. It comes from $\C O_1=(3)$ in
  $so(3)$. There are twelve represenations for $\C O_2$ and I think
  there are coincidences in the $\Theta-$lifts, and some characters
  do not lift.
\end{example}
\begin{example}
Let $\C O_2=(5,3,3)$. There are $48$ representations counting all
$O(a,b)$. I only see at most $36$ lifts from $\C O_1=(3,3)$.    
\end{example}


\newpage

%%%%%%%%%%%%%%%%%%%%%%%%%%%%%%%

\begin{thebibliography}{99}
\bibitem[ABV]{ABV}
jeffrey~Adams,  Dan~Barbasch, D.~Vogan, {\em The Langlands
  Classification and Irreducible Characters for Real Reductive
  Groups},  Progress in Mathematics, vol. 104, Birkh\"auser, (1992)


  
\bibitem[BV1]{BV1}
  Dan~Barbasch, D.~Vogan, {\em The local structure of Characters}, J. of Func. Analysis,
  vol. 37, issue 1, (1980) pp. 27-55

\bibitem[BV2]{BV2}
  Dan~Barbasch, D.~Vogan, {\em Unipotent representations of semisimple
  complex groups},
  Annals of Math.,
  vol. 121, issue 1 (1985) pp. 41-110
  
\bibitem[C]{C}
  L.~Casian, {\em Primitive Ideals and Representations }, Journal of Algebra, 101,(1986) pp 497-515

\bibitem[K]{K}
  D.~King, {\em The character polynomial of the annihilator of an
    irreducible Harish-Chandra module} Amer. J. Math. 103 (1981) pp 1195-1240

\bibitem[L]{L}
G.~Lusztig {\em Characters of Reductive Groups over a Finite Field} ,
Ann. of Math. Stud., vol. 107, Princeton  University Press, Princeton, 1984

\bibitem[McG]{McG}
G.~McGovern {\em Cells of Harish-Chandra Modules for Real Classical
  Groups}, American Journal of Mathematics, Vol. 120, No. 1 (Feb.,
1998), pp. 211-228 

\bibitem[RT]{RT}
D.~Renard, P.~Trapa, {\em Irreducible Genuine Characters of the
  Metaplectic Group : Kazhdan-Lusztig Algorithm and Vogan Duality},
Reprsentation Theory, vol. 4, (2000) 245–295 

  
\bibitem[V4]{V4}
D.~Vogan {\em  Irreducible Characters of Semisimple Lie Groups
  IV. Character-Multiplicity Duality}, Duke mathematical Journal,
vol. 49, no. 4, 1982, 943-1073

\end{thebibliography}











\end{document}


\begin{comment}

\bigskip

%%%%%%%%%%%%%%%%%%%%%%%%%%%%%%%%%%%%%%%%%%%%%%%%%%%%%%%%%%%%%%%%%%%%%%%%%

\end{comment}



\bigskip
Adding a pair $x\times x$ to \eqref{eq:10} amounts to
\begin{alignat}{3}\label{3.4.5}
&\text{Tableau }\qquad&\qquad\qquad&\vO\qquad          &\qquad\qquad&\CO\notag \\
\noalign{\medskip
\hrule
\medskip}
&\mat r\\ c\emat\times \mat r\\ \emptyset\emat&&\mat -+\\ -+\\ -+\emat&&
\mat +&-&+\\ +&-&+\\ -&&\emat\cup \mat +&-&+\\ -&+&-\\ +&&\emat\notag \\ 
\noalign{\medskip
\hrule
\medskip}
&\mat r\\ r\emat\times \mat r\\ \emptyset\emat&&\mat +-\\ +-\\ +-\emat&&
\mat +&-&+\\ +&-&+\\ -&&\emat\cup \mat +&-&+\\ -&+&-\\ +&&\emat\notag \\ 
\noalign{\medskip
\hrule
\medskip}
&\mat r\\ c'\emat\times \mat r\\ \emptyset\emat&&\mat +-\\ +-\\ +-\emat\cup\mat +-\\ +-\\ -+\emat &&
\mat +&-&+\\ +&-&+\\ +&&\emat\notag \\
\noalign{\medskip
\hrule
\medskip}
&\mat c\\ c'\emat\times \mat r\\ \emptyset\emat&&\mat -+\\ -+\\ +-\emat\cup\mat -+\\ -+\\ -+\emat &&
\mat +&-&+\\ +&-&+\\ +&&\emat\\
\noalign{\medskip
\hrule
\medskip}
&\mat \bullet\\ r\emat\times \mat \bullet\\ \emptyset\emat&&\mat +-\\ -+\\ +-\emat&&
\mat +&-&+\\ -&+&-\\ +&&\emat\notag \\
\noalign{\medskip
\hrule
\medskip}
&\mat \bullet\\ c\emat\times \mat \bullet\\ \emptyset\emat&&\mat +-\\ -+\\ -+\emat&&
\mat +&-&+\\ -&+&-\\ +&&\emat\notag \\
\noalign{\medskip
\hrule
\medskip}
&\mat \bullet\\ c'\emat\times \mat \bullet\\ \emptyset\emat
&&\mat +-\\ -+\\ +-\emat\cup\mat +-\\ -+\\ -+\emat&&
\mat -&+&-\\ -&+&-\\ +&&\emat \notag \\ 
\noalign{\medskip
\hrule
\medskip}
\notag
\end{alignat}

In general, adding a pair of equal rows is done as follows. Any switch
to make the number of $+$'s greater than the number of $-$'s in type B
is done at the end. 

\noindent\textbf{(A0)} A pair  $\bullet \dots \bullet c\times \bullet \dots
\bullet r$ adds a pair of rows $\mat -+\dots -+\\ -+\dots -+\emat$ to
the nilpotents in $\AV(\chp).$ It amounts to {\it real} induction for
$\AV(\pi)$ if the smallest  rows in the nilpotents in $\AV(\chp)$ have the same
sign. Otherwise use $\theta$-induction with an extra $-$ if there is a choice.

\noindent\textbf{(A1)} A pair  $\bullet \dots \bullet r\times \bullet \dots
\bullet r$ adds a pair of rows $\mat +-\dots +-\\ +-\dots +-\emat$ to
the nilpotents in $\AV(\chp).$  If a $c$ is present in the column of
$\tau_L$ that the $r$ is being added to, then the pair of rows
inherits the sign of the row with the $c.$ It amounts to {\it real}
induction for $\AV(\pi)$ if the smallest rows of the nilpotents in $\AV(\chp)$ have the
same sign. Otherwise use $\theta$-induction with an extra $-$ if there is a choice.

\noindent\textbf{(B0)} A pair  $\bullet \dots \bullet \times \bullet \dots
\bullet$ adds a pair of rows $\mat +-\dots +-\\ -+\dots -+\emat$ to
the nilpotents in $\AV(\chp).$ It amounts to $\theta$ induction for
$\AV(\pi).$ If there is a choice in the signature, use an extra $+.$ 

\noindent\textbf{(C0)} A pair  $\bullet \dots \bullet c'\times \bullet \dots
\bullet r$ adds a pair of rows $\mat +-\dots +-\\ -+\dots -+\emat$ to
the nilpotents in $\AV(\chp).$ It amounts to $\theta$ induction for
$\AV(\pi).$ If there is a choice in the signature, use an extra $-.$ 

\bigskip
Now suppose that the last row in $\tau_L$ is strictly larger than the
corresponding one in $\tau_R.$ Let $\tau'$ be the obtained from $\tau$
by truncating the last row of $\tau_L$ so that $\tau'_L$ and $\tau'_R$
have the largest rows of the same size. We build $\AV(\pi)$ from
$\AV(\pi')$ by describing the effect of adding a series of 
$x\times\emptyset$ to $\AV(\pi').$ For the first one the rules are:

\noindent\textbf{(D0)} A $c\times\eset$ that is added next to $\bullet\times
\bullet$ changes both rows for
$\bullet\dots\bullet\times\bullet\dots\bullet$ in $A(\chp)$ to $-$ and
increases one of them by $2.$
Suppose the result is formed of nilpotent orbits having their two
largest sized rows all of the
same sign. Let $\wti\pi$ be the representation corresponding to the
$\wti\tau$ obtained from $\tau$ by replacing the largest pair of rows
by  $\bullet\dots\bullet r\times\bullet\dots\bullet r.$ Then $\AV(\pi)$
is obtained from $\AV(\wti\pi)$ by adding a pair of rows of size 1 with
signs $\mat -\\ -\emat$ to all its nilpotents.
Otherwise add a pair of rows $\mat +\\ -\emat$ to each nilpotent in $\AV(\pi).$ 

\noindent\textbf{(E0)} If $c\times\eset$ is added next to $r\times r,$  it
increases one of the last two rows by 2 and changes its sign. The
nilpotents in $\AV(\pi)$ get a $\mat +\\-\emat.$

\noindent\textbf{(F0)} An $r\times \eset$ can only be added next to a
$\bullet\times\bullet;$ it changes the sign of $-+\dots$ and adds 2 to
it. For $\AV(\pi)$ the similar rule to (D0) applies. 

\noindent\textbf{(G0)} A $c'\times\eset$ increases a row in $\AV(\chp)$ by
2. If there are largest rows of both signs in a nilpotent, two of
them, one for each sign are added. For $\AV(\pi)$ add a pair of rows
$\mat +\\ +\emat.$  

More $c\times\emptyset$ and $c'\times \emptyset$ have the following effect:

\noindent\textbf{(E1)} Subsequent $c\times\eset$ change the sign and increase
the largest row by 2 in $\AV(\chp)$ and add a pair $\mat+\\ -\emat$ to
the nilpotents in $\AV(\pi)$. 

\noindent\textbf{(G1)} Subsequent $c'\times \eset$ increase the largest row by
2 in $\AV(\chp)$ and add pairs $\mat +\\ +\emat$ to $\AV(\pi).$ 



%%%%%%%%%%%%%%%%%%%%
\begin{comment}
{\clrblu We anticipate the shape of the nilpotent orbit}. When $c_0>0$
there are $c_0$ pairs of rows of size $2r+1$ each starting with a $+$
and a $-.$ \textbf{Assume that $c_0=0.$}
When $\mathbf{c_2>c_1}$, the $\bullet$'s and $c'$ in Cases (3) and (4) 
translate into pairs of rows of size $2r$ starting one with $+$, one
with $-.$ Case (1) corresponds to  pairs of rows starting with $+$. Case (2)
corresponds to $c_0$ pairs of rows starting with $-.$ 

When $\mathbf{c_1=c_2+1},$ there is an extra row of size $2r$. Cases
(1) and (2) acquire an extra row with 
$a+$ or a $-$ respectively. In Cases (3) and (4), there is a union of
nilpotent ortbits, one with an extra row with a $+$, one with an extra
row with  a $-$.    

\begin{theorem}
The number of representations with primitive ideal $\C I_{\CO_c}$ and infinitesimal character $\la_{\C O_c}$ equals the sum of the sizes $|A(\C  O)|.$ 
\end{theorem}

\begin{proof}
We do type C; the other cases are analogous.  

The number of unipotent representations (added over several real forms
of the Li algebra in types B, D) equals the sum of multiplicities of
the $\sig$ correspoinding to the primitive ideal cell $\C I_{\CO}$  in
the coherent continuation representation, counted with the
multiplicity  of the $sgn$ representation of the Weyl group of the
Levi component determined by the centralizer of $\la_{\C O}.$ From
[BV] and elsewhere this is one.  The size of $|A(\C O)|$ is the
product of $2^2$ for even sized rows with signatures both $\pm$, and
$2$ for even sized rows of just one signature. The number of rows of
size $2i$ is $n_{2i}:=2c_{2r-2i+1}-2c_{2r-2i}$. So it contributes
$2+4(n_{2i}-2)+2=4(n_{2i}-1).$ 

\medskip
Now consider the multiplicity of the special representation. Column
$c_0$ can only have $\bullet$ or $r.$ In any case, the rows forming
the top of column $c_{2r}$ to $c_2$ must have $\bullet$'s before the
entries in $c_0$. Consequently this has to be the case for the top
rows in columns  $c_{2r-1}$ to $c_3$. Columns $c_1$ and $c_0$ muat
have the same number of $\bullet$'s, and column $c_0$ can only be
filled $r.$ Let $0\le x\le c_0$ be the number of $\bullet$'s. The
remaining entries can only be filled by  
\begin{enumerate}
\item all $r$
\item all $r$ and one $c$
\item all $r$ and one $c'$
\item all $r$, one $c$ and one $c'.$ 
\end{enumerate}
Cases (1) and (2) with no $\bullet$'s correspond to  signatures all
$+$ and all $-.$ The number of  $\bullet$'s and $c'$ count the pairs
of rows with a $+$ and a $-$. The reamining all $r$, or all $r$ and
one $c$ count the reamining rows all with the same sign, $+$ or $-.$
So the rows with a single sign have been counted once, the rows with
both signs twice; this adds up to $2(n_{2i}-1).$ The other
$2(n_{2i}-1)$ come from the representation with $c_1$ and $c_0$ were
interchanged.    


\end{proof}

\end{comment}
%%%%%%%%%%%%%%%%%%%%%%%
More precisely, the cases of the last two pairs of columns  are
$$
\begin{aligned}
&\mat
\bullet &\bullet&\times &\bullet&\bullet\\
\vdots &\vdots& &\vdots&\vdots\\
\bullet &\bullet & &\bullet&\bullet\\
\bullet& r & &\bullet&r\\
\vdots &\vdots& &\vdots&\vdots\\
\bullet& r & &\bullet&r\\
\bullet&r &&\bullet  &x\\
r&c&& r\\
\vdots &&& \vdots&\\
r&&&y&\\
b&&&&
\\
\emat
&\qquad \text{ and }\qquad\mat
\bullet &\bullet&\times &\bullet&\bullet\\
\vdots &\vdots& &\vdots&\vdots\\
\bullet &\bullet & &\bullet&\bullet\\
\bullet& r & &\bullet&r\\
\vdots &\vdots& &\vdots&\vdots\\
\bullet& r & &\bullet&x\\
\bullet&a &&\bullet  &\\
\vdots &&& \vdots&\\
r&&&r&\\
r&&&y&\\
b&&&&
\emat
\end{aligned}
$$
The rules for the nilpotent orbits are
\begin{enumerate}
\item the rows $\bullet\bullet\times\bullet\bullet$ get pairs of rows $\mat
  +-&\hdots &+-\\ -+&\hdots &-+\emat$
\item the rows $\bullet r\times\bullet r$ get pairs $\mat
  +-&\hdots &+-\\ +-&\hdots &+-\emat$ or $\mat
  -+&\hdots &-+\\ -+&\hdots &-+\emat$ depending on $x=r$ or $x=c.$
\item the row  $\bullet a\times\bullet$ gets $\mat
  +-&\hdots &+-&+-\\ -+&\hdots &-+&\emat$ or $\mat
  -+&\hdots &-+&-+\\ +-&\hdots &+-&\emat$ depending on whether $a=r$
  or $a=c$. The rows $r\times r$ get the sign of $y$.
  \item $rc\times r$ gets $\mat +-&\dots&+-&+-\\+-&\dots&+-&\emat$ if $y=r$ and
    $\mat -+&\dots&-+&+-\\-+&\dots&-+&\emat$ if $y=c.$ The pairs $r\times r$ follow the
    sign of the smaller row.
\item $b\times\emptyset$ gets $+-\dots$ or $-+\dots $ depending on
  $b=r$ or $b=c.$  
\end{enumerate}
\bigskip
In the general case when pairs of equal columns occur, we construct
$\AV$ by induction. Suppose $\AV(\tau_L'\times\tau_R'$  for the
case for smaller size are known. There are several cases. We sort them
out by the edges:


\begin{comment}
\begin{enumerate}
\item $r\times r$ comes from $r\times\emptyset.$ Extend the next
  smaller row   by 2 with the same sign as the larger. Same for  $c\times c$ coming from $c\times\emptyset$.
\item For $r\times c$ or $c\times r$ increase a smaller row by 2 according to
preserving the sign; make a union if there is a choice.  
\end{enumerate}
For example,
\begin{description}
\item[A] $\mat \bullet\\ r\emat\times \mat\bullet\\ r\emat$ gives
  $\mat +-\\-+\\+-\\+-\emat$

  \bigskip
\item[B]
$\mat \bullet\\ r\emat\times \mat\bullet\\ c\emat$ gives $\mat
+-\\-+\\+-\\+-\emat\cup \mat +-\\-+\\+-\\-+\emat$

\bigskip
\item[C] $\mat r\\ r\emat\times \mat r\\ c\emat$ gives $\mat
+-\\+-\\+-\\+-\emat\cup \mat +-\\+-\\+-\\-+\emat$

  
\end{description}
\subsubsection{Triangular Nilpotents} The  \textit{triangular case} is defined as the one where each  row size, even,  occurs exactly twice, except for the largest one which occurs just once. Each matching pair of rows in $\tau_L\times \tau_R$ has sizes $(r_i+1)\times r_i$ in increasing order $r_i<r_{i+1}.$So $\tau_L$ has row sizes $1<2<\dots <k.$ Typical rows are labeled
$$
\mat
\dots &\bullet&\alpha_1&\alpha_2&\times&\bullet&x&\\
\dots &\beta_1&\beta_2&         &\times&y& \\
\dots &\gamma &       &         &      & &
\emat
$$
$\al_1$ and $\beta_1$ can only be $\bullet$ or $r$, while $\beta_2=r$ or $c.$ Similarly, $\al_2=r$ or $c.$ 
\begin{enumerate}
\item when $\beta_1=\bullet,$ $y=\bullet$ as well. If $\al_1=\bullet$ as well, $x=\bullet.$ Pair up $(\beta_2,\al_2)$. The cases are  
  $
  \pmat\beta_2\\ \al_2\epmat=
  \begin{cases}
    \mat r\\ r\emat & \ \mat +-\\-+\emat, \\
    \mat r\\c\emat & \ \mat +-\\+-\emat \\
    \mat c\\ r\emat & \ \mat -+\\-+\emat \\
    \mat c\\ c\emat & \ \mat -+\\+-\emat
  \end{cases}
  $
\item when $\beta_1=\bullet,$ and  $\al_1=r,$ then $\alpha_2=c.$ Pair up $(\beta_2,x).$ The cases are
  $
  \pmat\beta_2\\ x\epmat=
  \begin{cases}
    \mat r\\ r\emat & \ \mat +-\\+-\emat \\
    \mat r\\ c\emat & \ \mat -+\\+-\emat \\
    \mat c\\ r\emat & \ \mat +-\\-+\emat \\
    \mat c\\ c\emat & \ \mat -+\\-+\emat
  \end{cases}
  $
\item when $\beta_1=r,$ then $\beta_2=c$. If $\al_1=\bullet,$ then $x=\bullet$. Pair up $(y,\al_2).$ The cases are
  $
  \pmat y\\ \al_2\epmat=
  \begin{cases}
    \mat r\\ r\emat & \ \mat -+\\+-\emat \\
    \mat r\\ c\emat & \ \mat -+\\+-\emat \\
    \mat c\\ r\emat & \ \mat +-\\+-\emat \\
    \mat c\\ c\emat & \ \mat -+\\-+\emat
  \end{cases}
  $
  \item when $\beta_1=r,$ then $\beta_2=c$. If $\al_1=r,$ then $\al_2=c$. Pair up $(x,y).$ The cases are
  $
  \pmat x\\ y\epmat=
  \begin{cases}
    \mat r\\ r\emat & \ \mat +-\\+-\emat \\
    \mat r\\ c\emat & \ \mat +-\\-+\emat \\
    \mat c\\ r\emat & \ \mat -+\\+-\emat \\
    \mat c\\ c\emat & \ \mat -+\\-+\emat
  \end{cases}
  $  
\end{enumerate}
\end{comment}

\begin{description}
\item[(a)] Remove the $x$, and construct $\AV'.$ If $x=\al,$
  extend a row with the same sign as for $\al$ by two. For $x\ne \al$
form a union of nilpotent orbits with a row extended by two in all
possible ways.
\item[(b)] Extend the rows by 2 and follow the sign of $y.$
\item[(c)] Extend the rows by 2 and add the signs as in the case of a
  single pair of columns.
\item[(d)] Extend the rows by 2 and follow the signs of $x$.
\item[(e)] Same as case (a).
\item[(f)] Same as (b).
\item[(g)] Same as (c).
\item[(h)] Same as (d).
\end{description}

\begin{comment}
Larger rows go up by one. The rows end in $\bullet\times\bullet,$
$r\times r$, $c\times r$ or $c'\times r$ as before.
Apply the same rules with real induction  from
Levi components $gl(2m,\bR)$ for $c\times r$ and $r\times r,$ and  with
$u(m,m)$ if $c'\times r$ is not present, and  $u(m+1,a-1)$ or
$u(m-1,m+1)$ otherwise. 


\bigskip
{\clrr \%\%\%\%\%\%\%\%\%\%\%\%\%\% }

\bigskip
Now consider the case when there are pairs of columns in $\tau_L\times
\tau_R$ of unequal length; differing by 1. The same holds for the rows; the difference
between the corresponding rows in $\tau_L$ and $\tau_R$ is $\le 1.$
If there is a single row  of size 2, start with
\begin{alignat}{3}
&\text{Tableau }\qquad     &\qquad\qquad&\CO\subset
Sp\qquad&\qquad\qquad&\vO\subset So\notag\\
\noalign{\smallskip}
&c\times \emptyset           &&-+ &&\mat +\\ +\\ -\emat\notag\\ 
\noalign{\smallskip}
&r\times \emptyset            &&+- &&\mat +\\ +\\ -\emat\label{eq:11}\\ 
\noalign{\smallskip}
&c'\times \emptyset          &&+-\cup -+ &&\mat +\\ +\\ +\emat\notag
\end{alignat}
If there are  more pairs of rows of size 2, move $c'\times r$ right
below the $\bullet\times\bullet'$s. first. The last row is $r\times\emptyset$ or $c\times
\emptyset.$ They get assigned as in the table above. If there is no
$c'\times r$, the $\bullet\times\bullet$
correspond to $\theta-$induction from a $u(m,m)$. If yes, use
$\theta-$induction with a $u(m+1,m-1)$ or $u(m-1,m+1);$ there is only
one choice.  
\end{comment}


The nilpotent orbit corresponding to the case when all
rows end in $c,r,$  has nilpotent orbit with signature $+$ for the
largest row, and pairs $\pm$ for all pairs of equal rows.
Adding pairs of rows ending in
$r,c$ corresponds to $\rho-$induction in $\vO.$ Adding pairs ending in
$\bullet/c'$ corresponds to $\theta-$induction following the previous
case of rows of only size 2.

\begin{comment}
The coherent continuation representation of type
$B$ is obtained from type $C$ by tensoring with $sgn.$ So we use the
coherent continuation representation of type $C$ with its associated
$\C O,$ and compute the corresponding $\vO$ in type B. Recall from the
previous section that we only need to compute the cases when both $\C O$
and $\vO$ are even. 

The labelling of $\tau_R$ is completely determined by its shape
together with 
the labelling of $\tau_L,$ so we often only give $\tau_L$.
Consider first the case where each size row occurs exactly once in
$\tau_L,$ say $(m,m-1,\dots ,1),$ and the corresponding rows in
$\tau_R$ are equal, or just one less than the corresponding ones for
$\tau_L.$  We start with a single $+$.

If the first pair is $x\times x,$ then
\begin{alignat}{4}
&\text{Tableau }\qquad     &\qquad&\CO\subset
Sp\qquad&\qquad&\vO\subset So&Induced\notag\\
\noalign{\vspace{0.1in}}
&c\times r            &&\mat -+\\ -+\emat&&\mat -&+&-\\ +&\\
-&&\emat&& gl(2,\bR),\notag\\ 
\noalign{\vspace{0.1in}}
&r\times r            &&\mat +-\\ +-\emat&&\mat -&+&-\\ +&&\\
-&&\emat && gl(2,\bR),\label{eq:11}\\ 
\noalign{\vspace{0.1in}}
&c'\times r           &&\mat +-\\ -+\emat&&\mat -&+&-\\ -&&\\ -&&\emat
&& u(0,2),\notag\\
\noalign{\vspace{0.1in}}
&\bullet\times \bullet&&\mat +-\\ -+\emat&&\mat -&+&-\\ +&&\\ +&&\emat
&& u(1,1),\notag
\end{alignat}
In case $x\times\emptyset,$
\begin{alignat}{3}
&\text{Tableau }\qquad     &\qquad&\CO\subset
Sp\qquad&\qquad&\vO\subset So\notag\\
&c\times \emptyset &&\mat-+\emat&&\mat +\\-\\ +\emat\notag\\
\noalign{\vspace{0.2in}}
&r\times \emptyset &&\mat+-\emat&&\mat +\\-\\ +\emat\label{eq:12}\\
\noalign{\vspace{0.2in}}
&c'\times \emptyset &&\mat-+\emat\cup \mat +-\emat&&\mat +\\+\\ +\emat\notag
\end{alignat}
The general rules are
\begin{itemize}
\item Adding equal pairs of rows amounts to $\rho-$induction if the pair ends
  in $c\times r$ or $r\times r.$
\item  Adding equal pairs of rows ending in $\bullet\times\bullet$ amounts  to
  $\theta-$induction with one extra sign same as the largest row.
\item Adding equal pairs of rows ending in $c'\times r$ amounts  to
  $\theta-$induction with one extra sign opposite to  the largest row.
\item When adding an unequal pair of rows, use the same rules as before with the last
entry in $\tau_L$  removed. Then adding $c\times\emptyset,\
r\times\emptyset$ amounts to adding a $\mat +\\-\emat$. Adding 
$c'\times\emptyset$ amounts to adding a $\mat -\\-\emat.$
An exception occurs when the row in $\tau_L$ ends in $r$ or $c,$ and the last/previous
column end in the same. In that case, compute
the tableau with the $r,c$ removed, and with one $\bullet$ in the
previous column replaced by $r$. Then add a $\mat +\\+\emat$ at the bottom of the tableau.
\end{itemize}
The general case reduces to this by applying the rules above to the
tableau where all duplicate pairs of rows of the same size ending in $\bullet\times\bullet$ and
$c\times r,\ r\times r$ are removed. Then add the pairs with
$\bullet\times\bullet$ as $\theta-$ induction and  $c\times r,\
r\times r$ as $\rho-$induction.

One can flip all the signs to get a nilpotent orbit in $So(p,q)$ with  $p>q.$

\end{comment}
