% !TeX program = xelatex
\documentclass[12pt,a4paper]{amsart}
\usepackage[margin=2.5cm,marginpar=2cm]{geometry}

\usepackage[bookmarksopen,bookmarksdepth=2,hidelinks,colorlinks=false]{hyperref}
\usepackage[nameinlink]{cleveref}

% \usepackage[color]{showkeys}
% \makeatletter
%   \SK@def\Cref#1{\SK@\SK@@ref{#1}\SK@Cref{#1}}%
% \makeatother
%% FONTS

\usepackage{amssymb}
%\usepackage{amsmath}
\usepackage{mathrsfs}
\usepackage{mathtools}
%\usepackage{amsrefs}
\usepackage{mathbbol,mathabx}
\usepackage{amsthm}
\usepackage{graphicx}
\usepackage{braket}
%\usepackage[pointedenum]{paralist}
%\usepackage{paralist}


\usepackage{amsrefs}

\usepackage[all,cmtip]{xy}
\usepackage{rotating}
\usepackage{leftidx}
%\usepackage{arydshln}

%\DeclareSymbolFont{bbold}{U}{bbold}{m}{n}
%\DeclareSymbolFontAlphabet{\mathbbold}{bbold}


%\usepackage[dvipdfx,rgb,table]{xcolor}
\usepackage[rgb,table]{xcolor}
%\usepackage{mathrsfs}

\setcounter{tocdepth}{1}
\setcounter{secnumdepth}{2}

%\usepackage[abbrev,shortalphabetic]{amsrefs}


\usepackage[normalem]{ulem}

% circled number
\usepackage{pifont}
\makeatletter
\newcommand*{\circnuma}[1]{%
  \ifnum#1<1 %
    \@ctrerr
  \else
    \ifnum#1>20 %
      \@ctrerr
    \else
      \mbox{\ding{\numexpr 171+(#1)\relax}}%
     \fi
  \fi
}
\makeatother

\usepackage[centertableaux]{ytableau}


% Ytableau tweak
\makeatletter
\pgfkeys{/ytableau/options,
  noframe/.default = false,
  noframe/.is choice,
  noframe/true/.code = {%
    \global\let\vrule@YT=\vrule@none@YT
    \global\let\hrule@YT=\hrule@none@YT
  },
  noframe/false/.code = {%
    \global\let\vrule@YT=\vrule@normal@YT
    \global\let\hrule@YT=\hrule@normal@YT
  },
  noframe/on/.style = {noframe/true},
  noframe/off/.style = {noframe/false},
}

\def\hrule@enon@YT{%
  \hrule width  \dimexpr \boxdim@YT + \fboxrule *2 \relax
  height 0pt
}
\def\vrule@enon@YT{%
  \vrule height \dimexpr  \boxdim@YT + \fboxrule\relax
     width \fboxrule
}

\def\enon{\omit\enon@YT}
\newcommand{\enon@YT}[2][clear]{%
  \def\thisboxcolor@YT{#1}%
  \let\hrule@YT=\hrule@enon@YT
  \let\vrule@YT=\vrule@enon@YT
  \startbox@@YT#2\endbox@YT
  \nullfont
}

\makeatother
%\ytableausetup{noframe=on,smalltableaux}
\ytableausetup{noframe=off,boxsize=1.3em}
\let\ytb=\ytableaushort

\newcommand{\tytb}[1]{{\tiny\ytb{#1}}}


%\usepackage[mathlines,pagewise]{lineno}
%\linenumbers

\usepackage{enumitem}
%% Enumitem
\newlist{enumC}{enumerate}{1} % Conditions in Lemma/Theorem/Prop
\setlist[enumC,1]{label=(\alph*),wide,ref=(\alph*)}
\crefname{enumCi}{condition}{conditions}
\Crefname{enumCi}{Condition}{Conditions}
\newlist{enumT}{enumerate}{3} % "Theorem"=conclusions in Lemma/Theorem/Prop
\setlist[enumT]{label=(\roman*),wide}
\setlist[enumT,1]{label=(\roman*),wide}
\setlist[enumT,2]{label=(\alph*),ref ={(\roman{enumTi}.\alph*)}}
\setlist[enumT,3]{label=(\arabic*), ref ={(\roman{enumTi}.\alph{enumTii}.\alph*)}}
\crefname{enumTi}{}{}
\Crefname{enumTi}{Item}{Items}
\crefname{enumTii}{}{}
\Crefname{enumTii}{Item}{Items}
\crefname{enumTiii}{}{}
\Crefname{enumTiii}{Item}{Items}
\newlist{enumPF}{enumerate}{3}
\setlist[enumPF]{label=(\alph*),wide}
\setlist[enumPF,1]{label=(\roman*),wide}
\setlist[enumPF,2]{label=(\alph*)}
\setlist[enumPF,3]{label=\arabic*).}
\newlist{enumS}{enumerate}{3} % Statement outside Lemma/Theorem/Prop
\setlist[enumS]{label=\roman*)}
\setlist[enumS,1]{label=\roman*)}
\setlist[enumS,2]{label=\alph*)}
\setlist[enumS,3]{label=\arabic*.}
\newlist{enumI}{enumerate}{3} % items
\setlist[enumI,1]{label=\roman*),leftmargin=*}
\setlist[enumI,2]{label=\alph*), leftmargin=*}
\setlist[enumI,3]{label=\arabic*), leftmargin=*}
\newlist{enumIL}{enumerate*}{1} % inline enum
\setlist*[enumIL]{label=\roman*)}
\newlist{enumR}{enumerate}{1} % remarks
\setlist[enumR]{label=\arabic*.,wide,labelwidth=!, labelindent=0pt}
\crefname{enumRi}{remark}{remarks}

\crefname{equation}{}{}
\Crefname{equation}{Equation}{Equations}
\Crefname{lem}{Lemma}{Lemma}
\Crefname{thm}{Theorem}{Theorem}

\newlist{des}{description}{1}
\setlist[des]{font=\sffamily\bfseries}

% editing macros.
\blendcolors{!80!black}
\long\def\okay#1{\ifcsname highlightokay\endcsname
{\color{red} #1}
\else
{#1}
\fi
}
\long\def\editc#1{{\color{red} #1}}
\long\def\mjj#1{{{\color{blue}#1}}}
\long\def\mjjr#1{{\color{red} (#1)}}
\long\def\mjjd#1#2{{\color{blue} #1 \sout{#2}}}
\def\mjjb{\color{blue}}
\def\mjje{\color{black}}
\def\mjjcb{\color{green!50!black}}
\def\mjjce{\color{black}}

\long\def\sun#1{{{\color{cyan}#1}}}
\long\def\sund#1#2{{\color{cyan}#1  \sout{#2}}}
\long\def\mv#1{{{\color{red} {\bf move to a proper place:} #1}}}
\long\def\delete#1{}

%\reversemarginpar
\newcommand{\lokec}[1]{\marginpar{\color{blue}\tiny #1 \mbox{--loke}}}
\newcommand{\mjjc}[1]{\marginpar{\color{green}\tiny #1 \mbox{--ma}}}

\newcommand{\trivial}[2][]{\if\relax\detokenize{#1}\relax
  {%\hfill\break
   % \begin{minipage}{\textwidth}
      \color{orange} \vspace{0em} $[$  #2 $]$
  %\end{minipage}
  %\break
      \color{black}
  }
  \else
\ifx#1h
\ifcsname showtrivial\endcsname
{%\hfill\break
 % \begin{minipage}{\textwidth}
    \color{orange} \vspace{0em}  $[$ #2 $]$
%\end{minipage}
%\break
    \color{black}
}
\fi
\else {\red Wrong argument!} \fi
\fi
}

\newcommand{\byhide}[2][]{\if\relax\detokenize{#1}\relax
{\color{orange} \vspace{0em} Plan to delete:  #2}
\else
\ifx#1h\relax\fi
\fi
}



\newcommand{\Rank}{\mathrm{rk}}
\newcommand{\cqq}{\mathscr{D}}
\newcommand{\rsym}{\mathrm{sym}}
\newcommand{\rskew}{\mathrm{skew}}
\newcommand{\fraksp}{\mathfrak{sp}}
\newcommand{\frakso}{\mathfrak{so}}
\newcommand{\frakm}{\mathfrak{m}}
\newcommand{\frakp}{\mathfrak{p}}
\newcommand{\pr}{\mathrm{pr}}
\newcommand{\rhopst}{\rho'^*}
\newcommand{\Rad}{\mathrm{Rad}}
\newcommand{\Res}{\mathrm{Res}}
\newcommand{\Hol}{\mathrm{Hol}}
\newcommand{\AC}{\mathrm{AC}}
%\newcommand{\AS}{\mathrm{AS}}
\newcommand{\WF}{\mathrm{WF}}
\newcommand{\AV}{\mathrm{AV}}
\newcommand{\AVC}{\mathrm{AV}_\bC}
\newcommand{\VC}{\mathrm{V}_\bC}
\newcommand{\bfv}{\mathbf{v}}
\newcommand{\depth}{\mathrm{depth}}
\newcommand{\wtM}{\widetilde{M}}
\newcommand{\wtMone}{{\widetilde{M}^{(1,1)}}}

\newcommand{\nullpp}{N(\fpp'^*)}
\newcommand{\nullp}{N(\fpp^*)}
%\newcommand{\Aut}{\mathrm{Aut}}

\def\mstar{{\medstar}}


\newcommand{\bfone}{\mathbf{1}}
\newcommand{\piSigma}{\pi_\Sigma}
\newcommand{\piSigmap}{\pi'_\Sigma}


\newcommand{\sfVprime}{\mathsf{V}^\prime}
\newcommand{\sfVdprime}{\mathsf{V}^{\prime \prime}}
\newcommand{\gminusone}{\mathfrak{g}_{-\frac{1}{m}}}

\newcommand{\eva}{\mathrm{eva}}

% \newcommand\iso{\xrightarrow{
%    \,\smash{\raisebox{-0.65ex}{\ensuremath{\scriptstyle\sim}}}\,}}

\def\Ueven{{U_{\rm{even}}}}
\def\Uodd{{U_{\rm{odd}}}}
\def\ttau{\tilde{\tau}}
\def\Wcp{W}
\def\Kur{{K^{\mathrm{u}}}}

\def\Im{\operatorname{Im}}

\providecommand{\bcN}{{\overline{\cN}}}



\makeatletter

\def\gen#1{\left\langle
    #1
      \right\rangle}
\makeatother

\makeatletter
\def\inn#1#2{\left\langle
      \def\ta{#1}\def\tb{#2}
      \ifx\ta\@empty{\;} \else {\ta}\fi ,
      \ifx\tb\@empty{\;} \else {\tb}\fi
      \right\rangle}
\def\binn#1#2{\left\lAngle
      \def\ta{#1}\def\tb{#2}
      \ifx\ta\@empty{\;} \else {\ta}\fi ,
      \ifx\tb\@empty{\;} \else {\tb}\fi
      \right\rAngle}
\makeatother

\makeatletter
\def\binn#1#2{\overline{\inn{#1}{#2}}}
\makeatother


\def\innwi#1#2{\inn{#1}{#2}_{W_i}}
\def\innw#1#2{\inn{#1}{#2}_{\bfW}}
\def\innv#1#2{\inn{#1}{#2}_{\bfV}}
\def\innbfv#1#2{\inn{#1}{#2}_{\bfV}}
\def\innvi#1#2{\inn{#1}{#2}_{V_i}}
\def\innvp#1#2{\inn{#1}{#2}_{\bfV'}}
\def\innp#1#2{\inn{#1}{#2}'}

% choose one of then
\def\simrightarrow{\iso}
\def\surj{\twoheadrightarrow}
%\def\simrightarrow{\xrightarrow{\sim}}

\newcommand\iso{\xrightarrow{
   \,\smash{\raisebox{-0.65ex}{\ensuremath{\scriptstyle\sim}}}\,}}

\newcommand\riso{\xleftarrow{
   \,\smash{\raisebox{-0.65ex}{\ensuremath{\scriptstyle\sim}}}\,}}









\usepackage{xparse}
\def\usecsname#1{\csname #1\endcsname}
\def\useLetter#1{#1}
\def\usedbletter#1{#1#1}

% \def\useCSf#1{\csname f#1\endcsname}

\ExplSyntaxOn

\def\mydefcirc#1#2#3{\expandafter\def\csname
  circ#3{#1}\endcsname{{}^\circ {#2{#1}}}}
\def\mydefvec#1#2#3{\expandafter\def\csname
  vec#3{#1}\endcsname{\vec{#2{#1}}}}
\def\mydefdot#1#2#3{\expandafter\def\csname
  dot#3{#1}\endcsname{\dot{#2{#1}}}}

\def\mydefacute#1#2#3{\expandafter\def\csname a#3{#1}\endcsname{\acute{#2{#1}}}}
\def\mydefbr#1#2#3{\expandafter\def\csname br#3{#1}\endcsname{\breve{#2{#1}}}}
\def\mydefbar#1#2#3{\expandafter\def\csname bar#3{#1}\endcsname{\bar{#2{#1}}}}
\def\mydefhat#1#2#3{\expandafter\def\csname hat#3{#1}\endcsname{\hat{#2{#1}}}}
\def\mydefwh#1#2#3{\expandafter\def\csname wh#3{#1}\endcsname{\widehat{#2{#1}}}}
\def\mydeft#1#2#3{\expandafter\def\csname t#3{#1}\endcsname{\tilde{#2{#1}}}}
\def\mydefu#1#2#3{\expandafter\def\csname u#3{#1}\endcsname{\underline{#2{#1}}}}
\def\mydefr#1#2#3{\expandafter\def\csname r#3{#1}\endcsname{\mathrm{#2{#1}}}}
\def\mydefb#1#2#3{\expandafter\def\csname b#3{#1}\endcsname{\mathbb{#2{#1}}}}
\def\mydefwt#1#2#3{\expandafter\def\csname wt#3{#1}\endcsname{\widetilde{#2{#1}}}}
%\def\mydeff#1#2#3{\expandafter\def\csname f#3{#1}\endcsname{\mathfrak{#2{#1}}}}
\def\mydefbf#1#2#3{\expandafter\def\csname bf#3{#1}\endcsname{\mathbf{#2{#1}}}}
\def\mydefc#1#2#3{\expandafter\def\csname c#3{#1}\endcsname{\mathcal{#2{#1}}}}
\def\mydefsf#1#2#3{\expandafter\def\csname sf#3{#1}\endcsname{\mathsf{#2{#1}}}}
\def\mydefs#1#2#3{\expandafter\def\csname s#3{#1}\endcsname{\mathscr{#2{#1}}}}
\def\mydefcks#1#2#3{\expandafter\def\csname cks#3{#1}\endcsname{{\check{
        \csname s#2{#1}\endcsname}}}}
\def\mydefckc#1#2#3{\expandafter\def\csname ckc#3{#1}\endcsname{{\check{
      \csname c#2{#1}\endcsname}}}}
\def\mydefck#1#2#3{\expandafter\def\csname ck#3{#1}\endcsname{{\check{#2{#1}}}}}

\cs_new:Npn \mydeff #1#2#3 {\cs_new:cpn {f#3{#1}} {\mathfrak{#2{#1}}}}

\cs_new:Npn \doGreek #1
{
  \clist_map_inline:nn {alpha,beta,gamma,Gamma,delta,Delta,epsilon,varepsilon,zeta,eta,theta,vartheta,Theta,iota,kappa,lambda,Lambda,mu,nu,xi,Xi,pi,Pi,rho,sigma,varsigma,Sigma,tau,upsilon,Upsilon,phi,varphi,Phi,chi,psi,Psi,omega,Omega,tG} {#1{##1}{\usecsname}{\useLetter}}
}

\cs_new:Npn \doSymbols #1
{
  \clist_map_inline:nn {otimes,boxtimes} {#1{##1}{\usecsname}{\useLetter}}
}

\cs_new:Npn \doAtZ #1
{
  \clist_map_inline:nn {A,B,C,D,E,F,G,H,I,J,K,L,M,N,O,P,Q,R,S,T,U,V,W,X,Y,Z} {#1{##1}{\useLetter}{\useLetter}}
}

\cs_new:Npn \doatz #1
{
  \clist_map_inline:nn {a,b,c,d,e,f,g,h,i,j,k,l,m,n,o,p,q,r,s,t,u,v,w,x,y,z} {#1{##1}{\useLetter}{\usedbletter}}
}

\cs_new:Npn \doallAtZ
{
\clist_map_inline:nn {mydefsf,mydeft,mydefu,mydefwh,mydefhat,mydefr,mydefwt,mydeff,mydefb,mydefbf,mydefc,mydefs,mydefck,mydefcks,mydefckc,mydefbar,mydefvec,mydefcirc,mydefdot,mydefbr,mydefacute} {\doAtZ{\csname ##1\endcsname}}
}

\cs_new:Npn \doallatz
{
\clist_map_inline:nn {mydefsf,mydeft,mydefu,mydefwh,mydefhat,mydefr,mydefwt,mydeff,mydefb,mydefbf,mydefc,mydefs,mydefck,mydefbar,mydefvec,mydefdot,mydefbr,mydefacute} {\doatz{\csname ##1\endcsname}}
}

\cs_new:Npn \doallGreek
{
\clist_map_inline:nn {mydefck,mydefwt,mydeft,mydefwh,mydefbar,mydefu,mydefvec,mydefcirc,mydefdot,mydefbr,mydefacute} {\doGreek{\csname ##1\endcsname}}
}

\cs_new:Npn \doallSymbols
{
\clist_map_inline:nn {mydefck,mydefwt,mydeft,mydefwh,mydefbar,mydefu,mydefvec,mydefcirc,mydefdot} {\doSymbols{\csname ##1\endcsname}}
}



\cs_new:Npn \doGroups #1
{
  \clist_map_inline:nn {GL,Sp,rO,rU,fgl,fsp,foo,fuu,fkk,fuu,ufkk,uK} {#1{##1}{\usecsname}{\useLetter}}
}

\cs_new:Npn \doallGroups
{
\clist_map_inline:nn {mydeft,mydefu,mydefwh,mydefhat,mydefwt,mydefck,mydefbar} {\doGroups{\csname ##1\endcsname}}
}


\cs_new:Npn \decsyms #1
{
\clist_map_inline:nn {#1} {\expandafter\DeclareMathOperator\csname ##1\endcsname{##1}}
}

\decsyms{Mp,id,SL,Sp,SU,SO,GO,GSO,GU,GSp,PGL,Pic,Lie,Mat,Ker,Hom,Ext,Ind,reg,res,inv,Isom,Det,Tr,Norm,Sym,Span,Stab,Spec,PGSp,PSL,tr,Ad,Br,Ch,Cent,End,Aut,Dvi,Frob,Gal,GL,Gr,DO,ur,vol,ab,Nil,Supp,rank,Sign}

\def\abs#1{\left|{#1}\right|}
\def\norm#1{{\left\|{#1}\right\|}}


% \NewDocumentCommand\inn{m m}{
% \left\langle
% \IfValueTF{#1}{#1}{000}
% ,
% \IfValueTF{#2}{#2}{000}
% \right\rangle
% }
\NewDocumentCommand\cent{o m }{
  \IfValueTF{#1}{
    \mathop{Z}_{#1}{(#2)}}
  {\mathop{Z}{(#2)}}
}


\def\fsl{\mathfrak{sl}}
\def\fsp{\mathfrak{sp}}


%\def\cent#1#2{{\mathrm{Z}_{#1}({#2})}}


\doallAtZ
\doallatz
\doallGreek
\doallGroups
\doallSymbols
\ExplSyntaxOff


% \usepackage{geometry,amsthm,graphics,tabularx,amssymb,shapepar}
% \usepackage{amscd}
% \usepackage{mathrsfs}


\usepackage{diagbox}
% Update the information and uncomment if AMS is not the copyright
% holder.
%\copyrightinfo{2006}{American Mathematical Society}
%\usepackage{nicematrix}
\usepackage{arydshln}

\usepackage{tikz}
\usetikzlibrary{matrix,arrows,positioning,cd,backgrounds}
\usetikzlibrary{decorations.pathmorphing,decorations.pathreplacing}

\usepackage{upgreek}

\usepackage{listings}
\lstset{
    basicstyle=\ttfamily\tiny,
    keywordstyle=\color{black},
    commentstyle=\color{white}, % white comments
    stringstyle=\ttfamily, % typewriter type for strings
    showstringspaces=false,
    breaklines=true,
    emph={Output},emphstyle=\color{blue},
} 

\newcommand{\BA}{{\mathbb{A}}}
%\newcommand{\BB}{{\mathbb {B}}}
\newcommand{\BC}{{\mathbb {C}}}
\newcommand{\BD}{{\mathbb {D}}}
\newcommand{\BE}{{\mathbb {E}}}
\newcommand{\BF}{{\mathbb {F}}}
\newcommand{\BG}{{\mathbb {G}}}
\newcommand{\BH}{{\mathbb {H}}}
\newcommand{\BI}{{\mathbb {I}}}
\newcommand{\BJ}{{\mathbb {J}}}
\newcommand{\BK}{{\mathbb {U}}}
\newcommand{\BL}{{\mathbb {L}}}
\newcommand{\BM}{{\mathbb {M}}}
\newcommand{\BN}{{\mathbb {N}}}
\newcommand{\BO}{{\mathbb {O}}}
\newcommand{\BP}{{\mathbb {P}}}
\newcommand{\BQ}{{\mathbb {Q}}}
\newcommand{\BR}{{\mathbb {R}}}
\newcommand{\BS}{{\mathbb {S}}}
\newcommand{\BT}{{\mathbb {T}}}
\newcommand{\BU}{{\mathbb {U}}}
\newcommand{\BV}{{\mathbb {V}}}
\newcommand{\BW}{{\mathbb {W}}}
\newcommand{\BX}{{\mathbb {X}}}
\newcommand{\BY}{{\mathbb {Y}}}
\newcommand{\BZ}{{\mathbb {Z}}}
\newcommand{\Bk}{{\mathbf {k}}}

\newcommand{\CA}{{\mathcal {A}}}
\newcommand{\CB}{{\mathcal {B}}}
\newcommand{\CC}{{\mathcal {C}}}

\newcommand{\CE}{{\mathcal {E}}}
\newcommand{\CF}{{\mathcal {F}}}
\newcommand{\CG}{{\mathcal {G}}}
\newcommand{\CH}{{\mathcal {H}}}
\newcommand{\CI}{{\mathcal {I}}}
\newcommand{\CJ}{{\mathcal {J}}}
\newcommand{\CK}{{\mathcal {K}}}
\newcommand{\CL}{{\mathcal {L}}}
\newcommand{\CM}{{\mathcal {M}}}
\newcommand{\CN}{{\mathcal {N}}}
\newcommand{\CO}{{\mathcal {O}}}
\newcommand{\CP}{{\mathcal {P}}}
\newcommand{\CQ}{{\mathcal {Q}}}
\newcommand{\CR}{{\mathcal {R}}}
\newcommand{\CS}{{\mathcal {S}}}
\newcommand{\CT}{{\mathcal {T}}}
\newcommand{\CU}{{\mathcal {U}}}
\newcommand{\CV}{{\mathcal {V}}}
\newcommand{\CW}{{\mathcal {W}}}
\newcommand{\CX}{{\mathcal {X}}}
\newcommand{\CY}{{\mathcal {Y}}}
\newcommand{\CZ}{{\mathcal {Z}}}


\newcommand{\RA}{{\mathrm {A}}}
\newcommand{\RB}{{\mathrm {B}}}
\newcommand{\RC}{{\mathrm {C}}}
\newcommand{\RD}{{\mathrm {D}}}
\newcommand{\RE}{{\mathrm {E}}}
\newcommand{\RF}{{\mathrm {F}}}
\newcommand{\RG}{{\mathrm {G}}}
\newcommand{\RH}{{\mathrm {H}}}
\newcommand{\RI}{{\mathrm {I}}}
\newcommand{\RJ}{{\mathrm {J}}}
\newcommand{\RK}{{\mathrm {K}}}
\newcommand{\RL}{{\mathrm {L}}}
\newcommand{\RM}{{\mathrm {M}}}
\newcommand{\RN}{{\mathrm {N}}}
\newcommand{\RO}{{\mathrm {O}}}
\newcommand{\RP}{{\mathrm {P}}}
\newcommand{\RQ}{{\mathrm {Q}}}
%\newcommand{\RR}{{\mathrm {R}}}
\newcommand{\RS}{{\mathrm {S}}}
\newcommand{\RT}{{\mathrm {T}}}
\newcommand{\RU}{{\mathrm {U}}}
\newcommand{\RV}{{\mathrm {V}}}
\newcommand{\RW}{{\mathrm {W}}}
\newcommand{\RX}{{\mathrm {X}}}
\newcommand{\RY}{{\mathrm {Y}}}
\newcommand{\RZ}{{\mathrm {Z}}}

\DeclareMathOperator{\absNorm}{\mathfrak{N}}
\DeclareMathOperator{\Ann}{Ann}
\DeclareMathOperator{\LAnn}{L-Ann}
\DeclareMathOperator{\RAnn}{R-Ann}
\DeclareMathOperator{\ind}{ind}
%\DeclareMathOperator{\Ind}{Ind}



\newcommand{\cod}{{\mathrm{cod}}}
\newcommand{\cont}{{\mathrm{cont}}}
\newcommand{\cl}{{\mathrm{cl}}}
\newcommand{\cusp}{{\mathrm{cusp}}}

\newcommand{\disc}{{\mathrm{disc}}}
\renewcommand{\div}{{\mathrm{div}}}



\newcommand{\Gm}{{\mathbb{G}_m}}



\newcommand{\I}{{\mathrm{I}}}

\newcommand{\Jac}{{\mathrm{Jac}}}
\newcommand{\PM}{{\mathrm{PM}}}


\newcommand{\new}{{\mathrm{new}}}
\newcommand{\NS}{{\mathrm{NS}}}
\newcommand{\N}{{\mathrm{N}}}

\newcommand{\ord}{{\mathrm{ord}}}

%\newcommand{\rank}{{\mathrm{rank}}}

\newcommand{\rk}{{\mathrm{k}}}
\newcommand{\rr}{{\mathrm{r}}}
\newcommand{\rh}{{\mathrm{h}}}

\newcommand{\Sel}{{\mathrm{Sel}}}
\newcommand{\Sim}{{\mathrm{Sim}}}

\newcommand{\wt}{\widetilde}
\newcommand{\wh}{\widehat}
\newcommand{\pp}{\frac{\partial\bar\partial}{\pi i}}
\newcommand{\pair}[1]{\langle {#1} \rangle}
\newcommand{\wpair}[1]{\left\{{#1}\right\}}
\newcommand{\intn}[1]{\left( {#1} \right)}
\newcommand{\sfrac}[2]{\left( \frac {#1}{#2}\right)}
\newcommand{\ds}{\displaystyle}
\newcommand{\ov}{\overline}
\newcommand{\incl}{\hookrightarrow}
\newcommand{\lra}{\longrightarrow}
\newcommand{\imp}{\Longrightarrow}
%\newcommand{\lto}{\longmapsto}
\newcommand{\bs}{\backslash}

\newcommand{\cover}[1]{\widetilde{#1}}

\renewcommand{\vsp}{{\vspace{0.2in}}}

\newcommand{\Norma}{\operatorname{N}}
\newcommand{\Ima}{\operatorname{Im}}
\newcommand{\con}{\textit{C}}
\newcommand{\gr}{\operatorname{gr}}
\newcommand{\ad}{\operatorname{ad}}
\newcommand{\der}{\operatorname{der}}
\newcommand{\dif}{\operatorname{d}\!}
\newcommand{\pro}{\operatorname{pro}}
\newcommand{\Ev}{\operatorname{Ev}}
% \renewcommand{\span}{\operatorname{span}} \span is an innernal command.
%\newcommand{\degree}{\operatorname{deg}}
\newcommand{\Invf}{\operatorname{Invf}}
\newcommand{\Inv}{\operatorname{Inv}}
\newcommand{\slt}{\operatorname{SL}_2(\mathbb{R})}
%\newcommand{\temp}{\operatorname{temp}}
%\newcommand{\otop}{\operatorname{top}}
\renewcommand{\small}{\operatorname{small}}
\newcommand{\HC}{\operatorname{HC}}
\newcommand{\lef}{\operatorname{left}}
\newcommand{\righ}{\operatorname{right}}
\newcommand{\Diff}{\operatorname{DO}}
\newcommand{\diag}{\operatorname{diag}}
\newcommand{\sh}{\varsigma}
\newcommand{\sch}{\operatorname{sch}}
%\newcommand{\oleft}{\operatorname{left}}
%\newcommand{\oright}{\operatorname{right}}
\newcommand{\open}{\operatorname{open}}
\newcommand{\sgn}{\operatorname{sgn}}
\newcommand{\triv}{\operatorname{triv}}
\newcommand{\Sh}{\operatorname{Sh}}
\newcommand{\oN}{\operatorname{N}}

\newcommand{\oc}{\operatorname{c}}
\newcommand{\od}{\operatorname{d}}
\newcommand{\os}{\operatorname{s}}
\newcommand{\ol}{\operatorname{l}}
\newcommand{\oL}{\operatorname{L}}
\newcommand{\oJ}{\operatorname{J}}
\newcommand{\oH}{\operatorname{H}}
\newcommand{\oO}{\operatorname{O}}
\newcommand{\oS}{\operatorname{S}}
\newcommand{\oR}{\operatorname{R}}
\newcommand{\oT}{\operatorname{T}}
%\newcommand{\rU}{\operatorname{U}}
\newcommand{\oZ}{\operatorname{Z}}
\newcommand{\oD}{\textit{D}}
\newcommand{\oW}{\textit{W}}
\newcommand{\oE}{\operatorname{E}}
\newcommand{\oP}{\operatorname{P}}
\newcommand{\PD}{\operatorname{PD}}
\newcommand{\oU}{\operatorname{U}}

\newcommand{\g}{\mathfrak g}
\newcommand{\gC}{{\mathfrak g}_{\C}}
\renewcommand{\k}{\mathfrak k}
\newcommand{\h}{\mathfrak h}
\newcommand{\p}{\mathfrak p}
%\newcommand{\q}{\mathfrak q}
\renewcommand{\a}{\mathfrak a}
\renewcommand{\b}{\mathfrak b}
\renewcommand{\c}{\mathfrak c}
\newcommand{\n}{\mathfrak n}
\renewcommand{\u}{\mathfrak u}
\renewcommand{\v}{\mathfrak v}
\newcommand{\e}{\mathfrak e}
\newcommand{\f}{\mathfrak f}
\renewcommand{\l}{\mathfrak l}
\renewcommand{\t}{\mathfrak t}
\newcommand{\s}{\mathfrak s}
\renewcommand{\r}{\mathfrak r}
\renewcommand{\o}{\mathfrak o}
\newcommand{\m}{\mathfrak m}
\newcommand{\z}{\mathfrak z}
%\renewcommand{\sl}{\mathfrak s \mathfrak l}
\newcommand{\gl}{\mathfrak g \mathfrak l}


\newcommand{\re}{\mathrm e}

\renewcommand{\rk}{\mathrm k}

\newcommand{\Z}{\mathbb{Z}}
\DeclareDocumentCommand{\C}{}{\mathbb{C}}
\newcommand{\R}{\mathbb R}
\newcommand{\Q}{\mathbb Q}
\renewcommand{\H}{\mathbb{H}}
%\newcommand{\N}{\mathbb{N}}
\newcommand{\K}{\mathbb{K}}
%\renewcommand{\S}{\mathbf S}
\newcommand{\M}{\mathbf{M}}
\newcommand{\A}{\mathbb{A}}
\newcommand{\B}{\mathbf{B}}
%\renewcommand{\G}{\mathbf{G}}
\newcommand{\V}{\mathbf{V}}
\newcommand{\W}{\mathbf{W}}
\newcommand{\F}{\mathbf{F}}
\newcommand{\E}{\mathbf{E}}
%\newcommand{\J}{\mathbf{J}}
\renewcommand{\H}{\mathbf{H}}
\newcommand{\X}{\mathbf{X}}
\newcommand{\Y}{\mathbf{Y}}
%\newcommand{\RR}{\mathcal R}
\newcommand{\FF}{\mathcal F}
%\newcommand{\BB}{\mathcal B}
\newcommand{\HH}{\mathcal H}
%\newcommand{\UU}{\mathcal U}
%\newcommand{\MM}{\mathcal M}
%\newcommand{\CC}{\mathcal C}
%\newcommand{\DD}{\mathcal D}
\def\eDD{\mathrm{d}^{e}}
\def\DD{\nabla}
\def\DDc{\boldsymbol{\nabla}}
\def\gDD{\nabla^{\mathrm{gen}}}
\def\gDDc{\boldsymbol{\nabla}^{\mathrm{gen}}}
%\newcommand{\OO}{\mathcal O}
%\newcommand{\ZZ}{\mathcal Z}
\newcommand{\ve}{{\vee}}
\newcommand{\aut}{\mathcal A}
\newcommand{\ii}{\mathbf{i}}
\newcommand{\jj}{\mathbf{j}}
\newcommand{\kk}{\mathbf{k}}

\newcommand{\la}{\langle}
\newcommand{\ra}{\rangle}
\newcommand{\bp}{\bigskip}
\newcommand{\be}{\begin {equation}}
\newcommand{\ee}{\end {equation}}

\newcommand{\LRleq}{\stackrel{LR}{\leq}}

\numberwithin{equation}{section}


\def\flushl#1{\ifmmode\makebox[0pt][l]{${#1}$}\else\makebox[0pt][l]{#1}\fi}
\def\flushr#1{\ifmmode\makebox[0pt][r]{${#1}$}\else\makebox[0pt][r]{#1}\fi}
\def\flushmr#1{\makebox[0pt][r]{${#1}$}}


%\theoremstyle{Theorem}
% \newtheorem*{thmM}{Main Theorem}
% \crefformat{thmM}{main theorem}
% \Crefformat{thmM}{Main Theorem}
\newtheorem*{thm*}{Theorem}
\newtheorem{thm}{Theorem}[section]
\newtheorem{thml}[thm]{Theorem}
\newtheorem{lem}[thm]{Lemma}
\newtheorem{obs}[thm]{Observation}
\newtheorem{lemt}[thm]{Lemma}
\newtheorem*{lem*}{Lemma}
\newtheorem{whyp}[thm]{Working Hypothesis}
\newtheorem{prop}[thm]{Proposition}
\newtheorem{prpt}[thm]{Proposition}
\newtheorem{prpl}[thm]{Proposition}
\newtheorem{cor}[thm]{Corollary}
%\newtheorem*{prop*}{Proposition}
\newtheorem{claim}{Claim}
\newtheorem*{claim*}{Claim}
%\theoremstyle{definition}
\newtheorem{defn}[thm]{Definition}
\newtheorem{dfnl}[thm]{Definition}
\newtheorem*{IndH}{Induction Hypothesis}

\newtheorem*{eg*}{Example}
\newtheorem{eg}[thm]{Example}

\theoremstyle{remark}
\newtheorem*{remark}{Remark}
\newtheorem*{remarks}{Remarks}


\def\cpc{\sigma}
\def\ccJ{\epsilon\dotepsilon}
\def\ccL{c_L}

\def\wtbfK{\widetilde{\bfK}}
%\def\abfV{\acute{\bfV}}
\def\AbfV{\acute{\bfV}}
%\def\afgg{\acute{\fgg}}
%\def\abfG{\acute{\bfG}}
\def\abfV{\bfV'}
\def\afgg{\fgg'}
\def\abfG{\bfG'}

\def\half{{\tfrac{1}{2}}}
\def\ihalf{{\tfrac{\mathbf i}{2}}}
\def\slt{\fsl_2(\bC)}
\def\sltr{\fsl_2(\bR)}

% \def\Jslt{{J_{\fslt}}}
% \def\Lslt{{L_{\fslt}}}
\def\slee{{
\begin{pmatrix}
 0 & 1\\
 0 & 0
\end{pmatrix}
}}
\def\slff{{
\begin{pmatrix}
 0 & 0\\
 1 & 0
\end{pmatrix}
}}\def\slhh{{
\begin{pmatrix}
 1 & 0\\
 0 & -1
\end{pmatrix}
}}
\def\sleei{{
\begin{pmatrix}
 0 & i\\
 0 & 0
\end{pmatrix}
}}
\def\slxx{{\begin{pmatrix}
-\ihalf & \half\\
\phantom{-}\half & \ihalf
\end{pmatrix}}}
% \def\slxx{{\begin{pmatrix}
% -\sqrt{-1}/2 & 1/2\\
% 1/2 & \sqrt{-1}/2
% \end{pmatrix}}}
\def\slyy{{\begin{pmatrix}
\ihalf & \half\\
\half & -\ihalf
\end{pmatrix}}}
\def\slxxi{{\begin{pmatrix}
+\half & -\ihalf\\
-\ihalf & -\half
\end{pmatrix}}}
\def\slH{{\begin{pmatrix}
   0   & -\mathbf i\\
\mathbf i & 0
\end{pmatrix}}
}

\ExplSyntaxOn
\clist_map_inline:nn {J,L,C,X,Y,H,c,e,f,h,}{
  \expandafter\def\csname #1slt\endcsname{{\mathring{#1}}}}
\ExplSyntaxOff


\def\Mop{\fT}

\def\fggJ{\fgg_J}
\def\fggJp{\fgg'_{J'}}

\def\NilGC{\Nil_{\bfG}(\fgg)}
\def\NilGCp{\Nil_{\bfG'}(\fgg')}
\def\Nilgp{\Nil_{\fgg'_{J'}}}
\def\Nilg{\Nil_{\fgg_{J}}}
%\def\NilP'{\Nil_{\fpp'}}
\def\nNil{\Nil^{\mathrm n}}
\def\eNil{\Nil^{\mathrm e}}


\NewDocumentCommand{\NilP}{t'}{
\IfBooleanTF{#1}{\Nil_{\fpp'}}{\Nil_\fpp}
}

\def\KS{\mathsf{KS}}
\def\MM{\bfM}
\def\MMP{M}

\NewDocumentCommand{\KTW}{o g}{
  \IfValueTF{#2}{
    \left.\varsigma_{\IfValueT{#1}{#1}}\right|_{#2}}{
    \varsigma_{\IfValueT{#1}{#1}}}
}
\def\IST{\rho}
\def\tIST{\trho}

\NewDocumentCommand{\CHI}{o g}{
  \IfValueTF{#1}{
    {\chi}_{\left[#1\right]}}{
    \IfValueTF{#2}{
      {\chi}_{\left(#2\right)}}{
      {\chi}}
  }
}
\NewDocumentCommand{\PR}{g}{
  \IfValueTF{#1}{
    \mathop{\pr}_{\left(#1\right)}}{
    \mathop{\pr}}
}
\NewDocumentCommand{\XX}{g}{
  \IfValueTF{#1}{
    {\cX}_{\left(#1\right)}}{
    {\cX}}
}
\NewDocumentCommand{\PP}{g}{
  \IfValueTF{#1}{
    {\fpp}_{\left(#1\right)}}{
    {\fpp}}
}
\NewDocumentCommand{\LL}{g}{
  \IfValueTF{#1}{
    {\bfL}_{\left(#1\right)}}{
    {\bfL}}
}
\NewDocumentCommand{\ZZ}{g}{
  \IfValueTF{#1}{
    {\cZ}_{\left(#1\right)}}{
    {\cZ}}
}

\NewDocumentCommand{\WW}{g}{
  \IfValueTF{#1}{
    {\bfW}_{\left(#1\right)}}{
    {\bfW}}
}




\def\gpi{\wp}
\NewDocumentCommand\KK{g}{
\IfValueTF{#1}{K_{(#1)}}{K}}
% \NewDocumentCommand\OO{g}{
% \IfValueTF{#1}{\cO_{(#1)}}{K}}
\NewDocumentCommand\XXo{d()}{
\IfValueTF{#1}{\cX^\circ_{(#1)}}{\cX^\circ}}
\def\bfWo{\bfW^\circ}
\def\bfWoo{\bfW^{\circ \circ}}
\def\bfWg{\bfW^{\mathrm{gen}}}
\def\Xg{\cX^{\mathrm{gen}}}
\def\Xo{\cX^\circ}
\def\Xoo{\cX^{\circ \circ}}
\def\fppo{\fpp^\circ}
\def\fggo{\fgg^\circ}
\NewDocumentCommand\ZZo{g}{
\IfValueTF{#1}{\cZ^\circ_{(#1)}}{\cZ^\circ}}

% \ExplSyntaxOn
% \NewDocumentCommand{\bcO}{t' E{^_}{{}{}}}{
%   \overline{\cO\sb{\use_ii:nn#2}\IfBooleanTF{#1}{^{'\use_i:nn#2}}{^{\use_i:nn#2}}
%   }
% }
% \ExplSyntaxOff

\NewDocumentCommand{\bcO}{t'}{
  \overline{\cO\IfBooleanT{#1}{'}}}

\NewDocumentCommand{\oliftc}{g}{
\IfValueTF{#1}{\boldsymbol{\vartheta} (#1)}{\boldsymbol{\vartheta}}
}
\NewDocumentCommand{\oliftr}{g}{
\IfValueTF{#1}{\vartheta_\bR(#1)}{\vartheta_\bR}
}
\NewDocumentCommand{\olift}{g}{
\IfValueTF{#1}{\vartheta(#1)}{\vartheta}
}
% \NewDocumentCommand{\dliftv}{g}{
% \IfValueTF{#1}{\ckvartheta(#1)}{\ckvartheta}
% }
\def\dliftv{\vartheta}
\NewDocumentCommand{\tlift}{g}{
\IfValueTF{#1}{\wtvartheta(#1)}{\wtvartheta}
}

\def\slift{\cL}

\def\BB{\bB}


\def\thetaO#1{\vartheta\left(#1\right)}

\def\bbThetav{\check{\mathbbold{\Theta}}}
\def\Thetav{\check{\Theta}}
\def\thetav{\check{\theta}}

\DeclareDocumentCommand{\NN}{g}{
\IfValueTF{#1}{\fN(#1)}{\fN}
}
\DeclareDocumentCommand{\RR}{m m}{
\fR({#1},{#2})
}

%\DeclareMathOperator*{\sign}{Sign}

\NewDocumentCommand{\lsign}{m}{
{}^l\mathrm{Sign}(#1)
}



\NewDocumentCommand\lnn{t+ t- g}{
  \IfBooleanTF{#1}{{}^l n^+\IfValueT{#3}{(#3)}}{
    \IfBooleanTF{#2}{{}^l n^-\IfValueT{#3}{(#3)}}{}
  }
}


% Fancy bcO, support feature \bcO'^a_b = \overline{\cO'^a_b}
\makeatletter
\def\bcO{\def\O@@{\cO}\@ifnextchar'\@Op\@Onp}
\def\@Opnext{\@ifnextchar^\@Opsp\@Opnsp}
\def\@Op{\afterassignment\@Opnext\let\scratch=}
\def\@Opnsp{\def\O@@{\cO'}\@Otsb}
\def\@Onp{\@ifnextchar^\@Onpsp\@Otsb}
\def\@Opsp^#1{\def\O@@{\cO'^{#1}}\@Otsb}
\def\@Onpsp^#1{\def\O@@{\cO^{#1}}\@Otsb}
\def\@Otsb{\@ifnextchar_\@Osb{\@Ofinalnsb}}
\def\@Osb_#1{\overline{\O@@_{#1}}}
\def\@Ofinalnsb{\overline{\O@@}}

% Fancy \command: \command`#1 will translate to {}^{#1}\bfV, i.e. superscript on the
% lift conner.

\def\defpcmd#1{
  \def\nn@tmp{#1}
  \def\nn@np@tmp{@np@#1}
  \expandafter\let\csname\nn@np@tmp\expandafter\endcsname \csname\nn@tmp\endcsname
  \expandafter\def\csname @pp@#1\endcsname`##1{{}^{##1}{\csname @np@#1\endcsname}}
  \expandafter\def\csname #1\endcsname{\,\@ifnextchar`{\csname
      @pp@#1\endcsname}{\csname @np@#1\endcsname}}
}

% \def\defppcmd#1{
% \expandafter\NewDocumentCommand{\csname #1\endcsname}{##1 }{}
% }



\defpcmd{bfV}
\def\KK{\bfK}\defpcmd{KK}
\defpcmd{bfG}
\def\A{\!A}\defpcmd{A}
\def\K{\!K}\defpcmd{K}
\def\G{G}\defpcmd{G}
\def\J{\!J}\defpcmd{J}
\def\L{\!L}\defpcmd{L}
\def\eps{\epsilon}\defpcmd{eps}
\def\pp{p}\defpcmd{pp}
\defpcmd{wtK}
\makeatother

\def\fggR{\fgg_\bR}
\def\rmtop{{\mathrm{top}}}
\def\dimo{\dim^\circ}

\NewDocumentCommand\LW{g}{
\IfValueTF{#1}{L_{W_{#1}}}{L_{W}}}
%\def\LW#1{L_{W_{#1}}}
\def\JW#1{J_{W_{#1}}}

\def\floor#1{{\lfloor #1 \rfloor}}

\def\KSP{K}
\def\UU{\rU}
\def\UUC{\rU_\bC}
\def\tUUC{\widetilde{\rU}_\bC}
\def\OmegabfW{\Omega_{\bfW}}


\def\BB{\bB}


\def\thetaO#1{\vartheta\left(#1\right)}

\def\Thetav{\check{\Theta}}
\def\thetav{\check{\theta}}

\def\Thetab{\bar{\Theta}}

\def\cKaod{\cK^{\mathrm{aod}}}

\DeclareMathOperator{\sspan}{span}


\def\sp{{\mathrm{sp}}}

\def\bfLz{\bfL_0}
\def\sOpe{\sO^\perp}
\def\sOpeR{\sO^\perp_\bR}
\def\sOR{\sO_\bR}

\def\ZX{\cZ_{X}}
\def\gdliftv{\vartheta}
\def\gdlift{\vartheta^{\mathrm{gen}}}
\def\bcOp{\overline{\cO'}}
\def\bsO{\overline{\sO}}
\def\bsOp{\overline{\sO'}}
\def\bfVpe{\bfV^\perp}
\def\bfEz{\bfE_0}
\def\bfVn{\bfV^-}
\def\bfEzp{\bfE'_0}

\def\totimes{\widehat{\otimes}}
\def\dotbfV{\dot{\bfV}}

\def\aod{\mathrm{aod}}
\def\unip{\mathrm{unip}}

\def\Piunip{\Pi^{\mathrm{unip}}}
\def\cf{\emph{cf.} }
\def\Groth{\mathrm{Groth}}
\def\Irr{\mathrm{Irr}}

\def\edrc{\mathrm{DRC}^{\mathrm e}}
\def\drc{\mathrm{DRC}}
\def\LS{\mathrm{LS}}
\def\Unip{\mathrm{Unip}}


% Ytableau tweak
\makeatletter
\pgfkeys{/ytableau/options,
  noframe/.default = false,
  noframe/.is choice,
  noframe/true/.code = {%
    \global\let\vrule@YT=\vrule@none@YT
    \global\let\hrule@YT=\hrule@none@YT
  },
  noframe/false/.code = {%
    \global\let\vrule@YT=\vrule@normal@YT
    \global\let\hrule@YT=\hrule@normal@YT
  },
  noframe/on/.style = {noframe/true},
  noframe/off/.style = {noframe/false},
}
\makeatother 


\def\wAV{\AV^{\mathrm{weak}}}
\def\ckG{\check{G}}
\def\ckGc{\check{G}_{\bC}}
\def\dBV{d_{\mathrm{BV}}}
\def\CP{\mathsf{CP}}
\def\YD{\mathsf{YD}}
\def\SYD{\mathsf{SYD}}
\def\DD{\nabla}

\def\lamck{\lambda_\ckcO}
\def\lamckb{\lambda_{\ckcO_b}}
\def\lamckg{\lambda_{\ckcO_g}}
\def\Wint#1{W_{[#1]}}
\def\CLam{\Coh_{\Lambda}}
\def\Cint#1{\Coh_{[#1]}}
\def\PP#1{\mathrm{PP}_{#1}}
\def\BOX#1{\mathrm{Box}(#1)}
\DeclareDocumentCommand{\bigtimes}{}{\mathop{\scalebox{1.7}{$\times$}}}


\def\Gc{G_\bC}
\def\hha{{}^a\fhh}
\def\aX{{}^aX}
\def\aQ{{}^aQ}
\def\aP{{}^aP}
\def\aR{{}^aR}
\def\aRp{{}^aR^+}
\def\asRp{{}^a \Delta^+}
\def\Gfin{\cG(\Gc)}
\def\PiGfin{\Pi_{\mathrm{fin}}( \Gc )}
\def\PiGlfin{\Pi_{\Lambda_0}( \Gc )}
\def\adGfin{\cG_{\mathrm{ad}}(\Gc)}
\def\Ggk{\cG(\fgg,K)}
\def\WT#1{\Delta(F)}
\def\WG{W(\Gc)}
\def\ch{\mathrm{ch}\,}
\def\Wlam{W_{[\lambda]}}
\def\aLam{a_{\Lambda}}
\def\WLam{W_{\Lambda}}
\def\WLamck{W_{[\lambda_{\ckcO}]}}
\def\Wlamck{W_{\lamck}}
\def\Rlam{R_{[\lambda]}}
\def\RLam{R_\Lambda}
\def\RLamp{R_\Lambda^+}
\def\Rplam{R^+(\lambda)}
\def\Glfin{\cG_{\Lambda}(\Gc)}
\def\LC{{}^L\sC}
\def\ckLC{{}^L\check{\sC}}

\def\Con{\sfC}
\def\bCon{\overline{\sfC}}
\def\Re{\mathrm{Re}}
\def\Im{\mathrm{Im}}
\def\AND{\quad \text{and} \quad}
\def\Coh{\mathrm{Coh}}
\def\Cohlm{\Coh_{\Lambda}(\cM)}
\def\ev#1{{\mathrm{ev}_{#1}}}

\def\ppp{\times}
\def\mmm{\slash}


\def\cuprow{{\stackrel{r}{\sqcup}}}
\def\cupcol{{\stackrel{c}{\sqcup}}}

\def\Spr{\mathrm{Springer}}
\def\Prim{\mathrm{Prim}}


\def\CQ{\overline{\sfA}}% Lusztig's canonical quotient
\def\CPP{\mathfrak{P}}


\def\ceil#1{\lceil #1 \rceil}
\def\symb#1#2{{\left(\substack{{#1}\\{#2}}\right)}}
\def\cboxs#1{\mbox{\scalebox{0.25}{\ytb{\ ,\vdots,\vdots,\ }}}_{#1}}

\begin{document}


\title[]{Counting special unipotent representations of real classical groups}

\author [D. Barbasch] {Dan M. Barbasch}
\address{the Department of Mathematics\\
  310 Malott Hall, Cornell University, Ithaca, New York 14853 }
\email{dmb14@cornell.edu}

\author [J.-J. Ma] {Jia-jun Ma}
\address{School of Mathematical Sciences\\
  Shanghai Jiao Tong University\\
  800 Dongchuan Road, Shanghai, 200240, China} \email{hoxide@sjtu.edu.cn}

\author [B. Sun] {Binyong Sun}
% MCM, HCMS, HLM, CEMS, UCAS,
\address{Academy of Mathematics and Systems Science\\
  Chinese Academy of Sciences\\
  Beijing, 100190, China} \email{sun@math.ac.cn}

\author [C.-B. Zhu] {Chen-Bo Zhu}
\address{Department of Mathematics\\
  National University of Singapore\\
  10 Lower Kent Ridge Road, Singapore 119076} \email{matzhucb@nus.edu.sg}




\subjclass[2000]{22E45, 22E46} \keywords{orbit method, unitary dual, unipotent
  representation, classical group, theta lifting, moment map}

% \thanks{Supported by NSFC Grant 11222101}
\maketitle


\tableofcontents
\section{Introduction}

% and $\mathfrak h$
% be the abstract Cartan subalgebra of $\mathfrak g$.
%

Barbasch-Vogan \cite{BVUni,BV.W} developed a formula  counting the number of
special unipotent representations.

In this paper, we will calculate the number of special unipotent representations
for real classical groups explicitly.


\section{Counting formula}

Let $G$ be a real reductive group in the Harish-Chandra class.
Fixing a Cartan subalgebra $\mathfrak h$ in $\mathfrak g$,
we identify the set of infinitesimal character with $\mathfrak h^{*}/W$
where $W$ is the Weyl group acting on $\mathfrak h$.
Let $Q$ be the root lattice and write $[\mu]:= \mu+Q$ for the coset
$\mathfrak h^{*}/Q$ containing $\mu$.
Denote
$\mathrm{Coh}_{[\mu]}(G)$ the space of coherent families based on $[\mu]$ which has a $W_{[\mu]}$-action.

\begin{thm}[Barbasch-Vogan]\label{thm:count}
  For a complex nilpotent orbit $\cO$,
  define
  \[
    S_{\cO} = \left\{\sigma \in \widehat{W_{[\mu]}}|
      \Spr(j_{W_{[\mu]}}^{W} \sigma_{s}) = \cO
    \right\}
  \]
  where $\sigma_{s}$ is the special representation in the double cell containing
  $\sigma$.

  Let $\Pi_{\cO,\mu}(G)$ denote the set of irreducible admissible $G$-module with
  complex associated variety $\overline{\cO}$.
  % Let $W_{\mu}$ be the stabilizer of $\mu$ and $W_{[\mu]}$ be the stabilizer of
  % the lattice
  Then
  \[
    \# \Pi_{\cO,\mu}(G) =
    \sum_{\sigma\in S_{\cO}} [\sigma: \mathrm{Coh}_{[\mu]}(G)] \cdot
    [1_{W_{\mu}}, \sigma|_{W_{\mu}}].
  \]
  % Here $[\sgima : \ ]$ denote the multiplicity of $\sigma$
  % and  $W_{\mu}$ is the stabilizer of $\mu$.
\end{thm}

\subsection{Counting special unipotent representations}

Now let $\ckcO$ be a nilpotent orbit in $\ckcG$.
It determine an infinitesimal character $\lamck$.

Let $\WLamck$ be the integral Weyl group and $\Wlamck$ be the stabilizer of
$\lamck$.

Special unipotent representations are defined to be the set of
Harish-Chandra modules with minimal GK-dimension.

Let $a(\sigma)$ be the generic degree of a Weyl group representation $\sigma$.

In view of \Cref{thm:count}, we need the following lemma by Barabsch-Vogan. %in \cite{BVUni}.
\begin{lem}[{\cite{BVUni}*{(5.26)}}]
  Let
  \[
    a_{\ckcO} = \max\set{a(\sigma)| \sigma \in \Irr(\WLamck) \text{ and }
    [1_{\Wlamck}: \sigma]\neq 0}.
  \]
  Let
  \[
    \LC_{\ckcO} =
    \set{\sigma \in \Irr(\WLamck) | a(\sigma) = a_{\ckcO}
      \text{ and } [1_{\Wlamck}: \sigma]\neq   0
    }.
  \]
  Then
  \begin{itemize}
   \item $\Wlamck$ is a Levi subgroup of $\WLamck$, and
 \item
  $\cC_{\ckcO}$ is a left cell of $\WLamck$ given by
  \[
    (J_{\Wlamck}^{\WLamck} \sgn )\otimes \sgn
  \]
  which contains a unique special representation
  \[
    (j_{\Wlamck}^{\WLamck} \sgn )\otimes \sgn.
  \]
  \end{itemize}
  \qed
\end{lem}
\trivial{
  WLOG, we can assume $\lamck$ is integral.
  Now this is essentially contained in \cite{BVUni}.

  We adapt the notation in \cite{BVUni}: two special representations
  $\sigma \LRleq \sigma'$ if and only if $\cO_{\sigma}\supseteq \cO_{\sigma'}$
  where $\cO_{\sigma}:=\Spr(\sigma)$. The generic degree of $\sigma$ is denoted
  by $a(\sigma)$.
  Note that the ordering of double cells/special representation is the same as
  the closure relation on special nilpotent orbits, see \cite{BVUni}*{Prop
    3.23}.

  Note that induction maps left cone representation to a left cone
  representation \cite{BVUni}*{Prop~4.14~(a)}. Therefore
  $\Ind_{\Wlamck}^{\WLamck}\sgn$ is a left cone representation.
  $J_{\Wlamck}^{\WLamck}\sgn$ consists the constituents in the induced
  representation with the minimal generic degree. In particular,
  $J_{\Wlamck}^{\WLamck}\sgn$ is the set of minimal representations under the
  $LR$-order and it is contained in a unique double cell $\cD$.


  Recall that tensoring with sign (or rather twisting $w_{0}$) is an order
  reversing bijection of left cells in $\WLamck$ and induces a $LR$-order
  reversing bijection on $\Irr(\Wlamck)$, see \cite{BV2}*{Prop.~2.25}. Therefore
  $\Ind_{\Wlamck}^{\WLamck} 1 = \left(\Ind_{\Wlamck}^{\WLamck} \sgn\right)\otimes \sgn$
  has a set of constituents which is maximal under the $LR$-order, in particular
  the generic degree takes maximal value on these representations.

  Hence we conclude that $\cC_{\ckcO}\otimes \sgn = J_{\Wlamck}^{\WLamck} \sgn$.
  % Therefore we conclude that $\cC_{\ckcO}\otimes \sgn$ consists the
  % representations minimal under $LR$-order in $\Ind_{\Wlamck}^{\WLamck}\sgn$.
  % In particular, $\cC_{\ckcO}\otimes \sgn$ consists of representations having
  % minimal generic degree, which is exactly $J_{\Wlamck}^{\WLamck} \sgn$.
  % By our counting theorem, the block of highest weight modules with
  % infinitesimal character $\lamck$ is in one-one correspondence with
  % $\WLamck/\Wlamck$. As a set of representations, it is a union of left cells
  % since the trivial representation is a left cell of $\Wlam$ and parabolic induction
  % preserves left cone representation \cite{BVUni}*{Prop~4.14~(a)}. By
  % Joseph-Vogan, there is a maximal primitive ideal in the set of primitive
  % ideals with infinitesimal character $\lambda$. In particular, the associated
  % variety of the maximal primitive ideal is the unique one with minimal
  % dimension. By the relation between generic degree and dimension of the associated
  % varieties of primitive ideals, we conclude that the set of representations in
  % $\Ind_{\Wlamck}^{\WLamck}1_{\Wlamck}$ which have the maximal generic degree
  % is contained in a unique left cell. These representations are
  % maximal in the $LR$-order. (Warning: induction or truncated induction does
  % not maps double cell onto a double cell in general. The notion of smoothly
  % induction means double cell maps to double cell, see
  % \cite{BVUni}*{Prop~4.14~last paragraph})
}

\section{Counting in type A}
The results in this section are well known to the experts.


Let $\YD$ be the set of Young diagrams viewed as a finite multiset of positive integers. 
The set of nilpotent orbits in $\GL_n(\bC)$ is identified with Young diagram of $n$ boxes.


Let $\ckGc = \GL_n(\bC)$. 
Fix
an orbit $\ckcO\in \Nil(ckG)$, let $\ckcO_e$ (resp. $\ckcO_o$) be the partition
consists of all even (resp. odd) rows in $\ckcO$.

Let $S_n$ denote the Weyl group of $\GL_n(\bC)$.
Let $W_n := S_n \ltimes \set{\pm 1}^n$ denote the Weyl group of type $B_n$ or $C_n$.  
Let $\sgn$ denote the sign representation of the Weyl group. 
The group $W_n$ is naturally embedded in $S_{2n}$.
For $W_n$, let $\epsilon$ denote the unique non-trivial character which is trivial on $S_n$. 
Note that $\epsilon$ is also the restriction of the $\sgn$ of $S_{2n}$ on $W_n$. 


\subsection{Special unipotent representations of $G = \GL_n(\bC)$}
By \cite{BVUni}, the set of unipotent representations
of $G = \GL_n(\bC)$ one-one corresponds to nilpotent orbits in $\Nil(\ckGc)$. 
Suppose $\ckcO$ has rows 
\[
\bfrr_1(\ckcO)\geq \bfrr_2(\ckcO)\geq \cdots\geq 
\bfrr_k(\ckcO) >0. 
\]
Then $\cO := \dBV(\ckcO)$ has columns $\bfcc_i(\cO) = \bfrr_i(\ckcO)$ for all 
$i\in \bN^+$. 
The map $\ckcO \mapsto \cO$ is a bijection.


We set $\CP =  \YD$ be the set of Young diagrams. 
For $\uptau \in \CP$ which has $k$ columns,
let $1_{c}$ be the trivial representations of $\GL_c(\bC)$. 
\[
 \pi_\uptau = 1_{\bfcc_1(\uptau)}\times  1_{\bfcc_2(\uptau)}\times \cdots 
 \times 1_{\bfcc_k(\uptau)}.
\]

The Vogan duality gives a duality between Harish-Chandra cells. 
In this case, Harish-Chandra cells is the double cell  
of Lusztig.  
Now we have a duality 
\[
 \pi_\uptau \leftrightarrow \pi_{\uptau^t}. 
\]

Let $\uptau' := \DD(\uptau)$ be the partition obtained by deleting the first column 
of $\uptau$. 
Let $\theta_{a,b}$ (resp. $\Theta_{a,b}$) be the  theta lift (resp. big theta lift) from $\GL_a(\bC)$ to  
$\GL_b(\bC)$. 
Then we have 
\[
  \pi_{\uptau} = \theta_{{\abs{\uptau'}},{\abs{\uptau}}} (\pi_{\uptau}). 
\]

\subsection{Counting unipotent representations of $\GL_n(\bR)$}
Now let $\ckcO\in \Nil(\ckGc)$. 
Recall the decomposition $\ckcO  = \ckcO_e \cup \ckcO_o$.
Let $n_e = \abs{\ckcO_e}, n_o = \abs{\ckcO_o}$ and $\lambda_\ckcO = \half \ckhh$. 

Then 
\[
  W(\lambda_\ckcO)  \cong S_{\abs{\ckcO_e}}\times S_{\abs{\ckcO_o}}. 
\]
\[
 W_{\lambda_\ckcO} = \prod_j S_{\bfcc_j(\ckcO_e)}\times \prod_j S_{\bfcc_j(\ckcO_o)} 
\]
By the formula of $a$-function, one can easily see that 
The cell in $W(\lamck)$ consists of the unique representation $J_{W_{\lamck}}^{\Wint{\lamck}} (1)$.
Now the $W$-cell is $J_{W_{\lamck}}^W 1$ corresponds to the orbit $\cO= \ckcO^t $ under the Springer
correspondence. 
\trivial{
WLOG, we assume $\ckcO =  \ckcO_o$.

Let $\sigma\in \widehat{S_n}$. We identify $\sigma$ with a Young diagram. 
Let $c_i = \bfcc_i(\sigma)$.
Then $\sigma = J^{S_n}_{W'} \epsilon_{W'}$ where $W' = \prod S_{c_i}$
(see Carter's book). 
This implies Lusztig's a-function takes value
\[
a(\sigma) = \sum_i c_i(c_i-1) /2
\]
Compairing the above with the dimension formula of nilpotent Orbits
\cite{CM}*{Collary~6.1.4}, we get (for the formula, see Bai ZQ-Xie Xun's paper on 
GK dimension of $SU(p,q)$)
\[
\half \dim(\sigma) = \dim(L(\lambda)) = n(n-1)/2 - a(\sigma).
\]
Here $\dim(\sigma)$ is the dimension of nilpotent orbit attached to the Young
diagram of $\sigma$ (it is the Springer correspondence, regular orbit maps to
trivial representation, note that $a(\triv)=0$), $L(\lambda)$ is any highest
weight module in the cell of $\sigma$. 


Return to our question, let $S' = \prod_i S_{\bfcc_i(\ckcO)}$. We want to find
the component $\sigma_0$ in $\Ind_{S'}^{S_n} 1$ whose $a(\sigma_0)$ is maximal,
i.e. the Young diagram of $\sigma_0$ is minimal. 

 By the branaching rule, $\sigma \subset \Ind_{S'}^{S_n} 1$ is given by adding
 rows of lenght $\bfcc_i(\ckcO)$ repeatly (Each time add at most one box in each
 column). 
 Now it is clear that $\sigma_0 = \ckcO^t$ is desired. 

 This agrees with the Barbasch-Vogan duality $\dBV$ given by 
 \[
  \ckcO \xrightarrow{Springer}\ckcO \xrightarrow{\otimes \sgn} \ckcO^t 
  \xrightarrow{Springer} \ckcO^t.
 \]
}

The $\Wint{\lamck}$-module $\Cint{\lamck}$ is given by the following formula:
\[
  \begin{split}
  \Cint{\lamck} &\cong \cC_{n_e}\otimes \cC_{n_o} \quad \text{with} \\ 
 \cC_n &:= \bigoplus_{\substack{s,a,b\\2s+a+b=n}} 
 \Ind_{W_s\times S_a\times S_b}^{S_{n}} \epsilon \otimes 1\otimes 1. % \text{ is a $S_n$-module.} 
  \end{split}
\] 

According to Vogan duality,  we can obtain the above formula by tensoring $\sgn$
on the forumla of the unitary groups in \cite{BV.W}*{Section~4}.

By branching rules of the symmetric groups,  $\Unip_{\ckcO}(G)$ can be parameterized by painted partition. 

\[
\PP{\ckcO} = \set{\uptau\colon \BOX{\ckcO^t}\rightarrow \set{\bullet,c,d}|
  \begin{minipage}{12em} ``$\bullet$'' occures with even\\ mulitplicity in each column
  \end{minipage}
}.  
\]
\trivial{
The typical diagram of all columns with even length $2c$ are
\[
\ytb{\bullet\cdots\bullet\bullet\cdots\bullet,\vdots\vdots\vdots\vdots\vdots\vdots,
\bullet\cdots\bullet c\cdots c,
\bullet\cdots\bullet d\cdots d
}  
\]

The typical diagram of all columns with odd length $2c+1$ are
\[
\ytb{\bullet\cdots\bullet\bullet\cdots\bullet,\vdots\vdots\vdots\vdots\vdots\vdots,
\bullet\cdots\bullet \bullet\cdots\bullet ,
c\cdots c d\cdots d
}  
\]
}

Let $\sgn_n\colon \GL_n(\bR)\rightarrow \set{\pm 1}$ be the sign of determinant. 
Let $1_n$ be the trivial representation of $\GL_n(\bR)$. 
For $\uptau\in \PP{\ckcO}$, we attache the representation 
\begin{equation}\label{eq:u.GLR}
\pi_\uptau := 
\bigtimes_{j} \underbrace{1_j \times \cdots \times 1_j}_{c_j\text{-terms}}\times
\underbrace{\sgn_j \times \cdots \times {\sgn_j} }_{d_j\text{-terms}}.
\end{equation}
Here 
\begin{itemize}
  \item 
$j$ running over all column lengths in $\ckcO^t$, 
\item $d_j$ is the number of
columns of length $j$ ending with the symbol ``d'',
\item  $c_j$ is the number of
columns of length $j$ ending with the symbol ``$\bullet$'' or ``$c$'', and 
\item  ``$\times$'' denote the parabolic induction.  
\end{itemize}

\subsection{Special Unipotent representations of $G=\GL_{m}(\bH)$}

Suppose that $\cO$ is the complexificiation of a rational nilpotent $\GL_{m}(\bH)$-orbit. 
Then $\cO$ has only even length columns. 
Therefore, $\Unip_\ckcO(G) \neq\emptyset$ only if $\ckcO = \ckcO_e$. 

In this case the coherent continuation representation is given by  
\[
  \Cint{\lamck}(G) = \Ind_{W_m}^{S_2m}\epsilon 
\]
and $\Unip_\ckcO(G)$ is a singleton. %We use partition $\tau:= \ckcO^t$ to parameter special unipotent representations of $\GL_{m}(\bH)$. 
For each partition $\tau$ only having even columns, we define 
\[
  \pi_{\tau} := \bigtimes_i 1_{\bfcc_i(\tau)/2}. 
\] 

\subsection{Counting special unipotent repesentations of $\rU(p,q)$}
We call the parity of $\abs{\ckcO}$ the ``good pairity''.  The other pairity is called the ``bad parity''. 
We write $\ckcO = \ckcO_g\cup \ckcO_b$ where $\ckcO_g$ and $\ckcO_b$ consist of good pairity length rows 
and bad pairity rows respectively.  

Let $(n_g,n_b) = (\abs{\ckcO_g},\abs{\ckcO_b})$.
Now as the $S_{n_g}\times S_{n_b}$ 
\[
\bigoplus_{\substack{p,q\in \bN\\p+q=n}} \Cint{\lamck}(\rU(p,q)) = \cC_{gd}\otimes \cC_{bd}
\]
where 
\[
  \begin{split}
 \cC_{gd} &= \bigoplus_{\substack{s,a,b\in \bN\\2s+a+b=n_g}} \Ind_{W_{s}\times S_a\times S_b}^{S_{n_g}}
 1\otimes \sgn\otimes \sgn \\
 \cC_{bd} &= \begin{cases}
  \Ind_{W_{\frac{n_b}{2}}}^{S_{n_b}} 1 & \text{if $n_b$ is even}\\
  0 & \text{otherwise}. 
 \end{cases}
  \end{split}
\]

By the above formula, we have
\begin{lem}
  \begin{enumT}
    \item
The set $\Unip_{\ckcO_b}(\rU(p,q))\neq \emptyset$ if and only if $p=q$ and 
each row lenght in $\ckcO$ has even multiplicity.
\item
Suppose $\Unip_{\ckcO_b}(\rU(p,p))\neq \emptyset$, let $\ckcO'$ be the Young diagram 
such that $\bfrr_i(\ckcO') = \bfrr_{2i}(\ckcO_b)$ and $\pi'$ be the unique special 
uinpotent representation in $\Unip_{\ckcO'}(\GL_{p}(\bC))$. 
Then the unique element in $\Unip_{\ckcO_b}(\rU(p,p))$  is given by 
\[
  \pi := \Ind_{P}^{\rU(p,p)} \pi'
\]
where $P$ is a parabolic subgroup in $\rU(p,p)$ with Levi factor equals
to $\GL_p(\bC)$.
\item 
In general, when $\Unip_{\ckcO_b}(\rU(p,p))\neq \emptyset$, we have a natural bijection 
\[  
  \begin{array}{rcl}
  \Unip_{\ckcO_g}(\rU(n_1,n_2)) &\longrightarrow& \Unip_{\ckcO}(\rU(n_1+p,n_2+p))\\
  \pi_0 & \mapsto & \Ind_P^{\rU(n_1+p,n_2+p)} \pi'\otimes \pi_0
  \end{array}
\]
where $P$ is a parabolic subgroup with Levi factor $\GL_p(\bC)\times \rU(n_1,n_2)$. 
  \end{enumT}
\end{lem}

The above lemma ensure us to reduce the problem to the case when $\ckcO = \ckcO_g$. 
Now assume $\ckcO = \ckcO_g$ and so $\Cint{\ckcO}$ corresponds to the blocks of 
the infinitesimal character of the trivial representation.   

By \cite{BV.W}*{Theorem~4.2},  Harish-Chandra cells in $\Cint{\ckcO}$ are in one-one
correspondence to real nilpotent orbits in $\cO:=\dBV(\ckcO)=\ckcO^t$. 

\trivial{
From the branching rule, the cell is parametered by painted partition 
\[
\PP{}(\rU):=\set{\uptau\in \PP{}| \begin{array}{l}\Im (\uptau) \subseteq  \set{\bullet, s,r}\\
  \text{``$\bullet$'' occures even times in each row}
\end{array} 
  }.  
\]

The bijection $\PP{}(\rU)\rightarrow \SYD, \uptau\mapsto \sO$ is given by the following recipe:
The shape of $\sO$ is the same as that of $\uptau$. 
$\sO$ is the unique (upto row switching) signed Young diagram such that
\[
  \sO(i,\bfrr_i(\uptau)) := \begin{cases}
    +,  & \text{when }\uptau(i,\bfrr_i(\uptau))=r;\\
    -,  & \text{otherwise, i.e. }\uptau(i,\bfrr_i(\uptau))\in \set{\bullet,s}.
  \end{cases}
\] 

\begin{eg}
  \[
 \ytb{\bullet\bullet\bullet\bullet r,\bullet\bullet , sr,s,r}   
 \quad
 \mapsto\quad
 \ytb{+-+-+,+-, -+,-,+}   
  \]
\end{eg}
}

Now the following lemma is clear. 
\begin{lem}
When $\ckcO=\ckcO_g$, the associated varity of every special unipotent representations in $\Unip_\ckcO(\rU)$  
is irreducible. Moreover, the following map  is a bijection. 
\[  
  \begin{array}{rcl}
  \Unip_{\ckcO_g}(\rU(n_1,n_2)) &\longrightarrow& \set{\text{rational forms of $\ckcO^t$}}\\
  \pi_0 & \mapsto & \wAV(\pi_0).
  \end{array}
\]
\qed
\end{lem}
\begin{remark}
  Note that the parabolic induction of an rational nilpotent orbit can be reducible. 
  Therefore, when $\ckcO_b\neq \emptyset$, the special unipotent representations can have
  reducible associated variety. Meanwhile, it is easy to see that the map
  $\Unip_{\ckcO}(\rU) \ni \pi \mapsto\wAV(\pi)$ is still injective. 
\end{remark}

We will show that every elements in $\Unip_{\ckcO_g}$ can be constructed by iterated theta lifting.  
For each $\uptau$, let $\sO$ be the corresponding real nilpotent orbit. Let
$\Sign(\sO)$ be the signature of $\sO$, $\DD(\sO)$ be the signed Young diagram
obtained by deleteing the first column of $\sO$. 
Suppose $\sO$ has $k$-columns. Inductively we have a sequence of unitary groups
$\rU(p_i,q_i)$ with $(p_i,q_i) = \Sign(\DD^i(\sO))$ for $i=0, \cdots, k$. Then 
\begin{equation}\label{eq:u.U}
  \pi_\tau = \theta^{\rU(p_0,q_0)}_{\rU(p_1,q_1)} \theta^{\rU(p_1,q_1)}_{\rU(p_2,q_2)}\cdots   
\theta^{\rU(p_{k-1},q_{k-1})}_{\rU(p_k,q_k)}(1)
\end{equation}
where $1$ is the trivial representation of $\rU(p_k,q_k)$. 


Suppose $\ckcO = \ckcO_g$. Form the duality between cells of $\rU(p,q)$ and
$\GL(n,\bR)$. We have an ad-hoc (bijective) duality between unipotent
representations: 
\[
  \begin{array}{rcl}
 \dBV\colon \Unip_{\ckcO}(\rU)& \rightarrow &\Unip_{\ckcO^t}(\GL(\bR)) \\
 \pi_\uptau &\mapsto& \pi_{\dBV(\uptau)} \\ 
  \end{array}
\]

Here $\ckcO^t = \dBV(\ckcO)$ and $\dBV(\uptau)$ is the pained bipartition
obtained by transposeing $\uptau$ and replace $s$ and $r$ by $c$ and $d$
respectively. See \eqref{eq:u.U} and \eqref{eq:u.GLR} for the definition of
special unipotent representations on the two sides.  

\section{Counting in type BCD}
\def\tsgn{\widetilde{\sgn}}
\def\PBP{\mathsf{PBP}}

\def\ckstar{{\check \star}}

In this section, we consider the case when $\ckstar \in \set{B,C,D}$, i.e
$\star \in \set{B,\wtC, C,D,C^{*}, D^{*}}$.

Recall that
\[
  \text{good pairity} =
\begin{cases}
 \text{odd} & \text{when} \ckstar\in \set{B,D}\\
 \text{even} & \text{when} \ckstar = C\\
\end{cases}
\]


\subsection{The left cell $\LC_{\ckcO}$}
We compute the left cell $\LC_{\ckcO}$ case by case.
Let $\ckcO = \ckcO_{b}\cupcol \ckcO_{g}$.

\begin{lem}
  Suppose $\star = B$, $\ckcO_{b}$ has $2k$ rows. Here each row in $\ckcO_{b}$ has
  odd length and each row in $\ckcO_{g}$ has even length, and
  $\WLamck = W_{b}\times W_{g}$ where $b=\half \abs{\ckcO_{b}}$ and
  $g = \half\abs{\ckcO_{g}}$.
  Let
  \[
    \begin{split}
      \tau_{b} =  & \left( (\frac{\bfrr_{2}(\ckcO_{b})+1}{2}, \frac{\bfrr_{4}(\ckcO_{b})+1}{2}, \cdots, \frac{\bfrr_{2k}(\ckcO_{b})+1}{2}),\right.\\
        &\ \ \left. (\frac{\bfrr_{2}(\ckcO_{b})-1}{2}, \frac{\bfrr_{4}(\ckcO_{b})-1}{2}, \cdots, \frac{\bfrr_{2k}(\ckcO_{b})-1}{2})\right) \in \Irr(W_{b}).
    \end{split}
  \]
  Set
  \[
    \CPP(\ckcO_{g}) = \set{(2i,2i+1)| \bfrr_{2i+1}(\ckcO_{g})> \bfrr_{2i}(\ckcO_{g}), \text{
        and } i\in \bN^{+}}
  \]
  and $\CQ(\ckcO)= \bF_{2}[\CPP(\ckcO_{g})]$.

  For $\sP \in \CQ(\ckcO)$, let
  \[
    \tau_{\sP} := (\imath,\jmath) \in \Irr(W_{b})
  \]
  such that
  \[
    \bfcc_{1}(\jmath)  := \half\bfrr_{1}(\ckcO_{g})
  \]
  and for all $i\geq 1$
  \[
  (\bfcc_{i}(\imath), \bfcc_{i+1}(\jmath)):=
  \begin{cases}
    (\half \bfrr_{2i}(\ckcO_{g}), \half \bfrr_{2i+1}(\ckcO_{g}))
    & \text{if } (2i,2i+1)\notin \sP,\\
    (\half \bfrr_{2i+1}(\ckcO_{g}),\half \bfrr_{2i}(\ckcO_{g})) & \text{otherwise.}
  \end{cases}
  \]

  Then we have the following bijection
  \[
    \begin{array}{ccc}
      \CQ(\ckcO) &\longrightarrow & \LC(\ckcO)\\
      \sP & \mapsto & \tau_{b}\otimes \tau_{\sP},
    \end{array}
  \]
  such that $\tau_{b}\otimes \tau_{\sP}$ is the special representation in
  $\LC(\ckcO)$.
\end{lem}

\trivial{
  In this case, bad parity is odd and every odd row occurs with with even times.
  We take the convention that
  % $2\cO = [2r_{i}]$ if $\cO = [r_{i}]$.
  % We also write $[r_{i}]\cup [r_{j}] = [r_{i},r_{j}]$.
  $\dagger \cO = [r_{i}+1]$.
  By abuse of notation, let $\dagger_{n} \sigma$  denote the
  $j_{S_{n} \times W_{\abs{\sigma}}}^{W_{n+\abs{\sigma}}} \sgn\otimes \sigma$.
  We can write
  \[
    \ckcO_{b} = [2r_{1}+1, 2r_{1}+1, \cdots, 2r_{k}+1,2r_{k}+1]
    = (2c_{0},2c_{1},2c_{1}, \cdots, 2c_{l}, 2c_{l})
  \]
  with $k = c_{0}$ and $l = r_{1}$.

\[
\begin{split}
  W_{\lamckb} &= W_{c_{0}} \times S_{2c_{1}} \times S_{2c_{2}}\times \cdots \times S_{2c_{l}}\\
  \cksigma_{b} &:= \sigma_{b}\otimes \sgn = j_{W_{\lamckb}}^{W_{b}} \sgn \\
  & = \dagger_{2c_{l}}\cdots \dagger_{2c_{1}}
  \binom{0, 1, \cdots, c_{0}}{1, \cdots, c_{0}}\\
  & =
  \binom{0, 1+r_{k}, 2+r_{k-1}\cdots, c_{0}+r_{1}}{1+r_{k},2+r_{k-1}, \cdots, c_{0}+r_{1}}\\
  & = ([r_{1},r_{2},\cdots, r_{k}],[r_{1}+1,r_{2}+1,\cdots,r_{k}+1])\\
  &= ((c_{1},c_{2},\cdots, c_{k}),(c_{0},c_{1}, \cdots, c_{l}))\\
\end{split}
\]

Therefore
\[
  \begin{split}
    \sigma_{b} &= \cksigma_{b}\otimes \sgn = ((r_{1}+1,r_{2}+1,\cdots,r_{k}+1),(r_{1},r_{2},\cdots, r_{k})) \\
    & = j_{S_{2r_{1}+1}\times \cdots S_{2r_{k}+1}}^{W_{b}} \sgn\\
    & = j_{S_{b}}^{W_{b}} (2r_{1}+1, 2r_{2}+1, \cdots, 2r_{k}+1)
  \end{split}
\]
which corresponds to the orbit
\[
  \cO_{b} = (2r_{1}+1, 2r_{1}+1,2r_{2}+1, 2r_{2}+1,  \cdots,2r_{k}+1, 2r_{k}+1 ) = \ckcO_{b}^{t}.
\]
(Note that $\cO'_{b} = (2r_{1}+1,2r_{2}+1, \cdots, 2r_{k}+1)$ which corresponds
to $j_{W_{L_{b}}}^{S_{b}}\sgn$ and $\ind_{L}^{G} \cO'_{b} = \cO_{b}$.
)
% This implies the unique special representation is
% \[
%   \sigma_{b} = (j_{W_{\lamckb}}^{W_{b}}\sgn), \quad \text{where } W_{L,b} = \prod_{i=1}^{k} S_{2r_{i}+1}.
% \]
The $J$-induction is calculated by \cite{Lu}*{(4.5.4)}.
It is easy to see that in our case $J_{W_{\lamckb}}^{W_{b}} \sgn$ consists of
the single special representation by induction.


Now we consider the good parity parts.


%First assume that there is even number of rows.
Consider
\[
\cO_{g} = [2r_{1},2r_{2}, \cdots, 2r_{2k-1},2r_{2k}]
= (C_{1},C_{1}, C_{2},C_{2},\cdots, C_{l}, C_{l}).
\]
with $l = r_{1}$ and $k = \ceil{C_{1}/2}$.
Write  $\ckLC_{\ckcO} = J_{W_{\lamck}}^{W_{[\lamck]}}\sgn$.



Note that the trivial representation of the trivial group has symbol
\[
\binom{0,1, 2, \cdots, k\phantom{-1}}{0,1, \cdots, k-1}.
\]
Now it easy to deduce that
\[
\begin{split}
\cksigma_{[\underbrace{2r,2r, \cdots, 2r}_{2k+1}]}
=& ([\underbrace{r,r, \cdots, r}_{k+1}], [\underbrace{r,r, \cdots, r}_{k}]),
% \cksigma_{[\underbrace{2r,2r, \cdots, 2r}_{2k+1}]}
% =& ([\underbrace{r,r, \cdots, r}_{k+1}], [\underbrace{r,r, \cdots, r}_{k}])\quad
\text{and}\\
\ckLC_{[\underbrace{2r',2r', \cdots, 2r',2r}_{2k+1\text{ terms}}]}
=&
\begin{cases}
  ([\underbrace{r',r', \cdots, r',r'}_{k+1}], [\underbrace{r',r', \cdots, r',r'}_{k+1}]) &
  \text{if } r'=r\geq 0 \\
  ([\underbrace{r',r', \cdots, r',r}_{k+1}], [\underbrace{r',r', \cdots, r',r'}_{k}]) &\\
 +([\underbrace{r',r', \cdots, r',r'}_{k+1}], [\underbrace{r',r', \cdots, r',r}_{k}]) &
  \text{if } r'>r \geq 0\\
 \text{(the first term is special)}& \\
\end{cases}
\end{split}
\]
%where the first term is the special representation.

Now
\[
  \begin{split}
    \cksigma_{\ckcO_{g}} & =
    J_{S_{C_{1}}\times \cdots \times S_{C_{l}}}^{W_{a}} \sgn\\
    % =& ((\ceil{C_{1}/2},\ceil{C_{2}/2}, \ceil{C_{1}/2}),
    % (\floor{C_{1}/2},\floor{C_{2}/2}, \floor{C_{1}/2}))\\
    =& \ckLC_{[\underbrace{2r_{2k},2r_{2k}, \cdots,2r_{2k},2r_{2k+1}}_{2k+1\text{terms}}]}\\
    & \cuprow \LC_{[2(r_{1}-r_{2k}), 2(r_{2}-r_{2k}), \cdots, 2(r_{2k-1}-r_{2k})]}.
  \end{split}
\]
By induction on the number of columns, we conclude that $\ckLC_{\ckcO_{g}}$ is in one-one corresponds to
the subsets of
\[
\CPP(\ckcO_{g}) = \set{(2i,2i+1)| \bfrr_{2i+1}(\ckcO_{g})> \bfrr_{2i}(\ckcO_{g}), \text{ and
  } i\in \bN^{+}}.
\]
For $\sP\in \CQ(\ckcO_{g})$, let $\sigma_{\sP} = (\imath,\jmath)$
such that
\[
\begin{split}
  \bfrr_{1}(\imath) &:= \half \bfrr_{1}(\ckcO_{g})\\
  (\bfrr_{l+1}(\imath), \bfrr_{l}(\jmath))&:=
  \begin{cases}
    (\half \bfrr_{2l+1}(\ckcO_{g}), \half \bfrr_{2l}(\ckcO_{g}))
    & \text{if } (2l,2l+1)\notin \sP\\
    (\half \bfrr_{2l}(\ckcO_{g}), \half \bfrr_{2l+1}(\ckcO_{g}))
    & \text{otherwise}
  \end{cases}
\end{split}
\]


Note that according to Lusztig and BV, the subsets of $\CPP(\ckcO)$ is in
one-one correspondence to the canonical quotient $\CQ(\ckcO) = \bF_{2}[\CPP(\ckcO)]$.


}


\begin{lem}
  Suppose $\star = C$, $\ckcO_{b}$ has $2k$ rows and $\ckcO_{g}$
  has $2l+1$-rows. Here each row in $\ckcO_{b}$ has
  even length and each row in $\ckcO_{g}$ has odd length, and
  $\WLamck = W'_{b}\times W_{g}$ where $b= \frac{\abs{\ckcO_{b}}}{2}$ and
  $g = \frac{\abs{\ckcO_{g}}-1}{2}$.
  Let
  \[
    \tau_{b} = \left( (\half\bfrr_{2}(\ckcO_{b}), \half\bfrr_{4}(\ckcO_{b}),\cdots, \half\bfrr_{2k}(\ckcO_{b}), (\half\bfrr_{2}(\ckcO_{b}), \half\bfrr_{4}(\ckcO_{b}),\cdots, \half\bfrr_{2k}(\ckcO_{b}) \right) \in \Irr(W_{b}).
  \]
  Set
  \[
    \CPP(\ckcO_{g}) = \set{(2i-1,2i)| \bfrr_{2i-1}(\ckcO_{g})> \bfrr_{2i}(\ckcO_{g})>0, \text{
        and } i\in \bN^{+}}
  \]
  and $\CQ(\ckcO)= \bF_{2}[\CPP(\ckcO_{g})]$.

  For $\sP \in \CQ(\ckcO)$, let
  \[
    \tau_{\sP} := (\imath,\jmath) \in \Irr(W_{g})
  \]
  such that
  \[
    (\bfcc_{l+1}(\imath), \bfcc_{l+1}(\jmath))  := (\half(\bfrr_{2l+1}(\ckcO_{g})-1),0)
  \]
  and for all $1\leq i\leq l$
  \[
  (\bfcc_{i}(\imath), \bfcc_{i}(\jmath)):=
  \begin{cases}
    (\half (\bfrr_{2i-1}(\ckcO_{g})+1),
    \half (\bfrr_{2i}(\ckcO_{g})-1))
    & \text{if } (2i-1,2i)\notin \sP,\\
    (\half (\bfrr_{2i}(\ckcO_{g})+1),\half (\bfrr_{2i-1}(\ckcO_{g})-1)) & \text{otherwise.}
  \end{cases}
  \]

  Then we have the following bijection
  \[
    \begin{array}{ccc}
      \CQ(\ckcO) &\longrightarrow & \LC(\ckcO)\\
      \sP & \mapsto & \tau_{b}\otimes \tau_{\sP}.
    \end{array}
  \]
  such that $\tau_{b}\otimes \tau_{\sP}$ is the special representation in
  $\LC(\ckcO)$.
\end{lem}




\trivial{
  In this case, bad parity is even and each row length occur with even
  multiplicity. Suppose
  $\ckcO_{b} = (C_{1}, C_{1}, C_{2},C_{2}, \cdots, C_{k'},C_{k'})$ with
  $c_{1}=2k$ and $k' = \bfrr_{1}(\ckcO_{b})$.
  \[
    W_{\lamckb} = S_{C_{1}}\times S_{C_{2}}\times \cdots S_{C_{k'}}.
  \]
  The symbol of trivial representation of trivial group of type D is
  \[
    \binom{0,1, \cdots, k-1}{0,1, \cdots, k-1}.
  \]
  Now it is easy to see that
  \[
    J_{W_{\lamckb}}^{W_{b}}\sgn = ((\half C_{1}, \half C_{2},\cdots,
    \half C_{k'}),(\half C_{1}, \half C_{2},\cdots,
    \half C_{k'})).
  \]

  For the good parity part.
  Suppose
  $\ckcO_{g} = (2c_{1}+1, C_{2}, C_{2},C_{3},C_{3},\cdots, C_{k'},C_{k'})$ with
  $2c_{1}+1=2l+1$ and $2k'+1 = \bfrr_{1}(\ckcO_{g})$.
  \[
    W_{\lamckg} = W_{c_{1}}\times S_{C_{2}}\times \cdots
    \times S_{C_{k'}}.
  \]

  The symbol of sign representation of $W_{c_{1}}$ is
  \[
    \binom{0,1,2, \cdots, c_{1}}{1,2, \cdots, c_{1}}.
  \]

  By induction on number of columns,
  we see that when even column of length $2c$ occurs, it adds
  length $c$ columns on the both sides of the bipartition;
  when odd column $C_{i}=2c_{i}+1$ with $i>1$ and multiplicity $2r'$ occur,  the bifurcation happens:
  one can attach $r'$ columns of length $c_{i}+1$ on the right and $r'$ columns
  of length $c_{i}$ on the left (special representations) or
  attach $r'$ columns of length $c_{i}+1$ on the left and $r'$ columns
  of length $c_{i}$ on the right.

  Therefore,
  \[
    \begin{array}{ccc}
    J_{W_{\lamckg}}^{W_{g}} \sgn
    &\leftrightarrow&  \bF_{2}(\CPP(\ckcO_{g}))\\
    \cktau_{\sP}&\leftrightarrow & \sP
    \end{array}
  \]
  where
  \[
    \begin{split}
    \bfrr_{l+1}(\cktau_{R}) & =\half (\bfrr_{2l+1}(\ckcO_{g})-1)\\
    (\bfrr_{i}(\cktau_{L}), \bfrr_{i}(\cktau_{R})) & =
    \begin{cases}
     (\half(\bfrr_{2i}(\ckcO_{g})-1), \half(\bfrr_{2i-1}(\ckcO_{g})+1)) & (2i-1,2i)\notin \sP\\
     (\half(\bfrr_{2i-1}(\ckcO_{g})-1), \half(\bfrr_{2i}(\ckcO_{g})+1)) & (2i-1,2i)\in \sP
    \end{cases}
    \end{split}
  \]

  Now tensor with sign yields the result.
}


\begin{lem}
  Suppose $\star \in \set{D, D^{*}}$, $\ckcO_{b}$ has $2k$ rows and $\ckcO_{g}$
  has $2l$-rows. Here each row in $\ckcO_{b}$ has even length and each row in
  $\ckcO_{g}$ has odd length, and $\WLamck = W'_{b}\times W'_{g}$ where
  $b= \frac{\abs{\ckcO_{b}}}{2}$ and $g = \frac{\abs{\ckcO_{g}}}{2}$. Let
  \[
    \tau_{b} = \left( (\half\bfrr_{2}(\ckcO_{b}), \half\bfrr_{4}(\ckcO_{b}),\cdots, \half\bfrr_{2k}(\ckcO_{b}), (\half\bfrr_{2}(\ckcO_{b}), \half\bfrr_{4}(\ckcO_{b}),\cdots, \half\bfrr_{2k}(\ckcO_{b}) \right) \in \Irr(W'_{b}).
  \]
  Set
  \[
    \CPP(\ckcO_{g}) = \set{(2i,2i+1)| \bfrr_{2i}(\ckcO_{g})> \bfrr_{2i+1}(\ckcO_{g})>0, \text{
        and } i\in \bN^{+}}
  \]
  and $\CQ(\ckcO)= \bF_{2}[\CPP(\ckcO_{g})]$.

  For $\sP \in \CQ(\ckcO)$, let
  \[
    \tau_{\sP} := (\imath,\jmath) \in \Irr(W'_{g})
  \]
  such that
  \[
    \begin{split}
      \bfcc_{1}(\imath)  &:= \half(\bfrr_{1}(\ckcO_{g})+1)\\
      (\bfcc_{l+1}(\imath), \bfcc_{l}(\jmath))  &:= (0,\half(\bfrr_{2l}(\ckcO_{g})-1))
    \end{split}
  \]
  and for all $1\leq i< l$
  \[
  (\bfcc_{i+1}(\imath), \bfcc_{i}(\jmath)):=
  \begin{cases}
    (\half (\bfrr_{2i+1}(\ckcO_{g})+1),
    \half (\bfrr_{2i}(\ckcO_{g})-1))
    & \text{if } (2i,2i+1)\notin \sP,\\
    (\half (\bfrr_{2i}(\ckcO_{g})+1),\half (\bfrr_{2i+1}(\ckcO_{g})-1)) & \text{otherwise.}
  \end{cases}
  \]

  Then we have the following bijection
  \[
    \begin{array}{ccc}
      \CQ(\ckcO) &\longrightarrow & \LC(\ckcO)\\
      \sP & \mapsto & \tau_{b}\otimes \tau_{\sP}.
    \end{array}
  \]
  such that $\tau_{b}\otimes \tau_{\sP}$ is the special representation in
  $\LC(\ckcO)$.
\end{lem}
\trivial{
  The bad parity part is the same as that of the case when $\star = C$.

  For the good parity part.
  Suppose
  $\ckcO_{g} = (2c_{1}, C_{2}, C_{2},C_{3},C_{3},\cdots, C_{k'},C_{k'},C_{k'+1})$ with
  $2c_{1}=2l$ and $2k'+2 = \bfrr_{1}(\ckcO_{g})$.

  We use the two facts:
  \[
    W_{\lamckg} = W_{c_{1}}\times S_{C_{2}}\times \cdots
    \times S_{C_{k'}}.
  \]

  The symbol of sign representation of $W'_{c_{1}}$ is
  \[
    \binom{0,1, \cdots, c_{1}-1}{1,2, \cdots, c_{1}\phantom{-1}}.
  \]
  The bifurcation happens for even length columns.
  % Eg: 0,1,2,
  %     1,2,3,

}


\begin{lem}
  Suppose $\star = \wtC$, $\ckcO_{b}$ has $2k$ rows. Here each row in $\ckcO_{b}$ has
  odd length and each row in $\ckcO_{g}$ has even length, and
  $\WLamck = W_{b}\times W'_{g}$ where $b=\half \abs{\ckcO_{b}}$ and
  $g = \half\abs{\ckcO_{g}}$.
  Let
  \[
    \begin{split}
      \tau_{b} =  & \left( (\frac{\bfrr_{2}(\ckcO_{b})+1}{2}, \frac{\bfrr_{4}(\ckcO_{b})+1}{2}, \cdots, \frac{\bfrr_{2k}(\ckcO_{b})+1}{2}),\right.\\
        &\ \ \left. (\frac{\bfrr_{2}(\ckcO_{b})-1}{2}, \frac{\bfrr_{4}(\ckcO_{b})-1}{2}, \cdots, \frac{\bfrr_{2k}(\ckcO_{b})-1}{2})\right) \in \Irr(W_{b}).
    \end{split}
  \]
  Set
  \[
    \CPP(\ckcO_{g}) = \set{(2i-1,2i)| \bfrr_{2i-1}(\ckcO_{g})> \bfrr_{2i}(\ckcO_{g}), \text{
        and } i\in \bN^{+}}
  \]
  and $\CQ(\ckcO)= \bF_{2}[\CPP(\ckcO_{g})]$.

  For $\sP \in \CQ(\ckcO)$, let
  \[
    \tau_{\sP} := (\imath,\jmath) \in \Irr(W'_{g})
  \]
  such that for all $i\geq 1$
  \[
  (\bfcc_{i}(\imath), \bfcc_{i}(\jmath)):=
  \begin{cases}
    (\half \bfrr_{2i-1}(\ckcO_{g}), \half \bfrr_{2i}(\ckcO_{g}))
    & \text{if } (2i-1,2i)\notin \sP,\\
    (\half \bfrr_{2i}(\ckcO_{g}),\half \bfrr_{2i-1}(\ckcO_{g})) & \text{otherwise.}
  \end{cases}
  \]

  Then we have the following bijection
  \[
    \begin{array}{ccc}
      \CQ(\ckcO) &\longrightarrow & \LC(\ckcO)\\
      \sP & \mapsto & \tau_{b}\otimes \tau_{\sP},
    \end{array}
  \]
  such that $\tau_{b}\otimes \tau_{\sP}$ is the special representation in
  $\LC(\ckcO)$.
\end{lem}

\trivial{
  The bad parity part is the same as the case when $\star= B$.


  For the good parity part, note that the trivial representation of
  the trivial group has symbol
  \[
    \binom{0,1,\cdots, r}{0,1,\cdots, r}.
  \]
  Here we assume $\ckcO_{g}$ has at most $2r$ rows.

  Now the bifurcation happens for the odd length column.
  }

% \subsection{}
% In this section, we count special unipotent representations of real orthogonal
% groups (type $B$,$D$), symplectic groups (type $C$) and metaplectic groups (type
% $\wtC$).

% Recall that
% \[
%   \text{good pairity} =
% \begin{cases}
%  \text{odd} & \text{in type $C$ and $D$}\\
%  \text{even} & \text{in type $C$ and $D$}\\
% \end{cases}
% \]
% We decompose $\ckcO  = \ckcO_g\cup \ckcO_b$ as before.

% Let $\tsgn$ be the character of $W_n$ inflated from the sign character of $S_n$ via the
% natural map $W_n \rightarrow S_n$.
% Recall the following formula
% \[
%   \Ind_{W_b\ltimes \set{\pm 1}^b}^{W_2b} 1 \otimes \sgn = \sum_{\sigma} (\sigma,\sigma)
% \]
% where $\sigma$ running over all Young diagrams of size $b$.


%\subsection{Type C}

\subsection{Counting special unipotent representation of $G=\Sp(p,q)$}. 
The dual group of $\Gc = \Sp(2n,\bC)$ is $\ckGc = \SO(2n+1,\bC)$. 
We set $(2m+1,2m') = (\abs{\ckcO_g},\abs{\ckcO_n})$.
By Vogan duality, the dual group for class $[\lamck]$ is 
\[
 \SO(2m+1,\bC)\times \rO(2m',\bC).   
\]


Now we have 
\[
  \Cint{\lamck}(G) \otimes \sgn  = \cC_g\otimes \cC_b 
\]
with 
\[
  \begin{split}
 \cC_g &  = \bigoplus_{p+q=m} \Cint{\rho}(\Sp(p,q)) \\ 
 &=\bigoplus_{\substack{b,s,r\in\bN\\2b+s+r=m}}
 \Ind_{W_b\ltimes \set{\pm 1}^b \times W_s\times W_b}^{W_{m}} 
 1\otimes \sgn \otimes \sgn \otimes \sgn \\
& =\bigoplus_{\substack{b,s,r\in\bN\\2b+s+r=m}}
 \Ind_{W_{2b}\times W_s\times W_b}^{W_{m}} 
 (\sigma,\sigma)\otimes \sgn \otimes \sgn \\
 \cC_b &  = \Ind_{W_{\frac{m'}{2}}\ltimes \set{\pm 1}^{\frac{m'}{2}}}^{W_{m'}} \sgn 
 \\
  \end{split}
\]


In $\SO(2m',\bC)$, the dobule cell corresponds to $\ckcO_b$ consists of 
a single  representation  
\[
  \cktau_b = (\cksigma_b,\cksigma_b) \text{ such that } 
  \bfrr_i(\cksigma_b) = \bfrr_{2i}(\ckcO_b)/2
\]
Let $\ckcO'_b$ be the Young diagram such  that $\bfrr_i(\ckcO'_b) = \bfrr_{2i}(\ckcO_b)$ 
and $\pi'_b$ be the unique special unipotent representation attached to $\GL_p(\bH)$
\begin{lem}
The set $\Unip_{\ckcO_b}(\Sp(p,q))\neq \emptyset$ only if $p=q = \abs{\cksigma_b}$ and in this case it consists a single element
\[
  \pi_b := \Ind_P^{\Sp(p,p)} \pi'_{\cksigma_b} %^{\GL_p(\bH)} 
\]
where $P$ has the Levi subgroup $\GL_p(\bH)$.  
\end{lem}


The special reprsentation corresponds to $\ckcO_b$ is 
\[
 aa 
\]
Let 
\[
\PBP_{\ckcO_g}(C^*):= \Set{(\imath,\jmath,\cP,\cQ)| \begin{array}{l}
  (\imath,\jmath) = \cktau_g\\
  \Im \cP \subset \set{\bullet}, \Im \cQ \subset\set{\bullet, s,r}\\
\end{array} 
  }.  
\]

\begin{lem}
  The set of $\Unip_{\ckcO}(\Sp(p,q))$
  is paramterized by the set  such that 
  \[
    ss
  \]
\end{lem}


\trivial{Let $W = W(\Gc)$ where $\Gc$ is natrualy embeded in $\GL(n,\bC)$. Let
  $s_{\varepsilon_i i,\varepsilon_j j}$ be the permutation matrix of index
  $i,j$, where the $(i,j)$-th entry is $\varepsilon_i$, the $(j,i)$-th entry is
  $\varepsilon_j$ and the other place is the identity matrix. Let $w_{i,\pm j}^{\epsilon}$
  be the element such that 
  \begin{itemize}
   \item  it is in $\Gc$ and entries in $\set{0}\cup \mu_4$
   \item  it lifts the
  element $e_i \leftrightarrow \pm e_j$ in the Weyl group.
  \item $(w_{i,\pm j}^{\epsilon})^2 = \epsilon 1\in \set{\pm 1}$.  
  \end{itemize}
  Let \[h_{\pm i}^+  = \diag(1,\cdots, 1, \pm 1, 1, \cdots, 1)\] where $\pm 1$
    is the $i$-th place.  
    Let 
\[h_{\pm i}^-  = \diag(1,\cdots, 1, \pm \sqrt{-1}, 1, \cdots, 1)\] 
  where $\pm \sqrt{-1}$ is the $i$-th place.  
  Let $e_{\pm i} = s_{\pm i,\pm (n-i+1)}$.  

  Let $\ckww_{i,\pm j}^\pm$ be the lift of $e_i \leftrightarrow \pm e_j$ such that 
  \begin{itemize}
   \item  it is in $\Gc$ and entries in $\set{0}\cup \mu_4$
   \item  it lifts the
  element $e_i \leftrightarrow \pm e_j$ in the Weyl group.
  \item $(w_{i,\pm j}^{\epsilon})^2|_{[i,j]} = \epsilon 1\in \set{\pm 1}$ here
  ``$|_{[i,j]}$'' means restricts on the $e_i,e_j,-e_i,-e_j$-weights space.  
  \end{itemize}

Let 
\[
  \begin{split}
  x_{b,s,r} &= w_{1,2}^+\cdots w_{2b-1,2b}^+\, h_{2b+1}^+\cdots h_{2b+s}^+ \\
  y_{b,s,r}^+ &= \ckww_{1,-2}^+\cdots \ckww_{2b-1,-2b}^+\,
   e_{2b+1}\cdots e_{2b+s+r} \\
  y_{b,s,r}^- &= \ckww_{1,-2}^-\cdots \ckww_{2b-1,-2b}^- 
  \end{split}
\]


We compute the parameter space 
\[
  \begin{split}
\cZ_g &=  \bigcup_{2b+s+r=m} W_m\cdot (x_{b,s,r}^+, y_{b,s,r}^+)\\  
\cZ_b &=  \bigcup_{2b=m} W_m\cdot (x_{b,0,0}^+, y_{b,0,0}^-)\\  
  \end{split}
\]

}


\subsection{Counting special unipotent representation of $G = \Sp(2n,\bR)$}

Now we have 
\[
  \Cint{\lamck}(G) \otimes \sgn  = \cC_g\otimes \cC_b 
\]
with 
\[
  \begin{split}
 \cC_g &= \bigoplus_{p+q=2m+1}\Cint{\rho}(\SO(p,q))  \\
 &= 
 \bigoplus_{2b+s+r+c+d=m}\Ind_{W_b\ltimes \set{-1}^b\times W_s\times W_r\times W_c\times W_d}^{W_m} 1 \otimes \det \otimes \det \otimes 1 \otimes 1\\
 \cC_b &= \Cint{\rho}(\SO^*(2m)) = \bigoplus_{\substack{2s+b=m}}\Ind_{W_s\ltimes \set{-1}^s\times S_b}^{W_m} 1 \otimes\sgn\otimes \sgn   
  \end{split}
\]



\section{Examples}
\subsection{Symplectic group}
Let $G = \Sp(4,\bR)$. Then $\ckG_{\bC}= \SO(5,\bC)$.
Then $\ckcO$ has the following possibilities
\[
  \ckcO_{1}:= \ydiagram{5},\quad \ckcO_{2}:=\ydiagram{3,1,1}, \quad
  \ckcO_{3}:=\ydiagram{2,2,1}, \quad \ckcO_{4}:=\ydiagram{1,1,1,1,1}
\]
Here the orbit $\ckcO_{3}$ is non-special.
\[
  \cO_{1}:= \ydiagram{1,1,1,1},\quad \cO_{2}:=\ydiagram{2,2}, \quad
  \cO_{3}:=\ydiagram{2,2}, \quad \cO_{4}:=\ydiagram{4}
\]

For the real form of $\ckG_{\bC}$, we always consider the inner class
$\set{\SO(p,q)| q \text{ is even}}$.
Therefore, we have $3$ groups $\SO(1,4)$, $\SO(3,2)$ and $\SO(5,0)$.

\noindent{\bf When $\ckcO = \ckcO_1$}

Now
\[
  \Unip_{\ckcO_{1}}(\Sp_{4}(\bR) = \set{\triv}.
\]
and
\[
  \PBP_{C}(\ckcO) = \Set{\emptyset \times \ytb{s,s}}.
\]


The orbit $\cO_{1}$ has bad parity for $\SO_{5}$.
Now
\[
  \Unip_{\cO_{1}}(\SO_{5}) = \set{\Ind_{\GL_{1}\times \GL_{1}}^{\SO(3,2)} \sgn^{\epsilon_{1}}\otimes \sgn^{\epsilon_{2}}|
  \epsilon_{1},\epsilon_{2}\in \set{0,1}}
\]
which has $3$ elements.

\noindent{\bf When  $\ckcO = \ckcO_2$}

There are three unipotent Arthur parameters $\psi_{++}, \psi_{+-},\psi_{--}$.
These parameters corresponds to the three rational forms of $\ckcO$ respectively
\[
\ytb{-+-,+,+}, \ytb{+-+,+,-},\ytb{-+-,-,-}.
\]
Here we fix the inner class of $\SO(p,q)$ such that $q$ is even.

Now $\PBP_{C}(\ckcO)$  has $8$ elements.
The set $\PBP_{C}(\ckcO)$ one-one corresponds to the associated cycles.

The Arthur packet $\Pi_{\psi_{++}}$ contains two elements.
We expect that they consists of $\Sp(4,\bR)$ modules with associated variety
\[
  \ytb{-+,-+}.
\]
Similarly,
the Arthur packet $\Pi_{\psi_{--}}$ contains two elements and
they consists of $\Sp(4,\bR)$ modules with associated variety
\[
  \ytb{+-,+-}.
\]

The Arthur packet $\Pi_{\psi_{+-}}$ has 4-elements, we expect that
they consists of modules with associated variety
\[
  \ytb{-+,+-}.
\]

The explicit matching of the packet can be down by compare the K-types
in ABV's (27.17) with theta lifts.


\noindent{\bf When  $\ckcO = \ckcO_3$}
The orbit has bad parity part. There are two special unipotent representations given
by %parabolic induction
\[
\Ind_{\GL_{2}(\bR)}^{\Sp_{4}(\bR)} \sgn^{\epsilon} \qquad \epsilon \in \set{0,1}.
\]


\noindent{\bf When  $\ckcO = \ckcO_4$}
In this case $\Unip_{\Sp(2n,\bR)}(\ckcO)$ has $5$ elements which are
distinguished by  their associated cycles listed below.
\[
  \begin{split}
  \ytb{-+-+},& \quad
  \ytb{+-+-},\quad
  \ytb{\ppp\mmm\ppp\mmm},\quad
  \ytb{\mmm\ppp\mmm\ppp}, \\
  & \ytb{-+-+}\cup
  \ytb{-+-+},\quad
  \end{split}
\]

Totally there are $3$ Arthur parameters corresponds to
the trivial orbits of $\SO(1,4)$, $\SO(3,2)$ and $\SO(5,0)$.
We expect the Arthur packets are disjoint with each other.


The orbit $\cO_{4}$ has good parity for $\SO_{5}$.
So there must be a one-one correspondence between $\Unip_{\Sp_{2n}(\bR)}(\ckcO)$
In fact, there are totally $5$ special
unipotent representations of real forms in a inner class of $\SO_{5}$,
$2$ for $\SO(1,4)$, $2$ for $\SO(3,2)$ and $1$ for $\SO(5,0)$.
Theses representations are trivial when restricted on the connected component.




% \[
%   \PBP_{C}(\ckcO) =
%   \Set{
%     \begin{array}{c}
%     \ytb{c,d}\times \emptyset,
%     \ytb{r,d}\times \emptyset,
%     \ytb{r,c}\times \emptyset,
%       \ytb{r,r}\times \emptyset,
%       \vspace{.5em}
%       \\
%     \ytb{\bullet}\times \ytb{\bullet},
%     \ytb{r}\times \ytb{s},
%     \ytb{c}\times \ytb{s},
%     \ytb{d}\times \ytb{s},
%     \end{array}
%   }
% \]


% The orbit $\cO_{4}$ has good parity for $\SO_{5}$.
% So there is a one-one correspondence between $\Unip_{\Sp_{2n}(\bR)}(\ckcO)$
% and $\Unip_{\SO_{2n+1}}(\cO)$.

% We list the correspondence and associated variety
% below:
% \[
%   \ytb{\ppp\mmm,\ppp\mmm} \leftrightarrow  \ytb{c,d}\times \emptyset
%   \leftrightarrow \emptyset \times \ytb{sr} \leftrightarrow
%   \ytb{\ppp\mmm\ppp,+,\mmm}
% \]


% \[
%   \ytb{\mmm\ppp, \ppp\mmm} \leftrightarrow  \ytb{r,d}\times \emptyset
%   \leftrightarrow \emptyset \times \ytb{rd} \leftrightarrow
%   \ytb{+-+,+,+}
% \]


% \[
%   \ytb{\mmm\ppp, +-}\leftrightarrow  \ytb{r,c}\times \emptyset
%   \leftrightarrow \emptyset \times \ytb{sd} \leftrightarrow
%   \ytb{\mmm\ppp\mmm,+,\mmm}
% \]

% \[
%   \ytb{-+,-+} \leftrightarrow  \ytb{r,r}\times \emptyset
%   \leftrightarrow \emptyset \times \ytb{dd} \leftrightarrow
%   \ytb{-+-,+,+}
% \]



% \[
% \begin{array}{cccccccc}
%    & \AV(\pi)\subset \cO & \pi & \PBP_C & \PBP_{B} &\ckpi &  \AV(\ckpi)\subset \ckcO & \text{remark}\\
%   \hline
%   &
%     \vspace{.5em}
%     \ytb{+,+,-,-}
%     \vspace{.5em}
%                          & \triv &  \emptyset \times \ytb{s,s}
%       & na  & \Ind_{\GL(2,\bR)} 1 & \ytb{+-+-+}&   \text{$\cO$ has bad parity for $\SO(3,2)$}\\
%   \hline
%   &
%     \ytb{+,+,-,-} & \triv &  \emptyset \times \ytb{s,s}
%       & na  & \Ind_{\GL(2,\bR)} 1 & \ytb{+-+-+}&   \text{$\cO$ has bad parity for $\SO(3,2)$}\\
%     \ytb{+,+,-,-} & \Ind_{\GL(1,\bR)\times \Sp(2,\bR)}\triv &  \emptyset \times \ytb{s,s}
%       & na  & \Ind_{\GL(2,\bR)} 1 & \ytb{+-+-+}&   \text{$\cO$ has bad parity for $\SO(3,2)$}\\
%   \end{array}
% \]
%

\appendix
\section{Combinatorics of Weyl group representations in the classical types}

\subsection{The $j$-induction}
If $\mu$ and $\nu$ are two partitions representing two symmetric groups
representations.
Then
\[
  j_{S_{\abs{\mu}}\times S_{\abs{\nu}}}^{S_{\abs{\mu}+\abs{\nu}}}
  \mu\boxtimes \nu
  = \mu\cup \nu,
\]
where $\mu\cup \nu$ is the partition such that
\[
  \set{\bfcc_{i}(\mu\cup\nu) | i\in \bN^{+}} =
  \set{\bfcc_{i}(\mu) | i\in \bN^{+}}
  \cup
  \set{\bfcc_{i}(\nu) | i\in \bN^{+}}
\]
as multisets.

\trivial{
Use the inductive by stage of $j$-induction
\[
  \begin{split}
  &j_{S_{\abs{\mu}}\times S_{\abs{\nu}}}^{S_{\abs{\mu}+\abs{\nu}}}
  \mu\boxtimes \nu\\
  &=
j_{S_{\abs{\mu}}\times S_{\abs{\nu}}}^{S_{\abs{\mu}+\abs{\nu}}}
j^{S_{\abs{\mu}}\times S_{\abs{\nu}}}_{\prod_{i}S_{\bfcc_{i}(\mu)}\times
  \prod_{j} S_{\bfcc_{j}(\nu)}}\sgn\\
 &=
j^{S_{\abs{\mu}+\abs{\nu}}}_{\prod_{i}S_{\bfcc_{i}(\mu\cup \nu)}}\sgn\\
  &= \mu\cup \nu.
  \end{split}
\]
}



We have
\[
  j_{S_{n}}^{W_{n}} \sgn = \begin{cases}
    (\cboxs{k},\cboxs{k}) & \text{if $n=2k$ is even,}\\
    (\cboxs{k+1},\cboxs{k}) & \text{if $n=2k+1$ is odd.}\\
    \end{cases}
\]
\trivial{
  The symbol of trivial of trivial group is
  \[
    \symb{0, 1, \cdots, k}{0,\cdots, k-1} .
  \]
  Apply the formula \cite{Lu}*{4.5.4},
  When $n$ is even, the symbol of the induce is
  \[
    \symb{0,2, \cdots, k+1}{1,\cdots, k}.
  \]
  corresponds to $(\cboxs{k}, \cboxs{k})$.

  When $n$ is even, the symbol of the induce is
  \[
    \symb{1,2, \cdots, k+1}{1,\cdots, k}.
  \]
  corresponds to $(\cboxs{k+1}, \cboxs{k})$.
}

If $\tau = (\tau_{L}, \tau_{R})$ and $\sigma = (\sigma_{L}, \sigma_{R})$
be two bipartition. Then
\[
  j_{W_{\abs{\tau}}\times W_{\abs{\sigma}}}^{W_{\abs{\tau}+\abs{\sigma}}}
  \tau \boxtimes \sigma = (\tau_{L}\cup \sigma_{L}, \tau_{R}\cup \sigma_{R})
\]


\delete{
\subsection{Parameterize of Unipotent representations}
We fix an abstract complex Cartan subgroup $\bfH_a$ and $\fhh_a$ in $\bfG$ and a
set of simple roots $\Pi_a$.  Let $\cP(\bfG)$ be the set of all Langlands
parameters of $G$-modules with character $\rho$ (i.e. the infinitesimal
character of the trivial representation). For $\gamma\in \cP(\bfG)$, let
$\cL(\gamma)$, $\cS(\gamma)$ and $\Phi_\gamma$ be the corresponding Langlands
quotient, standard module and coherent family such that
$\Phi_\gamma(\rho) = \cL(\gamma)$. Let $\cM(\bfG)$ be the span of $\cL(\gamma)$.
Let $\set{\bB}$ be the set of all blocks. Then $\cP(\bfG) = \bigsqcup_\cB \cB$.
The Weyl group $W = W(G)$ acts on $\cM(\bfG)$ by coherent continuation.  Let
$\cM_{\cB}$ be the submodule of $\cM(\bfG)$ spanned by $\gamma\in\cB$, then
\[
  \cM(\bfG) = \bigoplus_\cB \cM_{\cB}
\]
Let $\tau(\gamma)\subset \Pi_a$ be the $\tau$-invariant of $\gamma$.

Let $\ckcO$ be even orbit. $\lambda= \half \ckhh$.  Define
\[
  S(\lambda) = \set{\alpha\in \Pi_a| \inn{\alpha}{\lambda}=0}.
\]
Let $\cP_{\lambda}(\bfG)$ be the set of all Langlands parameters with
infinitesimal character $\lambda$. Let $T_{\lambda,\rho}$ be the translation
functor.  Let
\[
  \cB(S) = \set{\gamma\in \cB|S\cap \tau(\gamma)=\emptyset}
\]
and
\[
  \cP(\bfG,S) = \bigsqcup_{\cB} \cB(S)
\]


Then
\[
  \begin{tikzcd}[row sep=0em]
    \cP(\bfG,S) \ar[r] & \cP_{\lambda}(\bfG)\\
    \gamma \ar[r, maps to]& T_{\lambda, \rho}(\gamma)
  \end{tikzcd}
\]

Let $\cO$ be a complex nilpotent orbit in $\fgg$.  Let
\[
  \cB(S,\cO) = \set{\gamma\in \cB(S)|\AVC(\cL(\gamma))\subset \bcO}
\]

Let
\[
  \begin{aligned}
    m_S(\sigma) &= [\sigma: \Ind_{W(S)}^{W}\bfone]\\
    m_\cB(\sigma)& = [\sigma: \cM_\cB]
  \end{aligned}
\]


Barbasch \cite{B10}*{Theorem~9.1} established the following theorem.
\begin{thm}
  \[
    |\cB(S,\cO)| = \sum_{\sigma} m_\cB(\sigma)m_S(\sigma)
  \]
  Here $\sigma\times \sigma$ running over the $W\times W$ appears in the double
  cell $\cC(\cO)$.
\end{thm}
\begin{proof}
  We need to take the graded module of $\cM(\bfG)$ with respect to the
  $\LRleq$. By abuse of notation, we identify the basis $\cP(\bfG)$ with its
  image in the graded module.  Note that $S\cap \tau(\lambda)=\emptyset$ if and
  only if $W(S)$ acts on $\gamma$ trivially by \cite[Lemma~14.7]{V4}.  On the
  other hand, by \cite[Theorem~14.10, and page 58]{V4},
  $\AVC(\cL(\gamma))\subset \bcO$ only if $\gamma$ generate a $W$-module in the
  double cell of $\cO$.
\end{proof}

Now assume $S=S(\lambda)$. By \cite[Cor~5.30 b) and c)]{BVUni},
$[\sigma: \Ind_{W(S)}^{W}\bfone]=[\bfone|_{W(S)}:\sigma]\leq 1$.

}



\section{Remarks on the Counting theorem of unipotent representations}


In this section,  let $\Gc$ be a connected complex reductive group and $\fgg$ is
its Lie algebra. Fix a antiholomorphic involution $\sigma$ on $\Gc$ and a
corresponding Cartan involution $\theta$ of $\Gc$. Let $G$ be a finite central
extension of a open subgroup os $\Gc^\sigma$ and
\[
\pr \colon G \rightarrow \Gc^\sigma
\]
be the canonical projection.
Let $K = \pr^{-1}(\Gc^\sigma)$ and $W = W(\Gc)$.

Let $\hha$ be the abstract Cartan subalgebra of $\fgg$ and $\aX$ be the lattice
of abstract weight spaces. Let $\aR\subseteq \aX$, $\aRp$ and $\aQ$ be the
abstract root system, the set of positive roots and the root lattice. Let
\[
  \Con = \Set{\mu\in \hha^*| \begin{array}{l}\text{either $\inn{\Re(\mu)}{\ckalpha}>0$ or }\\
    \text{ $\inn{\Re(\mu)}{\ckalpha}=0$  and $\sqrt{-1}\inn{\Im(\mu)}{\ckalpha}>0$}
  \end{array}}
\]
and $\bCon$ be the closure of $\Con$ in  $\hha$.

We identify the set of infinitesimal characters with $\hha^*/W$.

\subsection{Coherent family}
For each finite dimensional $\fgg$-module or $\Gc$-module $F$, let
$F^*$ be its contragredient representation and let
$\WT{F}\subseteq \aX$ denote the multi-set of weights in $F$.

Let $\PiGlfin$ be the set of irreducible finite dimensional representations of $\Gc$
with extreme weight in $\Lambda_0$ and $\Glfin$ be the subgroup generated by $\PiGlfin$.
Let
\[
\aP  := \set{\mu \in \aX| \text{$\mu$ is a $\hha$-weight of an $F\in \PiGfin$}}.
\]
Via the highest weight theory,
every $W$-orbit $W\cdot \mu$ in $\aP$ corresponds with the irreducible finite dimensional representation
$F\in \PiGfin$ with extremal weight $\mu$.

Now the Grothendieck group $\Gfin$ of finite dimensional representation of $\Gc$
is identified with $\bZ[\aP/W]$. In fact $\Gfin$ is a $\bZ$-algebra under the
tensor product and equipped with the involution $F\mapsto F^*$.

Fix a $W$-invariant sub-lattice $\Lambda_0\subset \aX$ containing $\aQ$.

%  Let $\Pi$
% $\Glfin$ be the $\star$-invariant subalgebra of $\Gfin$ generated by irreducible
% representations corresponds to $\Lambda_0/W$.


For any $\lambda\in \hha^{*}$, we define
\begin{equation}
  \label{eq:wlam}
  \begin{split}
  [\lambda ]  &:= \lambda  +  \aQ,\\
  R_{[\lambda]} &:= \Set{\alpha\in \aR| \inn{\lambda}{\ckalpha}\in \bZ},\\
  W_{[\lambda]} &:= \braket{s_\alpha|\alpha\in \Rlam} \subseteq W,\\
  R_{\lambda} &:= \Set{\alpha\in \aR| \inn{\lambda}{\ckalpha}=0}, \AND\\
  W_{\lambda} &:= \braket{s_\alpha|\alpha\in R_{\lambda}} = \braket{w\in W|w\cdot \lambda = \lambda} \subseteq W.
  \end{split}
\end{equation}

For any  lattice  $\Lambda = \lambda + \Lambda_0 \in \fhh^*/\Lambda_{0}$ with $\lambda \in \bCon$,
we define
\[
  W_{\Lambda} := \set{w\in W | w\cdot \Lambda  = \Lambda}.
\]
Clearly, we have
\[
  W_{[\lambda]} < W_{\Lambda}, \quad \forall \ \lambda \in \Lambda.
\]



\begin{defn}
Suppose $\cM$ is an abelian group with $\Glfin$-action
\[
  \Glfin\times \cM \ni(F,m)\mapsto F\otimes m.
\]
In addition,  we fix a subgroup $\cM_{\barmu}$ of $\cM$ for each
 for each $W_{\Lambda}$-orbit $\barmu = W_{\Lambda} \cdot \mu\in \Lambda/W_{\Lambda}$.

A function $f\colon \Lambda \rightarrow \cM$ is called
  a coherent family based on $\Lambda$ if it satisfies
  $f(\mu)\in \cM_\mu$ and
  \[
  F\otimes f(\mu)  = \sum_{\nu \in \WT{F}} f(\mu+\nu) \qquad \forall \mu\in \Lambda, F\in \PiGlfin.
  \]
Let $\Cohlm$ be the abelian group of all coherent families based on $\Lambda$ and value in $\cM$.
\end{defn}

\def\Grt{\cG}

In this paper, we will consider the following cases.

\begin{eg}
Suppose $\cM = \bQ$ and $F\otimes m = \dim(F)\cdot m$ for $F\in \PiGlfin$ and $m\in \cM$.
We let $\cM_{\barmu} = \cM$ for every $\mu\in \Lambda$.
When $\Lambda = \Lambda_0$, the set of $W$-harmonic polynomials on $\hha^*$ is naturally
identified with $\Cohlm$ via restriction (Vogan's result)

\end{eg}

\begin{eg}
Let $\Grt(\fgg,K)$ be the Grothendieck group of finite length $(\fgg,K)$-modules
and $\Grt_{\chi}(\fgg,K)$ be the subgroup of $\Grt(\fgg,K)$ generated by the
set of irreducible $(\fgg,K)$-modules with infinitesimal character $\chi$.

Then $\Coh_\Lambda(\cG(\fgg,K))$ is the group of coherent families of Harish-Chandra modules.
The space $\Coh_\Lambda(\cG(\fgg,K))$ is equipped with a $\WLam$-action by
  \[
    w\cdot f(\mu) =  f(w^{-1} \mu) \qquad \forall \mu\in \Lambda, w\in \WLam,
    f \in \Coh_{\Lambda}(\Grt(\fgg,K)).
  \]
\end{eg}

\begin{eg}
  Fix a $\Gc$-invariant closed subset $\cZ$ in the nilpotent cone of $\fgg$. Let
  $\Grt_{\cZ}(\fgg,K)$ be the Grothendieck group of $(\fgg,K)$-modules
  whose complex associated varieties are contained in $\cZ$.
  We define
  \[
    \Grt_{\chi,\cZ}(\fgg,K) := \Grt_{\chi}(\fgg,K)\cap \Grt_{\cZ}(\fgg,k).
  \]

  Now $\Coh_{\Lambda}(\Grt_{\cZ}(\fgg,K))$ is also a $\WLam$-submodule of
  $\Coh_\Lambda(\cG(\fgg,K))$.
\end{eg}

\begin{eg}
  Fixing a Borel subalgebra $\fbb = \fhh\oplus \fnn \subset \fgg$, let
  $\cG(\fgg,\fhh,\fnn)$ be the Grothendieck group of the category $\cO$.
  The space $\cG_{\cZ}(\fgg,\fhh,\fnn)$ is defined similarly.
  The
  space $\Coh_\Lambda(\cG(\fgg,\fhh,\fnn))$ and $\Coh_{\Lambda(\cG)}$defined
  similarly.

%Note that the lattice $\Lambda$ is stable under the $\Wlam$ action.
We can define $W_{\Lambda}$ action on $\Coh_\Lambda(\cM)$ by
\[
   w\cdot f(\mu) =  f(w^{-1} \mu) \qquad \forall \mu\in \Lambda, w\in \WLam.
\]
\end{eg}

\begin{eg}
For each infinitesimal character $\chi$ and
a close $G$-invariant set $\cZ\in \cN_{\fgg}$.
Let $\Grt_{\chi,\cZ}(\fgg,K)$ be the Grothendieck group of $(\fgg,K)$-module
with infinitesimal character $\chi$ and complex associated variety contained
$\cZ$.
Similarly, let $\Grt_{\chi,\cZ}()$
\end{eg}


\def\Parm{\mathrm{Parm}}
\def\cof{\Theta}
\subsection*{Translation principal assumption}
Recall that we have fixed a set of simple roots of $W_\Lambda$.

We make the following assumption for $\Coh_\Lambda(\cM)$.
\begin{itemize}
\item There is a basis $\set{\cof_\gamma|\gamma\in \Parm}$ of
$\Coh_\Lambda(\cM)$ where $\Parm$ is a parameter set;
\item For every $\mu\in \Lambda$, the evaluation at $\mu$ is surjective;
  \[
    \ev{\mu}\colon \Coh_\Lambda(\cM)\rightarrow \cM_{\barmu} \qquad f\mapsto f(\mu)
  \]
  is surjective;
\item a subset $\tau(\gamma)$ of the simple roots of $W_\Lambda$ is attached
to each $\gamma\in \Parm$ such that $s\cdot \cof_\gamma = - \cof_\gamma$;
\item $\cof_\gamma(\mu) =0$ if and only if $\tau(\gamma)\cap R_\mu \neq \emptyset$.
\item $\set{\cof_\gamma(\mu)| \tau(\gamma)\cap R_\mu = \emptyset}$ form a basis of
$\cM_{\bargamma}$.
\end{itemize}

The translation principle assumption implies
\begin{equation}\label{eq:kevmu}
  \Ker\ev{\mu} = \sspan\set{{\cof}_{\gamma}|\tau(\gamma)\cap R_\mu \neq \emptyset } %= \Ker \ev{\mu}.
\end{equation}


\def\cohm{\Coh_\Lambda(\cM)}
Now we have the following counting lemma.
\begin{lem}
 For each $\mu$, we have
 \[
    \dim \cM_{\barmu}  = \dim (\cohm)_{W_\mu} = [\cohm, 1_{W_\mu}].
 \]
\end{lem}
\begin{proof}
  Clearly
  \[
  \sspan\set{\cof_\gamma - w\cof_\gamma | \gamma\in \Parm, w\in W_\Lambda} \subseteq \ker \ev{\mu}
  \]
  since $w\cdot \cof_\gamma(\mu) = \cof_\gamma(w^{-1}\cdot \mu)=\cof_\gamma(\mu)$ for $w\in W_\mu$.
  Combine this with \eqref{eq:kevmu}, we conclude that
  \[
   \sspan\set{\cof_\gamma - w\cof_\gamma | \gamma\in \Parm, w\in W_\Lambda} = \ker \ev{\mu}.
  \]
  Therefore, $\ev{\mu}$ induces an isomorphism $(\cohm)_{W_\mu}\rightarrow \cM_{\barmu}$.
  Now the dimension equality follows.
\end{proof}

\begin{eg}
  For the case of category $\cO$. We can take $\Parm  = W$.
  Let $\cof_w(\mu) = L(w\mu)$ for each $w\in W_\Lambda$ with
  $\tau(w) = \set{s_\alpha |\alpha\in \Delta^+, w\alpha \not\in R^+ }$.
\end{eg}

\begin{eg}
 In the Harish-Chandra module case,
 $\Parm$ consists of certain irreducible $K$-equivariant local systems on a $K$-orbit of the flag variety of $G$.
 $\tau(\gamma)$ is the $\tau$-invariants of the parameter $\gamma$.
\end{eg}
  % Clearly the evaluation map factor through
  % \[
  % (\cohm)_{W_\mu}  = \cohm / \braket{\cof_\gamma - w\cof_\gamma | \gamma\in \Parm, w\in W_\Lambda}.
  % \]
  %



\subsection{Primitive ideals and left cells}
In this section, we recall some results about the primitive ideals and cells developed by Barbasch Vogan, Joseph and Lusztig etc.

\subsection{Highest weight module}
Let $\fgg$ be a reductive Lie algebra, $\fbb = \fhh \oplus \fnn$ is a fixed Borel.
%Fix $\lambda \in \fhh^*$ dominant i.e. $\inn{\lambda}{\ckalpha} \geq 0 $
Let
\[
  M(\lambda)  := \cU(\fgg)\otimes_{\cU(\fbb)} \bC_{\lambda-\rho}
\]
and $L(\lambda) $ be the unique irreducible quotient of $M(\lambda)$.

In the rest of this section, we fix a lattice $\Lambda  \in   \fhh^*/ X^*$.
Let $\RLam$ and $\RLamp$ be the set of integral root system and the set of positive integral roots.
Write $\WLam$ for the integral Weyl group.


For $\mu\in \Lambda$, % is called if
\[
  \begin{array}{ccccc}
  \mu \succeq 0&\Leftrightarrow &\mu \text{ is dominant} &\Leftrightarrow&  \inn{\lambda}{\ckalpha}\geq 0 \quad \forall \alpha \in \RLamp \\
  \mu \not\sim 0&\Leftrightarrow& \mu \text{ is regular} &\Leftrightarrow&   \inn{\lambda}{\ckalpha}\neq 0 \quad \forall \alpha \in \RLamp
  \end{array}
\] %for all $\alpha \in R^+_{\Lambda}$and is called
regular if $\inn{\lambda}{\ckalpha}\neq 0$.

For each $w\in W$, it give a coherent family such that
\[
M_w(\mu) = M(w\mu) \quad \forall \mu \in \Lambda %\text{ dominant}
\]
Let $L_w$ be the unique coherent family such that $L_w(\mu) = L(w\mu)$ for any
regular dominant $\mu$ in $\Lambda$.
\trivial{Note that the Grothendieck group
  $\cK(\cO)$ of category $\cO$ is naturally embedded in the space of formal
  character. $M_w$ is a coherent family: the formal character $\ch M_w (\mu)$ of
  $M_w(\mu)$ is
  \[
    \ch M_{w}(\mu)=\frac{e^{w\mu-\rho}}{\prod_{\alpha\in R^+} (1-e^{-\alpha})}
    =\frac{e^{w\mu}}{\prod_{\alpha\in R^+} (e^{\alpha/2}-e^{-\alpha/2})}
  \]
  It is clear
  that $\ch M_w$ satisfies the condition for coherent continuation.

  From now on, we fix a regular dominant weight $\lambda \in \fhh^{*}$.
  Then $w[\lambda] = [w\cdot \lambda]$ for any $w\in W$.

 Now $W_{w[\lambda]} = w\, W_{[\lambda]}\, w^{-1}$.

 As $W_{[\lambda]}$-module, we have the following decomposition
 \[
   \begin{split}
     \Coh_{[\lambda]} &= \bigoplus_{r\in W/W_{[\lambda]}}
     \Coh_{r}\quad \text{with}\\
     \Coh_{r} &=\Coh_{r\lambda} := \set{\cof\in \Coh_{[\lambda]}| \cof(\lambda)\in \cO'_{[r\cdot \lambda]}}
   \end{split}
 \]

 Here $\cO'_{S}$ is the set of highest weight module whose $\fhh$-weights are in
 $S\subset \fhh^{*}$.

 % Here
 % \[
 %   \begin{split}
 %     \Coh_{r}
 %    % &:= \sspan_{\bZ}\set{M_{w}| w \in r\, W_{[\lambda]}}\\
 %     &= \set{\cof\in \Coh_{[\lambda]}| \cof(\lambda)\in \cO'_{[r\cdot \lambda]}}
 %   \end{split}
 % \]

 Note that the following map
 \[
   \begin{array}{ccccc}
     \bC[W] & \longrightarrow & \Coh_{[\lambda]} &\longrightarrow & \bC[S]\\
     w & \mapsto & (\mu\mapsto M_{w}(\mu)) & \mapsto & w\cdot \lambda
   \end{array}
 \]
 is $W_{[\lambda]}$-equivariant
 where $W_{[\lambda]}$ acts by right translation on $\bC[W]$ and $S = W\cdot \lambda$.
 The action of $W_{[\lambda]}$ on $S$ is by transport of structure and so
 $(a, w\,\lambda)\mapsto wa^{-1}\,  \lambda$.

 Now $\bC[r W_{[\lambda]}] \subset \bC[W]$ and
 $\bC[r W_{[\lambda]}\cdot \lambda]$
 are identified with $\Coh_{r}$.


}



% Fix $\lambda \in \fhh^{*}$ and regular dominant.

% As $W_{\lambda}$ module, $\Coh_{[\lambda]}$ can be


For each $w\in \WLam$, the function
\[
  \wtpp_w(\mu) := \rank (\cU(\fgg)/\Ann (L(w \mu))) \qquad \forall \mu \in \Lambda \text{ dominant}.
\]
extends to a Harmonic polynomial on $\fhh^*$.
In particular, $\wtpp_w\in P(\fhh^*) = S(\fhh)$.
\trivial[]{
  Let $\Lambda^+$ be the set of dominant weights in $\Lambda$.
Assume $\lambda \succ 0$, i.e. dominant regular. Then
$\R^0_\lambda = \empty$, $w^{w\lambda} = w^{-1}$. The set
\[
 \hat F_{w\lambda} = \set{\mu \in \Lambda |w^{-1}\mu \succeq 0 }
=  w \Lambda^+.
\]
Joseph's $p_w(\mu) = \rank \cU(\fgg)/(\Ann L(w\mu))$ is defined on $\hat F_{w\lambda} = w\Lambda^+$.
Now his $\wtpp_w(\mu) = w^{-1} p_w$ is defined on $\Lambda^+$ and given by
$\wtpp_w(\mu) = \rank \cU(\fgg)/(\Ann L(w\mu))$.
}
\def\PIP#1{\cP^{#1}}

There is a unique function $\aLam \colon \WLam \times \WLam\rightarrow \bQ$ such that
\[
L_w  = \sum_{w'\in \WLam} \aLam (w,w') M_{w'}
\]
\trivial[h]{
In fact, we have, for any $\mu \succ 0$,
\[
L(w\mu) = \sum_{} \aLam(w,w') M(w'\mu)  %\qquad \forall \mu \succ 0.
\]
The coefficient is independent of $\mu$. } Suppose $V$ is a module in $\cO$ whose
Gelfand-Kirillov dimension $\leq d$ Let $c(V)$ be the Bernstein degree. Let
$\Cint{\Lambda}^d$ be the submodule of $\Cint{\Lambda}$ generated by $L_w$ such
that the $\dim L(w\mu)\leq d$. Then the map
\[
\PIP{d}: \CLam^d\rightarrow S(\fhh) \qquad \Theta \mapsto (\mu \mapsto c(\Theta(\mu)) )
\]
is $\WLam$-equivariant. Let $c_w := c\circ (L_w(\mu))\in S(\fhh)$.

The following result of Joseph implies that the set of primitive ideals can be
parameterized by Goldie rank polynomials.
\begin{thm}[Joseph] %, Barbasch-Vogan]
  \label{thm:GR}
  {\color{red} correct formulation?}
  For $\mu$ dominant, the primitive ideal $\Ann L(w\mu) = \Ann L(w'\mu)$ if and only if $p_w = p_w'$. (\cite[ref?]{J12})
%   Let $r = \dim \fnn$.
%   The following holds:
%  \begin{enumT}
%   \item \label{it:j.1}
%   For $\mu$ dominant, the primitive ideal $\Ann L(w\mu) = \Ann L(w'\mu)$ if and only if $p_w = p_w'$. (\cite[ref?]{J12})
%  \end{enumT}
\end{thm}

\begin{thm}
 Let $r = \dim \fnn$.  Suppose $\dim L(w\nu) = d$.
 \begin{enumT}
  \item Let $x\in \fhh$ such that $\alpha(x) = 1$ for every simple roots $\alpha$.
  Then there are non-zero rational numbers $a,a'$ such that  \cite{J12}*{Theorem~5.1 and 5.7}
  \[
   a \, \wtpp_w(\mu)  = a' \, c_w(\mu) = \sum_{w'\in \WLam} \aLam (w,w') \inn{\mu}{w'^{-1} x}^{r-d}.
  \]
  \item The submodule $\bC \WLam \wtpp_w$ in $S(\fhh)$ is
  a special  irreducible $\WLam$-representation.
  Moreover all special representations of $\WLam$ occurs as a certain
  $\sigma(w)$.
  See \cite{BV1}*{Theorem~D}, \cite{BV2}*{Theorem~1.1}.
  \item Let $\sigma(w)$ be the submodule of $S(\fhh)$ generated by
  $\wtpp_w$. Then
  $\sigma(w)$ is irreducible and it
  occurs in $S^{r-d}(\fhh)$ with multiplicity $1$, see \cite{J.hw}*{5.3},
  % occurs in
  % $\wtpp_{w^{-1}}$ also generate $\sigma(w)$ and the module
 \end{enumT}
\end{thm}
By \Cref{thm:GR}, the module $\sigma(w)$ only depends on the primitive ideal $\cJ :=\Ann L(w\mu)$. So we can also write
$\sigma(\cJ):= \sigma(w)$.

\trivial[h]{
A representation $\sigma \in \widehat{W}$ is called univalent, if $\sigma$ occur in
$S^{n(\sigma)}(\fhh)$ with multiplicity one. Here $n(\sigma)$ is the minimal degree such that $\sigma$ occur in $S(\fhh)$.
}

From the above theorem, we see that the map $\PIP{d}$ factor through
$\CLam^{d}/\CLam^{d-1}$ and its image is contained in the space $\cH^{d}$ of degree $d$-Harmonic
polynomials.

Note that the associated variety of a primitive ideal $\cJ$ is the closure of
a nilpotent orbit in $\fgg$.
%$\Ann L_w (\mu)$ for $\mu\succ 0$
Now the associated variety of $\cJ$  is computed by the following result of Joseph (and include Hotta ?).
\begin{thm}
  %Suppose $J = \Ann L(w\mu)$
  Let $\cJ$ be a primitive ideal in $\cU(\fgg)$.
  Then $\sigma(\cJ)$ is in the image of Springer correspondence.
  Moreover, the associated variety of $\cJ$ is the orbit $\cO$
  corresponding to $\sigma(\cJ)$.
See \cite{J.av}*{Proposition~2.10}.
\end{thm}


\begin{lem}
  Let $\mu\in \Lambda$, $W_{\mu} = \set{w\in W_{\Lambda}| w\cdot \mu = \mu}$ and
  \[
  a_{\mu} = \max\set{a(\sigma)|  \widehat{\WLam} \ni\sigma \text{ occurs in } \Ind_{W_{\mu}}^{\WLam} 1}.
  \]
  Suppose $W_{\mu}$ is a parabolic subgroup of $\WLam$.
  % Then there is a unique special representation $\sigma_{\mu}$ such that
  % $a(\sigma_{\mu}) = a_{\mu}$.
  Then the set of all irreducible representations $\sigma$ occurs in
  $\Ind_{W_{\mu}}^{\WLam} 1$ such that $a(\sigma) = a_{\mu}$ forms a left cell
  given by
  \[
  \left(J_{W_{\mu}}^{\WLam} \sgn\right) \otimes \sgn.
  \]
\end{lem}


Now the maximal primitive ideal having infinitesimal character $\mu$
has the associated variety $\cO$.

\def\Wb{W_{b}}
\def\Wg{W_{g}}
\def\WcOb{W_{\ckcO_b}}
\def\WcOg{W_{\ckcO_g}}

Recall that $\WLam = \Wb\times \Wg$.
Let
\[
\begin{split}
  \sigma_{b} &= (j_{\WcOb}^{\Wb}\sgn)\otimes \sgn\\
  \sigma_{g} &= (j_{\WcOg}^{\Wg}\sgn)\otimes \sgn\\
\end{split}
\]
Then $\sigma_{b}\otimes \sigma_{g}$ is a special representation of $\WLam$.
Moreover $\cO$ corresponds to the representation
\[
 \sigma := j_{\Wb\times \Wg}^{W} \sigma_{b}\otimes \sigma_{g}
\]

% Then $W_{[\lambda_{\ckcO}]} = W'_n$ and the left cell is given by
% \[
% (J_{W_\ckcO}^{W'_n} \sgn) \otimes \sgn \quad \text{ with } \quad W_\ckcO = \prod_{i\in \bN^+} S_{\bfcc_{2i}(\ckcO)}.
% \]
% Here $W_\ckcO$ is an subgroup of $S_n$ and the embedding of $S_n$ in $W'_n$ is fixed.
% When $n$ is even,we the symbol of $J_{S_n}^{W'_n} \sgn$ is degenerate and we label it by ``$I$''.

\trivial{
  We recall some facts about the $j$-induction and $J$-induction.
  If $W'$ is a parabolic subgroup in $W$ then $j_{W'}^{W}$ maps special
  representation to special representation.
  It maps irreducible representation to irreducible ones.
  The $j$ induction has induction by stage \cite{Carter}*{11.2.4}.

  For classical group, the representations satisfies property B.
  When $W'$ is parabolic, the orbit $\cO$ corresponds to $j_{W'}^{W}$ is the orbit
  $\Ind_{L}^{G} \cO$.

  The $J$-induction maps left cell to left cell, which is only defined for
  parabolic subgroups.


  The $j$-induction for classical group is in \cite{Carter}*{Section 11.4}.
  Basically, $W_{n}$ has $D_{c_{i}}$, $B/C_{c_{i}}$ factors,
  \[
    j_{\prod_{i}D_{c_{i}}\times \prod_{j} B/C_{c_{j}}} \sgn  =
      ((c_{i}), (c_{j})).
  \]
  Here $(c_{i})$ denote the Young diagram with columns $c_{i}$ etc.

  Moreover, we have the following formula for $j$-induction.
  \[
    \ind_{A_{2m}}^{W_{2m}} = ((m),(m)) \qquad
    \ind_{A_{2m+1}}^{W_{2m+1}} = ((m),(m+1)).
  \]
}


\subsection{Harish-Chandra cells and primitive ideals}

\def\dphi{\rdd \phi}
\def\CPH{C(H)}
\def\whCPH{\widehat{C(H)}}

For simplicity we assume $G$ is connected in this section.

We recall McGovern and Casian's works on the Harish-Chandra cells.

Fix a $\theta$ stable Cartan subgroup $H=TA$ of $G$ such that $T$ is the maximal
compact subgroup of $H$ and $A$ is the split part of $H$.
% Let $\cO'_{\whH}$ be the category of $(\fgg, T)$-module such that the $\fnn$-action
% is locally nilpotent.
%
Let $\whH$ denote the set of irreducible representations of $\whH$ (also viewed as an
$(\fhh,T)$-module).
Let $\CPH$ be component group of $H$, which is also the component group of $T$.
Let $\rdd \colon \whH\longrightarrow \fhh^{*}$ be the map takes $\phi\in \whH$
to its derivative.
Then $\whH$ is a $\whCPH$-torsor over $\fhh^{*}$.
\trivial{
  This means for each $\lambda\in \fhh^{*}$, its fiber
  $\rdd^{-1}(\lambda)$ is either empty or with a set with free transitive
  $\whCPH$-action (the tensor product action).
}

\def\Prim{\mathrm{Prim}}
\def\leqL{\stackrel{L}{\leq}}
\def\leqR{\stackrel{R}{\leq}}
\def\leqLR{\stackrel{LR}{\leq}}
\trivial{
  The classification of Primitive ideals.
  \[
    \begin{array}{cccc}
      \Wlam &\longrightarrow &  \Prim_{\lambda}&\\
      w & \mapsto & I(w\lambda)& := \Ann L(w\lambda).
    \end{array}
  \]

  We now recall some results about the blocks in category $\cO$.

  Assume $\lambda\in \sfC$ where
  $\sfC = \set{\nu\in \fhh^{*}|\inn{\nu}{\ckalpha}>0}$ is the positive cone.
  Let $\lambda' = w_{0}\lambda$ which is negative.

  Let $w_{0}$ (resp. $w_{[\lambda]}$)be the longest element in $W$
  (resp. $\Wlam$).

  Define
  (We adapt the convention that $e\leqL w_{[\lambda]}$.)
  \[
    \begin{split}
      w_{1} \leqL w_{2}
      & \Leftrightarrow
      I(w_{1}\lambda')\subseteq I(w_{2}\lambda')\\
      & \Leftrightarrow
      I(w_{1}w_{0}\lambda)\subseteq I(w_{2}w_{0}\lambda)\\
      & \Leftrightarrow
      [L(w_{2}^{-1}\lambda'), L(w_{1}^{-1}\lambda')\otimes S(\fgg)] \neq 0\\
      % \quad \text{for some f.d. repn $F$}.\\
      % & \Leftrightarrow
      % [L(w_{2}^{-1}\lambda'), L(w_{1}^{-1}\lambda')\otimes F] \neq 0
      % \quad \text{for some f.d. repn $F$}.
    \end{split}
  \]

  Note that
  \[
    w_{1}\leqL w_{2} \Leftrightarrow w_{2}^{-1}\leqR w_{1}^{-1}.
  \]
  Therefore, the last condition implies that
  \[
    \begin{split}
      \sV^{R}(w) &:=
      \sspan\set{L(w'\lambda) | w'\leqR w}\\
     &  =\sspan\set{L(w' \lambda) | w^{-1}\leqL w'^{-1}}
    \end{split}
  \]
}




We also fix a positive system $\Delta^{+}(\fgg,\fhh)$ and let $\fnn$ be the
span of positive roots.
Let
\[
Q:= \bZ[\Delta(\fgg,\fhh)]\subset \fhh^{*}
\]
be the root lattice of $\fgg$, which is exactly the set of $\fhh$-weights occur in $S(\fgg)$.

By \cite{Vg}*{0.4.6}, we can naturally identify $Q$ with the subset of $\whH$
consisting the characters occurs in $S(\fgg)$.

For each $\lambda\in \fhh^{*}$, let
$\braket{\lambda} := W_{[\lambda]}\cdot \lambda$.
A set $S\subset \fhh^{*}$ is called a \emph{type} if
\[
  S = \bigcup_{\lambda\in S} \braket{\lambda}.
\]
Clearly, $\braket{\lambda}$ is the smallest type containing $\lambda$.


Suppose $S$ is a type,
we define the category $\cO'_{S}$ to be the category of $\fgg$-modules
such that $M\in \cO'_{S}$ if and only if
\begin{itemize}
  \item the $\fbb$-action on $M$ is locally finite;
  \item $M$ is finitely generated $\cU(\fgg)$-module,
  \item $M = \sum_{\mu \in S} M_{\mu}$
        where $M_{\mu}$ is the $\mu$-isotypic component of $M$.
\end{itemize}
Clearly, $\cO'_{S}$ is a union of blocks in $\cO'$.


For each set $\tS\subset \whH$, let
$[\tS] := \set{\phi\otimes \alpha| \phi \in \tS, \alpha\in Q}$ be the
$Q$-translations of $\tS$.
We say $\tS$ is $Q$-saturated if $\tS = \rdd^{{-1}}(\rdd(\tS))\cap [\tS]$.
We say $\tS$ is a type if $\rdd(S)$ is a type.
Suppose $\tS$ is $Q$-saturated and

For $\phi\in \whH$, we write
\[
\braket{\phi}:= [\phi]\cap \phi^{-1}(\braket{\rdd \phi})
\]
which is the smallest $W$-invariant $Q$-saturated subset in $\whH$ containing
$\phi$.
Let $M(\phi) := \rU(\fgg)\otimes_{\fbb,T} \phi$ where $\phi$ is inflated to a
$(\fbb,T)$-module.

Suppose $\tS$ is $Q$-saturated type,
we define the category $\cO'_{S}$ to be the category of $(\fgg,T)$-module
such that $M\in \cO'_{S}$ if and only if
\begin{itemize}
  \item the $\fbb$-action is locally finite;
  \item $M$ is finitely generated $\cU(\fgg)$-module,
  \item $M = \sum_{\mu \in \tS} M_{\mu}$
        where $M_{\mu}$ is the $\mu$-isotypic component of $M$.
\end{itemize}
In particular, $\cO'_{\whH}$ is the category of finitely generated $(\fgg,T)$-module
such that $\fbb$-acts locally finite.

Let $\cF\colon \cO'_{\whH}\rightarrow \cO'_{\fhh}$ be the forgetful functor of
the $T$-module structure.
The following lemma is clear.
\begin{lem}\label{lem:eqcat}
  The forgetful functor $\cF$ yields an equivalence of
  category
  \[
    \cO'_{\braket{\phi}}\rightarrow \cO'_{\braket{\rdd\phi}}
  \]
  which induces an isomorphism of $W_{[\dphi]}$-module
  \[
    \Coh_{\phi}\rightarrow \Coh_{\rdd\phi}.
  \]
\end{lem}
Suppose $\gamma\in \whCPH$.
By \Cref{lem:eqcat}, we have an equivalent of category
\[
  \cO'_{\braket{\phi}}\cong \cO'_{\braket{\rdd\phi}}\cong \cO'_{\braket{\phi\otimes \gamma}}
\]
which sends $M(\phi)$ to $M(\phi\otimes \gamma)$.
This isomorphism induces a $W_{[\dphi]}$-module isomorphism
\[
    \Coh_{\phi}\xrightarrow{\otimes \gamma} \Coh_{\phi\otimes \gamma}.
\]

\def\tCoh{\widetilde{\Coh}}
Let $\tCoh_{[\lambda]}:= \cF^{-1}(\Coh_{[\lambda]})$.

In summary, we conclude that
%\[
$
\tCoh_{[\lambda]}\xrightarrow{\cF} \Coh_{[\lambda]}
$
%\]
is a $\whCPH$-torsor over $\Coh_{[\lambda]}$.

Fix a set $\set{r_{1}, \cdots, r_{k}}$ of representatives of
$W/W_{[\lambda]}$ and fix
an element $\phi_{i}\in \rdd^{-1}(r_{i}\lambda)$ for each $r_{i}$.
Let $\Phi := \set{\phi_{i}|i=1, 2, \cdots, k}$.

Now we have an isomorphism of $W_{[\lambda]}$-module
\[
  \cF_{\Phi}\colon \bigoplus_{i}\Coh_{\phi_{i}} \longrightarrow %\xrightarrow{\ \ \cF\ \ }
  \bigoplus_{i} \Coh_{r_{i}\lambda} = \Coh_{[\lambda]}.
\]
Now
\[
  \Coh_{[\lambda]}\otimes \bZ[\whCPH]
  \xrightarrow{\ \ \cong \ \ } \tCoh_{[\lambda]}.
\]
given by $\Theta \otimes \alpha \mapsto
\cF_{\Phi}^{-1}(\Theta)\otimes \alpha$ is a
$W_{[\lambda]}\times \whCPH$-module isomorphism.



\medskip
\def\Grt{\cG}
\def\DeltaRp{\Delta^{+}_{\bR}}
\def\approxLR{\substack{LR}{\approx}}
Let $K = G^{\theta}$.


Recall Casian's result:
\begin{thm}[\cite{Cas}*{Proposition~2.10, Theorem~3.1}]\label{thm:L1}
  Let $H$ be a $\theta$-stable Cartan subgroup of $G$.
  Fix a positive system of real roots $\DeltaRp$ and a positive system
  $\Delta^{+}$ of roots such that $\DeltaRp\subseteq \Delta^{+}$.
  Let $\bfnn$ be the maximal nilptent Lie subalgebra of $\fgg$ with spanned
  by roots in $\Delta^{+}$.
 Let $M$  be a Harish-Chandra $(\fgg,K)$-module, then
  \begin{enumT}
    \item the Lie algebra cohomology $H^{q}(\fnn,M)$ is finite dimensional;
    \item
    the localization $\gamma_{\fnn}^{q}M$ is in the category $\cO'_{\whH}$;
    (see \cite{Cas}*{Section~1})
    \item
    $\Ann M \subseteq \Ann (\gamma_{\fnn}^{q}M)$.
    \item
  Then the localization functor induces a homomorphism
  \[
    \begin{array}{cccc}
      \gamma_{\fnn}: &\Grt_{\chi,\cZ}(\fgg,K) &\longrightarrow & \Grt_{\chi,\cZ}(\cO_{\whH})\\
      & M &\mapsto & \sum_{q}  (-1)^{q} \gamma^{q}_{\fnn} M
    \end{array}
  \]
  \end{enumT}
  % Fix an infinitesimal character $\chi$, and a close $G$-invariant set
  % $\cZ\in \cN_{\fgg}$.
  \qed
\end{thm}

\begin{thm}[\cite{Cas}*{Theorem~3.1}]
  Let $H_{1}, H_{2}, \cdots, H_{s}$ form a set of representatives of the
  conjugacy class of $\theta$-stable Cartan subgroup of $G$. Fix maximal
  nilpotent Lie subalgebra $\fnn_{i}$ for each $H_{i}$ as in \Cref{thm:L1}.
  Then
  \[
    \begin{array}{cccc}
      \gamma:=\oplus_{i} \gamma_{\fnn_{i}}: &\Grt_{\chi}(\fgg,K)
      &\longrightarrow & \bigoplus_{i} \Grt_{\chi}(\cO_{\whH_{i}})\\
    \end{array}
  \]
  is an embedding of $\WLam$-module.
\end{thm}
\begin{proof}
  We retain the notation in \cite{Cas}.
  Let $H^{rs}_{i}$ be the set of regular semisimple elements in $H_{i}$. By
  Harish-Chandra, taking the character of the elements induces an embedding of
  $\Grt_{\chi}(\fgg,K)$ into the space of analytic functions on
  $\bigsqcup_{i} H^{rs}_{i}$. Now \cite{Cas}*{Theorem~3.1} implies that the
  global character $\Theta M$ of an element $M\in \Grt_{\chi}(\fgg,K)$ is
  completely determined by the formal character $\mathrm{ch}(\gamma(M))$.
\end{proof}

\def\VHC{\sV^{\mathrm{HC}}}
\begin{cor}
  Fix an irreducible $(\fgg,K)$-module $\pi$ with infinitesimal character
  $\chi_{\lambda}$. Let $\VHC(\pi)$ be the Harish-Chandra cell representation
  containing $\pi$ and $\cD$ be the double cell in $\widehat{W_{[\lambda]}}$
  containing the special representation $\sigma(\pi)$attached to $\Ann(\pi)$.
  Then $[\sigma, \VHC(\pi)]\neq 0$ only if $\sigma \in \cD$.
  Moreover, $\sigma(\pi)$ always occures in $\VHC(\pi)$
\end{cor}
\begin{proof}
  The occurrence of $\sigma(\pi)$ is a result of King.

  Note that we have an embedding
  \[
    \gamma \colon \Grt_{\chi}(\fgg,K)\longrightarrow \bigoplus_{i}\Grt(\fgg,\fbb,\lambda).
  \]
  where the left hand sides is identified with a finite copies of
  $\bC[W_{[\lambda]}]$.

  Since $\Ann(\pi)\subseteq \Ann (\gamma_{\fnn}^{q}(\pi))$,
  we conclude that $[\sigma, \VHC(\pi)]\neq 0$ implies that
  $\sigma(\pi)\leqLR \sigma$.

  By the Vogan duality, $\cD\otimes \sgn$ is also a Harish-Chandra cell.
  So we have $\sigma(\pi)\otimes \sgn \leqLR \sigma\otimes \sgn$.

  Therefore, $\sigma(\pi)\approxLR$
\end{proof}


\subsection{Primitive ideals for $\fgg=\fsp(2n,\bC)$ at half integral infinitesimal character}

Let $\ckcO$ be a metaplectic ``good'' nilpotent orbit in $\fsp(2n,\bC)$, i.e.
$\bfrr_i(\ckcO)$ is even for each $i\in \bN^+$.

%We write $ \bfcc_{2i} (\ckcO)  =: 2 c_i + \epsilon_i =: C_i$ such that $\epsilon_i \in \set{0,1}$.
Then $W_{[\lambda_{\ckcO}]} = W'_n$ and the left cell is given by
\[
(J_{W_\ckcO}^{W'_n} \sgn) \otimes \sgn \quad \text{ with } \quad W_\ckcO = \prod_{i\in \bN^+} S_{\bfcc_{2i}(\ckcO)}.
\]
Here $W_\ckcO$ is an subgroup of $S_n$ and the embedding of $S_n$ in $W'_n$ is fixed.
When $n$ is even,we the symbol of $J_{S_n}^{W'_n} \sgn$ is degenerate and we label it by ``$I$''.

\def\PPtC{\PP{\tC}}
 Let
 \[
  \PPtC(\ckcO) = \set{(2i-1,2i+2)|i\in \bN^+, \bfrr_{2i-1}(\ckcO)\neq \bfrr_{2i}(\ckcO)}.
 \]
  For each $\wp\subset \PPtC(\ckcO)$, let $\tau_\wp := (\imath_\wp,\jmath_\wp)$ be the bipartition
  given by
  \[
    (\bfrr_i(\imath_\wp),\bfrr_i(\jmath_\wp)) =\begin{cases}
      (\frac{\bfrr_{2i-1}(\ckcO)}{2}, \frac{\bfrr_{2i}(\ckcO)}{2}),   & \text{if } (2i-1,2i)\notin \wp, \\
      (\frac{\bfrr_{2i}(\ckcO)}{2}, \frac{\bfrr_{2i-1}(\ckcO)}{2}), & \text{otherwise.}
    \end{cases}
  \]
  If $\tau_\wp$ is degenerate, it represent the element in $\widehat{W'_n}$
  with label $I$.
  Let $\wp^c$ be the complement of $\wp$ in $\PPtC(\ckcO)$.
  Then $\tau_\wp$ and $\tau_{\wp^c}$ represent the same irreducible $W'_n$-module.

  \def\LCC{{}^L\sC'}
Using the induction formula in \cite{L}*{(4.6.6) (4.6.7)}, we have the following lemma.
\begin{lem}
The representation
\[
  \LCC(\ckcO) :=J_{W_\ckcO}^{W'_n} \sgn
\]
is multiplicity free.
The map
\[
  \bZ[\PPtC(\ckcO)]/\sim \longrightarrow \set{\tau\in \widehat{W'_n}| \tau\subset \LCC(\ckcO)}
  \qquad \wp \mapsto \tau_\wp
\]
is a bijection where $\sim$ is the equivalent relation identifying $\wp$ with $\wp^c$.
% irreducible components in $\LCC(\ckcO)$ are in one-one correspond
% to the quotient of  by the equivalent relation $\wp\sim \wp^c$
Moreover, the special representation in $\LCC(\ckcO)$ is $\tau_\emptyset$.
\qed
\end{lem}

In the following, we write
\[
\tau_{\ckcO}:=\tau_\emptyset
\]
for the unique special
representation in $\LCC(\ckcO)$.

\trivial[h]{
 Use the formula, if $\ckcO$ has only two rows $[2r_1,r_2]$.
 Then $W_{\ckcO} = \underbrace{S_2\times \cdots \times S_2}_{r_2} \times
 \underbrace{ S_1 \times \cdots \times S_1}_{r_1-r_2}$ and the claim is clear.

 The corresponding symbol is
 \[
  \binom{r_1}{r_2}
 \]

 We prove by induction on the number of rows.
 Suppose $\ckcO  = [2k+2r_1, 2r+2r_2, \ckcO'] $ where $2k$ is the number of columns in $\ckcO'$.
 If $r_1=r_2 =0$, then
 the symbol of $\ckcO$ are given by
 \[
  \binom{a_1,a_2, \cdots, a_k, 2k+r_1}{b_1, b_2, \cdots, b_k, 2k+r_2 }
  \text{ or }
  \binom{a_1,a_2, \cdots, a_k, 2k+r_2}{b_1, b_2, \cdots, b_k, 2k+r_1 }.
 \]
where
\[
  \binom{a_1,a_2, \cdots, a_k}{b_1, b_2, \cdots, b_k}
\]
is the symbol attached to $\ckcO'$.

This gives the claim.
}


Let $\tau_\ckcO  = (\imath,\jmath)$ such that $\bfrr_i(\imath)\leq \bfrr_i(\jmath)$ for all $i\in \bN^+$.
Then the unique special representation
in $\Ind_{W_\ckcO}^{W'_n} 1$ where the $a$ function take maximal value is given
by the bipartition
\[
  \tau_{\cO}:=(\jmath^t,\imath^t).
\]
Let $\sigma'(\cO)$ denote the $\tau_{\cO}$-isotypic component of $W'_n$-harmonic polynomials in $S(\fhh)$.
Comparing the  fake degree formulas of type $C$ and $D$ (see \cite{Carter}*{Proposition 11.4.3, 11.4.4}), we conclude that
the $W_n$-representation
\begin{equation}\label{eq:sigma.tC}
\sigma(\cO):= \bC[W_n]\cdot \sigma'(\cO)
\end{equation}
is irreducible whose type is also given by the
bipartition $\tau_{\cO}$. The type of $\sigma(\cO)$ is $j_{W'}^{W} \sigma'(\cO)$.

\begin{lem}\label{lem:MD1}
  Suppose $\sigma'$ is a $W'_n$-special representation.% and $\sigma = j_{W'_n}^{W_n}$.
  Let $\ckcO'$ be the partition of type $D$ corresponds to the $W'_n$-special representation
  $\sigma'\otimes \sgn_D$. Then $\sigma:=j_{W'_n}^{W_n} \sigma'$ corresponds to the partition $\ckcO'^t$ under the Springer correspondence of
  type C.
\end{lem}
\begin{proof}
  %By the algorithm computing the Springer correspondence, it is clear that
  First note that $\ckcO'$ is a specail nilpotent orbit of type $D_n$,
  therefore $\ckcO'^t$ is a partition of type $C_n$ (see \cite{CM}*{Proposition~6.3.7}).

  By the explicitly formula of the Lusztig-Spaltenstein duality, we known that the D-collapsing
  $(\ckcO'^t)_D$ of $\ckcO'^t$ corresponds to the special representation
  $\sigma'$ of type $D$.

  The Springer correspondence algorithm for classical groups can be naturally extends to all partitions.
  Sommers showed that two partitions are mapped to the same Weyl group representations
  if and only if they has the same $D$-collapsing \cite{So}*{Lemma~9}.
  Note that the algorithms computing the Springer correspondence for type C and D are essentially the same
  (see \cite{Carter}*{Section~13.3} or \cite{So}*{Section~7}).
  By the injectivity of the Springer correspondence, we conclude that the type $C$ partition $\ckcO'^t$
  must corresponds to $\sigma$.
  \trivial[h]{
    Two partition $\lambda\sim_X \mu$ if they have the same $X$-collapsing.
  Sommers showed that two partition $\lambda$ and $\mu$ maps to the same module $E_\lambda = E_\mu$
  if they have the same X-collapse. But the Springer correspondence is injective,
  so we have the map $\sP/\sim_X \xrightarrow{1-1} \sP_X \hookrightarrow \widehat{W}$
  which gives the claim.
  }
\end{proof}



%As a corrollary, we can obtain the following simple formule for
The following lemma is a direct consequence of \Cref{lem:MD1}.
\begin{lem}
  Suppose $\ckcO$ has good parity of type $\tC$. Under the Springer
  correspondence of type $C$, the representation $\sigma(\ckcO)$ (see
  \eqref{eq:sigma.tC}) corresponds to the nilpotent orbit $\cO$ defined by
  \[
    (\bfcc_{2i-1}(\cO),\bfcc_{2i}(\cO)) =\begin{cases}
      (\bfrr_{2i-1}(\ckcO),\bfrr_{2i}(\ckcO)), & \text{if } \bfrr_{2i-1}(\ckcO),\bfrr_{2i}(\ckcO)\\
      (\bfrr_{2i-1}(\ckcO)-1,\bfrr_{2i}(\ckcO)+1), & \text{otherwise} \\
    \end{cases}
  \]
  for all $i\in \bN^+$.
\end{lem}
\begin{proof}
  Apply the algorithm of Springer correspondence of type D to the representation $\sigma'(\ckcO)$ gives the above formula.
\end{proof}
\trivial{
A technical point, the representation of $W'_n$ is given by symbol $\binom{\xi}{\mu}$ or
$\binom{\mu}{\xi}$. However, using the Springer correspondence formula, only one of the arrangement
can give a valid type D partition.

It is a interesting fact that a type D orbit $\cO$ is special if and only if $\cO^t$ is of type C.



}



\def\tdBV{\tdd_{\text{BV}}}
\begin{lem}
Suppose $\ckcO = \ckcO_{b}\cup \ckcO_{g}$.
Then $\cO = \cO_{b} \cup \cO_{g}$ where $\cO_{b} = \ckcO_{b}^{t}$ and
$\cO_{g}=\tdBV(\ckcO_{g})$.
\end{lem}

% Let $\ckcO'_{g} = (\ckcO_{g})_{D}$. Then $\sigma_{D}(\ckcO_{g}) = \sigma_{D}(\ckcO'_{g})$.
% Let $\sigma_{\ckcO_{g}} = \sigma_{\ckcO'_{g}}$
% Let $\ckcO' = \ckcO_{b}\cup \ckcO'_{g}$.

\trivial[]{
  We take the convention that $2\cO = [2r_{i}]$ if $\cO = [r_{i}]$.
  We also write $[r_{i}]\cup [r_{j}] = [r_{i},r_{j}]$.
  $\dagger \cO = [r_{i}+1]$.

We suppose
\[
\ckcO_{b} = [2r_{1}+1, 2r_{1}+1, \cdots, 2r_{k}+1,2r_{k}+1]
= (2c_{0},2c_{1},2c_{1}, \cdots, 2c_{l}, 2c_{l})
\]
where $l = r_{1}$.

Now
\[
\begin{split}
  W_{\ckcO_{b}} &= W_{c_{0}} \times S_{2c_{1}} \times S_{2c_{2}}\times \cdots \times S_{2c_{l}}\\
  \cksigma_{b} &:= j_{W_{\ckcO_{b}}}^{W_{b}} \sgn = ((c_{1},c_{2},\cdots, c_{k}),(c_{0},c_{1}, \cdots, c_{l}))\\
  & = ([r_{1},r_{2},\cdots, r_{k}],[r_{1}+1,r_{2}+1,\cdots,r_{k}+1])
\end{split}
\]
Therefore
\[
\sigma_{b} = \cksigma_{b}\otimes \sgn = ((r_{1}+1,r_{2}+1,\cdots,r_{k}+1),(r_{1},r_{2},\cdots, r_{k}))
\]
which corresponds to the orbit
\[
  \cO_{b} = (2r_{1}+1, 2r_{1}+1,2r_{2}+1, 2r_{2}+1,  \cdots,2r_{k}+1, 2r_{k}+1 ) = \ckcO_{b}^{t}.
\]
This implies
\[
  \sigma_{b} = j_{W_{L_{b}}}^{W_{b}}\sgn, \quad \text{where } W_{L,b} = \prod_{i=1}^{k} S_{2r_{i}+1}.
\]
(Note that $\cO'_{b} = (2r_{1}+1,2r_{2}+1, \cdots, 2r_{k}+1)$ which corresponds
to $j_{W_{L_{b}}}^{S_{b}}\sgn$ and $\ind_{L}^{G} \cO'_{b} = \cO_{b}$.
)

Now we deduce that
\[
  \begin{split}
    \sigma &:= j_{W_{b}\times W_{g}}^{W_{n}} \sigma_{b}\otimes \sigma_{g}\\
    & = j_{W_{L_{b}}\times W_{g}}^{W_{n}} \sgn \otimes \sigma_{g}
  \end{split}
\]
where $W_{L_{b}}\times W_{g}$ is a parabolic subgroup of $W_{n}$
corresponds to the Levi factor $L$ of type
\[
A_{2r_{1}+1}\times A_{2r_{2}+1}\times \cdots \times A_{2r_{k}+1} \times W_{g}.
\]
Therefore
\[
\cO = \ind_{L}^{G} \triv\times \cO_{g} = \cO_{b}\cup \cO_{g}.
\]


% Note that
% \[
%   j_{S_{b}}^{W_{b}} 1
%   = (j_{S_{b}}^{W_{b}}\sgn)\otimes \sgn
%   = (J_{W_{\ckcO_{b}}}^{W_{b}} \sgn)\otimes
%   \sgn
% \]

% It suffice to compute
% \[
%   (J_{W_{\ckcO_{b}}\times W_{\ckcO_{g}}}^{\WLam} \sgn)\otimes \sgn
%   = (J_{W_{b}\times W_{g}}^{W} \sigma_{b} \otimes \sigma_{g}) \otimes \sgn
% \]
%
\def\ckfll{\check\fll}
\def\ckfgg{\check\fgg}

We claim that the map $\tdd\colon \ckcO \mapsto \cO$ defined here
coincide with our Metaplectic BV duality paper.

Note that $\tdBV$ is compatible with parabolic induction.
Suppose $\ckfll$ is a Levi subgroup of $\ckfgg$ and
$\ckcO_{\ckfll}:=\ckcO\cap \ckfll\neq \emptyset$.
Then
\[
\tdBV(\ckcO) = \ind_{\fll}^{\fgg} \tdBV(\ckcO_{\ckfll}).
\]
(Since $\tdBV$ commute with the descent map, the claim follows from the
prosperity of $\dBV$)

The map $\tdd$ also compatible with parabolic induction.
It suffice to consider the case where $\ckfll$ is a maximal parabolic of type
$A_{l}\times C_{n}$ and
the orbit is trivial on the $A_{l}$ factor.
Suppose $l$ is the bad parity, the claim is clear by our computation for the bad
parity case.

Now suppose $l=2m$ has good parity. If there is a $i$ such that
$\bfrr_{2i+1}(\ckcO)=\bfrr_{2i+2}(\ckcO)=2m$. Then
$\bfcc_{2i+1}(\cO)=\bfcc_{2i+2}(\cO)=2m$ and the claim follows.

Otherwise, we can assume
\[
R_{2i-1}:=\bfrr_{2i-1}(\ckcO)> \bfrr_{2i}(\ckcO)=\bfrr_{2i+1}(\ckcO) > \bfrr_{2i+2}(\ckcO)=:R_{2i+2}.
\]
and then
\[
\cO = (\cdots, R_{2i-1}-1, 2m+1, 2m-1, R_{2i+2}, \cdots)
\]
One check again that $\cO = \ind_{\fll}^{\fgg}\cO_{\fll}$.
(Note that the induction operation is add two length $2m$ columns and then apply
C-collapsing.)

Now it suffice to check that
$\tdd(\ckcO) = \tdBV(\ckcO)$ for every orbits $\ckcO$ whose
rows are multiplicity free) (i.e. $\ckcO$ is distinguished).
This is clear by the explicit formula for the both sides.
}



% \subsection{Embedding Harish-Chandra cells to cells in $\sO$}
% Let $H$ be a Cartan subgroup of $G$ and $T$ is the maximal compact subgroup of
% $H$. %Let $(\fgg, T)$-module.
% Let $\cO_{H}$ be the category


\begin{bibdiv}
  \begin{biblist}
% \bib{AB}{article}{
%   title={Genuine representations of the metaplectic group},
%   author={Adams, Jeffrey},
%   author = {Barbasch, Dan},
%   journal={Compositio Mathematica},
%   volume={113},
%   number={01},
%   pages={23--66},
%   year={1998},
% }

\bib{Ad83}{article}{
  author = {Adams, J.},
  title = {Discrete spectrum of the reductive dual pair $(O(p,q),Sp(2m))$ },
  journal = {Invent. Math.},
  number = {3},
 pages = {449--475},
 volume = {74},
 year = {1983}
}

%\bib{Ad07}{article}{
%  author = {Adams, J.},
%  title = {The theta correspondence over R},
%  journal = {Harmonic analysis, group representations, automorphic forms and invariant theory,  Lect. Notes Ser. Inst. Math. Sci. Natl. Univ. Singap., 12},
% pages = {1--39},
% year = {2007}
% publisher={World Sci. Publ.}
%}


\bib{ABV}{book}{
  title={The Langlands classification and irreducible characters for real reductive groups},
  author={Adams, J.},
  author={Barbasch, D.},
  author={Vogan, D. A.},
  series={Progress in Math.},
  volume={104},
  year={1991},
  publisher={Birkhauser}
}

\bib{AC}{article}{
  title={Algorithms for representation theory of
    real reductive groups},
  volume={8},
  DOI={10.1017/S1474748008000352},
  number={2},
  journal={Journal of the Institute of Mathematics of Jussieu},
  publisher={Cambridge University Press},
  author={Adams, Jeffrey},
  author={du Cloux, Fokko},
  year={2009},
  pages={209-259}
}

\bib{ArPro}{article}{
  author = {Arthur, J.},
  title = {On some problems suggested by the trace formula},
  journal = {Lie group representations, II (College Park, Md.), Lecture Notes in Math. 1041},
 pages = {1--49},
 year = {1984}
}


\bib{ArUni}{article}{
  author = {Arthur, J.},
  title = {Unipotent automorphic representations: conjectures},
  %booktitle = {Orbites unipotentes et repr\'esentations, II},
  journal = {Orbites unipotentes et repr\'esentations, II, Ast\'erisque},
 pages = {13--71},
 volume = {171-172},
 year = {1989}
}

\bib{AK}{article}{
  author = {Auslander, L.},
  author = {Kostant, B.},
  title = {Polarizations and unitary representations of solvable Lie groups},
  journal = {Invent. Math.},
 pages = {255--354},
 volume = {14},
 year = {1971}
}


\bib{B.Uni}{article}{
  author = {Barbasch, D.},
  title = {Unipotent representations for real reductive groups},
 %booktitle = {Proceedings of ICM, Kyoto 1990},
 journal = {Proceedings of ICM (1990), Kyoto},
   % series = {Proc. Sympos. Pure Math.},
 %   volume = {68},
     pages = {769--777},
 publisher = {Springer-Verlag, The Mathematical Society of Japan},
      year = {2000},
}

\bib{BV1}{article}{
   author={Barbasch, Dan},
   author={Vogan, David},
   title={Primitive ideals and orbital integrals in complex classical
   groups},
   journal={Math. Ann.},
   volume={259},
   date={1982},
   number={2},
   pages={153--199},
   issn={0025-5831},
   review={\MR{656661}},
   doi={10.1007/BF01457308},
}

\bib{BV2}{article}{
   author={Barbasch, Dan},
   author={Vogan, David},
   title={Primitive ideals and orbital integrals in complex exceptional
   groups},
   journal={J. Algebra},
   volume={80},
   date={1983},
   number={2},
   pages={350--382},
   issn={0021-8693},
   review={\MR{691809}},
   doi={10.1016/0021-8693(83)90006-6},
}

\bib{BV.W}{article}{
  author={Barbasch, Dan},
  author={Vogan, David},
  editor={Trombi, P. C.},
  title={Weyl Group Representations and Nilpotent Orbits},
  bookTitle={Representation Theory of Reductive Groups:
    Proceedings of the University of Utah Conference 1982},
  year={1983},
  publisher={Birkh{\"a}user Boston},
  address={Boston, MA},
  pages={21--33},
  %doi={10.1007/978-1-4684-6730-7_2},
}



\bib{B.Orbit}{article}{
  author = {Barbasch, D.},
  title = {Orbital integrals of nilpotent orbits},
 %booktitle = {The mathematical legacy of {H}arish-{C}handra ({B}altimore,{MD}, 1998)},
    journal = {The mathematical legacy of {H}arish-{C}handra, Proc. Sympos. Pure Math.},
    %series={The mathematical legacy of {H}arish-{C}handra, Proc. Sympos. Pure Math},
    volume = {68},
     pages = {97--110},
 publisher = {Amer. Math. Soc., Providence, RI},
      year = {2000},
}



\bib{B10}{article}{
  author = {Barbasch, D.},
  title = {The unitary spherical spectrum for split classical groups},
  journal = {J. Inst. Math. Jussieu},
% number = {9},
 pages = {265--356},
 volume = {9},
 year = {2010}
}



\bib{B17}{article}{
  author = {Barbasch, D.},
  title = {Unipotent representations and the dual pair correspondence},
  journal = {J. Cogdell et al. (eds.), Representation Theory, Number Theory, and Invariant Theory, In Honor of Roger Howe. Progress in Math.},
  %series ={Progress in Math.},
  volume = {323},
  pages = {47--85},
  year = {2017},
}

\bib{BVUni}{article}{
 author = {Barbasch, D.},
 author = {Vogan, D. A.},
 journal = {Annals of Math.},
 number = {1},
 pages = {41--110},
 title = {Unipotent representations of complex semisimple groups},
 volume = {121},
 year = {1985}
}

\bib{Br}{article}{
  author = {Brylinski, R.},
  title = {Dixmier algebras for classical complex nilpotent orbits via Kraft-Procesi models. I},
  journal = {The orbit method in geometry and physics (Marseille, 2000). Progress in Math.}
  volume = {213},
  pages = {49--67},
  year = {2003},
}

\bib{Carter}{book}{
   author={Carter, Roger W.},
   title={Finite groups of Lie type},
   series={Wiley Classics Library},
   %note={Conjugacy classes and complex characters;
   %Reprint of the 1985 original;
   %A Wiley-Interscience Publication},
   publisher={John Wiley \& Sons, Ltd., Chichester},
   date={1993},
   pages={xii+544},
   isbn={0-471-94109-3},
   %review={\MR{1266626}},
}

\bib{Cas}{article}{
   author={Casian, Luis G.},
   title={Primitive ideals and representations},
   journal={J. Algebra},
   volume={101},
   date={1986},
   number={2},
   pages={497--515},
   issn={0021-8693},
   review={\MR{847174}},
   doi={10.1016/0021-8693(86)90208-5},
}

\bib{Ca89}{article}{
 author = {Casselman, W.},
 journal = {Canad. J. Math.},
 pages = {385--438},
 title = {Canonical extensions of Harish-Chandra modules to representations of $G$},
 volume = {41},
 year = {1989}
}



\bib{Cl}{article}{
  author = {Du Cloux, F.},
  journal = {Ann. Sci. \'Ecole Norm. Sup.},
  number = {3},
  pages = {257--318},
  title = {Sur les repr\'esentations diff\'erentiables des groupes de Lie alg\'ebriques},
  url = {http://eudml.org/doc/82297},
  volume = {24},
  year = {1991},
}

\bib{CM}{book}{
  title = {Nilpotent orbits in semisimple Lie algebra: an introduction},
  author = {Collingwood, D. H.},
  author = {McGovern, W. M.},
  year = {1993},
  publisher = {Van Nostrand Reinhold Co.},
}


% \bib{Dieu}{book}{
%    title={La g\'{e}om\'{e}trie des groupes classiques},
%    author={Dieudonn\'{e}, Jean},
%    year={1963},
% 	publisher={Springer},
%  }

\bib{DKPC}{article}{
title = {Nilpotent orbits and complex dual pairs},
journal = {J. Algebra},
volume = {190},
number = {2},
pages = {518 - 539},
year = {1997},
author = {Daszkiewicz, A.},
author = {Kra\'skiewicz, W.},
author = {Przebinda, T.},
}

\bib{DKP2}{article}{
  author = {Daszkiewicz, A.},
  author = {Kra\'skiewicz, W.},
  author = {Przebinda, T.},
  title = {Dual pairs and Kostant-Sekiguchi correspondence. II. Classification
	of nilpotent elements},
  journal = {Central European J. Math.},
  year = {2005},
  volume = {3},
  pages = {430--474},
}


\bib{DM}{article}{
  author = {Dixmier, J.},
  author = {Malliavin, P.},
  title = {Factorisations de fonctions et de vecteurs ind\'efiniment diff\'erentiables},
  journal = {Bull. Sci. Math. (2)},
  year = {1978},
  volume = {102},
  pages = {307--330},
}

%\bibitem[DM]{DM}
%J. Dixmier and P. Malliavin, \textit{Factorisations de fonctions et de vecteurs ind\'efiniment diff\'erentiables}, Bull. Sci. Math. (2), 102 (4),  307-330 (1978).



%\bib{Du77}{article}{
% author = {Duflo, M.},
% journal = {Annals of Math.},
% number = {1},
% pages = {107-120},
% title = {Sur la Classification des Ideaux Primitifs Dans
%   L'algebre Enveloppante d'une Algebre de Lie Semi-Simple},
% volume = {105},
% year = {1977}
%}

\bib{Du82}{article}{
 author = {Duflo, M.},
 journal = {Acta Math.},
  volume = {149},
 number = {3-4},
 pages = {153--213},
 title = {Th\'eorie de Mackey pour les groupes de Lie alg\'ebriques},
 year = {1982}
}



\bib{GZ}{article}{
author={Gomez, R.},
author={Zhu, C.-B.},
title={Local theta lifting of generalized Whittaker models associated to nilpotent orbits},
journal={Geom. Funct. Anal.},
year={2014},
volume={24},
number={3},
pages={796--853},
}

\bib{EGAIV2}{article}{
  title = {\'El\'ements de g\'eom\'etrie alg\'brique IV: \'Etude locale des
    sch\'emas et des morphismes de sch\'emas. II},
  author = {Grothendieck, A.},
  author = {Dieudonn\'e, J.},
  journal  = {Inst. Hautes \'Etudes Sci. Publ. Math.},
  volume = {24},
  year = {1965},
}


\bib{EGAIV3}{article}{
  title = {\'El\'ements de g\'eom\'etrie alg\'brique IV: \'Etude locale des
    sch\'emas et des morphismes de sch\'emas. III},
  author = {Grothendieck, A.},
  author = {Dieudonn\'e, J.},
  journal  = {Inst. Hautes \'Etudes Sci. Publ. Math.},
  volume = {28},
  year = {1966},
}


\bib{HLS}{article}{
    author = {Harris, M.},
    author = {Li, J.-S.},
    author = {Sun, B.},
     title = {Theta correspondences for close unitary groups},
 %booktitle = {Arithmetic Geometry and Automorphic Forms},
    %series = {Adv. Lect. Math. (ALM)},
    journal = {Arithmetic Geometry and Automorphic Forms, Adv. Lect. Math. (ALM)},
    volume = {19},
     pages = {265--307},
 publisher = {Int. Press, Somerville, MA},
      year = {2011},
}

\bib{HS}{book}{
 author = {Hartshorne, R.},
 title = {Algebraic Geometry},
publisher={Graduate Texts in Mathematics, 52. New York-Heidelberg-Berlin: Springer-Verlag},
year={1983},
}

\bib{He}{article}{
author={He, H.},
title={Unipotent representations and quantum induction},
journal={arXiv:math/0210372},
year = {2002},
}

\bib{HL}{article}{
author={Huang, J.-S.},
author={Li, J.-S.},
title={Unipotent representations attached to spherical nilpotent orbits},
journal={Amer. J. Math.},
volume={121},
number = {3},
pages={497--517},
year={1999},
}


\bib{HZ}{article}{
author={Huang, J.-S.},
author={Zhu, C.-B.},
title={On certain small representations of indefinite orthogonal groups},
journal={Represent. Theory},
volume={1},
pages={190--206},
year={1997},
}



\bib{Howe79}{article}{
  title={$\theta$-series and invariant theory},
  author={Howe, R.},
  book = {
    title={Automorphic Forms, Representations and $L$-functions},
    series={Proc. Sympos. Pure Math},
    volume={33},
    year={1979},
  },
  pages={275-285},
}

\bib{HoweRank}{article}{
author={Howe, R.},
title={On a notion of rank for unitary representations of the classical groups},
journal={Harmonic analysis and group representations, Liguori, Naples},
pages={223-331},
year={1982},
}

\bib{Howe89}{article}{
author={Howe, R.},
title={Transcending classical invariant theory},
journal={J. Amer. Math. Soc.},
volume={2},
pages={535--552},
year={1989},
}

\bib{Howe95}{article}{,
  author = {Howe, R.},
  title = {Perspectives on invariant theory: Schur duality, multiplicity-free actions and beyond},
  journal = {Piatetski-Shapiro, I. et al. (eds.), The Schur lectures (1992). Ramat-Gan: Bar-Ilan University, Isr. Math. Conf. Proc. 8,},
  year = {1995},
  pages = {1-182},
}


\bib{J12}{article}{
   author={Joseph, A.},
   title={Goldie rank in the enveloping algebra of a semisimple Lie algebra.
   I, II},
   journal={J. Algebra},
   volume={65},
   date={1980},
   number={2},
   pages={269--283, 284--306},
   issn={0021-8693},
   review={\MR{585721}},
   doi={10.1016/0021-8693(80)90217-3},
}

\bib{J3}{article}{
   author={Joseph, A.},
   title={Goldie rank in the enveloping algebra of a semisimple Lie algebra.
   III},
   journal={J. Algebra},
   volume={73},
   date={1981},
   number={2},
   pages={295--326},
   issn={0021-8693},
   review={\MR{640039}},
   doi={10.1016/0021-8693(81)90324-0},
}
	


\bib{J.av}{article}{
   author={Joseph, Anthony},
   title={On the associated variety of a primitive ideal},
   journal={J. Algebra},
   volume={93},
   date={1985},
   number={2},
   pages={509--523},
   issn={0021-8693},
   review={\MR{786766}},
   doi={10.1016/0021-8693(85)90172-3},
}

\bib{J.ann}{article}{
   author={Joseph, Anthony},
   title={Annihilators and associated varieties of unitary highest weight
   modules},
   journal={Ann. Sci. \'{E}cole Norm. Sup. (4)},
   volume={25},
   date={1992},
   number={1},
   pages={1--45},
   issn={0012-9593},
   review={\MR{1152612}},
}

\bib{J.hw}{article}{
   author={Joseph, Anthony},
   title={On the variety of a highest weight module},
   journal={J. Algebra},
   volume={88},
   date={1984},
   number={1},
   pages={238--278},
   issn={0021-8693},
   review={\MR{741942}},
   doi={10.1016/0021-8693(84)90100-5},
}
	

\bib{JLS}{article}{
author={Jiang, D.},
author={Liu, B.},
author={Savin, G.},
title={Raising nilpotent orbits in wave-front sets},
journal={Represent. Theory},
volume={20},
pages={419--450},
year={2016},
}

\bib{Ki62}{article}{
author={Kirillov, A. A.},
title={Unitary representations of nilpotent Lie groups},
journal={Uspehi Mat. Nauk},
volume={17},
issue ={4},
pages={57--110},
year={1962},
}


\bib{Ko70}{article}{
author={Kostant, B.},
title={Quantization and unitary representations},
journal={Lectures in Modern Analysis and Applications III, Lecture Notes in Math.},
volume={170},
pages={87--208},
year={1970},
}


\bib{KP}{article}{
author={Kraft, H.},
author={Procesi, C.},
title={On the geometry of conjugacy classes in classical groups},
journal={Comment. Math. Helv.},
volume={57},
pages={539--602},
year={1982},
}

\bib{KR}{article}{
author={Kudla, S. S.},
author={Rallis, S.},
title={Degenerate principal series and invariant distributions},
journal={Israel J. Math.},
volume={69},
pages={25--45},
year={1990},
}


\bib{Ku}{article}{
author={Kudla, S. S.},
title={Some extensions of the Siegel-Weil formula},
journal={In: Gan W., Kudla S., Tschinkel Y. (eds) Eisenstein Series and Applications. Progress in Mathematics, vol 258. Birkh\"auser Boston},
%volume={69},
pages={205--237},
year={2008},
}





\bib{LZ1}{article}{
author={Lee, S. T.},
author={Zhu, C.-B.},
title={Degenerate principal series and local theta correspondence II},
journal={Israel J. Math.},
volume={100},
pages={29--59},
year={1997},
}

\bib{LZ2}{article}{
author={Lee, S. T.},
author={Zhu, C.-B.},
title={Degenerate principal series of metaplectic groups and Howe correspondence},
journal = {D. Prasad at al. (eds.), Automorphic Representations and L-Functions, Tata Institute of Fundamental Research, India,},
year = {2013},
pages = {379--408},
}

\bib{Li89}{article}{
author={Li, J.-S.},
title={Singular unitary representations of classical groups},
journal={Invent. Math.},
volume={97},
number = {2},
pages={237--255},
year={1989},
}

\bib{LiuAG}{book}{
  title={Algebraic Geometry and Arithmetic Curves},
  author = {Liu, Q.},
  year = {2006},
  publisher={Oxford University Press},
}

\bib{LM}{article}{
   author = {Loke, H. Y.},
   author = {Ma, J.},
    title = {Invariants and $K$-spectrums of local theta lifts},
    journal = {Compositio Math.},
    volume = {151},
    issue = {01},
    year = {2015},
    pages ={179--206},
}

\bib{DL}{article}{
   author={Deligne, P.},
   author={Lusztig, G.},
   title={Representations of reductive groups over finite fields},
   journal={Ann. of Math. (2)},
   volume={103},
   date={1976},
   number={1},
   pages={103--161},
   issn={0003-486X},
   review={\MR{393266}},
   doi={10.2307/1971021},
}

\bib{Lu}{book}{
   author={Lusztig, George},
   title={Characters of reductive groups over a finite field},
   series={Annals of Mathematics Studies},
   volume={107},
   publisher={Princeton University Press, Princeton, NJ},
   date={1984},
   pages={xxi+384},
   isbn={0-691-08350-9},
   isbn={0-691-08351-7},
   review={\MR{742472}},
   doi={10.1515/9781400881772},
}


\bib{LS}{article}{
   author = {Lusztig, G.},
   author = {Spaltenstein, N.},
    title = {Induced unipotent classes},
    journal = {j. London Math. Soc.},
    volume = {19},
    year = {1979},
    pages ={41--52},
}

\bib{Lu.I}{article}{
   author={Lusztig, G.},
   title={Intersection cohomology complexes on a reductive group},
   journal={Invent. Math.},
   volume={75},
   date={1984},
   number={2},
   pages={205--272},
   issn={0020-9910},
   review={\MR{732546}},
   doi={10.1007/BF01388564},
}
	

\bib{Ma}{article}{
   author = {Mackey, G. W.},
    title = {Unitary representations of group extentions},
    journal = {Acta Math.},
    volume = {99},
    year = {1958},
    pages ={265--311},
}


\bib{Mc}{article}{
   author = {McGovern, W. M},
    title = {Cells of Harish-Chandra modules for real classical groups},
    journal = {Amer. J.  of Math.},
    volume = {120},
    issue = {01},
    year = {1998},
    pages ={211--228},
}

\bib{Mo96}{article}{
 author={M{\oe}glin, C.},
    title = {Front d'onde des repr\'esentations des groupes classiques $p$-adiques},
    journal = {Amer. J. Math.},
    volume = {118},
    issue = {06},
    year = {1996},
    pages ={1313--1346},
}

\bib{Mo17}{article}{
  author={M{\oe}glin, C.},
  title = {Paquets d'Arthur Sp\'eciaux Unipotents aux Places Archim\'ediennes et Correspondance de Howe},
  journal = {J. Cogdell et al. (eds.), Representation Theory, Number Theory, and Invariant Theory, In Honor of Roger Howe. Progress in Math.}
  %series ={Progress in Math.},
  volume = {323},
  pages = {469--502}
  year = {2017}
}


\bib{MVW}{book}{
  volume={1291},
  title={Correspondances de Howe sur un corps $p$-adique},
  author={M{\oe}glin, C.},
  author={Vign\'eras, M.-F.},
  author={Waldspurger, J.-L.},
  series={Lecture Notes in Mathematics},
  publisher={Springer}
  ISBN={978-3-540-18699-1},
  date={1987},
}

\bib{NOTYK}{article}{
   author = {Nishiyama, K.},
   author = {Ochiai, H.},
   author = {Taniguchi, K.},
   author = {Yamashita, H.},
   author = {Kato, S.},
    title = {Nilpotent orbits, associated cycles and Whittaker models for highest weight representations},
    journal = {Ast\'erisque},
    volume = {273},
    year = {2001},
   pages ={1--163},
}

\bib{NOZ}{article}{
  author = {Nishiyama, K.},
  author = {Ochiai, H.},
  author = {Zhu, C.-B.},
  journal = {Trans. Amer. Math. Soc.},
  title = {Theta lifting of nilpotent orbits for symmetric pairs},
  volume = {358},
  year = {2006},
  pages = {2713--2734},
}


\bib{NZ}{article}{
   author = {Nishiyama, K.},
   author = {Zhu, C.-B.},
    title = {Theta lifting of unitary lowest weight modules and their associated cycles},
    journal = {Duke Math. J.},
    volume = {125},
    number= {03},
    year = {2004},
   pages ={415--465},
}



\bib{Ohta}{article}{
  author = {Ohta, T.},
  %doi = {10.2748/tmj/1178227492},
  journal = {Tohoku Math. J.},
  number = {2},
  pages = {161--211},
  publisher = {Tohoku University, Mathematical Institute},
  title = {The closures of nilpotent orbits in the classical symmetric
    pairs and their singularities},
  volume = {43},
  year = {1991}
}

\bib{Ohta2}{article}{
  author = {Ohta, T.},
  journal = {Hiroshima Math. J.},
  number = {2},
  pages = {347--360},
  title = {Induction of nilpotent orbits for real reductive groups and associated varieties of standard representations},
  volume = {29},
  year = {1999}
}

\bib{Ohta4}{article}{
  title={Nilpotent orbits of $\mathbb{Z}_4$-graded Lie algebra and geometry of
    moment maps associated to the dual pair $(\mathrm{U} (p, q), \mathrm{U} (r, s))$},
  author={Ohta, T.},
  journal={Publ. RIMS},
  volume={41},
  number={3},
  pages={723--756},
  year={2005}
}

\bib{PT}{article}{
  title={Some small unipotent representations of indefinite orthogonal groups and the theta correspondence},
  author={Paul, A.},
  author={Trapa, P.},
  journal={University of Aarhus Publ. Series},
  volume={48},
  pages={103--125},
  year={2007}
}


\bib{PV}{article}{
  title={Invariant Theory},
  author={Popov, V. L.},
  author={Vinberg, E. B.},
  book={
  title={Algebraic Geometry IV: Linear Algebraic Groups, Invariant Theory},
  series={Encyclopedia of Mathematical Sciences},
  volume={55},
  year={1994},
  publisher={Springer},}
}




%\bib{PPz}{article}{
%author={Protsak, V.} ,
%author={Przebinda, T.},
%title={On the occurrence of admissible representations in the real Howe
%    correspondence in stable range},
%journal={Manuscr. Math.},
%volume={126},
%number={2},
%pages={135--141},
%year={2008}
%}


\bib{PrzInf}{article}{
      author={Przebinda, T.},
       title={The duality correspondence of infinitesimal characters},
        date={1996},
     journal={Colloq. Math.},
      volume={70},
       pages={93--102},
}


\bib{Pz}{article}{
author={Przebinda, T.},
title={Characters, dual pairs, and unitary representations},
journal={Duke Math. J. },
volume={69},
number={3},
pages={547--592},
year={1993}
}

\bib{Ra}{article}{
author={Rallis, S.},
title={On the Howe duality conjecture},
journal={Compositio Math.},
volume={51},
pages={333--399},
year={1984}
}

\bib{Sa}{article}{
author={Sahi, S.},
title={Explicit Hilbert spaces for certain unipotent representations},
journal={Invent. Math.},
volume={110},
number = {2},
pages={409--418},
year={1992}
}

\bib{Se}{article}{
author={Sekiguchi, J.},
title={Remarks on real nilpotent orbits of a symmetric pair},
journal={J. Math. Soc. Japan},
%publisher={The Mathematical Society of Japan},
year={1987},
volume={39},
number={1},
pages={127--138},
}

\bib{SV}{article}{
  author = {Schmid, W.},
  author = {Vilonen, K.},
  journal = {Annals of Math.},
  number = {3},
  pages = {1071--1118},
  %publisher = {Princeton University, Mathematics Department, Princeton, NJ; Mathematical Sciences Publishers, Berkeley},
  title = {Characteristic cycles and wave front cycles of representations of reductive Lie groups},
  volume = {151},
year = {2000},
}

\bib{So}{article}{
author = {Sommers, E.},
title = {Lusztig's canonical quotient and generalized duality},
journal = {J. Algebra},
volume = {243},
number = {2},
pages = {790--812},
year = {2001},
}

\bib{SS}{book}{
  author = {Springer, T. A.},
  author = {Steinberg, R.},
  title = {Seminar on algebraic groups and related finite groups; Conjugate classes},
  series = {Lecture Notes in Math.},
  volume = {131},
publisher={Springer},
year={1970},
}

\bib{SZ1}{article}{
title={A general form of Gelfand-Kazhdan criterion},
author={Sun, B.},
author={Zhu, C.-B.},
journal={Manuscripta Math.},
pages = {185--197},
volume = {136},
year={2011}
}


%\bib{SZ2}{article}{
%  title={Conservation relations for local theta correspondence},
%  author={Sun, B.},
%  author={Zhu, C.-B.},
%  journal={J. Amer. Math. Soc.},
%  pages = {939--983},
%  volume = {28},
%  year={2015}
%}



\bib{Tr}{article}{
  title={Special unipotent representations and the Howe correspondence},
  author={Trapa, P.},
  year = {2004},
  journal={University of Aarhus Publication Series},
  volume = {47},
  pages= {210--230}
}

% \bib{Wa}{article}{
%    author = {Waldspurger, J.-L.},
%     title = {D\'{e}monstration d'une conjecture de dualit\'{e} de Howe dans le cas $p$-adique, $p \neq 2$ in Festschrift in honor of I. I. Piatetski-Shapiro on the occasion of his sixtieth birthday},
%   journal = {Israel Math. Conf. Proc., 2, Weizmann, Jerusalem},
%  year = {1990},
% pages = {267-324},
% }

\bib{Vg}{book}{
   author={Vogan, David A.},
   title={Representations of real reductive Lie groups},
   series={Progress in Mathematics},
   volume={15},
   publisher={Birkh\"{a}user, Boston, Mass.},
   date={1981},
   pages={xvii+754},
   isbn={3-7643-3037-6},
   review={\MR{632407}},
}

\bib{V4}{article}{
   author={Vogan, D. A. },
   title={Irreducible characters of semisimple Lie groups. IV.
   Character-multiplicity duality},
   journal={Duke Math. J.},
   volume={49},
   date={1982},
   number={4},
   pages={943--1073},
   issn={0012-7094},
   review={\MR{683010}},
}

\bib{VoBook}{book}{
author = {Vogan, D. A. },
  title={Unitary representations of reductive Lie groups},
  year={1987},
  series = {Ann. of Math. Stud.},
 volume={118},
  publisher={Princeton University Press}
}


\bib{Vo89}{article}{
  author = {Vogan, D. A. },
  title = {Associated varieties and unipotent representations},
 %booktitle ={Harmonic analysis on reductive groups, Proc. Conf., Brunswick/ME (USA) 1989,},
  journal = {Harmonic analysis on reductive groups, Proc. Conf., Brunswick/ME
    (USA) 1989, Prog. Math.},
 volume={101},
  publisher = {Birkh\"{a}user, Boston-Basel-Berlin},
  year = {1991},
pages={315--388},
  editor = {W. Barker and P. Sally},
}

\bib{Vo98}{article}{
  author = {Vogan, D. A. },
  title = {The method of coadjoint orbits for real reductive groups},
 %booktitle ={Representation theory of Lie groups (Park City, UT, 1998)},
 journal = {Representation theory of Lie groups (Park City, UT, 1998). IAS/Park City Math. Ser.},
  volume={8},
  publisher = {Amer. Math. Soc.},
  year = {2000},
pages={179--238},
}

\bib{Vo00}{article}{
  author = {Vogan, D. A. },
  title = {Unitary representations of reductive Lie groups},
 %booktitle ={Mathematics towards the Third Millennium (Rome, 1999)},
 journal ={Mathematics towards the Third Millennium (Rome, 1999). Accademia Nazionale dei Lincei, (2000)},
  %series = {Accademia Nazionale dei Lincei, 2000},
 %volume={9},
pages={147--167},
}


\bib{Wa1}{book}{
  title={Real reductive groups I},
  author={Wallach, N. R.},
  year={1988},
  publisher={Academic Press Inc. }
}

\bib{Wa2}{book}{
  title={Real reductive groups II},
  author={Wallach, N. R.},
  year={1992},
  publisher={Academic Press Inc. }
}


\bib{Weyl}{book}{
  title={The classical groups: their invariants and representations},
  author={Weyl, H.},
  year={1947},
  publisher={Princeton University Press}
}

\bib{Ya}{article}{
  title={Degenerate principal series representations for quaternionic unitary groups},
  author={Yamana, S.},
  year = {2011},
  journal={Israel J. Math.},
  volume = {185},
  pages= {77--124}
}



% \bib{EGAIV4}{article}{
%   title = {\'El\'ements de g\'eom\'etrie alg\'brique IV 4: \'Etude locale des
%     sch\'emas et des morphismes de sch\'emas},
%   author = {Grothendieck, Alexandre},
%   author = {Dieudonn\'e, Jean},
%   journal  = {Inst. Hautes \'Etudes Sci. Publ. Math.},
%   volume = {32},
%   year = {1967},
%   pages = {5--361}
% }



\end{biblist}
\end{bibdiv}


\end{document}


%%% Local Variables:
%%% coding: utf-8
%%% mode: latex
%%% TeX-engine: tex
%%% ispell-local-dictionary: "en_US"
%%% End:
