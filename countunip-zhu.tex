% !TeX program = xelatex
\documentclass[12pt,a4paper]{amsart}
\usepackage[margin=2.5cm,marginpar=2cm]{geometry}

\usepackage[bookmarksopen,bookmarksdepth=2,hidelinks,colorlinks=false]{hyperref}
\usepackage[nameinlink]{cleveref}

% \usepackage[color]{showkeys}
% \makeatletter
%   \SK@def\Cref#1{\SK@\SK@@ref{#1}\SK@Cref{#1}}%
% \makeatother
%% FONTS

\usepackage{amssymb}
%\usepackage{amsmath}
\usepackage{mathrsfs}
\usepackage{mathtools}
%\usepackage{amsrefs}
%\usepackage{mathbbol,mathabx}
\usepackage{amsthm}
\usepackage{graphicx}
\usepackage{braket}
%\usepackage[pointedenum]{paralist}
%\usepackage{paralist}
\usepackage{amscd}

\usepackage[alphabetic]{amsrefs}

\usepackage[all,cmtip]{xy}
\usepackage{rotating}
\usepackage{leftidx}
%\usepackage{arydshln}

%\DeclareSymbolFont{bbold}{U}{bbold}{m}{n}
%\DeclareSymbolFontAlphabet{\mathbbold}{bbold}


%\usepackage[dvipdfx,rgb,table]{xcolor}
\usepackage[rgb,table]{xcolor}
%\usepackage{mathrsfs}

\setcounter{tocdepth}{1}
\setcounter{secnumdepth}{2}

%\usepackage[abbrev,shortalphabetic]{amsrefs}


\usepackage[normalem]{ulem}

% circled number
\usepackage{pifont}
\makeatletter
\newcommand*{\circnuma}[1]{%
  \ifnum#1<1 %
    \@ctrerr
  \else
    \ifnum#1>20 %
      \@ctrerr
    \else
      \mbox{\ding{\numexpr 171+(#1)\relax}}%
     \fi
  \fi
}
\makeatother

\usepackage[centertableaux]{ytableau}


% Ytableau tweak
\makeatletter
\pgfkeys{/ytableau/options,
  noframe/.default = false,
  noframe/.is choice,
  noframe/true/.code = {%
    \global\let\vrule@YT=\vrule@none@YT
    \global\let\hrule@YT=\hrule@none@YT
  },
  noframe/false/.code = {%
    \global\let\vrule@YT=\vrule@normal@YT
    \global\let\hrule@YT=\hrule@normal@YT
  },
  noframe/on/.style = {noframe/true},
  noframe/off/.style = {noframe/false},
}

\def\hrule@enon@YT{%
  \hrule width  \dimexpr \boxdim@YT + \fboxrule *2 \relax
  height 0pt
}
\def\vrule@enon@YT{%
  \vrule height \dimexpr  \boxdim@YT + \fboxrule\relax
     width \fboxrule
}

\def\enon{\omit\enon@YT}
\newcommand{\enon@YT}[2][clear]{%
  \def\thisboxcolor@YT{#1}%
  \let\hrule@YT=\hrule@enon@YT
  \let\vrule@YT=\vrule@enon@YT
  \startbox@@YT#2\endbox@YT
  \nullfont
}

\makeatother
%\ytableausetup{noframe=on,smalltableaux}
\ytableausetup{noframe=off,boxsize=1.3em}
\let\ytb=\ytableaushort

\newcommand{\tytb}[1]{{\tiny\ytb{#1}}}


%\usepackage[mathlines,pagewise]{lineno}
%\linenumbers

\usepackage{enumitem}
%% Enumitem
\newlist{enumC}{enumerate}{1} % Conditions in Lemma/Theorem/Prop
\setlist[enumC,1]{label=(\alph*),wide,ref=(\alph*)}
\crefname{enumCi}{condition}{conditions}
\Crefname{enumCi}{Condition}{Conditions}
\newlist{enumT}{enumerate}{3} % "Theorem"=conclusions in Lemma/Theorem/Prop
\setlist[enumT]{label=(\roman*),wide}
\setlist[enumT,1]{label=(\roman*),wide}
\setlist[enumT,2]{label=(\alph*),ref ={(\roman{enumTi}.\alph*)}}
\setlist[enumT,3]{label=(\arabic*), ref ={(\roman{enumTi}.\alph{enumTii}.\alph*)}}
\crefname{enumTi}{}{}
\Crefname{enumTi}{Item}{Items}
\crefname{enumTii}{}{}
\Crefname{enumTii}{Item}{Items}
\crefname{enumTiii}{}{}
\Crefname{enumTiii}{Item}{Items}
\newlist{enumPF}{enumerate}{3}
\setlist[enumPF]{label=(\alph*),wide}
\setlist[enumPF,1]{label=(\roman*),wide}
\setlist[enumPF,2]{label=(\alph*)}
\setlist[enumPF,3]{label=\arabic*).}
\newlist{enumS}{enumerate}{3} % Statement outside Lemma/Theorem/Prop
\setlist[enumS]{label=\roman*)}
\setlist[enumS,1]{label=\roman*)}
\setlist[enumS,2]{label=\alph*)}
\setlist[enumS,3]{label=\arabic*.}
\newlist{enumI}{enumerate}{3} % items
\setlist[enumI,1]{label=\roman*),leftmargin=*}
\setlist[enumI,2]{label=\alph*), leftmargin=*}
\setlist[enumI,3]{label=\arabic*), leftmargin=*}
\newlist{enumIL}{enumerate*}{1} % inline enum
\setlist*[enumIL]{label=\roman*)}
\newlist{enumR}{enumerate}{1} % remarks
\setlist[enumR]{label=\arabic*.,wide,labelwidth=!, labelindent=0pt}
\crefname{enumRi}{remark}{remarks}

\crefname{equation}{}{}
\Crefname{equation}{Equation}{Equations}
\Crefname{lem}{Lemma}{Lemma}
\Crefname{thm}{Theorem}{Theorem}

\newlist{des}{description}{1}
\setlist[des]{font=\sffamily\bfseries}

% editing macros.
\blendcolors{!80!black}
\long\def\okay#1{\ifcsname highlightokay\endcsname
{\color{red} #1}
\else
{#1}
\fi
}
\long\def\editc#1{{\color{red} #1}}
\long\def\mjj#1{{{\color{blue}#1}}}
\long\def\mjjr#1{{\color{red} (#1)}}
\long\def\mjjd#1#2{{\color{blue} #1 \sout{#2}}}
\def\mjjb{\color{blue}}
\def\mjje{\color{black}}
\def\mjjcb{\color{green!50!black}}
\def\mjjce{\color{black}}

\long\def\sun#1{{{\color{cyan}#1}}}
\long\def\sund#1#2{{\color{cyan}#1  \sout{#2}}}
\long\def\mv#1{{{\color{red} {\bf move to a proper place:} #1}}}
\long\def\delete#1{}

%\reversemarginpar
\newcommand{\lokec}[1]{\marginpar{\color{blue}\tiny #1 \mbox{--loke}}}
\newcommand{\mjjc}[1]{\marginpar{\color{green}\tiny #1 \mbox{--ma}}}

\newcommand{\trivial}[2][]{\if\relax\detokenize{#1}\relax
  {%\hfill\break
   % \begin{minipage}{\textwidth}
      \color{orange} \vspace{0em} $[$  #2 $]$
  %\end{minipage}
  %\break
      \color{black}
  }
  \else
\ifx#1h
\ifcsname showtrivial\endcsname
{%\hfill\break
 % \begin{minipage}{\textwidth}
    \color{orange} \vspace{0em}  $[$ #2 $]$
%\end{minipage}
%\break
    \color{black}
}
\fi
\else {\red Wrong argument!} \fi
\fi
}

\newcommand{\byhide}[2][]{\if\relax\detokenize{#1}\relax
{\color{orange} \vspace{0em} Plan to delete:  #2}
\else
\ifx#1h\relax\fi
\fi
}



\newcommand{\Rank}{\mathrm{rk}}
\newcommand{\cqq}{\mathscr{D}}
\newcommand{\rsym}{\mathrm{sym}}
\newcommand{\rskew}{\mathrm{skew}}
\newcommand{\fraksp}{\mathfrak{sp}}
\newcommand{\frakso}{\mathfrak{so}}
\newcommand{\frakm}{\mathfrak{m}}
\newcommand{\frakp}{\mathfrak{p}}
\newcommand{\pr}{\mathrm{pr}}
\newcommand{\rhopst}{\rho'^*}
\newcommand{\Rad}{\mathrm{Rad}}
\newcommand{\Res}{\mathrm{Res}}
\newcommand{\Hol}{\mathrm{Hol}}
\newcommand{\AC}{\mathrm{AC}}
%\newcommand{\AS}{\mathrm{AS}}
\newcommand{\WF}{\mathrm{WF}}
\newcommand{\AV}{\mathrm{AV}}
\newcommand{\AVC}{\mathrm{AV}_\bC}
\newcommand{\VC}{\mathrm{V}_\bC}
\newcommand{\bfv}{\mathbf{v}}
\newcommand{\depth}{\mathrm{depth}}
\newcommand{\wtM}{\widetilde{M}}
\newcommand{\wtMone}{{\widetilde{M}^{(1,1)}}}

\newcommand{\nullpp}{N(\fpp'^*)}
\newcommand{\nullp}{N(\fpp^*)}
%\newcommand{\Aut}{\mathrm{Aut}}

\def\mstar{{\medstar}}


\newcommand{\bfone}{\mathbf{1}}
\newcommand{\piSigma}{\pi_\Sigma}
\newcommand{\piSigmap}{\pi'_\Sigma}


\newcommand{\sfVprime}{\mathsf{V}^\prime}
\newcommand{\sfVdprime}{\mathsf{V}^{\prime \prime}}
\newcommand{\gminusone}{\mathfrak{g}_{-\frac{1}{m}}}

\newcommand{\eva}{\mathrm{eva}}

\def\subset{\subseteq}


% \newcommand\iso{\xrightarrow{
%    \,\smash{\raisebox{-0.65ex}{\ensuremath{\scriptstyle\sim}}}\,}}

\def\Ueven{{U_{\rm{even}}}}
\def\Uodd{{U_{\rm{odd}}}}
\def\ttau{\tilde{\tau}}
\def\Wcp{W}
\def\Kur{{K^{\mathrm{u}}}}

\def\Im{\operatorname{Im}}

\providecommand{\bcN}{{\overline{\cN}}}



\makeatletter

\def\gen#1{\left\langle
    #1
      \right\rangle}
\makeatother

\makeatletter
\def\inn#1#2{\left\langle
      \def\ta{#1}\def\tb{#2}
      \ifx\ta\@empty{\;} \else {\ta}\fi ,
      \ifx\tb\@empty{\;} \else {\tb}\fi
      \right\rangle}
\def\binn#1#2{\left\lAngle
      \def\ta{#1}\def\tb{#2}
      \ifx\ta\@empty{\;} \else {\ta}\fi ,
      \ifx\tb\@empty{\;} \else {\tb}\fi
      \right\rAngle}
\makeatother

\makeatletter
\def\binn#1#2{\overline{\inn{#1}{#2}}}
\makeatother


\def\innwi#1#2{\inn{#1}{#2}_{W_i}}
\def\innw#1#2{\inn{#1}{#2}_{\bfW}}
\def\innv#1#2{\inn{#1}{#2}_{\bfV}}
\def\innbfv#1#2{\inn{#1}{#2}_{\bfV}}
\def\innvi#1#2{\inn{#1}{#2}_{V_i}}
\def\innvp#1#2{\inn{#1}{#2}_{\bfV'}}
\def\innp#1#2{\inn{#1}{#2}'}

% choose one of then
\def\simrightarrow{\iso}
\def\surj{\twoheadrightarrow}
%\def\simrightarrow{\xrightarrow{\sim}}

 \def\ckhha{{}^a \check\fhh}


\newcommand\iso{\xrightarrow{
   \,\smash{\raisebox{-0.65ex}{\ensuremath{\scriptstyle\sim}}}\,}}

\newcommand\riso{\xleftarrow{
   \,\smash{\raisebox{-0.65ex}{\ensuremath{\scriptstyle\sim}}}\,}}









\usepackage{xparse}
\def\usecsname#1{\csname #1\endcsname}
\def\useLetter#1{#1}
\def\usedbletter#1{#1#1}

% \def\useCSf#1{\csname f#1\endcsname}

\ExplSyntaxOn

\def\mydefcirc#1#2#3{\expandafter\def\csname
  circ#3{#1}\endcsname{{}^\circ {#2{#1}}}}
\def\mydefvec#1#2#3{\expandafter\def\csname
  vec#3{#1}\endcsname{\vec{#2{#1}}}}
\def\mydefdot#1#2#3{\expandafter\def\csname
  dot#3{#1}\endcsname{\dot{#2{#1}}}}

\def\mydefacute#1#2#3{\expandafter\def\csname a#3{#1}\endcsname{\acute{#2{#1}}}}
\def\mydefbr#1#2#3{\expandafter\def\csname br#3{#1}\endcsname{\breve{#2{#1}}}}
\def\mydefbar#1#2#3{\expandafter\def\csname bar#3{#1}\endcsname{\bar{#2{#1}}}}
\def\mydefhat#1#2#3{\expandafter\def\csname hat#3{#1}\endcsname{\hat{#2{#1}}}}
\def\mydefwh#1#2#3{\expandafter\def\csname wh#3{#1}\endcsname{\widehat{#2{#1}}}}
\def\mydeft#1#2#3{\expandafter\def\csname t#3{#1}\endcsname{\tilde{#2{#1}}}}
\def\mydefu#1#2#3{\expandafter\def\csname u#3{#1}\endcsname{\underline{#2{#1}}}}
\def\mydefr#1#2#3{\expandafter\def\csname r#3{#1}\endcsname{\mathrm{#2{#1}}}}
\def\mydefb#1#2#3{\expandafter\def\csname b#3{#1}\endcsname{\mathbb{#2{#1}}}}
\def\mydefwt#1#2#3{\expandafter\def\csname wt#3{#1}\endcsname{\widetilde{#2{#1}}}}
%\def\mydeff#1#2#3{\expandafter\def\csname f#3{#1}\endcsname{\mathfrak{#2{#1}}}}
\def\mydefbf#1#2#3{\expandafter\def\csname bf#3{#1}\endcsname{\mathbf{#2{#1}}}}
\def\mydefc#1#2#3{\expandafter\def\csname c#3{#1}\endcsname{\mathcal{#2{#1}}}}
\def\mydefsf#1#2#3{\expandafter\def\csname sf#3{#1}\endcsname{\mathsf{#2{#1}}}}
\def\mydefs#1#2#3{\expandafter\def\csname s#3{#1}\endcsname{\mathscr{#2{#1}}}}
\def\mydefcks#1#2#3{\expandafter\def\csname cks#3{#1}\endcsname{{\check{
        \csname s#2{#1}\endcsname}}}}
\def\mydefckc#1#2#3{\expandafter\def\csname ckc#3{#1}\endcsname{{\check{
      \csname c#2{#1}\endcsname}}}}
\def\mydefck#1#2#3{\expandafter\def\csname ck#3{#1}\endcsname{{\check{#2{#1}}}}}

\cs_new:Npn \mydeff #1#2#3 {\cs_new:cpn {f#3{#1}} {\mathfrak{#2{#1}}}}

\cs_new:Npn \doGreek #1
{
  \clist_map_inline:nn {alpha,beta,gamma,Gamma,delta,Delta,epsilon,varepsilon,zeta,eta,theta,vartheta,Theta,iota,kappa,lambda,Lambda,mu,nu,xi,Xi,pi,Pi,rho,sigma,varsigma,Sigma,tau,upsilon,Upsilon,phi,varphi,Phi,chi,psi,Psi,omega,Omega,tG} {#1{##1}{\usecsname}{\useLetter}}
}

\cs_new:Npn \doSymbols #1
{
  \clist_map_inline:nn {otimes,boxtimes} {#1{##1}{\usecsname}{\useLetter}}
}

\cs_new:Npn \doAtZ #1
{
  \clist_map_inline:nn {A,B,C,D,E,F,G,H,I,J,K,L,M,N,O,P,Q,R,S,T,U,V,W,X,Y,Z} {#1{##1}{\useLetter}{\useLetter}}
}

\cs_new:Npn \doatz #1
{
  \clist_map_inline:nn {a,b,c,d,e,f,g,h,i,j,k,l,m,n,o,p,q,r,s,t,u,v,w,x,y,z} {#1{##1}{\useLetter}{\usedbletter}}
}

\cs_new:Npn \doallAtZ
{
\clist_map_inline:nn {mydefsf,mydeft,mydefu,mydefwh,mydefhat,mydefr,mydefwt,mydeff,mydefb,mydefbf,mydefc,mydefs,mydefck,mydefcks,mydefckc,mydefbar,mydefvec,mydefcirc,mydefdot,mydefbr,mydefacute} {\doAtZ{\csname ##1\endcsname}}
}

\cs_new:Npn \doallatz
{
\clist_map_inline:nn {mydefsf,mydeft,mydefu,mydefwh,mydefhat,mydefr,mydefwt,mydeff,mydefb,mydefbf,mydefc,mydefs,mydefck,mydefbar,mydefvec,mydefdot,mydefbr,mydefacute} {\doatz{\csname ##1\endcsname}}
}

\cs_new:Npn \doallGreek
{
\clist_map_inline:nn {mydefck,mydefwt,mydeft,mydefwh,mydefbar,mydefu,mydefvec,mydefcirc,mydefdot,mydefbr,mydefacute} {\doGreek{\csname ##1\endcsname}}
}

\cs_new:Npn \doallSymbols
{
\clist_map_inline:nn {mydefck,mydefwt,mydeft,mydefwh,mydefbar,mydefu,mydefvec,mydefcirc,mydefdot} {\doSymbols{\csname ##1\endcsname}}
}



\cs_new:Npn \doGroups #1
{
  \clist_map_inline:nn {GL,Sp,rO,rU,fgl,fsp,foo,fuu,fkk,fuu,ufkk,uK} {#1{##1}{\usecsname}{\useLetter}}
}

\cs_new:Npn \doallGroups
{
\clist_map_inline:nn {mydeft,mydefu,mydefwh,mydefhat,mydefwt,mydefck,mydefbar} {\doGroups{\csname ##1\endcsname}}
}


\cs_new:Npn \decsyms #1
{
\clist_map_inline:nn {#1} {\expandafter\DeclareMathOperator\csname ##1\endcsname{##1}}
}

\decsyms{Mp,id,SL,Sp,SU,SO,GO,GSO,GU,GSp,PGL,Pic,Lie,Mat,Ker,Hom,Ext,Ind,reg,res,inv,Isom,Det,Tr,Norm,Sym,Span,Stab,Spec,PGSp,PSL,tr,Ad,Br,Ch,Cent,End,Aut,Dvi,Frob,Gal,GL,Gr,DO,ur,vol,ab,Nil,Supp,rank,Sign}

\def\abs#1{\left|{#1}\right|}
\def\norm#1{{\left\|{#1}\right\|}}


% \NewDocumentCommand\inn{m m}{
% \left\langle
% \IfValueTF{#1}{#1}{000}
% ,
% \IfValueTF{#2}{#2}{000}
% \right\rangle
% }
\NewDocumentCommand\cent{o m }{
  \IfValueTF{#1}{
    \mathop{Z}_{#1}{(#2)}}
  {\mathop{Z}{(#2)}}
}


\def\fsl{\mathfrak{sl}}
\def\fsp{\mathfrak{sp}}


%\def\cent#1#2{{\mathrm{Z}_{#1}({#2})}}


\doallAtZ
\doallatz
\doallGreek
\doallGroups
\doallSymbols
\ExplSyntaxOff


% \usepackage{geometry,amsthm,graphics,tabularx,amssymb,shapepar}
% \usepackage{amscd}
% \usepackage{mathrsfs}


\usepackage{diagbox}
% Update the information and uncomment if AMS is not the copyright
% holder.
%\copyrightinfo{2006}{American Mathematical Society}
%\usepackage{nicematrix}
\usepackage{arydshln}

\usepackage{tikz}
\usetikzlibrary{matrix,arrows,positioning,cd,backgrounds}
\usetikzlibrary{decorations.pathmorphing,decorations.pathreplacing}

\usepackage{upgreek}

\usepackage{listings}
\lstset{
    basicstyle=\ttfamily\tiny,
    keywordstyle=\color{black},
    commentstyle=\color{white}, % white comments
    stringstyle=\ttfamily, % typewriter type for strings
    showstringspaces=false,
    breaklines=true,
    emph={Output},emphstyle=\color{blue},
}

\newcommand{\BA}{{\mathbb{A}}}
%\newcommand{\BB}{{\mathbb {B}}}
\newcommand{\BC}{{\mathbb {C}}}
\newcommand{\BD}{{\mathbb {D}}}
\newcommand{\BE}{{\mathbb {E}}}
\newcommand{\BF}{{\mathbb {F}}}
\newcommand{\BG}{{\mathbb {G}}}
\newcommand{\BH}{{\mathbb {H}}}
\newcommand{\BI}{{\mathbb {I}}}
\newcommand{\BJ}{{\mathbb {J}}}
\newcommand{\BK}{{\mathbb {U}}}
\newcommand{\BL}{{\mathbb {L}}}
\newcommand{\BM}{{\mathbb {M}}}
\newcommand{\BN}{{\mathbb {N}}}
\newcommand{\BO}{{\mathbb {O}}}
\newcommand{\BP}{{\mathbb {P}}}
\newcommand{\BQ}{{\mathbb {Q}}}
\newcommand{\BR}{{\mathbb {R}}}
\newcommand{\BS}{{\mathbb {S}}}
\newcommand{\BT}{{\mathbb {T}}}
\newcommand{\BU}{{\mathbb {U}}}
\newcommand{\BV}{{\mathbb {V}}}
\newcommand{\BW}{{\mathbb {W}}}
\newcommand{\BX}{{\mathbb {X}}}
\newcommand{\BY}{{\mathbb {Y}}}
\newcommand{\BZ}{{\mathbb {Z}}}
\newcommand{\Bk}{{\mathbf {k}}}

\newcommand{\CA}{{\mathcal {A}}}
\newcommand{\CB}{{\mathcal {B}}}
\newcommand{\CC}{{\mathcal {C}}}

\newcommand{\CE}{{\mathcal {E}}}
\newcommand{\CF}{{\mathcal {F}}}
\newcommand{\CG}{{\mathcal {G}}}
\newcommand{\CH}{{\mathcal {H}}}
\newcommand{\CI}{{\mathcal {I}}}
\newcommand{\CJ}{{\mathcal {J}}}
\newcommand{\CK}{{\mathcal {K}}}
\newcommand{\CL}{{\mathcal {L}}}
\newcommand{\CM}{{\mathcal {M}}}
\newcommand{\CN}{{\mathcal {N}}}
\newcommand{\CO}{{\mathcal {O}}}
\newcommand{\CP}{{\mathcal {P}}}
\newcommand{\CQ}{{\mathcal {Q}}}
\newcommand{\CR}{{\mathcal {R}}}
\newcommand{\CS}{{\mathcal {S}}}
\newcommand{\CT}{{\mathcal {T}}}
\newcommand{\CU}{{\mathcal {U}}}
\newcommand{\CV}{{\mathcal {V}}}
\newcommand{\CW}{{\mathcal {W}}}
\newcommand{\CX}{{\mathcal {X}}}
\newcommand{\CY}{{\mathcal {Y}}}
\newcommand{\CZ}{{\mathcal {Z}}}


\newcommand{\RA}{{\mathrm {A}}}
\newcommand{\RB}{{\mathrm {B}}}
\newcommand{\RC}{{\mathrm {C}}}
\newcommand{\RD}{{\mathrm {D}}}
\newcommand{\RE}{{\mathrm {E}}}
\newcommand{\RF}{{\mathrm {F}}}
\newcommand{\RG}{{\mathrm {G}}}
\newcommand{\RH}{{\mathrm {H}}}
\newcommand{\RI}{{\mathrm {I}}}
\newcommand{\RJ}{{\mathrm {J}}}
\newcommand{\RK}{{\mathrm {K}}}
\newcommand{\RL}{{\mathrm {L}}}
\newcommand{\RM}{{\mathrm {M}}}
\newcommand{\RN}{{\mathrm {N}}}
\newcommand{\RO}{{\mathrm {O}}}
\newcommand{\RP}{{\mathrm {P}}}
\newcommand{\RQ}{{\mathrm {Q}}}
%\newcommand{\RR}{{\mathrm {R}}}
\newcommand{\RS}{{\mathrm {S}}}
\newcommand{\RT}{{\mathrm {T}}}
\newcommand{\RU}{{\mathrm {U}}}
\newcommand{\RV}{{\mathrm {V}}}
\newcommand{\RW}{{\mathrm {W}}}
\newcommand{\RX}{{\mathrm {X}}}
\newcommand{\RY}{{\mathrm {Y}}}
\newcommand{\RZ}{{\mathrm {Z}}}

\DeclareMathOperator{\absNorm}{\mathfrak{N}}
\DeclareMathOperator{\Ann}{Ann}
\DeclareMathOperator{\LAnn}{L-Ann}
\DeclareMathOperator{\RAnn}{R-Ann}
\DeclareMathOperator{\ind}{ind}
%\DeclareMathOperator{\Ind}{Ind}



\newcommand{\cod}{{\mathrm{cod}}}
\newcommand{\cont}{{\mathrm{cont}}}
\newcommand{\cl}{{\mathrm{cl}}}
\newcommand{\cusp}{{\mathrm{cusp}}}

\newcommand{\disc}{{\mathrm{disc}}}
\renewcommand{\div}{{\mathrm{div}}}



\newcommand{\Gm}{{\mathbb{G}_m}}



\newcommand{\I}{{\mathrm{I}}}

\newcommand{\Jac}{{\mathrm{Jac}}}
\newcommand{\PM}{{\mathrm{PM}}}


\newcommand{\new}{{\mathrm{new}}}
\newcommand{\NS}{{\mathrm{NS}}}
\newcommand{\N}{{\mathrm{N}}}

\newcommand{\ord}{{\mathrm{ord}}}

%\newcommand{\rank}{{\mathrm{rank}}}

\newcommand{\rk}{{\mathrm{k}}}
\newcommand{\rr}{{\mathrm{r}}}
\newcommand{\rh}{{\mathrm{h}}}

\newcommand{\Sel}{{\mathrm{Sel}}}
\newcommand{\Sim}{{\mathrm{Sim}}}

\newcommand{\wt}{\widetilde}
\newcommand{\wh}{\widehat}
\newcommand{\pp}{\frac{\partial\bar\partial}{\pi i}}
\newcommand{\pair}[1]{\langle {#1} \rangle}
\newcommand{\wpair}[1]{\left\{{#1}\right\}}
\newcommand{\intn}[1]{\left( {#1} \right)}
\newcommand{\sfrac}[2]{\left( \frac {#1}{#2}\right)}
\newcommand{\ds}{\displaystyle}
\newcommand{\ov}{\overline}
\newcommand{\incl}{\hookrightarrow}
\newcommand{\lra}{\longrightarrow}
\newcommand{\imp}{\Longrightarrow}
%\newcommand{\lto}{\longmapsto}
\newcommand{\bs}{\backslash}

\newcommand{\cover}[1]{\widetilde{#1}}

\renewcommand{\vsp}{{\vspace{0.2in}}}

\newcommand{\Norma}{\operatorname{N}}
\newcommand{\Ima}{\operatorname{Im}}
\newcommand{\con}{\textit{C}}
\newcommand{\gr}{\operatorname{gr}}
\newcommand{\ad}{\operatorname{ad}}
\newcommand{\der}{\operatorname{der}}
\newcommand{\dif}{\operatorname{d}\!}
\newcommand{\pro}{\operatorname{pro}}
\newcommand{\Ev}{\operatorname{Ev}}
% \renewcommand{\span}{\operatorname{span}} \span is an innernal command.
%\newcommand{\degree}{\operatorname{deg}}
\newcommand{\Invf}{\operatorname{Invf}}
\newcommand{\Inv}{\operatorname{Inv}}
\newcommand{\slt}{\operatorname{SL}_2(\mathbb{R})}
%\newcommand{\temp}{\operatorname{temp}}
%\newcommand{\otop}{\operatorname{top}}
\renewcommand{\small}{\operatorname{small}}
\newcommand{\HC}{\operatorname{HC}}
\newcommand{\lef}{\operatorname{left}}
\newcommand{\righ}{\operatorname{right}}
\newcommand{\Diff}{\operatorname{DO}}
\newcommand{\diag}{\operatorname{diag}}
\newcommand{\sh}{\varsigma}
\newcommand{\sch}{\operatorname{sch}}
%\newcommand{\oleft}{\operatorname{left}}
%\newcommand{\oright}{\operatorname{right}}
\newcommand{\open}{\operatorname{open}}
\newcommand{\sgn}{\operatorname{sgn}}
\newcommand{\triv}{\operatorname{triv}}
\newcommand{\Sh}{\operatorname{Sh}}
\newcommand{\oN}{\operatorname{N}}

\newcommand{\oc}{\operatorname{c}}
\newcommand{\od}{\operatorname{d}}
\newcommand{\os}{\operatorname{s}}
\newcommand{\ol}{\operatorname{l}}
\newcommand{\oL}{\operatorname{L}}
\newcommand{\oJ}{\operatorname{J}}
\newcommand{\oH}{\operatorname{H}}
\newcommand{\oO}{\operatorname{O}}
\newcommand{\oS}{\operatorname{S}}
\newcommand{\oR}{\operatorname{R}}
\newcommand{\oT}{\operatorname{T}}
%\newcommand{\rU}{\operatorname{U}}
\newcommand{\oZ}{\operatorname{Z}}
\newcommand{\oD}{\textit{D}}
\newcommand{\oW}{\textit{W}}
\newcommand{\oE}{\operatorname{E}}
\newcommand{\oP}{\operatorname{P}}
\newcommand{\PD}{\operatorname{PD}}
\newcommand{\oU}{\operatorname{U}}

\newcommand{\g}{\mathfrak g}
\newcommand{\gC}{{\mathfrak g}_{\C}}
\renewcommand{\k}{\mathfrak k}
\newcommand{\h}{\mathfrak h}
\newcommand{\p}{\mathfrak p}
%\newcommand{\q}{\mathfrak q}
\renewcommand{\a}{\mathfrak a}
\renewcommand{\b}{\mathfrak b}
\renewcommand{\c}{\mathfrak c}
\newcommand{\n}{\mathfrak n}
\renewcommand{\u}{\mathfrak u}
%\renewcommand{\v}{\mathfrak v}
\newcommand{\e}{\mathfrak e}
\newcommand{\f}{\mathfrak f}
\renewcommand{\l}{\mathfrak l}
\renewcommand{\t}{\mathfrak t}
\newcommand{\s}{\mathfrak s}
\renewcommand{\r}{\mathfrak r}
\renewcommand{\o}{\mathfrak o}
\newcommand{\m}{\mathfrak m}
\newcommand{\z}{\mathfrak z}
%\renewcommand{\sl}{\mathfrak s \mathfrak l}
\newcommand{\gl}{\mathfrak g \mathfrak l}


\newcommand{\re}{\mathrm e}

\renewcommand{\rk}{\mathrm k}

\newcommand{\Z}{\mathbb{Z}}
\DeclareDocumentCommand{\C}{}{\mathbb{C}}
\newcommand{\R}{\mathbb R}
\newcommand{\Q}{\mathbb Q}
\renewcommand{\H}{\mathbb{H}}
%\newcommand{\N}{\mathbb{N}}
\newcommand{\K}{\mathbb{K}}
%\renewcommand{\S}{\mathbf S}
\newcommand{\M}{\mathbf{M}}
\newcommand{\A}{\mathbb{A}}
\newcommand{\B}{\mathbf{B}}
%\renewcommand{\G}{\mathbf{G}}
\newcommand{\V}{\mathbf{V}}
\newcommand{\W}{\mathbf{W}}
\newcommand{\F}{\mathbf{F}}
\newcommand{\E}{\mathbf{E}}
%\newcommand{\J}{\mathbf{J}}
\renewcommand{\H}{\mathbf{H}}
\newcommand{\X}{\mathbf{X}}
\newcommand{\Y}{\mathbf{Y}}
%\newcommand{\RR}{\mathcal R}
\newcommand{\FF}{\mathcal F}
%\newcommand{\BB}{\mathcal B}
\newcommand{\HH}{\mathcal H}
%\newcommand{\UU}{\mathcal U}
%\newcommand{\MM}{\mathcal M}
%\newcommand{\CC}{\mathcal C}
%\newcommand{\DD}{\mathcal D}
\def\eDD{\mathrm{d}^{e}}
\def\DD{\nabla}
\def\DDD{{\check\nabla}}
\def\DDc{\boldsymbol{\nabla}}
\def\gDD{\nabla^{\mathrm{gen}}}
\def\gDDc{\boldsymbol{\nabla}^{\mathrm{gen}}}
%\newcommand{\OO}{\mathcal O}
%\newcommand{\ZZ}{\mathcal Z}
\newcommand{\ve}{{\vee}}
\newcommand{\aut}{\mathcal A}
\newcommand{\ii}{\mathbf{i}}
\newcommand{\jj}{\mathbf{j}}
\newcommand{\kk}{\mathbf{k}}

\newcommand{\la}{\langle}
\newcommand{\ra}{\rangle}
\newcommand{\bp}{\bigskip}
\newcommand{\be}{\begin {equation}}
\newcommand{\ee}{\end {equation}}

\newcommand{\LRleq}{\stackrel{LR}{\leq}}

\numberwithin{equation}{section}


\def\flushl#1{\ifmmode\makebox[0pt][l]{${#1}$}\else\makebox[0pt][l]{#1}\fi}
\def\flushr#1{\ifmmode\makebox[0pt][r]{${#1}$}\else\makebox[0pt][r]{#1}\fi}
\def\flushmr#1{\makebox[0pt][r]{${#1}$}}


%\theoremstyle{Theorem}
% \newtheorem*{thmM}{Main Theorem}
% \crefformat{thmM}{main theorem}
% \Crefformat{thmM}{Main Theorem}
\newtheorem*{thm*}{Theorem}
\newtheorem{thm}{Theorem}[section]
\newtheorem{thml}[thm]{Theorem}
\newtheorem{lem}[thm]{Lemma}
\newtheorem{obs}[thm]{Observation}
\newtheorem{lemt}[thm]{Lemma}
\newtheorem*{lem*}{Lemma}
\newtheorem{whyp}[thm]{Working Hypothesis}
\newtheorem{prop}[thm]{Proposition}
\newtheorem{prpt}[thm]{Proposition}
\newtheorem{prpl}[thm]{Proposition}
\newtheorem{cor}[thm]{Corollary}
%\newtheorem*{prop*}{Proposition}
\newtheorem{claim}{Claim}
\newtheorem*{claim*}{Claim}
%\theoremstyle{definition}
\newtheorem{defn}[thm]{Definition}
\newtheorem{dfnl}[thm]{Definition}
\newtheorem*{IndH}{Induction Hypothesis}

\newtheorem*{eg*}{Example}
\newtheorem{eg}[thm]{Example}
\newtheorem{conj}[thm]{Conjecture}

\theoremstyle{remark}
\newtheorem{remark}[thm]{Remark}
\newtheorem{remarks}[thm]{Remarks}


\def\cpc{\sigma}
\def\ccJ{\epsilon\dotepsilon}
\def\ccL{c_L}

\def\wtbfK{\widetilde{\bfK}}
%\def\abfV{\acute{\bfV}}
\def\AbfV{\acute{\bfV}}
%\def\afgg{\acute{\fgg}}
%\def\abfG{\acute{\bfG}}
\def\abfV{\bfV'}
\def\afgg{\fgg'}
\def\abfG{\bfG'}

\def\half{{\tfrac{1}{2}}}
\def\ihalf{{\tfrac{\mathbf i}{2}}}
\def\slt{\fsl_2(\bC)}
\def\sltr{\fsl_2(\bR)}

% \def\Jslt{{J_{\fslt}}}
% \def\Lslt{{L_{\fslt}}}
\def\slee{{
\begin{pmatrix}
 0 & 1\\
 0 & 0
\end{pmatrix}
}}
\def\slff{{
\begin{pmatrix}
 0 & 0\\
 1 & 0
\end{pmatrix}
}}\def\slhh{{
\begin{pmatrix}
 1 & 0\\
 0 & -1
\end{pmatrix}
}}
\def\sleei{{
\begin{pmatrix}
 0 & i\\
 0 & 0
\end{pmatrix}
}}
\def\slxx{{\begin{pmatrix}
-\ihalf & \half\\
\phantom{-}\half & \ihalf
\end{pmatrix}}}
% \def\slxx{{\begin{pmatrix}
% -\sqrt{-1}/2 & 1/2\\
% 1/2 & \sqrt{-1}/2
% \end{pmatrix}}}
\def\slyy{{\begin{pmatrix}
\ihalf & \half\\
\half & -\ihalf
\end{pmatrix}}}
\def\slxxi{{\begin{pmatrix}
+\half & -\ihalf\\
-\ihalf & -\half
\end{pmatrix}}}
\def\slH{{\begin{pmatrix}
   0   & -\mathbf i\\
\mathbf i & 0
\end{pmatrix}}
}

\ExplSyntaxOn
\clist_map_inline:nn {J,L,C,X,Y,H,c,e,f,h,}{
  \expandafter\def\csname #1slt\endcsname{{\mathring{#1}}}}
\ExplSyntaxOff


\def\Mop{\fT}

\def\fggJ{\fgg_J}
\def\fggJp{\fgg'_{J'}}

\def\NilGC{\Nil_{\bfG}(\fgg)}
\def\NilGCp{\Nil_{\bfG'}(\fgg')}
\def\Nilgp{\Nil_{\fgg'_{J'}}}
\def\Nilg{\Nil_{\fgg_{J}}}
%\def\NilP'{\Nil_{\fpp'}}
\def\nNil{\Nil^{\mathrm n}}
\def\eNil{\Nil^{\mathrm e}}


\NewDocumentCommand{\NilP}{t'}{
\IfBooleanTF{#1}{\Nil_{\fpp'}}{\Nil_\fpp}
}

\def\KS{\mathsf{KS}}
\def\MM{\bfM}
\def\MMP{M}

\NewDocumentCommand{\KTW}{o g}{
  \IfValueTF{#2}{
    \left.\varsigma_{\IfValueT{#1}{#1}}\right|_{#2}}{
    \varsigma_{\IfValueT{#1}{#1}}}
}
\def\IST{\rho}
\def\tIST{\trho}

\NewDocumentCommand{\CHI}{o g}{
  \IfValueTF{#1}{
    {\chi}_{\left[#1\right]}}{
    \IfValueTF{#2}{
      {\chi}_{\left(#2\right)}}{
      {\chi}}
  }
}
\NewDocumentCommand{\PR}{g}{
  \IfValueTF{#1}{
    \mathop{\pr}_{\left(#1\right)}}{
    \mathop{\pr}}
}
\NewDocumentCommand{\XX}{g}{
  \IfValueTF{#1}{
    {\cX}_{\left(#1\right)}}{
    {\cX}}
}
\NewDocumentCommand{\PP}{g}{
  \IfValueTF{#1}{
    {\fpp}_{\left(#1\right)}}{
    {\fpp}}
}
\NewDocumentCommand{\LL}{g}{
  \IfValueTF{#1}{
    {\bfL}_{\left(#1\right)}}{
    {\bfL}}
}
\NewDocumentCommand{\ZZ}{g}{
  \IfValueTF{#1}{
    {\cZ}_{\left(#1\right)}}{
    {\cZ}}
}

\NewDocumentCommand{\WW}{g}{
  \IfValueTF{#1}{
    {\bfW}_{\left(#1\right)}}{
    {\bfW}}
}




\def\gpi{\wp}
\NewDocumentCommand\KK{g}{
\IfValueTF{#1}{K_{(#1)}}{K}}
% \NewDocumentCommand\OO{g}{
% \IfValueTF{#1}{\cO_{(#1)}}{K}}
\NewDocumentCommand\XXo{d()}{
\IfValueTF{#1}{\cX^\circ_{(#1)}}{\cX^\circ}}
\def\bfWo{\bfW^\circ}
\def\bfWoo{\bfW^{\circ \circ}}
\def\bfWg{\bfW^{\mathrm{gen}}}
\def\Xg{\cX^{\mathrm{gen}}}
\def\Xo{\cX^\circ}
\def\Xoo{\cX^{\circ \circ}}
\def\fppo{\fpp^\circ}
\def\fggo{\fgg^\circ}
\NewDocumentCommand\ZZo{g}{
\IfValueTF{#1}{\cZ^\circ_{(#1)}}{\cZ^\circ}}

% \ExplSyntaxOn
% \NewDocumentCommand{\bcO}{t' E{^_}{{}{}}}{
%   \overline{\cO\sb{\use_ii:nn#2}\IfBooleanTF{#1}{^{'\use_i:nn#2}}{^{\use_i:nn#2}}
%   }
% }
% \ExplSyntaxOff

\NewDocumentCommand{\bcO}{t'}{
  \overline{\cO\IfBooleanT{#1}{'}}}

\NewDocumentCommand{\oliftc}{g}{
\IfValueTF{#1}{\boldsymbol{\vartheta} (#1)}{\boldsymbol{\vartheta}}
}
\NewDocumentCommand{\oliftr}{g}{
\IfValueTF{#1}{\vartheta_\bR(#1)}{\vartheta_\bR}
}
\NewDocumentCommand{\olift}{g}{
\IfValueTF{#1}{\vartheta(#1)}{\vartheta}
}
% \NewDocumentCommand{\dliftv}{g}{
% \IfValueTF{#1}{\ckvartheta(#1)}{\ckvartheta}
% }
\def\dliftv{\vartheta}
\NewDocumentCommand{\tlift}{g}{
\IfValueTF{#1}{\wtvartheta(#1)}{\wtvartheta}
}

\def\slift{\cL}

\def\BB{\bB}


\def\PhiO#1{\vartheta\left(#1\right)}

\def\bbThetav{\check{\mathbbold{\Phi}}}
\def\Phiv{\check{\Phi}}
\def\Phiv{\check{\Phi}}

\DeclareDocumentCommand{\NN}{g}{
\IfValueTF{#1}{\fN(#1)}{\fN}
}
\DeclareDocumentCommand{\RR}{m m}{
\fR({#1},{#2})
}

%\DeclareMathOperator*{\sign}{Sign}

\NewDocumentCommand{\lsign}{m}{
{}^l\mathrm{Sign}(#1)
}



\NewDocumentCommand\lnn{t+ t- g}{
  \IfBooleanTF{#1}{{}^l n^+\IfValueT{#3}{(#3)}}{
    \IfBooleanTF{#2}{{}^l n^-\IfValueT{#3}{(#3)}}{}
  }
}


% % Fancy bcO, support feature \bcO'^a_{\mathrm b} = \overline{\cO'^a_{\mathrm b}}
% \makeatletter
% \def\bcO{\def\O@@{\cO}\@ifnextchar'\@Op\@Onp}
% \def\@Opnext{\@ifnextchar^\@Opsp\@Opnsp}
% \def\@Op{\afterassignment\@Opnext\let\scratch=}
% \def\@Opnsp{\def\O@@{\cO'}\@Otsb}
% \def\@Onp{\@ifnextchar^\@Onpsp\@Otsb}
% \def\@Opsp^#1{\def\O@@{\cO'^{#1}}\@Otsb}
% \def\@Onpsp^#1{\def\O@@{\cO^{#1}}\@Otsb}
% \def\@Otsb{\@ifnextchar_\@Osb{\@Ofinalnsb}}
% \def\@Osb_#1{\overline{\O@@_{#1}}}
% \def\@Ofinalnsb{\overline{\O@@}}

% Fancy \command: \command`#1 will translate to {}^{#1}\bfV, i.e. superscript on the
% lift conner.

% \def\defpcmd#1{
%   \def\nn@tmp{#1}
%   \def\nn@np@tmp{@np@#1}
%   \expandafter\let\csname\nn@np@tmp\expandafter\endcsname \csname\nn@tmp\endcsname
%   \expandafter\def\csname @pp@#1\endcsname`##1{{}^{##1}{\csname @np@#1\endcsname}}
%   \expandafter\def\csname #1\endcsname{\,\@ifnextchar`{\csname
%       @pp@#1\endcsname}{\csname @np@#1\endcsname}}
% }

% \def\defppcmd#1{
% \expandafter\NewDocumentCommand{\csname #1\endcsname}{##1 }{}
% }



% \defpcmd{bfV}
% \def\KK{\bfK}\defpcmd{KK}
% \defpcmd{bfG}
% \def\A{\!A}\defpcmd{A}
% \def\K{\!K}\defpcmd{K}
% \def\G{G}\defpcmd{G}
% \def\J{\!J}\defpcmd{J}
% \def\L{\!L}\defpcmd{L}
% \def\eps{\epsilon}\defpcmd{eps}
% \def\pp{p}\defpcmd{pp}
% \defpcmd{wtK}
% \makeatother

\def\fggR{\fgg_\bR}
\def\rmtop{{\mathrm{top}}}
\def\dimo{\dim^\circ}
\def\GKdim{\text{GK-}\dim}

\NewDocumentCommand\LW{g}{
\IfValueTF{#1}{L_{W_{#1}}}{L_{W}}}
%\def\LW#1{L_{W_{#1}}}
\def\JW#1{J_{W_{#1}}}

\def\floor#1{{\lfloor #1 \rfloor}}

\def\KSP{K}
\def\UU{\rU}
\def\UUC{\rU_\bC}
\def\tUUC{\widetilde{\rU}_\bC}
\def\OmegabfW{\Omega_{\bfW}}


\def\BB{\bB}


\def\PhiO#1{\vartheta\left(#1\right)}

\def\Phiv{\check{\Phi}}
\def\Phiv{\check{\Phi}}

\def\Phib{\bar{\Phi}}

\def\cKaod{\cK^{\mathrm{aod}}}

\DeclareMathOperator{\sspan}{span}


\def\sp{{\mathrm{sp}}}

\def\bfLz{\bfL_0}
\def\sOpe{\sO^\perp}
\def\sOpeR{\sO^\perp_\bR}
\def\sOR{\sO_\bR}

\def\ZX{\cZ_{X}}
\def\gdliftv{\vartheta}
\def\gdlift{\vartheta^{\mathrm{gen}}}
\def\bcOp{\overline{\cO'}}
\def\bsO{\overline{\sO}}
\def\bsOp{\overline{\sO'}}
\def\bfVpe{\bfV^\perp}
\def\bfEz{\bfE_0}
\def\bfVn{\bfV^-}
\def\bfEzp{\bfE'_0}

\def\totimes{\widehat{\otimes}}
\def\dotbfV{\dot{\bfV}}

\def\aod{\mathrm{aod}}
\def\unip{\mathrm{unip}}
\def\IC{\mathfrak{I}}

\def\PI#1{\Pi_{\cI_{#1}}}
\def\Piunip{\Pi^{\mathrm{unip}}}
\def\cf{\emph{cf.} }
\def\Groth{\mathrm{Groth}}
\def\Irr{\mathrm{Irr}}
\def\Irrsp{\mathrm{Irr}^{\mathrm{sp}}}

\def\edrc{\mathrm{DRC}^{\mathrm e}}
\def\drc{\mathrm{DRC}}
\def\LS{\mathrm{LS}}
\def\Unip{\mathrm{Unip}}


% Ytableau tweak
\makeatletter
\pgfkeys{/ytableau/options,
  noframe/.default = false,
  noframe/.is choice,
  noframe/true/.code = {%
    \global\let\vrule@YT=\vrule@none@YT
    \global\let\hrule@YT=\hrule@none@YT
  },
  noframe/false/.code = {%
    \global\let\vrule@YT=\vrule@normal@YT
    \global\let\hrule@YT=\hrule@normal@YT
  },
  noframe/on/.style = {noframe/true},
  noframe/off/.style = {noframe/false},
}
\makeatother


\def\wAV{\AV^{\mathrm{weak}}}
\def\ckG{\check{G}}
\def\ckGc{\check{G}_{\bC}}
\def\dBV{d_{\mathrm{BV}}}
\def\CP{\mathsf{CP}}
\def\YD{\mathsf{YD}}
\def\SYD{\mathsf{SYD}}
\def\DD{\nabla}

\def\lamck{\lambda_\ckcO}
\def\Lamck{[\lambda_\ckcO]}
\def\lamckb{\lambda_{\ckcO_{\mathrm b}}}
\def\lamckg{\lambda_{\ckcO_{\mathrm g}}}
\def\Wint#1{W_{[#1]}}
\def\CLam{\Coh_{\Lambda}}
\def\Cint#1{\Coh_{[#1]}}
\def\PP{\mathsf{PAP}}
\def\PAP{\mathsf{PAP}}
\def\BOX#1{\mathrm{Box}(#1)}
\DeclareDocumentCommand{\bigtimes}{}{\mathop{\scalebox{1.7}{$\times$}}}
\providecommand\mapsfrom{\scalebox{-1}[1]{$\mapsto$}}

\def\ihh{{i_\fhh}}

\def\Gc{G_\bC}
\def\Gcad{G_\bC^{\text{ad}}}
\def\Gad{\Inn(\fgg)}

\def\hha{{}^a\fhh}
\def\ahh{\hha}
\def\aSR{{}^a\Sigma}
\def\aRp{{}^a\Delta^+}
\def\aX{{}^aX}
\def\aQ{{}^aQ}
\def\aP{{}^aP}
\def\aR{{}^aR}
\def\aRp{{}^aR^+}
\def\asRp{{}^a \Delta^+}
\def\Gfin{\cG(\Gc)}
\def\PiGfin{\Pi_{\mathrm{fin}}( \Gc )}
\def\PiGlfin{\Pi_{\Lambda_0}( \Gc )}
\def\adGfin{\cG_{\mathrm{ad}}(\Gc)}
\def\Ggk{\cG(\fgg,K)}
\def\WT#1{\Delta(#1)}
\def\WG{W(\Gc)}
\def\ch{\mathrm{ch}\,}
\def\Wlam{W_{[\lambda]}}
\def\aLam{a_{\Lambda}}
\def\WLam{W_{\Lam}}
\def\WLamck{W_{[\lambda_{\ckcO}]}}
\def\Wlamck{W_{\lamck}}
\def\Rlam{R_{[\lambda]}}
\def\RLam{R_\Lambda}
\def\RLamp{R_\Lambda^+}
\def\Rplam{R^+(\lambda)}
\def\Glfin{\cG_{\Lambda}(\Gc)}
\def\CL{{\sC}^{\scriptscriptstyle L}}
\def\CR{{\sC}^{\scriptscriptstyle R}}
\def\CLR{{\sC}^{\scriptscriptstyle LR}}
\def\LV{{}^{\scriptscriptstyle L}\sV}
\def\LC{{}^{\scriptscriptstyle L}\sC}
\def\RC{{}^{\scriptscriptstyle R}\sC}
\def\LRC{{}^{\scriptscriptstyle LR}\sC}
\def\ckLC{{}^{\scriptscriptstyle L}\check{\sC}}

\def\LV{{}^{\scriptscriptstyle L}\sV}
\def\ckLV{{}^{\scriptscriptstyle L}\check\sV}
\def\ckLC{{}^{\scriptscriptstyle L}\check\sC}

\def\tLV{{}^{\scriptscriptstyle L}\widetilde{\sV}}
\def\tLC{{}^{\scriptscriptstyle L}\widetilde{\sC}}

\def\brsgn{\breve{\sgn}}
\def\bsgn{\overline{\sgn}}

\def\Wb{W_{\mathrm b}}
\def\Wg{W_{\mathrm g}}


\def\nbb{n_{\mathrm b}}
\def\ngg{n_{\mathrm g}}
\def\tU{\widetilde{\rU}}

\newcommand{\cross}{\times}
\newcommand{\crossa}{\times^a}

\def\bVL{{\overline{\sV}}^{\scriptscriptstyle L}}
\def\bVR{{\overline{\sV}}^{\scriptscriptstyle R}}
\def\bVLR{{\overline{\sV}}^{\scriptscriptstyle LR}}
\def\VL{{\sV}^{\scriptscriptstyle L}}
\def\VR{{\sV}^{\scriptscriptstyle R}}
\def\VLR{{\sV}^{\scriptscriptstyle LR}}

\def\Con{\sfC}
\def\bCon{\overline{\sfC}}
\def\Re{\mathrm{Re}}
\def\Im{\mathrm{Im}}
\def\AND{\quad \text{and} \quad}
\def\Coh{\mathrm{Coh}}
\def\Cohlm{\Coh_{\Lambda}(\cM)}
\def\ev#1{{\mathrm{ev}_{#1}}}

\def\ppp{\times}
\def\mmm{\slash}


\def\cuprow{{\stackrel{r}{\sqcup}}}
\def\cupcol{{\stackrel{c}{\sqcup}}}

\def\Spr{\mathrm{Springer}}
\def\Prim{\mathrm{Prim}}



\def\imathp{\imath_{\wp}}
\def\jmathp{\jmath_{\wp}}

\def\CQ{\overline{\sfA}}% Lusztig's canonical quotient
\def\CPP{\mathrm{PP}}
\def\CPPs{\mathrm{PP}_{\star}}
%\def\CPP{\mathfrak{P}}
%\def\CPPs{\mathfrak{P}_{\star}}


\def\ceil#1{\lceil #1 \rceil}
\def\symb#1#2{{\left(\substack{{#1}\\{#2}}\right)}}
\def\cboxs#1{\mbox{\scalebox{0.25}{\ytb{\ ,\vdots,\vdots,\ }}}_{#1}}

\def\hsgn{\widetilde{\mathrm{sgn}}}

\def\tPBP{\widetilde{\mathsf{PBP}}}
\def\PBPe{\mathsf{PBP}^{\mathrm{ext}}}
\def\PBPes{\mathsf{PBP}^{\mathrm{ext}}_{\star}}
\def\PBPsb{\mathsf{PBP}_{\star,b}}

\def\bev#1{\overline{\mathrm{ev}}_{#1}}

\def\Prim{\mathrm{Prim}}
% \def\leqL{\stackrel{L}{\leq}}
% \def\leqR{\stackrel{R}{\leq}}
% \def\leqLR{\stackrel{LR}{\leq}}

% \def\leqL{{\leq_L}}
% \def\leqR{{\leq_R}}
% \def\leqLR{{\leq_{LR}}}


\def\Dsp{\cD^{\text{sp}}}
\def\Csp{\sfC^{\text{sp}}}


\def\lneqL{\mathrel{\mathop{\lneq}\limits_{\scriptscriptstyle L}}}
\def\lneqR{\mathrel{\mathop{\lneq}\limits_{\scriptscriptstyle R}}}
\def\lneqLR{\mathrel{\mathop{\lneq}\limits_{\scriptscriptstyle LR}}}

\def\leqL{\mathrel{\mathop{\leq}\limits_{\scriptscriptstyle L}}}
\def\leqR{\mathrel{\mathop{\leq}\limits_{\scriptscriptstyle R}}}
\def\leqLR{\mathrel{\mathop{\leq}\limits_{\scriptscriptstyle LR}}}


\def\approxL{\mathrel{\mathop{\approx}\limits_{\scriptscriptstyle L}}}
\def\approxR{\mathrel{\mathop{\approx}\limits_{\scriptscriptstyle R}}}
\def\approxLR{\mathrel{\mathop{\approx}\limits_{\scriptscriptstyle LR}}}


\def\dphi{\rdd \phi}
\def\CPH{C(H)}
\def\whCPH{\widehat{C(H)}}

\def\Greg{G_{\text{reg}}}
\def\Hnreg{H^-_{\text{reg}}}
\def\Hireg{H_{i,\text{reg}}}
\def\Hnireg{H^-_{i,\text{reg}}}


\def\tsgn{\widetilde{\sgn}}
\def\PBP{\mathsf{PBP}}

\def\ckstar{{\check \star}}
\def\ckfgg{{\check \fgg}}

\def\Inn{\mathrm{Ad}}

\providecommand{\nsubset}{\not\subset}

\def\cuprow{{\,\stackrel{r}{\sqcup}\,}}
\def\cupcol{{\,\stackrel{c}{\sqcup}\,}}


\def\ckcOp{\ckcO'}
\def\ckcOpp{\ckcO''}



\def\ckcOb{\ckcO_{\mathrm b}}
\def\ckcOpb{\ckcO'_{\mathrm b}}
\def\cOpb{\cO'_{\mathrm b}}
\def\ckcOg{\ckcO_{\mathrm g}}

\def\nng{n_{\mathrm g}}
\def\nnb{n_{\mathrm b}}
\def\Gb{G_{\mathrm b}}
\def\Gpb{G'_{\mathrm b}}
\def\Pb{P_{\mathrm b}}
\def\Gg{G_{\mathrm g}}
\def\ckGb{\ckG_{\mathrm b}}
\def\ckGg{\ckG_{\mathrm g}}

\def\bcOb{\overline{\cO_{\mathrm b}}}
\def\bcOg{\overline{\cO_{\mathrm g}}}

\def\lamb{\lambda_{\mathrm b}}
\def\lamg{\lambda_{\mathrm g}}


\def\tPBP{\widetilde{\mathsf{PBP}}}
\def\PBPs{\mathsf{PBP}_{\star}}
\def\PBPe{\mathsf{PBP}^{\mathrm{ext}}}
\def\PBPes{\mathsf{PBP}^{\mathrm{ext}}_{\star}}
\def\PBPsb{\mathsf{PBP}_{\star,b}}




\def\PBPop#1#2#3#4{\PBP_{#1}^{#2}(#3,#4)}
\newcommand{\PBPOP}[1][]{\PBPop{\star}{#1}{\ckcO}{\wp}}
% \def\PBPdOP{\PBPop{\star}{\mathrm{d}}{\ckcO}{\wp}}
% \def\PBPrcOP{\PBPop{\star}{\mathrm{rc}}{\ckcO}{\wp}}
% \def\PBPsOP{\PBPop{\star}{\mathrm{s}}{\ckcO}{\wp}}
\def\PBPOPp{\PBPop{\star'}{}{\ckcO'}{\wp'}}
%\def\PBPOPpp{\PBPop{\star}{}{\ckcO''}{\wp''}}
\newcommand{\PBPOPpp}[1][]{\PBPop{\star}{#1}{\ckcO''}{\wp''}}

% \def\PBPdOPpp{\PBPop{\star}{\mathrm{d}}{\ckcO''}{\wp}}
% \def\PBPrcOPpp{\PBPop{\star}{\mathrm{rc}}{\ckcO''}{\wp}}
% \def\PBPsOPpp{\PBPop{\star}{\mathrm{s}}{\ckcO''}{\wp}}

\def\PBPGOP{\PBPop{G}{}{\ckcO}{\wp}}
\def\PBPGOPp{\PBPop{G'}{}{\ckcO'}{\wp'}}
\def\PBPGOPpp{\PBPop{G''}{}{\ckcO''}{\wp''}}


\def\yrow#1{\left[#1\right]_{\mathrm{row}}}
\def\ycol#1{\left[#1\right]_{\mathrm{col}}}


\def\Ass{A_{\mathrm{s}}}
\def\Ans{A_{\mathrm{ns}}}
\def\wpu{\wp_{\uparrow}}
\def\wpm{\wp_{\downarrow}}
\def\wpd{\wp} % define the done-wp to be \wp
\def\uptauu{\uptau_{\uparrow}}
\def\uptaud{\uptau} % define the done-tau to be \uptau


 \def\tnaive{\mathrm{naive}}


\def\imathwpp{\imath_{\wp'}}
\def\jmathwpp{\jmath_{\wp'}}
\def\cPpn{\cP'_\mathrm{naive}}
\def\cQpn{\cQ'_\mathrm{naive}}


\def\PPm{\wp_{\downarrow}}
\def\tauwpp{\tau'_{\wp'}}
\def\tauwppp{\tau''_{\wp''}}
\def\uptaum{\uptau_{\downarrow}}
\def\uptaupn{\uptau'_{\tnaive}}
\def\alphapn{\alpha'_{\tnaive}}
\def\imathpn{\imath'_{\tnaive}}
\def\jmathpn{\jmath'_{\tnaive}}

\def\eDD{\mathrm{d}^{e}}
\def\ckDD{{\check\DD}}
\def\DD{\nabla}
\def\DDn{\nabla_{\tnaive}}
\def\ckDDn{{\ckDD}_{\tnaive}}
\def\DDD{{\check\nabla}}
\def\DDc{\boldsymbol{\nabla}}
\def\gDD{\nabla^{\mathrm{gen}}}
\def\gDDc{\boldsymbol{\nabla}^{\mathrm{gen}}}

\newcommand{\Lam}{{[\lambda]}}

\newcommand{\Rg}{\cR(\fgg)}
\newcommand{\Grt}{\cK}
\newcommand{\nckG}{n_{\ckG}}
%\newcommand{\nb}{n_{\mathrm b}}
%\newcommand{\ng}{n_{\mathrm g}}

\usepackage{xr}
\usepackage{subfiles} % Best loaded last in the preamble



\begin{document}


\title[Special unipotent representations]{Special unipotent representations of real classical groups: counting and reduction to good parity}

\author [D. Barbasch] {Dan Barbasch}
\address{Department of Mathematics\\
  310 Malott Hall, Cornell University, Ithaca, New York 14853 }
\email{dmb14@cornell.edu}

\author [J.-J. Ma] {Jia-Jun Ma}
\address{School of Mathematical Sciences\\
  Xiamen University\\
  Xiamen, China} \email{hoxide@xmu.edu.cn}

\author [B. Sun] {Binyong Sun}
% MCM, HCMS, HLM, CEMS, UCAS,
\address{Institute for Advanced Study in Mathematics\\
 Zhejiang University\\
  Hangzhou, China} \email{sunbinyong@zju.edu.cn}
  %310058

%\address{Academy of Mathematics and Systems Science\\
  %Chinese Academy of Sciences\\
  %Beijing, 100190, China} \email{sun@math.ac.cn}

\author [C.-B. Zhu] {Chen-Bo Zhu}
\address{Department of Mathematics\\
  National University of Singapore\\
  10 Lower Kent Ridge Road, Singapore 119076} \email{matzhucb@nus.edu.sg}




\subjclass[2010]{22E46, 22E47} \keywords{Special unipotent representation, associated variety, coherent continuation, primitive ideal, cell, classical group}

% \thanks{Supported by NSFC Grant 11222101}

\begin{abstract} Let $G$ be a real reductive group in Harish-Chandra's class. We derive some consequences of theory of coherent continuation representations
%, primitive ideals and cells
 to the counting of irreducible representations of $G$ with a given infinitesimal character and a given bound in the complex associated variety. When $G$ is a real classical group (including the real metaplectic group), we investigate the set of special unipotent representations of $G$ attached to $\check \CO$, in the sense of Barbasch and Vogan. Here $\check \CO$ is a nilpotent adjoint orbit in the Langlands dual of $G$ (or the metaplectic dual of $G$ when $G$ is a real metaplectic group). We give a precise count for the number of special unipotent representations of $G$ attached to $\check \CO$, when $\check \CO$ has good parity. We also reduce the problem of constructing special unipotent representations attached to $\check \CO$ to the case when $\check \CO$ has good parity. The paper is the first in a series of two papers on the classification of special unipotent representations of real classical groups.
\end{abstract}




\maketitle



\tableofcontents






\section{Introduction}\label{sec:intro}

\subsection{Background and goals} In \cite{ArPro,ArUni}, Arthur introduced certain families of representations of a semisimple algebraic group over $\R$ or $\C$, in connection with his conjecture on square-integrable automorphic forms. They are the special unipotent representations in the title of this paper and are attached to a nilpotent adjoint orbit $\check \CO$ in the Langlands dual of $G$. For its definition, we will use that of Barbasch-Vogan \cite{BVUni}, which refines/makes precise that of Arthur and is in the language of primitive ideals.

Apart from their clear interest for the theory of automorphic forms, special unipotent representations belong to a fundamental class of unitary representations which are associated to nilpotent coadjoint orbits in
the Kirillov-Kostant philosophy (the orbit method; see \cite{Ki62,Ko70,VoBook}). These are known informally as unipotent representations, which are expected to play a central role in the unitary dual problem and therefore have been a subject of great interest. See \cite{VoICM,VoBook,Vo89}.
%informal because we are still in search of the right definition.

In the context alluded to, a long standing problem in representation theory, known as the Arthur-Barbasch-Vogan conjecture \cite[Introduction]{ABV}, is to show that every special unipotent representation is unitary.

In a series of two papers, the authors will classify special unipotent representations of real classical groups, and will prove the Arthur-Barbasch-Vogan conjecture for these groups as a consequence of the classification.

The current paper is the first in the series, which have the following two goals. The first is to give a precise count of special unipotent representations attached to $\check \CO$, when $\check \CO$ has good parity. The second is to reduce the problem of constructing special unipotent representations attached to $\check \CO$ to the case when $\check \CO$ has good parity.

\subsection{Approach}
We will work in the category of Cassellman-Wallach representations \cite[Chapter 11]{Wa2}. The main tool for the counting of special unipotent representations is the coherent continuation representation for this category. It is a representation of a certain integral Weyl group, and it can be compared, through a certain embedding, with an analogous coherent continuation representation for the category of highest weight modules. The latter has been intensively studied in Kazhdan-Lusztig theory and is amenable to detailed analysis through the theory of cells and special representations (in the sense of Lusztig), as well as the theory of primitive ideals. In particular one may derive precise information on what representations of the integral Weyl group may enter in the coherent continuation representation. In effectively carrying out the mentioned steps, we build on earlier ideas of several authors including Vogan \cite{Vg}, Joseph \cite{J1,J2},  King \cite{King}, Barbasch-Vogan \cite{BVUni}, Casian \cite{Cas}, as well as Soergel \cite{Soergel}. Putting things together, we arrive at counting results of general nature on irreducible representations of $G$ with a given infinitesimal character and a given bound in the complex associated variety, which are valid for any real reductive group $G$ in Harish-Chandra's class.

Thanks to a version of the Langlands classification (\cite{Vg}), we may also calculate the coherent continuation representation explicitly (an unpublished result of Barbasch and Vogan). In the case of real classical groups, this will ultimately yield an upper bound of the count of special unipotent representations attached to $\check \CO$, when $\check \CO$ has good parity.

It is worthwhile to note, while the algebraic theory developed in the article yields an explicit upper bound of the count, we are unable to demonstrate the precise count using the algebraic theory alone, due to a certain technical issue on the relationship of a Harish-Chandra cell and a Lusztig double cell, which we have formerly stated as Conjecture \ref{conjcell}. In the case at hand, namely for the real classical groups, we rely on the analytic theory of theta lifting to construct the right number of special unipotent representations in the case of good parity (\cite{BMSZ2}, the second paper in the series), thus arriving at the precise count. It will be clearly desirable to demonstrate the precise count, without recourse to the analytic theory.

In general when $\check \CO$ is not necessarily of good parity, we introduce a $\check \CO$-relevant parabolic subgroup $P$, whose Levi quotient is of the form
 $\Gpb\times \Gg $, where $\Gpb$ is a general linear group, and $\Gg$ is of the same classical type as $G$. In their respective Langlands duals, we have nilpotent adjoint orbits
$\check \CO'_\mathrm b$ and $\check \CO_\mathrm g$ (determined by $\check \CO$). We will show that the normalized smooth parabolic induction from $P$ to $G$ yields a bijective map from a pair of special unipotent representations of $\Gpb$ (attached to $\check \CO'_\mathrm b$) and $\Gg$ (attached to $\check \CO_\mathrm g$) to special unipotent representations of $G$ (attached to $\check \CO$). We need two ingredients in achieving this reduction step. First by using naturality of coherent continuation representation and the well-known technique of translating to a regular infinitesimal character, we show that the afore-mentioned map is injective. Second, there is a quasi-split classical group $\Gb$ (of the same type as $G$) with the following properties:
\begin{itemize}
\item $\Gb\times \Gg $ is an endoscopy group of $G$;
\item $\Gpb$ is naturally isomorphic to the Levi quotient of a $\check \CO_\mathrm b$-relevant parabolic subgroup $P_\mathrm b$ of $\Gb$.
\end{itemize}
Here $\check \CO_\mathrm b =2\check \CO'_\mathrm b$ (in Young diagram notation) is a nilpotent adjoint orbit in the Langlands dual $\check \Gb$ of $\Gb$.
General considerations as well as detailed information about coherent continuation representation allow us to separate bad parity and good parity, namely we may describe special unipotent representations of $G$ in terms of those of $\Gb$ and $\Gg$. On the other hand, the normalized smooth parabolic induction from $P_\mathrm b$ to $\Gb$ yields a bijection from special unipotent representations of $\Gpb$ (attached to $\check \CO'_\mathrm b$) to special unipotent representations of $\Gb$ (attached to $\check \CO_\mathrm b$). In view of the injectivity, this will finally imply the bijectivity of the induction map from $P$ to $G$.


\subsection{Organization}
%\vskip.25in
%Here are some words on the contents and the organization of this article.
In Section 2, we state our main results on the counting of special unipotent representations of real classical groups. The answers are given in terms of certain combinatorial constructs called painted bipartitions. In Section 3, we develop some generalities on coherent continuation representations for highest weight modules. In Section 4, we prove some analogous results on coherent continuation representations of Casselman-Wallach representations and derive the consequences to the counting of irreducible representations of $G$ with a given infinitesimal character and a given bound in the complex associated variety (Theorems \ref{count1}, \ref{count2} and \ref{counteq}). In Section 4, we separate good parity and bad parity, as a preparation for the reduction step. In Section 5, we give explicit formulas for the coherent continuation representation, based on an unpublished result of Barbasch and Vogan. Sections 6 to 7 are devoted to proof of the main results on the counting of special unipotent representations, in the case of good parity. In Section 8, we carry out the reduction step to the case of good parity.




\section{The main results on special unipotent representations}

In this section, we formulate the main results on special unipotent representations of real classical groups that will be proved in this article.


\subsection{Notations}\label{secnot}

Let $\g$ be a reductive complex Lie algebra. Its universal enveloping algebra is denoted by
$\mathcal U(\g)$, and the center of $\mathcal U(\g)$ is denoted by $\CZ(\g)$. Let
 $\hha$ denote the  universal Cartan algebra of $\g$    (also called the abstract Cartan subalgbra in \cite{V4}). Let $W\subset \GL(\hha)$ denote the Weyl group.
 By the Harish-Chandra isomorphism, every $\nu \in \hha^*$ (a superscript $*$ indicates the dual space)
determines an algebraic character $\chi_\nu: \CZ(\g)\rightarrow \C$. By a result of Dixmier \cite[Section 3]{Bor}, there is a unique maximal ideal of $\oU(\g)$ containing the kernel of $\chi_\nu$, which is denoted by $I_\nu$.
Note that the maps
\[
  \hha^*\rightarrow \{\textrm{algebraic character of $\CZ(\g)$}\}, \quad \nu\mapsto \chi_\nu
\]
and
\[
  \hha^*\rightarrow \{\textrm{maximal ideal  of $\CU(\g)$}\}, \quad \nu\mapsto I_\nu
\]
are both surjective, and the fibers of these two maps are precisely the $W$-orbits.

Let $\mathrm{Ad}(\g)$ denote the inner automorphism group of $\g$. Let $\Nil(\g)$ denote the set of nilpotent elements in $[\g,\g]$. Put
\[
 \overline{\Nil}(\g):=\Inn(\g)\backslash \Nil(\g), 
 \]
the set of $\Inn(\check \g)$-orbits in $\Nil(\g)$ (which is finite). Likewise
let $\Nil(\g^*)$ denote the set of nilpotent elements in $[\g,\g]^*$ and put
\[
 \overline{\Nil}(\g^*):=\Inn(\g)\backslash \Nil(\g^*).
 \]
Since $\g$ is the direct sum of its center with $[\g, \g]$, $[\g,\g]^*$ is obviously viewed as a subspace of $\g^*$.  The Killing form on $[\g,\g]$ then yields an identification $\overline{\Nil}(\g)=\overline{\Nil}(\g^*)$.

For a real reductive group $G$, denote by $\Irr(G)$ the set of isomorphism classes of
irreducible Casselman-Wallach representations of $G$. 


 \subsection{Special unipotent representations of real classical groups}
 \label{sec:defunip}

% We are particularly interested in counting special unipotent representations of real classical groups.

 Let $\star$ be one of the 10 symbols
 \[
   \textrm{ $A^\R$, $A^\bH$, $A$, $\widetilde A$,  $B$, $D$, $C$, $\wtC$,
     $D^*$, $C^*$. }
 \]
 Suppose that $G$ is a classical Lie group of type $\star$, namely $G$
 respectively equals one of the following Lie groups:
 \[
   \begin{array}{c}
     \GL_n(\R),  \  \GL_{\frac{n}{2}}(\bH)\ (n \textrm{ is even}),\  \oU(p,q),\  \widetilde \oU(p,q), \smallskip\\
     \SO(p,q)\ (p+q\, \textrm{ is odd}),  \  \SO(p,q)\  (p+q\, \textrm{ is even}),  \  \Sp_{2n}(\R),  \   \widetilde \Sp_{2n}(\R), \ \smallskip\\
    \oO^*(2n), \  \Sp(\frac{p}{2},\frac{q}{2})  \ (p,q\, \textrm{ are even})\qquad (n, p, q\in \BN:=\{0,1,2,\cdots\}).
   \end{array}
 \]
 Here $\wtSp_{2n}(\R)$ denotes the metaplectic double cover of the symplectic
 group $\Sp_{2n}(\R)$ that does not split unless $n=0$, and $\tU(p,q)$ is the double cover of $\rU(p,q)$ defined by a square root of the determinant character.
Denote by $\g$ the complexified Lie algebra of $G$. % for the rest of this section.

Let
$\check \g$ denote the complex Lie algebra
\[
   \begin{array}{c}
     \g\l_n(\C),  \  \g\l_{n}(\C),\  \g\l_{p+q}(\C), \  \g\l_{p+q}(\C),\smallskip\\
     \s\p_{p+q-1}(\C),  \  \o_{p+q}(\C),\
     \o_{2n+1}(\C),  \ \s\p_{2n}(\C), \ \smallskip\\   \o_{2n}(\C), \ \textrm{or } \   \o_{p+q+1}(\C),
   \end{array}
 \]
 respectively.

 If $\star\neq \wtC$, then $\check \g$ is  the Langlands dual of $\g$ so that the dual space $\hha^*$ is identified with the universal Cartan algebra $\ckhha$ of $\check \g$.
 If  $\star=\wtC$, then $\check \g=\g$, to be  called the metaplectic Langlands dual of $\g$.  In this case, $\hha=\ckhha$, and  $\hha^*$ is identified with  $\hha$ by  using the half of the trace form on the Lie algebra $\g=\s\p_{2n}(\C)$. Thus in all cases we have the identification $\hha^*=\ckhha$.





% Define the Langlands dual $\ckG$ of $G$ to be respectively the complex group
% \[
%   \begin{array}{c}
%     \GL_n(\C),  \  \GL_{2n}(\C),\  \GL_{p+q}(\C), \  \GL_{p+q}(\C),\smallskip\\
%     \Sp_{p+q-1}(\C)\ (p+q\, \textrm{ is odd}),  \  \SO_{p+q}(\C)\  (p+q\, \textrm{ is even}),\smallskip\\
 %    \SO_{2n+1}(\C),  \ \Sp_{2n}(\C), \  \SO_{2n}(\C), \ \textrm{or } \   \SO_{2p+2q+1}(\C).
%   \end{array}
% \]
 % Write $\check \g$ for the Lie algebra of $\ckG$, and
 Let $\check \CO\in \overline{\mathrm{Nil}}(\check \g)$ be a nilpotent $\Inn(\check \g)$-orbit. Write $\mathbf d_{\check \CO}$ for the Young diagram attached to  $\check \CO$. It determines the nilpotent orbit $\check \CO$ unless $\check \g=\o_{4k}(\C)$ ($k\in \bN^+:=\{1,2,3, \dots\}$) and all the row lengths are even.
When there is no confusion, we will not distinguish $\check \CO$ and  $\mathbf d_{\check \CO}$.

 Let $\lambda_{\ckcO}^\circ \in \check \g$ be half of the neutral element in any
 $\fsl_{2}$ triple attached to $\ckcO$, as in \cite[Section 5]{BVUni}. It is a semisimple element and is uniquely determined up to conjugation by $\Inn(\check \g)$.
Using the identification
 \be\label{sse}
   \Inn(\check \g)\backslash  \{\textrm{semisimple element in $\check \g$}\}=W\backslash \ckhha=W\backslash \hha^*,
 \ee
we pick an arbitrary element $\lambda_{\check \CO}\in \hha^*$ that represents the same element in  $ W\backslash \hha^*$  as $\lambda_{\check \CO}^\circ$.
%By abuse of notation, we still use $\lambda_{\check \CO}$ to denote this element of $\hha^*$.
Write $I_{\check \CO}:=I_{\star, \check \CO}:=I_{\lambda_{\check \CO}}$.

 Let $\CO:=d_{\mathrm{BV}}(\check \CO)\in \overline{\Nil}(\g^*)$ denote the Barbasch-Vogan dual of $\check \CO$, namely the  unique Zariski open $\Ad(\g)$-orbit in the associated variety of $I_{\check \CO}$. Then a primitive ideal of $\CU(\g)$ equals $I_{\check \CO}$ if and only if it contains the kernel of $\chi_{\lambda_{\check \CO}}$ and its associated variety  equals the Zariski closure of $\CO$ in $\g^*$.   See \cite[Appendix]{BVUni} and \cite[Section 1]{BMSZ1}.

 % (the maximal ideal of $\mathcal U(\g)$ with infinitesimal character $\lambda_{\check \CO}$, namely  containing the kernel of $\chi_\lambda$).

%We remark that in the metaplectic case, namely when $\star=\wtC$ and $G=\widetilde \Sp_{2n}(\R)$,  $\hha$ is identified with $\hha^*$ by using the half of the trace form on the Lie algebra $\s\p_{2n}(\C)$, and hence \eqref{sse} still holds.

 %By using Harish-Chandra isomorphism, we view $\lambda_{\check \CO}$ as a character $\lambda_{\check \CO}: \mathcal Z(\g)\rightarrow \C$.


Following Barbasch-Vogan \cite{BVUni}, define the set of the special unipotent representations of $G$
 attached to $\ckcO$ by
\[
 %\begin{equation}\label{eq:defuni}
   \begin{split}
     \Unip_{\ckcO}(G):=&  \Unip_{\star, \ckcO}(G) \\
     :=& \begin{cases}
       % \{\pi\in \Irr(G)\mid \pi \textrm{ is annihilated by $ I_{\check \CO}$ or $I'_{\check \CO}$}\}, & \text{if } \star \in \set{D, D^\C, D^*};\\
       \{\pi\in \Irr(G)\mid \pi \textrm{ is genuine  and annihilated by } I_{\check \CO}\}, & \text{if } \star\in \{\widetilde A, \widetilde C\};\\
       \{\pi\in \Irr(G)\mid \pi \textrm{ is annihilated by } I_{\check \CO}\}, & otherwise.\\
     \end{cases}
   \end{split}
% \end{equation}
\]
 Here ``genuine" means that the central subgroup $\{\pm 1\}$ of $G$, which is the kernel of the covering homomorphism $\widetilde \oU(p,q)\rightarrow  \oU(p,q)$ or $\widetilde \Sp_{2n}(\R)\rightarrow \Sp_{2n}(\R)$, acts on $\pi$ through the nontrivial character.
 % of $G$ representation $\pi$ of $\widetilde \oU(p,q)$ or $\widetilde \Sp_{2n}(\R)$ does not descend to $\oU(p,q)$ or $\Sp_{2n}(\R)$, respectively.

The main goal of this paper and the second paper in the series \cite{BMSZ2} is to count the set $\Unip_{\check \CO}(G)$ and to construct all the representations in $\Unip_{\check \CO}(G)$.

%In view of \Cref{cor:bound}, we will explicitly determine both $\LC_\lambda$ and $\Coh_{\Lam}(\CK(G))$ (for $\lambda =\lambda_{\ckcO}$) and will express the sum $\sum_{\sigma\in \LC_\lambda} [\sigma: \Coh_{\Lam}(\CK(G))]$ as the count of certain combinatorial constructs. These combinatorial constructs provide the key linkage with the authors' second paper \cite{BMSZ2}, whose main goal is to construct all the representations in $\Unip_{\check \CO}(G)$.

\subsection{General linear groups}\label{sec:GLRH}


All special unipotent representations of $\GL_n(\bR)$ and $\GL_{\frac{n}{2}}(\bH)$  are obtained via normalized smooth parabolic induction from quadratic characters (see \cite{V.GL}*{Page 450}).
We will review their classifications in the framework of this article.


  For a Young diagram $\imath$, write
 \[
   \mathbf r_1(\imath)\geq \mathbf r_2(\imath)\geq \mathbf r_3(\imath)\geq \cdots
 \]
 for its row lengths, and similarly, write
 \[
   \mathbf c_1(\imath)\geq \mathbf c_2(\imath)\geq \mathbf c_3(\imath)\geq \cdots
 \]
 for its column lengths. Denote by
 $\abs{\imath}:=\sum_{i=1}^\infty \mathbf r_i(\imath)$ the total size of
 $\imath$.

For any Young diagram $\imath$, we introduce the set $\mathrm{Box}(\imath)$ of
boxes of $\imath$ %as the following subset of $\bN^+\times \bN^+$:
as follows: 
\[
% \begin{equation}\label{eq:BOX}
  \mathrm{Box}(\imath):=\Set{(i,j)\in\bN^+\times \bN^+| j\leq \bfrr_i(\imath)}.
%\end{equation}
\]

% We will also call a subset of $\bN^+\times \bN^+$ of the form \eqref{eq:BOX} a
% Young diagram.

% We say that a Young diagram $\imath'$ is contained in $\imath$ (and write
% $\imath'\subset \imath$) if
% \[
%   \mathbf r_i(\imath')\leq \mathbf r_i(\imath)\qquad \textrm{for all
% } i=1,2, 3, \cdots.
% \]
% When this is the case, $\mathrm{Box}(\imath')$ is viewed as a subset of
% $\mathrm{Box}(\imath)$ concentrating on the upper-left corner. We say that a
% subset of $\mathrm{Box}(\imath)$ is a Young subdiagram if it equals
% $\mathrm{Box}(\imath')$ for a Young diagram $\imath'\subset \imath$. In this
% case, we call $\imath'$ the Young diagram corresponding to this Young
% subdiagram.

\renewcommand{\CP}{\mathcal{P}} We also introduce five symbols $\bullet$, $s$,
$r$, $c$ and $d$, and make the following definitions.
\begin{defn}
  A painting on a Young diagram $\imath$ is a map
  \[
    \mathcal P: \mathrm{Box}(\imath) \rightarrow \{\bullet, s, r, c, d \}
  \]
  with the following properties:
  \begin{itemize}
    \item $\mathcal P^{-1}(S)$ is the set of boxes of a Young diagram when
          $S=\{\bullet\}, \{\bullet, s \}, \{\bullet, s, r\}$ or
          $\{\bullet, s, r, c \} $;
    \item when $S=\{s\}$ or $ \{r\}$, every row of $\imath$ has at most one box
          in $\CP^{-1}(S)$;
    \item when $S=\{c\}$ or $ \{d \}$, every column of $\imath$ has at most one
          box in $\CP^{-1}(S)$.
  \end{itemize}
A painted Young diagram is a pair $(\imath, \CP)$ consisting of a Young diagram $\imath$ and a painting $\CP$ on $\imath$.
\end{defn}



\begin{defn}\label{defpbp0}
  Suppose that $\star\in \{A^\R, A^\bH\}$. A painting $\CP$ on a Young diagram
  $\imath$ has type $\star$ if
  \begin{itemize}
    \item the image of $\CP$ is contained in
          \[
          \left\{
          \begin{array}{ll}
            \{\bullet, c, d\}, &\hbox{if $\star=A^\R$}; \smallskip\\
            \{\bullet\}, &\hbox{if $\star=A^\bH$},                      \end{array}
        \right.
          \]
    \item $\CP^{-1}(\bullet)$ has even number of
          boxes in every column of $\imath$.  \end{itemize}
  Denote by $\PAP_\star(\imath)$ the set of paintings on $\imath^{t}$ that has type $\star$, where $\imath^{t}$
  is the transpose of $\imath$.
   \end{defn}

%Note that in the definition of $\PAP_\star(\imath^{t})$, we have incorporated the transpose map in order to reconcile with the Barbasch-Vogan duality.

The middle letter $\mathrm A$ in $\PAP$ refers to the common $A$ in $\{A^\R, A^\bH\}$.

It is easy to check that if $\star=A^\R$, then
  \[
    \sharp(\PAP_\star(\check \CO))=\prod_{i\in \bN^+} (1+\textrm{the
      number of rows of length $i$ in $\check \CO$}),
  \]
  and if $\star=A^\bH$, then
  \[
    \sharp(\PAP_\star(\check \CO))= \left\{
      \begin{array}{ll}
        1, &\hbox{if all row lengths of $\check \CO$ are even}; \smallskip\\
        0, &\hbox{otherwise}.  \end{array}
    \right.
  \]
Here and henceforth $\sharp$ indicates the cardinality of a finite set.



% We assume that $G  = \GL_{n}(\bC), \GL_{n}(\bR), \GL_{\half n}(\bH)$ and $\star = A,A^{\bC}, A^{\bH}$ respectively.

%



% We list some easy facts below:
% \[
%   \begin{split}
%     W &= \sfS_{n} \\
%     W_{[\lamck]} & = \sfS_{\abs{\ckcO_{e}}}\times \sfS_{\abs{\ckcO_{o}}} \\
%     W_{\lamck} & = \prod_{i\in \bN^{+}}\sfS_{\bfcc_{i}(\ckcO_{e})}\times \prod_{i\in \bN^{+}}\sfS_{\bfcc_{i}(\ckcO_{o})}\\\
%     \tau_{\lamck} & := (j_{W_{\lamck}}^{W_{[\lamck]}}\sgn )\otimes \sgn =  \ckcO_{e}^{t}\boxtimes \ckcO_{o}^{t}\\
%     \LC_{\lamck} & = \LRC_{\lamck}= \set{\tau_{\lamck}}, \AND \\
%     \wttau_{\lamck} & = j_{W_{[\lamck]}}^{W}\tau_{\lamck} = \Spr^{-1}(\cO).
%   \end{split}
% \]


% Now let $\ckcO\in \Nil(\ckGc)$ and decompose
% \[
%   \ckcO = \ckcO_{e} \cuprow \ckcO_{o}
% \]
% where $\ckcO_e$ and $\ckcO_o$ are partitions consist of even and odd rows
% respectively.

% Every irreducible representation in $\Irr(W)$ is special. For the
% infinitesimal character $\lamck$,
% \[

% \]

% In all the cases, let $\DDD$ denote the dual descent of Young diagrams.
% Suppose $\ckcO$ is a Young diagram, it delete the longest row in $\ckcO$


% Let $\YD$ be the set of Young diagrams viewed as a finite multiset of positive
% integers. The set of nilpotent orbits in $\GL_n(\bC)$ is identified with Young
% diagram of $n$ boxes.


% Let $n_e = \abs{\ckcO_e}, n_o =
% \abs{\ckcO_o}$. % and $\lambda_\ckcO = \half \ckhh$.
% By the formula of $a$-function, one can easily see that The cell in
% $W(\lamck)$ consists of the unique representation
% $J_{W_{\lamck}}^{\Wint{\lamck}} (1)$. Now the $W$-cell
% $(J_{W_{\lamck}}^{W_{[\lamck]}} \sgn)\otimes \sgn$ consists a single
% representation
% \[
%   \tau_{\ckcO} = \ckcO_{e}^{t}\boxtimes \ckcO_{o}^{t}.
% \]
% The representation $j_{W_{\lamck}}^{S_{n}} \tau_{\ckcO}$ corresponds to the
% orbit $\cO= \ckcO^t $ under the Springer correspondence.

%In this subsection, we assume that  $\star=A^\R$ so that $G = \GL_{n}(\bR)$.

%We define the set of painted partitions of type $A$ as the following:
%\begin{equation*}%\label{eq:PP.AR}
%  \PP_{A}(\ckcO) = \Set{\uptau:=(\tau, \cP)|
%    \begin{array}{l}
%      \text{$\tau = \ckcO^{t}$}\\
%      \text{$\Im(\cP)\subseteq \set{\bullet,c,d}$}\\
%      \text{$\#\set{i|\cP(i,j)=\bullet}$ is even for all $j\in \bN^{+}$}
%    \end{array}
%  }.
%\end{equation*}


If $\star=A^\R$, for every
$\uptau\in \PP_{\star}(\ckcO)$ we attach the representation
\begin{equation}\label{eq:u.GLR}
  \pi_\uptau :=
  \bigtimes_{j} \underbrace{1_j \times \cdots \times 1_j}_{c_j\text{-terms}}\times
  \underbrace{\sgn_j \times \cdots \times {\sgn_j} }_{d_j\text{-terms}},
\end{equation}
where
\begin{itemize}
  \item $j$ runs over all nonzero row lengths in $\ckcO$,
  \item $1_j$ and $\sgn_j$ respectively denote the trivial and nontrivial quadratic characters on $\GL_j(\bR)$,
  \item $d_j$ is the number of columns of length $j$ ending with the symbol
        ``$d$'' in the painting $\uptau$,
  \item $c_j$ is the number of columns of length $j$ ending with the symbol
        ``$\bullet$'' or ``$c$'' in the painting $\uptau$, and
  \item ``$\times$'' denotes the normalized smooth parabolic induction.
\end{itemize}
If $\star=A^\bH$, for every
$\uptau\in \PP_{\star}(\ckcO)$ we attache the representation
\begin{equation}\label{eq:u.GLR}
  \pi_\uptau :=
 1_{\bfrr_1(\ckcO)}\times 1_{\bfrr_2(\ckcO)} \times \cdots
   \times  1_{\bfrr_k(\ckcO)},
\end{equation}
where
\begin{itemize}
  \item$k$ is the number of nonempty rows of $\ckcO$
\item $1_j$ denotes the trivial representation of $\GL_{\frac{j}{2}}(\bH)$ for every $j\in 2\bN^+$, and
  \item ``$\times$'' denotes the normalized smooth parabolic induction.
\end{itemize}
Then in both cases $\pi_{\uptau}$ is irreducible and belongs to $\Unip_{\ckcO}(G)$ (see  \cite[Theorem 3.8]{V.GL} and \cite[Example~27.5]{ABV}).

We summarize the classifications by the following theorem. 

\begin{thm}%(\cf \cite[Example 27.5]{ABV} and  \cite[Theorem 3.8]{V.GL})
 \label{thm:mainR00}
Suppose that  $\star\in \{A^\R, A^\bH\}$. Then the map
  \[
    \begin{array}{ccc}
      \PP_{\star}(\ckcO) & \rightarrow & \Unip_{\ckcO}(G),\\
      \uptau & \mapsto & \pi_{\uptau}
    \end{array}
  \]
  is bijective. \qed
\end{thm}

\subsection{Unitary groups}
%In this subsection, we  Then $\ckG = \GL_{2n}(\bC)$.


Note that if $\star\in \{A^\R, A^\bH, C, \wtC, D^*\}$, then $n$ equals the rank of $\g$. In all other cases, we also let $n$ denote the rank of $\g$.
We call an integer to have the good parity (depends on $\star$ and $n$) if it has the same parity as
\[
  \begin{cases}
    n, &  \text{if $\star \in \{A^\R, A^\BH,  A\}$}; \\
    1+ n, &  \text{if $\star = \wtA$}; \\
   1, & \text{if } \star \in \set{C,C^{*},D,D^{*}};\\
 \text{0}, & \text{if } \star \in \set{B,\wtC}.\\
  \end{cases}
\]
Otherwise we call the integer to have the bad parity.

Similar to \Cref{defpbp0}, we make the following definition.

\begin{defn}\label{defpbp1}
  Suppose that $\star\in \{A, \widetilde A\}$. A painting $\CP$ on a Young diagram
  $\imath$ has type $\star$ if
  \begin{itemize}
    \item the image of $\CP$ is contained in
                  $   \{\bullet, s, r\}$, and
                    \item $\CP^{-1}(\bullet)$ has even number of boxes in
          every row of $\imath$.
  \end{itemize}
  Denote by $\PAP_\star(\imath)$ the set of paintings on $\imath^{t}$ that has type $\star$, where $\imath^{t}$
  is the transpose of $\imath$.
   \end{defn}


Now suppose that $\imath$ is a Young diagram and $\CP$ is a painting on $\imath$
that has type $A$ or $\widetilde A$. Define the signature of $\CP$ to be the pair
\begin{equation}\label{eq:signature}
    (p_\CP, q_\cP): = \left (\frac{\sharp(\cP^{-1}(\bullet))}{2}+\sharp(\cP^{-1}(r)),\,
    \frac{ \sharp(\cP^{-1}(\bullet))}{2}+\sharp(\cP^{-1}(s))\right).
\end{equation}
\trivial[h]{ The first equation is the true definition of signature. The second
  one is an easy consequence of the definition of $\AC_\cP$. }

\begin{eg}
  Suppose
  that \[ \check \CO=\ytb{\ \ \ \ \ , \ \ \ , \ , \ , \ }\quad \textrm{and}\quad \CP=\ytb{\bullet\bullet\bullet\bullet r,\bullet\bullet , sr,s,r}\in \mathrm{PAP}_{A}(\check \CO) .
  \]
  Then $(p_\CP, q_\cP)=(6,5)$.

\end{eg}

% Given two Young diagrams $\imath$ and $\jmath$, write $\imath\cuprow \jmath$ for
% the Young diagram whose multiset of nonzero row lengths equals the union of
% those of $\imath$ and $\jmath$. Also write $2\imath =\imath\cuprow \imath$.



Given two Young diagrams $\imath$ and $\jmath$, write $\imath\cuprow \jmath$ for
the Young diagram whose multiset of nonzero row lengths equals the union of
those of $\imath$ and $\jmath$. Also write $2\imath =\imath\cuprow \imath$.
Similarly, we write $\imath\cupcol \jmath$ for
the Young diagram whose multiset of nonzero column lengths equals the union of
those of $\imath$ and $\jmath$.



For unitary groups, we have the following counting result.
\begin{thm}\label{thmu1}
  Suppose that $\star\in \{A, \widetilde A\}$. If there is a Young diagram decomposition
  \be\label{deccou}
    \ckcO=\ckcOg \cuprow 2\ckcOpb
  \ee
 such that all nonzero row lengths of   $\ckcOg$ have good parity and all nonzero row lengths of $\ckcOpb$ have bad parity,   then
  \[
    \sharp(\Unip_{\ckcO}(G))= \sharp \set{\CP\in \mathrm{PAP}_\star(\ckcOg)|(p_\CP+\abs{\ckcOpb}, q_\CP+\abs{\ckcOpb})=(p,q)}.
  \]
 Otherwise $\sharp(\Unip_{\check \CO}(G))=0$.

\end{thm}


In particular, when $\star=\widetilde A$ and $p+q$ is odd, the set $\Unip_{\check \CO}(\widetilde \oU(p,q))$ is empty.


Suppose that $\star\in \{A, \widetilde A\}$ so that $G=\oU(p,q)$ or $\widetilde \oU(p,q)$. By Theorem \ref{thmu1},
the set $\Unip_{\ckcO}(G)$ is empty unless there is a decomposition as in \eqref{deccou} such that
\be\label{decupq}
  p,q \geq \abs{\ckcOb'}.
\ee
Assume this holds, and define two groups
$G'_{\mathrm b}:=\GL_{\abs{\check \CO'_{\mathrm b}}}(\C)$ and
\[
  G_{\mathrm g} =
  \begin{cases}
    \rU(p-\abs{\check \CO'_{\mathrm b}},q-\abs{\check \CO'_{\mathrm b}}),  & \text{if }\star = A;\\
    \tU(p-\abs{\check \CO'_{\mathrm b}},q-\abs{\check \CO'_{\mathrm b}}),  & \text{if }\star = \wtA.
\end{cases}
\]
Then up to conjugation there is a unique parabolic subgroup $P$ of $G$ whose Levi quotient (namely its quotient by the unipotent radical) is naturally isomorphic to   $G'_{\mathrm b}\times G_{\mathrm g}$.
  Note that $\ckcO_{\mathrm g}$ has good parity with respect to $\star$ and $\abs{\ckcO_{\mathrm g}}$.

  Let $\pi_{\check \CO'_{\mathrm b}}$ denote the
unique element in $\Unip_{\check \CO'_{\mathrm b}}(G'_{\mathrm b})$ (see Section \ref{complex} for a review on complex classical groups). Then for
every $\pi_\mathrm g\in \Unip_{\ckcO_{\mathrm g}}(\Gg)$, the normalized smooth parabolic
induced representation (from $P$ to $G$) $\pi_{\ckcOpb}  \ltimes \pi_\mathrm g$ is irreducible by
\cite{Mat96}*{Theorem~3.2.2} and is an element of $\Unip_{\ckcO}(G)$ (\cf
\cite{MR.U}*{Theorem~5.3}).



\begin{thm}\label{thmu2}
 With the assumptions in \eqref{deccou} and \eqref{decupq}, and notation as above, the parabolic induction map
  \begin{equation}
  \label{bij00}
    \begin{array}{rcl}
      \mathrm{Unip}_{\check{\CO}_\mathrm g}(G_\mathrm g)&\longrightarrow &\mathrm{Unip}_{\ckcO}(G),\\
      \pi_{\mathrm g}& \mapsto & \pi_\mathrm g\ltimes \pi_{\ckcOpb} \\
    \end{array}
  \end{equation}
  is bijective.
\end{thm}


See Section \ref{secunit} for additional results on special unipotent representations of unitary groups.


\subsection{Orthogonal and symplectic groups: relevant parabolic subgroups}
\label{secrgp0}
In this subsection and the next two subsections,  we assume that
$\star\in \set{ B, C, \wtC,C^*,D, D^*}$.
Then there are Young diagram decompositions
\be\label{decdo}
 \mathbf d_\ckcO=\mathbf d_\mathrm b\cuprow \mathbf d_\mathrm g\quad\textrm{and}\quad \mathbf d_\mathrm b=2 \mathbf d_\mathrm b'
\ee
 such that $\mathbf d_\mathrm b$ has bad parity in the sense that all its nonzero row
lengths have bad parity, and $\mathbf d_\mathrm b$ has good parity in the sense that all its
nonzero row lengths have good parity.
Put
\be\label{nb000}
  n_\mathrm b:=\abs{\mathbf d_\mathrm b'}. %\quad\textrm{and}\quad n_\mathrm g:=n-n_\mathrm b.
\ee

Write $V$ for the standard module of $ \g$, which is either a complex symmetric bilinear space or a complex symplectic space. Recall that  $\CO\subset \g^*$ is the Barbasch-Vogan dual of $\check \CO$. We say that a parabolic subalgebra
 $\p$ of $\g$ is $\check \CO$-relevant if the following conditions are satisfied:
\begin{itemize}
\item
it is the stabilizer of a totally isotropic subspace $V'_{\mathrm b}$ of $V$ of dimension $n_\mathrm b$;
\item
the orbit $\CO$ equals the induction from $\p$ to $\g$ of an orbit in $\overline{\Nil}(\l^*)$, where $\l$ is the Levi quotient of $\p$.
%\item
%if $\star\in\{D,D^*\}$ and $\check \CO$ has bad parity, then
%the number of positive entries in $\lambda_{\check \CO}$ has the same parity as $ \frac{l}{2}$.
%Here we have used the identifications
%\[
 %   \hha=\b/[\b,\b]=\prod_{i=1}^l \gl(V_i/V_{i-1}) =\C^l
%\]
%so that $\lambda_{\check \CO}\in (\frac{1}{2}, \frac{1}{2}, \dots, \frac{1}{2})+\Z^l\subset \C^l=\hha^*$,
%where $0=V_0\subset V_1\subset \cdots V_{l-1} \subset V_l$ is a full flag in $V_l$, and $\b$ is the Borel subalgebra of $\g$ stabilizing this flag.
\end{itemize}
Note that the first condition implies the second one, except for the case when $\star\in \{D, D^*\}$ and $\check \CO$  is nonempty and has bad parity. In all cases,   up to conjugation by $\Ad(\g)$ there exists a unique  parabolic subalgebra of $\g$ that is $\check \CO$-relevant.

We say that a parabolic subgroup of $G$ is $\check \CO$-relevant if its complexified Lie algebra is. If such a parabolic subgroup exists, it is unique up to conjugation by $G$. In this case, the orbit $\check \CO$ is said to be $G$-relevant. 


If $\star=D^*$ and  $\check \CO$  is nonempty and has bad parity, then there are precisely two orbits in $\overline \Nil(\check \g)$ that has the same Young diagram as that of $\check \CO$. Between these two orbits, exactly one is $G$-relevant. Excluding this special case, $\check \CO$ is $G$-relevant if and only if
\be\label{existgl}
  \textrm{either $\star\in \{B,D, C^*\}$ and $p,q\geq l,\quad $ or  $\quad \star\in \{C, \wtC, D^*\}$}.
\ee

%On the other hand, if \eqref{existgl} holds, then up to conjugation there exists a unique $\check \CO$-relevant parabolic subgroup of $G$, except when $\star=D^*$ and $\check \CO$ is incompatible.

When \eqref{existgl} holds,  we put  \be\label{gg00}
  \Gg :=
  \begin{cases}
    \SO(p-n_\mathrm b,q-n_\mathrm b), & \textrm{if $\star\in \set{B,D}$};\\
  %  \SO_{n-2l}(\bC) &\textrm{if $\star\in \set{B^{\bC},D^{\bC}}$},\\
    \rO^{*}(2n-2n_\mathrm b), &\textrm{if $\star = D^{*}$};\\
    \Sp_{2n-2n_\mathrm b}(\bR), &\textrm{if $\star = C$};\\
    \wtSp_{2n-2n_\mathrm b}(\bR), &\textrm{if $\star = \wtC$};\\
  %  \Sp_{2n-2l}(\bC) &\textrm{if $\star \in \set{C^{\bC},\wtC^{\bC}}$},\\
    \Sp(\frac{p-n_\mathrm b}{2},\frac{q-n_\mathrm b}{2}), &\textrm{if $\star = C^{*}$}.\\
  \end{cases}
\ee
Then the Levi quotient of  every $\check \CO$-relevant parabolic subgroup  of $G$ is naturally isomorphic to   $\Gpb\times \Gg $ (or   $(\Gpb\times \Gg)/\{\pm 1\} $ when $\star=\wtC$), where
\begin{equation}\label{Gpb}
  \Gpb := \begin{cases}
    \GL_{n_\mathrm b}(\bR), & \text{if } \star \in \set{B,C,D}; \\
       \widetilde{ \GL}_{n_\mathrm b}(\bR), & \text{if } \star =\wtC; \\
    \GL_{\frac{n_\mathrm b}{2}}(\bH), & \text{if } \star \in \set{C^{*},D^{*}}. \\
  \end{cases}
\end{equation}
Here $ \widetilde{ \GL}_{n_\mathrm b}(\bR)$ is the double cover of $ \GL_{n_\mathrm b}(\bR)$ that fits the following Cartesian diagram of Lie groups:
\begin{equation}\label{wgll}
\begin{CD}
 \widetilde{ \GL}_{n_\mathrm b}(\bR)@>>>  \GL_{n_\mathrm b}(\bR)\\
  @VVV @VV g\mapsto \textrm{ sign of $\det(g)$} V\\
  \{\pm 1, \pm \sqrt{-1}\} @> x\mapsto x^2 >> \{\pm 1\}. \\
\end{CD}
\end{equation}

Let $\check \g'_\mathrm b$ denote the Langlands dual of $\g'_\mathrm b$, and let $\check \CO'_\mathrm b\in \overline{\Nil}(\check \g'_\mathrm b)$ denote the nilpotent orbit with Young diagram $\mathbf d'_\mathrm b$. Likewise let $\check \g_\mathrm g$ denote the Langlands dual (or the metaplectic Langlands dual when $\star=\wtC$) of $\g_\mathrm g$, and let $\check \CO_\mathrm g\in \overline{\Nil}(\check \g_\mathrm g)$ denote the (unique) nilpotent orbit with Young diagram $\mathbf d_\mathrm g$.

%Note that $\check \g_\mathrm g$ is the Langlands dual (or the metaplectic Langlands dual when $\star=\wtC$) of $\g_\mathrm g$.

 \subsection{Orthogonal and symplectic groups: reduction to good parity}
\label{secrgp0}
Define
 \[
      \Unip_{\ckcOpb}(\widetilde{ \GL}_{n_\mathrm b}(\bR)):=
       \{\pi\in \Irr(\widetilde{ \GL}_{n_\mathrm b}(\bR))\mid \pi \textrm{ is genuine  and annihilated by } I_{\ckcOpb}:= I_{A^\R, \ckcOpb}\}.
       \]
        Here and as before, ``genuine" means that the central subgroup $\{\pm 1\}$ acts through the nontrivial character.  Then we have a bijective map
 \[
    \Unip_{\ckcOpb}(\GL_{n_\mathrm b}(\bR))\rightarrow  \Unip_{\ckcOpb}(\widetilde{ \GL}_{n_\mathrm b}(\bR)), \quad \pi\mapsto \pi\otimes \tilde \chi_{n_\mathrm b},
 \]
 where $\tilde \chi_{n_\mathrm b}$ is the character given by the left vertical arrow of \eqref{wgll}.
\begin{thm}\label{reduction}
 If  $G$ has a $\check \CO$-relevant parabolic subgroup $P$, then the normalized smooth parabolic induction from $P$ to $G$ yields
   a bijection
   \[
 % \begin{equation}\label{eq:IND}
    \begin{array}{rccc}
 &\Unip_{\ckcO'_{\mathrm b}}( G'_{\mathrm b}) \times   \Unip_{\ckcO_{\mathrm g}}( G_{\mathrm g})  &         \longrightarrow &\Unip_{\ckcO }(G), \\
                &   (\pi',\pi_\mathrm g) & \mapsto & \pi'\ltimes \pi_\mathrm g.
    \end{array}
 % \end{equation}
 \]
  Otherwise,
  \[
    \Unip_{\ckcO}(G)=\emptyset.
  \]
\end{thm}


By \Cref{reduction}, we have the more specific results on counting as follows:
\begin{enumerate}[label=(\alph*)]
  \item Assume that $\star\in \{B,D\}$ so that $G=\SO(p,q)$. Then
        \[
        \sharp(\Unip_{\check \CO}(G))=
        \begin{cases}
          \sharp(\Unip_{\check \CO_{\mathrm g}}(G_{\mathrm g}))\times \sharp(\Unip_{\check \CO'_{\mathrm b}}(\GL_{n_\mathrm b}(\R)) ), &\hbox{if $p,q\geq n_\mathrm b$}; \smallskip\\
          0, &\hbox{otherwise.}
        \end{cases}
        \]
  \item Assume that $\star=C^*$ so that $G=\Sp(\frac{p}{2},\frac{q}{2})$. Then
        \[
        \sharp(\Unip_{\check \CO}(G))=
        \begin{cases}
          \sharp(\Unip_{\check \CO_{\mathrm g}}(G_{\mathrm g} )), &\hbox{if $p,q\geq n_\mathrm b$}; \smallskip\\
          0, &\hbox{otherwise.}
        \end{cases}
        \]

  \item Assume that $\star\in \{C,\widetilde C\}$ so that $G=\Sp_{2n}(\R)$ or
        $\widetilde \Sp_{2n}(\R)$. Then
        \[
        \sharp(\Unip_{\check \CO}(G))= \sharp(\Unip_{\check \CO_{\mathrm g}}(G_{\mathrm g}))\times \sharp(\Unip_{\check \CO'_{\mathrm b}}(\GL_{n_\mathrm b}(\R)) ). \]
  \item Assume that $\star =D^*$ so that $G=\oO^*(2n)$. Then
        \[
          \sharp(\Unip_{\check \CO}(G))=
          \begin{cases}
          0,&\quad \textrm{if $\check \CO$  is not $G$-relevant;}\\
          \sharp(\Unip_{\check \CO_{\mathrm g}}(G_{\mathrm g})),&\quad \textrm{otherwise}.
          \end{cases}
        \]
\end{enumerate}


 \subsection{Orthogonal and symplectic groups: the case of good parity}\label{secorgp0}
 We now assume that $\check \CO$ has good parity, namely
 $\check \CO=\check \CO_{\mathrm g}$. By Theorem \ref{reduction}, the counting
 problem in general is reduced to this case.



 \delete{
   \begin{defn}
     A $\star$-pair is a pair $(i,i+1)$ of consecutive positive integers such
     that
     \[
       \left\{
         \begin{array}{ll}
           i\textrm{ is odd}, \quad &\textrm{if $\star\in\{C, \widetilde{C}, C^*, C^\C, \widetilde C^\C\}$};  \\
           i \textrm{ is even}, \quad &\textrm{if $\star\in\{B, D, D^*, B^\C, D^\C\}$}. \\
         \end{array}
       \right.
     \]
     A $\star$-pair $(i,i+1)$ is said to be primitive in $\check \CO$ if
     $\mathbf r_i(\check \CO)-\mathbf r_{i+1}(\check \CO)$ is positive and even.
     Denote $\mathrm{PP}_\star(\check \CO)$ the set of all $\star$-pairs that
     are primitive in $\check \CO$.
   \end{defn}
 }



\begin{defn}\label{defn:PP}
  A $\star$-pair is a pair $(i,i+1)$ of consecutive positive integers such that
  \[
    \left\{
      \begin{array}{ll}
        i\textrm{ is odd}, \quad &\textrm{if $\star\in\{C, \widetilde{C}, C^*\}$};  \\
        i \textrm{ is even}, \quad &\textrm{if $\star\in\{B, D, D^*\}$}. \\
      \end{array}
    \right.
  \]
  A $\star$-pair $(i,i+1)$ is said to be
  \begin{itemize}
    \item vacant in $\check \CO$, if
          $\mathbf r_i(\check \CO)=\mathbf r_{i+1}(\check \CO)=0$;
    \item balanced in $\check \CO$, if
          $\mathbf r_i(\check \CO)=\mathbf r_{i+1}(\check \CO)>0$;
    \item tailed in $\check \CO$, if
          $\mathbf r_i(\check \CO)-\mathbf r_{i+1}(\check \CO)$ is positive and
          odd;
    \item primitive in $\check \CO$, if
          $\mathbf r_i(\check \CO)-\mathbf r_{i+1}(\check \CO)$ is positive and
          even.
  \end{itemize}
  Denote $\CPP_\star(\check \CO)$ the set of all $\star$-pairs that are
  primitive in $\check \CO$.
\end{defn}

%We continue with the counting problem of $\Unip_{\check \CO}(G)$, when $\check \CO$ has $\star$-good parity.

We attach to $\check \CO$ a pair of Young diagrams
\be\label{ijo}
  (\imath_{\check \CO}, \jmath_{\check \CO}):=(\imath_\star(\check \CO), \jmath_\star(\check \CO)),
\ee
as follows.

\medskip

\noindent {\bf The case when $\star=B$.} In this case,
\[
  \mathbf c_{1}(\jmath_{\check \CO})=\frac{\mathbf r_1(\check \CO)}{2},
\]
and for all $i\geq 1$,
\[
  \left (\mathbf c_{i}(\imath_{\check \CO}), \mathbf c_{i+1}(\jmath_{\check \CO})\right )= \left (\frac{\mathbf r_{2i}(\check \CO)}{2}, \frac{\mathbf r_{2i+1}(\check \CO)}{2}\right ).
\]

\medskip

\noindent {\bf The case when $\star=\widetilde C$.} In
this case, for all $i\geq 1$,
\[
  (\mathbf c_{i}(\imath_{\check \CO}), \mathbf c_{i}(\jmath_{\check \CO}))= \left (\frac{\mathbf r_{2i-1}(\check \CO)}{2}, \frac{\mathbf r_{2i}(\check \CO)}{2}\right).
\]

\medskip

\noindent {\bf The case when $\star=\{ C,C^*\}$.} In this case, for all
$i\geq 1$,
\[
  (\mathbf c_{i}(\jmath_{\check \CO}), \mathbf c_{i}(\imath_{\check \CO}))= \left\{
    \begin{array}{ll}
      (0,  0), &\hbox{if $(2i-1, 2i)$ is vacant  in $\check \CO$};\smallskip\\
      (\frac{\mathbf r_{2i-1}(\check \CO)-1}{2},  0), & \hbox{if $(2i-1, 2i)$ is tailed in $\check \CO$};\smallskip\\
      (\frac{\mathbf r_{2i-1}(\check \CO)-1}{2},  \frac{\mathbf r_{2i}(\check \CO)+1}{2}), &\hbox{otherwise}.\\
    \end{array}
  \right.
\]
\medskip

\noindent {\bf The case when $\star\in \{D,D^*\}$.} In this case,
\[
  \mathbf c_{1}(\imath_{\check \CO})= \left\{
    \begin{array}{ll}
      0,  &\hbox{if $\mathbf r_1(\check \CO)=0$}; \smallskip\\
      \frac{\mathbf r_1(\check \CO)+1}{2},   &\hbox{if $\mathbf r_1(\check \CO)>0$},\\
    \end{array}
  \right.
\]
and for all $i\geq 1$,
\[
  (\mathbf c_{i}(\jmath_{\check \CO}), \mathbf c_{i+1}(\imath_{\check \CO}))= \left\{
    \begin{array}{ll}
      (0,  0), &\hbox{if $(2i, 2i+1)$ is vacant in $\check \CO$};\smallskip\\
      \left  (\frac{\mathbf r_{2i}(\check \CO)-1}{2},  0\right ), & \hbox{if $(2i, 2i+1)$ is tailed in $\check \CO$};\smallskip\\
      \left  (\frac{\mathbf r_{2i}(\check \CO)-1}{2},  \frac{\mathbf r_{2i+1}(\check \CO)+1}{2}\right ), &\hbox{otherwise}.\\
    \end{array}
  \right.
\]




\begin{eg} Suppose that $\star=C$, and $\check \CO$ is the following Young
  diagram which has $\star$-good parity.
  \begin{equation*}%\label{eq:sp-nsp.C}
    \tytb{\ \ \ \ \  , \ \ \  , \ \ \ , \ \ \  , \ \ \ , \  ,\  }
  \end{equation*}
  Then
  \[
    \CPP_\star(\check \CO)=\{(1,2), (5,6)\}
  \]
  and
  \[
    (\imath_{\check \CO}, \jmath_{\check \CO})= \tytb{\ \ \ ,\ \ } \times \tytb{\ \ \ , \ }.
  \]


\end{eg}



\delete{
  \begin{eg} Suppose that $\star=C$, and $\check \CO$ is the following Young
    diagram which has $\star$-good parity.
    \begin{equation*}%\label{eq:sp-nsp.C}
      \tytb{\ \ \ \ \  , \ \ \  , \ \ \ , \ \ \  , \ \ \ , \  ,\  }
    \end{equation*}
    Then
    \[
      \mathrm{PP}_\star(\check \CO)=\{(1,2), (5,6)\}.
    \]
    and $(\imath_\star(\check \CO, \wp), \jmath_\star(\check \CO,
    \wp))$ %\in \mathrm{BP}_\star(\check \CO)$
    has the following form.

    \begin{equation*}%\label{eq:sp-nsp.C}
      \begin{array}{rclcrcl}
        \wp=\emptyset & : & \tytb{\ \ \ ,\ \  } \times \tytb{\ \ \ , \  }  & \qquad \quad &  \wp=\{(1,2)\}& : & \tytb{\ \ \  , \ \ , \   } \times \tytb{\ \ \  } \medskip \medskip \medskip \\
        \wp=\{(5,6)\} & : & \tytb{\ \ \ ,\ \ \ } \times \tytb{\ \ , \   }  & \qquad \quad &  \wp=\{(1,2), (5,6)\}  & : & \tytb{\ \ \  , \ \ \ ,  \ } \times \tytb{\ \   } \\
      \end{array}
    \end{equation*}

\end{eg}
}

Here and henceforth, when no confusion is possible, we write
$\alpha\times \beta$ for a pair $(\alpha, \beta)$. We will also write
$\alpha\times \beta\times \gamma$ for a triple $(\alpha, \beta, \gamma)$.


We introduce two more symbols $B^+$ and $B^-$, and make the following
definition.
\begin{defn} \label{def:pbp1}
  A painted bipartition is a triple
  $\tau=(\imath, \CP)\times (\jmath, \cQ)\times \alpha$, where $(\imath, \CP)$
  and $ (\jmath, \mathcal Q)$ are painted Young diagrams, and
  $\alpha\in \{B^+,B^-, C,D,\widetilde {C}, C^*, D^*\}$, subject to the
  following conditions:
  \begin{itemize}
          \delete{\item $(\imath, \jmath)\in \mathrm{BP}_\alpha$ if
          $\alpha\notin\{B^+,B^-\}$, and $(\imath, \jmath)\in \mathrm{BP}_{B}$
          if $\alpha\in\{B^+,B^-\}$;}

    \item $\CP^{-1}(\bullet)=\mathcal Q^{-1}(\bullet)$;
    \item the image of $\CP$ is contained in
          \[
          \left\{
          \begin{array}{ll}
            \{\bullet, c\}, &\hbox{if $\alpha=B^+$ or $B^-$}; \smallskip\\
            \{\bullet,  r, c,d\}, &\hbox{if $\alpha=C$}; \smallskip\\
            \{\bullet, s, r, c,d\}, &\hbox{if $\alpha=D$}; \smallskip\\
            \{\bullet, s, c\}, &\hbox{if $\alpha=\widetilde{ C}$}; \smallskip \\
            \{\bullet\}, &\hbox{if $\alpha=C^*$}; \smallskip \\
            \{\bullet, s\}, &\hbox{if $\alpha=D^*$},\\
          \end{array}
          \right.
          \]
    \item the image of $\mathcal Q$ is contained in
          \[
          \left\{
          \begin{array}{ll}
            \{\bullet, s, r, d\}, &\hbox{if $\alpha=B^+$ or $B^-$}; \smallskip\\
            \{\bullet, s\}, &\hbox{if $\alpha=C$}; \smallskip\\
            \{\bullet\}, &\hbox{if $\alpha=D$}; \smallskip\\
            \{\bullet, r, d\}, &\hbox{if $\alpha=\widetilde{ C}$}; \smallskip\\
            \{\bullet, s,r\}, &\hbox{if $\alpha=C^*$}; \smallskip \\
            \{\bullet, r\}, &\hbox{if $\alpha=D^*$}.
          \end{array}
          \right.
          \]

  \end{itemize}
\end{defn}

% \begin{remark}
%   The set of painted bipartition counts the multiplicities of an irreducible
%   representation of $W_{r_{\fgg}}$ occurs in the coherent continuation
%   representation at the infinitesimal character of the trivial representation.
%   For the relationship between painted bipartitions and the coherent
%   continuation representations of Harish-Chandra modules, see \cite{Mc}.
% \end{remark}

For any painted bipartition $\tau$ as in Definition \ref{def:pbp1}, we write
\[
  \imath_\tau:=\imath,\ \cP_\tau:=\cP,\ \jmath_\tau:=\jmath,\ \cQ_\tau:=\cQ,\ \alpha_\tau:=\alpha,\ \abs{\tau}:=\abs{\imath}+\abs{\jmath},
\]
and
\[
  \star_\tau:= \left\{
    \begin{array}{ll}
      B, &\hbox{if $\alpha=B^+$ or $B^-$}; \smallskip\\
      \alpha, & \hbox{otherwise}.           \end{array}
  \right.
\]
% Its leading column is then defined to be the first column of $(\jmath, \CQ)$
% when $\star_\tau\in \{B, C,C^*\}$, and the first column of $(\imath, \CP)$
% when $\star_\tau\in \{\widetilde C, D, D^*\}$.

We further define a pair $(p_{\tau}, q_{\tau})$ of natural numbers given by the
following recipe.
\begin{itemize}
  \item If $\star_\tau\in \{B, D, C^*\}$, then $(p_\tau, q_\tau)$ is given by
        counting the various symbols appearing in $(\imath, \CP)$,
        $(\jmath, \cQ)$ and $\{\alpha\}$ :
        \begin{equation*}%\label{ptqt}
          \left\{
            \begin{array}{l}
              p_\tau :=( \# \bullet)+ 2 (\# r) +(\# c )+ (\# d) + (\# B^+);\smallskip\\
              q_\tau :=( \# \bullet)+ 2 (\# s) + (\# c) + (\# d) + (\# B^-).\\
            \end{array}
          \right.
        \end{equation*}
        Here
        \[
        \#\bullet:=\#(\cP^{-1}(\bullet))+\#(\cQ^{-1}(\bullet))
        %\qquad (\textrm{$\#$        indicates the cardinality of a finite set}),
        \]
        and the other terms are similarly defined.
  \item If $\star_\tau\in \{C, \widetilde C, D^*\}$, then
        $p_\tau:=q_\tau:=\abs{\tau}$.
\end{itemize}
\smallskip

We also define a classical group
\begin{equation*}%\label{def:Gt}
  G_\tau:=
  \begin{cases}
    \SO(p_\tau, q_\tau), &\hbox{if $\star_\tau=B$ or $D$}; \smallskip\\
    \Sp_{2\abs{\tau}}(\R), &\hbox{if $\star_\tau=C$}; \smallskip\\
    \widetilde{\Sp}_{2\abs{\tau}}(\R), &\hbox{if $\star_\tau=\widetilde{ C}$}; \smallskip \\
    \Sp(\frac{p_\tau}{2}, \frac{q_\tau}{2}), &\hbox{if $\star_\tau=C^*$}; \smallskip \\
    \oO^*(2\abs{\tau}), &\hbox{if $\star_\tau=D^*$}.\\
  \end{cases}
\end{equation*}


Define
\begin{equation}\label{defpbp2222}
  \PBP_\star(\check \CO) :=\set{ \uptau\textrm{ is a painted
      bipartition} \mid \star_\uptau = \star, \text{ and
    } (\imath_\tau,\jmath_\tau) = (\imath_{\check \CO}, \jmath_{\check \CO})},
\end{equation}
and
\begin{equation*} %\label{defpbp3}
    \PBP_{G}(\ckcO) :=\set{\uptau\in \PBP_{\star}(\ckcO)| G_{\uptau} = G}.
\end{equation*}

\delete{
  \[
    \begin{array}{rl}
      \mathrm{PBP}_\star(\check \CO):=\{ &
                                           \tau\textrm{ is a painted bipartition}  \mid    \star_\tau = \star,
                                           \text{ and } \\  & (\imath_\tau,\jmath_\tau) = (\imath_{\check \CO}, \jmath_{\check \CO})\}.
    \end{array}
  \]
}


\begin{eg} Suppose that $\star=B$ and
  \[
    \check \CO =\tytb{\ \ \ \ \ \ , \ \ \ \ \ \ , \ \ , \ \ , \ \ }
  \]
  Then
  \[
    \tau:= \tytb{\bullet \bullet ,\bullet , c } \times \tytb{\bullet \bullet d ,\bullet , d }\times B^+\in \mathrm{PBP}_{\star}(\check \CO),
  \]
  and
  \[
    G_\tau=\SO(10,9).
  \]
\end{eg}


We now state our final result on the counting of special unipotent representations.


\begin{thm}\label{countup}
  Assume that $\star\in \{B, C,D,\widetilde {C}, C^*, D^*\}$, and $\check \CO$ has good parity. Then
 \[
   \sharp(\Unip_{\check \CO}(G)) \leq
    \left\{
    \begin{array}{ll}
       \sharp (\PBP_{G}(\ckcO)),  & \hbox{if $\star\in \{C^*,D^*\}$}; \smallskip\\
       2^{\sharp(\CPPs(\check \CO))} \cdot \sharp (\PBP_{G}(\ckcO)),  &\hbox{if $\star\in \{B, C,D,\widetilde {C}\}$}.
    \end{array}
  \right.
  \]
\end{thm}

In \cite{BMSZ2}, the authors construct a set of representations in
$\Unip_{\check \CO}(G)$ whose cardinality equals the upper bound in Theorem \ref{countup}, when $\check \CO$ has $\star$-good parity. See \cite[Theorem 4.1]{BMSZ2}. Thus the equality holds in \Cref{countup}.


\trivial[h]{
\begin{thm}\label{countup}
  Assume that $\star\in \{B, C,D,\widetilde {C}, C^*, D^*\}$, and $\check \CO$ has $\star$-good parity. Then
  \[
    \sharp(\Unip_{\ckcO}(G))\leq 2^{\sharp(\CPP_\star(\check \CO))} \cdot \sharp (\PBP_{\mathrm g}(\ckcO)).
  \]
\end{thm}

In \cite{BMSZ2}, the authors have constructed $2^{\sharp(\CPP_\star(\check \CO))} \cdot \sharp (\PBP_{\mathrm g}(\ckcO))$ number of representations in
$\Unip_{\check \CO}(G)$, when $\check \CO$ has $\star$-good parity. See \cite[Theorem 4.1]{BMSZ2}. Thus the equality holds in \Cref{countup}.

\medskip}

%\begin{remark}

\subsection{The case of complex classical groups}\label{complex}
Special unipotent representations of complex classical groups are well-understood (\cite{BVUni}, \cite{B89}). We briefly review their counting and constructions in what follows. As the methods of this paper and \cite{BMSZ2} work for complex classical groups as well, we will present the results in the complex case parallel to those of this paper and \cite{BMSZ2}. For this subsection, we introduce five more symbols $A^\C, B^\C,D^\C, C^\C$, and $\widetilde C^\C$, and let $\star$ be one of them. Let $G$ be a complex classical group of type $\star$, namely
\[
G=\GL_n(\C),\quad \SO_{2n+1}(\C),\quad \SO_{2n}(\C),\quad \Sp_{2n}(\C), \quad \textrm{or}\quad \Sp_{2n}(\C)\qquad (n\in \BN),
\]
respectively.
Let $\g_0$ denote the Lie algebra of $G$, which is  a complex Lie algebra.
The Langlands dual (or metaplectic Langlands dual when $\star=\widetilde C^\C$) $\check \g_0$ of $\g_0$ is respectively defined to be
\[
\g\l_n(\C),\quad \s\p_{2n}(\C), \quad \o_{2n}(\C), \quad \o_{2n+1}(\C), \quad \textrm{or}\quad  \s\p_{2n}(\C).
\]
 Let $\check \CO\in \overline{\Nil}(\check \g_0)$. As in the real case we have a maximal ideal $I_{\check \CO}:=I_{\star, \check \CO}$ of $\CU(\g_0)$.

 Write $\overline \g_0$ for the complex Lie algebra equipped with a conjugate linear isomorphism $\bar{\phantom a} :\g_0\rightarrow \overline{\g_0}$. The latter induces a  conjugate linear isomorphism $\bar{\phantom a} :\CU(\g_0)\rightarrow \CU( \overline{\g_0})$. Note that $\g_0\times \overline{\g_0}$ equals the complexified Lie algebra $\g$ of $G$.  Define the set of special unipotent representations of $G$
 attached to $\ckcO$ by
 \[
     \Unip_{\ckcO}(G):=  \Unip_{\star, \ckcO}(G)
     :=
       \{\pi\in \Irr(G)\mid \pi \textrm{ is annihilated by } I_{\check \CO}\otimes \CU(\overline{\g_0}) + \CU(\g_0)\otimes \overline{I_{\check \CO}}\, \}.
       \]

 If $\star=A^\C$ so that $G=\GL_n(\C)$, then $\Unip_{\ckcO}(G)$ is a singleton whose unique element is given by the normalized parabolic induction
 $\Ind_{P}^{G} 1_P$, where $P$ is the standard parabolic subgroup whose Levi component equals
 \[
 \GL_{\mathbf r_1(\check \CO)}(\C)\times \GL_{\mathbf r_2(\check \CO)}(\C)\times \dots \times \GL_{\mathbf r_{\mathbf c_1(\check \CO)}(\check \CO)}(\C),
 \]
 and $1_P$ denotes the trivial representation of $P$.
 %See \cite{V.GL}.

Now suppose that $\star\in \{B^\C,D^\C, C^\C, \widetilde C^\C\}$. Write
\[
 \mathbf d_\ckcO=\mathbf d_\mathrm b\cuprow \mathbf d_\mathrm g\quad\textrm{and}\quad \mathbf d_\mathrm b=2 \mathbf d_\mathrm b'
\]
as in \eqref{decdo}, and put $n_\mathrm b:=\abs{\mathbf d_\mathrm b'}$ as before. %\quad\textrm{and}\quad n_\mathrm g:=n-n_\mathrm b.
Let $\p_0$ be a parabolic subalgebra of $\g_0$ that is $\check \CO$-relevant (defined in \Cref{secrgp0}). Let $P$ be the parabolic subgroup of $G$ with Lie algebra $\p_0$. Then the Levi quotient of $P$  is naturally isomorphic to $\Gpb\times \Gg$, where $\Gpb=\GL_{n_\mathrm b}(\C)$ and
\[
  \Gg :=
  \begin{cases}
    \SO_{2n-2n_\mathrm b+1}(\C), & \textrm{if $\star=B^\C$};\\
    \SO_{2n-2n_\mathrm b}(\C), & \textrm{if $\star=D^\C$};\\
    \Sp_{2n-2n_\mathrm b}(\C), &\textrm{if $\star \in\{ C^\C, \widetilde C^\C \}$}.
      \end{cases}
\]

Define the set $\CPP_\star(\ckcOg)$ as in the real case. Then by the work of
Barbasch-Vogan (\cite[Corollary 5.29]{BVUni}, integral case) and Moeglin-Renard
(\cite{MR.C}, general case), we have that
   \[
    \sharp(\Unip_{\check \CO}(G))=\sharp(\Unip_{\ckcOg}(\Gg))=2^{\sharp(\CPP_\star(\ckcOg))}.
  \]
   As in the real case, every representation in $\Unip_{\check \CO}(G)$ is
   obtained through irreducible parabolic induction from $P$ to $G$ via those of
   $ \Unip_{\ckcO'_{\mathrm b}}( G'_{\mathrm b})\times \Unip_{\ckcO_{\mathrm g}}( G_{\mathrm g}) $
   (\cf Theorem \ref{reduction}), and every representation in
   $\Unip_{\ckcOg}(\Gg)$ is obtained via iterated theta lifting (see
   \cite[Theorem 3.5.1]{B17}, \cite{Mo17} and \cite{BMSZ2}).



\section{Preliminaries  on coherent families of highest weight modules}


%\section{Generalities on coherent continuation representations}

Most of the results of this section and the next one are known to experts.% and we will thus be brief.

We retain the notation of Section \ref{secnot}.
Write \[
\Delta^+\subset \Delta\subset \hha^*\quad \textrm{and}\quad \check \Delta^+\subset \check \Delta\subset \hha
\]
for the positive root system, the root system, the positive coroot system, and the coroot system, respectively, for the reductive complex Lie algebra $\g$.
Write $Q_\g$ and $ Q^\g$ for the root lattice and weight group of $\g$, respectively. More precisely, $Q_\g$ is the subgroup of $\hha^*$   spanned by $\Delta$, and
\[
Q^\g:=\{\nu\in \hha^*\mid \langle \nu, \check \alpha\rangle \in \Z\ \textrm{ for all }\check \alpha\in \check \Delta\}.
\]


\subsection{Coherent continuation representations}


 Throughout this article we take $\C$ as the coefficient ring to define  Grothendieck groups. When no confusion is possible, for every object $O$ in an abelian category, we  still use the same symbol to indicate the Grothendieck group element represented by the object $O$.




 Let $\Lambda\subset \hha^*$ be a $Q$-coset, where $Q$ is a $W$-stable subgroup of $Q^\g$ containing $Q_\g$, which is determined by $\Lambda$.
The stabilizer of $\Lambda$ in $W$ is denoted by $W_\Lambda$. Let $\mathrm{Rep}(\g, Q)$ denote the category of all finite-dimensional
representations of $\g$ whose weights (which are obviously defined as elements of $Q^\g\subset \hha^*$) are contained in $Q$. Write $\mathcal R(\g, Q)$ for the Grothendieck group of this category, which is
 a commutative $\bC$-algebra under the tensor
product of representations.

 Given  a $\mathcal R(\g, Q)$-module $\CK$, a  $\Lambda$-family in $\CK$ is a family
   $\{\CK_\nu\}_{\nu\in \Lambda}$ of subspaces of $\CK$ such that
\begin{itemize}
\item
  $\CK_{w \cdot \nu}=\CK_\nu\ $  for all $w\in W_\Lambda$ and $\nu\in \Lambda$;
  \item  for all representations $F$ in  $\mathrm{Rep}(\g, Q)$ and all $\nu\in \Lambda$,
  \[
   F\cdot \CK_\nu\subset \sum_{\mu\in \hha^* \textrm{ is a weight of $F$} } \CK_{\nu+\mu}.
  \]
 \end{itemize}
For example $\{\CK\}_{\nu\in \Lambda}$ is a  $\Lambda$-family in $\CK$, to be called the full $\Lambda$-family.


\begin{defn}\label{defcoh00}
  Let $\CK$ be  a $\mathcal R(\g, Q)$-module  with a $\Lambda$-family $\{\CK_\nu\}_{\nu\in \Lambda}$ in it.
  A $\CK$-valued coherent family on
  $\Lambda$ is a map
  \[
    \Psi: \Lambda\rightarrow \CK%, \qquad \lambda\mapsto \Phi_\lambda
  \]
  satisfying the following two conditions:
  \begin{itemize}
    \item for all $\nu\in \Lambda$, $\Psi(\nu)\in \CK_\nu$;
    \item for all representations $F$ in $\mathrm{Rep}(\g, Q)$
          and all $\nu\in \Lambda$,
          \[
          F \cdot (\Psi(\nu)) = \sum_{\mu} \Psi(\nu+\mu),
          \]
          where $\mu$ runs over all weights of $F$, counted with multiplicities.%  and $F$ is viewed as an element of $\mathcal R(\g)$.
  \end{itemize}
\end{defn}


In the notation of \Cref{defcoh00}, let $\Coh_{\Lambda}(\CK)$ denote the
vector space of all $\mathcal K$-valued coherent families on $\Lambda$. It is a
representation of $W_{\Lambda}$ under the action
\[
  (w \cdot \Psi)(\nu) = \Psi(w^{-1}\cdot \nu), \qquad \textrm{for all
  }\ w\in W_\Lambda, \ \Psi\in \Coh_{\Lambda}(\CK), \   \nu\in \Lambda.
\]
This is called a coherent continuation representation. When specifying a coherent continuation representation $\Coh_{\Lambda}(\CK)$, we will often explicitly describe $\CK$ as a Grothendieck group, while the $\mathcal R(\g, Q)$-module structure and the $\Lambda$-family  $\{\CK_\nu\}_{\nu\in \Lambda}$ are the ones which are clear from the context (which are often specified by the infinitesimal characters).

The assignment $\Coh_{\Lambda}(\CK)$ is functorial in $\CK$ in the following sense: suppose that $\CK'$ is another  $\mathcal R(\g, Q)$-module  with a $\Lambda$-family $\{\CK'_\nu\}_{\nu\in \Lambda}$ in it, and $\eta: \CK\rightarrow \CK'$ is a $\mathcal R(\g, Q)$-homomorphism such that $\eta(\CK_\nu)\subset \CK'_\nu$ for all $\nu\in \Lambda$, then
\[
 \eta_*: \Coh_{\Lambda}(\CK)\rightarrow \Coh_{\Lambda}(\CK'), \quad \Psi\mapsto \eta\circ \Psi
\]
is a well-defined $W_\Lambda$-equivariant linear map.


\newcommand{\Rep}{\mathrm{Rep}}




\subsection{Coherent continuation representations for highest weight modules}



Let $\b$ be a  Borel subalgebra of $\g$, and let $\Rep(\g,\b)$ denote the category of finitely generated $\g$-modules that are unions of finite-dimensional $\b$-submodules.
For each $\nu\in \hha^*$, let  $\Rep_\nu(\g,\b)$ denote the  full subcategory of $\Rep(\g,\b)$ consisting of the modules that have generalized infinitesimal character $\nu$ (by definition, a $\g$-module has  generalized infinitesimal character $\nu$ if and only if every vector in it is annihilated by $(\ker(\chi_\nu))^k$ for some $k\in \BN^+$).
Write $\CK(\g,\b)$ for the Grothendieck groups of $\Rep(\g,\b)$, and similarly define the Grothendieck groups $\CK_{\nu}(\g,\b)$.
The  Grothendieck group $\CK(\g,\b)$ is a $\mathcal R(\g,Q)$-module under the tensor products. Using the $\Lambda$-family   $\{\CK_\nu(\g,\b)\}_{\nu\in \Lambda}$, we form the coherent continuation representation $\Coh_{\Lambda}( \CK(\g,\b))$.

 %With these extra structures, $\CK(\g,\b)$ is  a basal $\mathcal R(\g, Q)$-module  with $\Lambda$-family.

%Likewise $\CK_\sfS(\g,\b)$ is also a basal $\mathcal R(\g, Q)$-module  with $\Lambda$-family.




Write $\rho\in \hha^*$ for the half sum of the positive roots.
For each $\nu\in \hha^*$, define the Verma module
\[
  \mathrm M(\nu):=\mathrm M(\g,\b,\nu):=\CU(\g)\otimes_{\CU(\b)} \C_{\nu-\rho},
\]
where  $\C_{\nu-\rho}$ is the one-dimensional $\hha$-module corresponds to the character $\nu-\rho\in \hha^*$, and every $\hha$-module is viewed an a $\b$-module via the canonical map $\b\rightarrow \hha$. Write
$\oL(\lambda):=\oL(\g,\b,\lambda)$ for the unique irreducible quotient of $ \mathrm M(\lambda)$.

Recall that an element $\nu\in \hha^*$ is said to be dominant if
\be\label{dominant}
    \la \nu, \check \alpha\ra\notin -\bN^+ \qquad\textrm{for all $\check \alpha\in \check \Delta^+$},
  \ee
  and is said to be regular if
  \[
    \la \nu, \check \alpha\ra\neq 0 \qquad\textrm{for all $\check \alpha\in \check \Delta$}.
  \]



For every $w\in W$, define a map
\[
\begin{array}{rcl}
  \Psi_{w}: \Lambda&\rightarrow  &\CK(\g,\b), \\
   \nu&\mapsto& \mathrm M(w \cdot \nu).
   \end{array}
\]
Then $\Psi_w\in \Coh_{\Lambda}( \CK(\g,\b))$, and
\be\label{basis}
\textrm{ $\{\Psi_w\}_{w\in W}$ is a basis of $\Coh_{\Lambda}( \CK(\g,\b))$. }
\ee
There is also a unique element $\overline \Psi_w\in \Coh_{\Lambda}( \CK(\g,\b))$ such that
\[
  \overline \Psi_w(\nu)=\oL(w\cdot  \nu)\quad \textrm{for all regular dominant element $\nu\in \Lambda$}.
\]
Then $\{\overline \Psi_w\mid w\in W\}$ is  also a   basis of the coherent continuation representation $\Coh_{\Lambda}( \CK(\g,\b))$. See \cite[Section 7.10]{H}.

Let $W$ acts on  $\CK(\g,\b)$ as $\mathcal R(\g, Q)$-module automorphisms by
\[
  w\cdot (\mathrm M(\nu))=\mathrm M(w\cdot \nu)\quad \textrm{for all }\nu\in \hha^*.
\]
By the functority this yields an acton of $W$ on  $\Coh_{\Lambda}( \CK(\g,\b))$ as  automorphisms of $W_\Lambda$-representations. Note that
the resulting action of $W\times W_\Lambda$ on $\Coh_{\Lambda}( \CK(\g,\b))$ is explicitly given by
 \[
   (w_1, w_2)\cdot  \Psi_{w}=\Psi_{w_1 w w_2^{-1}}\quad\textrm{for all $w_1\in W$, $w_2\in W_\Lambda$,  $w\in W$.}
 \]


Let $\Lambda^\g\subset \hha^*$ denote the $Q^\g$-coset containing $\Lambda$, and let  $\Lambda_\g\subset \Lambda$ be a $Q_\g$-coset.
Put
\[
  \Delta(\Lambda):=\{\alpha\in \Delta\mid \la \check \alpha, \nu\ra\in \Z \textrm{ for some (and all) }\nu\in \Lambda\}.
\]
Here and henceforth,  $\check \alpha\in \hha$ denotes the coroot corresponding to $\alpha$.
This is a root system with the corresponding coroots
\[
  \check \Delta(\Lambda):=\{\check \alpha\in \check \Delta \mid \la \check \alpha, \nu\ra\in \Z \textrm{ for some (and all) }\nu\in \Lambda\}.
\]
Let $W(\Lambda)\subset W$ denote the Weyl group of the root system $\Delta(\Lambda)$. Then
\[
  W(\Lambda_\g)=W(\Lambda)=W(\Lambda^\g)=W_{\Lambda_\g}\subset W_{\Lambda}\subset  W_{\Lambda^\g}.
\]


The Coherent continuation representation $\Coh_{\Lambda}( \CK(\g,\b))$ only depends on $\Lambda^\g$ by the following lemma, which obviously follows from \eqref{basis}.
\begin{lem}\label{restco0}
The restriction map
\[
  \Coh_{\Lambda^\g}( \CK(\g,\b))\rightarrow \Coh_{\Lambda}( \CK(\g,\b))
\]
is a $W\times W_\Lambda$-equivariant  linear isomorphisms,
and the restriction map
\[
 \Coh_{\Lambda}( \CK(\g,\b))\rightarrow \Coh_{\Lambda_\g}( \CK(\g,\b))
\]
is a $W\times W(\Lambda)$-equivariant  linear isomorphisms.

\qed
\end{lem}


\subsection{Jantzen matrix}

The Jantzen matrix $\{a_{\Lambda}(w_1, w_2)\in \Z\}_{w_1, w_2\in W}$ of integers is defined by
\[
  \overline \Psi_{w_1}=\sum_{w_2\in W}  a_{\Lambda}(w_1, w_2) \cdot  \Psi_{w_2}   \qquad (\textrm{as elements of $\CK(\g,\b)$}), \quad w_1\in W.
\]


\begin{lem}\label{lem33} (\cf \cite[Section 5.2]{H})
 Let $w_1, w_2\in W$. If $w_1 W(\Lambda)\neq w_2 W(\Lambda)$, then $a_{\Lambda}(w_1, w_2)=0$.
\qed
\end{lem}

\trivial[h]{\begin{proof}

For each $Q_\g$-coset  $\Lambda'\subset \hha^*$,
 let $\mathrm{Rep}_{\Lambda'}(\g, \b)$ denote the full subcategory of $\mathrm{Rep}(\g, \b)$ consisting of the modules whose weights are contained in $\Lambda'-\rho$.
 Write $\CK_{ \Lambda'}(\g, \b)$ for the Grothendieck group of this subcategory. Then
 \[
   \CK(\g, \b)=\bigoplus_{\Lambda'\in Q_\g\backslash \hha^*} \CK_{ \Lambda'}(\g, \b)
 \]
 and
 \be\label{decLam00}
 \Coh_{\Lambda}( \CK(\g,\b))=\bigoplus_{w\in W/W(\Lambda)}\Coh_{\Lambda,w}(\CK(\g, \b) ),
\ee
where
 \[
   \Coh_{\Lambda,w}( \CK(\g,\b)):=\{\Psi\in \Coh_{\Lambda}( \CK(\g,\b))\mid \Psi(\lambda)\in \CK_{w\lambda+Q_\g}(\g, \b)\ \textrm{for all } \lambda\in \Lambda\},
 \]
 which is a $W(w\cdot \Lambda)\times W(\Lambda)$-subrepresentation of  $\Coh_{\Lambda}( \CK(\g,\b))$. This implies the lemma by noting that $\Psi_{w_1}, \overline{\Psi}_{w_1}\in  \Coh_{\Lambda}( \CK(\g,\b))$ for all $w_1\in w  W(\Lambda)$. \end{proof}
}

Put
\[
  \Delta^+(\Lambda):=\Delta(\Lambda)\cap \Delta^+
\]
and
\be\label{wprime}
  W'(\Lambda):=\{w\in W\mid  w\cdot (\Delta^+(\Lambda) )= \Delta^+(w\cdot \Lambda)\}.
\ee
Then the group multiplications yield a bijective map
\[
  W'(\Lambda)\times W(\Lambda)\rightarrow W.
\]

The following lemma is a direct consequence of a theorem of Soergel (\cite[Section 2.5, Theorem 11]{Soergel}).
\begin{lem} \label{wprime0}
For all  $w'\in W'(\Lambda)$ and $w_1, w_2\in W(\Lambda)$,  $$a_{\Lambda}(w'w_1, w'w_2)= a_{\Lambda}(w_1, w_2).$$
Moreover, the matrix   $\{a_{\Lambda}(w_1, w_2)\}_{w_1, w_2\in W(\Lambda)}$ only depends on the set of simple reflections in the Weyl group $W(\Lambda)$.  More precisely, suppose that $(\g', \Lambda',  W(\Lambda'))$ is a triple as $(\g, \Lambda, W(\Lambda))$, and $\eta: W(\Lambda)\rightarrow W(\Lambda')$ is a group isomorphism that restricts to a bijection between the sets of  simple reflections, then $$a_{\Lambda}(w_1, w_2)=a_{\Lambda'}(\eta(w_1), \eta(w_2))$$ for all $w_1, w_2\in W(\Lambda)$.
\qed
\end{lem}


Recall that a polynomial function on $\hha^*$ or $\hha$ is said to be $W(\Lambda)$-harmonic if it is  annihilated by all the $W(\Lambda)$-invariant constant coefficient differential operators without constant term.
% Write $\CH_\Lambda(\hha^*)$ for the space of the $W(\Lambda)$-harmonic polynomial functions, which is naturally a representation of $W(\Lambda)$.


For every $w\in W$, define a polynomial function $\tilde p_{\Lambda, w}$ on $\hha\times \hha^*$ by
\be\label{polynomial0000}
  \tilde p_{\Lambda, w}(x,\nu):= \sum_{w_1\in W} a_{\Lambda}(w, w_1)\cdot  \la x, w_1\cdot \nu\ra^{m_{\Lambda, w}},\quad  \textrm{for all }  \ x\in \hha, \nu\in \hha^*,
\ee
where $m_{\Lambda, w}$ is the smallest non-negative integer (which always exists) that makes the right-hand side of \eqref{polynomial0000} a nonzero polynomial function.
Then there is a $W(w\cdot\Lambda)$-harmonic polynomial function $p'_{\Lambda, w}$ on $\hha$ and a  $W(\Lambda)$-harmonic polynomial function $p_{\Lambda, w}$ on $\hha^*$ such that
\[
  \tilde  p_{\Lambda,  w}(x,\nu)= p'_{\Lambda,w}(x)\cdot  p_{\Lambda,w}(\nu), \quad\textrm{for all }  x\in \h,  \ \nu\in \h^*.
\]
See \cite{King} and \cite[Section 5.1]{J.hw}. These two harmonic polynomial functions are nonzero and are uniquely determined up to scalar multiplication.



%\subsection{Basal representations and cells}\label{seccell00}
\subsection{Left, right, and double cells}\label{seccell}

We define a basal vector space to be a complex vector space $V$ equipped with a basis $\mathcal D\subset V$, and call elements of $\mathcal D$ the basal elements in $V$. A subspace of a basal space $V$ is called a basal subspace if it is spanned by a set of basal elements of $V$. Every basal subspace is obviously a basal space.

As an example, if $\CK$ is the Grothendieck group of an abelian category in which all objects have finite length, then $\CK$ is a basal vector space with the irreducible objects as the basal elements.

%In particular, $\CK(G)$ is a basal vector space.




\begin{defn}
Let $E$ be a  finite group. A basal representation of $E$ is basal vector space carrying a representation of $E$. A basal subrepresentation of a basal representation $V$ is a subrepresentation of $V$ that is simultaneously a basal subspace.
\end{defn}




Let $V$ be a basal representation of a finite group $E$, with basal elements
$\mathcal D\subset V$. For each subset $\mathcal C\subset \mathcal D$, write $\la \CC\ra$ for
the smallest basal subrepresentation of $V$ containing $\CC$. For each
$\phi\in \mathcal D$, write $\la \phi\ra:=\la \{\phi\}\ra$ for simplicity. We define an equivalence relation $\approx$ on $\mathcal D$  by
\[
  \phi_1 \approx \phi_2 \quad \textrm{ if and only if
  } \quad \la \phi_1 \ra =\la \phi_2 \ra \qquad (\phi_1, \phi_2\in \mathcal D).
\]
An equivalence class of the relation $\approx$ on the set $\mathcal D$ is called a cell in $V$.

We say that a subset $\CC$ of $\mathcal D$ is order closed if for every $\phi\in \mathcal D$, we have that $\phi\in \CC$ whenever $\la \phi\ra\subset \la \phi'\ra $ for some $\phi'\in \CC$.
This is equivalent to saying that $\la \CC\ra$ is spanned by $\CC$. In general, write $\overline \CC$ for the smallest order closed subset of $\mathcal D$ containing $\CC$.

When $\CC$ is a cell in $V$, define the cell representation attached to $\CC$ by
\[
  V(\CC):=\la \overline \CC \ra/ \la \overline \CC\setminus \CC\ra.
\]
Note that the set $ \overline \CC\setminus \CC$ is order closed, and $\{\phi+ \la \overline \CC\setminus \CC\ra\}_{ \phi\in \CC}$ is a basis of $V(\CC)$.
%More genera


We view $\Coh_{\Lambda}( \CK(\g,\b))$ as a basal representation of $W\times W_\Lambda$ with the basal elements
 $\{\overline \Psi_{w}\mid w\in W\}$.
%Let $w\in W$. The space  $\Coh_{\Lambda,w}( \CK(\g,\b))$   is a basal $W(w\cdot \Lambda)\times W(\Lambda)$-subrepresentation of $\Coh_{\Lambda}( \CK(\g,\b))$.
Write
\[
\Coh^{LR}_{\Lambda}( \CK(\g,\b)):=\Coh_{\Lambda}( \CK(\g,\b)),
\]
 to be viewed as a basal representation of $W\times W(\Lambda)$.
Likewise, write
\[
\Coh^{L}_{\Lambda}( \CK(\g,\b)):=\Coh_{\Lambda}( \CK(\g,\b)),
\]
 to be viewed as a basal representation of $W$, and write
 \[
 \Coh^{R}_{\Lambda}( \CK(\g,\b)):=\Coh_{\Lambda}( \CK(\g,\b)),
 \]
  to be viewed as a basal representation of $W(\Lambda)$ (as a subgroup of $W_\Lambda$).

Cells in $\Coh^{L}_{\Lambda}( \CK(\g,\b))$, $\Coh^{R}_{\Lambda}( \CK(\g,\b))$, and $\Coh^{LR}_{\Lambda}( \CK(\g,\b))$ are respectively call left, right, and double cells in  $\Coh_{\Lambda}( \CK(\g,\b))$. We say that two basal elements of $\Coh_{\Lambda}( \CK(\g,\b))$ lay in the same brick if they lay in a common left cell as well as a common right cell. This defines an equivalence relation and an equivalence class of this equivalence relation is called a brick in $\Coh_{\Lambda}( \CK(\g,\b))$.
%A set of the form
%\[
% \CB_{\Lambda, w'}:=
%   \{\overline \Psi_{w'w_0}\mid w_0\in W(\Lambda)\},
%\]
%where $w'\in W'(\Lambda)$, is called a block in $\Coh_{\Lambda}( \CK(\g,\b))$.
It is clear that every left (right) cell is contained in a double cell, and every brick is contained in left  (right) cell.

For every set $\CC$ of basal elements in $\Coh_{\Lambda}( \CK(\g,\b))$, write $\la \CC\ra_L$ for the smallest basal subrepresentation of $\Coh^L_{\Lambda}( \CK(\g,\b))$ containing $\CC$.
Similarly define $\la \CC\ra_R$ and $\la \CC\ra_{LR}$.



\begin{lem}\label{leftp}
Let  $w_1, w_2\in  W$.

\noindent
(a) The basal elements $\overline \Psi_{w_1}$ and $\overline \Psi_{w_2}$ lay in a common left cell if and only if
\[
\C\cdot p_{\Lambda, w_1}=\C\cdot p_{\Lambda, w_2}.
\]

\noindent
(b) The basal elements $\overline \Psi_{w_1}$ and $\overline \Psi_{w_2}$ lay in a common right cell if and only if
\[
\C\cdot p'_{\Lambda, w_1}=\C\cdot p'_{\Lambda, w_2}.
\]



%The basal elements $\overline \Psi_{w_1}$ and $\overline \Psi_{w_2}$ lay in a common double cell if and only if
%\[
%\sigma_{\Lambda, w_1}=\sigma_{\Lambda, w_2}.
%\]

\noindent
(c) The basal elements $\overline \Psi_{w_1}$ and $\overline \Psi_{w_2}$ lay in a common brick if and only if
\[
\C\cdot \tilde p_{\Lambda, w_1}=\C \cdot \tilde p_{\Lambda, w_2}.
\]

\end{lem}
\begin{proof}
This follows by \Cref{lem33}, \Cref{wprime0}, \cite[Theorems 5.4 and 5.5]{J2} and ?\cite[Proposition 2.9]{j}.
\end{proof}

Using Lemma \ref{leftp}, we obviously define a $W(\Lambda)$-harmonic polynomial function   $p_{\CC^L}$  on $\hha^*$ for every left cell $\CC^L$ in $\Coh_{\Lambda}( \CK(\g,\b))$,   a $W$-harmonic  polynomial function $ p'_{\CC^R}$ on $\hha$  for every right cell $\CC^R$ in $\Coh_{\Lambda}( \CK(\g,\b))$, and a polynomial function $\tilde p_\CB$ on $\hha\times \hha^*$ for every brick $\CB$ in  $\Coh_{\Lambda}( \CK(\g,\b))$. They are uniquely determined up to scalar multiplication.


\subsection{Cells and special representations}

For every $\sigma\in \Irr(W)$, its fake degree is defined to be
 \begin{equation}\label{eq:fdeg}
 a(\sigma):=\min\{k\in \BN\,|\, \sigma \textrm{ occurs in the $k$-th symmetric power $\oS^k(\hha)$}\}.
 \end{equation}
This is well-defined since every  $\sigma\in \Irr(W)$ occurs in $\oS(\hha^*)$. The representation $\sigma$ is said to be univalent if it occurs in $\oS^{a(\sigma)}(\hha_{\mathrm s})$ with multiplicity one, where $\hha_{\mathrm s}:=\Span(\check \Delta)$ denotes the span of the coroots. Recall that every special  (in the sense of Lusztig) irreducible representation of $W$ is univalent.



We have a decomposition
\[
  \hha=(\Delta(\Lambda))^\perp\oplus \Span( \check \Delta(\Lambda))
  \]
where
\[
(\Delta(\Lambda))^\perp:=\{x\in \hha\mid \la x , \alpha \ra=0\textrm{ for all $\alpha\in \Delta(\Lambda)$}\}.
\]
 For every univalent irreducible representation $\sigma_0$ of $W(\Lambda)$, whenever it is convenient, we view it as a subrepresentation of  $\oS^{a(\sigma_0)}(\hha)$ via the inclusions
\[
\sigma_0=  \C\otimes \sigma_0\subset \oS^0((\Delta(\Lambda))^\perp)\otimes \oS^{a(\sigma_0)}(\Span( \check \Delta(\Lambda)))\subset  \oS^{a(\sigma_0)}(\hha)
\]
 Recall that the $W$-subrepresentation of $\oS^{a(\sigma_0)}(\hha)$ generated by $\sigma_0$
is irreducible and univalent, with the same fake degree as that of $\sigma_0$ (This result is due to Macdonald, Lusztig, and Spaltenstein. See \cite[Chapter 11]{Carter}). This irreducible representation of $W$ is called the $j$-induction of $\sigma_0$, to be denoted by $j_{W(\Lambda)}^W(\sigma_0)$.


If  $\sigma_0$ is special, then the $j$-induction $j_{W(\Lambda)}^W \sigma_0$  is Springer in the sense that it corresponds to the trivial local system on a nilpotent orbit in $\g^*$, via the Springer correspondence. Write $\CO_{\sigma_0}\in \overline \Nil(\g^*)$ for this  nilpotent orbit. Recall that
\[
  \dim \CO_{\sigma_0}=2\cdot (\sharp (\Delta^+) - a(\sigma_0)).
\]



Suppose that $\CC$ is a double cell in $\Coh_{\Lambda}( \CK(\g,\b))$.
Put $m_\CC:=m_{\Lambda, w}$, where $w\in W$ is an element such that $\overline \Psi_{w}\in \CC$. By Lemma \ref{leftp},  this is  independent of the choice of $w$.
For every $\Psi=\sum_{w\in W} a_{w} \Psi_{w}\in \la \CC\ra_{LR}$, define a polynomial function $\tilde p_\Psi$ on $\hha\times \hha^*$ by
\be\label{polynomial00}
  \tilde p_{\Psi}(x,\nu):= \sum_{w\in W} a_{w}\cdot  \la x, w \cdot \nu\ra^{m_{\CC}},\quad  \textrm{for all }  \ x\in \hha, \nu\in \hha^*,
\ee
Then the linear map
\be\label{lara}
    \la \CC\ra_{LR}\rightarrow \oS(\hha^*)\otimes \oS(\hha), \quad \Psi\mapsto \tilde p_{\Psi}\quad (\oS\textrm{ indicates the symmetric algebra})
\ee
is $W\times W(\Lambda)$-equivariant  and descends to a  map
\be\label{lara2}
    \Coh^{LR}_{\Lambda}( \CK(\g,\b))( \CC) \rightarrow \oS(\hha^*)\otimes \oS(\hha).%, \quad \Psi\mapsto \tilde p_{\Psi}
\ee

For every left cell $\CC^L$ that is contained in $\CC$, the cell representation  $\Coh^{L}_{\Lambda}( \CK(\g,\b))( \CC^L)$ is naturally identified with a subspace of $\Coh^{LR}_{\Lambda}( \CK(\g,\b))( \CC)$ and thus \eqref{lara3} restricts to a  $W$-equivariant linear map
\be\label{lara3}
    \Coh^{L}_{\Lambda}( \CK(\g,\b))( \CC^L) \rightarrow \oS(\hha^*)\otimes \oS(\hha).%, \quad \Psi\mapsto \tilde p_{\Psi}
\ee
Similarly, for every right cell $\CC^R$ that is contained in $\CC$, \eqref{lara2} restricts to a  $W(\Lambda)$-equivariant linear map
\be\label{lara4}
    \Coh^{R}_{\Lambda}( \CK(\g,\b))( \CC^R) \rightarrow \oS(\hha^*)\otimes \oS(\hha).%, \quad \Psi\mapsto \tilde p_{\Psi}
\ee


Let  $\Irr^{\mathrm{sp}}(W(\Lambda))$ denote the subset of $\Irr(W(\Lambda))$ consisting of the special (in the sense of Lusztig) irreducible representations.
\begin{prop}\label{cell00}
\noindent (a) There is a unique special irreducible representation $\sigma_\CC$ of $W(\Lambda)$ with fake degree $m_\CC$ such that  the image of \eqref{lara2} equals $\left(j_{W(\Lambda)}^W\sigma_\CC\right)\otimes \sigma_\CC$.  Moreover, $\left(j_{W(\Lambda)}^W\sigma_\CC\right)\otimes \sigma_\CC$  occurs in  $\Coh^{LR}_{\Lambda}( \CK(\g,\b))( \CC)$ with multiplicity one, and
\[
\{\tilde p_{\CB}\}_{\CB\textrm{ is a brick contained in $\CC$}}
\]
 is a basis of $\left(j_{W(\Lambda)}^W\sigma_\CC\right)\otimes \sigma_\CC$.


\noindent
(b)
For every  left cell $\CC^L$ that is contained in $\CC$, the image of \eqref{lara3} equals $\left(j_{W(\Lambda)}^W\sigma_\CC\right)\otimes  p_{\CC^L}$. Moreover, $j_{W(\Lambda)}^W\sigma_\CC$ occurs in $\Coh^{L}_{\Lambda}( \CK(\g,\b))( \CC^L)$ with multiplicity one, and
\[
\{\tilde p_{\CB} \}_{\CB\textrm{ is a brick contained in $\CC^L$}}
\]
 is a basis of the image of  \eqref{lara3}.

\noindent (d)
For every  right cell $\CC^R$ that is contained in $\CC$, the image of \eqref{lara4} equals $p'_{\CC^R}\otimes \sigma_\CC$. Moreover, $\sigma_\CC$ occurs in $\Coh^{R}_{\Lambda}( \CK(\g,\b))( \CC^R)$ with multiplicity one, and
\[
\{\tilde p_{\CB} \}_{\CB\textrm{ is a brick contained in $\CC^R$}}
\]
 is a basis of the image of  \eqref{lara4}.

\noindent (d)
The irreducible representation $\sigma_\CC$ is the unique special irreducible representation of $W(\Lambda)$ that occurs in $\Coh^{LR}_{\Lambda}( \CK(\g,\b))( \CC)$. Moreover, the map
\[
  \{\textrm{double cell in $ \Coh_{\Lambda}( \CK(\g,\b))$}\}\rightarrow \Irr^{\mathrm{sp}}(W(\Lambda)), \quad \CC\mapsto \sigma_\CC
\]
is bijective.

\end{prop}


Let $\sigma_\CC$ be as in \Cref{cell00}.

We say that two irreducible representation of  $W(\Lambda)$ lay in a common double cell if there is a double cell $\CC$ in $\Coh_{\Lambda}( \CK(\g,\b))$ such that both of the representations occur in $\Coh^{LR}_{\Lambda}( \CK(\g,\b))( \CC)$. This defines an equivalence relation on the set $\Irr(W(\Lambda))$, and an equivalence class of this equivalence relation is called a double cell in  $\Irr(W(\Lambda))$. Similar to \Cref{wprime0}, this notion of double cells only depends on the set of simple reflections in $W(\Lambda)$.

\begin{prop}\label{dcrep}
As a representation of $W\times W(\Lambda)$,
\[
 \Coh^{LR}_{\Lambda}( \CK(\g,\b))( \CC)\cong \bigoplus_{\sigma\in \Irr(W(\Lambda)), \textrm{$\sigma$ lies in the same double cell with $\sigma_\CC$}} \left(\Ind_{W(\Lambda)}^W\sigma\right)\otimes \sigma.
\]
\end{prop}
\subsection{Primitive ideals}


Let $w\in W$.

\trivial[h]{
\begin{lem}
 For every simple reflection $s\in W(\Lambda)$, either $s\cdot  \overline \Psi_w=- \overline \Psi_w$ or $s\cdot  \overline \Psi_w$ is a nonnegative integral linear combination of basal elements in $\Coh_{\Lambda}( \CK(\g,\b))$.
 \end{lem}
\begin{proof}
This is proved in the setting of Harish-Chandra modules in \cite[Corollary 7.3.9]{Vg}. The same proof works here for highest weight modules.

\end{proof}

}
\begin{lem}
 For every simple reflection $s\in W(\Lambda)$,
\begin{eqnarray*}
&&
s\cdot  \overline \Psi_w=- \overline \Psi_w\\
&\Longleftrightarrow & s\cdot p_{\Lambda,w}=-p_{\Lambda, w}\\
  &  \Longleftrightarrow &  w(\alpha)\in -\Delta^+(\Lambda),
  \end{eqnarray*}
where $\alpha\in \Delta^+(\Lambda)$ is the simple root corresponding to $s$.
\end{lem}
%\begin{proof}
%\cite[Corollary 7.3.9]{Vg}.
%\end{proof}

Define the $\tau$-invariant of $ \overline \Psi_w$ to be
\[
  \tau_{\Lambda, w}:=\{s\in W(\Lambda) \textrm{ is a simple reflection }\mid   s\cdot p_{\Lambda,w}=-p_{\Lambda, w}\}.
\]
This only depends on the left cell containing $\overline \Psi_w$, and thus the $\tau$-invariant $\tau_{\CC^L}$ is obviously defined for each left cell $\CC^L$ in  $\Coh_{\Lambda}( \CK(\g,\b))$.




%Note that for $W_\lambda$ is naturally a Coxeter group, and   if $\lambda \in \Lambda$, then
%\[
 % \{ \textrm{simple reflection in $W(\Lambda)$ that fixes $\lambda$}\}= \{ \textrm{simple reflection in $W_\lambda$}\}.
%\]



\begin{prop}
Let $\nu\in \Lambda$ be a dominant element. Then
\[
\overline \Psi_w(\nu)=\left\{
                  \begin{array}{ll}
                    \oL(w\cdot \nu),\quad &\textrm{if there is no element of  $\tau_{\Lambda, w}$ that fixes $\nu$};\\
                    0, \quad &\textrm{otherwise}.
                    \end{array}
                    \right.
                    \]
              %      If $\overline \Psi_w(\nu)\neq 0$, then the associated variety of   $\mathrm J(w\cdot \nu)$ equals the Zariski closure $\overline \CO_{\Lambda, w}\subset \g^*$ of $\CO_{\Lambda, w}$.



\end{prop}


Let $\Ann(M)\subset \mathcal U(\g)$ denote the annihilator ideal of a $\mathcal U(\g)$-molude $M$.

%For every $\lambda\in \hha^*$, let $\mathrm J(\lambda)$ denote the annihilator ideal of $\oL(\lambda)$.
\begin{prop}\label{primitivei}
Let $\nu\in \Lambda$ be a dominant element. Then  the map
\begin{eqnarray*}
 && \{w_1\in W\mid \textrm{ there is no element of  $\tau_{\Lambda, w}$ that fixes $\nu$}\}\\
 &\xrightarrow{w_1\mapsto \Ann( \oL(w_1\cdot \nu))}  & \{\textrm{primitive ideal of $\CU(\g)$ containing $\ker \chi_\nu$}\}
\end{eqnarray*}
is surjective, and for all $w_1, w_2$ in the domain of this map,
\[
  \Ann(\oL(w_1\cdot \nu))=\Ann(\oL(w_2\cdot \nu))\quad \textrm{if and only if}\quad \C\cdot p_{\Lambda, w_1}=\C\cdot p_{\Lambda, w_2}.
\]

   \end{prop}

By Proposition \ref{primitivei}, for every dominant element $\nu\in \Lambda$, we have an obvious bijection
\begin{eqnarray*}
 && \{\CC^L \textrm{ is a left cell in $\Coh_{\Lambda}( \CK(\g,\b))$}
 \mid \textrm{ there is no element of  $\tau_{\CC^L}$ that fixes $\nu$}\}\\
 &\xrightarrow{\sim} & \{\textrm{primitive ideal of $\CU(\g)$ containing $\ker \chi_\nu$}\}.
\end{eqnarray*}
For every $I$ in the codomain of this map, write $\CC^L_{\nu, I}$ for the left cell that corresponds to it under this bijection.
We define the Goldie rank representation of $I$ to be
\be\label{grrj}
\sigma_{\nu, I}:=\sigma_{\CC^{LR}_{\nu,I} }\in \Irr^{\mathrm{sp}}(W(\Lambda)),
\ee
where $\CC^{LR}_{\nu,I}$ is the double cell containing $\CC^L_{\nu, I}$.

\begin{lem}\label{primitivei}
Let $\nu\in \Lambda$ be a dominant element. Then for all primitive ideals $I_1$ and $I_2$ of $\CU(\g)$  containing $\ker \chi_\nu$,
\[
  I_1\subset I_2\quad \textrm{if and only if}\quad \la \CC^L_{\nu, I_1}\ra_L\supseteq \la \CC^L_{\nu, I_2}\ra_L.
\]

\end{lem}





\subsection{Associated varieties}

\begin{prop}\label{assv}
Let $\nu\in \Lambda$ be a dominant element and let $I$ be a primitive ideal of $\CU(\g)$  containing $\ker \chi_\nu$.   Then the associated variety of   $I$ equals the Zariski closure
$\overline{\CO_{\sigma_{\nu,I}}}\subset \g^*$ of $\CO_{\sigma_{\nu,I}}$.



\end{prop}





Let $\sfS\subset \mathrm{Nil}(\g^*)$ be an $\Inn(\g)$-stable Zariski closed subset.
Let  $\Rep_\sfS(\g,\b)$ denote the full   subcategory of $\Rep(\g,\b)$ of the modules  $M$ such that the  associated variety of $\Ann(M)$  is contained in $\sfS$.
We obviously have the coherent continuation representation  $\Coh_{\Lambda}(\CK_\sfS(\g,\b))$.


Put $\CO_{\Lambda, w}:=\CO_{\sigma_\CC}$, where
                    $\CC$ is the double cell  in $\Coh_{\Lambda}( \CK(\g,\b))$ containing $\overline \Psi_w$.%, and $\sigma_\CC$ is the representation as in Proposition \ref{cell00}.



\begin{lem}\label{basalassv}
The coherent family $\overline \Psi_w$ belong to $\Coh_{\Lambda}(\CK_\sfS(\g,\b))$ if and only if  $\CO_{\Lambda, w}\subset \sfS$.
\end{lem}
\begin{proof}
In view of \Cref{assv}, the ``only if$\,$" part is obvious. The ``if$\,$" part may  be proved by  using \cite{Vg}*{Part (b) of Proposition~7.2.22} (which is formulated and proved for the case of Harish-Chandra module but is obviously generalized to  the case of highest weight modules).

\end{proof}



\begin{lem}\label{assv33}
Let $w_1, w_2\in W$. If $\la \overline \Psi_{w_1}\ra_L\supsetneq \la \overline \Psi_{w_2}\ra_L$, then $\overline{\CO_{\Lambda, w_1}}\supsetneq \overline{\CO_{\Lambda, w_2}}$.

\end{lem}
\begin{proof}
Pick a  regular dominant element $\nu\in \Lambda$.  Put $I_i:=\Ann(\Psi_{w_i}(\nu))$ ($i=1,2$). Lemma \ref{primitivei} implies that  $I_1\subsetneq I_2$ and hence
\[
  \AV(I_1)\supsetneq \AV(I_2)\qquad (\AV\textrm{ indicates the associated variety}).
\]
On the other hand, \Cref{assv} implies that  $\AV(I_i)=\overline{\CO_{\Lambda, w_i}}$ ($i=1,2$). Hence the lemma follows.
\end{proof}



\begin{prop}\label{basals}
The space $\Coh_{\Lambda}(\CK_\sfS(\g,\b))$ is a $W\times W_\Lambda$-subrepresentation of $\Coh_{\Lambda}(\CK(\g,\b))$ spanned by the basal elements
\be\label{spanb}
 \{\overline \Psi_{w_1}\mid w_1\in W,  \CO_{\Lambda, w_1}\subset \sfS\}.
\ee
\end{prop}
\begin{proof}
Lemma \ref{basalassv}  implies that  $\Coh_{\Lambda}(\CK(\g,\b))$ is spanned by  the set \eqref{spanb}.

The space $\Coh_{\Lambda}(\CK_\sfS(\g,\b))$ is  clearly $W_\Lambda$-stable, and \Cref{assv33} implies that it is also $W$-stable. This proves the proposition.  \end{proof}

Define
\[
  \Irr_\sfS^{\mathrm{sp}}(W(\Lambda)):= \Set{\sigma_0\in \Irr^{\mathrm{sp}}(W(\Lambda))| \cO_{\sigma_0}\subset \sfS}
\]
and
\begin{eqnarray*}%\label{sfc01}
 && \Irr_\sfS(W(\Lambda))\\
  &:=& \Set{\sigma_0\in \Irr(W(\Lambda))| \sigma_0 \textrm{ lies in the same double cell with an element of } \Irr_\sfS^{\mathrm{sp}}(W(\Lambda))}.
\end{eqnarray*}

By \Cref{basals}, $\Coh_{\Lambda}(\CK_\sfS(\g,\b))$ is a basal subrepresentation of  $\Coh^{LR}_{\Lambda}(\CK(\g,\b))$.

\begin{prop}\label{cohbbs}
As a representation of $W\times W(\Lambda)$, % we have
\[
  \Coh_{\Lambda}( \CK_{\sfS}(\g,\b))\cong \bigoplus_{\sigma\in \Irr_\sfS(W(\Lambda))} \left(\Ind_{W(\Lambda)}^W \sigma\right)\otimes \sigma.
\]
Here $\Ind$ indicates the induced representation.
\end{prop}

\begin{proof}
Write $\mathcal D_0$ for the set of basal elements in $ \Coh_{\Lambda}(\CK_\sfS(\g,\b))$ and
 choose a filtration
\[
  \mathcal D_0\supset \mathcal D_1\supset \cdots \supset \mathcal D_k=\emptyset \qquad (k\in \BN)
\]
such that
\begin{itemize}
\item
$\mathcal C_i:=\mathcal D_i\setminus \mathcal D_{i+1}$ is a double cell in $\Coh_{\Lambda}(\CK(\g,\b))$ for all $i=0,1, \dots, k-1$, and
\item
for all $i_1, i_2\in \{0,1, \dots, k-1\}$,   $\la \CC_{i_1}\ra_{LR} \subset \la \CC_{i_2}\ra_{LR}$ implies that $i_1\geq i_2$.
\end{itemize}
Then it is elementary to see that  $\la \mathcal D_i\ra_{LR}$ is spanned by $\mathcal D_i$ ($i=0, 1,2, \cdots ,k$).
%Let $\sigma_i\in \Irr^{\mathrm{sp}}(W(\Lambda))$ denote the unique special representation such that $(w\cdot \sigma_i)\otimes \sigma_i$ occurs in $\Coh_{\Lambda, w}( \CK_{\sfS}(\g,\b))(\CC_i)$.

Then \Cref{cell00} and  \Cref{basalassv} imply that
\[
\{\sigma_{\CC_0}, \sigma_{\CC_1}, \dots, \sigma_{\CC_{k-1}}\}= \Irr_\sfS^{\mathrm{sp}}(W(\Lambda)).
\]
Thus we have that
\begin{eqnarray*}
   &&\Coh_{\Lambda}(\CK_\sfS(\g,\b))
\\
   &\cong & \bigoplus_{i=0}^{k-1}  \la \mathcal D_i\ra_{LR}/\la \mathcal D_{i+1}\ra_{LR}\\
    &\cong & \bigoplus_{i=0}^{k-1}  \Coh_{\Lambda}(\CK(\g,\b))(\mathcal C_i)\\
    &\cong &\bigoplus_{\sigma\in \Irr_\sfS(W(\Lambda))} \left(\Ind_{W(\Lambda)}^W\sigma\right)\otimes \sigma\qquad(\textrm{by \Cref{dcrep}}).
\end{eqnarray*}

\end{proof}

 For every $\nu\in \hha^*$, write $\CO_\nu\in \overline{\mathrm{Nil}}(\g^*)$ for the nilpotent orbit whose Zariski closure $\overline{\CO_{\nu}}\subset \g^*$ equals the associated variety of the maximal ideal $I_\nu$.


\begin{lem}\label{leftcnu}
Let $\nu\in \Lambda$ be a dominant element.   Then
\[
  \Coh^L_{\Lambda}( \CK(\g,\b))(\CC^L_{ \nu, I_\nu})\cong \Ind_{W(\Lambda)}^W \left(\left(J_{W_{\nu}}^{W(\Lambda)} \sgn \right)\otimes \sgn\right)
  \]
and
\be\label{leftc3}
  \left(\left(J_{W_{\nu}}^{W(\Lambda)} \sgn \right)\otimes \sgn\right) \cong \bigoplus_{\sigma\in \Irr_{\overline{\CO_{\nu}}}(W(\Lambda)), \, [1_{W_\nu} :\sigma]\neq 0 }  \sigma.
\ee
 Moreover,  $\sigma_{\CC^{LR}_{ \nu, I_\nu}}\cong \left(j_{W_{\nu}}^{W(\Lambda)} \sgn \right)\otimes \sgn$, and it is  the unique special irreducible representation of $W(\Lambda)$ that  occurs in \eqref{leftc3}.
Here $ \CC^{LR}_{ \nu, I_\nu}$ denote the double cell containing $\CC^{L}_{ \nu, I_\nu}$.

\end{lem}


The isomorphism \eqref{leftc3} implies that
\be\label{mulone}
 [1_{W_\nu} :\sigma]\leq 1
 \ee
for all $\nu\in \Lambda$ and all $ \sigma\in \Irr_{\overline{\CO_\nu}}(W(\Lambda))$.
 Here and henceforth, $[\ : \ ]$ indicates the multiplicity of the first
(irreducible) representation in the second one.
We define the Lusztig left cell attached to $\nu\in \Lambda$ to be the set
\be \label{ll}
\LC_{\nu}:=\left\{\sigma\in \Irr(W(\Lambda))\mid \sigma \textrm{ occurs in } \left(J_{W_{\nu}}^{W(\Lambda)} \sgn \right)\otimes \sgn\right \}.
\ee




\begin{lem}\label{leftcnu2}
Let $\nu\in \Lambda$ be a dominant element and let $I$ be a primitive ideal of $\CU(\g)$ containing $\ker \chi_\nu$.    Then $I=I_\nu$ if and only if $\sigma_{\nu, I}\cong \left(j_{W_{\nu}}^{W(\Lambda)} \sgn \right)\otimes \sgn$.




\end{lem}
\begin{proof}
The ``only if$\,$" part  follows from \Cref{leftcnu}. For the proof of the ``if$\,$" part, we suppose that $\sigma_{\nu, I}\cong \sigma_{\nu, I_\nu}$. Then \Cref{assv} implies that $I$ and $I_\nu$ has the same associated variety. Thus $I=I_\nu$ and the proof is finished.

\end{proof}
%In the notation of the last subsection, we have that $a(\sigma_\CC)=m_\CC$.

% Note that $\h_{\mathrm s}=\h/\c=\h/\h^W$, where $\h^W$ denotes the $W$-fixed vectors in $\h$.



%Let $\sigma_{\Lambda, w}$ denote the $W(\Lambda)$-subrepresentation of $\CH_\Lambda(\hha^*)$ generated by $ q_{\Lambda,w}$, which is a special irreducible representation  of $W(\Lambda)$ with fake degree $m_{\Lambda, w}$.
%Then $p_{\Lambda,w}(x)$ generates an irreducible $W(w\cdot\Lambda)$-subrepresentation of $\CH_{w\cdot\Lambda}(\hha)$ that is isomorphic to $w\cdot \sigma_{\Lambda, w}$,



%In particular, $\sigma_{\Lambda, w}$ is univalent and the
% the $j$-induction $j_{W(\Lambda)}^W \sigma_{\Lambda, w}$ is the $W$-subrepresentation of $ \oS^{m_{\Lambda, w}}(\hha)$ generated by $q_{\Lambda,w}$.
%Moreover, the representation $j_{W(\Lambda)}^W \sigma_{\Lambda, w}$ is Springer in the sense that it corresponds to the trivial local system on a nilpotent orbit in $\g^*$, via the Springer correspondence. Write $\CO_{\Lambda, w}\in \overline \Nil(\g^*)$ for this  nilpotent orbit.




\section{Preliminaries  on coherent families of Casselman-Wallach representations}



%\subsection{Coherent continuation representations for real reductive groups}

Let $G$ be a real reductive group in Harish-Chandra's class (which may be
linear or non-linear). Suppose that $\g$ is the complexified Lie algebra of $G$. In the rest of  this paper, unless otherwise mentioned, we use the corresponding lowercase gothic letter to denote the complexified Lie algebra of a Lie group. Let $\Rep(G)$ denote the category of Casselman-Wallach representations of $G$, and write $\CK(G)$ for the Grothedieck group of this category.
For every $\nu\in \hha^*$, write $\mathrm{Rep}_\nu(G)$ for the category of Casselman-Wallach representations of $G$ of generalized infinitesimal character $\nu$, and write   $\CK_{\nu}(G)$ for its  Grothedieck group.

% In particular,  $\fgg$ is the complexified Lie algebra of $G$.



\subsection{The parameter set for  coherent continuation representations}\label{extcoh}

Suppose that we are given a connected reductive complex Lie group $G_\C$ together with a Lie group homomorphism $\iota: G\rightarrow G_\C$ such that its differential $\mathrm d \iota: \Lie(G)\rightarrow \Lie(G_\C)$ has the following two properties:
\begin{itemize}
  \item the kernel of $\mathrm d \iota$ is contained in the center of the Lie algebra $ \Lie(G)$ of $G$;
  \item the image of  $\mathrm d \iota$ is a real form  of the Lie algebra $ \Lie(G_\C)$.
\end{itemize}
The  analytic character lattice of $ G_\C$ is identified with a subgroup of $\hha^*$ via $\mathrm d\iota$. We write $Q_\iota\subset \hha^*$ for this subgroup.

When $G_\C=\mathrm{Ad}(\g)$ and $\iota$ is the adjoint representation, $Q_\iota$ equals the root lattice $Q_\g$. In general $Q_\iota$ is $W$-stable and $Q_\g\subseteq Q_\iota \subseteq Q^\g$.
In the rest of this section we assume that $Q=Q_\iota$. Recall that $\Lambda\subset \hha^*$ is a $Q$-coset.


%Write $[\lambda]_\iota:=\lambda+Q_\iota\subset \hha^*$, and denote by $W_{[\lambda]_\iota}$ its stabilizer in $W$. Then $W_{[\lambda]_\iota}\supseteq W_\Lam$.

The algebra $\mathcal R(\g, Q)$ is obviously identified with  the Grothendieck group of the category of holomorphic finite-dimensional representations of $G_\C$.  Thus $\CK(G)$ is naturally a $\mathcal R(\g, Q)$-module by using the homomorphism $\iota$. As before, we form the coherent continuation representation $\Coh_\Lambda(\CK(G))$.


%It is a commutative $\C$-algebra under the tensor product of the finite-dimensional representations.  Generalizing Definition \ref{defcoh}, we make the following definition.




 Suppose that $H$ is a  Cartan subgroup of $G$. Write $\t$ for the complexified Lie algebra of the unique maximal compact subgroup of $H$. As before, denote by $\Delta_\h\subset \h^*$ the root system of $\g$. A root $\alpha\in \Delta_\h$ is called imaginary if $\alpha^\vee\in \t$. An imaginary root $\alpha\in \Delta_\h$ is said to be compact if the root spaces $\g_\alpha$ and $\g_{-\alpha}$ are contained in  the  complexified Lie algebra of a common compact subgroup of $G$.



Note that every Casselman-Wallach representation of $H$ is finite dimensional, and for every $\Gamma\in \Irr(H)$, there is a unique element $\mathrm d \Gamma\in \h^*$ such that the differential of $\Gamma$ is isomorphic to a direct sum of  one-dimensional representations of $\h$ attached to $\mathrm d  \Gamma$.

For every Borel subalgebra $\b$ of $\g$ containing $\h$, write
\be\label{xib00}
  \xi_\b: \hha\rightarrow \h
\ee
for the linear isomorphism attached to $\b$, whose transpose inverse is still denoted by $ \xi_\b: \hha^*\rightarrow \h^*$. Write
\[
  W(\hha^*, \h^*):=\{ \xi_\b: \hha^*\rightarrow \h^*\mid \textrm{$\b$ is a Borel subalgebra of $\g$ containing $\h$}\}.
  \]

Recall that $ \Delta^+\subset \hha^*$ denotes the set of positive roots. For every element $\xi\in  W(\hha^*, \h^*)$,  put
\be\label{deltawb}
  \delta(\xi):=\frac{1}{2}\cdot \sum_{\alpha \textrm{ is an imaginary root in $\xi \Delta^+$ }} \alpha- \sum_{\beta \textrm{ is a compact imaginary root in $\xi \Delta^+$ }}\beta\in \h^*.
\ee

%Denote by $Q_\h\subset \h^*$ the root lattice of $\g$. For every $\nu\in \h^*$, put $[\nu]:=\nu+Q_\h$.

Write $\widetilde \cP_{\Lambda}(G)$ for the set of all triples $\gamma=(H, \xi, \Gamma)$ where $H$ is a Cartan subgroup of $G$, $\xi\in  W(\hha^*, \h^*)$, and
\[
\Gamma:   \Lambda\rightarrow \Irr(H), \quad \nu\mapsto \Gamma_\nu
\]
 is a map
with the following properties:
\begin{itemize}
  \item $\Gamma_{\nu+\beta}=\Gamma_\nu\otimes \xi(\beta)$ for all $\beta\in Q$ and $\nu\in \Lambda$;
  \item $\mathrm d  \Gamma_\nu- \xi(\nu)=\delta(\xi)$ for all $\nu\in  \Lambda$.
  \end{itemize}
Here $\xi(\beta)$ is naturally viewed as a character of $H$ by using the homomorphism $\iota: H\rightarrow H_\C$, and $H_\C$ is the Cartan subgroup of $G_\C$ containing $\iota(H)$.


\def\olgamma{\overline\gamma}
\def\olPsi{\overline\Psi}
\def\olrX{\overline\rX}

The group $G$ obviously acts on $\widetilde \cP_{\Lambda}(G)$, and we define the  set of parameters for $\Coh_{\Lambda}(\CK(G))$
to be \begin{equation}\label{eq:Pram}
  \cP_{\Lambda}(G) := G\backslash  \widetilde \cP_{\Lambda}(G).
\end{equation}



For each $\overline\gamma \in\cP_{\Lambda}(G)$ that is represented by $\gamma=(H, \xi, \Gamma)$, we  have two $\CK(G)$-valued coherent families  $\Psi_{\olgamma}$ and $\olPsi_{\olgamma}$ on $\Lambda$  such that
\begin{equation}\label{eq:psigamma}
  \Psi_{\olgamma}(\nu)=\rX{((\Gamma_\nu,\xi(\nu))}
  \AND
  \olPsi_{\olgamma}(\nu)=\olrX{(\Gamma_\nu,\xi(\nu))}
\end{equation}
for all  regular dominant element $\nu\in \Lambda$.
Here $\rX{((\Gamma_\nu,\xi(\nu))}$ is the standard representation defined in \cite{Vg}*{Notational
  Convention~6.6.3}  and  $\olrX(\Gamma_\nu,\xi(\nu)))$ is its unique irreducible subrepresentation. See
\cite{Vg}*{Theorem~6.5.10 and Theorem 7.2.10}.

By Langlands classification,
$\set{\olPsi_{\olgamma}}_{\olgamma\in \cP_{\Lambda}(G)}$ is a basis of
$\Coh_{\Lambda}(\cK(G))$   (see \cite{Vg}*{Theorem~6.6.14}), and we view $\Coh_\Lambda(\cK(G))$ as a basal representation with this basis. %implies that.  % \cite{Vg}*{??? cc}.
The family  $\set{\Psi_{\olgamma}}_{\olgamma\in \cP_{\Lambda}(G)}$ is also a basis of $\Coh_{\Lambda}(\cK(G))$ (\cite{Vg}*{Proposition~6.6.7}).




The coherent continuation representation  $\Coh_{\Lambda}(\CK(G))$ is independent of $Q$ in the sense of the following lemma.
\begin{lem}%(\cf \cite[Lemma 7.2.6]{Vg})
Let $\Lambda_\g$ be a $Q_\g$ coset in $\Lambda$.  Form the coherent continuation representation $\Coh_{\Lambda_\g}(\CK(G))$ by using the adjoint representation $G\rightarrow \Ad(\g)$. Then the restriction yields a linear isomorphism
\[
  \Coh_{\Lambda}(\CK(G))\xrightarrow{\sim} \Coh_{\Lambda_\g}(\CK(G)).
\]
\end{lem}
\begin{proof}
Note that the restriction yields a bijective map
\[
   \cP_{\Lambda}(G)\rightarrow  \cP_{\Lambda_\g}(G).
\]
This implies the lemma. %Pick a regular element $\lambda\in \Lam$. Extending Lemma \ref{lem21} (with the same proof), we know that
%the evaluation map   (at $\lambda$)
%  \[
%   \mathrm{ev}_{\lambda , \sfS}: \Coh_{[\lambda]_\iota}(\CK_\sfS(G))\rightarrow \CK_{\lambda,\sfS}(G)
%\]
%is also bijective. This clearly implies the lemma.
\end{proof}


We have the cross action of $W_{\Lambda}$ on the set $ \widetilde \cP_{\Lambda}(G)$ (see \cite[Definition 8.3.1]{Vg}):
\[
  w\cross(H, \xi, \Gamma)=\left(H, \xi w^{-1}, (\nu\mapsto \Gamma_\nu\otimes (\xi w^{-1}\nu+\delta(\xi w^{-1})-\xi \nu-\delta(\xi)) )\right)
\]
This commutes with the action of $G$ and thus descends to an action
\[
 W_{\Lambda}\times   \cP_{\Lambda}(G)\rightarrow \cP_{\Lambda}(G), \quad (w, \bar \gamma)\mapsto w\cross \bar \gamma.
 \]


Since $G$ is in the Harish-Chandra's class, the real Weyl group
\[
W_H:=\textrm{(the normalize of $H$ in $G$)}/H
\]
 is identified with a subgroup of the Weyl group $W_\h$ of $\g$ with respect to $\h$.
Write $\t_{\mathrm{im}}$ for the subspace of $\t$ spanned by the set $\{\alpha^\vee\mid \alpha \textrm{ is an imaginary root in $\Delta_\h$}\}$. Then $W_H$ stabilizers $\t_{\mathrm{im}}$, and we define a character
\[
  \mathrm{sgn}_\mathrm{im}: W_H\rightarrow \C^\times, \quad w\mapsto \textrm{(the determinant of the map $w: \t_{\mathrm{im}}\rightarrow \t_{\mathrm{im}}$)}.
\]
This is a quadratic character.

Let $\overline \gamma\in \cP_{\Lambda}(G)$. Write $W_{\overline \gamma}$ for the stabilizer of $\overline \gamma$ in $W_{\Lambda}$ under the cross action.
Pick an element $\gamma=(H, \xi, \Gamma)\in  \widetilde  \cP_{\Lambda}(G)$ that represents $\overline \gamma$.
It is clear that $\xi w \xi^{-1}\in W_H$ for all $w\in W_{\overline \gamma}$. We define a quadratic character
\[
  \mathrm{sgn}_{\overline \gamma}: W_{\overline \gamma}\rightarrow \C^\times, \qquad w\mapsto \mathrm{sgn}_\mathrm{im}(\xi w \xi^{-1}).
\]
This is independent of the choice of $\gamma$.


The coherent continuation representation may be computed by the use of the basis  $\set{\Psi_{\olgamma}}_{\olgamma\in \cP_{\Lam}(G)}$ (see \cite[Section 14]{V4}).
The following result is due to Barbasch-Vogan, in a suitably modified form from \cite{BV.W}*{Proposition~2.4}. As its proof follows the same line as that of \cite{BV.W}*{Proposition~2.4}, we will be content to state the precise result.

%based on the results of \cite{Vg}*{Chapter 8}.

\begin{prop}[{\cf \cite{BV.W}*{Proposition~2.4}}]
  \label{thm:cohHC}
As a representation of $W_{\Lambda}$,
  \[
    \Coh_{\Lambda}(\CK(G)) \cong \bigoplus_{\overline \gamma}
    \Ind_{W_{\overline \gamma}}^{W_{\Lambda}}  \mathrm{sgn}_{\overline \gamma},
  \]
  % \[
  %   \Coh_{[\lambda]}(\CK(G)) \cong \bigoplus_{H} \bigoplus_{\overline \gamma\in \overline \Irr_\Lam(H)}
  %   \Ind_{W_{\overline \gamma}}^{W_\Lam}  \mathrm{sgn}_{\overline \gamma},
  % \]
  where $\overline \gamma$ runs over a representative set of the $W_{\Lambda}$-orbits
  in $\cP_{\Lambda}(G)$ under the cross action.
\end{prop}



\subsection{Some properties of the representation $\Coh_{\Lambda}(\CK(G))$}


For every $\nu\in \Lambda$, by evaluating at $\nu$ we get a linear map
   \[
    \mathrm{ev}_{\nu } \, :\,  \Coh_{\Lambda}(\CK(G)) \rightarrow \Grt_{\nu}(G).
  \]

We present a number of lemmas, all of which may be found in (or easily deduced from) \cite{Vg,V4}.

 \begin{lem}\label{lem21}
Let  $\nu\in \Lambda$. The map  $\mathrm{ev}_{\nu}$ is surjective, and it is bijective when $\nu$ is regular.
     \end{lem}
\begin{proof}
The surjectivity is due to Schmid and Zuckerman, see  \cite{Vg}*{Theorem~7.2.7}. The injectivity (for $\nu$ regular)  is due to Schmid, see \cite{Vg}*{Proposition~7.2.23}.
\end{proof}

For every $\bar \gamma\in \cP_\Lambda(G)$,  we define the $\tau$-invariant of  $ \overline \Psi_{\bar \gamma}$ to be
\[
  \tau_{\bar\gamma}:=\{s\in W(\Lambda) \textrm{ is a simple reflection }\mid   s\cdot \overline \Psi_{\bar \gamma}=-\overline \Psi_{\bar \gamma}\}.
\]



\begin{prop}\label{lemirr} (\cite{V4}*{Corollary~7.3.23})
Let $\bar \gamma\in \cP_\Lambda(G)$ and let $\nu\in \Lambda$ be a dominant element. Then  $\overline \Psi_{\bar \gamma}(\nu)$ is
\[
\left\{
                  \begin{array}{ll}
                   irreducible,\quad &\textrm{if there is no element of   $ \tau_{\bar\gamma}$  that fixes $\nu$};\\
                    0, \quad &\textrm{otherwise}.
                    \end{array}
                    \right.
                    \]
                    Moreover, for all  $\bar \gamma_1, \bar \gamma_2\in \cP_\Lambda(G)$, if  $\overline \Psi_{\bar \gamma_1}(\nu)=\overline \Psi_{\bar \gamma_2}(\nu)\neq 0$ for some   dominant element $\nu\in \Lambda$, then $\bar \gamma_1=\bar \gamma_2$.

              %      If $\overline \Psi_w(\nu)\neq 0$, then the associated variety of   $\mathrm J(w\cdot \nu)$ equals the Zariski closure $\overline \CO_{\Lambda, w}\subset \g^*$ of $\CO_{\Lambda, w}$.



\end{prop}


For every Casselman-Wallach representation $\pi$ of $G$, its complex variety, which is denoted by $\AV_\C(\pi)$, is defined to be the associated variety of $\Ann(\pi)\subset \CU(\g)$.
 Recall that $\sfS$ is a $\Ad(\g)$-stable Zariski closed subset of $\Nil(\g^*)$. Let $\mathrm{Rep}_\sfS(G)$ denote the
category of Casselman-Wallach representations of $G$ whose complex associated
variety is contained in $\sfS$. Write
 $\CK_{\sfS}(G)$ for the Grothendieck group of this category.

\begin{prop}\label{hcass22}
View  $\Coh_{\Lambda}(\CK(G))$ as a basal representation of $W(\Lambda)$ and let $\CC$ be a cell in it. Then among all irreducible representations of $W(\Lambda)$ that occur in
$\Coh_{\Lambda}(\CK(G))(\CC)$, there is a unique one, to be denote by $\sigma_\CC$, that has minimal fake degree. Moreover, for all   dominant element $\nu\in \Lambda$ and  all $\overline \Psi_{\bar \gamma}\in  \CC$ such that  $\overline{\Psi}_{\bar \gamma}(\nu)\neq 0$, $\sigma_{\nu, \mathrm{Ann}(\overline{\Psi}_{\bar \gamma}(\nu))}\cong \sigma_\CC$.


\end{prop}

For every cell $\CC$ in the basal representation $\Coh_{\Lambda}(\CK(G))$ of $W(\Lambda)$, let $\sigma_\CC\in \Irr^{\mathrm{sp}}(W(\Lambda))$ be as in Proposition \ref{hcass22}.
For every $\overline{\Psi}_{\bar \gamma}\in \CC$, put $\CO_{\bar \gamma}:=\CO_{\sigma_\CC}\in  \overline{\Nil}(\g^*)$.

\begin{prop}\label{asshc00}
Let  $\bar \gamma \in \cP_\Lambda(G)$.

\noindent (a)
 For all  dominant element $\nu\in \Lambda$ such that $\overline{\Psi}_{\bar \gamma}(\nu)\neq 0$, the associated variety of $\Ann(\overline \Psi_{\bar \gamma}(\nu))$ equals the Zariski closure $\overline{\CO_{\bar \gamma}}$ of $\CO_{\bar \gamma}$.

\noindent (b)
For all $\lambda\in \Lambda$, $\overline{\Psi}_{\bar \gamma}(\lambda)\in \CK_{\overline{\CO_{\bar \gamma}}}(G)$.


\end{prop}





\begin{prop}\label{hcass222}
Let $\bar \gamma\in \cP_\Lambda(G)$ and let $\nu\in \Lambda$ be a dominant element. Then $\overline \Psi_{\bar \gamma}(\nu)\in \Irr_{\nu, \overline{\CO_\nu}}(G)$ if and only if $\sigma_\CC\cong \left( j_{W_\nu}^{W(\Lambda)} \sgn\right)\otimes \sgn $ and there is no element of   $ \tau_{\bar\gamma}$  that fixes $\nu$.  Here $\CC$ is the cell containing  $\overline \Psi_{\bar \gamma}$ in the basal  representation  $\Coh_{\Lambda}(\CK(G))$ of $W(\Lambda)$.


\end{prop}
\begin{proof}
Fist suppose that $\sigma_\CC\cong \left( j_{W_\nu}^{W(\Lambda)} \sgn\right)\otimes \sgn$ and there is no element of   $ \tau_{\bar\gamma}$  that fixes $\nu$.
Then \Cref{lemirr} implies that $\overline \Psi_{\bar \gamma}(\nu)\in \Irr(G)$ and \Cref{hcass22}  implies that
\[
\sigma_{\nu, \mathrm{Ann}(\overline{\Psi}_{\bar \gamma}(\nu))}\cong  \left( j_{W_\nu}^{W(\Lambda)} \sgn\right)\otimes \sgn.
\]
Thus $ \mathrm{Ann}(\overline{\Psi}_{\bar \gamma}(\nu))=J_\nu$ by \Cref{leftcnu2}, and hence $\overline \Psi_{\bar \gamma}(\nu)\in \Irr_{\nu, \overline{\CO_\nu}}(G)$.

Now we suppose that $\overline \Psi_{\bar \gamma}(\nu)\in \Irr_{\nu, \overline{\CO_\nu}}(G)$. \Cref{lemirr} implies that there is no element of   $ \tau_{\bar\gamma}$  that fixes $\nu$.
As $ \mathrm{Ann}(\overline{\Psi}_{\bar \gamma}(\nu))=J_\nu$, \Cref{leftcnu2} and  \Cref{hcass22} imply that $\sigma_\CC\cong \left( j_{W_\nu}^{W(\Lambda)} \sgn\right)\otimes \sgn$. This complete the proof of the proposition.

\end{proof}



The following conjecture is widely anticipated, although to our knowledge no proof has appeared in the literature. See \cite[page 1055]{V4}.

\begin{conj}\label{conjcell}
View  $\Coh_{\Lambda}(\CK(G))$ as a basal representation of $W(\Lambda)$ and let $\CC$ be a cell in it.  Then the set
\[
 \{ \sigma\in \Irr(W(\Lambda))\mid \sigma\textrm{ occurs in the cell representation $\Coh_{\Lambda}( \CK(G))(\CC)$}\}
\]
is contained in a single double cell in $\Irr(W(\Lambda))$.
\end{conj}

\begin{remark}
We call a double cell in  $\Irr(W_\Lambda)$ a Lusztig double cell. We also call a cell in  $\Coh_{\Lambda}( \CK(G))$ a Harish-Chandra cell, and the associated cell representation a Harish-Chandra cell representation.
\end{remark}

\begin{remark}
We note McGovern's observation (\cite[Page 213]{Mc}) which amounts to the assertion in \Cref{conjcell}. It appears to us that the argument is inadequate as presented.
\end{remark}



\subsection{An embedding of  coherent continuation representations}


For every $\lambda\in\hha^*$, write $\Rep_\lambda(G)$ for the category of Casselman-Wallach representations of $G$ of generalized infinitesimal $\lambda$.
Write $\Rep_{\lambda, \sfS}(G)$ for the full subcategory of $\Rep_\lambda(G)$ consisting of the representations that are also $\Rep_\sfS(G)$.
Denote by $\CK_\lambda(G)$ and  $\CK_{\lambda, \sfS}(G)$ for the Grothedieck groups of  $\Rep_{\lambda}(G)$ and $\Rep_{\lambda, \sfS}(G)$ respectively.


The purpose of this subsection is to prove the following proposition.

 \begin{prop}\label{lem0033}
The representation $\Coh_{\Lambda}(\CK_\sfS(G))$  of $W_{\Lambda}$ is isomorphic to a subrepresentation of $(\Coh_{\Lambda}(\CK_\sfS(\g,\b)))^k$, for some $k\in \BN$.
     \end{prop}



    Let $H$ be a Cartan subgroup of $G$ such that its complexified Lie algebra $\h$ is contained in $\b$. Recall that a $(\g, H)$-module is defined to be a $\g$-module $V$ together with a locally-finite representation of $H$ on it such that
     \begin{itemize}
     \item
        $h\cdot (X\cdot (h^{-1}\cdot u))=(\Ad_h(X))\cdot u$, for all $h\in H, X\in \CU(\g), u\in V$ ($\Ad$ stands for the Adjoint representation);
        \item the differential of the representation of $H$ and the restriction of the representation of $\g$ yields the same representation of $\h$ on $V$.
     \end{itemize}

Let $\Rep(\g,H,\b)$ denote the category of finitely generated $(\g, H)$-modules that  are unions of finite-dimensional $\b$-submodules.
As before, we obviously define the subcategory $\Rep_\sfS(\g,H,\b)$ of $\Rep(\g,H,\b)$,  the Grothendieck groups $\CK(\g,H, \b)$ and $\CK_\sfS(\g,H, \b)$, and the coherent continuation representations $\Coh_{\Lambda}(\CK(\g,H, \b))$ and  $\Coh_{\Lambda}(\CK_\sfS(\g,H, \b))$.

 \begin{lem}\label{lem0022}
The representation $\Coh_{\Lambda}(\CK_\sfS(\g,H, \b))$ of $W_{\Lambda}$ is isomorphic to a subrepresentation of $(\Coh_{\Lambda}(\CK_\sfS(\g, \b)))^k$, for some $k\in \BN$.
     \end{lem}
\begin{proof}


Write $H_\C$ for the Cartan subgroup of $G_\C$ containing $\iota(H)$. Using $\b$,   the lattice $Q$ is identified  with the lattice of holomorphic characters on $H_\C$. By pulling-back through the homomorphism $H\rightarrow H_\C$, we also view $Q$ as a set of characters on $H$.
The tensor product $\beta\otimes \gamma\in \Irr(H)$ is defined for every $\beta\in Q$ and $\gamma\in \Irr(H)$. This yields a free action of $Q$ on the set $\Irr(H)$.

For each $Q$-orbit $\Gamma\subset \Irr(H)$, write $\Rep_{\sfS,\Gamma}(\g,H,\b)$ for the full subcategory of $\Rep_\sfS(\g,H,\b)$ whose objects are  the modules $V$ such that every irreducible subquotient of $V|_H$ ($V$ viewed as a representation of $H$) belongs to $\Gamma$. Write $\CK_{\sfS,\Gamma}(\g,H,\b)$ for the  Grothendieck group of the category $\Rep_{\sfS,\Gamma}(\g,H,\b)$.
Then  we have a decomposition
\[
\CK_\sfS(\g,H,\b)=\bigoplus_{\Gamma\in Q\backslash \Irr(H)} \CK_{\sfS,\Gamma}(\g,H,\b),
\]
of $\mathcal R(\g)$-modules, and
\[
\Coh_{\Lambda}(\CK_\sfS(\g,H, \b))=\bigoplus_{i=1}^k  \Coh_{\Lambda}(\CK_{ \sfS,\Gamma_i}(\g,H,\b)),
\]
for a finite number of orbits $\Gamma_1, \Gamma_2, \cdots, \Gamma_k\in Q\backslash \Irr(H)$ ($k\in \bN$). Thus it remains to show that $ \Coh_{\Lambda}(\CK_{ \sfS,\Gamma}(\g,H,\b))$ is isomorphic to a subrepresentation of  $\Coh_{\Lambda}(\CK_\sfS(\g, \b))$.



For each $\gamma\in \Gamma$, put
\[
  M(\gamma):=\CU(\g)\otimes_{\CU(\b)} \gamma,
\]
which is a module in $\Rep_\Gamma(\g,H,\b)$, where the $\CU(\g)$-action is given by the left multiplication, and the $H$-action is given by
\[
 h\cdot (X\otimes u):=\Ad_h(X)\otimes h \cdot u, \qquad h\in H, \, X\in \CU(\g),\, u\in \gamma.
\]

Note that $\{ M(\gamma)\}_{\gamma\in \Gamma}$ is a basis of the space % (reference?)
\[
\CK_{\Gamma}(\g,H,\b):=\CK_{\Nil(\g^*),\Gamma}(\g,H,\b).
\]
Thus the forgetful functor
\[
   \Rep_{\Gamma}(\g,H,\b):=\Rep_{\Nil(\g^*),\Gamma}(\g,H,\b)\rightarrow  \Rep(\g,\b)
\]
induces an injective linear map
\[
    \CK_{\Gamma}(\g,H,\b)\rightarrow  \CK(\g,\b).
\]
This map is a $\mathcal R(\g)$-module homomorphism, and induces an injective
$\mathcal R(\g)$-module homomorphism
\[
    \CK_{\Gamma,\sfS}(\g,H,\b)\rightarrow  \CK_\sfS(\g,\b).
\]
The above homomorphism induces an embedding
\[
    \Coh_{\Lam}(\CK_{ \sfS,\Gamma}(\g,H,\b))\rightarrow  \Coh_{\Lam}(\CK_\sfS(\g,\b)),
\]
and the proposition follows.
\end{proof}




%\subsection{A result of Casian}

 Similar to the subspace $\Grt_{\lambda, \sfS}(G)\subset \Grt(G)$ ($\lambda\in \hha^*$), we define the subspace $\Grt_{\lambda, \sfS}(\g,H,\b)\subset \Grt(\fgg,H,\b)$ in the obvious way.
Let $\{H_1, H_2, \cdots, H_r\}$ ($r\in \bN^+$) be  a set of representatives of the
conjugacy classes of Cartan subgroups of $G$. For each $i=1,2,\cdots, r$, fix a Borel subalgebra $\b_i$ of $\g$ that contains the complexified Lie algebra of $H_i$.

 \begin{lem}\label{cor:HC.embed}
  % [\cite{Cas}*{Theorem~3.1}]
 There is an injective $\mathcal R(\g, Q)$-module homomorphism
 \[
\gamma_{G}: \Grt(G)\rightarrow  \bigoplus_{i=1}^{r} \Grt(\fgg,H_{i},\b_{i})
 \]
 such that
 \be\label{gammag}
   \gamma_{G}(\Grt_{\lambda, \sfS}(G))\subset  \bigoplus_{i=1}^{r} \Grt_{\lambda, \sfS}(\fgg,H_{i},\b_{i})
 \ee
 for all $\lambda\in \hha^*$ and all $\Ad(\g)$-stable Zariski closed subset $\sfS$ of $\Nil(\g^*)$.

 \end{lem}
\begin{proof}
This follows from the work of Casian (\cite{Cas}). See also {\cite{Mc}}. Since the proposition is not explicitly stated in \cite{Cas}, we briefly recall the argument of Casian for the convenience of the reader.



 Let $\n_i$ denote the nilpotent radical of $[\g,\g]\cap \b_i$ ($i=1,2, \cdots,r$).
 For every $q\in \Z$, let $\gamma_{\n_i}^q$ denote the $q$-th right derived functor of the following left exact functor from the category of $\g$-modules to itself:
 \[
   V\rightarrow \{u\in V\,|\, \n_i^k \cdot v=0\textrm{ for some $k\in \bN^+$}\}.
 \]

Fix a Cartan involution $\theta$ of $G$ and write $K$ for its fixed point group (which is a maximal compact subgroup of $G$).  Without loss of generality we assume that all $H_i$'s are $\theta$-stable.

For every Casselman-Wallach representation $V$ of $G$, write $V_{[K]}$ for the space of $K$-finite vectors in $V$, which is a $(\g,K)$-module of finite length. Then   $\gamma_{\n_i}^q(V_{[K]})$ is naturally a representation in $\Rep(\g, H_i, \b_i)$ (\cite[Corollary 4.9]{Cas}).

We define a linear map
 \[
\gamma_{G}: \Grt(G)\rightarrow  \bigoplus_{i=1}^{r} \Grt(\fgg,H_{i},\b_{i})
 \]
 given by
 \[
   \gamma_{G}(V)= \left\{\sum_{q\in \Z} (-1)^{q} \gamma^{q}_{\n_i}(V_{[K]})\right\}_{i=1,2, \cdots, r}
 \]
for every Casselman-Wallach representation $V$ of $G$. The Osborne conjecture (see \cite[Theorem 3.1]{Cas}) and \cite[Corollary 4.9]{Cas}) implies that the map $\gamma_{G}$ is injective.

Proposition 4.11 of \cite{Cas} implies that the functor $\gamma_{\n_i}^q$ commutes with tensor product with the finite-dimensional representations. Thus $\gamma_{G}$ is a $\mathcal R(\g)$-homomorphism. Finally, \cite[Corollary 4.15]{Cas} implies that $\gamma_{G}$ satisfies the property in \eqref{gammag}.
\end{proof}




\Cref{cor:HC.embed} implies that the representation $\Coh_{\Lambda}(\CK_\sfS(G))$ of $W(\Lambda)$ is isomorphic to a subrepresentation of
$\bigoplus_{i=1}^r \Coh_{\Lambda}(\CK_{\sfS}(\g,H_i,\b_i))$. Together with  \Cref{lem0022}, this implies \Cref{lem0033}.






\subsection{Counting irreducible representations with a bounded complex
  associated variety} %\label{sec12}



\begin{lem}\label{prop:ev00000}
  For all $\nu\in \Lambda$, the map
  \[
   \mathrm{ev}_\nu:  \Coh_{\Lambda}(\CK_\sfS(G))\rightarrow \CK_{\nu, \sfS}(G), \quad \Psi\mapsto \Psi(\nu)
  \]
  descends to a linear isomorphism
  \[
     \Coh_{\Lambda}(\CK_\sfS(G))_{W_{\nu}} \xrightarrow{\sim}\CK_{\nu, \sfS}(G).
  \]
  Here the subscript group $W_{\nu}$ indicates the coinvariant space. \end{lem}
\begin{proof}
  \def\BS{\cB_{\cS}} Without loss of generality we assume that $\nu$ is
  dominant. Put
  \[
    \CB:=\{\Psi_{\bar \gamma}\mid \bar \gamma\in \cP_\Lambda(G), \CO_{\bar \gamma}\subseteq \sfS \},
  \]
  which is a basis of  $\Coh_{\Lambda}(\CK_\sfS(G))$ by \Cref{asshc00}.

  % By \Cref{sur111}, there is a surjective evaluation map
  % \[
  %   \ev{\nu, \sfS} \, :\, \Coh_{[\lambda]}(\CK_\sfS(G)) \longrightarrow \Grt_{\nu, \sfS}(G).
  % \]
  % This map is $W_\lambda$-invariant.
  We have that
  \begin{eqnarray*}
    \ker (\ev{\nu})& = & \Span \Set{\Psi\in \CB \mid
                                        \Psi(\nu)= 0 }\quad (\textrm{by \Cref{lemirr}}) \\
    &=& \Span \Set{\Psi - s\cdot\Psi \mid
                                        \Psi\in  \CB,
                                           s  \text{ is a simple reflection in $W_\nu$}
                                        } \quad (\textrm{by \Cref{lemirr}})\\
                          & \subseteq &\Span\Set{\Psi- w\cdot \Psi | \Psi\in  \Coh_{\Lambda}(\CK), \, w\in W_{\nu}} \\
                          &  \subseteq &  \ker (\ev{\nu}). %\hspace{5em} \text{(by $(\Phi-w\cdot \Phi)(\nu) = \Phi(\nu)-\Phi(w^{-1}\cdot \nu) = 0$)}\\
  \end{eqnarray*}
Therefore
  \[
    \ker (\ev{\nu})=\Span\Set{\Phi- w \cdot\Phi \mid \Phi\in \Coh_{\Lambda}(\CK), \, w\in W_\nu}.
  \]
Together with \Cref{lem21}, this implies the lemma.
\end{proof}



For every $\nu\in \hha^*$, let $\Irr_{\nu,\sfS}(G)$ denote the set of isomorphism classes of irreducible representations in $\Rep_{\nu, \sfS}(G)$.


\begin{thm}[Vogan]\label{count1}
 For every $\nu\in \Lambda$,
  \[
    \sharp(\Irr_{\nu,\sfS}(G)) = [1_{W_{\nu}}:\Coh_{\Lambda}(\CK_\sfS(G))].
  \]

  % \[
  %   \dim {\barmu} = \dim (\cohm)_{W_\lambda} = [\cohm, 1_{W_\lambda}].
  % \]
\end{thm}
\begin{proof}
Since $ \sharp(\Irr_{\nu,\sfS}(G))=\dim \CK_{\nu, \sfS}(G)$, this is implied by \Cref{prop:ev00000}.

\end{proof}


\begin{thm}\label{count2}
For all $\sigma\in \Irr(W(\Lambda))\setminus \Irr_\sfS(W(\Lambda))$,  %$\CO_\sigma\nsubset \sfS$.
  \[
    [\sigma:\Coh_{\Lambda}(\CK_\sfS(G))]=0.
  \]

\end{thm}
\begin{proof}
This is implied by \Cref{lem0022} and \Cref{cohbbs}.
\end{proof}


Theorem \ref{count1} and \ref{count2} imply that  for every $\nu\in \Lambda$,
\begin{equation}\label{leq002}
  \sharp(\Irr_{\nu,\sfS}(G)) = \sum_{\sigma \in \Irr_\sfS(W(\Lambda))} [1_{W_{\nu}}: \sigma]\cdot [\sigma:\Coh_{\Lambda}(\CK_\sfS(G))].
  \end{equation}
  % \[
  %   \dim {\barmu} = \dim (\cohm)_{W_\lambda} = [\cohm, 1_{W_\lambda}].
  % \]
Consequently,
\begin{equation}\label{leq2}  \sharp(\Irr_{\nu,\sfS}(G)) \leq  \sum_{\sigma \in \Irr_\sfS(W(\Lambda))} [1_{W(\Lambda)}: \sigma]\cdot [\sigma:\Coh_{\Lambda}(\CK(G))].
\end{equation}

%Recall the notion of a Harish-Chandra cell representation in $\Coh_{\Lam}(\CK(G))$ (which is a subquotient of $\Coh_{\Lam}(\CK(G))$). See \cite{V4}*{Section 14} or Section \ref{seccell}.

%Also recall the notion of Lusztig double cells in $ \Irr(W_{[\lambda]}$ (see Section \ref{seccell}).
% For every Harish-Chandra cell $C$  in $\Coh_{\Lam}(\CK(G))$, write $\CV(C)$ for the  Harish-Chandra cell representation attached to $C$, which is a subquotient representation of $\Coh_{\Lam}(\CK(G))$.

 \begin{thm}\label{counteq}
   % Under the notation of \Cref{lem:lcell.BV}, we have
   Assume that \Cref{conjcell} holds for $G$. Then for all $\nu\in \Lambda$,
  \begin{equation*}%\label{boundc}
    \sharp(\Irr_{\nu,\sfS}(G)) = \sum_{\sigma \in \Irr_\sfS(W(\Lambda))} [1_{W_\nu}: \sigma]\cdot [\sigma:\Coh_{\Lambda}(\CK(G))].
  \end{equation*}
    \end{thm}

\begin{proof}
View  $\Coh_{\Lambda}(\CK(G))$ as a basal representation of $W(\Lambda)$. Note that
\[
  \Coh_{\Lambda}(\CK(G))\\
   \cong \bigoplus_{\CC \textrm{ is a cell in $\Coh_{\Lambda}(\CK(G))$}} \Coh_{\Lambda}(\CK(G))(\CC),
\]
and
\Cref{asshc00}  and \Cref{hcass22}  imply that
 \[
   \Coh_{\Lambda}(\CK_\sfS(G))\cong  \bigoplus_{\CC \textrm{ is a cell in $\Coh_{\Lambda}(\CK(G))$, $\sigma_\CC\in \Irr^{\mathrm{sp}}_\sfS(W(\Lambda))$}} \Coh_{\Lambda}(\CK(G))(\CC).
 \]
 Here $\sigma_\CC$ is as in \Cref{hcass22}. Therefore under the assumption of the theorem, we have that
\be\label{eqsigma}
[ \sigma: \Coh_{\Lambda}(\CK_\sfS(G))]=[\sigma:\Coh_{\Lambda}(\CK(G))]\quad \textrm{ for all  $\sigma\in \Irr_\sfS(W(\Lambda))$.}
\ee
 Together with \eqref{leq002}, this implies the theorem.
\end{proof}

% the equality always holds. Combining Theorem \ref{count1}, Theorem
% \ref{count2} and \eqref{leq1}, we conclude that
% \begin{equation}\label{leq2}
%   \sharp(\Irr_{\lambda,\sfS}(G)) \leq \sum_{\sigma\in \Irr(W_\Lam), \CO_\sigma\subset \sfS} [1_{W_{\lambda}}: \sigma]\cdot [\sigma: \Coh_{\Lam}(G)].
% \end{equation}

Recall the Lusztig left cell
$\LC_{\nu}$ from \eqref{ll}.




 \begin{cor}\label{counteq}
   % Under the notation of \Cref{lem:lcell.BV}, we have
   For  all $\nu\in \Lambda$,
  \begin{equation*}%\label{boundc}
    \sharp(\Irr_{\nu,\overline{\CO_{\nu}}}(G)) \leq \sum_{\sigma \in \LC_{\nu}}   [\sigma:\Coh_{\Lambda}(\CK(G))],
  \end{equation*}
     and the equality holds if  \Cref{conjcell} holds for $G$.
    \end{cor}
  \begin{proof}
 In view of \eqref{mulone} and \Cref{leftcnu}, this follows from \eqref{leq002} and \Cref{counteq}.
  \end{proof}


\section{Separating good parity and bad parity}

%Let $\star$, $G$, $\check G$, $\g$, $\check \g$, $\check \CO$, $\lambda_{\check \CO}$, $I_{\check \CO}$ and be as in

We continue the notation of the last section. We further assume that $G$ is a classical Lie group as in Sections \ref{sec:defunip}-\ref{secorgp0},
\be\label{gc}
  G_\C :=
  \begin{cases}
   \GL_{n}(\C), & \textrm{if $\star\in \{A^\R, A^\bH\}$};\\
     \GL_{p+q}(\C), & \textrm{if $\star\in \{A, \widetilde A\}$};\\
    \SO_{p+q}(\C), & \textrm{if $\star\in \set{B,D}$};\\
  %  \SO_{n-2l}(\bC) &\textrm{if $\star\in \set{B^{\bC},D^{\bC}}$},\\
    \SO_{2n}(\C), &\textrm{if $\star = D^{*}$};\\
    \Sp_{2n}(\C), &\textrm{if $\star \in \{C, \wtC\}$};\\
    %  \Sp_{2n-2l}(\bC) &\textrm{if $\star \in \set{C^{\bC},\wtC^{\bC}}$},\\
    \Sp_{p+q}(\C), &\textrm{if $\star = C^{*}$},\\
  \end{cases}
\ee
%If $\star=\wtC$, then $G_\C=\check G$, otherwise $G_\C$ and $\check G$
and $\iota: G\rightarrow G_\C$ is the usual complexification homomorphism.

In the rest of the paper we will also freely use  the notation of Sections \ref{sec:defunip}-\ref{secorgp0}.
If $\star\in \{\widetilde A, \widetilde C\}$, we let $\CK'(G)$ denote the Grothendieck group of the category of genuine Casselman-Wallach representations of $G$. Otherwise, put $\CK'(G):=\CK(G)$. Similar notations such as $ \CK'_{\sfS}(G)$, $\CK'_{\nu,\sfS}(G)$, and $\cP'_\Lambda(G)$ will be used without further explanation.


\subsection{Standard representations of  classical groups}

Define a partition
\[
\{\{A^\R\}, \{A^\bH\}, \{A, \wtA\}, \{B,D\}, \{C, \wtC\}, \{C^*\},\{D^*\} \}
\]
of the set of the symbols. Let $[\star]$ denote the equivalence class containing $\star$ of the corresponding equivalence relation.


We define a $[\star]$-space to be a   finite dimensional complex vector space $V$ equipped with a $[\star]$-structure that is defined as inwhat follows.

\noindent {\bf The case when $\star\in \set{A^\R, A^\BH}$.} In this case, a $[\star]$-structure on $V$ is a conjugate linear automorphism $\mathbf j: V\rightarrow V$ such
that
\[
  \mathbf j^2= \begin{cases}
  1,  &  \text{if $\star=A^\R$};\\
  -1,    &  \text{if $\star=A^\bH$}.
    \end{cases}
\]


\noindent {\bf The case when $\star\in \set{A, \wtA}$.} In this case, a $[\star]$-structure on $V$ is a non-degenerate Hermitian form
$\la\,,\,\ra: V\times V\rightarrow \C$ (which is by convention linear on the first variable and conjugate linear on the second variable).


\noindent {\bf The case when $\star\in \set{C, \wtC, C^*}$.} In this case, a $[\star]$-structure on $V$ is a symplectic form
$\la\,,\,\ra: V\times V\rightarrow \C$ together will a conjugate linear automorphism $\mathbf j: V\rightarrow V$ such
that
\[
  \mathbf j^2= \begin{cases}
  1,  &  \text{if $\star\in\{C, \wtC\}$};\\
  -1,    &  \text{if $\star=C^*$}
    \end{cases}
\]
 and
 \[
   \la \mathbf j(u), \mathbf j(v)\ra =\textrm{the complex conjugate of  $\la u, v\ra$ for all $u,v\in V$}.
 \]



\noindent {\bf The case when $\star\in \set{B,D, D^*}$.} In this case, a $[\star]$-structure on $V$ is a triple $(\la\,,\,\ra, \mathbf j, \omega)$ where
 $ \la\,,\,\ra: V\times V\rightarrow \C $ is a non-degenerate symmetric bilinear form,  $\mathbf j: V\rightarrow V$ is a conjugate linear automorphism, and $\omega\in \wedge^{\dim V} V$, subject to the following conditions:
 \begin{itemize}
 \item
 \[
  \mathbf j^2= \begin{cases}
  1,  &  \text{if $\star\in\{B,D\}$};\\
  -1,    &  \text{if $\star=D^*$}
    \end{cases}
\]
\item
 \[
   \la \mathbf j(u), \mathbf j(v)\ra =\textrm{the complex conjugate of  $\la u, v\ra$ for all $u,v\in V$}.
 \]
 \item
$
 \la \omega, \omega\ra=1,$
where $\la\,,\,\ra: \wedge^{\dim V} V\times \wedge^{\dim V} V\rightarrow \C$ is the symmetric bilinear form given by
\be\label{laraw}
  \la u_1\wedge u_2\wedge \dots \wedge u_{\dim V}, v_1\wedge v_2\wedge \dots \wedge v_{\dim V} \ra=\det \left( [\la u_i, v_j \ra]_{i,j=1,2, \dots, \dim V}\right).
\ee
 \item
if $\star=D^*$, then
\be\label{dstaro}
 \omega=\sqrt{-1} u_1\wedge \mathbf j(u_1)\wedge \sqrt{-1} u_2\wedge \mathbf j(u_2)\wedge \dots \wedge \sqrt{-1} u_{\frac{\dim V}{2}} \wedge \mathbf j(u_{\frac{\dim V}{2}}),
 \ee
 for some  vectors $u_1, u_2, \dots, u_{\frac{\dim V}{2}}\in V$,
 \item
if $\star\in \{B,D\}$ and $\dim V=0$, then $\omega=1$ (as an element of $\wedge^{\dim V} V=\C$).
 \end{itemize}

We remark that when  $\star=D^*$,  the existence of $\mathbf j$ implies that $\dim V$ is even, and up to a positive scalar multiplication
the vector
 \be\label{topw}
 \sqrt{-1} u_1\wedge \mathbf j(u_1)\wedge \sqrt{-1} u_2\wedge \mathbf j(u_2)\wedge \dots \wedge \sqrt{-1} u_{\frac{\dim V}{2}} \wedge \mathbf j(u_{\frac{\dim V}{2}})
 \ee
 is independent of the choice of  $u_1, u_2, \dots, u_{\frac{\dim V}{2}}\in V$ that makes \eqref{topw} a nonzero vector.


Suppose that $V$ is a $[\star]$-space, with a $[\star]$-structure as above. Whenever it is defined, write $\mathbf j_V:=\mathbf j$, $\la\,,\,\ra_V:=\la\,,\,\ra$, and $\omega_V:=\omega$.

Isomorphisms of $[\star]$-spaces are obviously defined. Given two $[\star]$-spaces $V_1$ and $V_2$, the product $V_1\times V_2$ is obviously a $[\star]$-space. We remark that when $[\star]=\{B,D\}$ and both $\dim V_1$ and $\dim V_2$ are odd, the switching map $V_1\times V_2\rightarrow V_2\times V_1$ does not preserve the orientation and  $V_1\times V_2$ and $V_2\times V_1$ are not canonically isomorphic to each other as $[\star]$-spaces (although they are isomorphic to each other).





 We define $\mathsf G(V)$ to be the subgroup of $\GL(V)$ fixing the $[\star]$-structure, which is a real classical group.  Let $\mathsf G_\C(V)$ denote the Zariski closure of $\mathsf G(V)$ in $\GL(V)$.
Put
\[
  \mathsf G_\star(V):=\begin{cases}
   \textrm{the $\det^{\frac{1}{2}}$-double cover of $\mathsf G(V)$},  &  \text{if $\star=\wtA$};\\
 \textrm{the metaplectic double cover of $\mathsf G(V)$},  &  \text{if $\star=\wtC$};\\
  \mathsf G(V),    &  \text{otherwise.}
    \end{cases}
\]
Note that  $\mathsf G_\star(V)$ may be a classical group of type $D$ even when $\star=B$, and vice versa.


Now we assume that $G$ is identified with $\mathsf G_\star(V)$.  We call $V$ the standard representation of $G$.
Recall than $n$ is the rank of $\g$ in all cases. Take a  flag
\be\label{flaga}
\{ 0\}= V_0 \subset V_1\subset \dots \subset  V_{n}
\ee
in $V$ such that
\begin{itemize}
 \item
$\dim V_i=i$ for all $i=1,2, \dots, n$.

\item
if $\star\notin \{A^\R, A^\bH, A, \wtA\}$, then $
  V_{n}
$ is totaly isotropic (with respect to the bilinear form $\la\,,\,\ra_V: V\times V\rightarrow \C$);

\item
if $\star\in \{D, D^*\}$, then $V_n$ is $\omega_V$-compatible    in the sense that
\[
  \omega_V= u_1\wedge u_2\wedge \dots \wedge u_{2n}%\wedge u_n^* \wedge \dots \wedge u_{2}\wedge u_1
\]
for some elements $u_1, u_2, \dots, u_{2n}\in V_{2n}$ such that $u_1, u_2, \cdots, u_n\in V_n$ and
\[
\la u_i, u_{j}\ra=\begin{cases}
   1, &\quad \textrm{if $i+j=2n+1$},\\
   0,&  \quad \textrm{if $i+j\neq 2n+1$}\\
  \end{cases}
\]
for all $i,j=1,2, \cdots, 2n$.
% where
%\[
%  V_{j}:=V_{ 2n-j}^\perp:=\{u\in V\mid \la u,v\ra=0\textrm{ for all }v\in V_{2n-j}\}
%  \]
 % for all $ j=n+1, n+2, \dots, 2n$.

\end{itemize}
The flag \eqref{flaga} is unique up to the conjugation by $G_\C$. We remark that in the case when $\star=D^*$  and $n$ is even,  the condition \eqref{dstaro} insures that there exists a
$\mathbf j_V$-stable totally isotropic subspace of $V$ of dimension $n$ that is $\omega_V$-compatible.

The stabilizer of the flag  \eqref{flaga} in $\g$ is a Borel subalgebra of $\g$. Using this Borel subgroup, we get an identification
\be\label{idabs}
   \hha= \prod_{i=1}^{n} \g\l(V_i/  V_{i-1})=\C^n.
\ee
This identification is independent of the choice of the flag \eqref{flaga}. Note that the analytic weight lattice
\be\label{weightl}
  Q_\iota=\Z^n\subset \C^n=(\C^n)^*=\hha^*,
\ee
and the positive roots are
\[
 \Delta^+= \begin{cases}
    \{e_i-e_j\mid 1\leq i<j\leq n\}, &  \text{if $\star \in \{A^\R, A^\BH,  A, \wtA\}$}; \\
   \{e_i\pm e_j\mid 1\leq i<j\leq n\}, &  \text{if $\star \in \{D, D^*\}$}; \\
   \{e_i\pm e_j\mid 1\leq i<j\leq n\} \cup \{e_i\mid 1\leq i\leq n\}, &  \text{if $\star =B$}; \\
   \{e_i\pm e_j\mid 1\leq i<j\leq n\} \cup \{2 e_i\mid 1\leq i\leq n\}, &  \text{if $\star\in \{C, \wtC, C^*\}$}. \\
  \end{cases}
\]
Here $e_1, e_2, \dots, e_n$ is the standard basis of $\C^n$. The Weyl group
\[
    W= \begin{cases}
    \sfS_n, &  \text{if $\star \in \{A^\R, A^\BH,  A, \wtA\}$}; \\
    \sfW_n, &  \text{if $\star\in \{B, C, \wtC, C^*\}$}; \\
      \sfW_n' , &  \text{if $\star \in \{D, D^*\}$}; \\
  \end{cases}
\]
where $\sfS_n\subset \GL_n(\Z)$ is the group of the permutation matrices, $\sfW_n\subset \GL_n(\Z)$ is the subgroup generated by $\sfS_n$ and all the diagonal matrices with diagonal entries $\pm 1$, and $\sfW'_n\subset \GL_n(\Z)$ is the subgroup generated by $\sfS_n$ and all the diagonal matrices with  determinant $1$ and diagonal entries $\pm 1$.




\subsection{Good parity and bad parity in the Langlands dual}

Recall that an integer has  good parity (depends on $\star$ and $n=\rank \check \g$) if it has the same parity as
\[
  \begin{cases}
    n, &  \text{if $\star \in \{A^\R, A^\BH,  A\}$}; \\
    1+ n, &  \text{if $\star = \wtA$}; \\
   1, & \text{if } \star \in \set{C,C^{*},D,D^{*}};\\
 \text{0}, & \text{if } \star \in \set{B,\wtC}.\\
  \end{cases}
\]
Otherwise it has bad parity.


%Recall the  nilpotent $\Ad(\check \g)$-orbit $\ckcO\subset \mathrm{Nil}(\check \g)$.

Let $\check V$ denote the standard representation of $\check \g$ so that $\check \g$ is identified with a Lie subalgebra of $\g\l(\check V)$.
When $\star\in \{A^\R, A^\bH, A, \wtA\}$,  $\g=\g\l(\check V)$. When $\star\in \{B, \wtC\}$,
the space $\check V$ is equipped with a $\check \g$-invariant symplectic form $\la\,,\,\ra_{\check V}: \check V\times \check V\rightarrow \C$.  When $\star\in \{C, C^*, D, D^*\}$, the space $\check V$ is equipped with a $\check \g$-invariant non-degenerate symmetric bilinear form $\la\,,\,\ra_{\check V}: \check V\times \check V\rightarrow \C$ together with an element $\omega_{\check V}\in \wedge^{\dim \check V} \check V$ with $\la \omega_{\check V}, \omega_{\check V}\ra_{\check V}=1$ (see \eqref{laraw}).

As in \eqref{idabs} we also have that $\check \hha=\C^{n}$. Thus   we have an identification
  $ \check \hha=\hha^*$
that identifies $\check \g$ as the Langlands dual (or the metaplectic Langlands dual) of $\g$.

Recall the semisimple element  $\lambda^\circ_{\check \CO}\in \check \g$ which is uniquely determined by $\check \CO$ up to $\Ad(\check \g)$-conjugation. We have a Lie algebra homomorphism
\[
  \eta: \s\l_2(\C)\rightarrow \check \g
\]
such that
\be\label{sl2}
\eta\left(\left[
                                            \begin{array}{cc}
                                            0&1\\
                                            0&0
                                            \end{array}
                                            \right]\right)\in \check \CO\quad \textrm{and}\quad
                                        \eta\left(\left[
                                            \begin{array}{cc}
                                            1&0\\
                                            0&-1
                                            \end{array}
                                            \right]\right)=2\lambda_{\check \CO}^\circ.
\ee

We view $\check V$ as
an $\s\l_2(\C)$-module via
$\eta$.
Put
\[
 \check V_{\mathrm b}:=\textrm{sum of  irreducible $\s\l_2(\C)$-submodules of $\check V$ with bad parity dimension.}
 \]
 Define $ \check V_{\mathrm g}$ similarly by using the good parity  so that
 \be\label{dgb}
 \check V= \check V_{\mathrm b}\oplus  \check V_{\mathrm g}.
 \ee



 Write $\check \g_\mathrm b$ for the Lie subalgebra of $\check \g$ consisting the elements that annihilate $\check V_{\mathrm g}$ and stabilize $\check V_{\mathrm b}$.
 Define $\check \g_\mathrm g$ similarly so that $\check \g_\mathrm b\times \check \g_\mathrm g$ is the stabilizer of the decomposition \eqref{dgb} in $\check \g$.
  We have natural isomorphisms
\[
  (\check \g_\mathrm b,\check \g_\mathrm g)\cong
  \begin{cases}
   (\g\l_{\nnb}(\bC),\g\l_{\nng}(\C)), &\quad  \text{if } \star \in \set{A^\R, A^\BH, A, \wtA}; \\
    (\s\p_{2\nnb}(\bC),\s\p_{2\nng}(\C)), &\quad  \text{if } \star \in \set{B,\wtC}; \\
    (\o_{2\nnb}(\bC),\o_{2\nng+1}(\bC)), & \quad  \text{if } \star \in \set{C,C^{*}}; \\
    (\o_{2\nnb}(\bC),\o_{2\nng}(\bC)), &\quad \text{if } \star \in \set{D,D^{*}},\\
  \end{cases}
\]
 where  $n_{\mathrm b}$ and $n_\mathrm g$ are ranks of $\check \g_\mathrm b$ and $\check \g_\mathrm g$ respectively so that $n_{\mathrm b}+n_{\mathrm g}=n$.
Note that when $\star\in \{B, C,\wtC, C^*, D, D^*\}$, $n_\mathrm b$ agrees with the one defined in \eqref{nb000}.

When $\star\in \{B, C,\wtC, C^*, D, D^*\}$,  \eqref{dgb} is an orthogonal decomposition, and the form $\la\,,\,\ra_{\check V}$ on $\check V$ restricts to a bilinear forms on $\check V_{\mathrm b}$ and $\check V_{\mathrm g}$, to be respectively denoted by $\la\,,\,\ra_{\check V_\mathrm b}$ and $\la\,,\,\ra_{\check V_\mathrm b}$.
 When $\star\in \{C, C^*, D, D^*\}$, we equip on  $\check V_{\mathrm b}$  an element $\omega_{\check V_\mathrm b}\in \wedge^{\dim \check V_\mathrm b} \check V_\mathrm b$
 and equip on $\check V_{\mathrm g}$ an element $\omega_{\check V_\mathrm g}\in \wedge^{\dim \check V_\mathrm g} \check V_\mathrm g$ such that
 \be\label{omega12}
   \la \omega_{\check V_\mathrm b}, \omega_{\check V_\mathrm b}\ra_{\check V_\mathrm b}=\la \omega_{\check V_\mathrm g}, \omega_{\check V_\mathrm g}\ra_{\check V_\mathrm g}=1\quad \textrm{and}\quad  \omega_{\check V_\mathrm b}\wedge  \omega_{\check V_\mathrm g}= \omega_{\check V}.
 \ee
Then  as in \eqref{idabs},   in all cases  we have identifications
   \[
   \check \hha_\mathrm b=\C^{n_\mathrm b},  \quad \check \hha_\mathrm g=\C^{n_\mathrm g}\quad  \textrm{and}\quad \check \hha= \check \hha_\mathrm b\times \check \hha_\mathrm g=\C^{n},
   \]
   where $ \check \hha_\mathrm b$ and $ \check \hha_\mathrm g$ are the abstract Cartan algebras of $\check \g_\mathrm b$ and $\check \g_\mathrm g$, respectively.





 The homomorphism $\eta$ obviously induces Lie algebra homomorphisms
\[
  \eta_\mathrm g: \s\l_2(\C)\rightarrow \check \g_\mathrm g\quad \textrm{and}\quad  \eta_\mathrm b: \s\l_2(\C)\rightarrow \check \g_\mathrm b.
  \]
  By using these two homomorphisms, we obtain $\check \CO_\mathrm g\in \overline{\Nil}(\check \g_\mathrm g)$,  $\lambda_{\CO_\mathrm g}^\circ \in \check \g_\mathrm g$,  $\check \CO_\mathrm b\in \overline{\Nil}(\check \g_\mathrm b)$, and  $\lambda_{\CO_\mathrm b}^\circ \in \check \g_\mathrm b$
  as in \eqref{sl2}.
  Then \[
  \mathbf d_\ckcO = \mathbf d_{\ckcO_{\mathrm g}}\cuprow \mathbf d_{\ckcOb}.
  \]


 Put
  \[
   \star_\mathrm g:=
  \begin{cases}
   A, &  \text{if $\star=\wtA$ and $p+q$ is odd}; \\
   \star, &  \text{otherwise}\\
  \end{cases}
\]
and
 \[
   \star_\mathrm b:=
  \begin{cases}
   \wtA, &  \text{if $\star=A$ and $p+q$ is odd}; \\
     D, &  \text{if $\star=C$}; \\
       D^*, &  \text{if $\star=C^*$}. \\
   \star, &  \text{otherwise}.\\
  \end{cases}
\]
The orbit $\ckcO_{\mathrm g}$ has good parity in the sense that all its nonzero row lengths has good parity with respect to $\star_\mathrm g$ and $\check \g_\mathrm g$, and likewise the orbit $\ckcOb$ has bad parity.

Similar to \eqref{dominant}, we say that an element $\check \nu\in \check \hha$ is dominant if
\be\label{dominant2}
    \la \check \nu, \alpha\ra\notin -\bN^+ \qquad\textrm{for all positive root $\alpha\in \check \hha^*$ of $\check \g$}.
  \ee
The same terminology of course also applies to elements of  $ \check \hha_\mathrm b$ and  $ \check \hha_\mathrm b$.
Let $\lambda_{\check \CO_\mathrm b}\in \check \hha_\mathrm b$ be the unique  dominant element  that represents the $\Ad(\check \g_\mathrm b)$-conjugacy class of $\lambda_{\check \CO_\mathrm b}^\circ$, and likewise let $\lambda_{\check \CO_\mathrm g}\in \check \hha_\mathrm g$ be the unique  dominant element  that represents the $\Ad(\check \g_\mathrm g)$-conjugacy class of $\lambda_{\check \CO_\mathrm g}^\circ$. Put $\lambda_{\check \CO}:=(\lambda_{\check \CO_\mathrm b}, \lambda_{\check \CO_\mathrm g})\in \check \hha$, which is a dominant element that represents the $\Ad(\check \g)$-conjugacy class of $\lambda_{\check \CO}^\circ$.


Define two subsets
\[
\Lambda_\mathrm b:=\lambda_{\check \CO_\mathrm b}+\Z^{n_\mathrm b}\subset \check\C^{n_\mathrm b}= \hha_\mathrm b
\]
 and
 \[
 \Lambda_\mathrm g:=\lambda_{\check \CO_\mathrm b}+\Z^{n_\mathrm g}\subset \C^{n_\mathrm g}=\check \hha_\mathrm b.
 \]
 Then
\[
  \Lambda_\mathrm b:= \begin{cases}
    \Z^{n_\mathrm b}, &  \text{if $\star \in \{A^\R,  A\}$ and $n$ is even, or $\star \in \{A^\BH,  B, \wtC\}$}; \\
    (\frac{1}{2},\frac{1}{2},\dots,  \frac{1}{2})+ \Z^{n_\mathrm b}, &  \text{otherwise}
      \end{cases}
\]
and
\[
  \Lambda_\mathrm g:= \begin{cases}
   (\frac{1}{2},\frac{1}{2},\dots,  \frac{1}{2})+ \Z^{n_\mathrm g}, &  \text{if $\star \in \{A^\R,  A\}$ and $n$ is even, or $\star \in \{A^\BH,  B, \wtC\}$}; \\
     \Z^{n_\mathrm g}, &  \text{otherwise.}
      \end{cases}
\]
Suppose that
\[
  \Lambda= \Lambda_\mathrm b\times  \Lambda_\mathrm g\subset \hha_\mathrm b^*\times  \hha_\mathrm g^*= \C^{n_\mathrm b}\times  \C^{n_\mathrm g}= \C^{n}= \check \hha=\hha^*.
\]
%Then $\Lambda$ contains an element which represents the conjugacy class of $\lambda^\circ_\ckcO$.

\subsection{Separating the abstract Cartan algebra}
When $\star\in \set{ B, C, \wtC,C^*,D, D^*}$, we have defined in Section \ref{secrgp0} the notion of $\check \CO$-relevant parabolic subgroups of $G$. We extend this notion to the other cases: If $\star\in \{A^\R, A^\bH\}$, then a parabolic subgroup of $G$ is said to be  $\check \CO$-relevant if it is the stabilizer group of $\mathbf j$-stable subspace of $V$ of dimension $n_\mathrm b$.  If $\star\in \{A,\wtA\}$, then a parabolic subgroup of $G$ is said to be  $\check \CO$-relevant if it is the stabilizer group of totally isotropic subspace of  $V$ of dimension $\frac{n_\mathrm b}{2}$ (in particular, the existence of such a parabolic subgroup implies that $n_\mathrm b$ is even).



% Note that if $\star\in \{A^\bH, A, C, C^*, D, D^* \}$, then $ n_\mathrm b $ is even.



\begin{prop}\label{leme}
For every $\gamma=(H, \xi, \Gamma)\in \widetilde \cP'_\Lambda(G)$, $H$ is contained in a parabolic subgroups of $G$ that is ${\check \CO}'$-relevant for some ${\check \CO}'\in \overline{\Nil}(\check \g)$ with $\mathbf d_{\check \CO'}=\mathbf d_{\check \CO}$.
  \end{prop}

In the rest of this paper we assume that $G$ has  a parabolic subgroups that is ${\check \CO}'$-relevant for some ${\check \CO}'\in \overline{\Nil}(\check \g)$ with $\mathbf d_{\check \CO'}=\mathbf d_{\check \CO}$.
Otherwise
\Cref{leme} and \Cref{lem21} imply that $\Unip_\ckcO(G)$ is empty.
The assumption is equivalent to saying that
\be\label{nonemp0}
\textrm{ $n_\mathrm b$ is even if $\star=\wtA$},
\ee
 and
\be\label{nonemp00}
  p,q\geq  \begin{cases}
   \frac{n_\mathrm b}{2},  &  \text{if $\star\in\{A, \wtA\}$};\\
  n_\mathrm b,    &  \text{if $\star\in\{B,C^*, D\}$}.
    \end{cases}
  \ee
  In particular, $\star_\mathrm g=\star$ in all cases.
% In the rest of this paper we assume that  these two conditions are  satisfied. Consequently, $\star_\mathrm g=\star$.

Define two classical groups
\[
  (G_\mathrm b, G_{\mathrm g}) =
  \begin{cases}
   ( \GL_{n_{\mathrm b}}(\R), \GL_{n_{\mathrm g}}(\R) ), &\quad  \text{if } \star = A^\R; \\
  (\GL_{\frac{n_{\mathrm b}}{2}}(\BH), \GL_{\frac{n_{\mathrm g}}{2}}(\BH)  ), &\quad  \text{if } \star = A^\bH; \\
   (\oU(\frac{n_{\mathrm b}}{2}, \frac{n_\mathrm b}{2} ), \oU(p_{\mathrm g}, q_\mathrm g)), &\quad  \text{if $ \star = A$ and $p+q$ is even}; \\
    ( \widetilde \oU(\frac{n_{\mathrm b}}{2}, \frac{n_\mathrm b}{2} ),\oU(p_{\mathrm g}, q_\mathrm g)), &\quad  \text{if $ \star = A$ and $p+q$ is odd}; \\
     (\widetilde \oU(\frac{n_{\mathrm b}}{2}, \frac{n_\mathrm b}{2} ), \widetilde \oU(p_{\mathrm g}, q_\mathrm g)), &\quad  \text{if $ \star = \wtA$}; \\
    (\SO(n_{\mathrm b},n_{\mathrm b}+1), \SO(p_{\mathrm g},q_{\mathrm g})), &\quad  \text{if } \star = B; \\
    (\SO(n_{\mathrm b}, n_{\mathrm b}), \Sp_{2n_{\mathrm g}}(\bR)), &\quad  \text{if } \star = C; \\
    (\oO^*(2n_\mathrm b), \Sp(\frac{p_{\mathrm g}}{2},\frac{q_{\mathrm g}}{2})), &\quad  \text{if }  \star = C^{*}; \\
    (\widetilde{\Sp}_{2n_{\mathrm b}}(\bR), \widetilde{\Sp}_{2n_{\mathrm g}}(\bR)), &\quad  \text{if } \star = \wtC; \\
     (\SO(n_{\mathrm b},n_{\mathrm b}), \SO(p_{\mathrm g},q_{\mathrm g}) ), &\quad  \text{if }  \star = D;\\
   (\rO^{*}(2 n_{\mathrm b}), \rO^{*}(2n_{\mathrm g})) , &\quad  \text{if } \star = D^{*}, \\
    \end{cases}
  \]
  where
\[
  (p_\mathrm g, q_{\mathrm g}) =
  \begin{cases}
  (p-\frac{n_\mathrm b}{2}, q-\frac{n_\mathrm b}{2} ),  &\quad  \text{if } \star \in \{A, \wtA\};\\
   (p-n_\mathrm b, q-n_\mathrm b),  &\quad  \text{if } \star \in \{B, C^*, D\}.\\
  \end{cases}
\]
Then $G_\mathrm b$ has type $\star_\mathrm b$ and $G_\mathrm g$ has type $\star$.

As in \eqref{gc}, we have the complex groups $G_{\mathrm b,\C}$ and $G_{\mathrm g,\C}$, and the complexification homomorphisms $\iota_\mathrm b: G_\mathrm b\rightarrow G_{\mathrm b,\C}$ and  the complexification homomorphisms $\iota_\mathrm g: G_\mathrm g\rightarrow G_{\mathrm g,\C}$.



Suppose that $G_\mathrm b$ is identified with $\mathsf G_{\star_\mathrm b}(V_\mathrm b)$ so that $G_{\mathrm b, \C}$ is identified with $\mathsf G_\C(V_\mathrm b)$,  where $V_\mathrm b$ is a  $[\star_\mathrm b]$-space.  Similarly,  suppose that $G_\mathrm g$ is identified with $\mathsf G_{\star_\mathrm g}(V_\mathrm g)$ so that $G_{\mathrm g, \C}$ is identified with $\mathsf G_\C(V_\mathrm g)$,  where $V_\mathrm g$ is a  $[\star]$-space.

%we have a  $[\star_\mathrm g]$-space $V_\mathrm g$ such that $G_{\mathrm g}$ is identified with  $\mathsf G_{\star_\mathrm g}(V_\mathrm g)$ and  $G_{\mathrm g, \C}$ is identified with $\mathsf G_\C(V_\mathrm g)$.






 Respectively write $\hha_\mathrm b$ and $\hha_\mathrm g$ for the abstract Cartan algebras for $\g_\mathrm b$ and $\g_\mathrm g$.  Then we have identifications
 \[
   \hha_\mathrm b^*=\C^{n_\mathrm b}=\check \hha_\mathrm b\qquad\textrm{and}\qquad  \hha_\mathrm g^*=\C^{n_\mathrm g}=\check \hha_\mathrm g.
 \]
 Using these identifications, we view $\check \g_\mathrm b$ as the Langlands dual (or the metaplectic Langlands dual when $\star=\wtC$) of  $\g_\mathrm b$, and view $\check \g_\mathrm g$ as the Langlands dual (or the metaplectic Langlands dual when $\star=\wtC$) of $\g_\mathrm g$.

 We also have identifications
   \[
    \hha_\mathrm b^*\times  \hha_\mathrm g^*= \C^{n_\mathrm b}\times  \C^{n_\mathrm g}= \C^{n}= \hha^*.
 \]
 Under this identification, we have that
 \[
   \Delta_\mathrm b^+=\Delta_\mathrm b\cap \Delta^+\qquad\textrm{and}\qquad  \Delta_\mathrm g^+=\Delta_\mathrm g\cap \Delta^+,
 \]
 where $\Delta_\mathrm b^+\subset \Delta_\mathrm b \subset \hha_\mathrm b^*$ are respectively the positive root system and the root system of $\g_\mathrm b$, and likewise $ \Delta_\mathrm g^+ \subset \Delta_\mathrm g\subset \hha_\mathrm g^*$ are respectively the positive root system and the root system of $\g_\mathrm g$.




 Similar to \eqref{weightl}, we have the analytic weight lattices
\[
  Q_{\iota_\mathrm b}=\Z^{n_\mathrm b}\subset \C^{n_\mathrm b}=(\C^{n_\mathrm b})^*=\hha_\mathrm b^*,
\]
and
\[
  Q_{\iota_\mathrm g}=\Z^{n_\mathrm g}\subset \C^{n_\mathrm g}=(\C^{n_\mathrm g})^*=\hha_\mathrm g^*.
\]
Then $\Lambda_\mathrm b$ is the unique $Q_{\iota_\mathrm b}$-coset in $\hha_\mathrm b^*$ that contains some (and all) elements in $\hha_\mathrm b^*$ that represents the $\Ad(\check \g_\mathrm b)$-conjugacy class of $\lambda_{\check \CO_\mathrm b}^\circ$. Likewise,$\Lambda_\mathrm g$ is the unique $Q_{\iota_\mathrm g}$-coset in $\hha_\mathrm g^*$ that contains some (and all) elements in $\hha_\mathrm g^*$ that represents the $\Ad(\check \g_\mathrm g)$-conjugacy class of $\lambda_{\check \CO_\mathrm g}^\circ$.



 Write $W_\mathrm b\subset \GL(\hha_\mathrm b)$ for the  Weyl group for $\g_\mathrm b$ so that
 \[
   W_\mathrm b= \begin{cases}
    \sfS_{n_\mathrm b}, &  \text{if $\star \in \{A^\R, A^\bH, A, \wtA\} $}; \\
    \sfW_{n_\mathrm b}, &  \text{if $\star \in \{B,\wtC\} $};\\
     \sfW'_{n_\mathrm b}, &  \text{if $\star \in \{C,C^*, D, D^*\} $},\\
      \end{cases}
 \]
 and write $W_\mathrm g\subset \GL(\hha_\mathrm g)$ for the  Weyl group for $\g_\mathrm b$ so that
 \[
    W_\mathrm g= \begin{cases}
    \sfS_{n_\mathrm g}, &  \text{if $\star \in \{A^\R, A^\bH, A, \wtA\} $}; \\
    \sfW_{n_\mathrm g}, &  \text{if $\star \in \{B, C, \wtC, C^*\} $};\\
     \sfW'_{n_\mathrm g}, &  \text{if $\star \in \{ D, D^*\} $}.\\
      \end{cases}
 \]
Let $W'_\mathrm g\subset \GL(\hha_\mathrm g)$ denote the integral Weyl group of $\Lambda_\mathrm g$, namely the Weyl group of the root system
\[
  \{\alpha\in \Delta_\mathrm g\mid \la \check \alpha, \nu\ra\in \Z\textrm{ for some (and all) $\nu\in \Lambda_\mathrm g$}\}.
\]
 Then \[
   W'_\mathrm g= \begin{cases}
     \sfW'_{n_\mathrm g}, &  \text{if $\star =\wtC$};\\
     W_\mathrm g, &\text{otherwise}.
      \end{cases}
 \]
%We remark that $W_\mathrm b$ agrees with the integral  Weyl group of $\Lambda_\mathrm g$ in all cases, and $W_\mathrm g$ agrees with the   Weyl group of $\g_\mathrm g$ except when $\star=\wtC$.
Note that
 \be\label{wl00}
   W(\Lambda)=
    W_{\mathrm b}\times W'_{\mathrm g}\subset W_{\mathrm b}\times W_{\mathrm g}\subset W_\Lambda. \ee
% and
% \[
%   W_\Lambda= \begin{cases}
%    \sfW_n'\cap (\sfW_{n_\mathrm b}\times \sfW_{n_\mathrm g}), &  \text{if $\star \in \{D, D^*\}$}; \\
%    W_\mathrm b\times W_\mathrm g, &  \text{otherwise.}
%      \end{cases}
% \]


Let $\lambda_{\check \CO_\mathrm b}\in \Lambda_\mathrm b$ be the unique  dominant element  that represents the $\Ad(\check \g_\mathrm b)$-conjugacy class of $\lambda_{\check \CO_\mathrm b}^\circ$. Also take a dominant element $\lambda_{\check \CO_\mathrm g}\in \Lambda_\mathrm g$ that represents the $\Ad(\check \g_\mathrm g)$-conjugacy class of $\lambda_{\check \CO_\mathrm g}^\circ$, which is unique unless $\star=\wtC$. Put $\lambda_{\check \CO}:=(\lambda_{\check \CO_\mathrm b}, \lambda_{\check \CO_\mathrm g})\in \Lambda$.

 \begin{lem}\label{dmlam}
 The element $\lambda_{\check \CO}\in \hha^*$ is dominant and represents the conjugacy class of $\lambda_{\check \CO}^\circ$.
 \end{lem}
 \begin{proof}
 Note that for all $\alpha\in \Delta$,
 \[
   \la \lambda_{\check \CO}, \check \alpha\ra\in \Z \qquad \textrm{implies}\qquad \alpha\in \Delta_\mathrm b\sqcup \Delta_\mathrm g.
 \]
 This implies that $ \lambda_{\check \CO}$ is dominant.

  It is easy to see that there exist an element $\lambda'_{\check \CO_\mathrm b}\in \hha^*_\mathrm b$ that represents the conjugacy class of $\lambda_{\check \CO_\mathrm b}^\circ$ and an  element $\lambda_{\check \CO_\mathrm g}\in \hha^*_\mathrm g$ that represents the conjugacy class of $\lambda_{\check \CO_\mathrm g}^\circ$ such that
  $\lambda'_{\check \CO}:=(\lambda'_{\check \CO_\mathrm b}, \lambda'_{\check \CO_\mathrm g})$ represents the conjugacy class of $\lambda_{\check \CO}$.
  Since $\lambda_{\check \CO_\mathrm b}$ is $W_\mathrm b$-conjugate to $\lambda'_{\check \CO_\mathrm b}$, and $\lambda_{\check \CO_\mathrm g}$ is $W_\mathrm g$-conjugate
    to $\lambda'_{\check \CO_\mathrm b}$, $\lambda_{\check \CO}$ is $W$-conjugate to $\lambda'_{\check \CO}$. Thus $\lambda_{\check \CO}\in \hha^*$ also represents the conjugacy class of $\lambda_{\check \CO}^\circ$.

   \end{proof}


 We say that a $[\star]$-space $V_0$ is split if
\begin{itemize}
\item $\star\in \{A^\R, A^\bH, C, \wtC\}$, or

\item $\star\in \{A, \wtA\}$ and there exists a totally isotropic subspace of $V_0$ of dimension $\frac{\dim V_0}{2}$, or
\item
 $\star\in \{B, D, C^*, D^*\}$ and  there exists a $\mathbf j$-stable totally isotropic subspace of $V$ of dimension $\frac{\dim V}{2}$.
 \end{itemize}
Up to isomorphism a split $[\star]$-spaces is  determined by its dimension.


Let $V_{\mathrm b}^\mathrm g$ be a split $[\star]$-space such that
\[
 \dim V_{\mathrm b}^\mathrm g=
 \begin{cases}
   \dim V_{\mathrm b}-1,  &  \text{if $\star=B$};\\
   \dim V_{\mathrm b},    &  \text{otherwise},
    \end{cases}
 \]
For simplicity, write $G_\mathrm b^\mathrm g:=\mathsf G_{\star_\mathrm g}(V_\mathrm b^\mathrm g)$. Note that
\be\label{isostar}
  V\cong V_\mathrm b^\mathrm g \times V_\mathrm g
\ee
as $[\star]$-spaces.



\subsection{Matching relevant Cartan subgroups}




By a $\star$-representation of a Lie group $E$, we mean a $[\star]$-space $V_1$ together with a Lie group homomorphism $E\rightarrow \mathsf G_\star(V_1)$. An isomorphism of $\star$-representation is defined to be a $[\star]$-sapce  isomorphism $f: V_1\rightarrow V_2$ of  $\star$-representations that makes the diagram
\[
\begin{tikzcd}[column sep={4cm,between origins}]
      & E
      \ar[dl," "']\ar[dr,"  "]&\\
      \mathsf G_\star(V_1)\ar[rr,"\textrm{the isomorphism induced by $f$}"]& &\mathsf G_\star(V_1).\\
    \end{tikzcd}
\]
commutes. The product $V_1\times V_2$ is obviously a $[\star]$-representation of $E$ whenever $V_1$ and $ V_2$ is are $[\star]$-representation of $E$.
Note that a $B$-representation is the same as a $D$-representation.

We say that a Cartan subgroup $H_\mathrm b$ of $G_\mathrm b$ is relevant if
\begin{itemize}
\item
$\star_\mathrm b\in \{A^\R, A^\bH\}$, or
\item $\star\in \{A, \wtA\}$ and $H_\mathrm b$ stabilizes a totally isotropic subspace of $V$ of dimension $\frac{n_\mathrm b}{2}$, or
\item
 $\star\in \{B,C,\wtC, D, C^*, D^*\}$ and $H_\mathrm b$ stabilizes  a $\mathbf j_{V_\mathrm b}$-stable totally isotropic subspace of $V$ of dimension $n_\mathrm b$.
\end{itemize}

Suppose that
 $H_\mathrm b$ be a relevant Cartan subgroup of $G_\mathrm b$. Then $V_\mathrm b$ is naturally a $\star_\mathrm b$-representation of $H_\mathrm b$.

 \begin{lem}\label{match1}
Suppose that $\star=B$. Then  there is a $\star$-representation of $H_\mathrm b$  on the $\star$-space $V_\mathrm b^\mathrm g$ such that for some trivial one-dimensional $\star$-representation $V_1$ of $H_\mathrm b$,
\[
  V_\mathrm b\cong V_1\times V_\mathrm b^\mathrm g
\]
as $\star$-representations of $H_\mathrm b$. Moreover, such a $\star$-representation of $H_\mathrm b$ is unique up to isomorphism.

 \end{lem}


 \begin{lem}\label{match2}
Suppose that $\star\in \{C, C^*\}$. Then  there is a $\star$-representation of $H_\mathrm b$  on the $\star$-space $V_\mathrm b^\mathrm g$ such that
\[
 V_\mathrm b^\mathrm g\cong V_\mathrm b
\]
as representations of $H_\mathrm b$. Moreover, such a $\star$-representation of $H_\mathrm b$ is unique up to isomorphism.

 \end{lem}


  The following lemma is obvious.
 \begin{lem}\label{match3}
Suppose that $\star\notin \{B, C, C^*\}$. Then  there is a $\star$-representation of $H_\mathrm b$  on the $\star$-space $V_\mathrm b^\mathrm g$ such that
\[
 V_\mathrm b^\mathrm g\cong V_\mathrm b
\]
as $\star$-representations of $H_\mathrm b$. Moreover, such a $\star$-representation of $H_\mathrm b$ is unique up to isomorphism.
\qed
 \end{lem}


We view  $V_\mathrm b^\mathrm g$  as a $\star$-representation of  $H_\mathrm b$ as in Lemmas \ref{match1}-\ref{match3}.



  Let $H_\mathrm g$ be a Cartan subgroup of $G_\mathrm g$ so that $V_\mathrm g$ is naturally an $H_\mathrm g$-representation.
Then $V_\mathrm b^\mathrm g\times V_\mathrm g$ is naturally a $\star$-representation of $H_\mathrm b\times H_\mathrm g$. The isomorphism \eqref{isostar} yields a
 a Lie group homomorphism
\be\label{matchmap}
H_\mathrm b\times H_\mathrm g\rightarrow G.
\ee
Write $H$ for the image of the homomorphism \eqref{matchmap}, and let $\zeta: H_\mathrm b\times H_\mathrm g\rightarrow H$ denote the homomorphism induced by \eqref{matchmap}.
Note that $H$ is a Cartan subgroup of $G$. We call the pair $(H, \zeta)$ a matching of $H_\mathrm b\times H_\mathrm g$, which is uniquely defined up to conjugation by $G$.


\subsection{Matching the parameters}
Given a parameter $\overline\gamma_\mathrm b \in\cP'_{\Lambda_\mathrm b}(G_\mathrm b)$ that is represented by $(H_\mathrm b, \xi_\mathrm b, \Gamma_\mathrm b)$, and  a parameter $\overline\gamma_\mathrm g \in\cP'_{\Lambda_\mathrm g}(G_\mathrm g)$ that is represented by $(H_\mathrm g, \xi_\mathrm g, \Gamma_\mathrm g)$, we  define  a parameter $\varphi(\overline\gamma_\mathrm b, \overline \gamma_\mathrm g) \in\cP'_{\Lambda}(G)$ as in what follows. Note that \Cref{leme} implies that $H_\mathrm b$ is  relevant. Take a matching $(H, \zeta)$ of $H_\mathrm b\times H_\mathrm g$. Let $\xi$ be the composition of
\[
    \hha^*=\hha_\mathrm b^*\times  \hha_\mathrm g^*\xrightarrow{\xi_\mathrm b\times \xi_\mathrm g} \h_\mathrm b^*\times \h_\mathrm g^*\xrightarrow{\textrm{the transpose inverse of the complexified differential of $\zeta$}} \h^*.
\]
Let $\Gamma: \Lambda\rightarrow \Irr'(G)$ to be the map
\[
  \nu=(\nu_\mathrm b, \nu_\mathrm g)\mapsto \Gamma_\mathrm b(\nu_\mathrm b)\otimes \Gamma_\mathrm g(\nu_\mathrm g),
\]
where  $\Gamma_\mathrm b(\nu_\mathrm b)\otimes \Gamma_\mathrm g(\nu_\mathrm g)$ (which is originally defined as an irreducible representation of $H_\mathrm b\times H_\mathrm g$) is viewed as a representation of $H$ via the descent through the homomorphism $\xi: H_\mathrm b\times H_\mathrm g\rightarrow H$.




\begin{lem}
The triple $(H, \xi, \Gamma)$ defined above is an element of $\widetilde \cP'_{\Lambda}(G)$.
\end{lem}
\begin{proof}
Note that
every imaginary root of $G$ with respects to $H$ is either an  imaginary roots of $G_\mathrm b$ with respect to $H_\mathrm b$ or an  imaginary roots of $G_\mathrm g$ with respect to $H_\mathrm g$ (here $\h$ is identified with $\h_\mathrm b\times \h_\mathrm g$ via $\xi$). This  implies that $\delta(\xi)=\delta(\xi_\mathrm b)+\delta(\xi_\mathrm g)$, and the lemme then easily follows.


\end{proof}





Now we define  $\varphi(\overline\gamma_\mathrm b, \overline \gamma_\mathrm g) \in\cP'_{\Lambda}(G)$ to be the $G$-orbit of the triple  $(H, \xi, \Gamma)$ defined above.
This is independent of the choices of the representatives $(H_\mathrm b, \xi_\mathrm b, \Gamma_\mathrm b)$, $(H_\mathrm b, \xi_\mathrm b, \Gamma_\mathrm b)$, and the matching $(H, \zeta)$. It is easy to see that the map
\be\label{match0}
  \varphi: \cP'_{\Lambda_\mathrm b}(G_\mathrm b)\times \cP'_{\Lambda_\mathrm g}(G_\mathrm g)\rightarrow \cP'_{\Lambda}(G)
\ee
is $W_\mathrm b\times W_\mathrm g$-equivariant under the cross actions.






 \begin{prop}\label{para}
 If either $\star=C^*$ and $n_\mathrm b>0$, or $\star=D^*$ and $n_\mathrm b, n_\mathrm g>0$, then the map \eqref{match0} is injective and
 \[
   \cP'_{\Lambda}(G)=\varphi(\cP'_{\Lambda_\mathrm b}(G_\mathrm b)\times \cP'_{\Lambda_\mathrm g}(G_\mathrm g))\sqcup w\cdot\left(\varphi(\cP'_{\Lambda_\mathrm b}(G_\mathrm b)\times \cP'_{\Lambda_\mathrm g}(G_\mathrm g))\right)
 \]
 for all $w\in W_\Lambda\setminus (W_\mathrm b\times W_\mathrm g)$. The map \eqref{match0} is bijective in all other cases.
 \end{prop}

Define a linear map
\be\label{match1}
  \varphi: \Coh_{\Lambda_\mathrm b}(\CK'(G_\mathrm b))\otimes \Coh'_{\Lambda_\mathrm g}(\CK'(G_\mathrm g))\rightarrow \Coh_{\Lambda}(\CK'(G))
\ee
such that
\[
 \varphi(\Psi_{\bar \gamma_\mathrm b}\otimes \Psi_{\bar \gamma_\mathrm g})=\Psi_{\varphi(\bar \gamma_\mathrm b, \bar \gamma_\mathrm g)}
\]
for all $\overline\gamma_\mathrm b \in\cP'_{\Lambda_\mathrm b}(G_\mathrm b)$ and $\overline\gamma_\mathrm g \in\cP'_{\Lambda_\mathrm g}(G_\mathrm g)$.


\begin{prop}\label{matchth}
The linear map \eqref{match1} is $W_\mathrm b\times W_\mathrm g$-equivariant. Moreover
\[
 \varphi(\overline \Psi_{\bar \gamma_\mathrm b}\otimes \overline \Psi_{\bar \gamma_\mathrm g})=\overline \Psi_{\varphi(\bar \gamma_\mathrm b, \bar \gamma_\mathrm g)}
\]
for all $\overline\gamma_\mathrm b \in\cP'_{\Lambda_\mathrm b}(G_\mathrm b)$ and $\overline\gamma_\mathrm g \in\cP'_{\Lambda_\mathrm g}(G_\mathrm g)$.

\end{prop}




\begin{cor}
 If $\star\in \{C^*, D^*\}$, then
\[
 \Coh_{\Lambda}(\CK'(G))\cong \Ind^{W_\Lambda}_{W_\mathrm b\times W_\mathrm g} (\Coh_{\Lambda_\mathrm b}(\CK'(G_\mathrm b))\otimes \Coh_{\Lambda_\mathrm g}(\CK'(G_\mathrm g)))
 \]
 as representations of $W_\Lambda$. In all the other cases,
\[
 \Coh_{\Lambda}(\CK'(G))\cong \Coh_{\Lambda_\mathrm b}(\CK'(G_\mathrm b))\otimes \Coh_{\Lambda_\mathrm g}(\CK'(G_\mathrm g))
 \]
 as representations of $W_\mathrm b\times W_\mathrm g$.

\end{cor}



% \subsection{Irreducible representations and special unipotent representations}

\begin{prop}\label{propKL33}
Let $\nu=(\nu_\mathrm b, \nu_\mathrm g)\in \Lambda_\mathrm b\times \Lambda_\mathrm g=\Lambda$. Then there is a unique linear map $\varphi_\nu: \CK'_{\nu_\mathrm b}(G_\mathrm b)\otimes \CK'_{\nu_\mathrm g}(G_\mathrm g)\rightarrow \CK'_{\nu}(G)$ that makes the diagram
\[
 \begin{CD}
          \Coh_{ \Lambda_\mathrm b}(\CK'(G_\mathrm b))\otimes  \Coh_{ \Lambda_\mathrm g}(\CK'(G_\mathrm g))
                  @>   \varphi  >>  \Coh_{ \Lambda}(\CK'(G))\\
            @V   \mathrm{ev}_{\nu_\mathrm b} \otimes  \mathrm{ev}_{\nu_\mathrm g}  VV         @ VV  \mathrm{ev}_{\nu} V \\
      \CK'_{\nu_\mathrm b}(G_\mathrm b)\otimes \CK'_{\nu_\mathrm g}(G_\mathrm g) @> \varphi_\nu >>  \CK'_{\nu}(G) \\
  \end{CD}
\]
commutes, where $\mathrm ev$ indicates the evaluation maps. Moreover, $\varphi_\nu$ is injective,  and
\[
\varphi_\nu(\Irr'_{\nu_\mathrm b}(G_\mathrm b)\times  \Irr'_{\nu_\mathrm g}(G_\mathrm g))\subset \Irr'_{\nu}(G).
\]
\end{prop}
\begin{proof}
The first assertion direct follows from \Cref{prop:ev00000}, since the map $\varphi$ is $W(\lambda)=W_\mathrm b\times W'_\mathrm g$-equivariant. Since $\varphi$ is injective,  \Cref{prop:ev00000} also implies that $\varphi_\nu$ is injective.


For the proof of the second assertion, we assume without loss of generality that $\nu$ is dominant. Let $\pi_\mathrm b\in \Irr'_{\nu_\mathrm b}(G_\mathrm b)$ and $\pi_\mathrm g\in \Irr'_{\nu_\mathrm g}(G_\mathrm g)$. Pick  basal elements
\[
  \Psi_\mathrm b \in \Coh_{ \Lambda_\mathrm b}(\CK'(G_\mathrm b)) \quad \textrm{and}\quad \Psi_\mathrm g\in  \Coh_{ \Lambda_\mathrm g}(\CK'(G_\mathrm g))
   \]
 such that $\Psi_\mathrm b(\nu_\mathrm b)=\pi_\mathrm b$ and $\Psi_\mathrm g(\nu_\mathrm g)=\pi_\mathrm g$.
 Then
\[
  \varphi_\nu(\pi_\mathrm b\otimes \pi_\mathrm g)=\mathrm{ev}_{\nu}(\varphi(\Psi_\mathrm b\otimes \Psi_\mathrm g)).
\]
\Cref{matchth} implies that $\varphi(\Psi_\mathrm b\otimes \Psi_\mathrm g)$ is a basel element of  $\Coh_{\Lambda}(\CK'(G))$. Thus by \Cref{lemirr}, $\mathrm{ev}_{\nu}(\varphi(\Psi_\mathrm b\otimes \Psi_\mathrm g))$ is either irreducible or zero. But it is nonzero since $\varphi_\nu$ is injective. This proves the last assertion of the proposition.


\end{proof}



\section{Special unipotent representations in type A}\label{sec:GL}




%Recall the Barbasch-Vogan duality map  (\cite[Appendix]{BVUni} and \cite[Section 1]{BMSZ1})
%\be\label{bvd0}
% \begin{array}{rcl}
% \dBV: \overline{\Nil}(\check \g)&\rightarrow &\overline{\Nil}(\g^*),\\
% \check \CO&\mapsto &\textrm{the unique open $\Inn(\g)$-orbit in the associated variety of $I_{\star, \check \CO}$},
 % \end{array}
% \ee
%where $\overline{\Nil}(\check \g)$ denotes the set of
 %$\Ad(\check \g)$-orbits in ${\Nil}(\check \g)$, and $\overline{\Nil}(\g^*)$ denotes  the set of
 %$\Ad(\g)$-orbits in ${\Nil}( \g^*)$.

As an obvious variant of \Cref{counteq} for $\lambda=\lambda_{\check \CO}$, we have  that
 \begin{equation}\label{boundc22}
     \sharp(\Unip_{\check \CO}(G)) =\sum_{\sigma\in \LC_{\lambda_{\check \CO}}} [\sigma: \Coh_{\Lambda}(\CK_{\overline{\CO}}'(G))]  \leq \sum_{\sigma\in \LC_{\lambda_{\check \CO}}} [\sigma: \Coh_{\Lambda}(\CK'(G))],
   \end{equation}
   and the equality holds if the analogous condition for a Harish-Chandra cell in $\Coh_{\Lam}(\CK'(G))$ holds.
Recall that $\CO:=\dBV(\check \CO)$ and $\overline{\CO}$ denotes its Zariski closure.



\subsection{Some Weyl group representations}


The group $\sfS_n$ is  identified with the permutation group of the set $\{1,2, \dots, n\}$, and
 $\sfW_n$ is identified with $\sfS_n \ltimes \set{\pm 1}^n$.

 Define a quadratic character
\be\label{defep}
  \varepsilon: \sfW_n\rightarrow \{\pm 1\}, \quad (s,(x_{1}, x_{2}, \cdots, x_{n}))\mapsto x_{1}x_{2}\cdots x_{n}.
\ee
Then $\sfW_n'$ is the kernel of this character. As always, $\sgn$ denotes the sign character (of an appropriate Weyl group).
Since $\sfS_{n}$ is a quotient of $\sfW_{n}$, we may inflate the sign character of $\sfS_{n}$ to obtain a character of $\sfW_{n}$, to be denoted by $\bsgn$. Then we have that
\[\varepsilon = \sgn \otimes \bsgn.\]


%Denote by $\epsilon$ the unique non-trivial character of $\sfW_n$ which is trivial on $\sfS_n$, which is given by
%\be\label{defep}
  %(s,(x_{1}, x_{2}, \cdots, x_{n}))\mapsto x_{1}x_{2}\cdots x_{n}.
%\ee


The group $\sfW_n$ is naturally embedded in $\sfS_{2n}$ via the homomorphism determined by
\be\label{wn1}
(i, i+1)\in \sfS_n\mapsto (2 i-1, 2i+1)(2i, 2i+2), \qquad (1\leq i\leq n-1),
\ee
and
       \be\label{wn2}
       (1,\cdots,1, \underbrace{-1}_{j\text{-th
        term}}, 1, \cdots, 1)\in \set{\pm 1}^n  \mapsto (2j-1, 2j), \qquad (1\leq j\leq n).
        \ee
Here $(i, i+1)$, $(2 i-1, 2i+1)$, etc., indicate the involutions in the permutation groups.  Note that $\varepsilon $ is also the restriction of $\sgn$ of $\sfS_{2n}$ to $\sfW_n$.


 Let $\YD_{n}$ be the set of Young diagrams of total size $n$.
We identify $\YD_{n}$ with the set $\overline{\Nil}(\g):= \GL_{n}(\bC)\backslash \Nil(\g\l_n(\C))$ of complex nilpotent orbits and
also with the set  $\Irr(\sfS_{n})$ via the Springer
correspondence (see \cite{Carter}*{11.4}).
More specifically, for $ \cO' \in \overline \Nil( \g)=\YD_{n}$, the Springer
correspondence is given by Macdonald's construction for $\sfS_{n}$ via $j$-induction:
\[
  \Spr( \cO') = j_{\prod_{j}\sfS_{\bfcc_{j}(\cO')}}^{\sfS_{n}} \sgn.
\]
%Recall that $\bfcc_{j}(\cO)$ is the $j$-th column length of $\cO$, and $\sgn$ denotes the sign character.

\trivial[h]{
  %\begin{minipage}{\textwidth}
  Here is the diagram to remember
  \[
    \begin{tikzcd}[ampersand replacement=\&]
     \Nil(\GL_{n}) \ar[rr,"\Spr"]\& \& \Irr(\sfS_{n})\\
      \&\YD_{n} \ar[lu,"\text{Jordan canonical forms}"] \ar[ru,"\text{Macdonald's $j$-induction}"']\&
    \end{tikzcd}
  \]
  % \end{minipage}
}

%For $\ckcO\in \Nil(\GL_{n}(\bC))$, we set % its Barbasch-Vogan dual is
%\[
 % \cO:=\dBV(\ckcO) = \ckcO^{t} \in \Nil(\GL_{n}(\bC)).
%\]


Suppose that $\star\in  \{A^\R, A^\BH, A, \widetilde A\}$ in the rest of this section.
Recall that  $W(\Lambda)  = \sfS_{n_{\mathrm b}}\times \sfS_{n_{\mathrm g}}$. It is easy to verify that
\begin{equation}\label{eq:WA}
  \begin{split}
      W_{\lamck} & = \prod_{i\in \bN^{+}}\sfS_{\bfcc_{i}(\ckcO_{\mathrm b})}\times \prod_{i\in \bN^{+}}\sfS_{\bfcc_{i}(\ckcO_{\mathrm g})},\\
    \LC_{\lamck} & = \set{\tau_{\lamck}}, \quad \textrm{where }  \tau_{\lamck} := (j_{W_{\lamck}}^{W(\Lambda)}\sgn )\otimes \sgn =  \ckcOb^{t}\otimes \ckcOg^{t}.
    %, \AND \\
    %\LC_{\lamck} & = \LRC_{\lamck}= \set{\tau_{\lamck}}, \AND \\
   % \wttau_{\lamck} & := j_{W_{[\lamck]}}^{W}\tau_{\lamck} = \ckcO^{t}.
  \end{split}
\end{equation}
Here and as before, a superscript ``$t$" indicates the transpose of a Young diagram.

  The proof of the following Propositions \ref{count000}-\ref{count002}  follows from Theorem \ref{thm:cohHC} by direct computation. We  omit the details.

   \begin{prop} \label{count000}
  %Suppose $\star=A^\R$ so that $G = \GL_{n}(\bR)$.
 Suppose that $\star=A^\R$.  For all $l\in \BN$, put
  \[
    \cC_l := \bigoplus_{\substack{t,c,d\in \bN \\2t+c+d=l}} \Ind_{\sfW_t\times \sfS_c\times \sfS_d}^{\sfS_{l}} \varepsilon \otimes 1\otimes 1.
  \]
  Then
  \[
    \Coh_{\Lambda_\mathrm b}(\CK'(G_\mathrm b)) \cong \cC_{n_{\mathrm b}}\quad\textrm{and}\quad   \Coh_{\Lambda_\mathrm g}(\CK'(G_\mathrm g)) \cong \cC_{n_{\mathrm g}}.
  \]
 \end{prop}


\begin{prop}\label{count001}
 Suppose that $\star=A^\BH$.    For all even number $l\in \BN$, put
  \[
  \cC_{l}: =
    \Ind_{\sfW_{\frac{l}{2}}}^{\sfS_{l}}\epsilon,
  \]
  Then \[
    \Coh_{\Lambda_\mathrm b}(\CK'(G_\mathrm b)) \cong \cC_{n_{\mathrm b}}\quad\textrm{and}\quad   \Coh_{\Lambda_\mathrm g}(\CK'(G_\mathrm g)) \cong \cC_{n_{\mathrm g}}.
   \]

\end{prop}



\begin{prop}\label{count002}
Suppose that $\star\in \{A, \wtA\}$.   Then
    \[
       \Coh_{\Lambda_\mathrm b}(\CK'(G_\mathrm b)) \cong  \Ind_{\sfW_{\frac{n_{\mathrm b}}{2}}}^{\sfS_{n_{\mathrm b}}} 1
    \]
    and
    \[
       \Coh_{\Lambda_\mathrm g}(\CK'(G_\mathrm g)) \cong  \bigoplus_{\substack{t,s,r\in \bN\\t+r=p_\mathrm g, t+s = q_\mathrm g}}
    \Ind_{\sfW_{t}\times \sfS_s\times \sfS_r}^{\sfS_{n_\mathrm g}}
 1\otimes \sgn \otimes \sgn.
    \]
    \end{prop}




% Fix $\ckcO\in \Nil(\ckG) = \YD_{n}$ and set
% \[
%   \cO :=\dBV(\ckcO) = \ckcO^{t} \AND \wttau := \Spr^{-1}(\cO).
% \]
% Here $\wttau$ has the partition $\cO$.

% In all the cases, $\ckG = \GL_{n}(\bC)$ and $n$ is even if
% $G = \GL_{\frac{n}{2}}(\bH)$.
% The nilpotent orbit $\ckcO$ is parameterized by Young diagrams.


% \subsection{Definition of special unipotent representation }

% In this section, we let $G = \GL_{n}(\bC)$ and $\fgg:=\Lie(G) = \fgl_{n}(\bC)$.
% Then $\fggC:= \fgg \otimes_{\bR}\bC$ is naturally identified with
% $\fgg\times \fgg$ and $\cU(\fggC)=\cU(\fgg)\otimes \cU(\fgg)$.


% We make the following definition
% \begin{defn}
% For $\ckcO\in \Nil(\GL_{n}(\bC))$, let $\cIck$ be the maximal primitive ideal in
% $\cU(\fgg)$ with infinitesimal character $\lamck$.
% We call an irreducible admissible representation of $G$ is a special unipotent
% representation attached to $\ckcO$
% \end{defn}

% % For $\ckcO\in \Nil(\GL_{n}(\bC))$, let
% % \[
% % \lamckC := (\lambck,\lamck)\in \fhh^{*}\times \fhh^{*}.
% % \]
% % Via Harish-Chandra homomorphism, $\lamckC$ determine an infinitesimal character
% % of $\cZ(\fggC)$.

% % see \cite{BVUni}*{Introduction}.

% Following \cite{BVUni}*{Defintion~1.17}, we make the following definition of
% unipotent representation:
% \begin{defn}
% \end{defn}

% Since $\Lie(G)\otimes_{\bR}\bC$ is naturally

\trivial[h]{\subsection{Special unipotent representations of $G = \GL_n(\bC)$}
\def\lamckC{\lambda^\bC_{\ckcO}}
\def\fggR{\fgg_{\bR}}
\def\fggC{\fgg_{\bC}}
\def\cIck{\cI_{\ckcO}}
\def\WC{W^{\bC}}

The classification of special unipotent representations of $\GL_n(\bC)$ with integral infinitesimal character is a special case of the results of \cite{BVUni}. The classification in general is also well known to the experts. %, see \cite{V.GL}.

\begin{thm} Let $\star =A^{\bC}$ so that $G =\GL_n(\bC)$. Suppose $\ckcO\in \YD_{n} = \Nil(\check \g)$ and $\ckcO$ has $k$ rows.
Let
\[
 \pi_{\ckcO} := 1_{\bfrr_1(\ckcO)}\times  1_{\bfrr_2(\ckcO)}\times \cdots
 \times 1_{\bfrr_k(\ckcO)}
\]
be the normalized parabolic induction where $1_{c}$ denotes the trivial
representations of $\GL_c(\bC)$.
Then
\[
  \Unip_{\ckcO}(G) = \Set{\pi_{\ckcO}}.
\]
\end{thm}
\begin{proof}
  The irreducibility of $\pi_{\ckcO}$ is well known, see
  \cite{V.GL}. It is clearly an element in $\Unip_{\ckcO}(G)$ by the construction.

  %The computation of complex groups doubles that of real groups.
  We let $\WC$ denote the Weyl group of the complexification of $G$. As in Section \ref{sec:intro}, let $\lamckC:=(\lamck,\overline{\lamck})$, which is an element in the dual $\ahh^{*}$ of the  universal Cartan algebra.
  Recall the notation in \cref{eq:WA}.
  We have the following facts:
  \[
    \begin{split}
      \WC  &\cong  \sfS_{n}\times \sfS_{n}\\
      \WC_{[\lamckC]} &= W_{[\lamck]}\times W_{[\lamck]}\\
      \LC_{\lamckC} &= \set{\tau_{\lamck}\boxtimes \tau_{\lamck}}\\
    \end{split}
  \]
  As a representation of $\WC_{[\lamckC]}$, $\Coh_{[\lamckC]}(\CK(G))$ is isomorphic
  to the regular representation of $ \bC[W_{[\lamck]}]$ (under the left and
  right translation), see \cite{BVUni}*{(3.15)}.
  Therefore,
  \[
    \sharp \Unip_{\ckcO}(G)\leq [\tau_{\lamck}\boxtimes \tau_{\lamck}: \bC[W_{[\lamck]}]] = 1
  \]
  by \Cref{cor:bound}.
  This finishes the proof of the theorem.
\end{proof}

}
% In this section, we let $G = \GL_{n}(\bC)$ and

% Then $\fggC:= \fgg \otimes_{\bR}\bC$ is naturally identified with
% $\fgg\times \fgg$ and $\cU(\fggC)=\cU(\fgg)\otimes \cU(\fgg)$.

% When $\lamck$

\trivial[h]{
Since the Lusztig canonical quotient of $\ckcO$ is trivial, the set of unipotent representations
of $G = \GL_n(\bC)$ one-one corresponds to nilpotent orbits in
$\Nil(\ckGc) = \YD_{n}$ if $\lamck$ is integral by
\cite{BVUni}.


\begin{remark}
The Vogan duality gives a duality between Harish-Chandra cells.
In this case, Harish-Chandra cell is the double cell
of Lusztig.
Now we have a duality
\[
 \pi_\ckcO \leftrightarrow \pi_{\ckcO^t}.
\]
\end{remark}

We record the following easy lemma which is a baby case of our results on other
classical groups.
\begin{lem}
  Let $\ckcO' := \DDD(\ckcO)$ be the partition obtained by deleting the first
  column of $\ckcO$. Let $\Phi_{\GL_{a}(\bC),\GL_{b}(\bC)}$ (resp.
  $\Phi_{\GL_{a}(\bC),\GL_{b}(\bC)}$) be the theta lift (resp. big theta lift)
  from $\GL_a(\bC)$ to $\GL_b(\bC)$. Then we have
  \[
    \pi_{\ckcO} = \Phi_{\GL_{\abs{\ckcO'}}(\bC),\GL_{\abs{\ckcO}}(\bC)} (\pi_{\ckcO'})= \Phi_{\GL_{\abs{\ckcO'}}(\bC),\GL_{\abs{\ckcO}}(\bC)} (\pi_{\ckcO'}).
  \]
\end{lem}

One should be able to deduce the local system attached to $\pi_{\ckcO}$ using
the inductive formula.
}

\subsection{Special unipotent representations of $\GL_n(\bR)$ and $\GL_n(\bH)$}\label{sec:GLRH}


%Special unipotent representations of general linear groups $\GL_{n}(F)$, where $F= \bR$ or $\bH$, are well-known and comprise
%representations of the form
%\[
 % \Ind_{P}^{\GL_{n}(F)}\chi,
%\]
%where $P$ is a parabolic subgroup of $\GL_{n}(F)$,
%and $\chi$ is a character of $P$ trivial on the connected component of $P$. See \cite{V.GL}*{Page 450}. We will review their classifications in the framework of this article.


% We assume that $G  = \GL_{n}(\bC), \GL_{n}(\bR), \GL_{\half n}(\bH)$ and $\star = A,A^{\bC}, A^{\bH}$ respectively.

%



% We list some easy facts below:
% \[
%   \begin{split}
%     W &= \sfS_{n} \\
%     W_{[\lamck]} & = \sfS_{\abs{\ckcO_{e}}}\times \sfS_{\abs{\ckcO_{o}}} \\
%     W_{\lamck} & = \prod_{i\in \bN^{+}}\sfS_{\bfcc_{i}(\ckcO_{e})}\times \prod_{i\in \bN^{+}}\sfS_{\bfcc_{i}(\ckcO_{o})}\\\
%     \tau_{\lamck} & := (j_{W_{\lamck}}^{W_{[\lamck]}}\sgn )\otimes \sgn =  \ckcO_{e}^{t}\boxtimes \ckcO_{o}^{t}\\
%     \LC_{\lamck} & = \LRC_{\lamck}= \set{\tau_{\lamck}}, \AND \\
%     \wttau_{\lamck} & = j_{W_{[\lamck]}}^{W}\tau_{\lamck} = \Spr^{-1}(\cO).
%   \end{split}
% \]


% Now let $\ckcO\in \Nil(\ckGc)$ and decompose
% \[
%   \ckcO = \ckcO_{e} \cuprow \ckcO_{o}
% \]
% where $\ckcO_e$ and $\ckcO_o$ are partitions consist of even and odd rows
% respectively.

% Every irreducible representation in $\Irr(W)$ is special. For the
% infinitesimal character $\lamck$,
% \[

% \]

% In all the cases, let $\DDD$ denote the dual descent of Young diagrams.
% Suppose $\ckcO$ is a Young diagram, it delete the longest row in $\ckcO$


% Let $\YD$ be the set of Young diagrams viewed as a finite multiset of positive
% integers. The set of nilpotent orbits in $\GL_n(\bC)$ is identified with Young
% diagram of $n$ boxes.


% Let $n_e = \abs{\ckcO_e}, n_o =
% \abs{\ckcO_o}$. % and $\lambda_\ckcO = \half \ckhh$.
% By the formula of $a$-function, one can easily see that The cell in
% $W(\lamck)$ consists of the unique representation
% $J_{W_{\lamck}}^{\Wint{\lamck}} (1)$. Now the $W$-cell
% $(J_{W_{\lamck}}^{W_{[\lamck]}} \sgn)\otimes \sgn$ consists a single
% representation
% \[
%   \tau_{\ckcO} = \ckcO_{e}^{t}\boxtimes \ckcO_{o}^{t}.
% \]
% The representation $j_{W_{\lamck}}^{S_{n}} \tau_{\ckcO}$ corresponds to the
% orbit $\cO= \ckcO^t $ under the Springer correspondence.
\trivial[h]{ WLOG, we assume $\ckcO = \ckcO_o$.

  Let $\sigma\in \widehat{S_n}$. We identify $\sigma$ with a Young diagram. Let
  $c_i = \bfcc_i(\sigma)$. Then $\sigma = J^{S_n}_{W'} \epsilon_{W'}$ where
  $W' = \prod S_{c_i}$ (see Carter's book). This implies Lusztig's a-function
  takes value
  \[
    a(\sigma) = \sum_i c_i(c_i-1) /2
  \]
  Comparing the above with the dimension formula of nilpotent Orbits
  \cite{CM}*{Collary~6.1.4}, we get (for the formula, see Bai ZQ-Xie Xun's paper
  on GK dimension of $SU(p,q)$)
  \[
    \half \dim(\sigma) = \dim(L(\lambda)) = n(n-1)/2 - a(\sigma).
  \]
  Here $\dim(\sigma)$ is the dimension of nilpotent orbit attached to the Young
  diagram of $\sigma$ (it is the Springer correspondence, regular orbit maps to
  trivial representation, note that $a(\triv)=0$), $L(\lambda)$ is any highest
  weight module in the cell of $\sigma$.


  Return to our question, let $S' = \prod_i S_{\bfcc_i(\ckcO)}$. We want to find
  the component $\sigma_0$ in $\Ind_{S'}^{S_n} 1$ whose $a(\sigma_0)$ is
  maximal, i.e. the Young diagram of $\sigma_0$ is minimal.

  By the branching rule, $\sigma \subset \Ind_{S'}^{S_n} 1$ is given by adding
  rows of length $\bfcc_i(\ckcO)$ repeatedly (Each time add at most one box in
  each column). Now it is clear that $\sigma_0 = \ckcO^t$ is desired.

  This agrees with the Barbasch-Vogan duality $\dBV$ given by
  \[
    \ckcO \xrightarrow{\Spr}\ckcO \xrightarrow{\otimes \sgn} \ckcO^t \xrightarrow{\Spr} \ckcO^t.
  \]
}

%In this subsection, we assume that  $\star=A^\R$ so that $G = \GL_{n}(\bR)$.
%Recall from \Cref{defpbp0} the set $\PP_{A^\R}(\ckcO)$, which is the set of paintings on $\ckcO^{t}$ that has type $A^\R$.
%We define the set of painted partitions of type $A$ as the following:
%\begin{equation*}%\label{eq:PP.AR}
%  \PP_{A}(\ckcO) = \Set{\uptau:=(\tau, \cP)|
%    \begin{array}{l}
%      \text{$\tau = \ckcO^{t}$}\\
%      \text{$\Im(\cP)\subseteq \set{\bullet,c,d}$}\\
%      \text{$\#\set{i|\cP(i,j)=\bullet}$ is even for all $j\in \bN^{+}$}
%    \end{array}
%  }.
%\end{equation*}
Recall from \Cref{defpbp0} and  \Cref{defpbp1} the set $\PP_{\star}(\ckcO)$, which is the set of paintings on $\ckcO^{t}$ that has type $\star\in \{A^\R, A^\bH, A, \wtA\}$.
Recall that if $\star=A^\R$, then
\begin{equation}\label{eq:PPA.count}
  \#(\PP_{\star}(\ckcO)) = \prod_{r\in \bN^{+}} (\#\set{i\in \bN^{+}| \bfrr_{i}(\ckcO)=r}+1).
\end{equation}
Also note that
if $\star=A^\bH$, then
\begin{equation}\label{eq:PPA.count2}
  \#(\PP_{\star}(\ckcO)) = \begin{cases}
   1,
    & \text{ if $\check \CO=\check \CO_\mathrm g$;}  \\
      0, & \text{ otherwise.} \\
    \end{cases}
\end{equation}

\begin{prop} \label{lem:GL.count2}
  Suppose that $\star=\in\{A^\R, A^\bH\}$.  Then
   \begin{equation*}%\label{eq:A.count}
    [\tau_{\lamck}: \Cint{\Lambda}(\CK'(G))] = \# (\PP_{\star}(\ckcOg))\times
    \# (\PP_{\star}(\ckcOb)) = \# (\PP_{\star}(\ckcO)).
  \end{equation*}
\end{prop}
\begin{proof}
  In view of Proposition \ref{count000} and the description of the left cell representation $\tau_{\lamck}$ in \eqref{eq:WA}, the  first equality  follows from  Pieri's rule (\cite[Corollary 9.2.4]{GW}) and
  the following branching formula (see \cite{BV.W}*{Lemma~4.1~(b)}):
\begin{equation}\label{IndFor}
  \Ind_{\sfW_{t}}^{\sfS_{2t}} \varepsilon = \bigoplus_{\substack{\sigma\in \YD_{2t}\\
      \bfcc_{i}(\sigma) \text{ is even for all }i\in \bN^+}} \sigma\qquad (t\in \bN).
\end{equation}
The last equality
  follows from \eqref{eq:PPA.count} and \eqref{eq:PPA.count2}.
\end{proof}
% The $\Wint{\lamck}$-module $\Cint{\lamck}$ is given by the following formula:
% According to Vogan duality, we can obtain the above formula by tensoring
% $\sgn$ on the formula of the unitary groups in \cite{BV.W}*{Section~4}.



Recall from \Cref{thm:mainR00} the map
  \[
    \begin{array}{ccc}
      \PP_{\star}(\ckcO) & \rightarrow & \Unip_{\ckcO}(G),\\
      \uptau & \mapsto & \pi_{\uptau}.
    \end{array}
  \]
 It  is proved in  \cite[Theorem 3.8]{V.GL} that the above map is injective. Then   the map is bijective by \eqref{boundc22} and \Cref{lem:GL.count2}. This proves  \Cref{thm:mainR00}.

\subsection{Special unipotent representations of unitary groups}\label{secunit}

In this subsection, we suppose  $\star \in \set{A, \wtA}$ so that
\[
  G =
  \begin{cases}
    \rU(p,q),  & \text{if }\star = A;\\
    \tU(p,q),  & \text{if }\star = \wtA.
\end{cases}
\]

%When it is convenient, we also use the symbol $U$ to represent either $A^{*}$ or $\wtA^{*}$.

%We define the set of painted partitions of type $U$ as the following:
%\begin{equation*}%\label{eq:PP.U}
%  \PP_{U}(\ckcO) = \Set{\uptau:=(\tau, \cP)|
%    \begin{array}{l}
%      \text{$\tau = \ckcO^{t}$}\\
%      \text{$\Im(\cP)\subseteq \set{\bullet,s,r}$}\\
%      \text{$\#\set{j|\cP(i,j)=\bullet}$ is even for all $i\in \bN^{+}$}
%    \end{array}
%  }.
%\end{equation*}
For $\cP \in \PP_{\star}(\ckcO)$, we have defined its signature $(p_\CP, q_\cP)$ in \eqref{eq:signature}. Recall that $\CO:=\dBV(\check \CO)=\ckcO^t\subset \Nil(\g^*)$.  Let $\overline \Nil_{G}(\cO)$ denote the set of $G$-orbits in $(\sqrt{-1}\g_0^*)\cap \CO$, where $\g_0$ denotes the Lie algebra of $G$ which equals $\u(p,q)$.
%For a painted partition $\cP$ of type $U$, define the signature of $\CP$ to be the pair
%\[
%    (p_\CP, q_\cP): = \left (\frac{\sharp(\cP^{-1}(\bullet))}{2}+\sharp(\cP^{-1}(r)),\,
%    \frac{ \sharp(\cP^{-1}(\bullet))}{2}+\sharp(\cP^{-1}(s))\right).
%\]

We first consider the case when  $\ckcO = \ckcO_{\mathrm g}$.
%Now assume $\ckcO = \ckcO_{\mathrm g}$.
% and so $\Cint{\ckcO}(G)$ corresponds to the blocks of the infinitesimal
% character of the trivial representation.
In this good parity setting, we will state a counting result on
$\Unip_{\ckcO}(G)$. The elements in $\Unip_{\ckcO}(G)$ can be constructed by
cohomological induction explicitly and they are irreducible and unitary due to
\cite{Mat96,Tr.U}, see also \cite{Tr.H}*{Section~2} and \cite{MR.U}*{Section~4}.
%\trivial[h]{See \cite{MR.U}*{First several paragraphs of Section~4}.}
We refer the reader to \cite{BMSZ2} for the construction of all elements of $\Unip_{\ckcO}(G)$ by the method of theta lifting.


\begin{thm}[\cf {\cite[Theorem 4.2]{BV.W} and \cite{Tr.H}*{Theorem~2.1}}]\label{thmunit0}
  Suppose that $\ckcO = \ckcO_{\mathrm g}$. Then
  \begin{equation} \label{sharpuni0} \sharp(\Unip_{\ckcO}(G))= \sharp \set{\CP\in \mathrm{PAP}_{A^*}(\ckcO)|(p_\CP, q_\CP)=(p,q)}=\sharp(\overline \Nil_{G}(\cO)).
  \end{equation}
  Moreover, for every $\pi\in \Unip_{\ckcO}(G)$, its wavefront set $\WF(\pi)$ is
  the closure in $\sqrt{-1}\g_0^*$ of a unique orbit
  $\sO_\pi\in \overline \Nil_{G}(\cO)$, and the map
  \begin{equation}\label{sharpuni1}
    \begin{array}{rcl}
      \Unip_{\ckcO}(G) &\longrightarrow& \overline \Nil_{G}(\cO), \\%\Nil_{\mathrm g}(\cO),\\
      \pi & \mapsto & \sO_\pi.
    \end{array}
  \end{equation}
  is bijective.
\end{thm}

\begin{proof} The Harish-Chandra cell representations in $\Coh_{\Lambda}(\CK'(G))$  are all irreducible and explicitly
  described in terms of the Springer correspondence by \cite{Bo}*{Lemma~4}. Thus the equality holds in \eqref{boundc22}, and the first equality  in \eqref{sharpuni0}
  follows from Proposition \ref{count002}, \eqref{eq:WA},  \eqref{IndFor}, and Pieri's rule (\cite[Corollary 9.2.4]{GW}).
  The second equality in \eqref{sharpuni0} will follow directly from
  the bijectivity of \eqref{sharpuni1}.
  \trivial[h]{
  {delete the following sentence: }
  The second equality in \eqref{sharpuni0}
  follows from (an easy) direct computation of numbers of elements both-sides.
  }
  % We now prove the second equality,  there is a one-one
  % correspondence of Harish-Chandra cells in $\Cint{\lambda_\ckcO}(\CK'(G))$ with
  % the orbits in $\overline \Nil_{G}(\cO)$ by \cite{BV.W}*{Theorem~4.2} and
  % \cite[Theorem 5]{Bo} (we use \cite{SV}*{Theorem~1.4} to restate the result in
  % terms of real nilpotent ).
  %

The assignment of wavefront set yields a bijection
 \[
   \{\textrm{cell in the basal representation $\Cint{\Lambda}(\CK_{\overline{\CO}}'(G))/\Cint{\Lambda}(\CK_{\overline{\CO}\setminus \CO}'(G))$}\} \rightarrow \overline \Nil_{G}(\cO),
 \]
and every cell representation in  $\Cint{\Lambda}(\CK_{\overline{\CO}}'(G))/\Cint{\Lambda}(\CK_{\overline{\CO}\setminus \CO}'(G))$ is irreducible and isomorphic to $\tau_{\lambda_{\check \CO}}$. See \cite{BV.W}*{Theorem~4.2} and
  \cite[Theorem 5]{Bo} (we use \cite{SV}*{Theorem~1.4} to rephrase the result in
  terms of real nilpotent orbits).

 Note that \eqref{mulone} implies   that
 \[
 [1_{W_{\lambda_{\ckcO}}} : \tau_{\lambda_{\check \CO}}]=1.
 \]
 Recall from \Cref{dmlam} that $\lambda_{\ckcO}\in \hha^*$ is dominant. As in the proof of Theorem \ref{count2},  for each cell $\CC$ in $\Cint{\lambda_\ckcO}(\CK_{\overline{\CO}}'(G))$ that is not a cell in $\Cint{\lambda_\ckcO}(\CK_{\overline{\CO}\setminus \CO}'(G))$, there is a unique element $\Psi_\CC\in \CC$ such that $ \Psi_\CC(\lambda_{\ckcO})\neq 0$.
 Then
 \[
    \Unip_{\ckcO}(G) =\{ \Psi_\CC(\lambda_{\ckcO})\}_{\CC\textrm{ is a cell  in $\Cint{\lambda_\ckcO}(\CK_{\overline{\CO}}'(G))$ that is not a cell in $\Cint{\lambda_\ckcO}(\CK_{\overline{\CO}\setminus \CO}'(G))$}}.
 \]
 This implies the bijectivity assertion of the theorem.   % The second assertion follows from the main result of
  % \cite{SV}*{Theorem~1.4}, \cite[Theorem 5]{Bo}, and \eqref{sharpuni0}.
\end{proof}

% \trivial[]{
%   Let
%   $\lambda := (\lambda_{1},\cdots,\lambda_{n})=(\underbrace{1/2, \cdots, 1/2}_{a},\underbrace{0,\cdots,0}_{\mathrm b})$
%   how to compute  $\Coh_{\Lam}(\wtU(p,q))$ of the genuine $\wtU(p,q)$ at the
%   infinitesimal character coset $[\lambda]$?

%   Twist the genuine $\wtU(p,q)$ representations with $\det^{1/2}$ yields
%   an isomorphism between $\Coh_{\Lam}(\wtU(p,q))$ with
%   $\Coh_{\Lam+\half}(U(p,q))$.
% }

% If $p,q \geq \half \nbb$, we write
% \[
% \Gg :=\begin{cases}

% \end{cases}
% \]

\begin{lem}\label{thmca0000}
  If  $ \sharp(\Unip_{\ckcO}(G))\neq 0$, then
  $
    \ckcOb= 2\ckcOpb
  $
  for some Young diagram $\ckcOpb$.

\end{lem}
\begin{proof}
Suppose that  $ \sharp(\Unip_{\ckcO}(G))\neq 0$. Then
\be\label{tau5}
   [\tau_{\lambda_{\check \CO_\mathrm b}}: \Coh_{\Lambda_{\mathrm b}}(\CK'(G_\mathrm b))]\neq 0.
\ee
Recall from Proposition \ref{count002} that
  \[
       \Coh_{\Lambda_\mathrm b}(\CK'(G_\mathrm b)) \cong  \Ind_{\sfW_{\frac{n_{\mathrm b}}{2}}}^{\sfS_{n_{\mathrm b}}} 1,
    \]
Thus \eqref{tau5}
 is the same as saying that
\[
 [ {\check \CO_\mathrm b}^t: \Ind_{\sfW_{t}}^{\sfS_{2t}} 1 ]\neq 0
\]
Similar to \eqref{IndFor}, we have that
\begin{equation}\label{IndFor3}
  \Ind_{\sfW_{t}}^{\sfS_{2t}} 1= \bigoplus_{\substack{\sigma\in \YD_{2t}\\
      \bfrr_{i}(\sigma) \text{ is even for all }i\in \bN^+}} \sigma\qquad (t\in \bN).
\end{equation}
This implies the lemma.
\end{proof}

Now we assume that $
    \ckcOb= 2\ckcOpb
  $
  for some Young diagram $\ckcOpb$.

The group $G$ has an $\check \CO$-relevant parabolic subgroup $P$ whose Levi quotient  is naturally isomorphic to $G'_{\mathrm b}\times G_{\mathrm g}$, where
$G'_{\mathrm b}:=\GL_{\frac{n_{\mathrm b}}{2}}(\C)$. Let $\pi_{\ckcOpb}$ denote the
unique element in $\Unip_{\ckcOpb}(\GL_{\frac{n_{\mathrm b}}{2}}(\C))$. Then for
every $\pi_\mathrm g\in \Unip_{\ckcO_{\mathrm g}}(\Gg)$, the normalized smooth parabolic
induction $\pi_{\ckcOpb}\rtimes \pi_{\mathrm g}$ is irreducible by
\cite{Mat96}*{Theorem~3.2.2} and is an element of $\Unip_{\ckcO}(G)$ (\cf
\cite{MR.U}*{Theorem~5.3}).

\begin{thm}\label{thmunit012}
  The equality  \begin{equation}\label{unitarred1}
    \sharp(\Unip_{\ckcO}(G)) =
    \sharp(\Unip_{\ckcO_{\mathrm g}}(\Gg))
  \end{equation}
  holds,
  and the map
  \begin{equation}
  \label{bij00}
    \begin{array}{ccc}
      \Unip_{\ckcOg}(\Gg)&\longrightarrow &\Unip_{\ckcO}(G),\\
      \pi_{0}& \mapsto & \pi_{\ckcOpb}\rtimes \pi_{0} \\
    \end{array}
  \end{equation}
  is bijective. %Consequently, % Then
  % the map
  % \[
  %   \begin{array}{ccc}
  %     \Unip_{\ckcO_{\mathrm g}}(\Gg)&\longrightarrow &\Unip_{\ckcO}(G),\\
  %     \pi_{0}& \mapsto & \pi_{\ckcOpb}\rtimes \pi_{0} \\
  %   \end{array}
  % \]
  % is bijective. Consequently, \be\label{unitarred1} \sharp(\Unip_{\ckcO}(G)) =
  % \sharp(\Unip_{\ckcO_{\mathrm g}}(\Gg)). \ee
\end{thm}

\begin{proof}
 As in the good parity case,  the structure of cells in the basal representation
  $\Coh_{\Lambda}(\CK'(G))$ (see \cite[Theorem 5]{Bo}) implies
  the equality in \eqref{boundc22}. Hence
  \[
    \sharp(\Unip_{\check \CO}(G)) =[\tau_{\lamck}: \Coh_{\Lambda}(\CK'(G))].
  \]
  Similarly,
  \[
    \sharp(\Unip_{\check \CO_\mathrm g}(G_\mathrm g)) = [\tau_{\lambda_{\check \CO_\mathrm g}}: \Coh_{\Lambda_{\mathrm g}}(\CK'(G_\mathrm g))].
  \]
 The proof of \Cref{thmca0000} shows that
 \[
 [\tau_{\lambda_{\check \CO_\mathrm b}}:\Coh_{\Lambda_\mathrm b}(\CK'(G_\mathrm b))]=1.
 \]
  The above three equalities clearly imply \eqref{unitarred1}.

 By the  calculation of  the wavefront set of the induced representations (\cite[Corollary 5.0.10]{B.Orbit}),  Theorem \ref{thmunit0} implies that all the representations $\pi_{\ckcOpb}\rtimes \pi_{\mathrm g}$, where $\pi_\mathrm g$ varies in $\Unip_{\ckcO_{\mathrm g}}(\Gg)$, have pairwise distinct wavefront sets.   Thus the map \eqref{bij00} is injective. Hence it is bijection by the counting assertion \eqref{unitarred1}.
\end{proof}

  % \begin{enumT}
  %   \item The set $\Unip_{\ckcO_b}(\rU(p,q))\neq \emptyset$ if and only if $p=q$
  %   and each row lenght in $\ckcO$ has even multiplicity.
  %   \item Suppose $\Unip_{\ckcO_b}(\rU(p,p))\neq \emptyset$, let $\ckcO'$ be the
  %   Young diagram such that $\bfrr_i(\ckcO') = \bfrr_{2i}(\ckcO_b)$ and $\pi'$
  %   be the unique special uinpotent representation in
  %   $\Unip_{\ckcO'}(\GL_{p}(\bC))$. Then the unique element in
  %   $\Unip_{\ckcO_b}(\rU(p,p))$ is given by
  %   \[
  %     \pi := \Ind_{P}^{\rU(p,p)} \pi'
  %   \]
  %   where $P$ is a parabolic subgroup in $\rU(p,p)$ with Levi factor equals to
  %   $\GL_p(\bC)$.
  %   \item In general, when $\Unip_{\ckcO_b}(\rU(p,p))\neq \emptyset$, we have a
  %   natural bijection
  %   \[
  %     \begin{array}{rcl}
  %       \Unip_{\ckcO_{\mathrm g}}(\rU(n_1,n_2)) &\longrightarrow& \Unip_{\ckcO}(\rU(n_1+p,n_2+p))\\
  %       \pi_0 & \mapsto & \Ind_P^{\rU(n_1+p,n_2+p)} \pi'\otimes \pi_0
  %     \end{array}
  %   \]
  %   where $P$ is a parabolic subgroup with Levi factor
  %   $\GL_p(\bC)\times \rU(n_1,n_2)$.
  % \end{enumT}

\trivial[h]{
From the branching rule, the cell is parameterized by painted partition
\[
\PP{}(\rU):=\set{\uptau\in \PP{}| \begin{array}{l}\Im (\uptau) \subseteq  \set{\bullet, s,r}\\
  \text{``$\bullet$'' occurs even times in each row}
\end{array}
  }.
\]

The bijection $\PP{}(\rU)\rightarrow \SYD, \uptau\mapsto \sO$ is given by the following recipe:
The shape of $\sO$ is the same as that of $\uptau$.
$\sO$ is the unique (up to row switching) signed Young diagram such that
\[
  \sO(i,\bfrr_i(\uptau)) := \begin{cases}
    +,  & \text{when }\uptau(i,\bfrr_i(\uptau))=r;\\
    -,  & \text{otherwise, i.e. }\uptau(i,\bfrr_i(\uptau))\in \set{\bullet,s}.
  \end{cases}
\]

\begin{eg}
  \[
 \ytb{\bullet\bullet\bullet\bullet r,\bullet\bullet , sr,s,r}
 \quad
 \mapsto\quad
 \ytb{+-+-+,+-, -+,-,+}
  \]
\end{eg}
}

Theorems \ref{thmu1} and \ref{thmu2} then follows from Theorems \ref{thmunit0} and \ref{thmunit012}.

\begin{remark}
  When $\check \CO\neq \check \CO_{\mathrm g}$, the wavefront set $\WF(\pi)$, where $\pi\in \Unip_{\ckcO}(G)$,  may not be the closure of a single orbit in
   $\overline \Nil_{G}(\cO)$.
  % the parabolic induction of a real nilpotent orbit may be reducible. When $\ckcO_{\mathrm b}\neq \emptyset$, a special unipotent representation may thus have
 % a reducible associated variety. Nevertheless, the map
  % $\Unip_{\ckcO}(G) \ni \pi \mapsto \WF(\pi)$ remains to be injective in the case of unitary groups.
\end{remark}



\section{Special unipotent representations in type BCD : counting}


%\section{Counting of special unipotent representations in type BCD}


%\begin{remark}
%If
%\end{remark}

In this section, we assume that $\star \in \set{B,C,\wtC,C^{*},D,D^{*}}$.
If  $\star\in \{C, C^*, D, D^*\}$, we say that $\check \CO_\mathrm b$ has type I if
  \be\label{assumi}
  \textrm{the number of negative entries of $\lambda_{\check \CO_\mathrm b}$ has the same parity as $\frac{n_\mathrm b}{2}$};
  \ee
  otherwise we say that $\check \CO_\mathrm b$ has type II.


\subsection{The coherent continuation representations}
 Let
$\sfH_{t} := \sfW_t\ltimes \set{\pm 1}^t$ ($t\in\bN$), to be viewed as a  subgroup in $\sfW_{2t}$ such
that
\begin{itemize}
  \item the first factor $\sfW_{t}$ sits in $\sfS_{2t}\subset \sfW_{2t}$ as in \eqref{wn1} and \eqref{wn2},
    \item the element $(1,\cdots,1, \underbrace{-1}_{i\text{-th
        term}}, 1, \cdots, 1)\in \set{\pm 1}^{t}$ acts on $\bC^{2t}$ by
        \[
          % (x_{1},\cdots, x_{2i-2}, x_{2i-1}, x_{2i},x_{2i+1},\cdots, x_{2t} )
        (x_{1},x_{2},\cdots, x_{2t} ) \mapsto (x_{1},\cdots, x_{2i-2}, -x_{2i},-x_{2i-1},x_{2i+1},\cdots, x_{2t}).
        \]
\end{itemize}
Note that $\sfH_{t}$ is also a subgroup of $\sfW'_{2t}$. Define a quadratic
character
\[
  \begin{array}{rcl}
    \tilde{\varepsilon} :  \sfH_{t}=  \sfW_{t}\ltimes \set{\pm 1}^{t}& \rightarrow & \set{\pm 1},\\
                                  (g,(a_{1},a_{2},\cdots, a_{t})) & \mapsto & a_{1}a_{2}\cdots a_{t}.
  \end{array}
\]







     The proof of the following Propositions \ref{count003} and \ref{count004}  also follows from Theorem \ref{thm:cohHC} by direct computation. We  omit the details.




\begin{prop}\label{count003}
As representations of $W_\mathrm b$,
    \[
       \Coh_{\Lambda_\mathrm b}(\CK'(G_\mathrm b)) \cong
        \begin{cases}
  \bigoplus_{\substack{2t+c+d=n_\mathrm b}} \Ind_{\sfH_{t} \times \sfW_{c}\times \sfW_{d}}^{\sfW_{n_\mathrm b}}
         \tilde{\varepsilon}\otimes 1\otimes 1, &  \text{if $\star \in \{B,  \wtC \}$}; \medskip\\
      \bigoplus_{\substack{2t+a=n_\mathrm b}} %\Res_{\sfW_{n}}^{\sfW'_{n}} \left(
          \Ind_{\sfH_{t} \times \sfS_a}^{\sfW_{n_\mathrm b}}\tilde{\varepsilon}\otimes 1, &  \text{if $\star \in \{C, D \}$};\medskip \\
         %\Res_{\sfW_{n}}^{\sfW'_{n}} \left(
          \Ind_{\sfH_{\frac{n_\mathrm b}{2}}}^{\sfW'_{n_\mathrm b}}\tilde{\varepsilon}, &  \text{if $\star \in \{C^*, D^* \}$},
     \end{cases}
       \]
       where in the case when $\star\in \{C,D\}$, the right-hand side space is viewed as a representation of $W_\mathrm b=\sfW_{n_\mathrm b}'$ by restriction.
      \end{prop}

\begin{prop}\label{count004}
As representations of $W_\mathrm g$,
    \[
       \Coh_{\Lambda_\mathrm g}(\CK'(G_\mathrm g)) \cong
        \begin{cases}
     \bigoplus_{\substack{0\leq p_\mathrm g-(2t+a+2r)\leq 1,\\0\leq q_\mathrm g - (2t+a+2s)\leq 1}} \Ind_{\sfH_{t} \times \sfS_{a}\times \sfW_s\times \sfW_r}^{\sfW_{n_\mathrm g}}
         \tilde{\varepsilon} \otimes 1 \otimes \sgn \otimes \sgn, &  \text{if $\star=B$}; \medskip\\
     \bigoplus_{2t+a+c+d=n_\mathrm g}\Ind_{\sfH_{t} \times \sfS_{a} \times \sfW_c\times \sfW_d}^{W_{n_\mathrm g}}\tilde{\varepsilon} \otimes
          \sgn \otimes 1 \otimes 1, &  \text{if $\star =C$};\medskip \\
          \bigoplus_{2t+a+a'=n_\mathrm g}\Ind_{\sfH_{t} \times \sfS_{a} \times \sfS_{a'}}^{\sfW_{n_\mathrm g}}\tilde{\varepsilon} \otimes
          \sgn \otimes 1, &  \text{if $\star =\wtC$};\medskip \\
          \bigoplus_{\substack{(t+s,t+r)=(\frac{p_\mathrm g}{2},\frac{q_\mathrm g}{2})}} \Ind_{\sfH_{t} \times \sfW_s\times \sfW_t}^{\sfW_{n_\mathrm g}}
         \tilde{\varepsilon} \otimes \sgn \otimes \sgn, &  \text{if $\star =C^*$}; \medskip\\
        \bigoplus_{\substack{2t+c+d+2r=p_\mathrm g\\2t+c+d+2s=q_\mathrm g}}
          \Ind_{\sfH_{t} \times \sfW_s\times \sfW_r\times \sfW'_{c}\times \sfW_{d} }^{\sfW_{n_\mathrm g}}\tilde{\varepsilon} \otimes \bsgn \otimes \bsgn \otimes 1\otimes
          1, &  \text{if $\star =D$}; \medskip \\
          \bigoplus_{\substack{2t+a=n_\mathrm g}} \Ind_{\sfH_{t} \times \sfS_{a}}^{\sfW'_{n_\mathrm g}}
         \tilde{\varepsilon} \otimes \sgn, &  \text{if $\star =D^*$}, \\
     \end{cases}
       \]
       where in the case when $\star=D$,  the right-hand side space is viewed as a representation of $W_\mathrm g=\sfW_{n_\mathrm g}'$ by restriction.
      \end{prop}



   \subsection{The Lusztig  left cells}
  \label{sec:LCBCD}

To ease the notation, for every sequence $a_1\geq a_2\geq \dots \geq a_k\geq 0$ ($k\geq 0$) of integers,   we let $[a_1, a_2, \cdots, a_k]_{\mathrm{col}}$ denote the Young diagram
whose $i$-th column has length $a_i$ if $1 \leq i \leq k$ and length $0$
otherwise. Likewise, we let $[a_1, a_2, \cdots, a_k]_{\mathrm{row}}$ denote the Young diagram
whose $i$-th  row has length $a_i$ if $1 \leq i \leq k$ and length $0$ otherwise.

 As usual, we identify $\Irr(\sfW_{t})$ ($t\in \BN$) with the set of bipartitions $\tau =(\tau_{L},\tau_{R})$ of total size $t$ (\cite[Section 11.4]{Carter}). Here the total size refers to
$\abs{\tau_{L}}+\abs{\tau_{R}}$.
%Given the chosen embedding of $\sfS_{n}$ into $\sfW'_{n}$,
We also let $(\tau_{L},\tau_{R})_{I}\in \Irr(\sfW'_t)$ denote the  irreducible representation  given by
  \begin{itemize}
    \item the restriction of $(\tau_{L},\tau_{R})\in \Irr(\sfW_{t})$  if
    $\tau_{L}\neq \tau_{R}$, and
    \item
    the induced representation
    $\Ind_{\sfS_{t}}^{\sfW'_{t}} \tau_{L}$ if $\tau_{L}=\tau_{R}$.
  \end{itemize}
  Take an element $w\in \sfW_t$ such that $w \sfW_t'$ generate the group $\sfW_t/\sfW'_t$. Define   $(\tau_{L},\tau_{R})_{II}\in \Irr(\sfW'_t)$ to be the twist of $(\tau_{L},\tau_{R})_{I}$ by the conjugation by $w$, namely there is a linear isomorphism $\kappa: (\tau_{L},\tau_{R})_{I}\rightarrow (\tau_{L},\tau_{R})_{II}$ such that
  \[
    \kappa(g\cdot u)= (wgw^{-1})\cdot (\kappa(u)), \quad \textrm{for all }g\in \sfW'_t, \ u\in (\tau_{L},\tau_{R})_{I}.
  \]


  Note that
  \[
     (\tau_{L},\tau_{R})_{I}=(\tau_{R},\tau_{L})_{I}\quad \textrm{and}\quad (\tau_{L},\tau_{R})_{II}=(\tau_{R},\tau_{L})_{II}
  \]
in all cases, and  $(\tau_{L},\tau_{R})_{I}=(\tau_{R},\tau_{L})_{II}$ when     $\tau_{L}\neq \tau_{R}$.



  In the rest of this subsection, we describe the Lusztig left cell $\LC_{\lambda_{\ckcO}}$
  attached to $\lambda_{\ckcO}$.
 Define two Young diagrams
 \begin{equation}\label{eq:taub}
    \begin{split}
      \tau_{L,\mathrm b} := \begin{cases}
        \big[\half(\bfrr_{1}(\ckcO'_{\mathrm b})+1), \half(\bfrr_{2}(\ckcO'_{\mathrm b})+1), \cdots, \half(\bfrr_{c}(\ckcO'_{\mathrm b})+1)\big]_{\mathrm{col}},
               &\quad \text{if } \star \in \set{B,\wtC}; \\% \smallskip \\
         \big[\half\bfrr_{1}(\ckcO'_{\mathrm b}), \half\bfrr_{2}(\ckcO'_{\mathrm b}),\cdots, \half\bfrr_{c}(\ckcO'_{\mathrm b})\big]_{\mathrm{col}},
        &\quad  \text{if } \star \in \set{C,C^{*}, D,D^{*}},\\
      \end{cases}
    \end{split}
  \end{equation}
  and
   \begin{equation}\label{eq:taub2}
    \begin{split}
      \tau_{R,\mathrm b} := \begin{cases}
        \big(\half(\bfrr_{1}(\ckcO'_{\mathrm b})-1), \half(\bfrr_{2}(\ckcO'_{\mathrm b})-1), \cdots, \half(\bfrr_{c}(\ckcO'_{\mathrm b})-1)\big)_{\mathrm{col}},
               &\quad \text{if } \star \in \set{B,\wtC}; \\% \smallskip \\
         \big(\half\bfrr_{1}(\ckcO'_{\mathrm b}), \half\bfrr_{2}(\ckcO'_{\mathrm b}),\cdots, \half\bfrr_{c}(\ckcO'_{\mathrm b})\big)_{\mathrm{col}},
        &\quad  \text{if } \star \in \set{C,C^{*}, D,D^{*}},\\
      \end{cases}
    \end{split}
  \end{equation}
 where $c:= \bfcc_{1}(\ckcO'_{\mathrm b})$.

 %Recall  from \eqref{wl00} that
% \[
%   W(\Lambda)=
   % W_{\mathrm b}\times W'_{\mathrm g},
 %\]
% where
%  \[
 % W'_\mathrm g: = \begin{cases}
  %   \sfW'_{n_\mathrm g}, &  \text{if $\star =\wtC $}; \\
  % W_\mathrm g, &  \text{otherwise}
  %    \end{cases}
%  \]
%which is the Weyl group of the integral root system
 % \[
 %  \{\alpha\in \Delta_\mathrm g\mid \la \lambda_{\check \CO_\mathrm g},  \check \alpha\ra\in \Z\}.
%  \]
 Define an irreducible  representation $\tau_{\mathrm b}\in \Irr(W_{\mathrm b})$ attached to $\ckcO_{\mathrm b}$ by
\begin{equation}\label{eq:taub}
 \tau_{\mathrm b}:= \left\{
     \begin{array}{ll}
       ( \tau_{L,\mathrm b}, \tau_{R,\mathrm b}), \quad
       & \text{if } \star \in \set{B,\wtC}; \medskip\\
         ( \tau_{L,\mathrm b}, \tau_{R,\mathrm b})_I, \quad & \text{if $\star \in \set{C,C^{*}, D,D^{*}}$ and $\check \CO_\mathrm b$ has type I;}\medskip\\
        ( \tau_{L,\mathrm b}, \tau_{R,\mathrm b})_{II}, \quad & \text{if $\star \in \set{C,C^{*}, D,D^{*}}$ and $\check \CO_\mathrm b$ has type II}.
\end{array}
  \right.
\end{equation}



Recall the set  $\CPPs(\ckcO_{\mathrm g})$ from   \Cref{defn:PP}.   Put
  \[
    {\mathrm A}(\ckcO) := {\mathrm A}(\ckcO_{\mathrm g}):= \textrm{the power set of $\CPPs(\ckcO_{\mathrm g})$},
    \]
    which is identified with the free $\bF_2$-vector space with free basis $\CPP(\ckcO_{\mathrm g})$. Here $\bF_{2}:=\bZ/2\bZ$ is the field with two elements only.
Note that   $\{\emptyset, \CPP(\ckcO_{\mathrm g})\}$ is a subgroup of ${\mathrm A}(\ckcO)$.  Define
   \begin{equation*}%\label{def:barA}
  \bar{\mathrm A}(\ckcO):= \bar{\mathrm A}(\ckcO_{\mathrm g}):=
  \begin{cases}
 {\mathrm A}(\ckcO)/\{\emptyset, \CPP(\ckcO_{\mathrm g})\}, & \quad \text{if  } \star =\wtC;\\
 {\mathrm A}(\ckcO),  & \quad \text{otherwise.}
  \end{cases}
    % \begin{cases}
    %   \wtA(\ckcO)/\wp\sim\wp^{c} & \text{when } \star \in \set{\wtC,D,D^{*}}.\\
    %   \wtA(\ckcO) & \text{when } \star \in \set{B,C,C^{*}},\\
    % \end{cases}
  \end{equation*}

  Generalizing \eqref{ijo}, for each $\wp\in  {\mathrm A}(\ckcO)$,    we define a pair \[
(\imath_\wp, \jmath_\wp):=(\imath_\star(\check \CO, \wp), \jmath_\star(\check \CO, \wp))
\]
 of Young diagrams  as in what follows.

If $\star=B$, then
 \[
   \mathbf c_{1}(\jmath_\wp)=\frac{\mathbf r_1(\check \CO_{\mathrm g})}{2},
\]
and for all $i\geq 1$,
\[
(\mathbf c_{i}(\imath_\wp), \mathbf c_{i+1}(\jmath_\wp))=
   \left\{
     \begin{array}{ll}
           (\frac{\mathbf r_{2i+1}(\check \CO_{\mathrm g})}{2},  \frac{\mathbf r_{2i}(\check \CO_{\mathrm g})}{2}), &\hbox{if $(2i, 2i+1)\in \wp$}; \smallskip\\
            (\frac{\mathbf r_{2i}(\check \CO_{\mathrm g})}{2},  \frac{\mathbf r_{2i+1}(\check \CO_{\mathrm g})}{2}), &\hbox{otherwise}.\\
            \end{array}
   \right.
\]


If $\star=\widetilde{C}$, then for all $i\geq 1$,
\[
(\mathbf c_{i}(\imath_\wp), \mathbf c_{i}(\jmath_\wp))=
   \left\{
     \begin{array}{ll}
           (\frac{\mathbf r_{2i}(\check \CO_{\mathrm g})}{2},  \frac{\mathbf r_{2i-1}(\check \CO_{\mathrm g})}{2}), &\hbox{if $(2i-1, 2i)\in \wp$}; \smallskip\\
            (\frac{\mathbf r_{2i-1}(\check \CO_{\mathrm g})}{2},  \frac{\mathbf r_{2i}(\check \CO_{\mathrm g})}{2}), &\hbox{otherwise}.\\
            \end{array}
   \right.
\]


If $\star\in\{D,D^*\}$, then
 \[
   \mathbf c_{1}(\imath_\wp)= \left\{
     \begin{array}{ll}
            \frac{\mathbf r_1(\check \CO_{\mathrm g})+1}{2},   &\hbox{if $\mathbf r_1(\check \CO_{\mathrm g})>0$}; \smallskip\\
       0,  &\hbox{if $\mathbf r_1(\check \CO_{\mathrm g})=0$},\\
            \end{array}
   \right.
 \]
and for all $i\geq 1$,
\[
(\mathbf c_{i}(\jmath_\wp), \mathbf c_{i+1}(\imath_\wp))=
   \left\{
     \begin{array}{ll}
            (\frac{\mathbf r_{2i+1}(\check \CO_{\mathrm g})-1}{2},  \frac{\mathbf r_{2i}(\check \CO_{\mathrm g})+1}{2}), &\hbox{if $(2i, 2i+1)\in \wp$}; \smallskip\\
        (\frac{\mathbf r_{2i}(\check \CO_{\mathrm g})-1}{2},  0), & \hbox{if $(2i, 2i+1)$ is tailed in $\check \CO_{\mathrm g}$};\smallskip\\
         (0,  0), &\hbox{if $(2i, 2i+1)$ is empty in $\check \CO_{\mathrm g}$};\\
         (\frac{\mathbf r_{2i}(\check \CO)-1}{2},  \frac{\mathbf r_{2i+1}(\check \CO_{\mathrm g})+1}{2}), &\hbox{otherwise}.\\
            \end{array}
   \right.
\]


If $\star\in\{C,C^*\}$, then for all $i\geq 1$,
\[
(\mathbf c_{i}(\jmath_\wp), \mathbf c_{i}(\imath_\wp))=
   \left\{
     \begin{array}{ll}
            (\frac{\mathbf r_{2i}(\check \CO_{\mathrm g})-1}{2},  \frac{\mathbf r_{2i-1}(\check \CO_{\mathrm g})+1}{2}), &\hbox{if $(2i-1, 2i)\in \wp$}; \smallskip\\
        (\frac{\mathbf r_{2i-1}(\check \CO_{\mathrm g})-1}{2},  0), & \hbox{if $(2i-1, 2i)$ is tailed in $\check \CO_{\mathrm g}$};\smallskip\\
         (0,  0), &\hbox{if $(2i-1, 2i)$ is empty in $\check \CO_{\mathrm g}$};\\
         (\frac{\mathbf r_{2i-1}(\check \CO_{\mathrm g})-1}{2},  \frac{\mathbf r_{2i}(\check \CO_{\mathrm g})+1}{2}), &\hbox{otherwise}.\\
            \end{array}
   \right.
\]


We define an element $\tau_{\wp}\in \Irr(\Wg')$ by
  \begin{equation}\label{eq:tauwp}
    \tau_{\wp} :=
    \begin{cases}
      (\imathp,\jmathp),  \quad &  \text{if } \star \in \set{B,C, C^{*} }; \\
      (\imathp,\jmathp)_{I}= (\imathp,\jmathp)_{II},  \quad &  \text{if   $ \star \in \set{D,D^{*}}$, };\\
       (\imathp,\jmathp)_I, \quad & \text{if $\star =\wtC$  and $ \frac{n_\mathrm g}{2}$ is even;}\medskip \\
            (\imathp,\jmathp)_{II}, \quad & \text{if $\star =\wtC$ and  $ \frac{n_\mathrm g}{2}$ is odd.}
    \end{cases}
  \end{equation}
  Note that  if $\star=\wtC$, then $\tau_{\wp} = \tau_{\wp^{c}}$, where $\wp^{c}$ is the complement of $\wp$ in $\CPPs(\ckcO_{\mathrm g})$.
Therefore in all cases,   $\tau_{\bar \wp}\in \Irr(\Wg')$ is obviously defined for every $\bar \wp\in \bar{\mathrm A}(\check \CO)$.



\begin{remark} When $\star\neq \wtC$, $\mathrm A(\ckcO)$ gives another description of Lusztig's canonical
  quotient attached to $\ckcO$. The set $\CPP_{\star}(\ckcO)$ appears implicitly in
\cite{So}*{Section~5}.

  \trivial[h]{ This can be seen from the following
    lemma, c.f. \cite{BVUni}*{Proposition~5.28}. }
\end{remark}

  To simplify the notation, we write $\LC_{\ckcO}:= \LC_{\lambda_{\ckcO}}$. Recall that $\LC_{\ckcO}$ is the set of all $\sigma\in \Irr(W(\Lambda))$  that occurs in the multiplicity free representation
   \[
    \left(J_{\Wlamck}^{W(\Lambda)} \sgn\right) \otimes \sgn.
  \]


  \begin{lem}[\cf Barbasch-Vogan {\cite{BVUni}*{Proposition~5.28}}]
    \label{lem:Lcell}
    The
  map    \[
      \begin{array}{rcl}
        \bar{\mathrm A}(\ckcO) & \rightarrow & \LC_{\ckcO},\\
                       \bar \wp & \mapsto &  \tau_{\mathrm b} \otimes \tau_{\bar \wp}
      \end{array}
    \]
    is well-defined and bijective, and
        \[
      \tau_{\ckcO}:=\tau_{\mathrm b}\otimes \tau_{\emptyset}
    \] is the unique special representation in $\LC_{\ckcO}$.
    % \begin{equation}\label{eq:dBV.W}
    %   \Spr ^{-1}(j_{\WLamck}^{W}(\tau_{\ckcO})) = \textrm{the unique Zariski open orbit in $\mathrm{AV}(I_{\star, \ckcO})$}.
    %   \end{equation} Here
    %   $\mathrm{AV}$ indicates the associated variety.
 Moreover,
    \begin{equation}\label{eq:dBV.W}
      \Spr ^{-1}(j_{W(\Lambda)}^{W}(\tau_{\ckcO}))
      = \dBV(\ckcO),
       \end{equation}
    as $\Ad(\g)$-orbits,
    and
  \begin{equation}\label{eq:dBV.W2}
     \dBV(\ckcO)=   (\ckcOpb)^{t} \cupcol (\ckcOpb)^{t}\cupcol \dBV(\ckcOg)
   \end{equation}
as Young diagrams.
        \end{lem}

          % \[
          %   (\bfcc_{i}(\imathp), \bfcc_{i}(\jmathp)):=
          %   \begin{cases}
          %     (\half (\bfrr_{2i-1}(\ckcO_{\mathrm g})+1),\half (\bfrr_{2i}(\ckcO_{\mathrm g})-1)) &\text{if } (2i-1,2i)\in \wp, \\
          %     (\half (\bfrr_{2i}(\ckcO_{\mathrm g})+1), \half (\bfrr_{2i-1}(\ckcO_{\mathrm g})-1))
          %     & \text{if } (2i-1,2i)\notin \wp\\
          %     & \text{ and }\bfrr_{2i}(\ckcO_{\mathrm g})\neq 0,
          %     \\
          %     (0,0)
          %     & \text{if } \bfrr_{2i-1}(\ckcO_{\mathrm g})=0,\\
          %     (0, \half (\bfrr_{2i-1}(\ckcO_{\mathrm g})-1)) & \text{otherwise}
          %   \end{cases}
          % \]

          % \[
          %   (\bfcc_{l+1}(\imathp), \bfcc_{l+1}(\jmathp)) := (0,\half(\bfrr_{2l+1}(\ckcO_{\mathrm g})-1))
          % \]
          % and for all $1\leq i\leq l$

    \trivial[h]{ The last equality could be checked using Sommer's formula on
      Springer correspondence directly: double columns $(2c+1,2c+1)$ corresponds
      to
      $ B_{c=\alpha_{2i-1}}\times D_{c+1=\alpha_{2i}+1}=D_{c+1=\alpha_{2i-1}+1}\times C_{c=\alpha_{2i}}$
      factor in type $B,\wtC$. double columns $(2c,2c)$ corresponds to factor
      $D_{c=\beta_{2i-1}}\times C_{c=\beta_{2i}}=D_{c=\beta_{2i-1}}\times B_{c=\beta_{2i}}$
      in type $C,C^{*},D,D^{*}$. Here $\dBV$ is the metaplectic dual if
    $\star=\wtC$ and is the Barbasch-Vogan dual otherwise.
}

  \begin{proof}
    When  $\ckcO=\ckcO_{\mathrm g}$ and $\star\neq \wtC$, the lemma is proved in \cite{BVUni}*{Proposition~5.28}.  When $\star\neq \wtC$, the equalities
    \eqref{eq:dBV.W} and \eqref{eq:dBV.W2} are proved in \cite{BVUni}*{Proposition~A2}. In general, the lemma follows from
    %from an induction on number of columns of $\ckcO_{\mathrm g}$ using
    Lusztig's formula of $J$-induction in \cite{Lu}*{\S 4.4-4.6}.

    \trivial[h]{{ {\bf Suppose $\star=C$.}
      In this case, bad parity is even and each row length occur with even
      multiplicity. Suppose
      $\ckcO_{\mathrm b} = (C_{1}, C_{1}, C_{2},C_{2}, \cdots, C_{k'},C_{k'})$ with
      $c_{1}=2k$ and $k' = \bfrr_{1}(\ckcO_{\mathrm b})$.
      \[
        W_{\lamckb} = S_{C_{1}}\times S_{C_{2}}\times \cdots S_{C_{k'}}.
      \]
      The symbol of trivial representation of trivial group of type D is
      \[
        \binom{0,1, \cdots, k-1}{0,1, \cdots, k-1}.
      \]
      Now it is easy to see that (use the similar computation as below)
      \[
        J_{W_{\lamckb}}^{W_{\mathrm b}}\sgn = ((\half C_{1}, \half C_{2},\cdots, \half C_{k'}),(\half C_{1}, \half C_{2},\cdots, \half C_{k'})).
      \]


      For the good parity part. Let
      $r'_{i} = \floor{\half(\bfrr_{i}(\ckcO_{\mathrm g})-\bfrr_{i+1}(\ckcO_{\mathrm g}))}$.
      Suppose $\ckcO_{\mathrm g}$ has $2l+1$ columns (superscripts denote the
      multiplicity)
      \[
        \ckcO_{\mathrm g} = ((2l+1)^{2r'_{2l+1}+1}, 2l^{2r'_{2l}}, (2l-1)^{2r'_{2l-1}}, \cdots, 2^{2r'_{2}}, 1^{2r'_{1}} )
      \]
      and
      % $\ckcO_{\mathrm g} = (2c_{1}+1, C_{2}, C_{2},C_{3},C_{3},\cdots, C_{k'},C_{k'})$
      % with $2c_{1}+1=2l+1$ and $2k'+1 = \bfrr_{1}(\ckcO_{\mathrm g})$.
      \[
        W_{\lamckg} = W_{l}\times \underbrace{S_{2l+1}\times \cdots \times S_{2l+1}}_{2r'_{2l+1}\text{-terms}} \times \prod_{i<2l+1} \underbrace{S_{i}\times \cdots\times S_{i}}_{r'_{i}\text{-terms}}
      \]

      The symbol of sign representation of $W_{l}$ is
      \[
        \binom{0,1,2, \cdots, l}{1,2, \cdots, l}.
      \]
      The induction begins with the longest columns to the shorter columns

      Induce to include all $2l+1$-length columns yields
      \[
        \binom{r'_{2l+1}+0,r'_{2l+1}+1,r'_{2l+1}+2, \cdots, r'_{2l+1}+l}{ r'_{2l+1}+1,r'_{2l+1}+2, \cdots, r'_{2l+1}+l}.
      \]
      Now move the shorter columns, we see that when even columns
      $(2i)^{2r'_{2i}}$ occurs, it adds $(i)^{r'_{2i}}$ columns on the both
      sides of the bipartition; when odd columns $(2i+1)^{r'_{2i+1}}$ occur, the
      bifurcation happens: one can
      \begin{itemize}
        \item attach columns $(i+1)^{r'_{2i+1}}$ on the left and columns
              $(i)^{r'_{2i+1}}$ on the right, which corresponds to
              $(2i+1,2i+2)\neq \wp$, or
        \item attach columns $(i)^{r'_{2i+1}}$ on the left and columns
              $(i+1)^{r'_{2i+1}}$ on the right, which corresponds to
              $(2i+1,2i+2)\in \wp$,
      \end{itemize}

      Therefore,
      \[
        \begin{array}{ccc}
          J_{W_{\lamckg}}^{W_{\mathrm g}} \sgn
          &\leftrightarrow&  \bF_{2}(\CPP(\ckcO_{\mathrm g}))\\
          (\cktau_{L},\cktau_{R}) =:\cktau_{\wp}&\leftrightarrow & \wp
        \end{array}
      \]
      where
      \[
        \bfrr_{l+1}(\cktau_{L}) = r'_{2l+1} = \half (\bfrr_{2l+1}(\ckcO_{\mathrm g})-1)
      \]
      and, if $(2i-1,2i)\notin \wp$,
      \[
        \begin{split}
          \bfrr_{i}(\cktau_{L}) & = \sum_{l\geq 2i-1} r'_{l}
          = \half(\bfrr_{2i-1}(\ckcO)-1)\\
          \bfrr_{i}(\cktau_{R}) & = 1 + \sum_{l\geq 2i} r'_{l} = \half(\bfrr_{2i}(\ckcO)+1)
        \end{split}
      \]
      if $(2i-1,2i)\in \wp$,
      \[
        \begin{split}
          \bfrr_{i}(\cktau_{L}) & = \sum_{l\geq 2i} r'_{l}
          = \half(\bfrr_{2i}(\ckcO)-1)\\
          \bfrr_{i}(\cktau_{R}) & = 1 + \sum_{l\geq 2i-1} r'_{l} = \half(\bfrr_{2i-1}(\ckcO)+1)
        \end{split}
      \]

      % \[
      %   \begin{split}
      %     \bfrr_{l+1}(\cktau_{L}) & = r'_{2l+1} =
      %     \half (\bfrr_{2l+1}(\ckcO_{\mathrm g})-1)\\
      %     (\bfrr_{i}(\cktau_{L}), \bfrr_{i}(\cktau_{R})) & =
      %     \begin{cases}
      %       (\half(\bfrr_{2i-1}(\ckcO_{\mathrm g})-1), \half(\bfrr_{2i}(\ckcO_{\mathrm g})+1)) & (2i-1,2i)\notin \wp\\
      %       (\half(\bfrr_{2i}(\ckcO_{\mathrm g})-1), \half(\bfrr_{2i-1}(\ckcO_{\mathrm g})+1)) & (2i-1,2i)\in \wp
      %     \end{cases}
      %   \end{split}
      % \]

      Since $\tau_{\wp} = \cktau_{\wp}\otimes \sgn$, we get the claim.

      We adopt the convention that
      \[
        \sfS_{\cO} := \prod_{i\in \bN^{+}}\sfS_{\bfcc_{i}(\cO)}
      \]
      so that $j_{\sfS_{\cO}}^{\sfS_{\abs{\cO}}}\sgn = \cO$ for each partition
      $\cO$.

      Now consider the orbit under the Springer correspondence.

      Let
      $\ckcO'_{\mathrm b}: = [\bfrr_{2}(\ckcO_{\mathrm b}), \bfrr_{4}(\ckcO_{\mathrm b}),\cdots, \bfrr_{2k}(\ckcO_{\mathrm b})]$,
      $\cO'_{\mathrm b}:=(\ckcO'_{\mathrm b})^{t}$ and $\cO_{\mathrm b}:=\cO'_{\mathrm b}\cupcol \cO'_{\mathrm b}$.
      Clearly, $\ckcO_{\mathrm b} = \ckcO'_{\mathrm b}\cuprow \ckcO'_{\mathrm b}$. Note that
      $\tau_{\mathrm b} = j_{S_{\cO'_{\mathrm b}}}^{W'_{\mathrm b}} \sgn$ (by the formula of fake degree
      see Lusztig or Carter's book). So, by induction by stage of $j$-induction,
      we have
      \[
        \wttau_{\cO}:= j_{W'_{\mathrm b}\times W_{\mathrm g}}^{W_{n}} (\tau_{\mathrm b}\otimes \tau_{\emptyset}) = j_{S_{\cO'_{\mathrm b}}\times W_{\mathrm g}}^{W_{n}} \sgn\otimes \tau_{\wp}.
      \]
      By Barbasch-Vogan, $\cO_{\mathrm g}:=\Spr(\tau_{\emptyset}) = d_{BV}(\ckcO_{\mathrm g})$,
      which is well know how to calculate. (In fact, one can deduce the result
      by our computation. )

      Since the Springer correspondence commutes with parabolic induction, we
      get
      $\Spr(\wttau) = \Ind_{\GL_{\cO'_{\mathrm b}}\times \Sp(2g)}^{\Sp(2n)} 0\times \cO_{\mathrm g} = \cO_{\mathrm b}\cupcol \cO_{\mathrm g}$.


      \medskip

      {\bf Suppose $\star=D$.}

      The bad parity part is the same as that of the case when $\star = C$.

      Now consider the good parity part.
      \[
        \ckcO_{\mathrm g} = ((2l)^{2r'_{2l}+1}, (2l-1)^{2r'_{2l-1}}, (2l-2)^{2r'_{2l-2}}, \cdots, 2^{2r'_{2}}, 1^{2r'_{1}} )
      \]
      and
      \[
        W_{\lamckg} = W'_{l}\times \underbrace{S_{2l}\times \cdots \times S_{2l}}_{2r'_{2l}\text{-terms}} \times \prod_{i<2l} \underbrace{S_{i}\times \cdots\times S_{i}}_{r'_{i}\text{-terms}}
      \]

      The symbol of sign representation of $W'_{l}$ is
      \[
        \binom{0,1, \cdots, l-1}{1,2, \cdots, l\phantom{-1}}.
      \]
      (Here we always made the choice of the top and bottom row to compatible
      with the type $C$ case. )

      Induce to include all $2l$-length columns yields
      \[
        \binom{r'_{2l}+0,r'_{2l}+1, \cdots, r'_{2l}+l-1}{ r'_{2l}+1,r'_{2l}+2, \cdots, r'_{2l}+l\phantom{-1}}.
      \]
      Now move the shorter columns. When odd columns $(2i+1)^{2r'_{2i+1}}$
      occurs, it adds $(i)^{r'_{2i+1}}$ columns on the left and
      $(i+1)^{r'_{2i+1}}$ on the right. When even columns $(2i)^{r'_{2i}}$
      occur, the bifurcation happens: one can
      \begin{itemize}
        \item attach columns $(i)^{r'_{2i}}$ on the left and columns
              $(i)^{r'_{2i}}$ on the right, which corresponds to
              $(2i,2i+1)\neq \wp$, or
        \item attach columns $(i-1)^{r'_{2i}}$ on the left and columns
              $(i+1)^{r'_{2i}}$ on the right, which corresponds to
              $(2i,2i+ 1)\in \wp$,
      \end{itemize}

      Therefore,
      \[
        \begin{array}{ccc}
          \bF_{2}(\CPP(\ckcO_{\mathrm g}))&\longrightarrow
          & J_{W_{\lamckg}}^{W_{\mathrm g}} \sgn \\
          \wp&\mapsto&    (\cktau_{L},\cktau_{R}) =:\cktau_{\wp}
        \end{array}
      \]
      where
      \[
        \bfrr_{l}(\cktau_{L}) = r'_{2l} = \half (\bfrr_{2l}(\ckcO_{\mathrm g})-1)
      \]
      \[
        \bfrr_{1}(\cktau_{R}) = 1+ \sum_{i} r'_{i} = \half (\bfrr_{1}(\ckcO_{\mathrm g})+1)
      \]
      and, if $(2i,2i+1)\notin \wp$,
      \[
        \begin{split}
          \bfrr_{i}(\cktau_{L}) & = \sum_{l\geq 2i} r'_{l}
          = \half(\bfrr_{2i}(\ckcO)-1)\\
          \bfrr_{i+1}(\cktau_{R}) & = 1 + \sum_{l\geq 2i+1} r'_{l} = \half(\bfrr_{2i+1}(\ckcO)+1)
        \end{split}
      \]
      if $(2i,2i+1)\in \wp$,
      \[
        \begin{split}
          \bfrr_{i}(\cktau_{L}) & = \sum_{l\geq 2i+1} r'_{l}
          = \half(\bfrr_{2i+1}(\ckcO)-1)\\
          \bfrr_{i}(\cktau_{R}) & = 1 + \sum_{l\geq 2i} r'_{l} = \half(\bfrr_{2i}(\ckcO)+1)
        \end{split}
      \]

      Also note that $\cktau_{\wp}=\cktau_{\wp^{c}}$. The rest parts are the
      same as that of type $C$.

      {\bf Suppose $\star=B$. }

      In this case, bad parity is odd and every odd row occurs with even
      times.

      We can write
      $r'_{i} := \floor{\half(\bfrr_{i}(\ckcO_{\mathrm b})-\bfrr_{i-1}(\ckcO_{\mathrm b}))}$
      \[
        \ckcO_{\mathrm b} % = [2r_{1}+1, 2r_{1}+1, \cdots, 2r_{k}+1,2r_{k}+1]
        % = (2c_{0},2c_{1},2c_{1}, \cdots, 2c_{l}, 2c_{l}).
        = ((2l)^{2r'_{2l}+1}, (2l-1)^{2r'_{2l-1}},\cdots, 1^{2r'_{1}})
      \]
            %             with $k = c_{0}$ and $l = r_{1}$.
      Then
      \[
        W_{\lamckb} = W_{l} \times \underbrace{S_{2l}\times \cdots \times S_{2l}}_{2r'_{2l}\text{-terms}} \times \prod_{i<2l} \underbrace{S_{i}\times \cdots\times S_{i}}_{r'_{i}\text{-terms}}
      \]
      (Note that in the product, $r'_{i}=0$ if $i$ is odd.) The computation of
      $\cksigma_{\mathrm b} = J_{W_{\lamckb}}^{W_{\mathrm b}} \sgn$ is similar to that of the
      good parity for type $C$ with no bifurcating, one deduce that
      $J$-induction and $j$-induction gives the same result.
      \[
        \begin{split}
          \cksigma_{\mathrm b} &=
          \binom{0, 1+r_{l}, 2+r_{l-1}\cdots, l+r_{1}}{1+r_{l},2+r_{l-1}, \cdots, l+r_{1}}\\
          & = ([r_{1},r_{2},\cdots, r_{l}],[r_{1}+1,r_{2}+1,\cdots,r_{l}+1])\\
        \end{split}
      \]
      with $r_{i} = \half\bfrr_{2i-1}(\ckcO_{\mathrm b}) = \half\bfrr_{2i}(\ckcO_{\mathrm b})$.
      Now
      \[
        \sigma_{\mathrm b} = ((r_{1}+1,r_{2}+1,\cdots,r_{l}+1), (r_{1},r_{2},\cdots, r_{l})) = j_{S_{\cO'_{\mathrm b}}}^{W_{\mathrm b}}\sgn
      \]
      where
      $\cO'_{\mathrm b}=(\bfrr_{2}(\ckcO_{\mathrm b}),\bfrr_{4}(\ckcO_{\mathrm b}),\cdots, \bfrr_{2l}(\ckcO_{\mathrm b}))$.
      Under the Springer correspondence of type $B$, it corresponds to
      $\Ind_{\GL_{\mathrm b}}^{\SO(2b+1)}\cO'_{\mathrm b} = \cO'_{\mathrm b}\cuprow \cO'_{\mathrm b}\cuprow (1)$.

      % \[
      %   \begin{split}
      %     \cksigma_{\mathrm b} &:= \sigma_{\mathrm b}\otimes \sgn = j_{W_{\lamckb}}^{W_{\mathrm b}} \sgn \\
      %     & = %\dagger_{2c_{l}}\cdots \dagger_{2c_{1}}
      %     \sigma_{\mathrm b}\otimes \sgn = j_{W_{\lamckb}}^{W_{\mathrm b}} \sgn\otimes
      %     \binom{0, 1, \cdots, c_{0}}{1, \cdots, c_{0}}\\
      %     & =
      %     \binom{0, 1+r_{k}, 2+r_{k-1}\cdots, c_{0}+r_{1}}{1+r_{k},2+r_{k-1}, \cdots, c_{0}+r_{1}}\\
      %     & = ([r_{1},r_{2},\cdots, r_{k}],[r_{1}+1,r_{2}+1,\cdots,r_{k}+1])\\
      %     &= ((c_{1},c_{2},\cdots, c_{k}),(c_{0},c_{1}, \cdots, c_{l}))\\
      %   \end{split}
      % \]


      % We take the convention that $\dagger \cO = [r_{i}+1]$. By abuse of
      % notation, let $\dagger_{n} \sigma$ denote the
      % $j_{S_{n} \times W_{\abs{\sigma}}}^{W_{n+\abs{\sigma}}} \sgn\otimes \sigma$.
      % We can write
      % \[
      %   \ckcO_{\mathrm b} = [2r_{1}+1, 2r_{1}+1, \cdots, 2r_{k}+1,2r_{k}+1] = (2c_{0},2c_{1},2c_{1}, \cdots, 2c_{l}, 2c_{l})
      % \]
      % with $k = c_{0}$ and $l = r_{1}$.

      % \[
      %   \begin{split}
      %     W_{\lamckb} &= W_{c_{0}} \times S_{2c_{1}} \times S_{2c_{2}}\times \cdots \times S_{2c_{l}}\\
      %     \cksigma_{\mathrm b} &:= \sigma_{\mathrm b}\otimes \sgn = j_{W_{\lamckb}}^{W_{\mathrm b}} \sgn \\
      %     & = \dagger_{2c_{l}}\cdots \dagger_{2c_{1}}
      %     \binom{0, 1, \cdots, c_{0}}{1, \cdots, c_{0}}\\
      %     & =
      %     \binom{0, 1+r_{k}, 2+r_{k-1}\cdots, c_{0}+r_{1}}{1+r_{k},2+r_{k-1}, \cdots, c_{0}+r_{1}}\\
      %     & = ([r_{1},r_{2},\cdots, r_{k}],[r_{1}+1,r_{2}+1,\cdots,r_{k}+1])\\
      %     &= ((c_{1},c_{2},\cdots, c_{k}),(c_{0},c_{1}, \cdots, c_{l}))\\
      %   \end{split}
      % \]

      % Therefore
      % \[
      %   \begin{split}
      %     \sigma_{\mathrm b} &= \cksigma_{\mathrm b}\otimes \sgn = ((r_{1}+1,r_{2}+1,\cdots,r_{k}+1),(r_{1},r_{2},\cdots, r_{k})) \\
      %     & = j_{S_{2r_{1}+1}\times \cdots S_{2r_{k}+1}}^{W_{\mathrm b}} \sgn\\
      %     & = j_{S_{\mathrm b}}^{W_{\mathrm b}} (2r_{1}+1, 2r_{2}+1, \cdots, 2r_{k}+1)
      %   \end{split}
      % \]
      % which corresponds to the orbit
      % \[
      %   \cO_{\mathrm b} = (2r_{1}+1, 2r_{1}+1,2r_{2}+1, 2r_{2}+1, \cdots,2r_{k}+1, 2r_{k}+1 ) = \ckcO_{\mathrm b}^{t}.
      % \]
      % (Note that $\cO'_{\mathrm b} = (2r_{1}+1,2r_{2}+1, \cdots, 2r_{k}+1)$ which
      % corresponds to $j_{W_{L_{\mathrm b}}}^{S_{\mathrm b}}\sgn$ and
      % $\ind_{L}^{G} \cO'_{\mathrm b} = \cO_{\mathrm b}$. ) This implies the unique special
      % representation is
      % \[
      %   \sigma_{\mathrm b} = (j_{W_{\lamckb}}^{W_{\mathrm b}}\sgn), \quad \text{where
      % } W_{L,b} = \prod_{i=1}^{k} S_{2r_{i}+1}.
      % \]
      % The $J$-induction is calculated by \cite{Lu}*{(4.5.4)}. It is easy to
      % see that in our case $J_{W_{\lamckb}}^{W_{\mathrm b}} \sgn$ consists of the
      % single special representation by induction.


      Now we consider the good parity part, where each row of $\ckcO_{\mathrm g}$ has
      even length.

      Assume $r'_{i} := \half(\bfrr_{i}(\ckcO_{\mathrm g})-\bfrr_{i-1}(\ckcO_{\mathrm g}))$ and
      so
      \[
        \ckcO_{\mathrm g} % = [2r_{1}+1, 2r_{1}+1, \cdots, 2r_{k}+1,2r_{k}+1]
        % = (2c_{0},2c_{1},2c_{1}, \cdots, 2c_{l}, 2c_{l}).
        = ((2l+1)^{2r'_{2l+1}}, (2l)^{2r'_{2l}},\cdots, 1^{2r'_{1}})
      \]
      % Consider
      % \[
      %   \cO_{\mathrm g} = [2r_{1},2r_{2}, \cdots, 2r_{2k-1},2r_{2k}] = (C_{1},C_{1}, C_{2},C_{2},\cdots, C_{l}, C_{l}).
      % \]
      with $l =\min\set{i|\bfrr_{2i+2}(\ckcO_{\mathrm g}) = 0}$.

      Then
      \[
        W_{\lamckg} = \times \prod_{i\leq 2l+1} \underbrace{S_{i}\times \cdots\times S_{i}}_{r'_{i}\text{-terms}}
      \]

      Note that the trivial representation of the trivial group has symbol
      \[
        \binom{0,1, 2, \cdots, l\phantom{-1}}{0,1, \cdots, l-1}.
      \]


      Induce to include all $2l+1$-length columns yields
      \[
        \binom{r'_{2l+1}+0,r'_{2l+1}+1,r'_{2l+1}+2,\cdots, r'_{2l+1}+l\phantom{-1}}{ r'_{2l+1}+0,r'_{2l+1}+1, \cdots, r'_{2l+1}+l-1}.
      \]
      Now move the shorter columns. When odd columns $(2i+1)^{2r'_{2i+1}}$
      occurs, it adds $(i+1)^{r'_{2i+1}}$ columns on the left and
      $(i)^{r'_{2i+1}}$ on the right. When even columns $(2i)^{r'_{2i}}$ occur,
      the bifurcation happens: one can
      \begin{itemize}
        \item attach columns $(i)^{r'_{2i}}$ on the left and columns
              $(i)^{r'_{2i}}$ on the right, which corresponds to
              $(2i,2i+1)\neq \wp$, or
        \item attach columns $(i-1)^{r'_{2i}}$ on the left and columns
              $(i+1)^{r'_{2i}}$ on the right, which corresponds to
              $(2i,2i+1)\in \wp$.
      \end{itemize}


      Therefore,
      \[
        \begin{array}{ccc}
          \bF_{2}(\CPP(\ckcO_{\mathrm g}))&\longrightarrow
          & J_{W_{\lamckg}}^{W_{\mathrm g}} \sgn \\
          \wp&\mapsto&    (\cktau_{L},\cktau_{R}) =:\cktau_{\wp}
        \end{array}
      \]
      where
      % \[
      %   \bfrr_{2l+1}(\cktau_{L}) = r'_{2l+1} = \half \bfrr_{2l+1}(\ckcO_{\mathrm g})
      % \]
      \[
        \bfrr_{1}(\cktau_{L}) = \sum_{i} r'_{i} = \half \bfrr_{1}(\ckcO_{\mathrm g})
      \]
      and, if $(2i,2i+1)\notin \wp$,
      \[
        \begin{split}
          \bfrr_{i+1}(\cktau_{L}) & = \sum_{l\geq 2i+1} r'_{l}
          = \half\bfrr_{2i+1}(\ckcO)\\
          \bfrr_{i}(\cktau_{R}) & = \sum_{l\geq 2i} r'_{l} = \half\bfrr_{2i}(\ckcO)
        \end{split}
      \]
      if $(2i,2i+1)\in \wp$,
      \[
        \begin{split}
          \bfrr_{i+1}(\cktau_{L}) & = \sum_{l\geq 2i} r'_{l}
          = \half\bfrr_{2i}(\ckcO)\\
          \bfrr_{i}(\cktau_{R}) & = \sum_{l\geq 2i+1} r'_{l} = \half\bfrr_{2i+1}(\ckcO)
        \end{split}
      \]

      Some remarks on the BV-dual. The calculation of $\cO_{\mathrm g}$ from
      $\tau_{\emptyset}$ can be reduced to the case of quasi-distinguished
      orbits (other case are deduced from this by parabolic induction,
      corresponds to attach two even columns for the balanced pairs). Compare
      Sommer's description of Springer correspondence with ours, we deduce that
      \[
        \cO_{\mathrm g} = (\bfrr_{1}(\ckcO_{1})+1,\bfrr_{2}(\ckcO_{2})-1,\bfrr_{3}(\ckcO_{3})+1, \cdots, \bfrr_{2l}(\ckcO_{2l})-1,\bfrr_{2l+1}(\ckcO_{2l+1})+1)
      \]
      The rest parts are similar to that of type $D$.
    }


    We give the main steps of the proof for the case when $\star = \wtC$.


    We first consider the good parity part $\ckcO_{\mathrm g}$, where each row has
    even length.

    Set $r'_{i} := \half(\bfrr_{i}(\ckcO_{\mathrm g})-\bfrr_{i-1}(\ckcO_{\mathrm g}))$,
    $l =\min\set{i|\bfrr_{2i+1}(\ckcO_{\mathrm g})=0}$, and write
    \[
      \ckcO_{\mathrm g} % = [2r_{1}+1, 2r_{1}+1, \cdots, 2r_{k}+1,2r_{k}+1]
      % = (2c_{0},2c_{1},2c_{1}, \cdots, 2c_{l}, 2c_{l}).
     % = ((2l)^{2r'_{2l}}, (2l-1)^{2r'_{2l-1}},\cdots, 1^{2r'_{1}})
      = (\underbrace{2l,\cdots, 2l}_{2r'_{2l}}, \underbrace{2l-1,\cdots, 2l-1}_{2r'_{2l-1}},\cdots,
      \underbrace{1,\cdots, 1}_{2r'_{1}})
    \]
    where $i^{r'}$ denotes $r'$-copies of length $i$ columns.
    % Consider
    % \[
    %   \cO_{\mathrm g} = [2r_{1},2r_{2}, \cdots, 2r_{2k-1},2r_{2k}] = (C_{1},C_{1}, C_{2},C_{2},\cdots, C_{l}, C_{l}).
    % \]
    The Weyl group $W_{\mathrm g}$ of good parity is $\sfW'_{n_{\mathrm g}}$ with
    $n_{\mathrm g} = \half\abs{\ckcO_{\mathrm g}}$. For $1\leq k\leq l$, let
    \[
      % S_{r,s} = \prod_{i=r}^{s} \underbrace{\sfS_{i}\times \cdots\times \sfS_{i}}_{r'_{i}\text{-terms}}
      \vec{S}_{i} = \underbrace{\sfS_{i}\times \cdots\times \sfS_{i}}_{r'_{i}\text{-times}} \AND n_{k} = \sum_{i=k}^{2l} i\cdot r'_{i}.
      % \AND n_{r,s} = \sum_{i=r}^{s} i\cdot r'_{i}.
    \]

    Then $W_{\lamckg}=\prod_{i=1}^{l} \vec{S}_{i}$ and
    \[
      \begin{split}
        \ckLV_{\ckcO_{\mathrm g}}& :=J_{W_{\lamckg}}^{\Wg}\sgn\\
        & = J_{\vec{S}_{1}\times \sfW'_{n_{2}}}^{\sfW'_{n_{1}}} \Big(\sgn \otimes J_{\vec{S}_{2}\times \sfW'_{n_{3}}}^{\sfW'_{n_{2}}}\Big(\sgn
        \otimes \cdots\big(J_{\vec{S}_{l}} \sgn\big)\cdots \Big)\Big) \\
      \end{split}
    \]
    Applying \cite{Lu}*{(4.6.2)} inductively, we see that the operation
    $J_{\vec{S}_{i}\times \sfW'_{n_{i+1}}}^{\sfW'_{n_{i}}}(\sgn \otimes \underline{\ \ \ })$ yields a multiplicity-free representation and
    doubles (resp. keeps) the number of irreducible constituents if $i$ is odd (resp. even). Informally we will say that a bifurcation occurs when attaching an odd length column.
    Denote by $\ckLC_{\ckcO_{\mathrm g}}$ the multiset of irreducible constituents of $\ckLV_{\ckcO_{\mathrm g}}$.

    First assume $\CPPs(\ckcO_{\mathrm g}) = \emptyset$. Define
    \[
      \tA'(\ckcO):= \bZ_{2}[\emptyset]
    \]
    to be the trivial group.
    Then $\ckLV_{\ckcO_{\mathrm g}}$ is
    irreducible (with the corresponding bipartition marked by the label $I$).  Hence, $\LC_{\ckcO_{\mathrm g}}$
    %$\LC_{\ckcO)}$
    and $\tA'(\ckcO)$ can be obviously identified.


    Now assume $\CPPs(\ckcO_{\mathrm g})\neq \emptyset$. Then the two parts of the
    bipartition of an irreducible constituent after each operation
    $J_{\vec{S}_{i}\times \sfW'_{n_{i+1}}}^{\sfW'_{n_{i}}}(\sgn \otimes \underline{\ \ \ })$ are different. Let
    \[i_{0}:= \min\Set{i| (2i-1,2i)\in \CPPs(\ckcO_{\mathrm g})}.\]
    % $(2i_{0}-1,2i_{0})$ be the element in such that $i_{0}$ is minimal.
    Then we will have a bijection
    \[
      \tA'(\ckcO):= \set{\wp\in \bZ_{2}[\CPPs(\ckcO_{\mathrm g})]|(2i_{0}-1,2i_{0})\notin \wp} \longrightarrow \ckLC_{\ckcO_{\mathrm g}}
    \]
    which records the bifurcation when attaching odd length columns. More precisely, by Lusztig's formula of $J$-induction, a bijection is given by sending
      $\wp$ to $\cktau_{\wp}:= (\cktau_{L},\cktau_{R})$, where
    \[
      (\bfrr_{i}(\cktau_{L}),\bfrr_{i}(\cktau_{R})) := \begin{cases} (\half\bfrr_{2i}(\ckcO_{\mathrm g}),\half\bfrr_{2i-1}(\ckcO_{\mathrm g}))
        & \text{if } (2i-1,2i)\notin \wp ,\\
        (\half\bfrr_{2i-1}(\ckcO_{\mathrm g}),\half\bfrr_{2i}(\ckcO_{\mathrm g}))
        & \text{if } (2i-1,2i)\in \wp .\\
      \end{cases}
    \]
    By interchanging the two parts of a bipartition belonging to $\ckLC_{\ckcO_{\mathrm g}}$, we thus obtain a bijection of $\tA'(\ckcO)$ with $\LC_{\ckcO_{\mathrm g}}$, which sends $\wp$ to $\tau_{\wp}$.


    \trivial[h]{ Note that the trivial representation of the trivial group is
      represented by the symbol
      \[
        \binom{0,1, 2, \cdots, l-1}{0,1,2, \cdots, l-1}_{I}.
      \]
      % Induce to include all $2l$-length columns yields
      % \[
      %   \binom{r'_{2l+1}+0,r'_{2l+1}+1,r'_{2l+1}+2,\cdots, r'_{2l+1}+l\phantom{-1}}{ r'_{2l+1}+0,r'_{2l+1}+1, \cdots, r'_{2l+1}+l-1}.
      % \]

      Now move the shorter columns. When even columns $(2i)^{2r'_{2i}}$
      occurs, it adds $(i)^{r'_{2i}}$ columns on the left and $(i)^{r'_{2i}}$ on
      the right. When odd columns $(2i-1)^{r'_{2i-1}}$ occur, the bifurcation
      happens: one can
      \begin{itemize}
        \item attach columns $(i-1)^{r'_{2i-1}}$ on the left and columns
              $(i)^{r'_{2i-1}}$ on the right, which corresponds to
              $(2i-1,2i)\neq \wp$, or
        \item attach columns $(i)^{r'_{2i-1}}$ on the left and columns
              $(i-1)^{r'_{2i-1}}$ on the right, which corresponds to
              $(2i-1,2i)\in \wp$.
      \end{itemize}
      Note that when we first encounter the longest odd column, we make the
      choice that the size of left part is larger than that of the right part.
      Now If $(2i-1,2i)\notin \wp$,
      \[
        \begin{split}
          \bfrr_{i}(\cktau_{L}) & = \sum_{l\geq 2i} r'_{l}
          = \half\bfrr_{2i}(\ckcO_{\mathrm g})\\
          \bfrr_{i}(\cktau_{R}) & = \sum_{l\geq 2i-1} r'_{l} = \half\bfrr_{2i-1}(\ckcO_{\mathrm g})
        \end{split}
      \]
      if $(2i-1,2i)\in \wp$,
      \[
        \begin{split}
          \bfrr_{i}(\cktau_{L}) & = \sum_{l\geq 2i-1} r'_{l}
          = \half\bfrr_{2i-1}(\ckcO_{\mathrm g})\\
          \bfrr_{i}(\cktau_{R}) & = \sum_{l\geq 2i} r'_{l} = \half\bfrr_{2i}(\ckcO_{\mathrm g})
        \end{split}
      \]
    }


    Now we consider the bad parity part, where each row has
    odd length.
    %For a partition $\cO$, we set
    %\[
    % \sfS_{\cO} := \prod_{i\in \bN^{+}}\sfS_{\bfcc_{i}(\cO)}
    %\]
    %so that $j_{\sfS_{\cO}}^{\sfS_{\abs{\cO}}}\sgn = \cO$.



    Suppose $\ckcOpb$ is nonempty.
    % such that
    % \[
    %   \ckcO_{\mathrm b} = (c_{0},c_{1}, c_{1}, c_{2},c_{2}, \cdots, c_{k}, c_{k})
    % \]
    % where
    Let $2k+1=\bfrr_{1}(\ckcOpb)$ and $c_{i} = \bfcc_{2i+1}(\ckcOpb)$.
    Now
    \[
      W_{\lamckb} = \sfW_{c_{0}} \times \sfS_{2c_{1}} \times \sfS_{2c_{2}}\times \cdots \times \sfS_{2c_{k}}
    \]
    and %$J_{W_{\lamckb}}^{W_{\mathrm b}}\sgn$ is irreducible by
    \[
      \begin{split}
        \cktau_{\mathrm b} &:= J_{W_{\lamckb}}^{\Wb} \sgn
        = \big((c_{1},c_{2},\cdots, c_{k}),(c_{0},c_{1}, \cdots, c_{l})\big)\\
        & = \big([\half(\bfrr_{1}(\ckcOpb)-1),\half(\bfrr_{2}(\ckcOpb)-1),\cdots, \half(\bfrr_{c_{0}}(\ckcOpb)-1)],\\
        & \hspace{2em} [\half(\bfrr_{1}(\ckcOpb)+1),\half(\bfrr_{2}(\ckcOpb)+1),\cdots, \half(\bfrr_{c_{0}}(\ckcOpb)+1)]\big)
      \end{split}
    \]
    is irreducible by \cite{Lu}*{(4.5.4)}. Tensoring with sign yields the
    formula of $\tau_{\mathrm b}$. Moreover, by the fake degree formula (see
    \cite{Carter}*{p~376}), we have
    \[
      \tau_{\mathrm b} = \cktau_{\mathrm b}\otimes \sgn = j_{\sfS_{\cO'_{\mathrm b}}}^{\sfW_{n_{\mathrm b}}} \sgn,
    \]
    where
    $ \cOpb:= (\ckcOpb)^{t}$
    % $:=(\bfrr_{1}(\ckcO),\bfrr_{4}(\ckcO),\cdots , \bfrr_{2c_{0}}(\ckcO))$,
    and $\sfS_{\cOpb} := \prod_{1\leq i\leq c_{0}}\sfS_{\bfrr_{i}(\ckcOpb)}$. The proof for the first part is now complete.

    \medskip \def\ckfll{\check{\fll}}

    Next we give the main steps of the proof for the second part, when $\star = \wtC$.
    %the proof of \eqref{eq:dBV.W}.

    Recall the definition of the metaplectic Barbasch-Vogan dual in
    \cite{BMSZ1}. A key property is that this duality map commutes with parabolic induction: Suppose
    $\ckfll\subset \ckfgg$ is a parabolic subalgebra of $\ckfgg$ and
    $\fll$ is the corresponding parabolic subalgebra in $\fgg$, then
    \begin{equation*}%\label{eq:inddBV}
      \dBV(\ckcO) =  \Ind_{\fll}^{\fgg}(\dBV(\ckcO_{\ckfll}))
    \end{equation*}
    for each nilpotent orbit $\ckcO$ in $\ckfgg$ such that
    $\ckcO_{\ckfll}:=\ckcO\cap \ckfll\neq \emptyset$. This is clear by reducing
    to the type $B$ case, as in \cite{BMSZ1}*{Proposition~3.8}. By removing pairs
    of rows with the same lengths in $\ckcO$, we are reduced to check the equality in \eqref{eq:dBV.W}
    in the case when $\ckcOpb=\emptyset$ and
    $\bfrr_{2i-1}(\ckcO_{\mathrm g})>\bfrr_{2i}(\ckcO_{\mathrm g})$ for all $i$ such that
    $i\leq \bfcc_{1}(\ckcO_{\mathrm g})$. In this case, both sides of \eqref{eq:dBV.W}
    can be easily computed and are equal to
    \[
      (\bfrr_{1}(\ckcOg)-1, \bfrr_{2}(\ckcOg)+1, \cdots,\bfrr_{2c-1}(\ckcOg)-1,\bfrr_{2c}(\ckcOg)+1),
    \]
    where $c = \min\set{i|\bfrr_{2i+1}(\ckcOg)=0}$. The general case follows
    from the aforementioned compatibility with parabolic induction.
    }
  \end{proof}



    % Now we compare the metaplectic dual defined in \cite{BMSZ1} with the Weyl
    % group representations.
    \trivial[h]{ Compare Sommer's description of Springer correspondence we
      deduce that the RHS is
      \[
        \cO_{\mathrm g} = (\bfrr_{1}(\ckcO_{1})-1,\bfrr_{2}(\ckcO_{2})+1,\bfrr_{3}(\ckcO_{3})+1, \cdots, \bfrr_{2l-1}(\ckcO_{2l-1})-1,\bfrr_{2l}(\ckcO_{2l})+1)
      \]
      The LHS is calculated by $((((\ckcO^{t})_{D})^{+})^{-})_{C}$. We write
      $R_{i}=\bfrr_{i}(\ckcO)=2r_{i}$. Now under our assumption,
      $R_{2i-1}>R_{2i}$, we have
      \[
        \begin{split}
          ((((\ckcO^{t})_{D})^{+})^{-})_{C} &=
          ((((R_{1},R_{2}, \cdots, R_{2l-1},R_{2l})_{D})^{+})^{-})_{C}\\
          &=((R_{1}-1,R_{2}, \cdots, R_{2l-1},R_{2l},1))_{C}\\
          &=(R_{1}-1,R_{2}+1, \cdots, R_{2l-1}-1,R_{2l}+1)\\
        \end{split}
      \]
      So the proof is done.

    }

    %At last, one can see that
    %$\dBV(\ckcO_{\mathrm b}\cuprow \ckcO_{\mathrm g}) = \ckcO_{\mathrm b}^{t} \cupcol \dBV(\ckcO_{\mathrm g})$
    %using \eqref{eq:inddBV}.


  % \begin{remark}
  %   When $\star=\wtC$, one can see that
  %   $\dBV(\ckcO_{\mathrm b}\cuprow \ckcO_{\mathrm g}) = \ckcO_{\mathrm b}^{t} \cupcol \dBV(\ckcO_{\mathrm g})$
  %   using \eqref{eq:inddBV}. \trivial[]{ This could be checked using the
  %   formula of Springer correspondence directly, see Sommer's. }
  % \end{remark}
%

Recall that $\Wg' =\sfW'_{n_{\mathrm g}}$ when $\star \in \set{\wtC,D,D^{*}}$. Since the representation theory of $\sfW_{n}$ is more elementary than that of
$\sfW'_{n}$, we prefer to work with $\sfW_n$ instead of $\sfW_n'$ in some situations. For this reason, when $\star \in \set{\wtC,D,D^{*}}$ we also define
\begin{equation}\label{eq:ttauwp}
\wttau_{\wp} = (\imath_{\wp},\jmath_{\wp})\in \Irr(\sfW_{n_\mathrm g}), \qquad \wp\in {\mathrm A}(\ckcO).
\end{equation}
See \eqref{eq:tauwp} for the description of $\imath_{\wp}$ and $\jmath_{\wp}$.

For later use, we record the following lemma, which follows immediately from our explicit descriptions of  $\tau_{\mathrm b}$ and $\tau_{\wp}$.

\begin{lem}\label{lem:WLcell}
Let $\wp\in {\mathrm A}(\ckcO)$. If $\star=\wtC$, then
  \[
    \Ind_{\sfW_{n_{\mathrm b}}\times \sfW'_{n_{\mathrm g}}}^{\sfW_{n_{\mathrm b}}\times \sfW_{n_{\mathrm g}}} \tau_{\mathrm b}\otimes\tau_{\wp}  =
    \begin{cases}
       \tau_{\mathrm b}\otimes \wttau_{\emptyset}, &\quad  \text{if } n_{\mathrm g}=0; \\
      (\tau_{\mathrm b}\otimes \wttau_{\wp}) \oplus ( \tau_{\mathrm b}\otimes \wttau_{\wp^{c}}),
      &\quad \text{otherwise}.
    \end{cases}
  \]
If $\star\in\set{D,D^{*}}$, then
  \[
    \Ind_{\sfW'_{n_{\mathrm b}}\times \sfW'_{n_{\mathrm g}}}^{\sfW_{n_{\mathrm b}}\times \sfW_{n_{\mathrm g}}} \tau_{\mathrm b}\otimes \tau_{\wp} \cong
    \begin{cases}
      \wttau_{\mathrm b}, & \quad \text{if } n_{\mathrm g}=0; \\
      ( \wttau_{\mathrm b}\otimes \wttau_{\wp})  \oplus ( \wttau_{\mathrm b}\otimes \wttau_{\wp}^{s}),
      &\quad \text{otherwise}.
    \end{cases}
  \]
  Here $\wttau_{\mathrm b} = \Ind_{\sfW'_{n_{\mathrm b}}}^{\sfW_{n_{\mathrm b}}}\tau_{\mathrm b}$ and $\wttau_{\wp}^{s}:= \wttau_{\wp}\otimes \varepsilon$ (recall the quadratic character $\varepsilon$ from \eqref{defep}).

  %\brsgn \neq \wttau_{\wp}$.

  % Then we have a bijection
  % \[
  %     \begin{array}{lccccccc}
  %       \tA(\ckcO)&=&\tA(\ckcO_{\mathrm g}) & \longrightarrow & \tLC(\ckcO_{\mathrm g})
  %       & \longrightarrow & \tLC(\ckcO)\\
  %                   &  &\wp & \mapsto & \wttau_{\wp} &
  %                                                    \mapsto & \tau_{\mathrm b}\otimes \wttau_{\wp}.
  %     \end{array}
  % \]

 \trivial[h]{
  Let
\[
  \tLV_{\ckcO}:= \Ind_{\sfW'_{n_{\mathrm g}}\times \sfW_{n_{\mathrm b}}}^{\sfW_{n_{\mathrm g}}\times \sfW_{n_{\mathrm b}}}\LV_{\ckcO}
\]
and $\tLC_{\ckcO}$ be the set of irreducible constituents of the (multiplicity free) $\sfW_{n_{\mathrm g}}\times \sfW_{n_{\mathrm b}}$-module $\tLV_{\ckcO}$.
Then we have a
bijection:
  \[
      \begin{array}{lccccccc}
        \tA(\ckcO)&\cong&\tA(\ckcO_{\mathrm g}) & \longrightarrow & \tLC(\ckcO_{\mathrm g})
        & \longrightarrow & \tLC_{\ckcO}\\
                  &  &\wp & \mapsto & \wttau_{\wp}
        & \mapsto & \wttau_{\wp}\otimes \tau_{\mathrm b}.
      \end{array}
  \]
  }
\end{lem}

\subsection{From coherent continuation representation to counting}




We have defined in \eqref{defpbp2222} the set $\PBPs(\ckcO)$ when $\ckcO$ has good parity. Similarly, we make the following definition in the bad parity case.
\begin{defn}
  Let $\PBPs(\ckcOb)$ be the set of all triples
  $\uptau = (\imath,\cP)\times(\jmath,\cQ)\times \star $ where $(\imath,\cP)$ and
  $(\jmath,\cQ)$ are painted partitions such that
  \begin{itemize}
    \item $(\imath,\jmath) = (\tau_{L,\mathrm b},\tau_{R,\mathrm b})$ (see \eqref{eq:taub});
    \item the image of $\cP$ is contained in
          \[
          \begin{cases}
            \set{\bullet, c,d},  & \text{if } \star\in \set{B,\wtC}; \\
            \set{\bullet, d},  & \text{if } \star\in \set{C,D};\\
            \set{\bullet},  & \text{if } \star\in \set{C^{*},D^{*}};\\
          \end{cases}
          \]
    \item the image of $\cQ$ is contained in

          \[
          \begin{cases}
            \set{\bullet, c},  & \text{if } \star\in \set{C,D};\\
            \set{\bullet},  & \text{if } \star\in \set{B,\wtC, C^{*},D^{*}}.\\
          \end{cases}
          \]
  \end{itemize}
\end{defn}

%We define the partition $\ckcOpb$ by $\ckcOb = 2\ckcOpb$. To be more precise, $\bfrr_{i}(\ckcO'_{\mathrm b}):= \bfrr_{2i}(\ckcO_{\mathrm b})$, for all $i\in \bN^{+}$. Let $\cO'_{\mathrm b}:= (\ckcO'_{\mathrm b})^{t}$.
  %\item $\cO_{\mathrm b}:= \cO'_{\mathrm b}\cupcol \cO'_{\mathrm b}$.
 %$\ckcOpb$ is a partition of $\half\abs{\ckcOb}$.

% Recall from \eqref{Gpb} the group
% \[
%   G'_{\mathrm b} := \begin{cases}
%     \GL_{n_{\mathrm b}}(\bR), & \text{if } \star \in \set{B,C,D}; \\
%     \widetilde{\GL}_{n_{\mathrm b}}(\bR), & \text{if } \star = \wtC; \\
%     \GL_{\frac{n_{\mathrm b}}{2}}(\bH), & \text{if } \star \in \set{C^{*},D^{*}}.\\
%   \end{cases}
% \]
%When $\star=\wtC$, let $\Unip_{\ckcOpb}(\Gpb)$ be the set of genuine unipotent representations of $\Gpb$ which is naturally identified with $\Unip_{\ckcOpb}(\GL(n_{\mathrm b},\bR))$.

% \begin{prop}\label{prop:BP.PP} In all cases,
% \[
%     \begin{split}
%       \sharp(\PBP_{\star}(\ckcO_{\mathrm b})) = \sharp(\PP_{\star '}(\ckcO'_{\mathrm b})) = \sharp(\Unip_{\ckcO'_{\mathrm b}}(G'_{\mathrm b})),
%     \end{split}
%   \]
% where
% \begin{equation}\label{def:star'}
% \star ':= \begin{cases}
%     A^{\bR}, & \text{if } \star \in \set{B,C,\wtC,D}; \\
%     A^{\bH}, & \text{if } \star \in \set{C^{*},D^{*}}.\\
%   \end{cases}
%   \end{equation}
% %is the set of painted partitions of type $A$ or $A^{\bH}$ attached to $\ckcO'_{\mathrm b}$.
% \end{prop}

% \begin{proof} Suppose that $\star \in \Set{C^{*},D^{*}}$. Then
% \[
%     \begin{split}
%       \sharp(\PBP_{\star}(\ckcO_{\mathrm b};\tau_{\mathrm b}))= \sharp(\PP_{A^{\bH}}(\ckcO'_{\mathrm b}))= 1.
%     \end{split}
% \]
%   Suppose that $\star \in \Set{B,C,\wtC,D}$.
%   %Recall the Young diagrams $\tau_{L,b}$ and $\tau_{R,b})$ from \eqref{eq:taub} and \eqref{eq:taub2}.
%    It is easy to see that we have a bijection
%   \[
%     \begin{array}{ccc}
%       \PBP_{\star}(\ckcO_{\mathrm b}) &  \rightarrow & \PP_{A^{\bR}}(\ckcO'_{\mathrm b}),\\
%       (\tau_{L,b},\cP)\times (\tau_{R,b},\cQ)\times & \mapsto & ((\ckcO'_{\mathrm b})^t,\cP'),
%     \end{array}
%   \]
%   where $\cP'$ is defined by the condition that
%   \[
%     \cP(\bfcc_{j}(\tau_{L,b}),j)=d \Longleftrightarrow \cP'(\bfrr_{j}(\ckcO'_{\mathrm b}),j)=d, \quad \textrm{ for all } j=1,2,\cdots, \bfcc_{1}(\ckcO'_{\mathrm b}).
%   \]
%   The last equality is in Theorems \ref{thm:mainR} and \ref{thm:mainH}.   \end{proof}


%   \trivial[h]{
%     % Let $\tau' = \ckcO'^{t}_{\mathrm b}$ and $\tau_{\mathrm b}=(\tau_{L,b}, \tau_{R,b})$. Here
%     % $\tau_{L,b}, \tau_{R,b}$.
%     Now the claim follows for the fact that the bottom rows in $\uptau_{L}$ can
%     be filled by $\bullet/c$ or $d$ and
%     \[
%       \bfcc_{i}(\tau_{L,b}) = \bfcc_{j}(\tau_{L,b}) \Leftrightarrow \bfcc_{i}(\cOpb) = \bfcc_{j}(\cOpb) \quad \textrm{ for all } i,j\in \bN^{+}.
%     \]
%   }





 We introduce some additional notation. For each bipartition $\tau$, let
\[
  \PBP_{\star}(\tau) := \Set{ \uptau \text{ is a painted bipartition }\mid  \star_{\uptau}=\star, (\imath_{\uptau},\jmath_{\uptau}) = \tau}
  % \uptau=(\imath, \cP)\times (\jmath,\cP)\times \alpha|}
\]
and
\[
  \PBP_{G}(\tau) := \Set{\uptau \text{ is a painted bipartition }\mid G_{\uptau}=G, (\imath_{\uptau},\jmath_{\uptau}) = \tau}.
  % \uptau=(\imath, \cP)\times (\jmath,\cP)\times \alpha|}
\]
Put
\[
  \tPBP_{\star}(\ckcO) := %\bigsqcup_{\tau\in\LC(\ckcOg)}\PBP_{\star}(\tau).
  \bigsqcup_{\wp \subseteq \CPP(\ckcO)}\PBP_{\star}(\wttau_{\wp})
\]
and
\[
  \tPBP_{G}(\check \CO):=   \bigsqcup_{\wp \subseteq \CPP(\ckcO)}\PBP_{G}(\wttau_{\wp}),
 \]
where $\wttau_{\wp} := (\imath_{\wp},\jmath_{\wp})$ (see \eqref{eq:ttauwp}).


%Recall the group $G_\mathrm g$ from \eqref{gg00} (with $l=\nnb$).  Its Langlands dual group is identified with $\check G_\mathrm g$.

Note the if  $\star =D^{*}$, then $\check \CO$ is not $G$-relevant if and only if $\check \CO$ has bad parity and type II.

\begin{prop}\label{prop:countBCD}
 If  $\star =D^{*}$ and $\check \CO$ is not $G$-relevant, then
  \[
  \sum_{\bar \wp\in \bar{\mathrm A}(\ckcO)} [\tau_{\mathrm b}\otimes \tau_{\bar \wp}: \Coh_{\Lambda} (\CK'(G))]=0.
  \]
In all other cases,   % \[
  %   [\tau_{\mathrm b}: \cC_{\mathrm b}] = \PBP_{\star,b}(\ckcO_{\mathrm b}) = \PBP_{G'}(\ckcO'_{n_{\mathrm b}}) = \Unip_{\ckcO'_{\mathrm b}}(G'_{n_{\mathrm b}})
  % \]
  % and
  % \[
  %   \sum_{\tau\in \LC_{\ckcO_{\mathrm g}}} [\tau:\cC_{\mathrm g}] = \PBP_{\mathrm g}(\ckcO_{\mathrm g}).
  % \]
 \[
 \sum_{\bar \wp\in \bar{\mathrm A}(\ckcO)} [\tau_{\mathrm b}\otimes \tau_{\bar \wp}: \Coh_{\Lambda} (\CK'(G))]=
           \sharp (\PBP_{\star}(\ckcO_{\mathrm b}))\cdot \sharp(\tPBP_{G_{\mathrm g}}(\ckcO_{\mathrm g})).
           \]
 \end{prop}



 \begin{proof}
   We use the following
  formulas (\cite{Mc}*{p220 (6)})  to compute the multiplicities:
  \begin{equation}\label{eq:indSW}
    \begin{split}
      \Ind_{\sfH_{t}}^{\sfW_{2t}} \tilde{\varepsilon}& \cong   \bigoplus_{\sigma\in \Irr(\sfS_{t})} (\sigma,\sigma),\\
      \Ind_{\sfH_{t}}^{\sfW'_{2t}} \tilde{\varepsilon} &\cong  \bigoplus_{\sigma\in \Irr(\sfS_{t})} (\sigma,\sigma)_{I},\\
      \Ind_{\sfS_{t}}^{\sfW_{t}}\sgn &\cong \bigoplus_{a+b=t}\Ind_{\sfW_{a}\times \sfW_{\mathrm b}}^{\sfW_{t}} \bsgn\otimes \sgn \cong \bigoplus_{a+b=t} ([a]_{\mathrm{col}},[b]_{\mathrm{col}}),\\
      \Ind_{\sfS_{t}}^{\sfW_{t}} 1 &\cong \bigoplus_{a+b=t}\Ind_{\sfW_{a}\times \sfW_{\mathrm b}}^{\sfW_{t}} 1 \otimes \epsilon \cong \bigoplus_{a+b=t} ([a]_{\mathrm{row}},[b]_{\mathrm{row}}),
    \end{split}
\end{equation}
where $t\in \bN$.

We skip the details when $\star \in \set{B,\wtC, C,D,C^{*}}$, and present the computation for $\star = D^{*}$, which is the most complicated case.
 Suppose that $\star=D^*$ so that $G=\oO^*(2n)$.  If $\check \CO$ has bad parity, then
  \begin{eqnarray*}
 && \sum_{\bar \wp\in \bar{\mathrm A}(\ckcO)} [\tau_{\mathrm b}\otimes \tau_{\bar \wp}: \Coh_{\Lambda} (\CK'(G))] \medskip \\
 &=&[\tau_{\mathrm b} : \Coh_{\Lambda} (\CK'(G))] \medskip \\
 &=& [\tau_\mathrm b:\Ind_{\sfH_{t}}^{\sfW'_{2t}} \tilde{\varepsilon}] \medskip \\
& =& \begin{cases}
    1=\sharp (\PBP_{\star}(\ckcO_{\mathrm b}))\cdot \sharp(\tPBP_{G_{\mathrm g}}(\ckcO_{\mathrm g})),  \qquad &  \text{if $\check \CO$ is $G$-relevant,}\medskip \\
          0, \qquad & \text{if $\check \CO$ is not $G$-relevant,.}
       \end{cases}
  \end{eqnarray*}


Now we assume that $\check \CO$ does not have bad parity so that $n_\mathrm g>0$.
 Put
 \[
   \sfW'_{\nbb, \ngg}:=\sfW'_n\cap (\sfW_{\nbb}\times \sfW_{\ngg})\supseteq \sfW'_{\nbb}\times \sfW'_{\ngg}= W_{\mathrm b}\times W_{\mathrm g}.
 \]


Recall that
  \[
    \wttau_{\mathrm b} := \Ind_{\sfW'_{\nbb}}^{\sfW_{\nbb}} \tau_{\mathrm b} = (\cOpb,\cOpb)\in \Irr(\sfW_{\nnb})
    \]
    and
    \[
    \wttau_{\wp}: = (\imath_{\wp},\jmath_{\wp})\in \Irr(\sfW_{\nng}) \quad \textrm{ for all } \wp \subseteq \CPP(\ckcOg).
  \]
  Note that $\imath_{\wp}\neq \jmath_{\wp}$ since
  $\bfcc_{1}(\imath_{\wp})> \bfcc_{1}(\jmath_{\wp})$, which implies that
  \begin{equation*}%\label{eq:W''}
    \Ind_{\sfW'_{\nbb}\times \sfW'_{\ngg}}^{\sfW_{n_{\mathrm b},n_{\mathrm g}}'} \tau_{\mathrm b}\otimes \tau_{\wp}
    \cong (\wttau_{\mathrm b}\otimes \wttau_{\wp})|_{\sfW_{n_{\mathrm b},n_{\mathrm g}}'}.
  \end{equation*}

  \trivial[h]{ When $\nbb=0$, $W'' = W_{\mathrm g} = \sfW'_{\ngg}$ and so
    $\wttau_{\wp}|_{\sfW'_{\ngg}} = \tau_{\wp}$.

    Now we assume $\nbb\neq 0$ and $\ngg\neq 0$ (the general case). This
    follows from the following points
    \begin{itemize}
      \item the dimension of the two sides are equal ($W_{\mathrm b}\times W_{\mathrm g}$ has
            index $2$ in $W''$).
      \item
            \[
            \begin{split}
              &[\Ind_{W_{\mathrm b}\times W_{\mathrm g}}^{W''}\tau_{\mathrm b}\otimes \tau_{\wp}:(\wttau_{\mathrm b}\otimes \wttau_{\wp})|_{W''}] \\
              =& [\Ind_{\sfW'_{n_{\mathrm b}}\times \sfW'_{n_{\mathrm g}}}^{\sfW_{n_{\mathrm b}}\times \sfW_{n_{\mathrm g}}}\tau_{\mathrm b}\otimes \tau_{\wp}:\wttau_{\mathrm b}\otimes \wttau_{\wp}]\\
              =& [\wttau_{\mathrm b}\otimes \wttau_{\wp} \oplus \wttau_{\mathrm b}\otimes (\wttau_{\wp}\otimes \brsgn):\wttau_{\mathrm b}\otimes \wttau_{\wp}] =1\\
            \end{split}
            \]
            where $\wttau_{\wp}\otimes \brsgn$ has the bipartition obtained by
            switching the left and right side of $\wttau_{\wp}$.
      \item the LHS is irreducible, by
            \[
            \begin{split}
              & [\Ind_{W_{\mathrm b}\times W_{\mathrm g}}^{W''}\tau_{\mathrm b}\otimes \tau_{\wp}:
              \Ind_{W_{\mathrm b}\times W_{\mathrm g}}^{W''}\tau_{\mathrm b}\otimes \tau_{\wp}]_{W''}\\
              =&  [\tau_{\mathrm b}\otimes \tau_{\wp} : (\Ind_{W_{\mathrm b}\times W_{\mathrm g}}^{W''}\tau_{\mathrm b}\otimes \tau_{\wp})|_{W_{\mathrm b}\times W_{\mathrm g}}]\\
              =& [\tau_{\mathrm b}\otimes \tau_{\wp} : \tau_{\mathrm b}\otimes \tau_{\wp} + (\cOpb,\cOpb)_{II} \otimes \tau_{\wp} ] = 1
            \end{split}
            \]
    \end{itemize}
  }

  For ease of notation, write $\sfW'':=\sfW_{n_{\mathrm b},n_{\mathrm g}}'$. Put
  \[
    \cC_{\mathrm b}:=   \Ind_{\sfH_{\frac{n_\mathrm b}{2}}}^{\sfW_{n_\mathrm b}}\tilde{\varepsilon} \quad\textrm{and}\quad  \cC_\mathrm g:=  \bigoplus_{\substack{2t+a=n_\mathrm g}} \Ind_{\sfH_{t} \times \sfS_{a}}^{\sfW'_{n_\mathrm g}}
         \tilde{\varepsilon} \otimes \sgn,
           \]
           where $\cC_\mathrm b$ is viewed as a representation of $\sfW_{n_\mathrm b}'$ by restriction.
   For every finite group $E$ and any  two finite-dimensional  representations $V_1$ and $V_2$ of $E$, put
  \[
    [V_1, V_2]_E:=\dim \Hom_E(V_1, V_2).
  \]
   For each $\wp\in \mathrm A (\ckcOg)$, we have that
  \[
    \begin{split}
    [\tau_{\mathrm b}\otimes \tau_{\wp}: \Coh_{\Lambda} (\CK'(G))]
      = & [\tau_{\mathrm b}\otimes \tau_{\wp} :
      \Ind_{\sfW'_{n_{\mathrm b}}\times \sfW'_{n_{\mathrm g}}}^{\sfW''} \cC_{\mathrm b} \otimes \cC_{\mathrm g}]_{\sfW'_{n_{\mathrm b}}\times \sfW'_{n_{\mathrm g}}}\\
      = & [\Ind_{\sfW'_{n_{\mathrm b}}\times \sfW'_{n_{\mathrm g}}}^{\sfW''} \tau_{\mathrm b}\otimes \tau_{\wp} :
      \Ind_{\sfW'_{n_{\mathrm b}}\times \sfW'_{n_{\mathrm g}}}^{\sfW''} \cC_{\mathrm b} \otimes \cC_{\mathrm g}]_{\sfW''}\\
      = & [(\wttau_{\mathrm b}\otimes \wttau_{\wp})|_{\sfW''}:
      \Ind_{\sfW'_{n_{\mathrm b}}\times \sfW'_{n_{\mathrm g}}}^{\sfW''} \cC_{\mathrm b} \otimes \cC_{\mathrm g}]_{\sfW''}\\
      = & [\wttau_{\mathrm b}\otimes \wttau_{\wp}:
      \Ind_{\sfW'_{n_{\mathrm b}}\times \sfW'_{n_{\mathrm g}}}^{\sfW_{n_{\mathrm b}}\times \sfW_{n_{\mathrm g}}} \cC_{\mathrm b} \otimes \cC_{\mathrm g}]_{\sfW_{n_{\mathrm b}}\times \sfW_{n_{\mathrm g}}}\\
       = & [\wttau_{\mathrm b}:\Ind_{\sfW'_{n_{\mathrm b}}}^{\sfW_{n_{\mathrm b}}} \cC_{\mathrm b}]_{\sfW_{n_{\mathrm b}}}\cdot
           [\wttau_{\wp}:\Ind_{\sfW'_{n_{\mathrm g}}}^{\sfW_{n_{\mathrm g}}} \cC_{\mathrm g}]_{\sfW_{n_{\mathrm g}}}\\
      =& \sharp(\PBP_{\star}(\ckcOb)))\cdot \sharp(\PBP_{\Gg}(\wttau_{\wp})).
    \end{split}
  \]
  The last equality follows from the branching rules
  % of $\sfW_{n}$ in
  % \eqref{eq:CC.C} and
  \eqref{eq:indSW} and
  Pieri's rule (\cite[Corollary 9.2.4]{GW}).
  % , where the factor $\sfH_{t}$ amounts to
  % painting ``$\bullet$'' on $\imath_{\wp}$ and $\jmath_{\wp}$, the factor
  % $\sfS_{a}$ amounts to painting ``$s$'' on $\imath_{\wp}$ and painting ``$r$''
  % on $\jmath_{\wp}$, each ``permissible" way of painting (see \Cref{def:pbp1})
  % contributing $1$ to the multiplicity
  % $[\wttau_{\wp}:\Ind_{\sfW'_{n_{\mathrm g}}}^{\sfW_{n_{\mathrm g}}} \cC_{\mathrm g}^{p_{\mathrm g},q_{\mathrm g}}]_{\sfW_{n_{\mathrm g}}}$.
\end{proof}


Now \Cref{prop:countBCD}, \Cref{lem:Lcell} and \eqref{boundc22} imply the following corollary.
\begin{cor}\label{prop:countBCD22}
 The inequality
  % \[
  %   [\tau_{\mathrm b}: \cC_{\mathrm b}] = \PBP_{\star,b}(\ckcO_{\mathrm b}) = \PBP_{G'}(\ckcO'_{n_{\mathrm b}}) = \Unip_{\ckcO'_{\mathrm b}}(G'_{n_{\mathrm b}})
  % \]
  % and
  % \[
  %   \sum_{\tau\in \LC_{\ckcO_{\mathrm g}}} [\tau:\cC_{\mathrm g}] = \PBP_{\mathrm g}(\ckcO_{\mathrm g}).
  % \]
  \begin{eqnarray*}
 && \sharp(\Unip_{\ckcO}(G))\\
  & \leq & \begin{cases}
    0,  \qquad &  \text{if $\star=D^*$ and $\check \CO$ is not $G$-relevant;}\medskip \\
          \sharp (\PBP_{\star}(\ckcO_{\mathrm b}))\cdot \sharp(\tPBP_{G_{\mathrm g}}(\ckcO_{\mathrm g})), \qquad & \text{otherwise.}
       \end{cases}
  \end{eqnarray*}
  holds.
 \end{cor}



When $\ckcO $ has good parity, we will see from \Cref{prop:PBP1} and \Cref{prop:PBP2} that
  \[
    \sharp(\tPBP_{G}(\ckcO)) =
    \left\{
    \begin{array}{ll}
       \sharp (\PBP_{G}(\ckcO)),  & \hbox{if $\star\in \{C^*,D^*\}$}; \smallskip\\
       2^{\sharp(\CPPs(\check \CO))} \cdot \sharp (\PBP_{G}(\ckcO)),  &\hbox{if $\star\in \{B, C,D,\widetilde {C}\}$}.
    \end{array}
  \right.
  \]
 \Cref{prop:countBCD22} will thus imply Theorem \ref{countup}.% in the introductory section.

% \begin{prop}\label{prop:countBCD} Assume that $\check \CO$ has $\star$-good parity.  Then
%   % \[
%   %   [\tau_{\mathrm b}: \cC_{\mathrm b}] = \PBP_{\star,b}(\ckcO_{\mathrm b}) = \PBP_{G'}(\ckcO'_{n_{\mathrm b}}) = \Unip_{\ckcO'_{\mathrm b}}(G'_{n_{\mathrm b}})
%   % \]
%   % and
%   % \[
%   %   \sum_{\tau\in \LC_{\ckcO_{\mathrm g}}} [\tau:\cC_{\mathrm g}] = \PBP_{\mathrm g}(\ckcO_{\mathrm g}).
%   % \]
%   \[
%   \sharp (\Unip_{\ckcO}(G))\leq
%      \sharp ({\tPBP_{G}(\ckcO)}).
%   \]
%  \end{prop}


% \begin{proof} Since we are at the good parity case, Lemma \ref{lem:Lcell} and \eqref{boundc22} imply that
% \[
% \sharp (\Unip_{\ckcO}(G))
%   \leq \sum_{\wp\in \bar{\mathrm A}(\ckcO)} [\tau_{\wp}: \Coh_{\Lambda}(\CK'(G))].
%   \]
%     For the rest of the proof, the necessary computation for all cases are similar. We will present the computation for $\star = D^{*}$, and skip the details when $\star \in \set{B,\wtC, C,D,C^{*}}$.

% Suppose that $\star=D^*$ so that $G=\oO^*(2n)$. Then $W_{\Lambda} = \sfW'_{n}$, and Proposition \ref{prop:cohBCD44} implies that
% \[
% \Coh_{\Lambda}(\CK'(G))\cong   \bigoplus_{\substack{2t+a=n}} \Ind_{\sfH_{t} \times \sfS_{a}}^{\sfW'_{n}}
%          \tilde{\varepsilon} \otimes \sgn.
%          \]

%          For each $\wp\in \bar{\mathrm A}(\ckcO)$, we have that
% \[
% \begin{split}
%  &[\tau_{\wp} : \Coh_{\Lambda}(\CK'(G))]\\
%  = &
% [\wttau_{\wp}|_{\sfW'_{n}}:\sum_{2t+a=n} \Ind_{\sfH_{t}\times \sfS_{a}}^{\sfW'_{n}}\tilde{\varepsilon} \otimes \sgn]_{\sfW'_{n}}\\
% = & [\wttau_{\wp}: \sum_{2t+a=n} \Ind_{\sfH_{t}\times \sfS_{a}}^{\sfW'_{n}}\tilde{\varepsilon} \otimes \sgn]_{\sfW_{n}}\\
% = & \PBP_{\star}(\ttau_\wp)
% \end{split}
% \]
%   The last equality follows from the branching rules of $\sfW_{n}$ in \eqref{eq:CC.C} and \eqref{eq:indSW}, where the factor $\sfH_{t}$ amounts to painting ``$\bullet$'' on $\imath_{\wp}$ and $\jmath_{\wp}$, the factor $\sfS_{a}$ amounts to painting ``$s$'' on $\imath_{\wp}$ and painting ``$r$'' on $\jmath_{\wp}$, each ``permissible" way of painting (see \Cref{def:pbp1})
%   contributing $1$ to the multiplicity $[\tau_{\wp} : \Coh_{\Lambda}(\CK'(G))]_{\sfW'_{n}}$.
% \end{proof}


%  \begin{remark} We will see from \Cref{prop:PBP1} and \Cref{prop:PBP2} that
%   \[
%     \sharp(\tPBP_{G}(\ckcO)) =
%     \left\{
%     \begin{array}{ll}
%        \sharp (\PBP_{G}(\ckcO)),  & \hbox{if $\star\in \{C^*,D^*\}$}; \smallskip\\
%        2^{\sharp(\CPPs(\check \CO))} \cdot \sharp (\PBP_{G}(\ckcO)),  &\hbox{if $\star\in \{B, C,D,\widetilde {C}\}$}.
%     \end{array}
%   \right.
%   \]
%  Proposition \ref{prop:countBCD} will thus imply Theorem \ref{countup} in the introductory section.
%  \end{remark}
%   % Suppose $n_{\mathrm b}=0$ then
%   % \[
%   %   \begin{split}
%   %     [\tau_{\wp} : \cC_{\mathrm g}]_{\sfW'_{n_{\mathrm g}}} = &
%   %     [\wttau_{\wp}|_{W_{\mathrm g}}:\sum_{2t+a=n_{\mathrm g}} \Ind_{\sfH_{t}\times \sfS_{a}}^{\sfW'_{n_{\mathrm g}}}\tsgn\otimes \sgn]_{\sfW'_{n_{\mathrm g}}}\\
%   %     = & [\wttau_{\wp}: \sum_{2t+a=n_{\mathrm g}} \Ind_{\sfH_{t}\times \sfS_{a}}^{\sfW'_{n_{\mathrm g}}}\tsgn\otimes \sgn]_{\sfW_{n_{\mathrm g}}}\\
%   %     = & \PBP_{\star}(\ttau_\wp)
%   %   \end{split}
%   % \]






\section{Special unipotent representations in type BCD :  reduction to good parity}

The goal of this section is to prove \Cref{reduction}.
\Cref{prop:countBCD22} implies that $\Unip_{\check \CO}(G)$ is empty if  $\check \CO$ is not $G$-relevant. Thus we further assume that $\check \CO$ is $G$-relevant. With our earlier assumptions \eqref{nonemp0} and \eqref{nonemp00}, this is equivalent to saying that $\check \CO$ has type I when $\star=D^*$ and $\check \CO$ has bad parity.


If $\star=C^*$, or $\star=D^*$ and $\check \CO $ does not have bad parity, by possibly changing $\omega_{\check V_\mathrm b}$ and $\omega_{\check V_\mathrm g}$ defined in \eqref{omega12} to their negatives, we
assume without loss of generality that  ${\check \CO_\mathrm b}$ has type I.

\subsection{Separating bad parity and good parity for special unipotent representations}

By Proposition \ref{propKL33},  we have an injective linear map
\[
 \varphi_{\lambda_{\check \CO}} : \CK'_{\lambda_{\check \CO_\mathrm b}}(G_{\mathrm b})\otimes  \CK'_{\lambda_{\check \CO_\mathrm g}}(G_{\mathrm g})\rightarrow\CK'_{\lambda_{\check \CO}}(G)
\]
and an injective map
\be\label{injirr}
   \varphi_{\lambda_{\check \CO}} : \Irr'_{\lambda_{\check \CO_\mathrm b}}(G_\mathrm b)\times  \Irr'_{\lambda_{\check \CO_\mathrm g}}(G_\mathrm g)\rightarrow  \Irr'_{\lambda_{\check \CO}}(G).
\ee

\begin{lem}\label{imageu}
The set $\Unip_{\ckcO}(G)$ is contained in the image of the map \eqref{injirr}.
\end{lem}
\begin{proof}
We  assume that $\star\in\{C^*, D^*\}$, and $n_\mathrm b, n_\mathrm g>0$. Otherwise the map \eqref{injirr} is bijective and the lemma is obviously true.
Let $\pi\in \Unip_{\ckcO}(G)$. Pick a basal  element $\Psi\in \Coh_{ \Lambda}(\CK'(G))$ such that $\Psi(\lambda_{\check \CO})=\pi$.
 Denote by $\CC$ the cell  containing $\Psi$ in the basal representation $ \Coh_{\Lambda}(\CK'(G))$ of $W(\Lambda)$. Then Proposition \ref{hcass222} implies that
 \[
  \sigma_\CC\cong  \left( j_{W_{\lambda_\ckcO}}^{W(\Lambda)} \sgn\right)\otimes \sgn\cong \tau_\mathrm b\otimes \tau_{\emptyset}.
  \]
  In particular, $\tau_\mathrm b$ occurs in the cell representation $\la \CC\ra /\la \overline \CC\setminus \CC\ra$ (in the notation of \Cref{seccell}).

  Write $\Coh:= \Coh_{\Lambda}(\CK'(G))$ and write $\Coh_I$ for the image of the map
\[
  \varphi: \Coh_{\Lambda_\mathrm b}(\CK'(G_\mathrm b))\otimes \Coh'_{\Lambda_\mathrm g}(\CK'(G_\mathrm g))\rightarrow \Coh_{\Lambda}(\CK'(G))
\]
which is defined in \eqref{match1}. Assume by contradiction that $\pi$ is not contained in the image of the map \eqref{injirr}. Then $\CC$ has no intersection with $\Coh_I$,  and the natural homormorphism
\[
\la \CC\ra /\la \overline \CC\setminus \CC\ra \rightarrow \Coh/(\Coh_I+\la \overline \CC\setminus \CC\ra )
\]
is injective. Note that $\tau_\mathrm b$ does not occur in  $\Coh/(\Coh_I+\la \overline \CC\setminus \CC\ra )$ since it does not occur in  $\Coh/\Coh_I$. This is a contradiction and the lemma is now proved.
\end{proof}

\begin{prop}\label{propKL333}
The map  \eqref{injirr} restricts to a bijective map
\[
    \varphi_{\lambda_{\check \CO}}: \Unip_{\ckcO_{\mathrm b}}(G_{\mathrm b})\times \Unip_{\ckcO_{\mathrm g}}(G_{\mathrm g})\rightarrow \Unip_{\ckcO}(G).
    \]

\end{prop}
\begin{proof}
Let $\pi_\mathrm b\in  \Irr'_{\lambda_{\check \CO_\mathrm b}}(G_\mathrm b)$ and $\pi_\mathrm g\in  \Irr'_{\lambda_{\check \CO_\mathrm g}}(G_\mathrm g)$.  Pick a basal  element $\Psi_\mathrm b$ of  $\Coh_{ \Lambda_\mathrm b}(\CK'(G_\mathrm b))$ such that $\Psi_\mathrm b(\lambda_{\check \CO_\mathrm b})=\pi_\mathrm b$. Likewise pick a basal  element $\Psi_\mathrm g$ of $ \Coh_{ \Lambda_\mathrm b}(\CK'(G_\mathrm g))$ such that $\Psi_\mathrm g(\lambda_{\check \CO_\mathrm g})=\pi_\mathrm g$.
Write $\Psi:=\varphi(\Psi_\mathrm b\otimes \Psi_\mathrm g)$. Denote by $\CC$ the cell in $ \Coh_{\Lambda}(\CK'(G))$ containing $\Psi$, and similarly define the
 cells $\CC_\mathrm g\ni \Psi_{\mathrm g}$ and $\CC_\mathrm b\ni \Psi_{\mathrm b}$.

Put $\pi:= \varphi_{\lambda_{\check \CO}}(\pi_\mathrm b, \pi_\mathrm g)=\Psi(\lambda_{\check \CO})$. Write
\[
  G_2:=G_\mathrm b \times G_\mathrm g, \qquad \pi_2:=\pi_\mathrm b\widehat \otimes \pi_\mathrm g\in \Irr(G_2),
  \]
  and
  \[
   I_{2, \lambda_{\check \CO}}:=\textrm{the maximal ideal of $\CU(\g_2)$ with infinitesimal character $\lambda_{\check \CO}$.}
  \]
Then
 \begin{eqnarray*}
   && \pi\in  \Unip_{\ckcO}(G)\\
    &\Longleftrightarrow& \sigma_\CC\cong  \left( j_{W_{\lambda_\ckcO}}^{W(\Lambda)} \sgn\right)\otimes \sgn \quad (\textrm{by Proposition \ref{hcass222}})\\
   &\Longleftrightarrow& \sigma_{\CC_\mathrm b}\cong  \left( j_{W_{\mathrm b, \lambda_{\ckcO_\mathrm b}}}^{W_\mathrm b} \sgn\right)\otimes \sgn \ \ \textrm{and}\ \  \sigma_{\CC_\mathrm g}\cong  \left( j_{W'_{\mathrm g, \lambda_{\ckcO_\mathrm g}}}^{W'_\mathrm g} \otimes \sgn\right)\otimes \sgn\\
 &\Longleftrightarrow& \pi_\mathrm b\in  \Unip_{\ckcO_{\mathrm b}}(G_{\mathrm b}) \ \ \textrm{and}\ \  \pi_\mathrm g\in  \Unip_{\ckcO_{\mathrm g}}(G_{\mathrm g}) \quad (\textrm{by Proposition \ref{hcass222}}).
 \end{eqnarray*}
 Here $W_{\mathrm b, \lambda_{\ckcO_\mathrm b}}$ denote the stabilizer of $\lambda_{\ckcO_\mathrm b}$ in $W_{\mathrm b, \lambda_{\ckcO_\mathrm b}}$, and likewise $W'_{\mathrm g, \lambda_{\ckcO_\mathrm g}}$ denote the stabilizer of $\lambda_{\ckcO_\mathrm g}$ in $W_{\mathrm g}$.

 Thus we have proved that the inverse image of $\Unip_{\ckcO}(G)$ under the map \eqref{injirr} equals  $\Unip_{\ckcO_{\mathrm b}}(G_{\mathrm b})\times \Unip_{\ckcO_{\mathrm g}}(G_{\mathrm g})$. Together with \Cref{imageu},  this proves the proposition. \end{proof}




%In  this section, we assume that if $\star=D^*$ and $\check \CO$ has bad parity, then  the number of
  %        positive entries in $\lambda_{\check \CO}$ has the same parity as $ \frac{n}{2}$.


\subsection{The case of bad parity}


% Recall from \Cref{cor:bound}, we have the inequality
% \[
% \begin{split}
% \sharp (\Unip_{\ckcO}(G))&\leq \sum_{\sigma\in \LC_{\ckcO}} [\sigma: \Coh_{\Lambda_{n_{\mathrm b},n_{\mathrm g}}}(\CK(G))]\\
%   & =\sum_{\wp\in \barA(\ckcO)} [\tau_{\mathrm b}\otimes \tau_{\wp}: \Coh_{\Lambda_{n_{\mathrm b},n_{\mathrm g}}}(\CK(G))] \quad (\textrm{by Lemma \ref{lem:Lcell}}).
% \end{split}
% \]


% %\subsection{From coherent continuation representation to counting}

% We have defined in \eqref{defpbp2222} the set $\PBPs(\ckcO)$ when $\ckcO$ has good parity. Similarly, we make the following definition in the bad parity case.
% \begin{defn}
%   Let $\PBPs(\ckcOb)$ be the set of all triples
%   $\uptau = (\imath,\cP)\times(\jmath,\cQ)\times \star $ where $(\imath,\cP)$ and
%   $(\jmath,\cQ)$ are painted partitions such that
%   \begin{itemize}
%     \item $(\imath,\jmath) = (\tau_{L,\mathrm b},\tau_{R,\mathrm b})$ (see \eqref{eq:taub});
%     \item the image of $\cP$ is contained in
%           \[
%           \begin{cases}
%             \set{\bullet, c,d},  & \text{if } \star\in \set{B,\wtC}; \\
%             \set{\bullet, d},  & \text{if } \star\in \set{C,D};\\
%             \set{\bullet},  & \text{if } \star\in \set{C^{*},D^{*}};\\
%           \end{cases}
%           \]
%     \item the image of $\cQ$ is contained in

%           \[
%           \begin{cases}
%             \set{\bullet, c},  & \text{if } \star\in \set{C,D};\\
%             \set{\bullet},  & \text{if } \star\in \set{B,\wtC, C^{*},D^{*}}.\\
%           \end{cases}
%           \]
%   \end{itemize}
% \end{defn}

%We define the partition $\ckcOpb$ by $\ckcOb = 2\ckcOpb$. To be more precise, $\bfrr_{i}(\ckcO'_{\mathrm b}):= \bfrr_{2i}(\ckcO_{\mathrm b})$, for all $i\in \bN^{+}$. Let $\cO'_{\mathrm b}:= (\ckcO'_{\mathrm b})^{t}$.
  %\item $\cO_{\mathrm b}:= \cO'_{\mathrm b}\cupcol \cO'_{\mathrm b}$.
 %$\ckcOpb$ is a partition of $\half\abs{\ckcOb}$.


Recall from \eqref{Gpb} the group
\[
  G'_{\mathrm b} := \begin{cases}
    \GL_{n_{\mathrm b}}(\bR), & \text{if } \star \in \set{B,C,D}; \\
    \widetilde{\GL}_{n_{\mathrm b}}(\bR), & \text{if } \star = \wtC; \\
    \GL_{\frac{n_{\mathrm b}}{2}}(\bH), & \text{if } \star \in \set{C^{*},D^{*}}.\\
  \end{cases}
\]
%When $\star=\wtC$, let $\Unip_{\ckcOpb}(\Gpb)$ be the set of genuine unipotent representations of $\Gpb$ which is naturally identified with $\Unip_{\ckcOpb}(\GL(n_{\mathrm b},\bR))$.

\begin{prop}\label{prop:BP.PP} %In all cases,
  The equalities
\[
    \begin{split}
      \sharp(\PBP_{\star}(\ckcO_{\mathrm b})) = \sharp(\PP_{\star '}(\ckcO'_{\mathrm b})) = \sharp(\Unip_{\ckcO'_{\mathrm b}}(G'_{\mathrm b}))
    \end{split}
  \]
  hold, where
\begin{equation}\label{def:star'}
\star ':= \begin{cases}
    A^{\bR}, & \text{if } \star \in \set{B,C,\wtC,D}; \\
    A^{\bH}, & \text{if } \star \in \set{C^{*},D^{*}}.\\
  \end{cases}
  \end{equation}
%is the set of painted partitions of type $A$ or $A^{\bH}$ attached to $\ckcO'_{\mathrm b}$.
\end{prop}

\begin{proof} Suppose that $\star \in \Set{C^{*},D^{*}}$. Then
\[
    \begin{split}
      \sharp(\PBP_{\star}(\ckcO_{\mathrm b};\tau_{\mathrm b}))= \sharp(\PP_{A^{\bH}}(\ckcO'_{\mathrm b}))= 1.
    \end{split}
\]
  Suppose that $\star \in \Set{B,C,\wtC,D}$.
  %Recall the Young diagrams $\tau_{L,b}$ and $\tau_{R,b})$ from \eqref{eq:taub} and \eqref{eq:taub2}.
   It is easy to see that we have a bijection
  \[
    \begin{array}{ccc}
      \PBP_{\star}(\ckcO_{\mathrm b}) &  \rightarrow & \PP_{A^{\bR}}(\ckcO'_{\mathrm b}),\\
      (\tau_{L,b},\cP)\times (\tau_{R,b},\cQ)\times \star & \mapsto & ((\ckcO'_{\mathrm b})^t,\cP'),
    \end{array}
  \]
  where $\cP'$ is defined by the condition that
  \[
    \cP(\bfcc_{j}(\tau_{L,b}),j)=d \Longleftrightarrow \cP'(\bfrr_{j}(\ckcO'_{\mathrm b}),j)=d, \quad \textrm{ for all } j=1,2,\cdots, \bfcc_{1}(\ckcO'_{\mathrm b}).
  \]
  The last equality is in Theorem \ref{thm:mainR00}.   \end{proof}


  \trivial[h]{
    % Let $\tau' = \ckcO'^{t}_{\mathrm b}$ and $\tau_{\mathrm b}=(\tau_{L,b}, \tau_{R,b})$. Here
    % $\tau_{L,b}, \tau_{R,b}$.
    Now the claim follows for the fact that the bottom rows in $\uptau_{L}$ can
    be filled by $\bullet/c$ or $d$ and
    \[
      \bfcc_{i}(\tau_{L,b}) = \bfcc_{j}(\tau_{L,b}) \Leftrightarrow \bfcc_{i}(\cOpb) = \bfcc_{j}(\cOpb) \quad \textrm{ for all } i,j\in \bN^{+}.
    \]
  }



Let $P_{\mathrm b}$ be a parabolic subgroup of $G_{\mathrm b}$ that is  $\check \CO_\mathrm b$-relevant as defined in Section \ref{secrgp0}. Then
  $G'_{\mathrm b}$ is naturally isomorphic to the Levi quotient of $P_\mathrm b$.


\def\fIb{\fI_{\mathrm b}}
\def\pib{\pi_{\mathrm b}}


\begin{prop}\label{lem:Unip.BP}
%We are in the setting of \Cref{prop:BP.PP}.
%Suppo For every we have a bijection
For every $\pi'\in \Unip_{\Gpb}(\ckcOpb)$, the normalized induced representation
$\Ind_{\Pb}^{\Gb}\pi'$ is irreducible and belongs to $\Unip_{\Gpb}(\ckcOpb)$.
Moreover, the map
  \be\label{badind}
    \begin{array}{rcl}
      \Unip_{\ckcO'_{\mathrm b}}(G'_{\mathrm b}) & \rightarrow & \Unip_{\ckcOb}(\Gb), \\
      \pi' & \mapsto & % \pib:=
                        \Ind_{\Pb}^{\Gb}\pi'
    \end{array}
  \ee
  is bijective.
\end{prop}

% \begin{prop}\label{lem:Unip.BP}
% %We are in the setting of \Cref{prop:BP.PP}.
% Suppose that $n_{\mathrm g}=0$.
% \begin{enumerate}
% \item Then we have a bijection
%   \[
%     \begin{array}{rccc}
%       \fIb: &\Unip_{\ckcO'_{\mathrm b}}(G'_{\mathrm b}) & \longrightarrow & \Unip_{\ckcO}(\Gb) \\
%       &\pi' & \mapsto & \pib:=\Ind_{\Pb}^{\Gb}\pi'.
%     \end{array}
%   \]
%   % where $P$ is a parabolic subgroup of $G$ whose Levi component is
%   % isomorphic to $G'_{\mathrm b}$.
%  \item Suppose by $\sO'$ the real nilpotent orbit in $G'_{\mathrm b}$ such that $\WF(\sO')=\overline{\sO'}$.  % \item Denote by $\sO'$ the real nilpotent orbit in $G'_{\mathrm b}$ such that $\sO'_{\bC}=\cO$, and let $\sO$ be the real induction of $\sO'$ to $\Gb$.
%   Then the wavefront cycle of $\pi $ is
%   \[
%     \WF(\pib) = \overline{\sO}
%   \]
%   where $\sO$ is the real induction of $\sO'$ to $\Gb$.
%  \end{enumerate}
% \end{prop}
\begin{proof}
  % Let $\sO_{\pi'}$ be the open $\Gpb$-orbit in the wavefront set of $\pi'$.
  % It occurs with multiplicity one in the wavefront cycle of $\pi'$ by .
%   where $\sO'$ is a $\Gpb$
% Suppose by $\sO'$ the real nilpotent orbit in $G'_{\mathrm b}$ such that the
% wave front set of $\pi'$ is $\sO'$ $\WF(\sO')
% =\overline{\sO'}$.  % \item Denote by $\sO'$ the real nilpotent orbit in %
% $G'_{\mathrm b}$ such that $\sO'_{\bC}=\cO$, and let $\sO$ be the real induction of $\sO'$ to $\Gb$.
% %   Then the wavefront cycle of $\pi $ is
% %   \[
% %     \WF(\pib) = \overline{\sO}
% %   \]

  By the construction of $\pi'$  (see \Cref{sec:GLRH}) and
  % first note that $\WF(\pi')=[\sO']$.
  % It follows from
  a result of Barbasch \cite[Corollary 5.0.10]{B.Orbit}, the
  wavefront cycle of $\Ind_{\Pb}^{\Gb}\pi'$ is a single orbit $\sO$ with
  multiplicity one and its complexification is $\dBV(\ckcOb)$.
  Note that every irreducible summand of  $\Ind_{\Pb}^{\Gb}\pi'$ belongs
  to $\Unip_{\ckcOb}(\Gb)$. Hence $\Ind_{\Pb}^{\Gb}\pi'$ has to be irreducible.

  % We claim that $\Ind_{\Pb}^{\Gb}\pi'$ is irreducible. If not, $\Ind_{\Pb}^{\Gb}\pi'$  must contain an irreducible subquotient with infinitesimal
  % character $\lambda_{\ckcOb}$ and GK-dimension $<\half\dim_{\bC}\dBV(\ckcOb)$, by \cite[Korollar 3.6]{BK}. This will
  % contradict to the assertion of \Cref{lem:Lcell} on the associated variety of the maximal primitive ideal $\cI_{\ckcOb}$.
  % \trivial[h]{
  %   First, it is not obvious to me that the wavefront of $\Ind_{P}^{G}\pi$
  %   must be contained in $\Ind\WF(\pi)$. But it is clear that the leading term
  %   must be $\sum_{\sO\text{ open in } \WF(\pi)}\Ind\sO$. So the boundaries has
  %   less GK-dimension.

  %   Suppose $\pi_{0}$ is the sub-quotient with less GK-dimension.
  %   On the other hand, the maximal primitive ideal $\cI_{\ckcO}$  with infinitesimal
  %   character must contain $\Ann\pi_{0}$. In other words,
  %   $\AV_{\bC}(\pi_{0})\supseteq \bcO$ which implies GK-dimension of
  %   $\pi_{0}\geq \half\dim_{\bC}\cO$, a contradiction.
  %   % Let $\lambda = \lamck$.
  %   % Note that every $W_{[\lambda]}$-module in $\Grt_{\lambda}(G)$ is in $\Ind_{W_{\lambda}}^{W_{[\lambda]}}$
  % }
  %


  We now suppose that $\star=D$ so that $\Gb=\SO(\nnb, \nnb)$. We will prove the injectivity of the map \eqref{badind}.
    Fix a split Cartan subgroup
    $$H=(\bR^\times)^{\nnb}\subseteq \Gpb\subseteq \Gb $$ and write $H = MA$ where
    $M = \set{\pm 1}^{\nnb}$ is the compact part of $H$ and
    $A = (\bR^\times_+)^{\nnb}$ is the split part of $H$. We identify a
    character of $A$ with a vector in $\bC^{\nnb}$ as usual. Let
    \[
    K:=\{(g_1, g_2)\in \rO(\nnb)\times \rO(\nnb)\mid \det(g_1)\cdot \det(g_2)=1\}
    \]
    be a maximal compact subgroup of $\Gb$, and let
    $B$ be a Borel subgroup of $\Gb$ containing $H$ and the unipotent radical of $\Pb$.

 For each integer $l$ such that $0\leq l \leq \nnb$, put
    \[
      \delta_{l} = \underbrace{1_{1}\otimes \cdots \otimes 1_{1}}_{l} \otimes
      \underbrace{\sgn_{1} \otimes \cdots \otimes \sgn_{1}}_{\nnb-l}\in \Irr(M),
    \]
    where $1_{1}$ denotes the trivial character of $\{\pm 1\}$ and $\sgn_{1}$ denotes the non-trivial character of $\{\pm 1\}$.
It is a fine $M$-type (\cite{Vg}*{Definition~4.3.8}). Let $\lambda_{l}$ be
    the restriction to $K$ of the irreducible
    $\rO(\nnb)\times \rO(\nnb)$-representation
    $\wedge^{l}\,\bC^{\nnb}\otimes \wedge^{0}\,\bC^{\nnb}$. Then $\lambda_{l}$ is a
    fine $K$-type, see \cite{BGG.M}*{\S 6}.

    \def\cusp{\fC}

    Vogan proves that there is a well-defined injective map
\[
      \begin{array}{lccc}
       \fX\colon &   W_H\backslash \Irr(H)&\rightarrow & \Irr(G_\mathrm b),\\
       & \textrm{the $W_H$-orbit of $\chi\in \Irr(H)$} & \mapsto & (\Ind_{B}^{\Gb}\chi)(\lambda_{l_\chi}),
      \end{array}
    \]
    where $W_H$ denotes the real Weyl group of $G_\mathrm b$ with respect to $H$,  $l_\chi\in\{0,1, \dots, n_\mathrm b\}$ is the integer such that $\chi|_M$ is conjugate to $\delta_{l_\chi}$ by $W_H$, and $(\Ind_{B}^{\Gb}\chi)(\lambda_{l_\chi})$ denotes the unique
    irreducible subquotient in the normalized induced representation
    $\Ind_{B}^{\Gb}\chi$ containing the $K$-type $\lambda_{l}$. See \cite{Vg}*{Theorem 4.4.8}.

        Now take
         \[
         \pi' = 1_{n_{1}}\times \cdots \times 1_{n_{r}}\times \sgn_{n_{r+1}}\times \cdots \times \sgn_{n_{k}}\in \Unip_{\Gpb}(\ckcOpb),
         \]
       where   $k\geq r\geq 0$,  and $n_1, n_2, \dots, n_k$ are positive integers with  $n_1+n_2+ \dots+ n_k=\nnb$.



    Write
    \begin{eqnarray*}
    \delta_{\pi'}&:=&\delta_{n_{1}+n_{2}+\dots + n_{r}}\in \Irr(M),\\
        \nu_{\pi'} &:=& \big( {\tfrac{n_{1}-1}{2},\tfrac{n_{1}-3}{2},\dots ,\tfrac{1-n_{1}}{2}, \dots , \tfrac{n_{k}-1}{2},\tfrac{n_{k}-3}{2},\dots ,\tfrac{1-n_{k}}{2}} \big)\in \Irr(A).
   \end{eqnarray*}
   Clearly the map
   \[
   \begin{array}{rcl}
     \fP:   \Unip_{\Gpb}(\ckcOpb)&\rightarrow &  W_H\backslash \Irr(H),\\
      \pi' &\mapsto & \textrm{the $W_H$-orbit of $\delta_{\pi'}\otimes \nu_{\pi'}$}
      \end{array}
      \]
       is well-defined and
    injective.

   Note that the map \eqref{badind} equals the composition $\fX\circ \fP$, and hence it is also injective. This proves the injectivity of \eqref{badind} in the case when $\star=D$. The same proof works in the case when $\star\in \{B, C, \wtC\}$ and we omit the details. When $\star\in \{C^*, D^*\}$, the map \eqref{badind} has to be injective since its domain is a singleton. Thus \eqref{badind} is injective in all cases.




  The bijection of \eqref{badind}  follows from the injectivity  and the counting
  inequalities below:
  \[
    \abs{\PP_{\star'}(\ckcO'_{\mathrm b})}=\abs{\Unip_{\ckcO'_{\mathrm b}}(G'_{\mathrm b})}\leq \abs{\Unip_{\ckcO_\mathrm b}(G_\mathrm b)}
    \leq \abs{\PBP_{\star}(\ckcO_{\mathrm b})} = \abs{\PP_{\star'}(\ckcO'_{\mathrm b})}.
  \]
  Here $\star' \in \set{A^\R,A^{\bH}}$ is as in \eqref{def:star'}, the first inequality follows from the injectivity of \eqref{badind}, the second inequality follows from  \Cref{prop:countBCD22}, and the last equality is in \Cref{prop:BP.PP}.

\end{proof}




\trivial[h]{
We introduce some additional notation. For each bipartition $\tau$, let
\[
  \PBP_{\star}(\tau) := \Set{ \uptau|\uptau \text{ is a painted bipartition and
    } \star_{\uptau}=\star, (\imath_{\uptau},\jmath_{\uptau}) = \tau}
  % \uptau=(\imath, \cP)\times (\jmath,\cP)\times \alpha|}
\]
and
\[
  \tPBP_{\star}(\ckcO_{\mathrm g}) := %\bigsqcup_{\tau\in\LC(\ckcOg)}\PBP_{\star}(\tau).
  \bigsqcup_{\wp \subseteq \CPP(\ckcO_{\mathrm g})}\PBP_{\star}(\wttau_{\wp}),
\]
where $\wttau_{\wp} := (\imath_{\wp},\jmath_{\wp})$ (see \eqref{eq:ttauwp}). Similarly, define
\[
  \tPBP_{G_{\mathrm g}}(\ast):= \Set{\uptau\in \tPBP_{\star}(\ast )|\Sign(\uptau)= \Sign(G_{\mathrm g})}, \quad  \quad
  \ast  = \ckcO_{\mathrm g} \text{ or } \tau.
\]

Recall from \Cref{cor:bound}, we have the inequality
\[
\begin{split}
\sharp (\Unip_{\ckcO}(G))&\leq \sum_{\sigma\in \LC_{\ckcO}} [\sigma: \Coh_{\Lambda_{n_{\mathrm b},n_{\mathrm g}}}(\CK(G))]\\
  & =\sum_{\wp\in \barA(\ckcO)} [\tau_{\mathrm b}\otimes \tau_{\wp}: \Coh_{\Lambda_{n_{\mathrm b},n_{\mathrm g}}}(\CK(G))] \quad (\textrm{by Lemma \ref{lem:Lcell}}).
\end{split}
\]


\begin{prop}\label{prop:countBCD}
  In all cases, we have
  % \[
  %   [\tau_{\mathrm b}: \cC_{\mathrm b}] = \PBP_{\star,b}(\ckcO_{\mathrm b}) = \PBP_{G'}(\ckcO'_{n_{\mathrm b}}) = \Unip_{\ckcO'_{\mathrm b}}(G'_{n_{\mathrm b}})
  % \]
  % and
  % \[
  %   \sum_{\tau\in \LC_{\ckcO_{\mathrm g}}} [\tau:\cC_{\mathrm g}] = \PBP_{\mathrm g}(\ckcO_{\mathrm g}).
  % \]
  \[
  \sum_{\wp\in \barA(\ckcO)} [\tau_{\mathrm b}\otimes \tau_{\wp}: \Coh_{\Lambda_{n_{\mathrm b},n_{\mathrm g}}}(\CK(G))]
      = \sharp (\PBP_{\star}(\ckcO_{\mathrm b};\tau _{\mathrm b}))\cdot \sharp(\tPBP_{G_{\mathrm g}}(\ckcO_{\mathrm g})).
  \]
 In particular, when $\ckcO = \ckcO_{\mathrm g}$, namely $\ckcO $ has good parity, we have
  \[
  \sharp (\Unip_{\ckcO}(G))\leq
     \sharp ({\tPBP_{G}(\ckcO)}).
  \]
 \end{prop}

 \begin{remark} When $\ckcO $ has good parity, we will see from \Cref{prop:PBP1} and \Cref{prop:PBP2} that
  \[
    \sharp(\tPBP_{G}(\ckcO)) =
    \left\{
    \begin{array}{ll}
       \sharp (\PBP_{G}(\ckcO)),  & \hbox{if $\star\in \{C^*,D^*\}$}; \smallskip\\
       2^{\sharp(\CPPs(\check \CO))} \cdot \sharp (\PBP_{G}(\ckcO)),  &\hbox{if $\star\in \{B, C,D,\widetilde {C}\}$}.
    \end{array}
  \right.
  \]
 Proposition \ref{prop:countBCD} will thus imply Theorem \ref{countup} in the introductory section.
 \end{remark}

\begin{proof} We use the descriptions of $\tau_{\mathrm b}$ and $\tau_{\wp}$ (in \eqref{eq:taub} and \eqref{eq:tauwp}) to compute the multiplicities, using branching rules of Weyl groups in \eqref{eq:CC.C} and \eqref{eq:indSW} as well as \Cref{lem:WLcell}. We skip the details when $\star \in \set{B,\wtC, C,D,C^{*}}$, and present the computation for $\star = D^{*}$, which is the most complicated case.

  Recall that $(W_{\mathrm b},W_{\mathrm g}) = (\sfW'_{\nbb},\sfW'_{\ngg})$, where $n_{\mathrm b}=\half \abs{\ckcO_{\mathrm b}}$ and $n_{\mathrm g}=\half\abs{\ckcO_{\mathrm g}}$.

  First suppose $\ngg = 0$. Then $\tau_{\mathrm b} = (\cOpb,\cOpb)_{I}$ and
  \[
    \begin{split}
      [\tau_{\mathrm b}:\cC_{\mathrm b}^{\nbb}] = &
      [\tau_{\mathrm b}: \Ind_{\sfH_{\frac{\nbb}{2}}}^{\sfW'_{\nbb}}\tsgn]\\
      = & [(\cOpb,\cOpb)_{I}: \bigoplus_{\sigma}(\sigma,\sigma)_{I}]\\
      = & 1.
    \end{split}
  \]

  Now suppose $\ngg>0$. Let
  \[
    \wttau_{\mathrm b} := \Ind_{\sfW'_{\nbb}}^{\sfW_{\nbb}} \tau_{\mathrm b} = (\cOpb,\cOpb) \AND \wttau_{\wp}: = (\imath_{\wp},\jmath_{\wp}) \quad \textrm{ for all } \wp \subseteq \CPP(\ckcOg).
  \]
  Note that $\imath_{\wp}\neq \jmath_{\wp}$ since
  $\bfcc_{1}(\imath_{\wp})> \bfcc_{1}(\jmath_{\wp})$. It is then easy to see that
  \begin{equation*}%\label{eq:W''}
    \Ind_{\sfW'_{\nbb}\times \sfW'_{\ngg}}^{\sfW_{n_{\mathrm b},n_{\mathrm g}}'} \tau_{\mathrm b}\otimes \tau_{\wp}
    = (\wttau_{\mathrm b}\otimes \wttau_{\wp})|_{\sfW_{n_{\mathrm b},n_{\mathrm g}}'}.
  \end{equation*}

  \trivial[h]{ When $\nbb=0$, $W'' = W_{\mathrm g} = \sfW'_{\ngg}$ and so
    $\wttau_{\wp}|_{\sfW'_{\ngg}} = \tau_{\wp}$.

    Now we assume $\nbb\neq 0$ and $\ngg\neq 0$ (the general case). This
    follows from the following points
    \begin{itemize}
      \item the dimension of the two sides are equal ($W_{\mathrm b}\times W_{\mathrm g}$ has
            index $2$ in $W''$).
      \item
            \[
            \begin{split}
              &[\Ind_{W_{\mathrm b}\times W_{\mathrm g}}^{W''}\tau_{\mathrm b}\otimes \tau_{\wp}:(\wttau_{\mathrm b}\otimes \wttau_{\wp})|_{W''}] \\
              =& [\Ind_{\sfW'_{n_{\mathrm b}}\times \sfW'_{n_{\mathrm g}}}^{\sfW_{n_{\mathrm b}}\times \sfW_{n_{\mathrm g}}}\tau_{\mathrm b}\otimes \tau_{\wp}:\wttau_{\mathrm b}\otimes \wttau_{\wp}]\\
              =& [\wttau_{\mathrm b}\otimes \wttau_{\wp} \oplus \wttau_{\mathrm b}\otimes (\wttau_{\wp}\otimes \brsgn):\wttau_{\mathrm b}\otimes \wttau_{\wp}] =1\\
            \end{split}
            \]
            where $\wttau_{\wp}\otimes \brsgn$ has the bipartition obtained by
            switching the left and right side of $\wttau_{\wp}$.
      \item the LHS is irreducible, by
            \[
            \begin{split}
              & [\Ind_{W_{\mathrm b}\times W_{\mathrm g}}^{W''}\tau_{\mathrm b}\otimes \tau_{\wp}:
              \Ind_{W_{\mathrm b}\times W_{\mathrm g}}^{W''}\tau_{\mathrm b}\otimes \tau_{\wp}]_{W''}\\
              =&  [\tau_{\mathrm b}\otimes \tau_{\wp} : (\Ind_{W_{\mathrm b}\times W_{\mathrm g}}^{W''}\tau_{\mathrm b}\otimes \tau_{\wp})|_{W_{\mathrm b}\times W_{\mathrm g}}]\\
              =& [\tau_{\mathrm b}\otimes \tau_{\wp} : \tau_{\mathrm b}\otimes \tau_{\wp} + (\cOpb,\cOpb)_{II} \otimes \tau_{\wp} ] = 1
            \end{split}
            \]
    \end{itemize}
  }

  For ease of notation, write $\sfW'':=\sfW_{n_{\mathrm b},n_{\mathrm g}}'$. We then have
  \[
    \begin{split}
      & [\tau_{\mathrm b}\otimes \tau_{\wp} :
      \Ind_{\sfW'_{n_{\mathrm b}}\times \sfW'_{n_{\mathrm g}}}^{\sfW''} \cC_{\mathrm b}^{\nbb} \otimes \cC_{\mathrm g}^{p_{\mathrm g},q_{\mathrm g}}]_{\sfW'_{n_{\mathrm b}}\times \sfW'_{n_{\mathrm g}}}\\
      = & [\Ind_{\sfW'_{n_{\mathrm b}}\times \sfW'_{n_{\mathrm g}}}^{\sfW''} \tau_{\mathrm b}\otimes \tau_{\wp} :
      \Ind_{\sfW'_{n_{\mathrm b}}\times \sfW'_{n_{\mathrm g}}}^{\sfW''} \cC_{\mathrm b}^{\nbb} \otimes \cC_{\mathrm g}^{p_{\mathrm g},q_{\mathrm g}}]_{\sfW''}\\
      = & [(\wttau_{\mathrm b}\otimes \wttau_{\wp})|_{\sfW''}:
      \Ind_{\sfW'_{n_{\mathrm b}}\times \sfW'_{n_{\mathrm g}}}^{\sfW''} \cC_{\mathrm b}^{\nbb} \otimes \cC_{\mathrm g}^{p_{\mathrm g},q_{\mathrm g}}]_{\sfW''}\\
      = & [\wttau_{\mathrm b}\otimes \wttau_{\wp}:
      \Ind_{\sfW'_{n_{\mathrm b}}\times \sfW'_{n_{\mathrm g}}}^{\sfW_{n_{\mathrm b}}\times \sfW_{n_{\mathrm g}}} \cC_{\mathrm b}^{\nbb} \otimes \cC_{\mathrm g}^{p_{\mathrm g},q_{\mathrm g}}]_{\sfW_{n_{\mathrm b}}\times \sfW_{n_{\mathrm g}}}\\
       = & [\wttau_{\mathrm b}:\Ind_{\sfW'_{n_{\mathrm b}}}^{\sfW_{n_{\mathrm b}}} \cC_{\mathrm b}^{\nbb}]_{\sfW_{n_{\mathrm b}}}\cdot
           [\wttau_{\wp}:\Ind_{\sfW'_{n_{\mathrm g}}}^{\sfW_{n_{\mathrm g}}} \cC_{\mathrm g}^{p_{\mathrm g},q_{\mathrm g}}]_{\sfW_{n_{\mathrm g}}}\\
      =& \sharp(\PBP_{\star}(\ckcO_{\mathrm b}; \tau _{\mathrm b})))\cdot \sharp(\PBP_{\star}(\wttau_{\wp})).
    \end{split}
  \]
  The last equality follows from the explicit description of $\cC_{\mathrm b}^{\nbb}$ and $\cC_{\mathrm g}^{p_{\mathrm g},q_{\mathrm g}}$ in Section \ref{sec:explicitCoh}, and
  the branching rules of $\sfW_{n}$ in \eqref{eq:CC.C} and \eqref{eq:indSW}, where the factor $\sfH_{t}$ amounts to painting ``$\bullet$'' on $\imath_{\wp}$ and $\jmath_{\wp}$, the factor $\sfS_{a}$ amounts to painting ``$s$'' on $\imath_{\wp}$ and painting ``$r$'' on $\jmath_{\wp}$, each ``permissible" way of painting (see \Cref{def:pbp1})
  contributing $1$ to the multiplicity $[\wttau_{\wp}:\Ind_{\sfW'_{n_{\mathrm g}}}^{\sfW_{n_{\mathrm g}}} \cC_{\mathrm g}^{p_{\mathrm g},q_{\mathrm g}}]_{\sfW_{n_{\mathrm g}}}$.
\end{proof}

}



  \trivial[h]{

    Suppose $\star =C^{*}$.

    For the bad parity, $n_{\mathrm b}=2n'_{\mathrm b}$ must be even.
    \[
      \begin{split}
        & [\tau_{\mathrm b}\otimes \tau_{\wp} :
        \cC_{\mathrm b}^{\nbb} \otimes \cC_{\mathrm g}^{\ngg,\ngg}]_{\sfW'_{n_{\mathrm b}}\times \sfW_{n_{\mathrm g}}}\\
        = & [\Ind_{\sfW'_{n_{\mathrm b}}\times \sfW_{n_{\mathrm g}}}^{\sfW_{n_{\mathrm b}}\times \sfW_{n_{\mathrm g}}} \tau_{\mathrm b}\otimes \tau_{\wp} :
        \cC_{\mathrm b}^{\nbb} \otimes \cC_{\mathrm g}^{\ngg,\ngg}]_{\sfW_{n_{\mathrm b}}\times \sfW_{n_{\mathrm g}}}\\
        = & [\wttau_{\mathrm b}\otimes \tau_{\wp}:
        \cC_{\mathrm b}^{\nbb} \otimes \cC_{\mathrm g}^{\ngg,\ngg}]_{\sfW_{n_{\mathrm b}}\times \sfW_{n_{\mathrm g}}}\\
        =& \# \PBP_{\star}(\ckcO_{\mathrm b})\cdot \# \PBP_{\star}(\tau_{\wp})
      \end{split}
    \]

    The alternative approach using restriction. For the bad parity,
    $n_{\mathrm b}=2n'_{\mathrm b}$ must be even.
    \[
      \begin{split}
        \cC_{\mathrm b}^{n_{\mathrm b}} &=
        \Res_{\sfW_{n_{\mathrm b}}}^{\sfW'_{n_{\mathrm b}}} \Ind_{\sfH_{n'_{\mathrm b}}}^{\sfW_{n_{\mathrm b}}}\tilde{\varepsilon} \\
        &= \bigoplus_{\sigma\in \Irr(\sfS_{n'_{\mathrm b}})} \left((\sigma,\sigma)_{I} \oplus (\sigma,\sigma)_{II}\right).
      \end{split}
    \]
    For the good parity,
    \[
      \begin{split}
        \cC_{\mathrm g}^{2p,2q} %& = \bigoplus_{p+q=m} \Cint{\rho}(\Sp(p,q)) \\
        & =\bigoplus_{\substack{(t+s,t+r)=(p,q)}} \bigoplus_{\sigma} \Ind_{\sfW_{2t}\times \sfW_s\times \sfW_r}^{\sfW_{p+q}}
        (\sigma,\sigma)\otimes \sgn \otimes \sgn \\
      \end{split}
    \]
    Now the branching rule implies the irreducible components of
    $\cC_{\mathrm g}^{2p,2q}$ are given by the dot-diagram attaching two columns on the
    right, which we mark them by $s$ and $r$ respectively. }


  % Suppose $\star=C$
  % \[
  %   \begin{split}
  %     & [\tau_{\mathrm b}\otimes \tau_{\wp} :
  %     \cC_{\mathrm b}^{\nbb} \otimes \cC_{\mathrm g}^{\ngg,\ngg}]_{\sfW'_{n_{\mathrm b}}\times \sfW_{n_{\mathrm g}}}\\
  %     = & [\Ind_{\sfW'_{n_{\mathrm b}}\times \sfW_{n_{\mathrm g}}}^{\sfW_{n_{\mathrm b}}\times \sfW_{n_{\mathrm g}}} \tau_{\mathrm b}\otimes \tau_{\wp} :
  %     \cC_{\mathrm b} \otimes \cC_{\mathrm g}]_{\sfW_{n_{\mathrm b}}\times \sfW_{n_{\mathrm g}}}\\
  %     = & [\wttau_{\mathrm b}\otimes \tau_{\wp}:
  %     \cC_{\mathrm b} \otimes \cC_{\mathrm g}]_{\sfW_{n_{\mathrm b}}\times \sfW_{n_{\mathrm g}}}\\
  %     =& \# \PBP_{\star}(\ckcO_{\mathrm b})\cdot \# \PBP_{\star}(\ckcO_{\mathrm g};\wp)
  %   \end{split}
  % \]
%
  % For the bad parity, $n_{\mathrm b}=2n'_{\mathrm b}$ must be even.
  % \[
  %   \begin{split}
  %     \cC_b^{n_{\mathrm b}} &= \Res_{\sfW_{n_{\mathrm b}}}^{\sfW'_{n_{\mathrm b}}} \left( \bigoplus_{2t+a=n_{\mathrm b}}\Ind_{\sfH_{t}\times \sfS_{a}}^{\sfW_{n_{\mathrm b}}}\tilde{\varepsilon} \otimes 1
  %     \right)\\
  %     &= \bigoplus_{2t+c+d=n_{\mathrm b}}\bigoplus_{\sigma\in \Irr(\sfS_{t})} \left((\sigma,\sigma)_{I} \oplus (\sigma,\sigma)_{II}\right) \times ([d,],[c,]).
  %   \end{split}
  % \]
  % Note that $\tau_{\mathrm b}=(\cO'_{\mathrm b},\cO'_{\mathrm b})_{I}$. Therefore we only need to
  % consider the case when $c=d$ in the above formula.
  % \[
  %   \begin{split}
  %     [\tau_{\mathrm b}: \cC_{\mathrm b}^{n_{\mathrm b}}] & = [\cO'_{\mathrm b}:\bigoplus_{\substack{t+d = n'_{\mathrm b}\\ \sigma}} \sigma \times 1].
  %   \end{split}
  % \]
  % By the counting of unipotent representation of $\GL(n'_{\mathrm b},\bR)$. We see
  % that $\PBP_{\star,b}(\ckcO_{\mathrm b})$ is identified with
  % $\PBP_{A^{\bR}}(\ckcO'_{\mathrm b})$ by send $(\uptau_{L},\uptau_{R})$ to the
  % painted partition $\uptau'$ such that
  % \[
  %   \uptau_{L}(i,j)=d \Leftrightarrow \uptau'(i,j)=d.
  % \]

  % For the good parity, this is clear by the branching rules.


  \trivial[h]{


    Suppose $\star = D$. Suppose that $n_{\mathrm g}\neq 0$. Since
    $\bfcc_{1}(\imath_{\wp})>\bfcc_{1}(\jmath_{\wp})$, we have
    $\wttau^{s}_{\wp}:=\wttau_{\wp}\otimes \brsgn\ncong\wttau_{\wp}$.
    \[
      \begin{split}
        & [\tau_{\mathrm b}\otimes \tau_{\wp} :
        \cC_{\mathrm b} \otimes \cC_{\mathrm g}]_{\sfW'_{n_{\mathrm b}}\times \sfW'_{n_{\mathrm g}}}\\
        = & [\Ind_{\sfW'_{n_{\mathrm b}}\times \sfW'_{n_{\mathrm g}}}^{\sfW_{n_{\mathrm b}}\times \sfW_{n_{\mathrm g}}} \tau_{\mathrm b}\otimes \tau_{\wp} :
        \cC_{\mathrm b} \otimes \cC_{\mathrm g}]_{\sfW_{n_{\mathrm b}}\times \sfW_{n_{\mathrm g}}}\\
        = & [\wttau_{\mathrm b}\otimes \wttau_{\wp}\oplus \wttau_{\mathrm b}\otimes \wttau_{\wp}^{s}:
        \cC_{\mathrm b} \otimes \cC_{\mathrm g}]_{W''}\\
        = & [\wttau_{\mathrm b}\otimes \wttau_{\wp}:
        \cC_{\mathrm b} \otimes \cC_{\mathrm g}]_{\sfW_{n_{\mathrm g}}\times \sfW_{n_{\mathrm b}}}\\
        =& \# \PBP_{\star}(\ckcO_{\mathrm b})\cdot \# \PBP_{\star}(\ckcO_{\mathrm g};\wp)
      \end{split}
    \]
    The terms involving $\wttau_{\sP}^{s}$ vanish since every irreducible
    component $(\sigma_{L},\sigma_{R})$ in $\cC_{\mathrm g}$ satisfies
    $\sigma_{L}\supseteq \sigma_{R}$ but
    $\bfcc_{1}(\imath_{\wp})>\bfcc_{1}(\jmath_{\wp})$.

    Suppose that $n_{\mathrm g} = 0$.
    \[
      \begin{split}
        & [\tau_{\mathrm b} : \cC_{\mathrm b}^{n_{\mathrm b}}]_{\sfW'_{n_{\mathrm b}}}\\
        =& [\Ind_{\sfW'_{n_{\mathrm b}}}^{\sfW_{n_{\mathrm b}}} \tau_{\mathrm b} : \bigoplus_{\substack{2t+a=n}}
        \Ind_{\sfH_{t}\times \sfS_{a}}^{\sfW_{n}}\tilde{\varepsilon}\otimes 1] \\
        =& [ \wttau_{\mathrm b} : \bigoplus_{\substack{2t+a=n}}
        \Ind_{\sfH_{t}\times \sfS_{a}}^{\sfW_{n}}\tilde{\varepsilon}\otimes 1] \\
      \end{split}
    \]
    In any cases, the counting formula holds. There is place to confuse: Why
    there shouldn't be double the size of special unipotent representations?

    In fact, $\AC_{\bC}(\pi)$ can only be the fixed type, say $\cO_{I}$! Note
    that we fixed an infinitesimal character which has half-integral values.
    This choice implicitly force us to fix real Siegel parabolic when we do
    induction from $\GL$! The non-trivial outer automorphism, say $c$, will
    permute the infinitesimal character to the another one and we then will have
    $\AC_{\bC}({}^{c}\pi)i = \cO_{II}$.


    Using Barbasch's formula of wavefront, we see that the induction
    $\pi_{I}:=\Ind_{\GL}^{\SO}\pi'$ must be irreducible, where $\pi'$ is a
    unipotent representation of $\GL$. This will also implies
    $\Ind_{\GL}^{\rO}\pi'$ is irreducible and restricted to two $\SO$-modules,
    $\pi_{I}$ and $\pi_{II}$.

  }


  \trivial[h]{ Suppose $\star = \wtC$. Then
    \[
      \begin{split}
        & [\tau_{\mathrm b}\otimes \tau_{\wp} :
        \cC_{\mathrm b} \otimes \cC_{\mathrm g}]_{\sfW_{n_{\mathrm b}}\times \sfW'_{n_{\mathrm g}}}\\
        = & [ \tau_{\mathrm b}\otimes \Ind_{\sfW'_{n_{\mathrm g}}}^{\sfW_{n_{\mathrm g}}}\tau_{\wp} : \cC_{\mathrm b} \otimes \cC_{\mathrm g}]_{\sfW_{n_{\mathrm b}}\times \sfW_{n_{\mathrm g}}}\\
        = &\sum_{\sP\in \tA(\ckcO)} [\tau_{\mathrm b}\otimes \wttau_{\wp}:
        \cC_{\mathrm b} \otimes \cC_{\mathrm g}]_{\sfW_{n_{\mathrm b}}\times \sfW_{n_{\mathrm g}}}\\
        =& \# \PBP_{\star}(\ckcO_{\mathrm b})\cdot \# \PBP_{\star}(\ckcO_{\mathrm g})
      \end{split}
    \]

    Suppose $\star = B$. Then
    \[
      \begin{split}
        & \sum_{\sP\in \tA'(\ckcO)}[\tau_{\mathrm b}\otimes \tau_{\wp} :
        \cC_{\mathrm b} \otimes \cC_{\mathrm g}]_{\sfW_{n_{\mathrm b}}\times \sfW'_{n_{\mathrm g}}}\\
        =& \# \PBP_{\star}(\ckcO_{\mathrm b})\cdot \# \PBP_{\star}(\ckcO_{\mathrm g};\wp)
      \end{split}
    \]
  }



\subsection{Coherent continuation representations and parabolic induction}





 The normalized smooth parabolic induction from $P_{\mathrm b}$ to $G_\mathrm b$ yields a linear map
\[
 \Ind:  \CK'(G'_{\mathrm b})\rightarrow \CK'(G_\mathrm b).
\]
This induces a linear map
\[
\begin{array}{rcl}
 \Ind:  \Coh_{\Lambda_\mathrm b}(\CK'(G'_{\mathrm b}))&\rightarrow &  \Coh_{\Lambda_\mathrm b}( \CK'(G_\mathrm b)),\\
   \Psi'_\mathrm b&\mapsto & \left(\nu\mapsto \Ind(\Psi'_\mathrm b(\nu))\right).
   \end{array}
\]

Let $P$ be a $\check \CO$-relevant parabolic subgroup of $G$ as in \Cref{reduction}.
Then  $G'_\mathrm b\times G_\mathrm g$ (or its quotient by $\{\pm 1\}$ when $\star=\wtC$) is naturally isomorphic to the  Levi quotient  of $P$. The
 normalized smooth parabolic induction using $P$ yields a linear map
\[
 \Ind:  \CK'(G'_{\mathrm b})\otimes \CK'(G_{\mathrm g})\rightarrow \CK'(G),
\]
which further induces a linear map
\[
\begin{array}{rcl}
 \Ind:  \Coh_{\Lambda_{\mathrm b}}(\CK'(G'_{\mathrm b}))\otimes  \Coh_{\Lambda_\mathrm g}(\CK'(G_{\mathrm g})) &\rightarrow &  \Coh_{\Lambda}( \CK'(G)),\\
  \Psi'_\mathrm b\otimes \Psi_\mathrm g&\mapsto & \left((\nu_\mathrm b,\nu_\mathrm g)\mapsto \Ind(\Psi_\mathrm b(\nu_\mathrm b))\otimes \Psi_\mathrm g(\nu_\mathrm g))\right).
   \end{array}
\]


  \begin{prop}\label{lem:cohred000}
       The  diagram
    \be\label{cdt}
    \begin{tikzcd}[column sep={4cm,between origins}]
      & \Coh_{\Lambda_{\mathrm b}}(\CK'(G'_{\mathrm b}))\otimes  \Coh_{\Lambda_{\mathrm g}}(\CK'(G_{\mathrm g}))
      \ar[dl,"\Ind \otimes  \mathrm{Id }"'] \ar[dr,"\Ind "]&\\
   \Coh_{\Lambda_\mathrm b}(\CK'(G_{\mathrm b}))   \otimes     \Coh_{\Lambda_{\mathrm g}}(\CK'(G_{\mathrm g})) \ar[rr,"\varphi"]& & \Coh_{\Lambda}(\CK'(G)).\\
    \end{tikzcd}
  \ee
   commutes.
 \end{prop}



  \begin{prop}\label{lem:cohred111}
    For all  $\pi'\in \Unip_{\ckcO'_{\mathrm b}}(G'_{\mathrm b})$ and  $\pi_{\mathrm g}\in  \Unip_{\ckcO_{\mathrm g}}(G_{\mathrm g})$, the  normalized smooth induction
    $\Ind_P^G (\pi_\mathrm g\widehat \otimes \pi')$ is irreducible and belongs to  $\Unip_{\ckcO}(G)$.
    %Here   $P$ is a parabolic subgroup of $G$ containing $G_\mathrm g\times G'_\mathrm b$ (or its quotient by $\{\pm 1\}$ when $\star=\wtC$) as a Levi subgroup.

 \end{prop}
\begin{proof}
By using the evaluation maps, the commutative  diagram \eqref{cdt} descends to a commutative diagram
\be\label{tri}
\begin{tikzcd}[column sep={4cm,between origins}]
      & \CK'_{\lambda_{\check \CO_\mathrm b}}(G'_{\mathrm b})   \otimes \CK'_{\lambda_{\check \CO_\mathrm g}}(G_{\mathrm g})
      \ar[dl," \Ind \otimes \mathrm{Id}"']\ar[dr,"\Ind "]&\\
      \CK'_{\lambda_{\check \CO_\mathrm b}}(G_{\mathrm b})\otimes  \CK'_{\lambda_{\check \CO_\mathrm g}}(G_{\mathrm g})\ar[rr,"\varphi_{\lambda_{\check \CO}}"]& & \CK'_{\lambda_{\check \CO}}(G).\\
    \end{tikzcd}
\ee
Here $\varphi_{\lambda_{\check \CO}}$ is as in Proposition \ref{propKL33}. Thus  $\Ind_P^G (\pi_\mathrm g\widehat \otimes \pi')$ is irreducible
    by Propositions \ref{lem:Unip.BP} and \ref{propKL33}. As before,   \cite[Corollary 5.0.10]{B.Orbit} implies that $\Ind_P^G (\pi_\mathrm g\widehat \otimes \pi')\in \Unip_{\ckcO}(G)$.
\end{proof}



By \Cref{lem:cohred111}, we have a well-defined map
\be\label{ind000}
  \Ind: \Unip_{\ckcO'_{\mathrm b}}(G'_{\mathrm b})\times \Unip_{\ckcO_{\mathrm g}}(G_{\mathrm g})   \longrightarrow \Unip_{\ckcO}(G).
  \ee
In view of the commutative diagram \eqref{tri},   Propositions \ref{propKL33} and \ref{lem:Unip.BP} implies that the map \eqref{ind000} is injective.
On the other hands, Propositions \ref{propKL333} and \ref{lem:Unip.BP} imply that the domain  and codomain of the map  \eqref{ind000} have the same cardinality. Thun the map  \eqref{ind000}  is bijective. This finishes the proof of  \Cref{reduction}.

% We would like to reduce the problem to consider the bad and good parts
% separately.
%

\def\fhhaso{(\fhh^a_1)^*}
\def\fhhast{(\fhh^a_2)^*}
\newcommand{\ff}{f}
\newcommand{\ffcoh}{\varphi}

\section{Combinatorics of painted bipartitions}


In this section, we assume that $\star \in \{B,C,\wtC,C^{*},D,D^{*}\}$, and $\ckcO = \ckcOg$, namely $\ckcO $ has $\star$-good parity.
Recall the set  $\CPPs(\ckcO)$ of primitive $\star$-pairs in $\ckcO$. For each subset $\wp$ of $\CPPs(\ckcO)$, we have defined a bipartition $\tau_{\wp}=(\imath_{\wp},\jmath_{\wp})$ in \Cref{sec:LCBCD}.

In this section, we study the set
\[
\PBPGOP := \set{\uptau \text{ is a painted bi
    partition}| G_{\uptau}=G,  (\imath_{\uptau},\jmath_{\uptau}) = \tau_{\wp}}
\]
for each $\wp\in \CPP(\ckcO)$.


\subsection{Main counting results}
%\subsection{Painted bipartitions in quaternionic group cases}
% The following two combinatorial results follow by induction on $\mathbf c_1(\check \CO)$. As the proof is quite tedious, we omit the details.
%
%\subsection{Non-existence of painted bipartition in quaternionic group cases}

For the quaternionic groups, we have the following vanishing results of painted
bipartition.

\begin{prop} \label{prop:PBP1} Suppose that $\star\in \set{C^{*}, D^{*}}$. Then
\[
    \PBP_{G}(\tau_{\wp}) = \emptyset, \quad \text{for all nonempty $\wp\subset \CPPs(\ckcO)$.}
  \]
 Consequently,
     \[
     \sharp(\tPBP_{G}(\ckcO)) = \sharp(\PBP_{G}(\ckcO)).
  \]
\end{prop}

\begin{proof}
Suppose that
  $\emptyset\neq \wp\subset \CPPs(\ckcO)$, and there was an element $\uptau = (\imath_{\wp}, \cP)\times (\jmath_{\wp},\cQ)\times \star\in \PBP_{G}(\tau_{\wp})$.

   First assume that  $\star = C^{*}$.  Pick an element $(2i-1, 2i)\in \wp$. Then we have that
  \begin{equation}\label{eq:res.C*}
    \bfcc_{i}(\imath_{\wp}) = \half(\bfrr_{2i-1}(\ckcO)+1)>
    \half(\bfrr_{2i}(\ckcO)-1) = \bfcc_{i}(\jmath_{\wp}).
      \end{equation}
   By the requirements of a painted bipartition, we also have that
  \begin{eqnarray*}
    \bfcc_{i}(\imath_{\wp})& = &\sharp\set{j\in \BN^+ \mid (i,j)\in \BOX{\imath_\wp}, \, \cP(i,j)=\bullet} \\
    &= &\sharp\set{j\in \BN^+\mid  (i,j)\in \BOX{\jmath_\wp}, \, \cQ(i,j)=\bullet} \\
    &\leq & \bfcc_{i}(\jmath_{\wp}).
  \end{eqnarray*}
 This contradicts \eqref{eq:res.C*} and therefore  $\PBP_{G}(\tau_{\wp})= \emptyset$.



  Now we assume that $\star = D^{*}$. Pick an element  $(2i, 2i+1)\in \wp$.
  Then we have that
  \begin{equation}\label{eq:res.D*}
    \bfcc_{i+1}(\imath_{\wp}) = \half(\bfrr_{2i}(\ckcO)+1)>
    \half(\bfrr_{2i+1}(\ckcO)-1) = \bfcc_{i}(\jmath_{\wp}).
  \end{equation}
      By the requirements of a painted bipartition, we also have that
  \begin{eqnarray*}
  \bfcc_{i+1}(\imath_{\wp})& \leq & \sharp\set{j\in \BN^+ \mid (i,j)\in \BOX{\imath_\wp}, \,  \cP(i,j)=\bullet} \\
  &=&\sharp\set{j\in \BN^+ \mid (i,j)\in \BOX{\jmath_\wp}, \,  \cQ(i,j)=\bullet} \\
  &\leq & \bfcc_{i}(\jmath_{\wp}).  \end{eqnarray*}
 This contradicts \eqref{eq:res.D*} and therefore  $\PBP_{G}(\tau_{\wp})= \emptyset$.
\end{proof}



%The purpose of this section is to prove the   following combinatorial result.


%We shall deal with the two quaternionic cases first, which are simple. When $\star \in \set{B, C, \wtC, D}$, the proof of the main statement of the above proposition involves an elaborate reduction argument (by removing elements from $\wp$ one-by-one), and will be handled separately in \Cref{lem:down} below.
%[Proof of {\Cref{prop:PBP}} the quaternionic case]



Because of the following vanishing proposition, we will only need to consider
quasi-distinguished nilpotent orbits $\ckcO$ when $\star\in \set{C^*, D^*}$.

\begin{prop}\label{prop:CD*}
  Suppose that $\star\in \set{C^*, D^*}$. If the set $\PBP_\star(\ckcO)$ is nonempty, then $\check \CO$ is quasi-distinguished.
\end{prop}
\begin{proof}
  Suppose that $\tau=(\imath,\cP)\times(\jmath,\cQ)\times \alpha \in  \mathrm{PBP}_\star(\check \CO)$. If  $\star=C^*$, then  the definition of painted bipartitions implies that
 \[
 \bfcc_i(\imath)\leq \bfcc_i(\jmath) \qquad \textrm{for all } i=1,2,3, \cdots.
 \]
This forces that $\check \CO$ is quasi-distinguished.

 If  $\star=D^*$, then  the definition of painted bipartitions implies that
 \[
 \bfcc_{i+1}(\imath)\leq \bfcc_i(\jmath) \qquad \textrm{for all } i=1,2,3, \cdots.
 \]
This  also forces that   $\check \CO$ is quasi-distinguished.
 \end{proof}

%\subsection{Descent of painted bi-partition}
The following proposition is clear by \Cref{lem:gf.C*}.
\begin{prop}\label{prop:count.CD*}
  Suppose that $\star\in \set{C^*, D^*}$ and $\ckcO$ is quasi-distinguished.
  Then $\# \PBP_{G}(\ckcO)$ equals to the number of real nilpotent orbits of $G$
  whose complexification is $\dBV(\ckcO)$.
\end{prop}



%\subsection{Counting in type $B,C,\wtC,D$}


%The following combinatorial result follows by induction on $\mathbf c_1(\check \CO)$.
In the rest of the section, we will define a notion of descent of painted
bipartition and then deduce the formulas of the generating functions on the
counting problem in \Cref{lem:gf.BD} and \Cref{lem:gf.C}. Then the following
proposition is a immediate consequence of the independence of $\wp$ of the
generating functions.

% we are devoted to establish the following counting
% result.
% which
% imply the following counting result.

\begin{prop} \label{prop:PBP2} Suppose that   $\star\in \set{B,C,\wtC,D}$. Then
  \[
    \sharp(\PBPOP) = \sharp(\PBPop{\star}{}{\ckcO}{\emptyset}), \quad \textrm{for all } \wp \subset \CPPs(\ckcO).
  \]
 Consequently,
     \[
     \sharp(\tPBP_{G}(\ckcO)) = 2^{\sharp(\CPPs(\ckcO))}\cdot \sharp(\PBP_{G}(\ckcO)).
  \]
\end{prop}

% We will define a notion of descent of painted
% bipartition and then deduce the formulas of the generating functions
% on the counting problem in \Cref{lem:gf.BD} and \Cref{lem:gf.C}.
% Then \Cref{prop:PBP2} is a immediate consequence of the independence of $\wp$ of
% the generating functions.

 \subsection{Shape shifting in type $C$ and $\wtC$}
 In this subsection, we assume $\star \in \set{C,\wtC}$ and $\ckcO$ is an orbit
 with good parity.



 Moreover, we assume that $(1,2)$ is a primitive pair, i.e.
 $(1,2)\in \CPPs(\ckcO)$. We have the following decomposition of the power set
 $A(\ckcO)$ of $\PP(\ckcO)$
 \[
   A(\ckcO) = \Ass(\ckcO) \sqcup \Ans(\ckcO)
 \]
 where
 \[
 \Ass(\ckcO): = \set{\wp | (1,2)\notin \wp}, \AND
 \Ans(\ckcO): = \set{\wp | (1,2)\in \wp}.
 \]
 For each $\wp\in \Ass(\ckcO)$, we
 define $\wp_{\uparrow}:= \wp\cup \set{(1,2)}$. Clearly, the map
 $\wp \mapsto \wp_{\uparrow}$ defines bijection from $\Ass(\ckcO)$ to
 $\Ans(\ckcO)$.
 % For each $\wp\in A(\ckcO)$, we define
 % $\wp_{\leftrightarrow}:= (\wp-\set{(1,2)})\cup (\set{(1,2)} - \wp)$. Clearly,
 % the map $\wp \mapsto \wp_{\leftrightarrow}$ defines bijection between
 % $\Ass(\ckcO)$ and $\Ans(\ckcO)$.
 %
 %
 Let $\wp\in \Ass(\ckcO)$. We have
  \[
    \begin{split}
      (\bfcc_{1}(\imath_{\wpu}), \bfcc_{t}(\jmath_{\wpu})) &=
      \begin{cases}
        (\bfcc_{1}(\jmath_{\wpd})+1, \bfcc_{t}(\imath_{\wpd})-1) & \text{when
        } \star = C\\
        (\bfcc_{1}(\jmath_{\wpd}), \bfcc_{t}(\imath_{\wpd}))& \text{when
        } \star = \wtC\\
      \end{cases}
      \\
     %  &= (b_{2},b_{1}),  \\
     % &= (b_{1}-1,b_{2}+1),\AND\\
      % (\bfcc_{t}(\imath_{\PPm}), \bfcc_{t}(\jmath_{\PPm})) &= (b_{2},b_{1}),  \\
      % (\bfcc_{t}(\imath_{\wp}), \bfcc_{t}(\jmath_{\wp})) &= (b_{1}-1,b_{2}+1),\AND\\
      \AND
      (\bfcc_{i}(\imath_{\wpu}),\bfcc_{i}(\jmath_{\wpu})) &=(\bfcc_{i}(\imath_{\wpd}),\bfcc_{i}(\jmath_{\wpd}))\quad \text{for $i\neq 1$}.
    \end{split}
  \]

 For
 $\uptau := (\imath_{\wp},\cP_{\uptau})\times (\jmath_{\wp},\cQ_{\uptau})\times \alpha \in \PBPOP $,
 we
 define
 \[
   \uptauu:= (\imath_{\wpu}, \cP_{\uptauu})\times (\jmath_{\wpu},\cQ_{\uptauu})\times \alpha
 \]
 by the following recipe.

 {\bfseries Suppose $\star = C$.}

 \newcommand{\localtextbulletone}{\textcolor{black}{\raisebox{.45ex}{\rule{.6ex}{.6ex}}}}

  For all $(i,j)\in \BOX{\jmath_{\wpu}}$, we define $\cP_{\uptauu}(i,j)$ case by
  case:
 \begin{enumerate}[label=(\alph*)]
   \item Suppose
   $\cP_{\uptaud}(\bfcc_{1}(\imath_{\wpd}),1)\neq \bullet$.
   \begin{enumerate}[label={\localtextbulletone}]
     \item If $\bfcc_{1}(\imath_{\wpd})\geq 2$ and
     $\cP_{\uptau}(\bfcc_{1}(\imath_{\wpd})-1,1) = c$,
     we define
     \[
       \cP_{\uptauu}(i,j) := \begin{cases}
         r ,& \text{if $j=1$ and $\bfcc_{1}(\imath_{\wpd})-1
           \leq i \leq \bfcc_{1}(\imath_{\wpu})-2$},\\
         c ,& \text{if $(i,j)=(\bfcc_{1}(\imath_{\wpu})-1,1)$},\\
         d ,& \text{if $(i,j)=(\bfcc_{1}(\imath_{\wpu}),1)$},\\
         \cP_{\uptaud}(i,j) ,&\text{otherwise};
       \end{cases}
     \]
     \item Otherwise, we define
     \[
       \cP_{\uptauu}(i,j) := \begin{cases}
         r ,& \text{if $j=1$ and $\bfcc_{1}(\imath_{\wpu})
           \leq i \leq \bfcc_{1}(\imath_{\wpu})-1$},\\
         \cP_{\uptaud}(\bfcc_{1}(\imath_{\wpd}),1) ,&
         \text{if $(i,j)=(\bfcc_{1}(\imath_{\wpu}),1)$},\\
         \cP_{\uptaud}(i,j) ,&\text{otherwise};
       \end{cases}
     \]
   \end{enumerate}
   \item Suppose $\cP_{\uptaud}(\bfcc_{1}(\imath_{\wpd}),1)=\bullet$.
   \begin{enumerate}[label={\localtextbulletone}]
     \item If $\bfcc_{2}(\imath_{\wpd}) = \bfcc_{1}(\imath_{\wpd})$
     and
     $\cP_{\uptaud}(\bfcc_{1}(\imath_{\wpd}),2) = r$,
     we define
     \[
       \cP_{\uptauu}(i,j) := \begin{cases}
         r ,& \text{if $j=1$ and $\bfcc_{t}(\imath_{\wpd})\leq i \leq \bfcc_{t}(\imath_{\wpu})-1$},\\
         c ,& \text{if $(i,j)=(\bfcc_{2}(\imath_{\wpd}),2)$},\\
         d ,& \text{if $(i,j)=(\bfcc_{1}(\imath_{\wpu}),1)$},\\
         \cP_{\uptaud}(i,j) ,&\text{otherwise}.
       \end{cases}
     \]
     \item Otherwise, we define
     \[
       \cP_{\uptauu}(i,j) := \begin{cases}
         r ,& \text{if $j=1$ and $\bfcc_{1}(\imath_{\wpd})\leq i \leq \bfcc_{1}(\imath_{\wpu})-2$},\\
         c ,& \text{if $(i,j)=(\bfcc_{1}(\imath_{\wpu})-1,t)$},\\
         d ,& \text{if $(i,j)=(\bfcc_{1}(\imath_{\wpu}),t)$},\\
         \cP_{\uptaud}(i,j) ,&\text{otherwise}.
       \end{cases}
     \]
   \end{enumerate}
 \end{enumerate}

  For all $(i,j)\in \BOX{\jmath_{\wpu}}$,
   \[
     \cQ_{\uptauu}(i,j) := \cQ_{\uptau}(i,j).
   \]

{\bfseries Suppose $\star = \wtC$.}

  For all $(i,j)\in \BOX{\imath_{\wpu}}$,
  \begin{equation} \label{eq:T.M1}
     \cP_{\uptauu}(i,j) :=
     \begin{cases}
       s& \text{if $j=1$ and  $\cQ_{\uptau}(i,j)=s$,}\\
       c& \text{if $j=1$ and  $\cQ_{\uptau}(i,j)=c$,}\\
       \cP_{\uptau}(i,j) &\text{otherwise.}
     \end{cases}
     % \cP_{\uptau}(i,j).
   \end{equation}

  For all $(i,j)\in \BOX{\jmath_{\wpu}}$,
   % For $\uptaum\in\PBPs(\tau_{\PPm})$, define $\uptau=:T_{\PPm,\wp}(\uptaum)$
   % by the following formula:
  \begin{equation} \label{eq:T.M2}
    \cQ_{\uptauu}(i,j) :=
    \begin{cases}
      r& \text{if $j=1$ and  $\cP_{\uptau}(i,j)=s$,}\\
      d& \text{if $j=1$ and  $\cP_{\uptau}(i,j)=c$,}\\
      \cQ_{\uptau}(i,j) &\text{otherwise.}
    \end{cases}
  \end{equation}


 % to an element $\uptauu\in \PBP_{\star}(\ttau_{\wpu})$ such that

\begin{lem}\label{lem:sn}
  Let $\wp\in \Ass(\ckcO)$. The map $\uptau\mapsto \uptauu$ defined above
  gives a well defined
  bijection
  \[
    T_{\wp,\wpu}\colon \PBPOP \rightarrow
    \PBPop{\star}{}{\ckcO}{\wpu} %. \quad \uptau \mapsto \uptauu
  \]
\end{lem}
\begin{proof}
  It is routine to check that $T_{\wp,\wpu}$ is well defined, i.e. $\uptauu$ is a valid painted bipartition.
  Similarly, it is routine to check that the map
  \[
    T_{\wpu,\wp}\colon
    \PBPop{\star}{}{\ckcO}{\wpu} \rightarrow \PBPOP
  \]
  given by the following recipe is well defined.

  {\bfseries Suppose $\star = C$.}
  For all $(i,j)\in \BOX{\jmath_{\wpd}}$, we define $\cP_{\uptaud}(i,j)$ and $\cQ_{\uptaud}(i,j)$ case by
  case:
 \begin{enumerate}[label=(\alph*)]
   \item Suppose
   $\cP_{\uptauu}(\bfcc_{1}(\imath_{\wpu})-1,1)=c$.
   \begin{enumerate}[label={\localtextbulletone}]
     \item
           If $\bfcc_{1}(\jmath_{\wpu})\geq 1$ and
           $\cP_{\uptau}(\bfcc_{1}(\jmath_{\wpu}),1) = r$,

     %       If $\bfcc_{1}(\imath_{\wpd})\geq 2$ and
     % $\cP_{\uptau}(\bfcc_{1}(\imath_{\wpd})-1,1) = r$,
     we define
     \[
       \cP_{\uptaud}(i,j) := \begin{cases}
         c ,& \text{if $(i,j)=(\bfcc_{1}(\imath_{\wpd})-1,1)$},\\
         d ,& \text{if $(i,j)=(\bfcc_{1}(\imath_{\wpd}),1)$},\\
         \cP_{\uptauu}(i,j) ,&\text{otherwise};
       \end{cases}
     \]
     and
     \[
       \cQ_{\uptaud}(i,j) := \begin{cases}
         s ,& \text{if $j=1$ and $\bfcc_{1}(\imath_{\wpd})\leq i \leq \bfcc_{1}(\jmath_{\wpd})$},\\
         \cQ_{\uptauu}(i,j) ,&\text{otherwise};
       \end{cases}
     \]
     %% Note that c_1(j_wpu)+1 =c_1(i_wpd)
     \item Otherwise, we define
     \[
       \cP_{\uptaud}(i,j) := \begin{cases}
         \bullet ,& \text{if $(i,j)=(\bfcc_{1}(\imath_{\wpd}),1)$},\\
         \cP_{\uptauu}(i,j) ,&\text{otherwise};
       \end{cases}
     \]
     and
     \[
       \cQ_{\uptaud}(i,j) := \begin{cases}
         \bullet ,& \text{if $(i,j)=(\bfcc_{1}(\imath_{\wpd}),1)$},\\
         s ,& \text{if $j=1$ and $\bfcc_{1}(\imath_{\wpd})+1\leq i \leq \bfcc_{1}(\jmath_{\wpd})$},\\
         \cQ_{\uptauu}(i,j) ,&\text{otherwise};
       \end{cases}
     \]
   \end{enumerate}
   \item Suppose $\cP_{\uptauu}(\bfcc_{1}(\imath_{\wpu})-1,1)=r$.
   \begin{enumerate}[label={\localtextbulletone}]
     \item If $\bfcc_{2}(\imath_{\wpu}) = \bfcc_{1}(\jmath_{\wpu})+1$
     and
     $(\cP_{\uptauu}(\bfcc_{2}(\imath_{\wpu}),2),\cP_{\uptauu}(\bfcc_{1}(\imath_{\wpu}),1) ) = (c,d)$,
     we define
     \[
       \cP_{\uptaud}(i,j) := \begin{cases}
         \bullet ,& \text{if $(i,j)=(\bfcc_{1}(\imath_{\wpd}),1)$},\\
         r ,& \text{if $(i,j)=(\bfcc_{2}(\imath_{\wpd}),2)$},\\
         \cP_{\uptaud}(i,j) ,&\text{otherwise}.
       \end{cases}
     \]
     and
     \[
       \cQ_{\uptaud}(i,j) := \begin{cases}
         \bullet ,& \text{if $(i,j)=(\bfcc_{1}(\imath_{\wpd}),1)$},\\
         s ,& \text{if $j=1$ and $\bfcc_{1}(\imath_{\wpd})+1\leq i \leq \bfcc_{1}(\jmath_{\wpd})$},\\
         \cQ_{\uptauu}(i,j) ,&\text{otherwise};
       \end{cases}
     \]
     \item Otherwise, we define
     \[
       \cP_{\uptauu}(i,j) := \begin{cases}
         \cP_{\uptaud}(\bfcc_{1}(\imath_{\wpu}),1), & \text{if $(i,j)=(\bfcc_{1}(\imath_{\wpd}),1)$},\\
         \cP_{\uptaud}(i,j) ,&\text{otherwise}.
       \end{cases}
     \]
     and
     \[
       \cQ_{\uptaud}(i,j) := \begin{cases}
         s ,& \text{if $j=1$ and $\bfcc_{1}(\imath_{\wpd})\leq i \leq \bfcc_{1}(\jmath_{\wpd})$},\\
         \cQ_{\uptauu}(i,j) ,&\text{otherwise};
       \end{cases}
     \]
   \end{enumerate}
 \end{enumerate}


  {\bfseries Suppose $\star = \wtC$.}
  The definition of $T_{\wpu,\wp}$ is given by the formula \eqref{eq:T.M1} and
  \eqref{eq:T.M2} with the
  role of $\wp$ and $\wpu$ switched.


 %   \begin{description}
 %     \item[STEP 1] We first recover $\cP'$.
 %           If $t=1$ or $\cP'(\bfcc_{t}(\imath_{\wp})-1,t-1)=\bullet$, then
 %           $\cP':= \cP_{\wp}$.
 %           Otherwise,
 %           $\cP'$ is given by $\cP_{\wp}$ except the $2\times 2$ square in
 %           \eqref{eq:modP} which is given by reversing the formula cited.
 %     \item[STEP 2]

 %           \def\xxn{\cP_{\uptaum}(\bfcc_t(\imath_{\PPm})-1,t)} %x_0
 %           \def\xxo{\cP_{\uptaum}(\bfcc_t(\imath_{\wp}),t)} %x_1
 %           \def\xxd{\cP_{\uptaum}(\bfcc_t(\imath_{\wp}),t+1)} %x_2
 %           \def\yyn{\cP'(\bfcc_t(\imath_{\PPm})-1,t)} %y_0
 %           \def\yyo{\cP'(\bfcc_t(\imath_{\wp})-1,t)} %y_1
 %           \def\yyt{\cP'(\bfcc_t(\imath_{\wp}),t)} %y_3
 %           \def\yyd{\cP'(\bfcc_t(\imath_{\wp}),t+1)} %y_2
 %           We have the following cases:
 %           \begin{enumerate}[label=(\alph*)]
 %             \item Suppose $\yyo=r$.
 %             \begin{itemize}
 %               \item If $\bfcc_{t+1}(\imath_{\wp}) = \bfcc_{t}(\imath_{\PPm})$
 %               and
 %               \[
 %                 (\yyd,\yyt) = (c,d),
 %               \]
 %               let
 %               \[
 %                 (\xxo,\xxd):=(\bullet, r)
 %               \]
 %               \item Otherwise, let \[
 %                 \xxo:=\yyt.
 %               \]
 %             \end{itemize}
 %             \item Suppose $\yyo=c$
 %             \begin{itemize}
 %               \item If $\bfcc_{t}(\imath_{\PPm})\geq 2$ and $\xxn=r$,
 %               then let
 %               \[
 %                 (\xxn,\xxo):=(c,d).
 %               \]
 %               \item Otherwise, let
 %               \[
 %                 \xxo :=\bullet.
 %               \]
 %             \end{itemize}
 %           \end{enumerate}
 %           For the boxes $(i,j)$ in $\BOX{\imath_{\uptaum}}$ which are not specified
 %           in the above procedure, set
 %           \[
 %           \cP_{\uptaum}(i,j):=\cP'(i,j).
 %           \]
 %     \item[STEP 3]
 %           Now $\cP_{\uptaum}$ uniquely determine the painted bipartition
 %           $\uptaum$.
 %   \end{description}
 %   The construction of the inverse map implies that $T_{\PPm,\wp}$ is a
 %   bijection.
 % }

\end{proof}

In the next section, we will define the descent map of painted biparttion and reduce the proof of
\Cref{prop:PBP2} to the basic case.
% % The rest of this section is devoted to the proof of the following lemma, which
% % clearly implies Proposition \ref{prop:PBP}.




% % Suppose $\star = \wtC$, $\wp\neq \emptyset$, and
% % $t:=\min{t|(2t-1,2t)\in \wp}$. Let $\PPm:=\wp - \set{(2t-1,2t)}$. Let
% % $\PPm:=\wp - \set{(2t-1,2t)}$.

% \begin{lem}\label{lem:down}
%   Suppose $\star \in \set{B, C, \wtC, D}$ and $\wp$ is a non-empty subset of
%   $\CPPs(\ckcO)$. Let
%   \[
%     t:=
%     \begin{cases}
%       \min\set{i|(2i-1,2i)\in \wp}, & \text{if $\star \in \set{C,\wtC}$};\\
%       \min\set{i|(2i,2i+1)\in \wp}, & \text{if $\star \in \set{B,D}$},\\
%     \end{cases}
%   \]
%   and
%   \[
%     \PPm:=
%     \begin{cases}
%       \wp \setminus \set{(2t-1,2t)},  & \text{if $\star \in \set{C,\wtC}$};\\
%       \wp \setminus   \set{(2t,2t+1)}, & \text{if $\star \in \set{B,D}$}.\\
%     \end{cases}
%   \]
%   Then
%   \[
%     \sharp(\PBP_{\star}(\tau_{\PPm})) = \sharp(\PBP_{\star}(\tau_{\wp})).
%   \]
% \end{lem}

% \begin{proof}
%   We prove the equality by defining a bijection
%   \[
%     T_{\PPm,\wp}\colon \PBP_{\star}(\tau_{\PPm}) \rightarrow \PBP_{\star}(\tau_{\wp})\quad \uptaum \mapsto \uptau
%   \]
%   % and its inverse $T_{\wp,\PPm}$
%   explicitly case by case. In the following,
%   $\uptau = (\imath_{\wp},\cP_{\uptau})\times (\jmath_{\wp},\cQ_{\uptau})$ will
%   always denote an element in $\PBP_{\star}(\tau_{\wp})$ and
%   $\uptaum = (\imath_{\PPm},\cP_{\uptaum})\times (\jmath_{\PPm},\cQ_{\uptaum})$
%   an element in $\PBP_{\star}(\tau_{\PPm})$.

%   \medskip
% \end{proof}

\subsection{Tails of painted bipartitions of type $B$ and $D$}
\label{sec:tail}
In this subsection, we assume that $\abs{\ckcO}>0$ and
$\star\in\set{B, D, C^*}$. We further assume that $\ckcO$ is quasi-distinguished
if $\star = C^{*}$.

We define the notion of ``tail'' of a painted bipartition.

%Let $(\imath,\jmath) = (\imath_{\star}(\ckcO),\jmath_{\star}(\ckcO))$.
%Note that $l\geq l'$ if $\star\in \{B,C^*\}$,   and $l\geq l'+1$ if $\star=D$.
\def\startt{\star_{\mathrm t}}
Put
\[
  \startt:= \begin{cases}
  D, & \ \text{ if $\star\in \set{B,D}$}; \\
  C^*, &\  \text{ if $\star=C^*$}.
\end{cases}
\quad
\text{and}
\quad
k := \begin{cases}
  \frac{\bfrr_{1}(\ckcO)-\bfrr_{2}(\ckcO)}{2} + 1 &
    \text{if $\star\in \{B,D\}$}; \\
\frac{\bfrr_{1}(\ckcO)-\bfrr_{2}(\ckcO)}{2} -1 &  \text{if $\star=C^*$}.
  \end{cases}
\]
In all the cases, $k$ is a positive integer when $\star \in \set{B,D}$ and $k$ is
non-negative.

\trivial[h]{
Note that $\bfrr_{2}(\ckcO)>0$ is odd in our assumption.
}

% We have $k = \bfcc_{1}(\jmath)-\bfcc_{1}(\imath)+1$,
% $\bfcc_{1}(\jmath)-\bfcc_{1}(\imath)$,
% and $\bfcc_{1}(\imath)-\bfcc_{1}(\jmath)$
% when $\star = B,D,C^{*}$, respectively.

From the pair $(\star, \ckcO)$, we define a Young diagram $\ckcO_{\bftt}$ as follows:
\begin{itemize}
  \item If $\star \in \set{B,D}$, then $\ckcO_{\bftt}$ consists of two rows with
        lengths $2k-1$ and $1$.
  \item If $\star =C^*$, then $\ckcO_{\bftt}$ consists of one row with length
        $2k+1$.
\end{itemize}
Note that in all three cases
 $\check \CO_{\mathbf t}$ has $\star_{\mathbf t}$-good parity and every element in $\PBP_{\star_\bftt}(\ckcO_\bftt)$ has the form
 \begin{equation}
 \label{tail0}
  \ytb{{x_1} , {x_2} , {\enon\vdots},{\enon{\vdots}},{x_k}  } \times \emptyset \times
  D,\qquad \qquad  \ytb{{x_1} , {x_2} , {\enon\vdots},{\enon{\vdots}},{x_k}  } \times \emptyset \times
  D\qquad\textrm{or}\qquad \emptyset \times  \ytb{{x_1} , {x_2} , {\enon\vdots},{\enon{\vdots}},{x_k}  } \times
 C^*,
\end{equation}
according to $\star=B, D$ or $C^*$, respectively. %Here $k$ can be zero if $\star = C^*$.

% Here $k=l-l'+1, l-l'$ or $l-l'$ respectively.
%\subsubsection{The case when $\star = B$}

\medskip


Let $ \uptau=(\imath,\cP)\times(\jmath,\cQ)\times \alpha \in
\mathrm{PBP}_\star(\check \CO) $. The tail $\uptau_\bftt$ of $\uptau$ will be a painted bipartition in
$\PBP_{\star_\bftt}(\ckcO_\bftt)$, which is defined below case by case.

{\bfseries The case when $\star = B$:}

In this case, we define the tail $\tau_\bftt$ to be the first painted bipartition in \eqref{tail0} such that the multiset $\{x_1, x_2, \cdots, x_k\}$ is the
union of the multiset
\[
  \set{\cQ(j,1)| \bfcc_{1}(\imath)+1 \leq j \leq  \bfcc_{1}(\jmath) }
    % \cQ(\bfcc_{1}(\imath)+1,1),\cQ(\bfcc_{1}(\imath)+2,1),\cdots, \cQ(\bfcc_{1}(\jmath),1)}
\]
with the set
\[
  \begin{cases}
 \set{c}, &
 \qquad
  \text{if $\alpha = B^+$, and either $\bfcc_{1}(\imath)=0$ or $\cQ(\bfcc_{1}(\imath),1)\in \set{\bullet,s}$};  \\
 \set{s},&
  \qquad \text{if $\alpha = B^-$, and either $\bfcc_{1}(\imath)=0$ or $\cQ(\bfcc_{1}(\imath),1)\in \set{\bullet,s}$}; \\
%  \qquad\text{when } \alpha_\tau = B^-, \text{ and, } l'=0 \textrm{ or } \cQ_\tau(l',1)\in \set{\bullet,s},  \\
\set{\cQ(\bfcc_{1}(\imath),1)},&
\qquad
\text{otherwise.}
%\text{if $\bfcc_{1}(\imath)>0$ and $\cQ(\bfcc_{1}(\imath),1)\in \{r,d\}$.}
\end{cases}
\]

{\bfseries The case when $\star = D$:}
In this case, we define the tail $\tau_\bftt$ to be the second painted
bipartition in \eqref{tail0} such that the multiset $\{x_1, x_2, \cdots, x_k\}$
is the union of the multiset
\[
\set{\cP(j,1)| \bfcc_{1}(\jmath)+2 \leq j \leq \bfcc_{1}(\imath)}
\]
with the set
\[
\begin{cases}
    \set{c},                          &
    \ \text{if $\bfrr_2(\ckcO)=\bfrr_3(\ckcO)$,}                                                                         \\
                                      & \quad \text{$(\cP(\bfcc_{1}(\jmath)+1,1) ,\cP(\bfcc_{1}(\jmath)+1,2)) = (r,c)$ } \\
                                      & \quad \text{ and $\cP(\bfcc_1(\imath),1)\in \set{r,d}$};                                       \\
    \set{\cP(\bfcc_{1}(\jmath)+1,1)}, &
    \    \text{otherwise.}
  \end{cases}
\]


{\bfseries The case $\star = C^*$:}

If $k=0$, we define the tail $\tau_{\bftt}$ to be
$\emptyset\times \emptyset \times C^{*}$.
If $k> 0$, we define the tail $\tau_\bftt$ to be the third painted bipartition in \eqref{tail0} such that
\[
  (x_1, x_2, \cdots, x_k)= (\cQ(\bfcc_{1}(\imath)+1,1),\cQ(\bfcc_{1}(\imath)+2,1),\cdots, \cQ(\bfcc_{1}(\jmath),1)).
\]


 When $\star \in \set{B,D}$, the symbol in the last box of the tail $\uptau_\bftt\in \PBP_{\star_\bftt}(\ckcO_\bftt)$ will be important for us. We write $x_\uptau$ for it, namely
\[
x_\uptau := \cP_{\uptau_\bftt}(k,1).
\]


\trivial[h]{

 The following technical lemma is easy to check.

\begin{lem}\label{tailtip}
If $\star=B$, then
\[
\hspace{9em} x_\tau=s\Longleftrightarrow
\begin{cases}
  \alpha=B^-;\\
  \bfcc_1(\jmath)=0 \text{ or }\cQ(\bfcc_{1}(\jmath),1) \in\set{\bullet, s},
  \end{cases}
%\quad \textrm{if and only if}\quad \alpha=B^- \ \textrm{ and }\  \cQ(l,1) = s,
\]
and
\[
x_\tau=d \Longleftrightarrow
%\quad \textrm{if and only if}\quad
\cQ(\bfcc_{1}(\jmath),1) =d.
\]
If $\star=D$, then
\[
x_\tau=s\Longleftrightarrow \cP(\bfcc_{1}(\imath),1) = s,
\]
and
\[
x_\tau=d\Longleftrightarrow \cP(\bfcc_{1}(\imath),1) =d.
\]
\qed
\end{lem}
}

\subsection{Descents of painted bipartitions}\label{sec:comb}


In this subsection, we define the descent of a painted bipartition.
%as alluded to in Section \ref{subsec:comTOrep}.
As before, let  $\star\in \{ B, C,  D, \widetilde{C},  C^*, D^*\}$ and let $\check \CO$ be a Young diagram that has $\star$-good parity.


Define a symbol
\[
\star':=\widetilde{C}, \ D, \  C, \ B, \ D^*\  \textrm{ or } \ C^*
\]
respectively if
\[
\star=B,\  C, \ D, \ \widetilde{C}, \ C^* \ \textrm{ or }\  D^*.
\]
We call $\star'$
 the Howe dual of $\star$.

 Define the naive dual descent of a Young diagram $\ckcO$ to be
 \[
  \ckDDn(\ckcO):= \textrm{the Young diagram obtained from $\check \CO$ by removing the first row}.
  \]
   By convention, $\check \nabla_{\mathrm{naive}}(\emptyset)=\emptyset$.

   The dual descent of $\check \CO$ is defined to be
  \[
   \check \CO':=\check \nabla(\check \CO):=\check \nabla_\star( \check \CO):=\begin{cases}
   \tytb{\   }\, , \quad& \textrm{if $\star\in \{D, D^*\}$ and $\check \CO=\emptyset$};\smallskip\\
   \check \nabla_{\mathrm{naive}}(\check \CO), \quad & \textrm{otherwise},
    \end{cases}
  \]
  where $\tytb{\   }$ denotes the Young diagram that has total size $1$.

  % Here $\emptyset$ stands for the empty Young diagram, and $\tytb{\ }$ stands for the Young diagram of total size $1$.

\delete{Put
\begin{equation}\label{lstarco}
  l:=l_{\star, \check \CO}:=\begin{cases}
 \frac{\bfrr_1(\ckcO)}{2}; & \quad \textrm{if } \star\in \{B, \widetilde C\};\smallskip\\
 \frac{\bfrr_1(\ckcO)-1}{2}, &\quad \textrm{if } \star\in \{C, C^* \};\smallskip\\
 \frac{\bfrr_1(\ckcO)+1}{2}, &\quad \textrm{if } \star\in \{ D, D^*\}.\\
\end{cases}
\end{equation}
This is the length of the leading column of every element of $\mathrm{PBP}_\star(\check \CO)$.
}

For a Young diagram $\imath$, its naive descent, which is denoted by $\nabla_\mathrm{naive}(\imath)$, is defined to be the Young diagram obtained from $\imath$ by removing the first column. By convention, $\nabla_\mathrm{naive}(\emptyset)=\emptyset$.

%In the rest of this section, we assume that $\check \CO\neq \emptyset$.

 %Put\[
%l':=l_{\star', \check \CO'}
%\]


\def\bipartl{\mathrm{bi\cP_L}}
\def\bipartr{\mathrm{bi\cP_R}}
\def\dsdiagl{\mathrm{DS_L}}
\def\dsdiagr{\mathrm{DS_R}}
\def\DDl{\eDD_\mathrm{L}}
\def\DDr{\eDD_\mathrm{R}}

We first define the naive descent of a painted bipartition. Let $\uptau=(\imath,\cP)\times (\jmath,\cQ)\times \alpha$ be a painted bipartition such that $\star_\tau=\star$. Put
\delete{\begin{equation} \label{eq:def.alphap}
\alpha'=\begin{cases} B^+,
& \textrm{if $\alpha = \wtC$ and $\cP_\tau(l_{\star,\ckcO},1),1) \neq c$;}\\
B^-,
& \textrm{if $\alpha = \wtC$ and $\cP_\tau(l_{\star,\ckcO},1),1)  = c$;}\\
\star', & \textrm{if $\alpha\neq \widetilde C$}.
\end{cases}
\end{equation}
}
  \begin{equation} \label{eq:def.alphap}
    \alphapn=\begin{cases} B^+,
  & \textrm{if $\alpha=\widetilde{C}$ and $c$ does not occur in the first column of $(\imath,\cP)$}; \smallskip \\
  B^-,
  & \textrm{if $\alpha=\widetilde{C}$ and  $c$ occurs in the first column of $(\imath,\cP)$}; \smallskip \\
  \star', & \textrm{if $\alpha\neq \widetilde C$}.
  \end{cases}
  \end{equation}

\begin{lem}\label{lemDDn1}
  If $\star \in \set{B,C,C^*}$, then there is a unique painted bipartition of the form $\uptaupn= (\imathpn,\cPpn)\times (\jmathpn,\cQpn)\times \alphapn$ with the following properties:
  \begin{itemize}
        \item $
   (\imathpn,\jmathpn)= (\imath,\DD_\mathrm{naive}(\jmath)); \smallskip
   $
   \item for all $(i,j)\in \BOX{\imathpn}$,
   \[
     \cPpn(i,j)=\begin{cases}
    \bullet \textrm{ or } s,&\textrm{ if  $\ \cP(i,j)\in \{\bullet, s\}$;} \smallskip \\
  \cP(i,j),& \textrm{ if $\ \cP(i,j)\notin \{\bullet, s\}$};\end{cases}
   \]
   \item for all $(i,j)\in \BOX{\jmathpn}$,
   \[
     \cQpn(i,j)=\begin{cases}
    \bullet \textrm{ or } s,&\textrm{ if  $\ \cQ(i,j+1)\in \{\bullet, s\}$;} \smallskip \\
  \cQ(i,j+1), & \textrm{ if $\ \cQ(i,j+1)\notin \{\bullet, s\}$}.  \end{cases}
   \]
    \end{itemize}
    \end{lem}




   \begin{proof}
    First assume that the images of $\cP$ and $\cQ$ are both contained in $\{\bullet, s\}$. Then  the image of $\cP$  is in fact contained in $\{\bullet\}$, and $(\imath, \jmath)$ is  right interlaced in the sense that
 \[
 \mathbf{c}_1(\jmath)\geq \mathbf{c}_1(\imath)\geq \mathbf{c}_2(\jmath)\geq \mathbf{c}_2(\imath)\geq \mathbf{c}_3(\jmath)\geq \mathbf{c}_3(\imath) \geq \cdots.
 \]
 Hence $ (\imath',\jmath'):= (\imath,\DD(\jmath))$ is left interlaced in the sense that
 \[
 \mathbf{c}_1(\imath')\geq \mathbf{c}_1(\jmath')\geq \mathbf{c}_2(\imath')\geq \mathbf{c}_2(\jmath')\geq \mathbf{c}_3(\imath')\geq \mathbf{c}_3(\jmath') \geq \cdots.
 \]
 Then it is clear that there is a unique painted bipartition of the form  $\uptaupn=(\imathpn,\cPpn)\times (\jmathpn,\cQpn)\times \alphapn$ such that images of $\cPpn$ and $\cQpn$ are both contained in $\{\bullet, s\}$. This proves the lemma in the special case when the images of $\cP$ and $\cQ$ are both contained in $\{\bullet, s\}$.

 The proof of the lemma in the general case is easily reduced to this special case.
   \end{proof}
    \begin{lem}\label{lemDDn2}
    If $\star \in \set{ \wtC, D,D^*}$, then there is a unique painted bipartition of the form $\uptaupn= (\imathpn,\cPpn)\times (\jmathpn,\cQpn)\times \alphapn$ with the following properties:
  \begin{itemize}
        \item $
   (\imathpn,\jmathpn)= (\DD_\mathrm{naive}(\imath),\jmath); \smallskip
   $
   \item for all $(i,j)\in \BOX{\imathpn}$,
   \[
     \cPpn(i,j)=\begin{cases}
    \bullet \textrm{ or } s,&\textrm{ if  $\ \cP(i,j+1)\in \{\bullet, s\}$;} \smallskip \\
  \cP(i,j+1),& \textrm{ if $\ \cP(i,j+1)\notin \{\bullet, s\}$};\end{cases}
   \]
   \item for all $(i,j)\in \BOX
          {\jmathpn}$,
   \[
     \cQpn(i,j)=\begin{cases}
    \bullet \textrm{ or } s,&\textrm{ if  $\ \cQ(i,j)\in \{\bullet, s\}$;} \smallskip \\
  \cQ(i,j), & \textrm{ if $\ \cQ(i,j)\notin \{\bullet, s\}$}.  \end{cases}
   \]

    \end{itemize}
\end{lem}
\begin{proof}
  The proof is similar to that of \Cref{lemDDn1}.
\end{proof}

\begin{defn}
 In the notation of \Cref{lemDDn1,lemDDn2}, we call $\uptaupn$ the naive descent of $\uptau$, to be denoted by $\DDn(\uptau)$.
\end{defn}




 \begin{eg} If
    \[
     \uptau = \ytb{\bullet\bullet\bullet {c},\bullet {s} {c},{s},{c}}
    \times \ytb{\bullet\bullet\bullet ,\bullet {r} {d},{d}{d}, \none}
    \times \widetilde C, \]
   then
   \[
    \nabla_{\mathrm{naive}}(\uptau) =\ytb{\bullet\bullet{c} ,\bullet{c},\none }
    \times  \ytb{\bullet\bullet {s} ,\bullet {r} {d},{d}{d}}\times B^-.
    \]

\end{eg}


From now on, we suppose that $\ckcO$ is non-empty.
Let $\ckcO' := \DD(\ckcO)$, $\wp':=\DD(\wp)$ and $\tauwpp = (\imathwpp,\jmathwpp)$ be the
bipartition defined by \eqref{eq:ttauwp} with respect to $\ckcO'$.

Suppose that $\wp\in A(\ckcO)$.
\[
\uptau=(\imath,\cP)\times(\jmath,\cQ)\times \alpha \in  \PBPOP.
\]
% and write
% \[
%   \uptaupn=(\imath'_{\mathrm{naive}}, \cP'_{\mathrm{naive}})\times (\jmath'_{\mathrm{naive}}, \cQ'_{\mathrm{naive}})\times \alpha'
% \]
% for the naive descent of $\uptau$.


We define
\[
  \uptau' := (\imathwpp, \cP')\times (\jmathwpp, \cQ')\times \alpha'
\]
such that $\cP'$ and $\cQ'$ are paintings on $\BOX{\imath_{\wp'}}$ and
$\BOX{\jmath_{\wp'}}$  % determined
by the following rules  respectively:

%the descent of the painted bipartition $\uptau$ by to be


% This is clearly an element of $  \PBP(\check \CO')$.
%Put
%\begin{equation}\label{lstarco}
%  l:=l_{\star, \check \CO}:=\begin{cases}
% \frac{\bfrr_2(\ckcO)}{2}; & \quad \textrm{if } \star\in \{B, \widetilde C\};\\
% \frac{\bfrr_2(\ckcO)+1}{2}, &\quad \textrm{if } \star\in \{C, C^* \};\\
% \frac{\bfrr_2(\ckcO)-1}{2}, &\quad \textrm{if } \star\in \{ D, D^*\}.\\
%\end{cases}
%\end{equation}

{\bfseries Case $\star = B$. }
We define $\alpha' := \alphapn$.
%\begin{enumerate}[label={\localtextbulletone}]
\begin{enumerate}[label=(\alph*)]
  \item  Suppose
                \[
                \begin{cases}
                  \alpha = B^+; & \\
                  (2,3) \notin \wp & \\
                  \bfrr_2(\ckcO)>0; & \\
                  \cQ(\bfcc_1(\imath_{\wp}),1)\in \set{r,d}.
                \end{cases}
                \]
                We define %
                \[
                \cP'(i,j) := \begin{cases}
                  s, & \ \text{ if $(i,j) = (\bfcc_1(\imathwpp),1)$;}\\
                  \cPpn(i,j), & \ \text{ otherwise},
                \end{cases}
                % \quad \text{for all $(i,j)\in \BOX{\imathwpp}$,}
                \] for all $(i,j)\in \BOX{\imathwpp}$, and
                $\cQ' := \cQpn $.
                \trivial[h]{
                  Note that $\bfcc_{1}(\imathwpp) = \bfcc_{1}(\imath_{\wp})$
                }
  \item Suppose
                \[
                \begin{cases}
                  \alpha = B^+; & \\
                  (2,3)\in \wp & \\
                  \cQ(\bfcc_2(\jmath_{\wp}),1)\in \set{r,d},
                \end{cases}
                \]
                We define
                $\cP' := \cPpn $ and
                \[
                \cQ'(i,j) := \begin{cases}
                  r, & \ \text{ if $(i,j) = (\bfcc_1(\jmathwpp),1)$;}\\
                  \cQpn(i,j), & \ \text{ otherwise},
                \end{cases}
                \] for all $(i,j)\in \BOX{\jmathwpp}$.
                \trivial[h]{
                  Note that $\bfcc_{1}(\jmathwpp) = \bfcc_{2}(\jmath_{\wp})$
                }
        \item Otherwise, we define $\cP' := \cPpn$ and $\cQ':= \cQpn$.
\end{enumerate}

{\bfseries Case $\star = D$. }
We define $\alpha' := \alphapn$.
\begin{enumerate}[label=(\alph*)]
  \item  Suppose
  \[
    \begin{cases}
      \bfrr_2(\ckcO)=\bfrr_{3}(\ckcO)>0; & \\
      \cP(\bfcc_{2}(\imath_{\wp}),2) = c;  &\\
      \cP(i,1)\in \set{r,d}, & \text{for all
        $\bfcc_{2}(\imath_{\wp})\leq i\leq \bfcc_{1}(\imath_{\wp})$}.
    \end{cases}
  \]
  We define %
  \[
    \cP'(i,j) := \begin{cases}
      r, & \ \text{ if $(i,j) = (\bfcc_1(\imathwpp),1)$;}\\
      \cPpn(i,j), & \ \text{ otherwise},
    \end{cases}
    % \quad \text{for all $(i,j)\in \BOX{\imathwpp}$,}
  \] for all $(i,j)\in \BOX{\imathwpp}$, and
  $\cQ' := \cQpn $.
  \item Suppose
  \[
    \begin{cases}
      (2,3)\in \wp & \\
      \cP(\bfcc_2(\imath_{\wp})-1,1)\in \set{r,c},
    \end{cases}
  \]
  We define
  \[
    \cP'(i,j) := \begin{cases}
      r, & \ \text{ if $(i,j) = (\bfcc_1(\imathwpp)-1,1)$;}\\
      \cP(\bfcc_2(\imath_{\wp})-1,1) & \ \text{ if $(i,j) = (\bfcc_1(\imathwpp),1)$;}\\
      \cPpn(i,j), & \ \text{ otherwise},
    \end{cases}
  \] for all $(i,j)\in \BOX{\imathwpp}$.
  $\cQ' := \cQpn $ and
  \item Otherwise, we define $\cP' := \cPpn$ and $\cQ':= \cQpn$.
\end{enumerate}

{\bfseries Case $\star \in \set{C,\wtC,C^{*},D^{*}}$}

\begin{enumerate}[label=(\alph*)]
  \item Suppose $(1,2)\notin \wp$. We define
  \[
    \uptau' := \DDn(\uptau).
  \]
  % \[
  %   \alpha' = \alphapn, \quad
  %   \cP' := \cPpn \AND
  %   \cQ' := \cQpn.
  % \]
  \item Suppose $(1,2)\in \wp$. We define
  \[
  \uptau' = \DDn(\uptau_{\wpm})
  \]
  where $\wpm := \wp - \set{(1,2)}$ and
  $\uptau_{\wpm}  := T_{\wpm,\wp}^{-1}(\uptau)$ (see \Cref{lem:sn}).
        (Note that when $\star \in \set{C^{*},D^{*}}$, case (b) never happen.)
\end{enumerate}

It is routine to check that the above definition do gives valid painted
bipartition and we record the following lemma.
\begin{lem}
  Suppose that $\ckcO$ is non-empty and $\ckcO' := \DD(\ckcO)$.
  For each $\wp\in \PP(\ckcO)$, the map $\uptau\mapsto \uptau'$ defined above
  gives a well defined map
  \begin{equation}
    \label{eq:DD.CC}
    \DD \colon \PBPOP \rightarrow \PBPOPp.
  \end{equation}
  \qed
\end{lem}

Note that, when $\star\in \set{C^{*}, D^{*}}$, we will only consider the case
when $\ckcO$ is quasi-distinguished and $\wp=\emptyset$ since
$\PBPOP = \emptyset $ otherwise.


The following proposition summaries the key properties of the descent map.
\begin{prop} \label{lem:PBPd.C}
  Suppose that $\star\in \set{C,\wtC, D^*}$. Let $\ckcO$ be a non-empty partition of good
  parity and $\ckcO' := \DD(\ckcO)$.
  \begin{enumerate}[label=(\alph*)]
    \item When $\star = D^{*}$ or $\bfrr_1(\ckcO)>\bfrr_2(\ckcO)$,
    the map \eqref{eq:DD.CC} is bijective. %Case $D^{*}$.
    \item When $\star\in \{C,\widetilde C\}$ and $\bfrr_1(\ckcO)=\bfrr_2(\ckcO)$,
    then the  map \eqref{eq:DD.CC} is injective and its image equals
    \[
      \Set{\uptau'\in \PBPOPp |  x_{\uptau'}\neq s}.
    \]
  \end{enumerate}
\end{prop}
\begin{proof}
  This is clear from the definition of the descent algorithm.
\end{proof}



\begin{prop}
\label{lem:delta}
Suppose that $\star \in \set{D,B,C^*}$ and $\bfrr_2(\ckcO)>0$. Write
$\ckcOpp := \ckDD(\ckcO')$, $\wp'':= \DD(\DD(\wp))$ and let $\tauwppp$ be the
bipartition attached to $\wp''$ with respect to $\ckcO''$. Consider the map
\begin{equation}\label{eq:delta}
  \delta  \colon \PBPOP \longrightarrow
    \PBPOPp \times \PBP_{\star_\bftt}(\ckcO_\bftt),
    \qquad \uptau \mapsto (\DD^2(\uptau),\uptau_\bftt).
\end{equation}
\begin{enumerate}[label=(\alph*)]
\item Suppose that
$\star = C^*$ or $\bfrr_2(\ckcO)>\bfrr_3(\ckcO)$. Then the map \eqref{eq:delta} is bijective, and for every $\uptau\in  \PBP_\star(\ckcO) $,
    % We have the following equation of signatures.
\begin{equation}\label{eq:sign.D}
\Sign(\uptau)
=(\bfcc_2(\cO),\bfcc_2(\cO))+\Sign(\DD^2(\uptau))+\Sign(\uptau_\bftt).
\end{equation}

\item Suppose that  $\star \in \set{B,D}$ and $\bfrr_2(\ckcO)=\bfrr_3(\ckcO)>0$.
Then the map \eqref{eq:delta} is an  injection and its  image equals
\begin{equation}\label{eq:delta.I}
    \Set{ (\uptau'',\uptau_0)  \in \PBPOPpp \times \PBP_D(\ckcO_\bftt)  |
    \begin{array}{l}
        \text{either
    $x_{\uptau''} = d$, or} \\
    \text{$x_{\uptau''}\in \set{r,c}$  and
    $\cP_{\uptau_0}^{-1}(\set{s,c})\neq \emptyset$}
    \end{array}
}.
\end{equation}
Moreover,  for every $\uptau\in  \PBPOP$,
\begin{equation}\label{eq:sign.GD}
\Sign(\uptau)
=(\bfcc_2(\cO)-1,\bfcc_2(\cO)-1)+\Sign(\DD^2(\uptau))+\Sign(\uptau_\bftt).
\end{equation}
\end{enumerate}
\end{prop}
\begin{proof}
  % One can deduce the inverse of $\delta$ (implemented in \cite{MU}) and signature formula via a patient analysis % The proposition can be verified
  % % by a patient analysis
  % of the descent algorithm.
  This follows from detailed description of the descent algorithm.
\end{proof}



We have the following immediate corollary.

\begin{cor}\label{prop:DD.BDinj}
  Suppose $\star \in \set{B, D,C^*}$.
  We define
  \[
    \varepsilon_{\uptau} = \begin{cases}
      0 & \text{if $\star\in \set{B,D}$, and $x_{\uptau}=d$,}\\
      1 & \text{otherwise.}
    \end{cases}
  \]
  Then the map
\begin{equation}
  \begin{array}{rcl}
   \PBPOP&\rightarrow&
   \PBPOPp \times \BN\times \bN\times \Z/2\Z, \smallskip\\
   \uptau& \mapsto & (\DD(\uptau), p_\uptau, q_\uptau, \varepsilon_\uptau)
   \end{array}
\end{equation}
is injective.
\end{cor}
\begin{proof}
  It is easy to verify the the basic case where $\bfrr_{3}(\ckcO)=0$ and
    \[
      (\bfrr_{1}(\ckcO), \bfrr_{2}(\ckcO))
      =
      \begin{cases}
        (2k-2,0) & \text{if } \star=B,\\
        (2k-1,1) & \text{if } \star=D,\\
        (2k-1,0) & \text{if } \star=C^{*}.\\
      \end{cases}
    \]
    \trivial[h]{
    Then $\ckcO$ is the regular nilpotent orbit.
    % % in $\check \fgg$
    and $\cO =\dBV(\ckcO) = (1^{\bfrr_{1}(\ckcO)+1})_{\star}$.
    }
    The other cases follows from the injectivity of $\DD$ in \Cref{lem:PBPd.C}
    and the signature formula in \Cref{lem:delta}.
\end{proof}



Now we define some generating functions of painted bipartition. %$(\ckcO,\wp)$

When $\star \in  \set{B,C^{*}, D}$, we define
\[
   f_{\star,\ckcO,\wp}(\bfpp,\bfqq) := \sum_{\uptau \in \PBPOP} \bfpp^{p_{\uptau}} \bfqq^{q_{\uptau}}.
\]
for each $\wp\in \CPPs(\ckcO)$.
The coefficient of $\bfpp^{p}\bfqq^{q}$ in $f_{\star,\ckcO,\wp}(\bfpp,\bfqq)$  equals the cardinality of
 $\PBPop{\SO(p,q)}{}{\ckcO}{\wp}$ if $\star\in \set{B,D}$ or
 $\PBPop{\Sp(\frac{p}{2},\frac{q}{2})}{}{\ckcO}{\wp}$ if $\star = C^{*}$.


\trivial[]{
Clearly $f_{\star,\ckcO,\wp}=0$, if $\star\in \set{C^*}$ and $\wp\neq \emptyset$.
}

When $\star \in \set{B,D}$,  for each subset $S\subset \set{c,d,r,s}$, we define
\[
  \PBPOP[S] = \set{\uptau\in \PBPOP|x_{\uptau}\in S}
\]
and the corresponding generating function
\[
   f_{\star,\ckcO,\wp}^{S}(\bfpp,\bfqq) := \sum_{\uptau \in \PBPOP[S]} \bfpp^{p_{\uptau}} \bfqq^{q_{\uptau}}
\]
in $\bZ[\bfpp,\bfqq]$ where $(p_{\uptau}, q_{\uptau})$ is the signature of
$\uptau$ and $\bfpp$
and $\bfqq$ are indeterminants.
In the following we will compute
$f_{\star,\ckcO,\wp}^{\set{s}}$, $f_{\star,\ckcO,\wp}^{\set{c,r}}$ and
$f_{\star,\ckcO,\wp}^{\set{d}}$ whose sum is the desired function
$f_{\star, \ckcO,\wp}$.
% % By to ease the notation,
% \[
%   f_{\star,\ckcO,\wp} %:= f_{\star,\ckcO,\wp}^{\set{c,d,r,s}} =
%   f_{\star,\ckcO,\wp}^{\set{s}}
%   + f_{\star,\ckcO,\wp}^{\set{c,r}}
%   + f_{\star,\ckcO,\wp}^{\set{d}}
% \]
% whose coefficient of $\bfpp^{p}\bfqq^{q}$ equals the cardinality of
% $\PBPop{\SO(p,q)}{}{\ckcO}{\wp}$.
% $\PBPGOP$
% with $\Sign(G)=(p,q)$.

\def\CSk#1#2{h_{#1}^{#2}}
\def\TSk#1#2{g_{#1}^{#2}}
\def\RS{\nu}

For an integer $k$,
we define
\[
\RS_{k} := \sum_{i=0}^{k} \bfpp^{2i}\bfqq^{2(k-i)}
\]
if $k\geq 0$ and $0$ otherwise.
\trivial[]{
This is the generating function of all PBP of type D with only one row filled
by r or s.
}

We make the following definition:
\[
  \begin{split}
  \TSk{k}{\set{d}} &:= f_{D,\yrow{2k-1,1},\emptyset}^{\set{d}}= \bfpp\bfqq\RS_{k-1} + \bfpp^{2}\bfqq^{2}\RS_{k-2}, \\
  \TSk{k}{\set{c,r}} &:= f_{D,\yrow{2k-1,1},\emptyset}^{\set{c,r}}= (\bfpp\bfqq+\bfpp^{2})\RS_{k-1},  \\
  \TSk{k}{\set{s}}& :=\CSk{k}{\set{s}} := f_{D,\yrow{2k-1,1},\emptyset}^{\set{s}}= \bfqq^{2n},\\
  \CSk{k}{\set{d}} &:= \bfpp^{2}\bfqq^{2}\RS_{k-2} + \bfpp\bfqq^{3}\RS_{k-2}, \AND \\
  \CSk{k}{\set{c,r}} &:= \bfpp\bfqq\RS_{k-1}+\bfpp^{2}\bfqq^{2}\RS_{k-2}.\\
  \end{split}
\]
\trivial[h]{
  For reference, rest = column filled by r and s.
\[
  \begin{split}
  \TSk{k}{\set{d}} &:=  rest + d\cup rest+cd \\
  \TSk{k}{\set{c,r}} &:= rest+c \cup rest+r\\
  \TSk{k}{\set{s}}& :=\CSk{k}{\set{s}} := ss\cdots s,\\
  \CSk{k}{\set{d}} &:= rest+cd \cup s+rest + d\\
  \CSk{k}{\set{c,r}} &:= rest+c \cup s+rest+r \\
  \end{split}
\]
}

It is also straight to see that, for non-negative  integer $k$,
\[
  \begin{split}
   f_{B,\yrow{2k},\emptyset}^{\set{d}} & = (\bfpp^{2}\bfqq+ \bfpp\bfqq^{3})\RS_{k-1}, \\
   f_{B,\yrow{2k},\emptyset}^{\set{c,r}} & = \bfpp\RS_{k} + \bfpp^{2}\bfqq\RS_{k-1},  \\
  f_{B,\yrow{2k},\emptyset}^{\set{s}} & = \bfqq^{2k+1},\\
  \end{split}
\]
\trivial[h]{
  For reference, rest := count columns filled by r and s.
\[
  \begin{split}
    f_{B,\yrow{2k},\emptyset}^{\set{d}} &:=  B^{+} + rest + d \cup B^{-}+rest+d, \\
    f_{B,\yrow{2k},\emptyset}^{\set{c,r}}   &:= B^{+} + rest \cup B^{-}+rest+r, \\
    f_{B,\yrow{2k},\emptyset}^{\set{s}} & := B^{-} + ss\cdots s.\\
  \end{split}
\]
}

By convention, we set
\[
  f_{D,\yrow{0}, \emptyset}^{\set{d}} := 1, \quad
  f_{D,\yrow{0}, \emptyset}^{\set{c,r}} := 0,
  \AND
  f_{D,\yrow{0}, \emptyset}^{\set{s}} := 0,
\]

\begin{lem}\label{lem:gf.BD}
  Suppose $\star\in\set{B,D}$ and $\ckcO$ is a good parity partition such that $\bfrr_{2}(\ckcO)>0$.

  Let $k:=\frac{\bfrr_{1}(\ckcO)-\bfrr_{2}(\ckcO)}{2}+1$.
  Then we have the following  recursive formulas of the generating functions
  with $S \in \set{\set{d}, \set{c,r}, \set{s}}$:
  \begin{enumerate}[label=(\alph*)]
    \item If $(2,3)\in \CPPs(\ckcO)$, then
    \[
      f_{\star,\ckcO, \wp}^{S} = (\bfpp\bfqq)^{\bfcc_{1}(\ckcO)} \TSk{k}{S} \cdot  f_{\star,\DD^{2}(\ckcO),\DD^{2}(\wp)}.
    \]
    \item If $(2,3)\notin \CPPs(\ckcO)$,
    then
    \[
      f_{\star,\ckcO, \wp}^{S} := (\bfpp\bfqq)^{\bfcc_{1}(\ckcO)-1} (\TSk{k}{S} \cdot f_{\star,\DD^{2}(\ckcO),\DD^{2}(\wp)}^{\set{d}}
      + \CSk{k}{S} \cdot f_{\star,\DD^{2}(\ckcO),\DD^{2}(\wp)}^{\set{c,r}}).
    \]
  \end{enumerate}
  In particular, the generating function is independent of $\wp$ and so
  \[
    f_{\star,\ckcO,\wp} = f_{\star,\ckcO,\emptyset} \quad \text{for all $\wp\in \CPPs(\ckcO)$.}
  \]
\end{lem}
\begin{proof}
  This follows immediately from \Cref{lem:delta}.
\end{proof}


\begin{lem}\label{lem:gf.C}
  Suppose $\star\in\set{C,\wtC}$ and $\ckcO$ is a good parity partition such that $\bfrr_{1}(\ckcO)>0$.
  Then % we have the following  formulas of the generating functions
  % with $S \in \set{\set{d}, \set{c,r}, \set{s}}$:
  \begin{enumerate}[label=(\alph*)]
    \item If $(1,2)\in \CPPs(\ckcO)$, then
    \[
      \# \PBPOP =  f_{\star',\DD(\ckcO), \DD(\wp)}(1,1).
    \]
    \item If $(1,2)\notin \CPPs(\ckcO)$,
    then
    \[
      \# \PBPOP = f_{\star',\DD(\ckcO), \DD(\wp)}^{\set{c,r}}(1,1) + f_{\star',\DD(\ckcO), \DD(\wp)}^{\set{d}}(1,1).
    \]
  \end{enumerate}
  In particular, the cardinality of $\PBPOP$ is independent of $\wp$ and so
  \[
    \# \PBPOP = \# \PBPop{\star}{}{\ckcO}{\emptyset} \quad \text{for all $\wp\in \CPPs(\ckcO)$.}
  \]
\end{lem}
\begin{proof}
  This follows immediately from \Cref{lem:delta} and \Cref{lem:gf.BD}.
\end{proof}

% In this section, we exhibit the inductive structure of painted bipartitions.
% The key is to define the descent of a painted bipartition.
% As before, let  $\star\in \set{B, C,  D, \wtC,  C^*, D^*}$ and let $\check \CO$ be a Young diagram that has $\star$-good parity.

\subsection{More precise result in type $C^{*}$ and $D^{*}$}
The following lemma is straight forward  to check.
\begin{lem}\label{lem:gf.C*}
  Suppose $\star = C^{*}$.
  \begin{enumerate}[label=(\alph*)]
    \item Then %The following equation holds:
          \[
          f_{\star, \yrow{2k-1},\emptyset} = \RS_{k}.
          \]
    \item Assume that $\ckcO$ is a non-empty good parity partition such that
          $\bfrr_{2}(\ckcO)>0$. Then $\DD^{2}(\ckcO)$ is also non-empty and there is
          the following recursive formula
          \[
          f_{\star,\ckcO,\emptyset} = (\bfpp\bfqq)^{\bfrr_{2}(\ckcO)+1} \RS_{k}\cdot f_{\star,\DD^{2}(\ckcO),\emptyset}
          \]
          where $k = \frac{\bfrr_{1}(\ckcO)-\bfrr_{2}(\ckcO)}{2}-1$.
  \end{enumerate}
\end{lem}


We have the following consequence of the above lemma.

\begin{lem}\label{lem:gf.C*}
  When $\star = D^{*}$ and $\ckcO$ is non-empty,
  the following equation holds
  \[
    \# \PBPGOP = f_{\star',\DD(\ckcO),\emptyset}(1,1).
  \]
  Suppose $\star\in \set{C^{*}, D^{*}}$. The cardinality of  $\PBPGOP(\ckcO)$ equals
  to the real nilpotent orbit of $G$ whose complexification is $\dBV(\ckcO)$.
  \qed
\end{lem}



% \section{Combinatorics of painted bipartitions}

% \trivial[h]{
% In this section, we assume that $\star \in \{B,C,\wtC,C^{*},D,D^{*}\}$, and $\ckcO = \ckcOg$, namely $\ckcO $ has $\star$-good parity.
% Recall the set  $\CPPs(\ckcO)$ of primitive $\star$-pairs in $\ckcO$. For each subset $\wp$ of $\CPPs(\ckcO)$, we have defined a bipartition $\tau_{\wp}=(\imath_{\wp},\jmath_{\wp})$ in \Cref{sec:LCBCD}.

% \begin{prop} \label{prop:PBP} If $\star\in \set{C^{*}, D^{*}}$, then
%   \[
%     \PBP_{\star}(\tau_{\wp}) = \emptyset, \quad \text{for all nonempty subset   $\wp\subset \CPPs(\ckcO)$.}
%   \]
% \end{prop}
% \begin{proof}
% Suppose that
%   $\emptyset\neq \wp\subset \CPPs(\ckcO)$.

%    First assume that  $\star = C^{*}$. Then we have that
%   \begin{equation}\label{eq:res.C*}
%     \bfcc_{i}(\imath_{\wp}) = \half(\bfrr_{2i-1}(\ckcO)+1)>
%     \half(\bfrr_{2i}(\ckcO)-1) = \bfcc_{i}(\jmath_{\wp}),
%     \quad \textrm{ for all } \,\, (2i-1, 2i)\in \wp,
%   \end{equation}
%   Let $\uptau = (\imath_{\wp}, \cP)\times (\jmath_{\wp},\cQ)\times \star$ be an element in $\PBP_{\star}(\tau_{\wp})$. By the requirements of a painted bipartition, we have that
%   \[
%     \bfcc_{i}(\imath_{\wp}) = \sharp\set{j| \cP(i,j)=\bullet} = \sharp\set{j| \cQ(i,j)=\bullet} \leq \bfcc_{i}(\jmath_{\wp}), \quad \textrm{ for all } \,\, i=1,2,3,\cdots,
%   \]
%   which contradicts \eqref{eq:res.C*}. Hence, $\PBP_{\star}(\tau_{\wp})= \emptyset$.



%   Now we assume that $\star = D^{*}$.
%   Then we have that
%   \begin{equation}\label{eq:res.D*}
%     \bfcc_{i+1}(\imath_{\wp}) = \half(\bfrr_{2i}(\ckcO)+1)>
%     \half(\bfrr_{2i+1}(\ckcO)-1) = \bfcc_{i}(\jmath_{\wp}),
%     \quad \textrm{ for all } \,\, (2i, 2i+1)\in \wp.
%   \end{equation}
%   Let $\uptau = (\imath_{\wp}, \cP)\times (\jmath_{\wp},\cQ)\times \star$ be an element in $\PBP_{\star}(\tau_{\wp})$. By the requirements of a painted bipartition, we have that
%   \[
%     \bfcc_{i+1}(\imath_{\wp}) \leq \sharp\set{j| \cP(i,j)=\bullet} =\sharp\set{j| \cQ(i,j)=\bullet} \leq \bfcc_{i}(\jmath_{\wp}), \quad \textrm{ for all } \,\, i = 1,2,3, \cdots,
%   \]
%   which contradicts \eqref{eq:res.D*}. Hence, $\PBP_{\star}(\tau_{\wp})= \emptyset$.

% \end{proof}



% %The purpose of this section is to prove the   following combinatorial result.


% %We shall deal with the two quaternionic cases first, which are simple. When $\star \in \set{B, C, \wtC, D}$, the proof of the main statement of the above proposition involves an elaborate reduction argument (by removing elements from $\wp$ one-by-one), and will be handled separately in \Cref{lem:down} below.
% %[Proof of {\Cref{prop:PBP}} the quaternionic case]

%     \smallskip


% \begin{prop} \label{prop:PBP} Suppose that   $\star\in \set{B,C,\wtC,D}$. Then
%   \[
%     \sharp(\PBP_{\star}(\tau_{\wp})) = \sharp(\PBP_{\star}(\tau_{\emptyset})), \quad \textrm{for all } \wp \subseteq \CPPs(\ckcO).
%   \]
%  Consequently,
%   \[
%     \sharp(\tPBP_{\star}(\ckcO)) = 2^{\sharp(\CPPs(\ckcO))}\cdot \sharp(\PBP_{\star}(\ckcO)), \,\, \text{ and } \,\, \sharp(\tPBP_{\mathrm g}(\ckcO)) = 2^{\sharp(\CPPs(\ckcO))}\cdot \sharp(\PBP_{\mathrm g}(\ckcO)).
%   \]
% \end{prop}

% The rest of this section is devoted to the proof of the following lemma, which clearly implies Proposition \ref{prop:PBP}.
% }

% In this section, we assume that $\star \in \{B,C,\wtC,C^{*},D,D^{*}\}$, and $\ckcO = \ckcOg$, namely $\ckcO $ has $\star$-good parity.
% Recall the set  $\CPPs(\ckcO)$ of primitive $\star$-pairs in $\ckcO$. For each subset $\wp$ of $\CPPs(\ckcO)$, we have defined a bipartition $\tau_{\wp}=(\imath_{\wp},\jmath_{\wp})$ in \Cref{sec:LCBCD}.

% The following two combinatorial results follow by induction on $\mathbf c_1(\check \CO)$. As the proof is quite tedious, we omit the details.
% \begin{prop} \label{prop:PBP1} Suppose that $\star\in \set{C^{*}, D^{*}}$. Then
% \[
%     \PBP_{G}(\tau_{\wp}) = \emptyset, \quad \text{for all nonempty $\wp\subset \CPPs(\ckcO)$.}
%   \]
%  Consequently,
%      \[
%      \sharp(\tPBP_{G}(\ckcO)) = \sharp(\PBP_{G}(\ckcO)).
%   \]
% \end{prop}

% \begin{proof}
% Suppose that
%   $\emptyset\neq \wp\subset \CPPs(\ckcO)$, and there was an element $\uptau = (\imath_{\wp}, \cP)\times (\jmath_{\wp},\cQ)\times \star\in \PBP_{G}(\tau_{\wp})$.

%    First assume that  $\star = C^{*}$.  Pick an element $(2i-1, 2i)\in \wp$. Then we have that
%   \begin{equation}\label{eq:res.C*}
%     \bfcc_{i}(\imath_{\wp}) = \half(\bfrr_{2i-1}(\ckcO)+1)>
%     \half(\bfrr_{2i}(\ckcO)-1) = \bfcc_{i}(\jmath_{\wp}).
%       \end{equation}
%    By the requirements of a painted bipartition, we also have that
%   \begin{eqnarray*}
%     \bfcc_{i}(\imath_{\wp})& = &\sharp\set{j\in \BN^+ \mid (i,j)\in \BOX{\imath_\wp}, \, \cP(i,j)=\bullet} \\
%     &= &\sharp\set{j\in \BN^+\mid  (i,j)\in \BOX{\jmath_\wp}, \, \cQ(i,j)=\bullet} \\
%     &\leq & \bfcc_{i}(\jmath_{\wp}).
%   \end{eqnarray*}
%  This contradicts \eqref{eq:res.C*} and therefore  $\PBP_{G}(\tau_{\wp})= \emptyset$.



%   Now we assume that $\star = D^{*}$. Pick an element  $(2i, 2i+1)\in \wp$.
%   Then we have that
%   \begin{equation}\label{eq:res.D*}
%     \bfcc_{i+1}(\imath_{\wp}) = \half(\bfrr_{2i}(\ckcO)+1)>
%     \half(\bfrr_{2i+1}(\ckcO)-1) = \bfcc_{i}(\jmath_{\wp}).  \end{equation}
%       By the requirements of a painted bipartition, we also have that
%   \begin{eqnarray*}
%   \bfcc_{i+1}(\imath_{\wp})& \leq & \sharp\set{j\in \BN^+ \mid (i,j)\in \BOX{\imath_\wp}, \,  \cP(i,j)=\bullet} \\
%   &=&\sharp\set{j\in \BN^+ \mid (i,j)\in \BOX{\jmath_\wp}, \,  \cQ(i,j)=\bullet} \\
%   &\leq & \bfcc_{i}(\jmath_{\wp}).  \end{eqnarray*}
%  This contradicts \eqref{eq:res.D*} and therefore  $\PBP_{G}(\tau_{\wp})= \emptyset$.
% \end{proof}



% %The purpose of this section is to prove the   following combinatorial result.


% %We shall deal with the two quaternionic cases first, which are simple. When $\star \in \set{B, C, \wtC, D}$, the proof of the main statement of the above proposition involves an elaborate reduction argument (by removing elements from $\wp$ one-by-one), and will be handled separately in \Cref{lem:down} below.
% %[Proof of {\Cref{prop:PBP}} the quaternionic case]

%     \smallskip

% The following combinatorial result follows by induction on $\mathbf c_1(\check \CO)$. We omit the details.

% \begin{prop} \label{prop:PBP2} Suppose that   $\star\in \set{B,C,\wtC,D}$. Then
%   \[
%     \sharp(\PBP_{G}(\tau_{\wp})) = \sharp(\PBP_{G}(\tau_{\emptyset})), \quad \textrm{for all } \wp \subset \CPPs(\ckcO).
%   \]
%  Consequently,
%      \[
%      \sharp(\tPBP_{G}(\ckcO)) = 2^{\sharp(\CPPs(\ckcO))}\cdot \sharp(\PBP_{G}(\ckcO)).
%   \]
% \end{prop}


% \trivial[h]{
% The rest of this section is devoted to the proof of the following lemma, which clearly implies Proposition \ref{prop:PBP}.



% \def\PPm{\wp_{\downarrow}}
% \def\uptaum{\uptau_{\downarrow}}

% % Suppose $\star = \wtC$, $\wp\neq \emptyset$, and
% % $t:=\min{t|(2t-1,2t)\in \wp}$. Let $\PPm:=\wp - \set{(2t-1,2t)}$. Let
% % $\PPm:=\wp - \set{(2t-1,2t)}$.

% \begin{lem}\label{lem:down}
%   Suppose $\star \in \set{B, C, \wtC, D}$ and
%   $\wp$ is a non-empty subset of $\CPPs(\ckcO)$.
%   Let
%   \[
%     t:=
%     \begin{cases}
%       \min\set{i|(2i-1,2i)\in \wp}, & \text{if $\star \in \set{C,\wtC}$};\\
%       \min\set{i|(2i,2i+1)\in \wp}, & \text{if $\star \in \set{B,D}$},\\
%     \end{cases}
%   \]
%   and
%   \[
%     \PPm:=
%     \begin{cases}
%       \wp \setminus \set{(2t-1,2t)},  & \text{if $\star \in \set{C,\wtC}$};\\
%       \wp \setminus   \set{(2t,2t+1)}, & \text{if $\star \in \set{B,D}$}.\\
%     \end{cases}
%   \]
%   Then
%   \[
%     \sharp(\PBP_{\star}(\tau_{\PPm})) = \sharp(\PBP_{\star}(\tau_{\wp})).
%   \]
% \end{lem}

% \begin{proof}
%   We prove the equality by defining a bijection
%   \[
%     T_{\PPm,\wp}\colon \PBP_{\star}(\tau_{\PPm}) \rightarrow \PBP_{\star}(\tau_{\wp})\quad \uptaum \mapsto \uptau
%   \]
%   %and its inverse $T_{\wp,\PPm}$
%   explicitly case by case.
%   In the following, $\uptau = (\imath_{\wp},\cP_{\uptau})\times (\jmath_{\wp},\cQ_{\uptau})$
%   will always denote an element in $\PBP_{\star}(\tau_{\wp})$ and
%   $\uptaum = (\imath_{\PPm},\cP_{\uptaum})\times (\jmath_{\PPm},\cQ_{\uptaum})$
%   an element in $\PBP_{\star}(\tau_{\PPm})$.

%   \medskip

%   % \[
%   %   T_{\wp,\PPm}\colon \PBP_{\star}(\tau_{\wp}) \rightarrow \PBP_{\star}(\tau_{\PPm}).
%   % \]


%   %We start with the simplest case.

%   \smallskip

%   Case $\star = \wtC$:
%   %Let $(b_{1},b_{2}) = (\frac{\bfrr_{2t-1}(\ckcO)}{2},\frac{\bfrr_{2t}(\ckcO)}{2})$.
%   We have
%   \[
%     \begin{split}
%       (\bfcc_{t}(\imath_{\PPm}), \bfcc_{t}(\jmath_{\PPm}))
%       &= (\bfcc_{t}(\jmath_{\wp}), \bfcc_{t}(\imath_{\wp})),\AND\\
%       % (\bfcc_{t}(\imath_{\PPm}), \bfcc_{t}(\jmath_{\PPm}))
%       % &= (b_{1},b_{2})= (\bfcc_{t}(\jmath_{\wp}), \bfcc_{t}(\imath_{\wp})),\AND\\
%       (\bfcc_{i}(\imath_{\PPm}), \bfcc_{i}(\jmath_{\PPm}) )
%       & = (\bfcc_{i}(\imath_{\wp}), \bfcc_{i}(\jmath_{\wp}) ) \quad \text{for $i\neq t$}.
%     \end{split}
%   \]

%   For $\uptaum\in\PBPs(\tau_{\PPm})$, define $\uptau=:T_{\PPm,\wp}(\uptaum)$ by the following formula:
%   \[
%     \begin{split}
%       \text{$\textrm{ for all } (i,j)\in \BOX{\imath_{\wp}}$,} \quad   \cP_{\uptau}(i,j) &=  \cP_{\uptaum}(i,j),\\
%       \text{$\textrm{ for all } (i,j)\in \BOX{\jmath_{\wp}}$,} \quad \cQ_{\uptau}(i,j) &= \begin{cases}
%         r& \text{if $j=t$ and  $\cP_{\uptaum}(i,j)=s$,}\\
%         d& \text{if $j=t$ and  $\cP_{\uptaum}(i,j)=c$,}\\
%         \cQ_{\uptaum}(i,j) &\text{otherwise.}
%       \end{cases}
%     \end{split}
%   \]
%  We easily check that the above formula defines a valid
%   painted bipartition $\uptau$ and construct the inverse map $T_{\wp,\PPm}$ by reversing the process.
%   This finishes the proof for the case when $\star=\wtC$. \medskip

%   \trivial[h]{The inverse map
%   \[
%     T_{\wp,\PPm}\colon \PBP_{\star}(\tau_{\wp}) \rightarrow \PBP_{\star}(\tau_{\PPm}).
%   \]
%   is given by the following formula:
%   \[
%     \begin{split}
%       \text{$\textrm{ for all } (i,j)\in \BOX{\imath_{\PPm}}$,} \quad \cP_{\uptaum}(i,j) &= \begin{cases}
%         s& \text{if $j=t$ and  $\cQ_{\uptau}(i,j)=r$,}\\
%         c& \text{if $j=t$ and  $\cQ_{\uptau}(i,j)=d$,}\\
%         \cP_{\uptau}(i,j) &\text{otherwise.}
%       \end{cases}\\
%       \text{$\textrm{ for all } (i,j)\in \BOX{\jmath_{\PPm}}$,} \quad   \cQ_{\uptaum}(i,j) &=  \cQ_{\uptau}(i,j).\\
%     \end{split}
%   \]
%   }
%   % We leave it to the reader to check that the above formula does define a valid
%   % painted bipartition $\uptaum$. Retain the above notation, it is easy to check that the
%   % inverse map
%   % \[
%   %   T_{\PPm,\wp}\colon \PBP_{\star}(\tau_{\PPm}) \rightarrow \PBP_{\star}(\tau_{\wp})\quad \uptaum \mapsto \uptau
%   % \]
%   % is given by the following formula:
%   % \[
%   %   \begin{split}
%   %     \text{$\textrm{ for all } (i,j)\in \BOX{\imath_{\wp}}$,} \quad   \cP_{\uptau}(i,j) &=  \cP_{\uptaum}(i,j),\\
%   %     \text{$\textrm{ for all } (i,j)\in \BOX{\jmath_{\wp}}$,} \quad \cQ_{\uptau}(i,j) &= \begin{cases}
%   %       r& \text{if $j=t$ and  $\cP_{\uptaum}(i,j)=s$,}\\
%   %       d& \text{if $j=t$ and  $\cP_{\uptaum}(i,j)=c$,}\\
%   %       \cQ_{\uptaum}(i,j) &\text{otherwise.}
%   %     \end{cases}
%   %   \end{split}
%   % \]

%   \medskip

%   Case $\star = C$:
%   % Let
%   % $(b_{1},b_{2}) = (\frac{\bfrr_{2t-1}(\ckcO)-1}{2},\frac{\bfrr_{2t}(\ckcO)+1}{2})$.
%   We have
%   \[
%     \begin{split}
%       (\bfcc_{t}(\imath_{\PPm}), \bfcc_{t}(\jmath_{\PPm})) &=
%       (\bfcc_{t}(\jmath_{\wp})+1, \bfcc_{t}(\imath_{\wp})-1) \AND \\
%      %  &= (b_{2},b_{1}),  \\
%      % &= (b_{1}-1,b_{2}+1),\AND\\
%       % (\bfcc_{t}(\imath_{\PPm}), \bfcc_{t}(\jmath_{\PPm})) &= (b_{2},b_{1}),  \\
%       % (\bfcc_{t}(\imath_{\wp}), \bfcc_{t}(\jmath_{\wp})) &= (b_{1}-1,b_{2}+1),\AND\\
%       (\bfcc_{i}(\imath_{\PPm}),\bfcc_{i}(\jmath_{\PPm})) &=(\bfcc_{i}(\imath_{\wp}),\bfcc_{i}(\jmath_{\wp}))\quad \text{for $i\neq t$}.
%     \end{split}
%   \]
%   % Let
%   % $a = \half(\bfrr_{2t-1}(\ckcO)-\bfrr_{2t}(\ckcO))-1 = \bfcc_{t}(\jmath_{\PPm})-\bfcc_{t}(\imath_{\PPm})$.

%   \trivial[h]{
%     The idea of the definition of $T_{\PPm,\wp}$ is that we move ``$s$'' appeared
%     in the $t$-th column of $\cQ_{\uptaum}$ to the $t$-th column of
%     $\cP_{\uptau}$.
%   }

%   For $\uptaum\in \PBPs(\tau_{\PPm})$, we define $\uptau$ by the following algorithm:
%   \begin{description}
%     \item[STEP~1] Define a map
%           $\cP'\colon \BOX{\imath_{\wp}}\rightarrow \set{\bullet,r,c,d}$ (as a candidate for $\cP_{\uptau}$), by the following rules:
%           \begin{enumerate}[label=(\alph*)]
%             \item Suppose
%             $\cP_{\uptaum}(\bfcc_{t}(\imath_{\PPm}),t)\neq \bullet$.
%             \begin{itemize}
%               \item If $\bfcc_{t}(\imath_{\PPm})\geq 2$ and
%               $\cP_{\uptaum}(\bfcc_{t}(\imath_{\PPm})-1,t) = c$,
%               we define
%               \[
%                 \cP'(i,j) := \begin{cases}
%                   r ,& \text{if $j=t$ and $\bfcc_{t}(\imath_{\PPm})-1
%                     \leq i \leq \bfcc_{t}(\imath_{\wp})-2$},\\
%                   c ,& \text{if $(i,j)=(\bfcc_{t}(\imath_{\wp})-1,t)$},\\
%                   d ,& \text{if $(i,j)=(\bfcc_{t}(\imath_{\wp}),t)$},\\
%                   \cP_{\uptaum}(i,j) ,&\text{otherwise}.
%                 \end{cases}
%               \]
%               \item Otherwise, we define
%               \[
%                 \cP'(i,j) := \begin{cases}
%                   r ,& \text{if $j=t$ and $\bfcc_{t}(\imath_{\PPm})
%                     \leq i \leq \bfcc_{t}(\imath_{\wp})-1$},\\
%                   \cP_{\uptaum}(\bfcc_{t}(\imath_{\PPm}),t) ,&
%                   \text{if $(i,j)=(\bfcc_{t}(\imath_{\wp}),t)$},\\
%                   \cP_{\uptaum}(i,j) ,&\text{otherwise}.
%                 \end{cases}
%               \]
%             \end{itemize}
%             \item Suppose $\cP_{\uptaum}(\bfcc_{t}(\imath_{\PPm}),t)=\bullet$.
%             \begin{itemize}
%               \item If $\bfcc_{t+1}(\imath_{\PPm}) = \bfcc_{t}(\imath_{\PPm})$
%               and
%               $\cP_{\uptaum}(\bfcc_{t}(\imath_{\PPm}),t+1) = r$,
%               we define
%               \[
%                 \cP'(i,j) := \begin{cases}
%                   r ,& \text{if $j=t$ and $\bfcc_{t}(\imath_{\PPm})\leq i \leq \bfcc_{t}(\imath_{\wp})-1$},\\
%                   c ,& \text{if $(i,j)=(\bfcc_{t+1}(\imath_{\PPm}),t+1)$},\\
%                   d ,& \text{if $(i,j)=(\bfcc_{t}(\imath_{\wp}),t)$},\\
%                   \cP_{\uptaum}(i,j) ,&\text{otherwise}.
%                 \end{cases}
%               \]
%               \item Otherwise, we define
%               \[
%                 \cP'(i,j) := \begin{cases}
%                   r ,& \text{if $j=t$ and $\bfcc_{t}(\imath_{\PPm})\leq i \leq \bfcc_{t}(\imath_{\wp})-2$},\\
%                   c ,& \text{if $(i,j)=(\bfcc_{t}(\imath_{\wp})-1,t)$},\\
%                   d ,& \text{if $(i,j)=(\bfcc_{t}(\imath_{\wp}),t)$},\\
%                   \cP_{\uptaum}(i,j) ,&\text{otherwise}.
%                 \end{cases}
%               \]
%             \end{itemize}
%           \end{enumerate}

%     \item[STEP 2] From the construction of $\cP'$, there are four possibilities for $\cP'$ to violate the requirements of a painting on $\imath_{\wp}$, which are detailed as follows. We must have $t>1$ and violations occur in positions of $(i,j)\in \BOX{\imath_{\wp}}$ inside the following $2\times 2$ square
%          \[
%           A :=
%           \begin{pmatrix}
%             (\bfcc_{t}(\imath_{\wp})-1,t-1) & (\bfcc_{t}(\imath_{\wp})-1,t) \\
%             (\bfcc_{t}(\imath_{\wp})\;\phantom{-1}\;,t-1) & (\bfcc_{t}(\imath_{\wp})\;\phantom{-1}\;,t) \\
%           \end{pmatrix}
%           \]

%           %\[
%           %A :=
%           %\begin{pmatrix}
%           %  \cP'(\bfcc_{t}(\imath_{\wp})-1,t-1) & \cP'(\bfcc_{t}(\imath_{\wp})-1,t) \\
%           %  \cP'(\bfcc_{t}(\imath_{\wp})\;\phantom{-1}\;,t-1) & \cP'(\bfcc_{t}(\imath_{\wp})\;\phantom{-1}\;,t) \\
%           %\end{pmatrix}
%           %\]
%           \trivial[h]{
%           Note that  $\cP'(\bfcc_{t}(\imath_{\wp})-1,t-1)$ is always equal to
%           $\bullet$.
%           }

%     Let $\cP_{\uptau}\colon \BOX{\imath_{\wp}}\rightarrow \set{\bullet,r,c,d}$
%           be the painting on $\imath_{\wp}$ defined as follows:
%           \begin{itemize}
%             \item When $(i,j)\in \BOX{\imath_{\wp}}$ and
%             $\set{i-\bfcc_{t}(\imath_{\wp})+1, j-t+1}\nsubseteq \set{0,1}$
%             (i.e. $(i,j)$ is not one of the four boxes corresponding to
%             $A$), we define
%             \[
%               \cP_{\uptau}(i,j):= \cP'(i,j).
%             \]

%             \item For the four boxes corresponding to $A$, we modify the symbols by
%             setting
%             \begin{equation} \label{eq:modP}
%               \begin{split}
%               %  &\begin{pmatrix}
%               %    \cP_{\uptau}(\bfcc_{t}(\imath_{\wp})-1,t-1) & \cP_{\uptau}(\bfcc_{t}(\imath_{\wp})-1,t) \\
%               %   \cP_{\uptau}(\bfcc_{t}(\imath_{\wp})\;\phantom{-1}\;,t-1)
%               %    & \cP_{\uptau}(\bfcc_{t}(\imath_{\wp})\;\phantom{-1}\;,t) \\
%               % \end{pmatrix}\\
%               \cP_{\uptau}|_A  :=&
%                 \begin{cases}
%                   \begin{pmatrix}
%                     r & c\\
%                     r & d
%                   \end{pmatrix}, & \text{if } \cP'|_A =
%                   \begin{pmatrix}
%                     \bullet & r\\
%                     r & r
%                   \end{pmatrix},\\[1.5em]
%                   \begin{pmatrix}
%                     r & c\\
%                     c & d
%                   \end{pmatrix}, & \text{if } \cP'|_A =
%                   \begin{pmatrix}
%                     \bullet & r\\
%                     c & r
%                   \end{pmatrix},\\[1.5em]
%                   \begin{pmatrix}
%                     r & c\\
%                     d & d
%                   \end{pmatrix}, & \text{if } \cP'|_A =
%                   \begin{pmatrix}
%                     \bullet & r\\
%                     d & r
%                   \end{pmatrix},\\[1.5em]
%                   \begin{pmatrix}
%                     c & c\\
%                     d & d
%                   \end{pmatrix}, & \text{if } \cP'|_A =
%                   \begin{pmatrix}
%                     \bullet & r\\
%                     d & c
%                   \end{pmatrix}.\\
%                 \end{cases}
%               \end{split}
%             \end{equation}
%           \end{itemize}

%     \item[STEP 3] The painted bipartition $\uptau$ is uniquely determined by
%           $\cP_{\uptau}$. More precisely, $\cQ_{\uptau}$ is given by the following
%           formula: for $(i,j)\in \BOX{\jmath_{\wp}}$,
%           \[
%           \cQ_{\uptau}(i,j) :=
%           \begin{cases}
%             s, & \begin{minipage}{17em}if $\cP'$ is not a valid painting on $\imath_{\wp}$\\
%               and $(i,j)= (\bfcc_{t}(\imath_{\wp})-1,t-1)$,
%               \end{minipage}\\
%             \cQ_{\uptaum}(i,j), & \text{otherwise.}
%             \end{cases}
%           \]
%   \end{description}

%  It is not difficult to check that $\uptau$ is a valid painted bipartition
%  and to construct the inverse map $T_{\wp,\PPm}$ by reversing the above steps.

%  \trivial[h]{
%    The inverse map $T_{\wp,\PPm}$ is given by the following algorithm:
%    \begin{description}
%      \item[STEP 1] We first recover $\cP'$.
%            If $t=1$ or $\cP'(\bfcc_{t}(\imath_{\wp})-1,t-1)=\bullet$, then
%            $\cP':= \cP_{\wp}$.
%            Otherwise,
%            $\cP'$ is given by $\cP_{\wp}$ except the $2\times 2$ square in
%            \eqref{eq:modP} which is given by reversing the formula cited.
%      \item[STEP 2]

%            \def\xxn{\cP_{\uptaum}(\bfcc_t(\imath_{\PPm})-1,t)} %x_0
%            \def\xxo{\cP_{\uptaum}(\bfcc_t(\imath_{\wp}),t)} %x_1
%            \def\xxd{\cP_{\uptaum}(\bfcc_t(\imath_{\wp}),t+1)} %x_2
%            \def\yyn{\cP'(\bfcc_t(\imath_{\PPm})-1,t)} %y_0
%            \def\yyo{\cP'(\bfcc_t(\imath_{\wp})-1,t)} %y_1
%            \def\yyt{\cP'(\bfcc_t(\imath_{\wp}),t)} %y_3
%            \def\yyd{\cP'(\bfcc_t(\imath_{\wp}),t+1)} %y_2
%            We have the following cases:
%            \begin{enumerate}[label=(\alph*)]
%              \item Suppose $\yyo=r$.
%              \begin{itemize}
%                \item If $\bfcc_{t+1}(\imath_{\wp}) = \bfcc_{t}(\imath_{\PPm})$
%                and
%                \[
%                  (\yyd,\yyt) = (c,d),
%                \]
%                let
%                \[
%                  (\xxo,\xxd):=(\bullet, r)
%                \]
%                \item Otherwise, let \[
%                  \xxo:=\yyt.
%                \]
%              \end{itemize}
%              \item Suppose $\yyo=c$
%              \begin{itemize}
%                \item If $\bfcc_{t}(\imath_{\PPm})\geq 2$ and $\xxn=r$,
%                then let
%                \[
%                  (\xxn,\xxo):=(c,d).
%                \]
%                \item Otherwise, let
%                \[
%                  \xxo :=\bullet.
%                \]
%              \end{itemize}
%            \end{enumerate}
%            For the boxes $(i,j)$ in $\BOX{\imath_{\uptaum}}$ which are not specified
%            in the above procedure, set
%            \[
%            \cP_{\uptaum}(i,j):=\cP'(i,j).
%            \]
%      \item[STEP 3]
%            Now $\cP_{\uptaum}$ uniquely determine the painted bipartition
%            $\uptaum$.
%    \end{description}
%    The construction of the inverse map implies that $T_{\PPm,\wp}$ is a
%    bijection.
%  }

%   \def\ckcOa{\ckcO^{\uparrow}}
%   \def\PPa{\wp^{\uparrow}}
%   \def\PPam{\wp^{\uparrow}_{\downarrow}}
%   \def\uptaua{\uptau^{\uparrow}}
%   \def\tauPPa{\tau_{\PPa}}
%   \def\tauPPam{\tau_{\PPam}}
%   \def\stara{\star^{\uparrow}}

%   \medskip

%   {Now suppose $\star \in \set{B,D}$.}
%   We will prove the lemma by appealing to the corresponding assertion for the case of $\wtC/C$.
%   Let
%   \[
%     \stara := \begin{cases}
%       \wtC ,& \text{when $\star=B$},\\
%       C ,& \text{when $\star=D$},
%     \end{cases}
%   \]
%   and $\ckcOa$ be the partition defined by
%   \[
%     \bfrr_{1}(\ckcOa) = \bfrr_{1}(\ckcO)+2, \AND \bfrr_{i+1}(\ckcOa)
%     = \bfrr_{i}(\ckcO)\quad \text{for all $i=1,2,3,\cdots,$}.
%   \]
%   Clearly
%   \[
%     \CPP_{\stara }(\ckcOa) = \set{(1,2)}\cup \set{(i+1,i+2)|(i,i+1)\in \CPP_{\star}(\ckcO)}.
%   \]




%   % Let $\PPa$ denote the subset in
%   % $\CPP(\ckcOa)$ defined by
%   % \[
%   %   \PPa:=\set{(i+1,i+2)|(i,i+1)\in \sP}.
%   % \]
%   % Let $r_{0}:= \half \bfrr_{1}(\ckcO)+1 = \bfcc_{1}(\imath_{\PPa})$. For $x=c$
%   % or $s$, define
%   % \[
%   %   \PBP_{\wtC}^{x}(\tauPPa):= \Set{\uptaua\in \PBP_{\wtC}(\tau_{\PPa})|\cP_{\uptaua}\left(r_{0},1\right)=x}
%   % \]
%   % where $\tauPPa$ is the bipartition defined with respect to $\ckcOa$.

%   % Let $\PPam:=(\PPa)_{\downarrow}$. It is easy to check that
%   % \begin{itemize}
%   %   \item
%   %   $\PBP_{\wtC}(\tauPPa) = \PBP_{\wtC}^{c}(\tauPPa)\sqcup \PBP_{\wtC}^{s}(\tauPPa)$,
%   %   \item the map $T_{\PPa,\PPam}$ restricted into a bijection from
%   %   $\PBP_{\wtC}^{x}(\tauPPa)$ onto $\PBP_{\wtC}^{x}(\tauPPam)$ (here $x=s$ or
%   %   $c$), and
%   %   \item the descent map restricted to bijections
%   %   \[
%   %     \text{
%   %         $\PBP_{\wtC}^{s}(\tau_{\PPa})\longrightarrow \PBP_{B^{+}}(\tau_{\sP})$
%   %         and
%   %         $\PBP_{\wtC}^{c}(\tau_{\PPa})\longrightarrow \PBP_{B^{-}}(\tau_{\sP})$.
%   %     }
%   %   \]
%   % \end{itemize}

%   % Now the bijection $T_{\wp,\wpm}$ is defined by making the following diagram
%   % commutative
%   % \[
%   %   \begin{tikzcd}
%   %     \PBP_{\wtC}^{s}(\tau_{\PPa}) \ar[r] \ar[d] & \PBP_{B^{+}}^{s}(\tau_{\wp}) \ar[d]\\
%   %     \PBP_{\wtC}^{s}(\tau_{\PPam}) \ar[r] & \PBP_{B^{+}}^{s}(\tau_{\PPm}) \\
%   %   \end{tikzcd}
%   % \]

%  % For each subset $\sP\subset \CPP(\ckcO)$,
%   Let $\PPa$ denote the subset in
%   $\CPP_{\stara }(\ckcOa)$ defined by
%   \[
%     \PPa:=\set{(i+1,i+2)|(i,i+1)\in \sP}.
%   \]
%   % Let $r_{0}:= \half \bfrr_{1}(\ckcO)+1 = \bfcc_{1}(\imath_{\PPa})$.
%   % For $x=c$
%   % or $s$, define
%   % \[
%   %   \PBP_{\wtC}^{x}(\tauPPa):= \Set{\uptaua\in \PBP_{\wtC}(\tau_{\PPa})|\cP_{\uptaua}\left(r_{0},1\right)=x}
%   % \]
%   % where $\tauPPa$ is the bipartition defined with respect to $\ckcOa$.

% Let $\PPam:=(\PPa)_{\downarrow}$. Recall from \cite[Section 2.3]{BMSZ2} the (naive) descent map $\DD$ of painted bipartitions.
% It is easy to check that the descent maps (horizontal arrows) in the following diagram are bijections.
%  %(c.f. \cite{BMSZ1}*{Lemma???}). % restricted to bijections
%  % \[
%  %   \PBP_{\wtC}(\tau_{\PPa})\longrightarrow \PBP_{B}(\tau_{\sP}).
%  % \]
%   \[
%     \begin{tikzcd}
%       \PBP_{\wtC}(\tau_{\PPam}) \ar[r,"\DD",two heads,hook] \ar[d,two heads,hook,"T_{\PPam,\PPa}"']
%       & \PBP_{B}(\tau_{\PPm}) \ar[d,dashed,"T_{\PPm,\wp}"]\\
%       \PBP_{\wtC}(\tau_{\PPa}) \ar[r,"\DD",two heads,hook] & \PBP_{B}(\tau_{\wp}) \\
%     \end{tikzcd}
%   \]
%  Since there is a bijection $T_{\PPam,\PPa}$ in the left vertical arrow by case $\wtC$, we may define a bijection $T_{\PPm,\wp}$ in the right vertical arrow by making the above diagram commutative.
%  This completes the proof for the case $B$. The case $D$ is entirely similar.
% \end{proof}


% }


\begin{bibdiv}
  \begin{biblist}
% \bib{AB}{article}{
%   title={Genuine representations of the metaplectic group},
%   author={Adams, Jeffrey},
%   author = {Barbasch, Dan},
%   journal={Compositio Mathematica},
%   volume={113},
%   number={01},
%   pages={23--66},
%   year={1998},
% }

% \bib{Ad83}{article}{
%   author = {Adams, J.},
%   title = {Discrete spectrum of the reductive dual pair $(O(p,q),Sp(2m))$ },
%   journal = {Invent. Math.},
%   number = {3},
%  pages = {449--475},
%  volume = {74},
%  year = {1983}
% }

%\bib{Ad07}{article}{
%  author = {Adams, J.},
%  title = {The theta correspondence over R},
%  journal = {Harmonic analysis, group representations, automorphic forms and invariant theory,  Lect. Notes Ser. Inst. Math. Sci. Natl. Univ. Singap., 12},
% pages = {1--39},
% year = {2007}
% publisher={World Sci. Publ.}
%}


\bib{ABV}{book}{
  title={The Langlands classification and irreducible characters for real reductive groups},
  author={Adams, J.},
  author={Barbasch, D.},
  author={Vogan, D. A.},
  series={Progress in Math.},
  volume={104},
  year={1991},
  publisher={Birkhauser}
}

\bib{AC}{article}{
  title={Algorithms for representation theory of
    real reductive groups},
  volume={8},
  DOI={10.1017/S1474748008000352},
  number={2},
  journal={Journal of the Institute of Mathematics of Jussieu},
  publisher={Cambridge University Press},
  author={Adams, J.},
  author={du Cloux, F.},
  year={2009},
  pages={209-259}
}

\bib{ArPro}{article}{
author = {Arthur, J.},
title = {On some problems suggested by the trace formula},
journal = {Lie group representations, II (College Park, Md.), Lecture Notes in Math. 1041},
pages = {1--49},
year = {1984}
}


\bib{ArUni}{article}{
author = {Arthur, J.},
title = {Unipotent automorphic representations: conjectures},
booktitle = {Orbites unipotentes et repr\'esentations, II},
journal = {Orbites unipotentes et repr\'esentations, II, Ast\'erisque},
pages = {13--71},
volume = {171-172},
year = {1989}
}

% \bib{AK}{article}{
%   author = {Auslander, L.},
%   author = {Kostant, B.},
%   title = {Polarizations and unitary representations of solvable Lie groups},
%   journal = {Invent. Math.},
%  pages = {255--354},
%  volume = {14},
%  year = {1971}
% }


% \bib{B.Uni}{article}{
%   author = {Barbasch, D.},
%   title = {Unipotent representations for real reductive groups},
%  %booktitle = {Proceedings of ICM, Kyoto 1990},
%  journal = {Proceedings of ICM (1990), Kyoto},
%    % series = {Proc. Sympos. Pure Math.},
%  %   volume = {68},
%      pages = {769--777},
%  publisher = {Springer-Verlag, The Mathematical Society of Japan},
%       year = {2000},
% }


\bib{B89}{article}{
  author = {Barbasch, D.},
  title = {The unitary dual for complex classical Lie groups},
  journal = {Invent. Math.},
  volume = {96},
  number = {1},
 pages = {103--176},
 year = {1989}
}

\bib{B.Orbit}{article}{
  author = {Barbasch, D.},
  title = {Orbital integrals of nilpotent orbits},
 %booktitle = {The mathematical legacy of {H}arish-{C}handra ({B}altimore,{MD}, 1998)},
    journal = {The mathematical legacy of {H}arish-{C}handra, Proc. Sympos. Pure Math.},
    %series={The mathematical legacy of {H}arish-{C}handra, Proc. Sympos. Pure Math},
    volume = {68},
     pages = {97--110},
 publisher = {Amer. Math. Soc., Providence, RI},
      year = {2000},
}


\bib{B10}{article}{
  author = {Barbasch, D.},
  title = {The unitary spherical spectrum for split classical groups},
  journal = {J. Inst. Math. Jussieu},
% number = {9},
 pages = {265--356},
 volume = {9},
 year = {2010}
}



\bib{B17}{article}{
  author = {Barbasch, D.},
  title = {Unipotent representations and the dual pair correspondence},
  journal = {J. Cogdell et al. (eds.), Representation Theory, Number Theory, and Invariant Theory, In Honor of Roger Howe. Progress in Math.},
  %series ={Progress in Math.},
  volume = {323},
  pages = {47--85},
  year = {2017},
}


\bib{Bo}{article}{
   author={Bo\v{z}i\v{c}evi\'{c}, M.},
   title={Double cells for unitary groups},
   journal={J. Algebra},
   volume={254},
   date={2002},
   number={1},
   pages={115--124},
   issn={0021-8693},
   review={\MR{1927434}},
   doi={10.1016/S0021-8693(02)00070-4},
}



\bib{BMSZ1}{article}{
      title={On the notion of metaplectic Barbasch-Vogan duality},
      year={2020},
      author={Barbasch, D.},
      author = {Ma, J.-J.},
      author = {Sun, B.},
      author = {Zhu, C.-B.},
      eprint={2010.16089},
      archivePrefix={arXiv},
      primaryClass={math.RT}
}

\bib{BMSZ2}{article}{
      title={Special unipotent representations of real classical groups: construction and unitarity},
      author={Barbasch, D.},
      author = {Ma, J.-J.},
      author = {Sun, B.},
      author = {Zhu, C.-B.},
      year={2021},
      eprint={arXiv:1712.05552},
      archivePrefix={arXiv},
      primaryClass={math.RT}
}



\bib{BV1}{article}{
   author={Barbasch, D.},
   author={Vogan, D. A.},
   title={Primitive ideals and orbital integrals in complex classical
   groups},
   journal={Math. Ann.},
   volume={259},
   date={1982},
   number={2},
   pages={153--199},
   issn={0025-5831},
   review={\MR{656661}},
   doi={10.1007/BF01457308},
}

\bib{BV2}{article}{
   author={Barbasch, D.},
   author={Vogan, D. A.},
   title={Primitive ideals and orbital integrals in complex exceptional
   groups},
   journal={J. Algebra},
   volume={80},
   date={1983},
   number={2},
   pages={350--382},
   issn={0021-8693},
   review={\MR{691809}},
   doi={10.1016/0021-8693(83)90006-6},
}

\bib{BV.W}{article}{
  author={Barbasch, D.},
  author={Vogan, D. A.},
  editor={Trombi, P. C.},
  title={Weyl Group Representations and Nilpotent Orbits},
  bookTitle={Representation Theory of Reductive Groups:
    Proceedings of the University of Utah Conference 1982},
  year={1983},
  publisher={Birkh{\"a}user Boston},
  address={Boston, MA},
  pages={21--33},
  %doi={10.1007/978-1-4684-6730-7_2},
}


\bib{BVUni}{article}{
 author = {Barbasch, D.},
 author = {Vogan, D. A.},
 journal = {Annals of Math.},
 number = {1},
 pages = {41--110},
 title = {Unipotent representations of complex semisimple groups},
 volume = {121},
 year = {1985}
}

% \bib{BB}{article}{
%    author={Beilinson, Alexandre},
%    author={Bernstein, Joseph},
%    title={Localisation de $\mathfrak g$-modules},
%    language={French, with English summary},
%    journal={C. R. Acad. Sci. Paris S\'{e}r. I Math.},
%    volume={292},
%    date={1981},
%    number={1},
%    pages={15--18},
%    issn={0249-6291},
%    review={\MR{610137}},
% }

\bib{BGG.M}{article}{
   author={Bernstein, I. N.},
   author={Gel'fand, I. M.},
   author={Gel'fand, S. I.},
   title={Models of representations of compact Lie groups},
   language={Russian},
   journal={Funkcional. Anal. i Prilo\v{z}en.},
   volume={9},
   date={1975},
   number={4},
   pages={61--62},
   %issn={0374-1990},
   %review={\MR{0414792}},
}

\bib{Bor}{article}{
 author = {Borho, W.},
 journal = {S\'eminaire Bourbaki, Exp. No. 489},
 pages = {1--18},
 title = {Recent advances in enveloping algebras of semisimple Lie-algebras},
 year = {1976/77}
}

\bib{BK}{article}{
author={Borho, W.},
author={Kraft, H.},
title={\"{U}ber die Gelfand-Kirillov-Dimension},
journal={Math. Ann.},
volume={220},
date={1976},
number={1},
pages={1--24},
issn={0025-5831},
review={\MR{412240}},
doi={10.1007/BF01354525},
}


% \bib{Br}{article}{
%   author = {Brylinski, R.},
%   title = {Dixmier algebras for classical complex nilpotent orbits via Kraft-Procesi models. I},
%   journal = {The orbit method in geometry and physics (Marseille, 2000). Progress in Math.}
%   volume = {213},
%   pages = {49--67},
%   year = {2003},
% }

\bib{BK}{article}{
   author={Brylinski, J.-L.},
   author={Kashiwara, M.},
   title={Kazhdan-Lusztig conjecture and holonomic systems},
   journal={Invent. Math.},
   volume={64},
   date={1981},
   number={3},
   pages={387--410},
   issn={0020-9910},
   review={\MR{632980}},
   doi={10.1007/BF01389272},
}

\bib{Carter}{book}{
   author={Carter, R. W.},
   title={Finite groups of Lie type},
   series={Wiley Classics Library},
   %note={Conjugacy classes and complex characters;
   %Reprint of the 1985 original;
   %A Wiley-Interscience Publication},
   publisher={John Wiley \& Sons, Ltd., Chichester},
   date={1993},
   pages={xii+544},
   isbn={0-471-94109-3},
   %review={\MR{1266626}},
}

\bib{Cas}{article}{
   author={Casian, L. G.},
   title={Primitive ideals and representations},
   journal={J. Algebra},
   volume={101},
   date={1986},
   number={2},
   pages={497--515},
   issn={0021-8693},
   review={\MR{847174}},
   doi={10.1016/0021-8693(86)90208-5},
}

% \bib{Ca89}{article}{
%  author = {Casselman, W.},
%  journal = {Canad. J. Math.},
%  pages = {385--438},
%  title = {Canonical extensions of Harish-Chandra modules to representations of $G$},
%  volume = {41},
%  year = {1989}
% }



% \bib{Cl}{article}{
%   author = {Du Cloux, F.},
%   journal = {Ann. Sci. \'Ecole Norm. Sup.},
%   number = {3},
%   pages = {257--318},
%   title = {Sur les repr\'esentations diff\'erentiables des groupes de Lie alg\'ebriques},
%   url = {http://eudml.org/doc/82297},
%   volume = {24},
%   year = {1991},
% }

\bib{CM}{book}{
  title = {Nilpotent orbits in semisimple Lie algebra: an introduction},
  author = {Collingwood, D. H.},
  author = {McGovern, W. M.},
  year = {1993},
  publisher = {Van Nostrand Reinhold Co.},
}


% \bib{Dieu}{book}{
%    title={La g\'{e}om\'{e}trie des groupes classiques},
%    author={Dieudonn\'{e}, Jean},
%    year={1963},
% 	publisher={Springer},
%  }

% \bib{DKPC}{article}{
% title = {Nilpotent orbits and complex dual pairs},
% journal = {J. Algebra},
% volume = {190},
% number = {2},
% pages = {518 - 539},
% year = {1997},
% author = {Daszkiewicz, A.},
% author = {Kra\'skiewicz, W.},
% author = {Przebinda, T.},
% }

% \bib{DKP2}{article}{
%   author = {Daszkiewicz, A.},
%   author = {Kra\'skiewicz, W.},
%   author = {Przebinda, T.},
%   title = {Dual pairs and Kostant-Sekiguchi correspondence. II. Classification
% 	of nilpotent elements},
%   journal = {Central European J. Math.},
%   year = {2005},
%   volume = {3},
%   pages = {430--474},
% }


\bib{DM}{article}{
  author = {Dixmier, J.},
  author = {Malliavin, P.},
  title = {Factorisations de fonctions et de vecteurs ind\'efiniment diff\'erentiables},
  journal = {Bull. Sci. Math. (2)},
  year = {1978},
  volume = {102},
  pages = {307--330},
}

%\bibitem[DM]{DM}
%J. Dixmier and P. Malliavin, \textit{Factorisations de fonctions et de vecteurs ind\'efiniment diff\'erentiables}, Bull. Sci. Math. (2), 102 (4),  307-330 (1978).



\bib{Du77}{article}{
  author = {Duflo, M.},
  journal = {Annals of Math.},
  number = {1},
  pages = {107-120},
  title = {Sur la Classification des Ideaux Primitifs Dans
    L'algebre Enveloppante d'une Algebre de Lie Semi-Simple},
  volume = {105},
  year = {1977}
}

% \bib{Du82}{article}{
%  author = {Duflo, M.},
%  journal = {Acta Math.},
%   volume = {149},
%  number = {3-4},
%  pages = {153--213},
%  title = {Th\'eorie de Mackey pour les groupes de Lie alg\'ebriques},
%  year = {1982}
% }



% \bib{GZ}{article}{
% author={Gomez, R.},
% author={Zhu, C.-B.},
% title={Local theta lifting of generalized Whittaker models associated to nilpotent orbits},
% journal={Geom. Funct. Anal.},
% year={2014},
% volume={24},
% number={3},
% pages={796--853},
% }

% \bib{EGAIV2}{article}{
%   title = {\'El\'ements de g\'eom\'etrie alg\'brique IV: \'Etude locale des
%     sch\'emas et des morphismes de sch\'emas. II},
%   author = {Grothendieck, A.},
%   author = {Dieudonn\'e, J.},
%   journal  = {Inst. Hautes \'Etudes Sci. Publ. Math.},
%   volume = {24},
%   year = {1965},
% }


% \bib{EGAIV3}{article}{
%   title = {\'El\'ements de g\'eom\'etrie alg\'brique IV: \'Etude locale des
%     sch\'emas et des morphismes de sch\'emas. III},
%   author = {Grothendieck, A.},
%   author = {Dieudonn\'e, J.},
%   journal  = {Inst. Hautes \'Etudes Sci. Publ. Math.},
%   volume = {28},
%   year = {1966},
% }

\bib{GI}{article}{
   author={Gan, W. T.},
   author={Ichino, A.},
   title={On the irreducibility of some induced representations of real
   reductive Lie groups},
   journal={Tunis. J. Math.},
   volume={1},
   date={2019},
   number={1},
   pages={73--107},
   issn={2576-7658},
   review={\MR{3907735}},
   doi={10.2140/tunis.2019.1.73},
}

\bib{GW}{book}{
   author={Goodman, R.},
   author={Wallach, N. R.},
   title={Symmetry, representations, and invariants},
   series={Graduate Texts in Mathematics},
   volume={255},
   publisher={Springer, Dordrecht},
   date={2009},
   pages={xx+716},
   isbn={978-0-387-79851-6},
   %review={\MR{2522486}},
   %doi={10.1007/978-0-387-79852-3},
}
% \bib{HLS}{article}{
%     author = {Harris, M.},
%     author = {Li, J.-S.},
%     author = {Sun, B.},
%      title = {Theta correspondences for close unitary groups},
%  %booktitle = {Arithmetic Geometry and Automorphic Forms},
%     %series = {Adv. Lect. Math. (ALM)},
%     journal = {Arithmetic Geometry and Automorphic Forms, Adv. Lect. Math. (ALM)},
%     volume = {19},
%      pages = {265--307},
%  publisher = {Int. Press, Somerville, MA},
%       year = {2011},
% }

% \bib{HS}{book}{
%  author = {Hartshorne, R.},
%  title = {Algebraic Geometry},
% publisher={Graduate Texts in Mathematics, 52. New York-Heidelberg-Berlin: Springer-Verlag},
% year={1983},
% }

% \bib{He}{article}{
% author={He, H.},
% title={Unipotent representations and quantum induction},
% journal={arXiv:math/0210372},
% year = {2002},
% }



\bib{Ho}{article}{
author={Hotta, R.},
title={On Joseph's construction of Weyl group representations},
journal={Tohoku Math. J.},
volume={36},
%number = {3},
pages={49--74 },
year={1984},
}




% \bib{Howe79}{article}{
%   title={$\Phi$-series and invariant theory},
%   author={Howe, R.},
%   book = {
%     title={Automorphic Forms, Representations and $L$-functions},
%     series={Proc. Sympos. Pure Math},
%     volume={33},
%     year={1979},
%   },
%   pages={275-285},
% }

% \bib{HoweRank}{article}{
% author={Howe, R.},
% title={On a notion of rank for unitary representations of the classical groups},
% journal={Harmonic analysis and group representations, Liguori, Naples},
% pages={223-331},
% year={1982},
% }

% \bib{Howe89}{article}{
% author={Howe, R.},
% title={Transcending classical invariant theory},
% journal={J. Amer. Math. Soc.},
% volume={2},
% pages={535--552},
% year={1989},
% }

% \bib{Howe95}{article}{,
%   author = {Howe, R.},
%   title = {Perspectives on invariant theory: Schur duality, multiplicity-free actions and beyond},
%   journal = {Piatetski-Shapiro, I. et al. (eds.), The Schur lectures (1992). Ramat-Gan: Bar-Ilan University, Isr. Math. Conf. Proc. 8,},
%   year = {1995},
%   pages = {1-182},
% }



% \bib{HL}{article}{
% author={Huang, J.-S.},
% author={Li, J.-S.},
% title={Unipotent representations attached to spherical nilpotent orbits},
% journal={Amer. J. Math.},
% volume={121},
% number = {3},
% pages={497--517},
% year={1999},
% }


% \bib{HZ}{article}{
% author={Huang, J.-S.},
% author={Zhu, C.-B.},
% title={On certain small representations of indefinite orthogonal groups},
% journal={Represent. Theory},
% volume={1},
% pages={190--206},
% year={1997},
% }


 \bib{H}{book}{
   author={Humphreys, J. E.},
    title={Representations of semisimple Lie algebras in the BGG category
    $\scr{O}$},
    series={Graduate Studies in Mathematics},
    volume={94},
   publisher={American Mathematical Society, Providence, RI},
   date={2008},
   pages={xvi+289},
  isbn={978-0-8218-4678-0},
    review={\MR{2428237}},
    doi={10.1090/gsm/094},
}



\bib{Jan}{book}{
   author={Jantzen, J. C.},
   title={Moduln mit einem h\"{o}chsten Gewicht},
   series={Lecture notes in Mathematics},
   volume={750},
   publisher={Springer-Verlag, Berlin/Heidelberg/New York},
   date={1979},
  % pages={xvi+289},
   % isbn={978-0-8218-4678-0},
   % review={\MR{2428237}},
   % doi={10.1090/gsm/094},
}




\bib{J1}{article}{
   author={Joseph, A.},
   title={Goldie rank in the enveloping algebra of a semisimple Lie algebra. I},
   journal={J. Algebra},
   volume={65},
   date={1980},
   number={2},
   pages={269--283},
   issn={0021-8693},
   review={\MR{585721}},
   doi={10.1016/0021-8693(80)90217-3},
}

\bib{J2}{article}{
   author={Joseph, A.},
   title={Goldie rank in the enveloping algebra of a semisimple Lie algebra. II},
   journal={J. Algebra},
   volume={65},
   date={1980},
   number={2},
   pages={284--306},
   issn={0021-8693},
   review={\MR{585721}},
   doi={10.1016/0021-8693(80)90217-3},
}

% \bib{J3}{article}{
%    author={Joseph, A.},
%    title={Goldie rank in the enveloping algebra of a semisimple Lie algebra.
%    III},
%    journal={J. Algebra},
%    volume={73},
%    date={1981},
%    number={2},
%    pages={295--326},
%    issn={0021-8693},
%    review={\MR{640039}},
%    doi={10.1016/0021-8693(81)90324-0},
% }

\bib{J.hw}{article}{
   author={Joseph, A.},
   title={On the variety of a highest weight module},
   journal={J. Algebra},
   volume={88},
   date={1984},
   number={1},
   pages={238--278},
   issn={0021-8693},
   review={\MR{741942}},
   doi={10.1016/0021-8693(84)90100-5},
}


\bib{J.av}{article}{
   author={Joseph, A.},
   title={On the associated variety of a primitive ideal},
   journal={J. Algebra},
   volume={93},
   date={1985},
   number={2},
   pages={509--523},
   issn={0021-8693},
   review={\MR{786766}},
   doi={10.1016/0021-8693(85)90172-3},
}

% \bib{J.ann}{article}{
%    author={Joseph, Anthony},
%    title={Annihilators and associated varieties of unitary highest weight
%    modules},
%    journal={Ann. Sci. \'{E}cole Norm. Sup. (4)},
%    volume={25},
%    date={1992},
%    number={1},
%    pages={1--45},
%    issn={0012-9593},
%    review={\MR{1152612}},
% }


% \bib{JLS}{article}{
% author={Jiang, D.},
% author={Liu, B.},
% author={Savin, G.},
% title={Raising nilpotent orbits in wave-front sets},
% journal={Represent. Theory},
% volume={20},
% pages={419--450},
% year={2016},
% }

\bib{King}{article}{
author={King, D. R.},
title={The character polynomial of the annihilator of an irreducible Harish-Chandra module},
journal={Amer. J. Math.},
volume={103},
%issue ={4},
pages={1195--1240},
year={1981},
}


\bib{Ki62}{article}{
author={Kirillov, A. A.},
title={Unitary representations of nilpotent Lie groups},
journal={Uspehi Mat. Nauk},
volume={17},
issue ={4},
pages={57--110},
year={1962},
}

\bib{Ko70}{article}{
author={Kostant, B.},
title={Quantization and unitary representations},
journal={Lectures in Modern Analysis and Applications III, Lecture Notes in Math.},
volume={170},
pages={87--208},
year={1970},
}


% \bib{KP}{article}{
% author={Kraft, H.},
% author={Procesi, C.},
% title={On the geometry of conjugacy classes in classical groups},
% journal={Comment. Math. Helv.},
% volume={57},
% pages={539--602},
% year={1982},
% }

% \bib{KR}{article}{
% author={Kudla, S. S.},
% author={Rallis, S.},
% title={Degenerate principal series and invariant distributions},
% journal={Israel J. Math.},
% volume={69},
% pages={25--45},
% year={1990},
% }


% \bib{Ku}{article}{
% author={Kudla, S. S.},
% title={Some extensions of the Siegel-Weil formula},
% journal={In: Gan W., Kudla S., Tschinkel Y. (eds) Eisenstein Series and Applications. Progress in Mathematics, vol 258. Birkh\"auser Boston},
% %volume={69},
% pages={205--237},
% year={2008},
% }





% \bib{LZ1}{article}{
% author={Lee, S. T.},
% author={Zhu, C.-B.},
% title={Degenerate principal series and local theta correspondence II},
% journal={Israel J. Math.},
% volume={100},
% pages={29--59},
% year={1997},
% }

% \bib{LZ2}{article}{
% author={Lee, S. T.},
% author={Zhu, C.-B.},
% title={Degenerate principal series of metaplectic groups and Howe correspondence},
% journal = {D. Prasad at al. (eds.), Automorphic Representations and L-Functions, Tata Institute of Fundamental Research, India,},
% year = {2013},
% pages = {379--408},
% }

% \bib{Li89}{article}{
% author={Li, J.-S.},
% title={Singular unitary representations of classical groups},
% journal={Invent. Math.},
% volume={97},
% number = {2},
% pages={237--255},
% year={1989},
% }

% \bib{LiuAG}{book}{
%   title={Algebraic Geometry and Arithmetic Curves},
%   author = {Liu, Q.},
%   year = {2006},
%   publisher={Oxford University Press},
% }

% \bib{LM}{article}{
%    author = {Loke, H. Y.},
%    author = {Ma, J.},
%     title = {Invariants and $K$-spectrums of local theta lifts},
%     journal = {Compositio Math.},
%     volume = {151},
%     issue = {01},
%     year = {2015},
%     pages ={179--206},
% }

% \bib{DL}{article}{
%    author={Deligne, P.},
%    author={Lusztig, G.},
%    title={Representations of reductive groups over finite fields},
%    journal={Ann. of Math. (2)},
%    volume={103},
%    date={1976},
%    number={1},
%    pages={103--161},
%    issn={0003-486X},
%    review={\MR{393266}},
%    doi={10.2307/1971021},
% }

% \bib{KL}{article}{
%    author={Kazhdan, David},
%    author={Lusztig, George},
%    title={Representations of Coxeter groups and Hecke algebras},
%    journal={Invent. Math.},
%    volume={53},
%    date={1979},
%    number={2},
%    pages={165--184},
%    issn={0020-9910},
%    review={\MR{560412}},
%    doi={10.1007/BF01390031},
% }

\bib{Lu}{book}{
   author={Lusztig, G.},
   title={Characters of reductive groups over a finite field},
   series={Annals of Mathematics Studies},
   volume={107},
   publisher={Princeton University Press, Princeton, NJ},
   date={1984},
   pages={xxi+384},
   isbn={0-691-08350-9},
   isbn={0-691-08351-7},
   review={\MR{742472}},
   doi={10.1515/9781400881772},
}


% \bib{Lu.I}{article}{
%    author={Lusztig, G.},
%    title={Intersection cohomology complexes on a reductive group},
%    journal={Invent. Math.},
%    volume={75},
%    date={1984},
%    number={2},
%    pages={205--272},
%    issn={0020-9910},
%    review={\MR{732546}},
%    doi={10.1007/BF01388564},
% }


% \bib{LS}{article}{
%    author = {Lusztig, G.},
%    author = {Spaltenstein, N.},
%     title = {Induced unipotent classes},
%     journal = {J. London Math. Soc.},
%     volume = {19},
%     year = {1979},
%     pages ={41--52},
% }


% \bib{Ma}{article}{
%    author = {Mackey, G. W.},
%     title = {Unitary representations of group extentions},
%     journal = {Acta Math.},
%     volume = {99},
%     year = {1958},
%     pages ={265--311},
% }

\bib{Mat96}{article}{
   author={Matumoto, H.},
   title={On the representations of ${\rm U}(m,n)$ unitarily induced from
   derived functor modules},
   journal={Compos. Math.},
   volume={100},
   date={1996},
   number={1},
   pages={1--39},
   issn={0010-437X},
   review={\MR{1377407}},
}

\bib{Mat}{article}{
   author={Matumoto, H.},
   title={On the representations of ${\rm Sp}(p,q)$ and ${\rm SO}^*(2n)$
   unitarily induced from derived functor modules},
   journal={Compos. Math.},
   volume={140},
   date={2004},
   number={4},
   pages={1059--1096},
   issn={0010-437X},
   review={\MR{2059231}},
   doi={10.1112/S0010437X03000629},
}

\bib{Mc}{article}{
   author = {McGovern, W. M.},
    title = {Cells of Harish-Chandra modules for real classical groups},
    journal = {Amer. J.  of Math.},
    volume = {120},
    issue = {01},
    year = {1998},
    pages ={211--228},
}


\bib{Mil}{article}{
   author = {Mili\v{c}i\'c, D.},
    title = {Localizations and representation theory of reductive Lie groups},
    journal = {preprint, http://www.math.utah.edu/~milicic/Eprints/book.pdf},
   % volume = {120},
    %issue = {01},
    %year = {1998},
   % pages ={211--228},
}


% \bib{Mo96}{article}{
%  author={M{\oe}glin, C.},
%     title = {Front d'onde des repr\'esentations des groupes classiques $p$-adiques},
%     journal = {Amer. J. Math.},
%     volume = {118},
%     issue = {06},
%     year = {1996},
%     pages ={1313--1346},
% }

\bib{Mo17}{article}{
  author={M{\oe}glin, C.},
  title = {Paquets d'Arthur Sp\'eciaux Unipotents aux Places Archim\'ediennes et Correspondance de Howe},
  journal = {J. Cogdell et al. (eds.), Representation Theory, Number Theory, and Invariant Theory, In Honor of Roger Howe. Progress in Math.}
  %series ={Progress in Math.},
  volume = {323},
  pages = {469--502}
  year = {2017}
}

\bib{MR.C}{article}{
   author={M{\oe}glin, C.},
   author={Renard, D.},
   title={Paquets d'Arthur des groupes classiques complexes},
   language={French, with English and French summaries},
   conference={
      title={Around Langlands correspondences},
   },
   book={
      series={Contemp. Math.},
      volume={691},
      publisher={Amer. Math. Soc., Providence, RI},
   },
   date={2017},
   pages={203--256},
   review={\MR{3666056}},
   doi={10.1090/conm/691/13899},
}

\bib{MR.U}{article}{
   author={M{\oe}glin, C.},
   author={Renard, D.},
   title={Sur les paquets d'Arthur des groupes unitaires et quelques
   cons\'{e}quences pour les groupes classiques},
   language={French, with English and French summaries},
   journal={Pacific J. Math.},
   volume={299},
   date={2019},
   number={1},
   pages={53--88},
   issn={0030-8730},
   review={\MR{3947270}},
   doi={10.2140/pjm.2019.299.53},
}


% \bib{MVW}{book}{
%   volume={1291},
%   title={Correspondances de Howe sur un corps $p$-adique},
%   author={M{\oe}glin, C.},
%   author={Vign\'eras, M.-F.},
%   author={Waldspurger, J.-L.},
%   series={Lecture Notes in Mathematics},
%   publisher={Springer}
%   ISBN={978-3-540-18699-1},
%   date={1987},
% }

% \bib{NOTYK}{article}{
%    author = {Nishiyama, K.},
%    author = {Ochiai, H.},
%    author = {Taniguchi, K.},
%    author = {Yamashita, H.},
%    author = {Kato, S.},
%     title = {Nilpotent orbits, associated cycles and Whittaker models for highest weight representations},
%     journal = {Ast\'erisque},
%     volume = {273},
%     year = {2001},
%    pages ={1--163},
% }

% \bib{NOZ}{article}{
%   author = {Nishiyama, K.},
%   author = {Ochiai, H.},
%   author = {Zhu, C.-B.},
%   journal = {Trans. Amer. Math. Soc.},
%   title = {Theta lifting of nilpotent orbits for symmetric pairs},
%   volume = {358},
%   year = {2006},
%   pages = {2713--2734},
% }


% \bib{NZ}{article}{
%    author = {Nishiyama, K.},
%    author = {Zhu, C.-B.},
%     title = {Theta lifting of unitary lowest weight modules and their associated cycles},
%     journal = {Duke Math. J.},
%     volume = {125},
%     number= {03},
%     year = {2004},
%    pages ={415--465},
% }



% \bib{Ohta}{article}{
%   author = {Ohta, T.},
%   %doi = {10.2748/tmj/1178227492},
%   journal = {Tohoku Math. J.},
%   number = {2},
%   pages = {161--211},
%   publisher = {Tohoku University, Mathematical Institute},
%   title = {The closures of nilpotent orbits in the classical symmetric
%     pairs and their singularities},
%   volume = {43},
%   year = {1991}
% }

% \bib{Ohta2}{article}{
%   author = {Ohta, T.},
%   journal = {Hiroshima Math. J.},
%   number = {2},
%   pages = {347--360},
%   title = {Induction of nilpotent orbits for real reductive groups and associated varieties of standard representations},
%   volume = {29},
%   year = {1999}
% }

% \bib{Ohta4}{article}{
%   title={Nilpotent orbits of $\mathbb{Z}_4$-graded Lie algebra and geometry of
%     moment maps associated to the dual pair $(\mathrm{U} (p, q), \mathrm{U} (r, s))$},
%   author={Ohta, T.},
%   journal={Publ. RIMS},
%   volume={41},
%   number={3},
%   pages={723--756},
%   year={2005}
% }

% \bib{PT}{article}{
%   title={Some small unipotent representations of indefinite orthogonal groups and the theta correspondence},
%   author={Paul, A.},
%   author={Trapa, P.},
%   journal={University of Aarhus Publ. Series},
%   volume={48},
%   pages={103--125},
%   year={2007}
% }


% \bib{PV}{article}{
%   title={Invariant Theory},
%   author={Popov, V. L.},
%   author={Vinberg, E. B.},
%   book={
%   title={Algebraic Geometry IV: Linear Algebraic Groups, Invariant Theory},
%   series={Encyclopedia of Mathematical Sciences},
%   volume={55},
%   year={1994},
%   publisher={Springer},}
% }




%\bib{PPz}{article}{
%author={Protsak, V.} ,
%author={Przebinda, T.},
%title={On the occurrence of admissible representations in the real Howe
%    correspondence in stable range},
%journal={Manuscr. Math.},
%volume={126},
%number={2},
%pages={135--141},
%year={2008}
%}


% \bib{PrzInf}{article}{
%       author={Przebinda, T.},
%        title={The duality correspondence of infinitesimal characters},
%         date={1996},
%      journal={Colloq. Math.},
%       volume={70},
%        pages={93--102},
% }


% \bib{Pz}{article}{
% author={Przebinda, T.},
% title={Characters, dual pairs, and unitary representations},
% journal={Duke Math. J. },
% volume={69},
% number={3},
% pages={547--592},
% year={1993}
% }

% \bib{Ra}{article}{
% author={Rallis, S.},
% title={On the Howe duality conjecture},
% journal={Compositio Math.},
% volume={51},
% pages={333--399},
% year={1984}
% }

\bib{RT1}{article}{
   author={Renard, D.},
   author={Trapa, P.},
   title={Irreducible genuine characters of the metaplectic group:
   Kazhdan-Lusztig algorithm and Vogan duality},
   journal={Represent. Theory},
   volume={4},
   date={2000},
   pages={245--295},
   review={\MR{1795754}},
   doi={10.1090/S1088-4165-00-00105-9},
}

\bib{RT2}{article}{
   author={Renard, D.},
   author={Trapa, P.},
   title={Irreducible characters of the metaplectic group. II.
   Functoriality},
   journal={J. Reine Angew. Math.},
   volume={557},
   date={2003},
   pages={121--158},
   issn={0075-4102},
   review={\MR{1978405}},
   doi={10.1515/crll.2003.028},
}

% \bib{RT3}{article}{
%    author={Renard, David A.},
%    author={Trapa, Peter E.},
%    title={Kazhdan-Lusztig algorithms for nonlinear groups and applications
%    to Kazhdan-Patterson lifting},
%    journal={Amer. J. Math.},
%    volume={127},
%    date={2005},
%    number={5},
%    pages={911--971},
%    issn={0002-9327},
%    review={\MR{2170136}},
% }


% \bib{Sa}{article}{
% author={Sahi, S.},
% title={Explicit Hilbert spaces for certain unipotent representations},
% journal={Invent. Math.},
% volume={110},
% number = {2},
% pages={409--418},
% year={1992}
% }

% \bib{Se}{article}{
% author={Sekiguchi, J.},
% title={Remarks on real nilpotent orbits of a symmetric pair},
% journal={J. Math. Soc. Japan},
% %publisher={The Mathematical Society of Japan},
% year={1987},
% volume={39},
% number={1},
% pages={127--138},
% }

\bib{SV}{article}{
author = {Schmid, W.},
author = {Vilonen, K.},
journal = {Annals of Math.},
number = {3},
pages = {1071--1118},
%publisher = {Princeton University, Mathematics Department, Princeton, NJ; Mathematical Sciences Publishers, Berkeley},
title = {Characteristic cycles and wave front cycles of representations of reductive Lie groups},
volume = {151},
year = {2000},
}


\bib{Soergel}{article}{
   author={Soergel, W.},
   title={Kategorie $\scr O$, perverse Garben und Moduln \"{u}ber den
   Koinvarianten zur Weylgruppe},
   language={German, with English summary},
   journal={J. Amer. Math. Soc.},
   volume={3},
   date={1990},
   number={2},
   pages={421--445},
   issn={0894-0347},
   review={\MR{1029692}},
   doi={10.2307/1990960},
}


\bib{So}{article}{
author = {Sommers, E.},
title = {Lusztig's canonical quotient and generalized duality},
journal = {J. Algebra},
volume = {243},
number = {2},
pages = {790--812},
year = {2001},
}

% \bib{SS}{book}{
%   author = {Springer, T. A.},
%   author = {Steinberg, R.},
%   title = {Seminar on algebraic groups and related finite groups; Conjugate classes},
%   series = {Lecture Notes in Math.},
%   volume = {131},
% publisher={Springer},
% year={1970},
% }

% \bib{SZ1}{article}{
% title={A general form of Gelfand-Kazhdan criterion},
% author={Sun, B.},
% author={Zhu, C.-B.},
% journal={Manuscripta Math.},
% pages = {185--197},
% volume = {136},
% year={2011}
% }


%\bib{SZ2}{article}{
%  title={Conservation relations for local theta correspondence},
%  author={Sun, B.},
%  author={Zhu, C.-B.},
%  journal={J. Amer. Math. Soc.},
%  pages = {939--983},
%  volume = {28},
%  year={2015}
%}

\bib{Tr.U}{article}{
   author={Trapa, P.},
   title={Annihilators and associated varieties of $A_{\mathfrak q}(\lambda)$
   modules for $\mathrm U(p,q)$},
   journal={Compositio Math.},
   volume={129},
   date={2001},
   number={1},
   pages={1--45},
   issn={0010-437X},
   review={\MR{1856021}},
   doi={10.1023/A:1013115223377},
}

\bib{Tr.H}{article}{
  title={Special unipotent representations and the Howe correspondence},
  author={Trapa, P.},
  year = {2004},
  journal={University of Aarhus Publication Series},
  volume = {47},
  pages= {210--230}
}

% \bib{Wa}{article}{
%    author = {Waldspurger, J.-L.},
%     title = {D\'{e}monstration d'une conjecture de dualit\'{e} de Howe dans le cas $p$-adique, $p \neq 2$ in Festschrift in honor of I. I. Piatetski-Shapiro on the occasion of his sixtieth birthday},
%   journal = {Israel Math. Conf. Proc., 2, Weizmann, Jerusalem},
%  year = {1990},
% pages = {267-324},
% }

\bib{VGK}{article}{
   author={Vogan, D. A.},
   title={Gel\cprime fand-Kirillov dimension for Harish-Chandra modules},
   journal={Invent. Math.},
   volume={48},
   date={1978},
   number={1},
   pages={75--98},
   issn={0020-9910},
   review={\MR{506503}},
   doi={10.1007/BF01390063},
}



\bib{Vg}{book}{
   author={Vogan, D. A.},
   title={Representations of real reductive Lie groups},
   series={Progress in Mathematics},
   volume={15},
   publisher={Birkh\"{a}user, Boston, Mass.},
   date={1981},
   pages={xvii+754},
   isbn={3-7643-3037-6},
   review={\MR{632407}},
}

\bib{V1}{article}{
   author={Vogan, D. A.},
   title={Irreducible characters of semisimple Lie groups. I},
   journal={Duke Math. J.},
   volume={46},
   date={1979},
   number={1},
   pages={61--108},
   issn={0012-7094},
   review={\MR{523602}},
}

\bib{V3}{article}{
   author={Vogan, D. A.},
   title={Irreducible characters of semisimple Lie groups. III. Proof of Kazhdan-Lusztig conjecture in the integral case},
   journal={Invent. Math.},
   volume={71},
   date={1983},
   number={2},
   pages={381--417},
}

\bib{V4}{article}{
   author={Vogan, D. A.},
   title={Irreducible characters of semisimple Lie groups. IV.
   Character-multiplicity duality},
   journal={Duke Math. J.},
   volume={49},
   date={1982},
   number={4},
   pages={943--1073},
   issn={0012-7094},
   review={\MR{683010}},
}



\bib{V.GL}{article}{
   author={Vogan, D. A.},
   title={The unitary dual of ${\rm GL}(n)$ over an Archimedean field},
   journal={Invent. Math.},
   volume={83},
   date={1986},
   number={3},
   pages={449--505},
   issn={0020-9910},
   review={\MR{827363}},
   doi={10.1007/BF01394418},
}

\bib{VoICM}{article}{
author={Vogan, D. A.},
   title={Representations of reductive Lie groups},
   journal={Proceedings of the International Congress of Mathematicians (Berkeley, Calif., 1986)},
   publisher = {Amer. Math. Soc.},
  year = {1987},
pages={246--266},
}

\bib{VoBook}{book}{
author = {Vogan, D. A.},
  title={Unitary representations of reductive Lie groups},
  year={1987},
  series = {Ann. of Math. Stud.},
 volume={118},
  publisher={Princeton University Press}
}


\bib{Vo89}{article}{
  author = {Vogan, D. A.},
  title = {Associated varieties and unipotent representations},
 %booktitle ={Harmonic analysis on reductive groups, Proc. Conf., Brunswick/ME (USA) 1989,},
  journal = {Harmonic analysis on reductive groups, Proc. Conf., Brunswick/ME
    (USA) 1989, Prog. Math.},
 volume={101},
  publisher = {Birkh\"{a}user, Boston-Basel-Berlin},
  year = {1991},
pages={315--388},
  editor = {W. Barker and P. Sally},
}

% \bib{Vo98}{article}{
%   author = {Vogan, D. A. },
%   title = {The method of coadjoint orbits for real reductive groups},
%  %booktitle ={Representation theory of Lie groups (Park City, UT, 1998)},
%  journal = {Representation theory of Lie groups (Park City, UT, 1998). IAS/Park City Math. Ser.},
%   volume={8},
%   publisher = {Amer. Math. Soc.},
%   year = {2000},
% pages={179--238},
% }



% \bib{Vo00}{article}{
%   author = {Vogan, D. A. },
%   title = {Unitary representations of reductive Lie groups},
%  %booktitle ={Mathematics towards the Third Millennium (Rome, 1999)},
%  journal ={Mathematics towards the Third Millennium (Rome, 1999). Accademia Nazionale dei Lincei, (2000)},
%   %series = {Accademia Nazionale dei Lincei, 2000},
%  %volume={9},
% pages={147--167},
% }


% \bib{Wa1}{book}{
%   title={Real reductive groups I},
%   author={Wallach, N. R.},
%   year={1988},
%   publisher={Academic Press Inc. }
% }

\bib{Wa2}{book}{
title={Real reductive groups II},
author={Wallach, N. R.},
year={1992},
publisher={Academic Press Inc. }
}


% \bib{Weyl}{book}{
%   title={The classical groups: their invariants and representations},
%   author={Weyl, H.},
%   year={1947},
%   publisher={Princeton University Press}
% }

% \bib{Ya}{article}{
%   title={Degenerate principal series representations for quaternionic unitary groups},
%   author={Yamana, S.},
%   year = {2011},
%   journal={Israel J. Math.},
%   volume = {185},
%   pages= {77--124}
% }


\bib{Zu}{article}{
  title={Tensor products of finite and infinite dimensional representations of semisimple Lie groups},
  author={Zuckerman, G.},
  year = {1977},
  journal={Ann. of Math. 106},
  volume = {106},
  pages= {295--308}
}


% \bib{EGAIV4}{article}{
%   title = {\'El\'ements de g\'eom\'etrie alg\'brique IV 4: \'Etude locale des
%     sch\'emas et des morphismes de sch\'emas},
%   author = {Grothendieck, Alexandre},
%   author = {Dieudonn\'e, Jean},
%   journal  = {Inst. Hautes \'Etudes Sci. Publ. Math.},
%   volume = {32},
%   year = {1967},
%   pages = {5--361}
% }



\end{biblist}
\end{bibdiv}

\end{document}


Write $\check V$ for the standard module of $\check \g$, which  either a complex symmetric bilinear space or a complex symplectic space. We say that a parabolic subalgebra
 $\check \p$ of $\check \g$ is $\check \CO$-relevant if the following conditions are satisfied:
\begin{itemize}
\item
it is the stabilizer of a totally isotropic subspace $\check V'_{\mathrm b}$ of $\check V$ of dimension $n_\mathrm b$;

\item
some (and hence all) Levi factor of $\check \p$ has nonempty intersection with $\check \CO$.
%\item
%if $\star\in\{D,D^*\}$ and $\check \CO$ has bad parity, then
%the number of positive entries in $\lambda_{\check \CO}$ has the same parity as $ \frac{l}{2}$.
%Here we have used the identifications
%\[
 %   \hha=\b/[\b,\b]=\prod_{i=1}^l \gl(V_i/V_{i-1}) =\C^l
%\]
%so that $\lambda_{\check \CO}\in (\frac{1}{2}, \frac{1}{2}, \dots, \frac{1}{2})+\Z^l\subset \C^l=\hha^*$,
%where $0=V_0\subset V_1\subset \cdots V_{l-1} \subset V_l$ is a full flag in $V_l$, and $\b$ is the Borel subalgebra of $\g$ stabilizing this flag.
\end{itemize}
Note that the first condition implies the second one, except for the case when $\star\in \{D, D^*\}$ and $\check \CO$ has bad parity and is nonempty. In all cases,   up to conjugation by $\Ad(\check \g)$ there exists a unique  parabolic subalgebra of $\check \g$ that is $\check \CO$-relevant. Fix such a parabolic subalgebra $\check \p$ and a Levi factor $\check \l$ of it. Then we have an obvious decomposition
\[
  \check \l= \check \g'_\mathrm b \times \check \g_\mathrm g
\]
such that there are natural isomorphism
\[
\check \g'_\mathrm b\cong \gl_{n_\mathrm b}(\C)
\]
and
\[
\check \g_\mathrm g\cong
  \begin{cases}
      \s\p_{2\nng}(\bC), &\quad  \text{if } \star \in \set{B,\wtC}; \\
  \o_{2\nng+1}(\bC), & \quad  \text{if } \star \in \set{C,C^{*}}; \\
    \o_{2\nng}(\bC), &\quad \text{if } \star \in \set{D,D^{*}}.\\
  \end{cases}
\]

We have a decomposition
\[
  \l\cap \check \CO=\check \CO'_\mathrm b\times \check \CO_\mathrm g
\]
where $\check \CO'_\mathrm b\in \overline \Nil(\check \g'_\mathrm b)$ is the nilpotent orbit whose Young diagram equals $\mathbf d'_\mathrm b$, and  $\check \CO_\mathrm g\in \overline \Nil(\check \g_\mathrm g)$ is the nilpotent orbit whose Young diagram equals $\mathbf d_\mathrm g$.


%%% Local Variables:
%%% coding: utf-8
%%% mode: latex
%%% ispell-local-dictionary: "en_US"
%%% End:
