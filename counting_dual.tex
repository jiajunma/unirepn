\documentclass[counting_main.tex]{subfiles}
\begin{document}
\externaldocument{\subfix{main}}


\subsection{Counting equality by Vogan duality}
In this section, we prove the counting equality for small representations
only assuming the existence of Vogan duality for $G$.

\def\dG{\ckM}
\def\CHC{\sC^{\mathrm{HC}}}

We recall the following key properties of Vogan duality. Fix a regular element
$\lambda$ in $\hha^{*}$, and $B$ is a block in $\cP_{\lambda}(G)$.
Suppose $B$ satisfies the Vogan duality. This means that
there are
\begin{itemize}
  \item
a real reductive group $\dG$ (depends on $\lambda$) in the Harish-Chandra,

\item a block $\ckB$ in $\cP_{\ckrho}(\ckcG)$ where $\ckrho$ is the half sun of
roots of $\ckG$, and
\item
is an bijection %$[\gamma]\in \cP_{\lambda}(G)$
  \[
    \begin{array}{rccc}
     &\cP_{\lambda}(G)&\rightarrow& \cP_{\ckrho}(\dG)\\
      &\gamma & \mapsto& \ckgamma
    \end{array}
  \]
\end{itemize}
such that the above bijection respects Harish-Chandra cells and for $\gamma$
\[
  \LV(\gamma) \cong \LV(\ckgamma)\otimes \sgn.
\]

Now we consider the partition of $\cP_{\lambda}(G)$ into Harish-Chandra cells.
Let
\[
\fC:= \set{\CHC_{\gamma}|\gamma\in \cP_{\lambda}(G)}
\]
be the set of Harish-Chandra cells.
Fix an element $\mu\in [\lambda]$ and define
\[
\begin{split}
  \fC_{\mu}  := &\set{\CHC_{\gamma} \in \fC| [1_{W_{\mu}}:\LV(\gamma)]\neq 0}, \AND\\
  \fC'_{\mu} :=& \fC - \fC_{\mu}.
\end{split}
\]

The following lemma is crucial for us.
Recall the definition of  the special representation $\sigma_{\mu}$.
\begin{lem}
For each cell $\CHC_{\gamma}\in \fC_{\mu}$.
We have the following possible cases
\begin{itemize}
  \item either
        $\sigma_{\mu}$ occurs in $\LV_{\gamma}$ and in which case every irreducible representation $\sigma$ occurs in  $\LV_{\gamma}$
        satisfies $\sigma\approxLR \sigma_{\mu}$, or
  \item $\sigma_{\mu}$ does not occurs in $\LV_{\gamma}$ and every irreducible representation $\sigma$ occurs in  $\LV_{\gamma}$
        satisfies $\sigma\lneqL \sigma_{\mu}$.
\end{itemize}
\end{lem}
\begin{proof}
  Note that $\sigma_{\mu}$ is the maximal irr. repn. under $\leqLR$ order
  satisfies $[1_{W_{\mu}}: \sigma_{\mu}]\neq 0$.

  By Vogan duality, every representation in $\LV_{\ckgamma}$ contains
  $\sgn_{W_{\mu}}$.
  Which implies that, every repn. $\sigma$ in $\CHC_{\gamma}$  satisfies
  \[
   \sigma_{\mu}\otimes \sgn\leqLR \sigma\otimes \sgn
   \Leftrightarrow \sigma \leqLR \sigma_{\mu}
  \]
  Suppose $\sigma_{\mu}$ occurs

\end{proof}


\end{document}

%%% Local Variables:
%%% mode: latex
%%% TeX-master: t
%%% End:
