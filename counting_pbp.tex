\documentclass[counting_main.tex]{subfiles}


\begin{document}

\section{Combinatorics of painted bi-partitions}

In this section, we assume $\ckcO = \ckcOg$ and $\star \in \Set{B,C,\wtC,C^{*},D,D^{*}}$.

Recall the definition of the set of primitive pairs $\CPP(\ckcO)$.

The following proposition is the main result of this section
\begin{prop} \label{prop:PBP}
  When $\star\in \set{C^{*}, D^{*}}$,
  \[
    \PBP_{\star}(\tau_{\wp}) = \emptyset \quad \text{if } \wp \neq \emptyset.
  \]
  When $\star\in \set{B,C,\wtC,D}$,
  \[
    \sharp(\PBP_{\star}(\tau_{\emptyset})) = \sharp(\PBP_{\star}(\tau_{\wp}))
    \quad \forall \wp \subseteq \CPP(\ckcO).
  \]
  In particular, we always have
  \[
    \sharp(\tPBP_{\star}(\ckcO))  = \sharp(\PBPe_{\star}(\ckcO)).
  \]
\end{prop}

%\subsection{The case of quaternionic groups}

The quaternionic case is simple.

\begin{proof}[Proof of {\Cref{prop:PBP}} the quaternionic case]

  Recall the definition of $\tau_{\sP}$ in \Cref{sec:LCBCD}.

  \smallskip

First consider the case when $\star = C^{*}$.
Suppose that $\wp \neq \emptyset$. Then we have
\begin{equation}\label{eq:res.C*}
  \bfcc_{i}(\imath_{\wp}) = \half(\bfrr_{2i-1}(\ckcO)+1)>
  \half(\bfrr_{2i}(\ckcO)-1) = \bfcc_{i}(\jmath_{\wp})
  \quad \forall (2i-1, 2i)\in \wp,
\end{equation}
On the other hand, suppose $\PBP_{\star}(\tau_{\wp})$ is non-empty and
$\uptau = (\imath_{\wp}, \cP)\times (\jmath_{\wp},\cQ)\times \star$
is an element in it.
By the requirement of painted bipartition, we have
\[
  \bfcc_{i}(\imath_{\wp}) = \sharp\set{j| \cP(i,j)=\bullet}
= \sharp\set{j| \cQ(i,j)=\bullet}
\leq \bfcc_{i}(\imath_{\wp}) \quad \forall i=1,2,3,\cdots,
\]
which is contradict to \eqref{eq:res.C*}.
Hence, $\PBP_{\star}(\tau_{\wp})= \emptyset$.

\smallskip

Now consider the case when $\star = D^{*}$.
Suppose that $\wp \neq \emptyset$. Then we have
\begin{equation}\label{eq:res.D*}
  \bfcc_{i+1}(\imath_{\wp}) = \half(\bfrr_{2i}(\ckcO)+1)>
  \half(\bfrr_{2i+1}(\ckcO)-1) = \bfcc_{i}(\jmath_{\wp})
  \quad \forall (2i, 2i+1)\in \wp.
\end{equation}
On the other hand, suppose $\PBP_{\star}(\tau_{\wp})$ is non-empty and
$\uptau = (\imath_{\wp}, \cP)\times (\jmath_{\wp},\cQ)\times \star$
is an element in it.
By the requirement of painted bipartition, we have
\[
  \bfcc_{i+1}(\imath_{\wp}) \leq \sharp\set{j| \cP(i,j)=\bullet}
  =\sharp\set{j| \cQ(i,j)=\bullet}
  \leq \bfcc_{i}(\jmath_{\wp}) \quad \forall i = 1,2,3, \cdots,
\]
which is contradict to \eqref{eq:res.D*}.
Hence, $\PBP_{\star}(\tau_{\wp})= \emptyset$.

\end{proof}


The rest of the section is devoted to the proof for
$\star \in \set{B,C,\wtC,D}$.


\begin{lem}
  Suppose $\star = \set{C,\wtC}$, $\wp\neq \emptyset$, and $t:=\min{t|(2t-1,2t)\in \wp}$.
  Let $\sQ:=\wp - \set{(2t-1,2t)}$.
  \[
  \sharp(\PBP_{\star}(\tau_{\sQ})) = \sharp(\PBP_{\star}(\tau_{\wp})).
  \]
\end{lem}
\begin{proof}
  We prove the equality by defining a bijection
  \[
    T_{\wp,\sQ}\colon \PBP_{\star}(\tau_{\wp}) \rightarrow \PBP_{\star}(\tau_{\sQ}).
  \]

  We begin with the simpler case.

  Suppose $\star = \wtC$. Let
  $(b_{1},b_{2}) = (\frac{\bfrr_{2t-1}(\ckcO)}{2},\frac{\bfrr_{2t}(\ckcO)}{2})$.
  Then
  \[
    (\bfcc_{t}(\imath_{\sQ}), \bfcc_{t}(\jmath_{\sQ})) = (b_{1},b_{2})= (\bfcc_{t}(\jmath_{\wp}), \bfcc_{t}(\imath_{\wp})).
  \]
  For other columns, we have
  \[
    \bfcc_{i}(\imath_{\sQ}) = \bfcc_{i}(\imath_{\wp}) \AND \bfcc_{i}(\imath_{\sQ}) = \bfcc_{i}(\imath_{\wp})\quad \text{when
      $i\neq t$}.
  \]
  For
  $\uptau': = (\imath_{\wp},\cP')\times (\jmath_{\wp},\cQ')\in \PBPs(\tau_{\wp})$,
  define
  \[
  T_{\wp,\sQ}(\uptau'):=\uptau := (\imath_{\sQ},\cP)\times (\jmath_{\sQ},\cQ)\in \PBP_{\star}(\tau_{\sQ})
  \]
  by the following formula
  \[
    \begin{split}
      \text{$\forall (i,j)\in \BOX{\imath_{\sQ}}$,} \quad \cP(i,j) &= \begin{cases}
        s& \text{if $j=t$ and  $\cQ'(i,j)=r$,}\\
        c& \text{if $j=t$ and  $\cQ'(i,j)=d$,}\\
        \cP'(i,j) &\text{otherwise.}
      \end{cases}\\
      \text{$\forall (i,j)\in \BOX{\jmath_{\sQ}}$,} \quad   \cQ(i,j) &=  \cQ'(i,j).\\
    \end{split}
  \]
  We leave it to the reader to check that the above formula does define an valid
  painted bipartition. Retain the above notation, it is easy to check that
  the inverse map
  \[
    T_{\sQ,\wp}\colon \PBP_{\star}(\tau_{\sQ}) \rightarrow \PBP_{\star}(\tau_{\wp})\quad
    \uptau \mapsto \uptau'
  \]
  is given by the following formula:
  \[
    \begin{split}
      \text{$\forall (i,j)\in \BOX{\imath_{\wp}}$,} \quad   \cP'(i,j) &=  \cP(i,j),\\
      \text{$\forall (i,j)\in \BOX{\jmath_{\wp}}$,} \quad \cQ'(i,j) &= \begin{cases}
        r& \text{if $j=t$ and  $\cP(i,j)=s$,}\\
        d& \text{if $j=t$ and  $\cP(i,j)=c$,}\\
        \cQ(i,j) &\text{otherwise.}
      \end{cases}
    \end{split}
  \]
\end{proof}


\end{document}

%%% Local Variables:
%%% mode: latex
%%% TeX-master: "counting_main"
%%% End:
