\documentclass[counting_main.tex]{subfiles}


\begin{document}

\section{Combinatorics of painted bi-partitions}

In this section, we assume $\ckcO = \ckcOg$ and
$\star \in \Set{B,C,\wtC,C^{*},D,D^{*}}$.

Recall the definition of the set of primitive pairs $\CPP(\ckcO)$.

The following proposition is the main result of this section
\begin{prop} \label{prop:PBP} When $\star\in \set{C^{*}, D^{*}}$,
  \[
    \PBP_{\star}(\tau_{\wp}) = \emptyset \quad \text{if } \wp \neq \emptyset.
  \]
  When $\star\in \set{B,C,\wtC,D}$,
  \[
    \sharp(\PBP_{\star}(\tau_{\emptyset})) = \sharp(\PBP_{\star}(\tau_{\wp})) \quad \forall \wp \subseteq \CPP(\ckcO).
  \]
  In particular, we always have
  \[
    \sharp(\tPBP_{\star}(\ckcO)) = \sharp(\PBPe_{\star}(\ckcO)).
  \]
\end{prop}

% \subsection{The case of quaternionic groups}

The quaternionic case is simple.

\begin{proof}%[Proof of {\Cref{prop:PBP}} the quaternionic case]

  Recall the definition of $\tau_{\wp}$ in \Cref{sec:LCBCD}.

  \smallskip

  First consider the case when $\star = C^{*}$. Suppose that
  $\wp \neq \emptyset$. Then we have
  \begin{equation}\label{eq:res.C*}
    \bfcc_{i}(\imath_{\wp}) = \half(\bfrr_{2i-1}(\ckcO)+1)>
    \half(\bfrr_{2i}(\ckcO)-1) = \bfcc_{i}(\jmath_{\wp})
    \quad \forall (2i-1, 2i)\in \wp,
  \end{equation}
  On the other hand, suppose $\PBP_{\star}(\tau_{\wp})$ is non-empty and
  $\uptau = (\imath_{\wp}, \cP)\times (\jmath_{\wp},\cQ)\times \star$ is an
  element in it. By the requirement of painted bipartition, we have
  \[
    \bfcc_{i}(\imath_{\wp}) = \sharp\set{j| \cP(i,j)=\bullet} = \sharp\set{j| \cQ(i,j)=\bullet} \leq \bfcc_{i}(\imath_{\wp}) \quad \forall i=1,2,3,\cdots,
  \]
  which is contradict to \eqref{eq:res.C*}. Hence,
  $\PBP_{\star}(\tau_{\wp})= \emptyset$.

  \smallskip

  Now consider the case when $\star = D^{*}$. Suppose that $\wp \neq \emptyset$.
  Then we have
  \begin{equation}\label{eq:res.D*}
    \bfcc_{i+1}(\imath_{\wp}) = \half(\bfrr_{2i}(\ckcO)+1)>
    \half(\bfrr_{2i+1}(\ckcO)-1) = \bfcc_{i}(\jmath_{\wp})
    \quad \forall (2i, 2i+1)\in \wp.
  \end{equation}
  On the other hand, suppose $\PBP_{\star}(\tau_{\wp})$ is non-empty and
  $\uptau = (\imath_{\wp}, \cP)\times (\jmath_{\wp},\cQ)\times \star$ is an
  element in it. By the requirement of painted bipartition, we have
  \[
    \bfcc_{i+1}(\imath_{\wp}) \leq \sharp\set{j| \cP(i,j)=\bullet} =\sharp\set{j| \cQ(i,j)=\bullet} \leq \bfcc_{i}(\jmath_{\wp}) \quad \forall i = 1,2,3, \cdots,
  \]
  which is contradict to \eqref{eq:res.D*}. Hence,
  $\PBP_{\star}(\tau_{\wp})= \emptyset$.


  The other cases follow  by removing elements from $\wp$ one-by-one using
  \Cref{lem:down} below.
\end{proof}

The rest of the section is devoted to the proof of \Cref{lem:down}.


\def\PPm{\wp_{\downarrow}}
\def\uptaum{\uptau_{\downarrow}}

% Suppose $\star = \wtC$, $\wp\neq \emptyset$, and
% $t:=\min{t|(2t-1,2t)\in \wp}$. Let $\PPm:=\wp - \set{(2t-1,2t)}$. Let
% $\PPm:=\wp - \set{(2t-1,2t)}$.

\begin{lem}\label{lem:down}
  Suppose $\star \in \set{B, C, \wtC, D}$ and
  $\wp$ is a non-empty subset of $\CPP(\ckcO)$.
  Let
  \[
    t:=
    \begin{cases}
      \min\set{t|(2t-1,2t)\in \wp} & \text{when $\star \in \set{C,\wtC}$}\\
      \min\set{t|(2t,2t+1)\in \wp} & \text{when $\star \in \set{B,D}$}\\
    \end{cases}
  \]
  and
  \[
    \PPm:=
    \begin{cases}
      \wp - \set{(2t-1,2t)}  & \text{when $\star \in \set{C,\wtC}$}\\
      \wp -  \set{(2t,2t+1)} & \text{when $\star \in \set{B,D}$}\\
    \end{cases}
  \]
  Then
  \[
    \sharp(\PBP_{\star}(\tau_{\PPm})) = \sharp(\PBP_{\star}(\tau_{\wp})).
  \]
\end{lem}
\begin{proof}
  We prove the equality by defining a bijection
  \[
    T_{\PPm,\wp}\colon \PBP_{\star}(\tau_{\PPm}) \rightarrow \PBP_{\star}(\tau_{\wp})\quad \uptaum \mapsto \uptau
  \]
  and its inverse $T_{\wp,\PPm}$ explicitly case by case.
  In the following discussion, let
  $\uptau = (\imath_{\wp},\cP_{\uptau})\times (\imath_{\wp},\cQ_{\uptau})$
  denote an element in $\PBP_{\star}(\tau_{\wp})$ and
  $\uptaum = (\imath_{\wp},\cP_{\uptaum})\times (\imath_{\wp},\cQ_{\uptaum})$
  denote an element in $\PBP_{\star}(\tau_{\PPm})$.

  \medskip

  % \[
  %   T_{\wp,\PPm}\colon \PBP_{\star}(\tau_{\wp}) \rightarrow \PBP_{\star}(\tau_{\PPm}).
  % \]


  We start with the simplest case.

  \smallskip

  Suppose $\star = \wtC$.
  %Let $(b_{1},b_{2}) = (\frac{\bfrr_{2t-1}(\ckcO)}{2},\frac{\bfrr_{2t}(\ckcO)}{2})$.
  Then
  \[
    \begin{split}
      (\bfcc_{t}(\imath_{\PPm}), \bfcc_{t}(\jmath_{\PPm}))
      &= (\bfcc_{t}(\jmath_{\wp}), \bfcc_{t}(\imath_{\wp})),\AND\\
      % (\bfcc_{t}(\imath_{\PPm}), \bfcc_{t}(\jmath_{\PPm}))
      % &= (b_{1},b_{2})= (\bfcc_{t}(\jmath_{\wp}), \bfcc_{t}(\imath_{\wp})),\AND\\
      (\bfcc_{i}(\imath_{\PPm}), \bfcc_{i}(\jmath_{\PPm}) )
      & = (\bfcc_{i}(\imath_{\wp}), \bfcc_{i}(\jmath_{\wp}) ) \quad \text{when $i\neq t$}.
    \end{split}
  \]

  For $\uptaum\in\PBPs(\tau_{\PPm})$ define $T_{\PPm,\wp}(\uptaum) = \uptau$ where
  $\uptau$ is given by the following formula:
  \[
    \begin{split}
      \text{$\forall (i,j)\in \BOX{\imath_{\wp}}$,} \quad   \cP_{\uptau}(i,j) &=  \cP_{\uptaum}(i,j),\\
      \text{$\forall (i,j)\in \BOX{\jmath_{\wp}}$,} \quad \cQ_{\uptau}(i,j) &= \begin{cases}
        r& \text{if $j=t$ and  $\cP_{\uptaum}(i,j)=s$,}\\
        d& \text{if $j=t$ and  $\cP_{\uptaum}(i,j)=c$,}\\
        \cQ_{\uptaum}(i,j) &\text{otherwise.}
      \end{cases}
    \end{split}
  \]
  We leave it to the reader to check that the above formula does define a valid
  painted bipartition $\uptau$. Retain the above notation, it is easy to check that the
  inverse map
  \[
    T_{\wp,\PPm}\colon \PBP_{\star}(\tau_{\wp}) \rightarrow \PBP_{\star}(\tau_{\PPm}).
  \]
  is given by the following formula
  \[
    \begin{split}
      \text{$\forall (i,j)\in \BOX{\imath_{\PPm}}$,} \quad \cP_{\uptaum}(i,j) &= \begin{cases}
        s& \text{if $j=t$ and  $\cQ_{\uptau}(i,j)=r$,}\\
        c& \text{if $j=t$ and  $\cQ_{\uptau}(i,j)=d$,}\\
        \cP_{\uptau}(i,j) &\text{otherwise.}
      \end{cases}\\
      \text{$\forall (i,j)\in \BOX{\jmath_{\PPm}}$,} \quad   \cQ_{\uptaum}(i,j) &=  \cQ_{\uptau}(i,j).\\
    \end{split}
  \]
  % We leave it to the reader to check that the above formula does define a valid
  % painted bipartition $\uptaum$. Retain the above notation, it is easy to check that the
  % inverse map
  % \[
  %   T_{\PPm,\wp}\colon \PBP_{\star}(\tau_{\PPm}) \rightarrow \PBP_{\star}(\tau_{\wp})\quad \uptaum \mapsto \uptau
  % \]
  % is given by the following formula:
  % \[
  %   \begin{split}
  %     \text{$\forall (i,j)\in \BOX{\imath_{\wp}}$,} \quad   \cP_{\uptau}(i,j) &=  \cP_{\uptaum}(i,j),\\
  %     \text{$\forall (i,j)\in \BOX{\jmath_{\wp}}$,} \quad \cQ_{\uptau}(i,j) &= \begin{cases}
  %       r& \text{if $j=t$ and  $\cP_{\uptaum}(i,j)=s$,}\\
  %       d& \text{if $j=t$ and  $\cP_{\uptaum}(i,j)=c$,}\\
  %       \cQ_{\uptaum}(i,j) &\text{otherwise.}
  %     \end{cases}
  %   \end{split}
  % \]
  This finish the prove for the case when $\star=\wtC$. \medskip

  \medskip

  Suppose $\star = C$.
  % Let
  % $(b_{1},b_{2}) = (\frac{\bfrr_{2t-1}(\ckcO)-1}{2},\frac{\bfrr_{2t}(\ckcO)+1}{2})$.
  Then
  \[
    \begin{split}
      (\bfcc_{t}(\imath_{\PPm}), \bfcc_{t}(\jmath_{\PPm})) &=
      (\bfcc_{t}(\jmath_{\wp})+1, \bfcc_{t}(\imath_{\wp})-1) \AND \\
     %  &= (b_{2},b_{1}),  \\
     % &= (b_{1}-1,b_{2}+1),\AND\\
      % (\bfcc_{t}(\imath_{\PPm}), \bfcc_{t}(\jmath_{\PPm})) &= (b_{2},b_{1}),  \\
      % (\bfcc_{t}(\imath_{\wp}), \bfcc_{t}(\jmath_{\wp})) &= (b_{1}-1,b_{2}+1),\AND\\
      (\bfcc_{i}(\imath_{\PPm}),\bfcc_{i}(\jmath_{\PPm})) &=(\bfcc_{i}(\imath_{\wp}),\bfcc_{i}(\jmath_{\wp}))\quad \text{when
        $i\neq t$}.
    \end{split}
  \]
  % Let
  % $a = \half(\bfrr_{2t-1}(\ckcO)-\bfrr_{2t}(\ckcO))-1 = \bfcc_{t}(\jmath_{\PPm})-\bfcc_{t}(\imath_{\PPm})$.

  \trivial[]{
    The idea of the definition of $T_{\PPm,\wp}$ is that we move ``$s$'' appeared
    in the $t$-th column of $\cQ_{\uptaum}$ to the $t$-th column of
    $\cP_{\uptau}$.
  }

  For $\uptaum\in \PBPs(\tau_{\PPm})$, we define $\uptau$ by the following
  algorithm:
  \begin{description}
    \item[STEP~1] Define a function
          $\cP'\colon \BOX{\imath_{\wp}}\rightarrow \set{\bullet,r,c,d}$
           by the following rules:
          \begin{enumerate}[label=(\alph*)]
            \item Suppose
            $\cP_{\uptaum}(\bfcc_{t}(\imath_{\PPm}),t)\neq \bullet$.
            \begin{itemize}
              \item If $\bfcc_{t}(\imath_{\PPm})\geq 2$ and
              $\cP_{\uptaum}(\bfcc_{t}(\imath_{\PPm})-1,t) = c$,
              we define
              \[
                \cP'(i,j) := \begin{cases}
                  r ,& \text{if $j=t$ and $\bfcc_{t}(\imath_{\PPm})-1
                    \leq i \leq \bfcc_{t}(\imath_{\wp})-2$},\\
                  c ,& \text{if $(i,j)=(\bfcc_{t}(\imath_{\wp})-1,t)$},\\
                  d ,& \text{if $(i,j)=(\bfcc_{t}(\imath_{\wp}),t)$},\\
                  \cP_{\uptaum}(i,j) ,&\text{otherwise}.
                \end{cases}
              \]
              \item Otherwise, we define
              \[
                \cP'(i,j) := \begin{cases}
                  r ,& \text{if $j=t$ and $\bfcc_{t}(\imath_{\PPm})
                    \leq i \leq \bfcc_{t}(\imath_{\wp})-1$},\\
                  \cP_{\uptaum}(\bfcc_{t}(\imath_{\PPm}),t) ,&
                  \text{if $(i,j)=(\bfcc_{t}(\imath_{\wp}),t)$},\\
                  \cP_{\uptaum}(i,j) ,&\text{otherwise}.
                \end{cases}
              \]
            \end{itemize}
            \item Suppose $\cP_{\uptaum}(\bfcc_{t}(\imath_{\PPm}),t)=\bullet$.
            \begin{itemize}
              \item If $\bfcc_{t+1}(\imath_{\PPm}) = \bfcc_{t}(\imath_{\PPm})$
              and
              $\cP_{\uptaum}(\bfcc_{t}(\imath_{\PPm}),t+1) = r$,
              we define
              \[
                \cP'(i,j) := \begin{cases}
                  r ,& \text{if $j=t$ and $\bfcc_{t}(\imath_{\PPm})\leq i \leq \bfcc_{t}(\imath_{\wp})-1$},\\
                  c ,& \text{if $(i,j)=(\bfcc_{t+1}(\imath_{\PPm}),t+1)$},\\
                  d ,& \text{if $(i,j)=(\bfcc_{t}(\imath_{\wp}),t)$},\\
                  \cP_{\uptaum}(i,j) ,&\text{otherwise}.
                \end{cases}
              \]
              \item Otherwise, we define
              \[
                \cP'(i,j) := \begin{cases}
                  r ,& \text{if $j=t$ and $\bfcc_{t}(\imath_{\PPm})\leq i \leq \bfcc_{t}(\imath_{\wp})-2$},\\
                  c ,& \text{if $(i,j)=(\bfcc_{t}(\imath_{\wp})-1,t)$},\\
                  d ,& \text{if $(i,j)=(\bfcc_{t}(\imath_{\wp}),t)$},\\
                  \cP_{\uptaum}(i,j) ,&\text{otherwise}.
                \end{cases}
              \]
            \end{itemize}
          \end{enumerate}

    \item[STEP 2] Let $\cP_{\uptau}:= \cP'$ if $\cP'$ is a valid painted partition.
          Otherwise, $t>1$ and the problem happens on the following $2\times 2$ square
          \[
          A :=
          \begin{pmatrix}
            \cP'(\bfcc_{t}(\imath_{\wp})-1,t-1) & \cP'(\bfcc_{t}(\imath_{\wp})-1,t) \\
            \cP'(\bfcc_{t}(\imath_{\wp})\;\phantom{-1}\;,t-1) & \cP'(\bfcc_{t}(\imath_{\wp})\;\phantom{-1}\;,t) \\
          \end{pmatrix}
          \]
          \trivial[]{
          Note that  $\cP'(\bfcc_{t}(\imath_{\wp})-1,t-1)$ always equal to
          $\bullet$.
          }

          There are four possibilities of $A$ (see \eqref{eq:modP}).
          Let $\cP_{\uptaum}\colon \BOX{\imath_{\PPm}}\rightarrow \set{\bullet,r,c,d}$
          be the painted partition defined in the following way:
          \begin{itemize}
            \item When $(i,j)\in \BOX{\imath_{\PPm}}$ and
            $\set{\bfcc_{t}(\imath_{\wp})-i,t-1}\cap\set{0,1}=\emptyset$
            (i.e. $(i,j)$ is not one of the four boxes corresponding to
            $A$), we define
            \[
              \cP_{\uptau}(i,j):= \cP'(i,j).
            \]

            \item For the four boxes corresponding to $A$, we modify them by
            setting
            \begin{equation} \label{eq:modP}
              \begin{split}
                &\begin{pmatrix}
                  \cP_{\uptau}(\bfcc_{t}(\imath_{\wp})-1,t-1) & \cP_{\uptau}(\bfcc_{t}(\imath_{\wp})-1,t) \\
                  \cP_{\uptau}(\bfcc_{t}(\imath_{\wp})\;\phantom{-1}\;,t-1)
                  & \cP_{\uptau}(\bfcc_{t}(\imath_{\wp})\;\phantom{-1}\;,t) \\
                \end{pmatrix}\\
                :=&
                \begin{cases}
                  \begin{pmatrix}
                    r & c\\
                    r & d
                  \end{pmatrix}, & \text{if } A =
                  \begin{pmatrix}
                    \bullet & r\\
                    r & r
                  \end{pmatrix},\\[1.5em]
                  \begin{pmatrix}
                    r & c\\
                    c & d
                  \end{pmatrix}, & \text{if } A =
                  \begin{pmatrix}
                    \bullet & r\\
                    c & r
                  \end{pmatrix},\\[1.5em]
                  \begin{pmatrix}
                    r & c\\
                    d & d
                  \end{pmatrix}, & \text{if } A =
                  \begin{pmatrix}
                    \bullet & r\\
                    d & r
                  \end{pmatrix},\\[1.5em]
                  \begin{pmatrix}
                    c & c\\
                    d & d
                  \end{pmatrix}, & \text{if } A =
                  \begin{pmatrix}
                    \bullet & r\\
                    d & c
                  \end{pmatrix}.\\
                \end{cases}
              \end{split}
            \end{equation}
          \end{itemize}
    \item[STEP 3] The painted bipartition $\uptau$ is uniquely determined by
          $\cP_{\uptau}$. In fact, $\cQ_{\uptau}$ is given by the following
          formula: for $(i,j)\in \BOX{\jmath_{\wp}}$,
          \[
          \cQ_{\uptau}(i,j) :=
          \begin{cases}
            s, & \begin{minipage}{17em}if $\cP'$ is not a valide pained partition\\
              and $(i,j)= (\bfcc_{t}(\imath_{\wp})-1,t-1)$,
              \end{minipage}\\
            \cQ_{\uptaum}(i,j), & \text{otherwise.}
            \end{cases}
          \]
  \end{description}

 We leave it to the reader to check that $\uptau$ is a valid painted-bipartition
 and to construct the inverse map $T_{\wp,\PPm}$ by reversing the above steps.

 \trivial[]{
   The inverse map $T_{\wp,\PPm}$ is given by the following algorithm:
   \begin{description}
     \item[STEP 1] We first recover $\cP'$.
           If $t=1$ or $\cP'(\bfcc_{t}(\imath_{\wp})-1,t-1)=\bullet$, then
           $\cP':= \cP_{\wp}$.
           Otherwise,
           $\cP'$ is given by $\cP_{\wp}$ except the $2\times 2$ square in
           \eqref{eq:modP} wihch is given by reversing the formula cited.
     \item[STEP 2]

           \def\xxn{\cP_{\uptaum}(\bfcc_t(\imath_{\PPm})-1,t)} %x_0
           \def\xxo{\cP_{\uptaum}(\bfcc_t(\imath_{\wp}),t)} %x_1
           \def\xxd{\cP_{\uptaum}(\bfcc_t(\imath_{\wp}),t+1)} %x_2
           \def\yyn{\cP'(\bfcc_t(\imath_{\PPm})-1,t)} %y_0
           \def\yyo{\cP'(\bfcc_t(\imath_{\wp})-1,t)} %y_1
           \def\yyt{\cP'(\bfcc_t(\imath_{\wp}),t)} %y_3
           \def\yyd{\cP'(\bfcc_t(\imath_{\wp}),t+1)} %y_2
           We have the following cases:
           \begin{enumerate}[label=(\alph*)]
             \item Suppose $\yyo=r$.
             \begin{itemize}
               \item If $\bfcc_{t+1}(\imath_{\wp}) = \bfcc_{t}(\imath_{\PPm})$
               and
               \[
                 (\yyd,\yyt) = (c,d),
               \]
               let
               \[
                 (\xxo,\xxd):=(\bullet, r)
               \]
               \item Otherwise, let \[
                 \xxo:=\yyt.
               \]
             \end{itemize}
             \item Suppose $\yyo=c$
             \begin{itemize}
               \item If $\bfcc_{t}(\imath_{\PPm})\geq 2$ and $\xxn=r$,
               then let
               \[
                 (\xxn,\xxo):=(c,d).
               \]
               \item Otherwise, let
               \[
                 \xxo :=\bullet.
               \]
             \end{itemize}
           \end{enumerate}
           For the boxes $(i,j)$ in $\BOX{\imath_{\uptaum}}$ which are not specified
           in the above procedure, set
           \[
           \cP_{\uptaum}(i,j):=\cP'(i,j).
           \]
     \item[STEP 3]
           Now $\cP_{\uptaum}$ uniquely determine the painted bipartition
           $\uptaum$.
   \end{description}
   The construction of the inverse map implies that $T_{\PPm,\wp}$ is a
   bijection.
 }

  \def\ckcOa{\ckcO^{\uparrow}}
  \def\PPa{\wp^{\uparrow}}
  \def\PPam{\wp^{\uparrow}_{\downarrow}}
  \def\uptaua{\uptau^{\uparrow}}
  \def\tauPPa{\tau_{\PPa}}
  \def\tauPPam{\tau_{\PPam}}
  \def\stara{\star^{\uparrow}}

  \medskip

  {Now suppose $\star \in \set{B,D}$.}
  We prove the lemma by reducing the problem to the case of $C/\wtC$.
  Let
  \[
    \stara := \begin{cases}
      \wtC ,& \text{when $\star=B$},\\
      C ,& \text{when $\star=D$},
    \end{cases}
  \]
  and $\ckcOa$ be the partition defined by
  \[
    \bfrr_{1}(\ckcOa) = \bfrr_{1}(\ckcO)+2, \AND \bfrr_{i+1}(\ckcOa)
    = \bfrr_{i}(\ckcO)\quad \text{for all $i=1,2,3,\cdots,$}.
  \]
  Clearly
  \[
    \CPP(\ckcOa) = \set{(1,2)}\cup \set{(i+1,i+2)|(i,i+1)\in \CPP(\ckcO)}.
  \]


  We refer to \cite{BMSZ1}*{Definition? } for the definition of (naive) descent $\DD$ of painted bipartitions.

  % Let $\PPa$ denote the subset in
  % $\CPP(\ckcOa)$ defined by
  % \[
  %   \PPa:=\set{(i+1,i+2)|(i,i+1)\in \sP}.
  % \]
  % Let $r_{0}:= \half \bfrr_{1}(\ckcO)+1 = \bfcc_{1}(\imath_{\PPa})$. For $x=c$
  % or $s$, define
  % \[
  %   \PBP_{\wtC}^{x}(\tauPPa):= \Set{\uptaua\in \PBP_{\wtC}(\tau_{\PPa})|\cP_{\uptaua}\left(r_{0},1\right)=x}
  % \]
  % where $\tauPPa$ is the bipartition defined with respect to $\ckcOa$.

  % Let $\PPam:=(\PPa)_{\downarrow}$. It is easy to check that
  % \begin{itemize}
  %   \item
  %   $\PBP_{\wtC}(\tauPPa) = \PBP_{\wtC}^{c}(\tauPPa)\sqcup \PBP_{\wtC}^{s}(\tauPPa)$,
  %   \item the map $T_{\PPa,\PPam}$ restricted into a bijection from
  %   $\PBP_{\wtC}^{x}(\tauPPa)$ onto $\PBP_{\wtC}^{x}(\tauPPam)$ (here $x=s$ or
  %   $c$), and
  %   \item the descent map restricted to bijections
  %   \[
  %     \text{
  %         $\PBP_{\wtC}^{s}(\tau_{\PPa})\longrightarrow \PBP_{B^{+}}(\tau_{\sP})$
  %         and
  %         $\PBP_{\wtC}^{c}(\tau_{\PPa})\longrightarrow \PBP_{B^{-}}(\tau_{\sP})$.
  %     }
  %   \]
  % \end{itemize}

  % Now the bijection $T_{\wp,\wpm}$ is defined by making the following diagram
  % commutative
  % \[
  %   \begin{tikzcd}
  %     \PBP_{\wtC}^{s}(\tau_{\PPa}) \ar[r] \ar[d] & \PBP_{B^{+}}^{s}(\tau_{\wp}) \ar[d]\\
  %     \PBP_{\wtC}^{s}(\tau_{\PPam}) \ar[r] & \PBP_{B^{+}}^{s}(\tau_{\PPm}) \\
  %   \end{tikzcd}
  % \]

 % For each subset $\sP\subset \CPP(\ckcO)$,
  Let $\PPa$ denote the subset in
  $\CPP(\ckcOa)$ defined by
  \[
    \PPa:=\set{(i+1,i+2)|(i,i+1)\in \sP}.
  \]
  % Let $r_{0}:= \half \bfrr_{1}(\ckcO)+1 = \bfcc_{1}(\imath_{\PPa})$.
  % For $x=c$
  % or $s$, define
  % \[
  %   \PBP_{\wtC}^{x}(\tauPPa):= \Set{\uptaua\in \PBP_{\wtC}(\tau_{\PPa})|\cP_{\uptaua}\left(r_{0},1\right)=x}
  % \]
  % where $\tauPPa$ is the bipartition defined with respect to $\ckcOa$.

  Let $\PPam:=(\PPa)_{\downarrow}$. It is easy to check that
 the descent maps (horizontal arrows) in the following diagram are bijections (c.f. \cite{BMSZ1}*{Lemma???}). % restricted to bijections
 % \[
 %   \PBP_{\wtC}(\tau_{\PPa})\longrightarrow \PBP_{B}(\tau_{\sP}).
 % \]
  \[
    \begin{tikzcd}
      \PBP_{\wtC}(\tau_{\PPam}) \ar[r,"\DD",two heads,hook] \ar[d,two heads,hook,"T_{\PPam,\PPa}"']
      & \PBP_{B}(\tau_{\PPm}) \ar[d,dashed,"T_{\PPm,\wp}"]\\
      \PBP_{\wtC}(\tau_{\PPa}) \ar[r,"\DD",two heads,hook] & \PBP_{B}(\tau_{\wp}) \\
    \end{tikzcd}
  \]
  The left vertical arrow is bijective by case $\wtC$.
  Now the bijection $T_{\PPm,\wp}$ is defined by making the above diagram
  commutative.
\end{proof}


\end{document}

%%% Local Variables:
%%% mode: latex
%%% TeX-master: "counting_main"
%%% End:
