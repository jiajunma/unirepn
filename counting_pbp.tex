\documentclass[counting_main.tex]{subfiles}


\begin{document}

\section{Combinatorics of painted bi-partitions}

In this section, we assume $\ckcO = \ckcOg$ and
$\star \in \Set{B,C,\wtC,C^{*},D,D^{*}}$.

Recall the definition of the set of primitive pairs $\CPP(\ckcO)$.

The following proposition is the main result of this section
\begin{prop} \label{prop:PBP} When $\star\in \set{C^{*}, D^{*}}$,
  \[
    \PBP_{\star}(\tau_{\wp}) = \emptyset \quad \text{if } \wp \neq \emptyset.
  \]
  When $\star\in \set{B,C,\wtC,D}$,
  \[
    \sharp(\PBP_{\star}(\tau_{\emptyset})) = \sharp(\PBP_{\star}(\tau_{\wp})) \quad \forall \wp \subseteq \CPP(\ckcO).
  \]
  In particular, we always have
  \[
    \sharp(\tPBP_{\star}(\ckcO)) = \sharp(\PBPe_{\star}(\ckcO)).
  \]
\end{prop}

% \subsection{The case of quaternionic groups}

The quaternionic case is simple.

\begin{proof}[Proof of {\Cref{prop:PBP}} the quaternionic case]

  Recall the definition of $\tau_{\sP}$ in \Cref{sec:LCBCD}.

  \smallskip

  First consider the case when $\star = C^{*}$. Suppose that
  $\wp \neq \emptyset$. Then we have
  \begin{equation}\label{eq:res.C*}
    \bfcc_{i}(\imath_{\wp}) = \half(\bfrr_{2i-1}(\ckcO)+1)>
    \half(\bfrr_{2i}(\ckcO)-1) = \bfcc_{i}(\jmath_{\wp})
    \quad \forall (2i-1, 2i)\in \wp,
  \end{equation}
  On the other hand, suppose $\PBP_{\star}(\tau_{\wp})$ is non-empty and
  $\uptau = (\imath_{\wp}, \cP)\times (\jmath_{\wp},\cQ)\times \star$ is an
  element in it. By the requirement of painted bipartition, we have
  \[
    \bfcc_{i}(\imath_{\wp}) = \sharp\set{j| \cP(i,j)=\bullet} = \sharp\set{j| \cQ(i,j)=\bullet} \leq \bfcc_{i}(\imath_{\wp}) \quad \forall i=1,2,3,\cdots,
  \]
  which is contradict to \eqref{eq:res.C*}. Hence,
  $\PBP_{\star}(\tau_{\wp})= \emptyset$.

  \smallskip

  Now consider the case when $\star = D^{*}$. Suppose that $\wp \neq \emptyset$.
  Then we have
  \begin{equation}\label{eq:res.D*}
    \bfcc_{i+1}(\imath_{\wp}) = \half(\bfrr_{2i}(\ckcO)+1)>
    \half(\bfrr_{2i+1}(\ckcO)-1) = \bfcc_{i}(\jmath_{\wp})
    \quad \forall (2i, 2i+1)\in \wp.
  \end{equation}
  On the other hand, suppose $\PBP_{\star}(\tau_{\wp})$ is non-empty and
  $\uptau = (\imath_{\wp}, \cP)\times (\jmath_{\wp},\cQ)\times \star$ is an
  element in it. By the requirement of painted bipartition, we have
  \[
    \bfcc_{i+1}(\imath_{\wp}) \leq \sharp\set{j| \cP(i,j)=\bullet} =\sharp\set{j| \cQ(i,j)=\bullet} \leq \bfcc_{i}(\jmath_{\wp}) \quad \forall i = 1,2,3, \cdots,
  \]
  which is contradict to \eqref{eq:res.D*}. Hence,
  $\PBP_{\star}(\tau_{\wp})= \emptyset$.

\end{proof}


The rest of the section is devoted to the proof for
$\star \in \set{B,C,\wtC,D}$.

We refer to \cite{BMSZ1} for the definition of descent $\DD$ of painted bipartition.


\def\PPm{\wp_{\downarrow}}
\def\uptaum{\uptau_{\downarrow}}

Suppose $\wp$ is a non-empty subset of $\CPP(\ckcO)$. In this section, we let
\[
  t:=
  \begin{cases}
    \min\set{t|(2t-1,2t)\in \wp} & \text{when $\star \in \set{C,\wtC}$}\\
    \min\set{t|(2t,2t+1)\in \wp} & \text{when $\star \in \set{B,D}$}\\
  \end{cases}
\]
and
\[
  \PPm:=
  \begin{cases}
    \wp - \set{(2t-1,2t)}  & \text{when $\star \in \set{C,\wtC}$}\\
    \wp -  \set{(2t,2t+1)} & \text{when $\star \in \set{B,D}$}\\
  \end{cases}
\]
% Suppose $\star = \wtC$, $\wp\neq \emptyset$, and
% $t:=\min{t|(2t-1,2t)\in \wp}$. Let $\PPm:=\wp - \set{(2t-1,2t)}$. Let
% $\PPm:=\wp - \set{(2t-1,2t)}$.

We begin with the simpler case.
\begin{lem}
  Suppose $\star \in \set{B, \wtC}$.
  \[
    \sharp(\PBP_{\star}(\tau_{\PPm})) = \sharp(\PBP_{\star}(\tau_{\wp})).
  \]
\end{lem}
\begin{proof}
  We prove the equality by defining a bijection
  \[
    T_{\wp,\PPm}\colon \PBP_{\star}(\tau_{\wp}) \rightarrow \PBP_{\star}(\tau_{\PPm}).
  \]
  and its inverse $T_{\PPm,\wp}$ explicitly. In the following discussion, let
  $\uptau = (\imath_{\wp},\cP_{\uptau})\times (\imath_{\wp},\cQ_{\uptau})$
  denote an element in $\PBP_{\star}(\tau_{\wp})$ and
  $\uptaum = (\imath_{\wp},\cP_{\uptaum})\times (\imath_{\wp},\cQ_{\uptaum})$
  denote an element in $\PBP_{\star}(\tau_{\PPm})$.

  Suppose $\star = \wtC$. Let
  $(b_{1},b_{2}) = (\frac{\bfrr_{2t-1}(\ckcO)}{2},\frac{\bfrr_{2t}(\ckcO)}{2})$.
  Then
  \[
    \begin{split}
      (\bfcc_{t}(\imath_{\PPm}), \bfcc_{t}(\jmath_{\PPm}))
      &= (b_{1},b_{2})= (\bfcc_{t}(\jmath_{\wp}), \bfcc_{t}(\imath_{\wp})),\AND\\
      (\bfcc_{i}(\imath_{\PPm}), \bfcc_{i}(\jmath_{\PPm}) )
      & = (\bfcc_{i}(\imath_{\wp}), \bfcc_{i}(\jmath_{\wp}) ) \quad \text{when $i\neq t$}.
    \end{split}
  \]
  % For
  % $\uptau': = (\imath_{\wp},\cP')\times (\jmath_{\wp},\cQ')\in \PBPs(\tau_{\wp})$,
  For $\uptau\in\PBPs(\tau_{\wp})$ define $T_{\wp,\PPm}(\uptau) = \uptaum$  with
  $\uptaum$ given
  by the following formula
  % by the
  % \[
  %   T_{\wp,\PPm}(\uptau):=\uptau := (\imath_{\PPm},\cP)\times (\jmath_{\PPm},\cQ)\in \PBP_{\star}(\tau_{\PPm})
  % \]
  \[
    \begin{split}
      \text{$\forall (i,j)\in \BOX{\imath_{\PPm}}$,} \quad \cP_{\uptaum}(i,j) &= \begin{cases}
        s& \text{if $j=t$ and  $\cQ_{\uptau}(i,j)=r$,}\\
        c& \text{if $j=t$ and  $\cQ_{\uptau}(i,j)=d$,}\\
        \cP_{\uptau}(i,j) &\text{otherwise.}
      \end{cases}\\
      \text{$\forall (i,j)\in \BOX{\jmath_{\PPm}}$,} \quad   \cQ_{\uptaum}(i,j) &=  \cQ_{\uptau}(i,j).\\
    \end{split}
  \]
  We leave it to the reader to check that the above formula does define a valid
  painted bipartition $\uptaum$. Retain the above notation, it is easy to check that the
  inverse map
  \[
    T_{\PPm,\wp}\colon \PBP_{\star}(\tau_{\PPm}) \rightarrow \PBP_{\star}(\tau_{\wp})\quad \uptaum \mapsto \uptau
  \]
  is given by the following formula:
  \[
    \begin{split}
      \text{$\forall (i,j)\in \BOX{\imath_{\wp}}$,} \quad   \cP_{\uptau}(i,j) &=  \cP_{\uptaum}(i,j),\\
      \text{$\forall (i,j)\in \BOX{\jmath_{\wp}}$,} \quad \cQ_{\uptau}(i,j) &= \begin{cases}
        r& \text{if $j=t$ and  $\cP_{\uptaum}(i,j)=s$,}\\
        d& \text{if $j=t$ and  $\cP_{\uptaum}(i,j)=c$,}\\
        \cQ_{\uptaum}(i,j) &\text{otherwise.}
      \end{cases}
    \end{split}
  \]
  This finish the prove for the case when $\star=\wtC$. \medskip

  \def\ckcOa{\ckcO^{\scalebox{.6}{$+$}}}
  \def\sPa{\sP^{\scalebox{.6}{$+$}}}
  \def\sPam{\sP^{\scalebox{.6}{$+$}}_{\downarrow}}
  \def\uptaua{\uptau^{\scalebox{.6}{$+$}}}
  \def\tausPa{\tau_{\sPa}}
  \def\tausPam{\tau_{\sPam}}


  Now suppose $\star = B$. Let $\ckcOa$ be the nilpotent orbit of $\Sp(2n,\bC)$
  such that
  \[
    \bfrr_{1}(\ckcOa) = \bfrr_{1}(\ckcO)+2, \AND \bfrr_{i+1}(\ckcOa) = \bfrr_{i}(\ckcO)\quad \text{for
      all $i=1,2,3,\cdots,$}.
  \]
  Clearly
  \[
    \CPP(\ckcOa) = \set{(1,2)}\cup \set{(i+1,i+2)|(i,i+1)\in \CPP(\ckcO)}.
  \]

  For each subset $\sP\subset \CPP(\ckcO)$, let $\sPa$ denote the subset in
  $\CPP(\ckcOa)$ defined by
  \[
    \sPa:=\set{(i+1,i+2)|(i,i+1)\in \sP}.
  \]
  Let $r_{0}:= \half \bfrr_{1}(\ckcO)+1 = \bfcc_{1}(\imath_{\sPa})$. For $x=c$
  or $s$, define
  \[
    \PBP_{\wtC}^{x}(\tausPa):= \Set{\uptaua\in \PBP_{\wtC}(\tau_{\sPa})|\cP_{\uptaua}\left(r_{0},1\right)=x}
  \]
  where $\tausPa$ is the bipartition defined with respect to $\ckcOa$.

  It is easy to check that
  \begin{itemize}
    \item
    $\PBP_{\wtC}(\tausPa) = \PBP_{\wtC}^{c}(\tausPa)\sqcup \PBP_{\wtC}^{s}(\tausPa)$,
    \item the map $T_{\sPa,\sPam}$ restricted into a bijection from
    $\PBP_{\wtC}^{x}(\tausPa)$ onto $\PBP_{\wtC}^{x}(\tausPam)$ (here $x=s$ or
    $c$), and
    \item the descent map restricted to bijections
    \[
      \text{
          $\PBP_{\wtC}^{s}(\tau_{\sPa})\longrightarrow \PBP_{B^{+}}(\tau_{\sP})$
          and
          $\PBP_{\wtC}^{c}(\tau_{\sPa})\longrightarrow \PBP_{B^{-}}(\tau_{\sP})$.
      }
    \]
  \end{itemize}


  % Define $\wp'':=\set{(i+1,i+2)|(i,i+1)\in \wp}$ and
  % $\PPm'':=\set{(i+1,i+2)|(i,i+1)\in \wp}$.

\end{proof}

Suppose $\star = C$. Let
$(b_{1},b_{2}) = (\frac{\bfrr_{2t-1}(\ckcO)-1}{2},\frac{\bfrr_{2t}(\ckcO)+1}{2})$.
Then
\[
  \begin{split}
    (\bfcc_{t}(\imath_{\PPm}), \bfcc_{t}(\jmath_{\PPm})) &= (b_{2},b_{1}),  \\
    (\bfcc_{t}(\jmath_{\wp}), \bfcc_{t}(\imath_{\wp})) &= (b_{1}-1,b_{2}+1),\AND\\
    (\bfcc_{i}(\imath_{\PPm}),\bfcc_{i}(\imath_{\PPm})) &=(\bfcc_{i}(\imath_{\wp}),\bfcc_{i}(\imath_{\wp}))\quad \text{when
      $i\neq t$}.
  \end{split}
\]

For $\uptau\in \PBPs(\tau_{\PPm})$, we define $\uptau'$ by the following recipe:
\begin{align*}
  aa
\end{align*}

\end{document}

%%% Local Variables:
%%% mode: latex
%%% TeX-master: "counting_main"
%%% End:
