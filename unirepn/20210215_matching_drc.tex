\documentclass[12pt,a4paper]{amsart}
\usepackage[margin=2.5cm,marginpar=2cm]{geometry}



% \usepackage{showkeys}
% \makeatletter
%   \SK@def\Cref#1{\SK@\SK@@ref{#1}\SK@Cref{#1}}
%   \SK@def\cref#1{\SK@\SK@@ref{#1}\SK@cref{#1}}
% \makeatother

\usepackage[bookmarksopen,bookmarksdepth=3]{hyperref}
\usepackage[nameinlink]{cleveref}

\usepackage{array}
%% FONTS
\usepackage{amssymb}
%\usepackage{boisik}
%\usepackage{amsmath}
\usepackage{mathrsfs}
%\usepackage{amsrefs}
\usepackage{mathbbol,mathabx}
\usepackage{amsthm}
\usepackage{graphicx}
\usepackage{braket}
%\usepackage[pointedenum]{paralist}
%\usepackage{paralist}

\usepackage{mathtools}

\usepackage{amsrefs}

\usepackage[all,cmtip]{xy}
\usepackage{rotating}
\usepackage{leftidx}
%\usepackage{arydshln}

%\DeclareSymbolFont{bbold}{U}{bbold}{m}{n}
%\DeclareSymbolFontAlphabet{\mathbbold}{bbold}


%\usepackage[dvipdfx,rgb,table]{xcolor}
\usepackage[rgb,table,dvipsnames]{xcolor}
%\usepackage{color}
%\usepackage{mathrsfs}

\setcounter{tocdepth}{1}
\setcounter{secnumdepth}{3}

%\usepackage[abbrev,shortalphabetic]{amsrefs}


\usepackage{imakeidx}
\def\idxemph#1{\emph{#1}\index{#1}}
\makeindex


\usepackage[normalem]{ulem}

% circled number
\usepackage{pifont}
\makeatletter
\newcommand*{\circnuma}[1]{%
  \ifnum#1<1 %
    \@ctrerr
  \else
    \ifnum#1>20 %
      \@ctrerr
    \else
      \mbox{\ding{\numexpr 171+(#1)\relax}}%
     \fi
  \fi
}
\makeatother

\usepackage[centertableaux]{ytableau}

%\usepackage[mathlines,pagewise]{lineno}
%\linenumbers

\usepackage{enumitem}
%% Enumitem
\newlist{enumC}{enumerate}{1} % Conditions in Lemma/Theorem/Prop
\setlist[enumC,1]{label=(\alph*),wide,ref=(\alph*)}
\crefname{enumCi}{condition}{conditions}
\Crefname{enumCi}{Condition}{Conditions}
\newlist{enumT}{enumerate}{3} % "Theorem"=conclusions in Lemma/Theorem/Prop
\setlist[enumT]{label=(\roman*),wide}
\setlist[enumT,1]{label=(\roman*),wide}
\setlist[enumT,2]{label=(\alph*),ref ={(\roman{enumTi}.\alph*)},left=2em}
\setlist[enumT,3]{label*=.(\arabic*), ref ={(\roman{enumTi}.\alph{enumTii}.\alph*)}}
\crefname{enumTi}{}{}
\Crefname{enumTi}{Item}{Items}
\crefname{enumTii}{}{}
\Crefname{enumTii}{Item}{Items}
\crefname{enumTiii}{}{}
\Crefname{enumTiii}{Item}{Items}
\newlist{enumPF}{enumerate}{3}
%\setlist[enumPF]{label=(\alph*),wide}
\setlist[enumPF,1]{label=(\roman*),wide}
\setlist[enumPF,2]{label=(\alph*),left=2em}
\setlist[enumPF,3]{label=\arabic*).,left=1em}
\newlist{enumS}{enumerate}{3} % Statement outside Lemma/Theorem/Prop
\setlist[enumS]{label=\roman*)}
\setlist[enumS,1]{label=\roman*)}
\setlist[enumS,2]{label=\alph*)}
\setlist[enumS,3]{label=\arabic*.}
\newlist{enumI}{enumerate}{3} % items
\setlist[enumI,1]{label=\roman*),leftmargin=*}
\setlist[enumI,2]{label=\alph*), leftmargin=*}
\setlist[enumI,3]{label=\arabic*), leftmargin=*}
\newlist{enumIL}{enumerate*}{1} % inline enum
\setlist*[enumIL]{label=\roman*)}
\newlist{enumR}{enumerate}{1} % remarks
\setlist[enumR]{label=\arabic*.,wide,labelwidth=!, labelindent=0pt}
\crefname{enumRi}{remark}{remarks}


%\definecolor{srcol}{RGB}{255,255,51}

%\definecolor{srcol}{RGB}{255,255,51}
\colorlet{srcol}{black!15}

\crefname{equation}{}{}
\Crefname{equation}{Equation}{Equations}
\Crefname{lem}{Lemma}{Lemma}
\Crefname{thm}{Theorem}{Theorem}

\newlist{des}{enumerate}{1}
\setlist[des]{font=\upshape\sffamily\bfseries, label={}}
%\setlist[des]{before={\renewcommand\makelabel[1]{\sffamily \bfseries ##1 }}}

% editing macros.
%\blendcolors{!80!black}
\long\def\okay#1{\ifcsname highlightokay\endcsname
{\color{red} #1}
\else
{#1}
\fi
}
\long\def\editc#1{{\color{red} #1}}
\long\def\mjj#1{{{\color{blue}#1}}}
\long\def\mjjr#1{{\color{red} (#1)}}
\long\def\mjjd#1#2{{\color{blue} #1 \sout{#2}}}
\def\mjjb{\color{blue}}
\def\mjje{\color{black}}
\def\mjjcb{\color{green!50!black}}
\def\mjjce{\color{black}}

\long\def\sun#1{{{\color{cyan}#1}}}
\long\def\sund#1#2{{\color{cyan}#1  \sout{#2}}}
\long\def\mv#1{{{\color{red} {\bf move to a proper place:} #1}}}
\long\def\delete#1{}

%\reversemarginpar
\newcommand{\lokec}[1]{\marginpar{\color{blue}\tiny #1 \mbox{--loke}}}
\newcommand{\mjjc}[1]{\marginpar{\color{green}\tiny #1 \mbox{--ma}}}


\def\showtrivial{\relax}

\newcommand{\trivial}[2][]{\if\relax\detokenize{#1}\relax
  {%\hfill\break
   % \begin{minipage}{\textwidth}
      \color{orange} \vspace{0em} $[$  #2 $]$
  %\end{minipage}
  %\break
      \color{black}
  }
  \else
\ifx#1h
\ifcsname showtrivial\endcsname
{%\hfill\break
 % \begin{minipage}{\textwidth}
    \color{orange} \vspace{0em}  $[$ #2 $]$
%\end{minipage}
%\break
    \color{black}
}
\fi
\else {\red Wrong argument!} \fi
\fi
}

\newcommand{\byhide}[2][]{\if\relax\detokenize{#1}\relax
{\color{orange} \vspace{0em} Plan to delete:  #2}
\else
\ifx#1h\relax\fi
\fi
}



\newcommand{\Rank}{\mathrm{rk}}
\newcommand{\cqq}{\mathscr{D}}
\newcommand{\rsym}{\mathrm{sym}}
\newcommand{\rskew}{\mathrm{skew}}
\newcommand{\fraksp}{\mathfrak{sp}}
\newcommand{\frakso}{\mathfrak{so}}
\newcommand{\frakm}{\mathfrak{m}}
\newcommand{\frakp}{\mathfrak{p}}
\newcommand{\pr}{\mathrm{pr}}
\newcommand{\rhopst}{\rho'^*}
\newcommand{\Rad}{\mathrm{Rad}}
\newcommand{\Res}{\mathrm{Res}}
\newcommand{\Hol}{\mathrm{Hol}}
\newcommand{\AC}{\mathrm{AC}}
%\newcommand{\AS}{\mathrm{AS}}
\newcommand{\WF}{\mathrm{WF}}
\newcommand{\AV}{\mathrm{AV}}
\newcommand{\AVC}{\mathrm{AV}_\bC}
\newcommand{\VC}{\mathrm{V}_\bC}
\newcommand{\bfv}{\mathbf{v}}
\newcommand{\depth}{\mathrm{depth}}
\newcommand{\wtM}{\widetilde{M}}
\newcommand{\wtMone}{{\widetilde{M}^{(1,1)}}}

\newcommand{\nullpp}{N(\fpp'^*)}
\newcommand{\nullp}{N(\fpp^*)}
%\newcommand{\Aut}{\mathrm{Aut}}

\def\mstar{{\medstar}}
\def\YD{{\mathsf{YD}}}
\def\SYD{{\mathsf{SYD}}}
\def\MYD{{\mathsf{MYD}}}

\def\KM{{\mathcal{K_{\mathsf{M}}}}}

\newcommand{\bfonenp}{\mathbf{1}^{-,+}}
\newcommand{\bfonepn}{\mathbf{1}^{+,-}}
\newcommand{\bfone}{\mathbf{1}}
\newcommand{\piSigma}{\pi_\Sigma}
\newcommand{\piSigmap}{\pi'_\Sigma}


\newcommand{\sfVprime}{\mathsf{V}^\prime}
\newcommand{\sfVdprime}{\mathsf{V}^{\prime \prime}}
\newcommand{\gminusone}{\mathfrak{g}_{-\frac{1}{m}}}

\newcommand{\eva}{\mathrm{eva}}

% \newcommand\iso{\xrightarrow{
%    \,\smash{\raisebox{-0.65ex}{\ensuremath{\scriptstyle\sim}}}\,}}

\def\Ueven{{U_{\rm{even}}}}
\def\Uodd{{U_{\rm{odd}}}}
\def\ttau{\tilde{\tau}}
\def\Wcp{W}
\def\Kur{{K^{\mathrm{u}}}}

\def\Im{\operatorname{Im}}


\providecommand{\bcN}{{\overline{\cN}}}



\makeatletter

\def\gen#1{\left\langle
    #1
      \right\rangle}
\makeatother

\makeatletter
\def\inn#1#2{\left\langle
      \def\ta{#1}\def\tb{#2}
      \ifx\ta\@empty{\;} \else {\ta}\fi ,
      \ifx\tb\@empty{\;} \else {\tb}\fi
      \right\rangle}
\def\binn#1#2{\left\lAngle
      \def\ta{#1}\def\tb{#2}
      \ifx\ta\@empty{\;} \else {\ta}\fi ,
      \ifx\tb\@empty{\;} \else {\tb}\fi
      \right\rAngle}
\makeatother

\makeatletter
\def\binn#1#2{\overline{\inn{#1}{#2}}}
\makeatother


\def\innwi#1#2{\inn{#1}{#2}_{W_i}}
\def\innw#1#2{\inn{#1}{#2}_{\bfW}}
\def\innv#1#2{\inn{#1}{#2}_{\bfV}}
\def\innbfv#1#2{\inn{#1}{#2}_{\bfV}}
\def\innvi#1#2{\inn{#1}{#2}_{V_i}}
\def\innvp#1#2{\inn{#1}{#2}_{\bfV'}}
\def\innp#1#2{\inn{#1}{#2}'}

% choose one of then
\def\simrightarrow{\iso}
\def\surj{\twoheadrightarrow}
%\def\simrightarrow{\xrightarrow{\sim}}

\newcommand\iso{\xrightarrow{
   \,\smash{\raisebox{-0.65ex}{\ensuremath{\scriptstyle\sim}}}\,}}

\newcommand\riso{\xleftarrow{
   \,\smash{\raisebox{-0.65ex}{\ensuremath{\scriptstyle\sim}}}\,}}









\usepackage{xparse}
\def\usecsname#1{\csname #1\endcsname}
\def\useLetter#1{#1}
\def\usedbletter#1{#1#1}

% \def\useCSf#1{\csname f#1\endcsname}

\ExplSyntaxOn

\def\mydefcirc#1#2#3{\expandafter\def\csname
  circ#3{#1}\endcsname{{}^\circ {#2{#1}}}}
\def\mydefvec#1#2#3{\expandafter\def\csname
  vec#3{#1}\endcsname{\vec{#2{#1}}}}
\def\mydefdot#1#2#3{\expandafter\def\csname
  dot#3{#1}\endcsname{\dot{#2{#1}}}}

\def\mydefacute#1#2#3{\expandafter\def\csname a#3{#1}\endcsname{\acute{#2{#1}}}}
\def\mydefbr#1#2#3{\expandafter\def\csname br#3{#1}\endcsname{\breve{#2{#1}}}}
\def\mydefbar#1#2#3{\expandafter\def\csname bar#3{#1}\endcsname{\bar{#2{#1}}}}
\def\mydefhat#1#2#3{\expandafter\def\csname hat#3{#1}\endcsname{\hat{#2{#1}}}}
\def\mydefwh#1#2#3{\expandafter\def\csname wh#3{#1}\endcsname{\widehat{#2{#1}}}}
\def\mydeft#1#2#3{\expandafter\def\csname t#3{#1}\endcsname{\tilde{#2{#1}}}}
\def\mydefu#1#2#3{\expandafter\def\csname u#3{#1}\endcsname{\underline{#2{#1}}}}
\def\mydefr#1#2#3{\expandafter\def\csname r#3{#1}\endcsname{\mathrm{#2{#1}}}}
\def\mydefb#1#2#3{\expandafter\def\csname b#3{#1}\endcsname{\mathbb{#2{#1}}}}
\def\mydefwt#1#2#3{\expandafter\def\csname wt#3{#1}\endcsname{\widetilde{#2{#1}}}}
%\def\mydeff#1#2#3{\expandafter\def\csname f#3{#1}\endcsname{\mathfrak{#2{#1}}}}
\def\mydefbf#1#2#3{\expandafter\def\csname bf#3{#1}\endcsname{\mathbf{#2{#1}}}}
\def\mydefc#1#2#3{\expandafter\def\csname c#3{#1}\endcsname{\mathcal{#2{#1}}}}
\def\mydefsf#1#2#3{\expandafter\def\csname sf#3{#1}\endcsname{\mathsf{#2{#1}}}}
\def\mydefs#1#2#3{\expandafter\def\csname s#3{#1}\endcsname{\mathscr{#2{#1}}}}
\def\mydefcks#1#2#3{\expandafter\def\csname cks#3{#1}\endcsname{{\check{
        \csname s#2{#1}\endcsname}}}}
\def\mydefckc#1#2#3{\expandafter\def\csname ckc#3{#1}\endcsname{{\check{
      \csname c#2{#1}\endcsname}}}}
\def\mydefck#1#2#3{\expandafter\def\csname ck#3{#1}\endcsname{{\check{#2{#1}}}}}

\cs_new:Npn \mydeff #1#2#3 {\cs_new:cpn {f#3{#1}} {\mathfrak{#2{#1}}}}

\cs_new:Npn \doGreek #1
{
  \clist_map_inline:nn {alpha,beta,gamma,Gamma,delta,Delta,epsilon,varepsilon,zeta,eta,theta,vartheta,Theta,iota,kappa,lambda,Lambda,mu,nu,xi,Xi,pi,Pi,rho,sigma,varsigma,Sigma,tau,upsilon,Upsilon,phi,varphi,Phi,chi,psi,Psi,omega,Omega,tG} {#1{##1}{\usecsname}{\useLetter}}
}

\cs_new:Npn \doSymbols #1
{
  \clist_map_inline:nn {otimes,boxtimes} {#1{##1}{\usecsname}{\useLetter}}
}

\cs_new:Npn \doAtZ #1
{
  \clist_map_inline:nn {A,B,C,D,E,F,G,H,I,J,K,L,M,N,O,P,Q,R,S,T,U,V,W,X,Y,Z} {#1{##1}{\useLetter}{\useLetter}}
}

\cs_new:Npn \doatz #1
{
  \clist_map_inline:nn {a,b,c,d,e,f,g,h,i,j,k,l,m,n,o,p,q,r,s,t,u,v,w,x,y,z} {#1{##1}{\useLetter}{\usedbletter}}
}

\cs_new:Npn \doallAtZ
{
\clist_map_inline:nn {mydefsf,mydeft,mydefu,mydefwh,mydefhat,mydefr,mydefwt,mydeff,mydefb,mydefbf,mydefc,mydefs,mydefck,mydefcks,mydefckc,mydefbar,mydefvec,mydefcirc,mydefdot,mydefbr,mydefacute} {\doAtZ{\csname ##1\endcsname}}
}

\cs_new:Npn \doallatz
{
\clist_map_inline:nn {mydefsf,mydeft,mydefu,mydefwh,mydefhat,mydefr,mydefwt,mydeff,mydefb,mydefbf,mydefc,mydefs,mydefck,mydefbar,mydefvec,mydefdot,mydefbr,mydefacute} {\doatz{\csname ##1\endcsname}}
}

\cs_new:Npn \doallGreek
{
\clist_map_inline:nn {mydefck,mydefwt,mydeft,mydefwh,mydefbar,mydefu,mydefvec,mydefcirc,mydefdot,mydefbr,mydefacute} {\doGreek{\csname ##1\endcsname}}
}

\cs_new:Npn \doallSymbols
{
\clist_map_inline:nn {mydefck,mydefwt,mydeft,mydefwh,mydefbar,mydefu,mydefvec,mydefcirc,mydefdot} {\doSymbols{\csname ##1\endcsname}}
}



\cs_new:Npn \doGroups #1
{
  \clist_map_inline:nn {GL,Sp,rO,rU,fgl,fsp,foo,fuu,fkk,fuu,ufkk,uK} {#1{##1}{\usecsname}{\useLetter}}
}

\cs_new:Npn \doallGroups
{
\clist_map_inline:nn {mydeft,mydefu,mydefwh,mydefhat,mydefwt,mydefck,mydefbar} {\doGroups{\csname ##1\endcsname}}
}


\cs_new:Npn \decsyms #1
{
\clist_map_inline:nn {#1} {\expandafter\DeclareMathOperator\csname ##1\endcsname{##1}}
}

\decsyms{Mp,id,SL,Sp,SU,SO,GO,GSO,GU,GSp,PGL,Pic,Lie,Mat,Ker,Hom,Ext,Ind,reg,res,inv,Isom,Det,Tr,Norm,Sym,Span,Stab,Spec,PGSp,PSL,tr,Ad,Br,Ch,Cent,End,Aut,Dvi,Frob,Gal,GL,Gr,DO,ur,vol,ab,Nil,Supp,rank,Sign}

\def\abs#1{\left|{#1}\right|}
\def\norm#1{{\left\|{#1}\right\|}}


% \NewDocumentCommand\inn{m m}{
% \left\langle
% \IfValueTF{#1}{#1}{000}
% ,
% \IfValueTF{#2}{#2}{000}
% \right\rangle
% }
\NewDocumentCommand\cent{o m }{
  \IfValueTF{#1}{
    \mathop{Z}_{#1}{(#2)}}
  {\mathop{Z}{(#2)}}
}


\def\fsl{\mathfrak{sl}}
\def\fsp{\mathfrak{sp}}


%\def\cent#1#2{{\mathrm{Z}_{#1}({#2})}}


\doallAtZ
\doallatz
\doallGreek
\doallGroups
\doallSymbols
\ExplSyntaxOff


% \usepackage{geometry,amsthm,graphics,tabularx,amssymb,shapepar}
% \usepackage{amscd}
% \usepackage{mathrsfs}


\usepackage{diagbox}
% Update the information and uncomment if AMS is not the copyright
% holder.
%\copyrightinfo{2006}{American Mathematical Society}
%\usepackage{nicematrix}
\usepackage{arydshln}
\usepackage[mode=buildnew]{standalone}% requires -shell-escape

\usepackage{tikz,etoolbox}
\usetikzlibrary{matrix,arrows,positioning,backgrounds}
\usetikzlibrary{decorations.pathmorphing,decorations.pathreplacing}
\usetikzlibrary{cd}
% \usetikzlibrary{external}
%   \tikzexternalize
% \usetikzlibrary{cd}

%  \AtBeginEnvironment{tikzcd}{\tikzexternaldisable}
%  \AtEndEnvironment{tikzcd}{\tikzexternalenable}

%  \usetikzlibrary{matrix,arrows,positioning,backgrounds}
%  \usetikzlibrary{decorations.pathmorphing,decorations.pathreplacing}

% % externalization not work properly
% % \usetikzlibrary{external}
% \tikzexternalize[prefix=figures/]
% % % activate the following such that you can check the macro expansion in
% % % *-figure0.md5 manually
% %\tikzset{external/up to date check=diff}
% \usepackage{environ}

% \def\temp{&} \catcode`&=\active \let&=\temp

% \newcommand{\mytikzcdcontext}[2]{
%   \begin{tikzpicture}[baseline=(maintikzcdnode.base)]
%     \node (maintikzcdnode) [inner sep=0, outer sep=0] {\begin{tikzcd}[#2]
%         #1
%     \end{tikzcd}};
%   \end{tikzpicture}}

% \NewEnviron{mytikzcd}[1][]{%
% % In the following, we need \BODY to expanded before \mytikzcdcontext
% % such that the md5 function gets the tikzcd content with \BODY expanded.
% % Howerver, expand it only once, because the \tikz-macros aren't
% % defined at this point yet. The same thing holds for the arguments to
% % the tikzcd-environment.
% \def\myargs{#1}%
% \edef\mydiagram{\noexpand\mytikzcdcontext{\expandonce\BODY}{\expandonce\myargs}}%
% \mydiagram%
% }

\usepackage{upgreek}

\usepackage{listings}
\lstset{
    basicstyle=\ttfamily\tiny,
    keywordstyle=\color{black},
    commentstyle=\color{white}, % white comments
    stringstyle=\ttfamily, % typewriter type for strings
    showstringspaces=false,
    breaklines=true,
    emph={Output},emphstyle=\color{blue},
} 

\newcommand{\BA}{{\mathbb{A}}}
%\newcommand{\BB}{{\mathbb {B}}}
\newcommand{\BC}{{\mathbb {C}}}
\newcommand{\BD}{{\mathbb {D}}}
\newcommand{\BE}{{\mathbb {E}}}
\newcommand{\BF}{{\mathbb {F}}}
\newcommand{\BG}{{\mathbb {G}}}
\newcommand{\BH}{{\mathbb {H}}}
\newcommand{\BI}{{\mathbb {I}}}
\newcommand{\BJ}{{\mathbb {J}}}
\newcommand{\BK}{{\mathbb {U}}}
\newcommand{\BL}{{\mathbb {L}}}
\newcommand{\BM}{{\mathbb {M}}}
\newcommand{\BN}{{\mathbb {N}}}
\newcommand{\BO}{{\mathbb {O}}}
\newcommand{\BP}{{\mathbb {P}}}
\newcommand{\BQ}{{\mathbb {Q}}}
\newcommand{\BR}{{\mathbb {R}}}
\newcommand{\BS}{{\mathbb {S}}}
\newcommand{\BT}{{\mathbb {T}}}
\newcommand{\BU}{{\mathbb {U}}}
\newcommand{\BV}{{\mathbb {V}}}
\newcommand{\BW}{{\mathbb {W}}}
\newcommand{\BX}{{\mathbb {X}}}
\newcommand{\BY}{{\mathbb {Y}}}
\newcommand{\BZ}{{\mathbb {Z}}}
\newcommand{\Bk}{{\mathbf {k}}}

\newcommand{\CA}{{\mathcal {A}}}
\newcommand{\CB}{{\mathcal {B}}}
\newcommand{\CC}{{\mathcal {C}}}

\newcommand{\CE}{{\mathcal {E}}}
\newcommand{\CF}{{\mathcal {F}}}
\newcommand{\CG}{{\mathcal {G}}}
\newcommand{\CH}{{\mathcal {H}}}
\newcommand{\CI}{{\mathcal {I}}}
\newcommand{\CJ}{{\mathcal {J}}}
\newcommand{\CK}{{\mathcal {K}}}
\newcommand{\CL}{{\mathcal {L}}}
\newcommand{\CM}{{\mathcal {M}}}
\newcommand{\CN}{{\mathcal {N}}}
\newcommand{\CO}{{\mathcal {O}}}
\newcommand{\CP}{{\mathcal {P}}}
\newcommand{\CQ}{{\mathcal {Q}}}
\newcommand{\CR}{{\mathcal {R}}}
\newcommand{\CS}{{\mathcal {S}}}
\newcommand{\CT}{{\mathcal {T}}}
\newcommand{\CU}{{\mathcal {U}}}
\newcommand{\CV}{{\mathcal {V}}}
\newcommand{\CW}{{\mathcal {W}}}
\newcommand{\CX}{{\mathcal {X}}}
\newcommand{\CY}{{\mathcal {Y}}}
\newcommand{\CZ}{{\mathcal {Z}}}


\newcommand{\RA}{{\mathrm {A}}}
\newcommand{\RB}{{\mathrm {B}}}
\newcommand{\RC}{{\mathrm {C}}}
\newcommand{\RD}{{\mathrm {D}}}
\newcommand{\RE}{{\mathrm {E}}}
\newcommand{\RF}{{\mathrm {F}}}
\newcommand{\RG}{{\mathrm {G}}}
\newcommand{\RH}{{\mathrm {H}}}
\newcommand{\RI}{{\mathrm {I}}}
\newcommand{\RJ}{{\mathrm {J}}}
\newcommand{\RK}{{\mathrm {K}}}
\newcommand{\RL}{{\mathrm {L}}}
\newcommand{\RM}{{\mathrm {M}}}
\newcommand{\RN}{{\mathrm {N}}}
\newcommand{\RO}{{\mathrm {O}}}
\newcommand{\RP}{{\mathrm {P}}}
\newcommand{\RQ}{{\mathrm {Q}}}
%\newcommand{\RR}{{\mathrm {R}}}
\newcommand{\RS}{{\mathrm {S}}}
\newcommand{\RT}{{\mathrm {T}}}
\newcommand{\RU}{{\mathrm {U}}}
\newcommand{\RV}{{\mathrm {V}}}
\newcommand{\RW}{{\mathrm {W}}}
\newcommand{\RX}{{\mathrm {X}}}
\newcommand{\RY}{{\mathrm {Y}}}
\newcommand{\RZ}{{\mathrm {Z}}}

\DeclareMathOperator{\absNorm}{\mathfrak{N}}
\DeclareMathOperator{\Ann}{Ann}
\DeclareMathOperator{\LAnn}{L-Ann}
\DeclareMathOperator{\RAnn}{R-Ann}
\DeclareMathOperator{\ind}{ind}
%\DeclareMathOperator{\Ind}{Ind}



\def\ckbfG{\check{\bfG}}

\newcommand{\cod}{{\mathrm{cod}}}
\newcommand{\cont}{{\mathrm{cont}}}
\newcommand{\cl}{{\mathrm{cl}}}
\newcommand{\cusp}{{\mathrm{cusp}}}

\newcommand{\disc}{{\mathrm{disc}}}



\newcommand{\Gm}{{\mathbb{G}_m}}



\newcommand{\I}{{\mathrm{I}}}

\newcommand{\Jac}{{\mathrm{Jac}}}
\newcommand{\PM}{{\mathrm{PM}}}


\newcommand{\new}{{\mathrm{new}}}
\newcommand{\NS}{{\mathrm{NS}}}
\newcommand{\N}{{\mathrm{N}}}

\newcommand{\ord}{{\mathrm{ord}}}

%\newcommand{\rank}{{\mathrm{rank}}}

\newcommand{\rk}{{\mathrm{k}}}
\newcommand{\rr}{{\mathrm{r}}}
\newcommand{\rh}{{\mathrm{h}}}

\newcommand{\Sel}{{\mathrm{Sel}}}
\newcommand{\Sim}{{\mathrm{Sim}}}

\newcommand{\wt}{\widetilde}
\newcommand{\wh}{\widehat}
\newcommand{\pp}{\frac{\partial\bar\partial}{\pi i}}
\newcommand{\pair}[1]{\langle {#1} \rangle}
\newcommand{\wpair}[1]{\left\{{#1}\right\}}
\newcommand{\intn}[1]{\left( {#1} \right)}
\newcommand{\sfrac}[2]{\left( \frac {#1}{#2}\right)}
\newcommand{\ds}{\displaystyle}
\newcommand{\ov}{\overline}
\newcommand{\incl}{\hookrightarrow}
\newcommand{\lra}{\longrightarrow}
\newcommand{\imp}{\Longrightarrow}
%\newcommand{\lto}{\longmapsto}
\newcommand{\bs}{\backslash}

\newcommand{\cover}[1]{\widetilde{#1}}

\renewcommand{\vsp}{{\vspace{0.2in}}}

\newcommand{\Norma}{\operatorname{N}}
\newcommand{\Ima}{\operatorname{Im}}
\newcommand{\con}{\textit{C}}
\newcommand{\gr}{\operatorname{gr}}
\newcommand{\ad}{\operatorname{ad}}
\newcommand{\der}{\operatorname{der}}
\newcommand{\dif}{\operatorname{d}\!}
\newcommand{\pro}{\operatorname{pro}}
\newcommand{\Ev}{\operatorname{Ev}}
% \renewcommand{\span}{\operatorname{span}} \span is an innernal command.
%\newcommand{\degree}{\operatorname{deg}}
\newcommand{\Invf}{\operatorname{Invf}}
\newcommand{\Inv}{\operatorname{Inv}}
\newcommand{\slt}{\operatorname{SL}_2(\mathbb{R})}
%\newcommand{\temp}{\operatorname{temp}}
%\newcommand{\otop}{\operatorname{top}}
%\renewcommand{\small}{\operatorname{small}}
\newcommand{\HC}{\operatorname{HC}}
\newcommand{\lef}{\operatorname{left}}
\newcommand{\righ}{\operatorname{right}}
\newcommand{\Diff}{\operatorname{DO}}
\newcommand{\diag}{\operatorname{diag}}
\newcommand{\sh}{\varsigma}
\newcommand{\sch}{\operatorname{sch}}
%\newcommand{\oleft}{\operatorname{left}}
%\newcommand{\oright}{\operatorname{right}}
\newcommand{\open}{\operatorname{open}}
\newcommand{\sgn}{\operatorname{sgn}}
\newcommand{\triv}{\operatorname{triv}}
\newcommand{\Sh}{\operatorname{Sh}}
\newcommand{\oN}{\operatorname{N}}

\newcommand{\oc}{\operatorname{c}}
\newcommand{\od}{\operatorname{d}}
\newcommand{\os}{\operatorname{s}}
\newcommand{\ol}{\operatorname{l}}
\newcommand{\oL}{\operatorname{L}}
\newcommand{\oJ}{\operatorname{J}}
\newcommand{\oH}{\operatorname{H}}
\newcommand{\oO}{\operatorname{O}}
\newcommand{\oS}{\operatorname{S}}
\newcommand{\oR}{\operatorname{R}}
\newcommand{\oT}{\operatorname{T}}
%\newcommand{\rU}{\operatorname{U}}
\newcommand{\oZ}{\operatorname{Z}}
\newcommand{\oD}{\textit{D}}
\newcommand{\oW}{\textit{W}}
\newcommand{\oE}{\operatorname{E}}
\newcommand{\oP}{\operatorname{P}}
\newcommand{\PD}{\operatorname{PD}}
\newcommand{\oU}{\operatorname{U}}

\newcommand{\gC}{{\mathfrak g}_{\C}}
%\renewcommand{\sl}{\mathfrak s \mathfrak l}
\newcommand{\gl}{\mathfrak g \mathfrak l}


\newcommand{\re}{\mathrm e}

\renewcommand{\rk}{\mathrm k}

\newcommand{\Z}{\mathbb{Z}}
\DeclareDocumentCommand{\C}{}{\mathbb{C}}
\newcommand{\R}{\mathbb R}
\newcommand{\Q}{\mathbb Q}
\renewcommand{\H}{\mathbb{H}}
%\newcommand{\N}{\mathbb{N}}
\newcommand{\K}{\mathbb{K}}
%\renewcommand{\S}{\mathbf S}
\newcommand{\M}{\mathbf{M}}
\newcommand{\A}{\mathbb{A}}
\newcommand{\B}{\mathbf{B}}
%\renewcommand{\G}{\mathbf{G}}
\newcommand{\V}{\mathbf{V}}
\newcommand{\W}{\mathbf{W}}
\newcommand{\F}{\mathbf{F}}
\newcommand{\E}{\mathbf{E}}
%\newcommand{\J}{\mathbf{J}}
\renewcommand{\H}{\mathbf{H}}
\newcommand{\X}{\mathbf{X}}
\newcommand{\Y}{\mathbf{Y}}
%\newcommand{\RR}{\mathcal R}
\newcommand{\FF}{\mathcal F}
%\newcommand{\BB}{\mathcal B}
\newcommand{\HH}{\mathcal H}
%\newcommand{\UU}{\mathcal U}
%\newcommand{\MM}{\mathcal M}
%\newcommand{\CC}{\mathcal C}
%\newcommand{\DD}{\mathcal D}
%\def\eDD{\mathrm{d}^{e}}
%\def\eDD{\bigtriangledown}
\def\eDD{\overline{\nabla}}
\def\eDDo{\overline{\nabla}_1}
%\def\eDD{\mathrm{d}}
\def\DD{\nabla}
\def\DDc{\boldsymbol{\nabla}}
\def\gDD{\nabla^{\mathrm{gen}}}
\def\gDDc{\boldsymbol{\nabla}^{\mathrm{gen}}}
%\newcommand{\OO}{\mathcal O}
%\newcommand{\ZZ}{\mathcal Z}
\newcommand{\ve}{{\vee}}
\newcommand{\aut}{\mathcal A}
\newcommand{\ii}{\mathbf{i}}
\newcommand{\jj}{\mathbf{j}}
\newcommand{\kk}{\mathbf{k}}

\newcommand{\la}{\langle}
\newcommand{\ra}{\rangle}
\newcommand{\bp}{\bigskip}
\newcommand{\be}{\begin {equation}}
\newcommand{\ee}{\end {equation}}

\newcommand{\LRleq}{\stackrel{LR}{\leq}}

\numberwithin{equation}{section}


\def\flushl#1{\ifmmode\makebox[0pt][l]{${#1}$}\else\makebox[0pt][l]{#1}\fi}
\def\flushr#1{\ifmmode\makebox[0pt][r]{${#1}$}\else\makebox[0pt][r]{#1}\fi}
\def\flushmr#1{\makebox[0pt][r]{${#1}$}}


%\theoremstyle{Theorem}
% \newtheorem*{thmM}{Main Theorem}
% \crefformat{thmM}{main theorem}
% \Crefformat{thmM}{Main Theorem}
\newtheorem*{thm*}{Theorem}
\newtheorem{thm}{Theorem}[section]
\newtheorem{thml}[thm]{Theorem}
\newtheorem{lem}[thm]{Lemma}
\newtheorem{obs}[thm]{Observation}
\newtheorem{lemt}[thm]{Lemma}
\newtheorem*{lem*}{Lemma}
\newtheorem{whyp}[thm]{Working Hypothesis}
\newtheorem{prop}[thm]{Proposition}
\newtheorem{prpt}[thm]{Proposition}
\newtheorem{prpl}[thm]{Proposition}
\newtheorem{cor}[thm]{Corollary}
%\newtheorem*{prop*}{Proposition}
\newtheorem{claim}[thm]{Claim}
\newtheorem*{claim*}{Claim}
%\theoremstyle{definition}
\newtheorem{defn}[thm]{Definition}
\newtheorem{dfnl}[thm]{Definition}
\newtheorem*{IndH}{Induction Hypothesis}

\newtheorem*{eg*}{Example}
\newtheorem{eg}[thm]{Example}

\theoremstyle{remark}
\newtheorem*{remark}{Remark}
\newtheorem*{remarks}{Remarks}


\def\cpc{\sigma}
\def\ccJ{\epsilon\dotepsilon}
\def\ccL{c_L}

\def\wtbfK{\widetilde{\bfK}}
%\def\abfV{\acute{\bfV}}
\def\AbfV{\acute{\bfV}}
%\def\afgg{\acute{\fgg}}
%\def\abfG{\acute{\bfG}}
\def\abfV{\bfV'}
\def\afgg{\fgg'}
\def\abfG{\bfG'}

\def\half{{\tfrac{1}{2}}}
\def\ihalf{{\tfrac{\mathbf i}{2}}}
\def\slt{\fsl_2(\bC)}
\def\sltr{\fsl_2(\bR)}

% \def\Jslt{{J_{\fslt}}}
% \def\Lslt{{L_{\fslt}}}
\def\slee{{
\begin{pmatrix}
 0 & 1\\
 0 & 0
\end{pmatrix}
}}
\def\slff{{
\begin{pmatrix}
 0 & 0\\
 1 & 0
\end{pmatrix}
}}\def\slhh{{
\begin{pmatrix}
 1 & 0\\
 0 & -1
\end{pmatrix}
}}
\def\sleei{{
\begin{pmatrix}
 0 & i\\
 0 & 0
\end{pmatrix}
}}
\def\slxx{{\begin{pmatrix}
-\ihalf & \half\\
\phantom{-}\half & \ihalf
\end{pmatrix}}}
% \def\slxx{{\begin{pmatrix}
% -\sqrt{-1}/2 & 1/2\\
% 1/2 & \sqrt{-1}/2
% \end{pmatrix}}}
\def\slyy{{\begin{pmatrix}
\ihalf & \half\\
\half & -\ihalf
\end{pmatrix}}}
\def\slxxi{{\begin{pmatrix}
+\half & -\ihalf\\
-\ihalf & -\half
\end{pmatrix}}}
\def\slH{{\begin{pmatrix}
   0   & -\mathbf i\\
\mathbf i & 0
\end{pmatrix}}
}

\ExplSyntaxOn
\clist_map_inline:nn {J,L,C,X,Y,H,c,e,f,h,}{
  \expandafter\def\csname #1slt\endcsname{{\mathring{#1}}}}
\ExplSyntaxOff


\def\Mop{\fT}

\def\fggJ{\fgg_J}
\def\fggJp{\fgg'_{J'}}

\def\NilGC{\Nil_{\bfG}(\fgg)}
\def\NilGCp{\Nil_{\bfG'}(\fgg')}
\def\Nilgp{\Nil_{\fgg'_{J'}}}
\def\Nilg{\Nil_{\fgg_{J}}}
%\def\NilP'{\Nil_{\fpp'}}
\def\peNil{\Nil^{\mathrm{pe}}}
\def\dpeNil{\Nil^{\mathrm{dpe}}}
\def\nNil{\Nil^{\mathrm n}}
\def\eNil{\Nil^{\mathrm e}}


\NewDocumentCommand{\NilP}{t'}{
\IfBooleanTF{#1}{\Nil_{\fpp'}}{\Nil_\fpp}
}

\def\KS{\mathsf{KS}}
\def\MM{\bfM}
\def\MMP{M}

\NewDocumentCommand{\KTW}{o g}{
  \IfValueTF{#2}{
    \left.\varsigma_{\IfValueT{#1}{#1}}\right|_{#2}}{
    \varsigma_{\IfValueT{#1}{#1}}}
}
\def\IST{\rho}
\def\tIST{\trho}

\NewDocumentCommand{\CHI}{o g}{
  \IfValueTF{#1}{
    {\chi}_{\left[#1\right]}}{
    \IfValueTF{#2}{
      {\chi}_{\left(#2\right)}}{
      {\chi}}
  }
}
\NewDocumentCommand{\PR}{g}{
  \IfValueTF{#1}{
    \mathop{\pr}_{\left(#1\right)}}{
    \mathop{\pr}}
}
\NewDocumentCommand{\XX}{g}{
  \IfValueTF{#1}{
    {\cX}_{\left(#1\right)}}{
    {\cX}}
}
\NewDocumentCommand{\PP}{g}{
  \IfValueTF{#1}{
    {\fpp}_{\left(#1\right)}}{
    {\fpp}}
}
\NewDocumentCommand{\LL}{g}{
  \IfValueTF{#1}{
    {\bfL}_{\left(#1\right)}}{
    {\bfL}}
}
\NewDocumentCommand{\ZZ}{g}{
  \IfValueTF{#1}{
    {\cZ}_{\left(#1\right)}}{
    {\cZ}}
}

\NewDocumentCommand{\WW}{g}{
  \IfValueTF{#1}{
    {\bfW}_{\left(#1\right)}}{
    {\bfW}}
}




\def\gpi{\wp}
\NewDocumentCommand\KK{g}{
\IfValueTF{#1}{K_{(#1)}}{K}}
% \NewDocumentCommand\OO{g}{
% \IfValueTF{#1}{\cO_{(#1)}}{K}}
\NewDocumentCommand\XXo{d()}{
\IfValueTF{#1}{\cX^\circ_{(#1)}}{\cX^\circ}}
\def\bfWo{\bfW^\circ}
\def\bfWoo{\bfW^{\circ \circ}}
\def\bfWg{\bfW^{\mathrm{gen}}}
\def\Xg{\cX^{\mathrm{gen}}}
\def\Xo{\cX^\circ}
\def\Xoo{\cX^{\circ \circ}}
\def\fppo{\fpp^\circ}
\def\fggo{\fgg^\circ}
\NewDocumentCommand\ZZo{g}{
\IfValueTF{#1}{\cZ^\circ_{(#1)}}{\cZ^\circ}}

% \ExplSyntaxOn
% \NewDocumentCommand{\bcO}{t' E{^_}{{}{}}}{
%   \overline{\cO\sb{\use_ii:nn#2}\IfBooleanTF{#1}{^{'\use_i:nn#2}}{^{\use_i:nn#2}}
%   }
% }
% \ExplSyntaxOff

\NewDocumentCommand{\bcO}{t'}{
  \overline{\cO\IfBooleanT{#1}{'}}}

\NewDocumentCommand{\oliftc}{g}{
\IfValueTF{#1}{\boldsymbol{\vartheta} (#1)}{\boldsymbol{\vartheta}}
}
\NewDocumentCommand{\oliftr}{g}{
\IfValueTF{#1}{\vartheta_\bR(#1)}{\vartheta_\bR}
}
\NewDocumentCommand{\olift}{g}{
\IfValueTF{#1}{\vartheta(#1)}{\vartheta}
}
% \NewDocumentCommand{\dliftv}{g}{
% \IfValueTF{#1}{\ckvartheta(#1)}{\ckvartheta}
% }
\def\dliftv{\vartheta}
\NewDocumentCommand{\tlift}{g}{
\IfValueTF{#1}{\wtvartheta(#1)}{\wtvartheta}
}

\def\slift{\cL}

\def\BB{\bB}


\def\thetaO#1{\vartheta\left(#1\right)}

\def\bbThetav{\check{\mathbbold{\Theta}}}
\def\Thetav{\check{\Theta}}
\def\thetav{\check{\theta}}

\DeclareDocumentCommand{\NN}{g}{
\IfValueTF{#1}{\fN(#1)}{\fN}
}
\DeclareDocumentCommand{\RR}{m m}{
\fR({#1},{#2})
}

%\DeclareMathOperator*{\sign}{Sign}

% \NewDocumentCommand{\lsign}{m}{
% {}^l\mathrm{Sign}(#1)
% }

% \NewDocumentCommand{\bsign}{m}{
% {}^b\mathrm{Sign}(#1)
% }
%
\def\tsign{{}^t\mathrm{Sign}}
\def\lsign{{}^l\mathrm{Sign}}
\def\bsign{{}^b\mathrm{Sign}}
\def\ssign{\mathrm{Sign}}


\NewDocumentCommand\lnn{t+ t- g}{
  \IfBooleanTF{#1}{{}^l n^+\IfValueT{#3}{(#3)}}{
    \IfBooleanTF{#2}{{}^l n^-\IfValueT{#3}{(#3)}}{}
  }
}


% Fancy bcO, support feature \bcO'^a_b = \overline{\cO'^a_b}
\makeatletter
\def\bcO{\def\O@@{\cO}\@ifnextchar'\@Op\@Onp}
\def\@Opnext{\@ifnextchar^\@Opsp\@Opnsp}
\def\@Op{\afterassignment\@Opnext\let\scratch=}
\def\@Opnsp{\def\O@@{\cO'}\@Otsb}
\def\@Onp{\@ifnextchar^\@Onpsp\@Otsb}
\def\@Opsp^#1{\def\O@@{\cO'^{#1}}\@Otsb}
\def\@Onpsp^#1{\def\O@@{\cO^{#1}}\@Otsb}
\def\@Otsb{\@ifnextchar_\@Osb{\@Ofinalnsb}}
\def\@Osb_#1{\overline{\O@@_{#1}}}
\def\@Ofinalnsb{\overline{\O@@}}

% Fancy \command: \command`#1 will translate to {}^{#1}\bfV, i.e. superscript on the
% lift conner.

\def\defpcmd#1{
  \def\nn@tmp{#1}
  \def\nn@np@tmp{@np@#1}
  \expandafter\let\csname\nn@np@tmp\expandafter\endcsname \csname\nn@tmp\endcsname
  \expandafter\def\csname @pp@#1\endcsname`##1{{}^{##1}{\csname @np@#1\endcsname}}
  \expandafter\def\csname #1\endcsname{\,\@ifnextchar`{\csname
      @pp@#1\endcsname}{\csname @np@#1\endcsname}}
}

% \def\defppcmd#1{
% \expandafter\NewDocumentCommand{\csname #1\endcsname}{##1 }{}
% }



\defpcmd{bfV}
\def\KK{\bfK}\defpcmd{KK}
\defpcmd{bfG}
\def\A{\!A}\defpcmd{A}
\def\K{\!K}\defpcmd{K}
\def\G{G}\defpcmd{G}
\def\J{\!J}\defpcmd{J}
\def\L{\!L}\defpcmd{L}
\def\eps{\epsilon}\defpcmd{eps}
\def\pp{p}\defpcmd{pp}
\defpcmd{wtK}
\makeatother

\def\fggR{\fgg_\bR}
\def\rmtop{{\mathrm{top}}}
\def\dimo{\dim^\circ}

\NewDocumentCommand\LW{g}{
\IfValueTF{#1}{L_{W_{#1}}}{L_{W}}}
%\def\LW#1{L_{W_{#1}}}
\def\JW#1{J_{W_{#1}}}

\def\floor#1{{\lfloor #1 \rfloor}}

\def\KSP{K}
\def\UU{\rU}
\def\UUC{\rU_\bC}
\def\tUUC{\widetilde{\rU}_\bC}
\def\OmegabfW{\Omega_{\bfW}}


\def\BB{\bB}


\def\thetaO#1{\vartheta\left(#1\right)}

\def\Thetav{\check{\Theta}}
\def\thetav{\check{\theta}}

\def\Thetab{\bar{\Theta}}

\def\cKaod{\cK^{\mathrm{aod}}}

%G_V's or G
%%%%%%%%%%%%%%%%%%%%%%%%%%%
% \def\GVr{G_{\bfV}}
% \def\tGVr{\wtG_{\bfV}}
% \def\GVpr{G_{\bfV'}}
% \def\tGVpr{\wtG_{\bfV'}}
% \def\GVpr{G_{\abfV}}
% \def\tGVar{\wtG_{\abfV}}
% \def\GV{\bfG_{\bfV}}
% \def\GVp{\bfG_{\bfV'}}
% \def\KVr{K_{\bfV}}
% \def\tKVr{\wtK_{\bfV}}
% \def\KV{\bfK_{\bfV}}
% \def\KaV{\bfK_{\acute{V}}}

% \def\KV{\bfK}
% \def\KaV{\acute{\bfK}}
% \def\acO{\acute{\cO}}
% \def\asO{\acute{\sO}}
%%%%%%%%%%%%%%%%%%%%%%%%%%%
\def\GVr{G}
\def\tGVr{\wtG}
\def\GVpr{G'}
\def\tGVpr{\widetilde{G'}}
\def\GVar{G'}
\def\tGVar{\wtG'}
\def\GV{\bfG}
\def\GVp{\bfG'}
\def\KVr{K_{\bfV}}
\def\tKVr{\wtK_{\bfV}}
\def\KV{\bfK_{\bfV}}
\def\KaV{\bfK_{\acute{V}}}

\def\KV{\bfK}
\def\KaV{\acute{\bfK}}
\def\acO{{\cO'}}
\def\asO{{\sO'}}

\DeclareMathOperator{\sspan}{span}

%%%%%%%%%%%%%%%%%%%%%%%%%%%%

\def\sp{{\mathrm{sp}}}

\def\bfLz{\bfL_0}
\def\sOpe{\sO^\perp}
\def\sOpeR{\sO^\perp_\bR}
\def\sOR{\sO_\bR}

\def\ZX{\cZ_{X}}
\def\gdliftv{\vartheta}
\def\gdlift{\vartheta^{\mathrm{gen}}}
\def\bcOp{\overline{\cO'}}
\def\bsO{\overline{\sO}}
\def\bsOp{\overline{\sO'}}
\def\bfVpe{\bfV^\perp}
\def\bfEz{\bfE_0}
\def\bfVn{\bfV^-}
\def\bfEzp{\bfE'_0}

\def\totimes{\widehat{\otimes}}
\def\dotbfV{\dot{\bfV}}

\def\aod{\mathrm{aod}}
\def\unip{\mathrm{unip}}


\def\ssP{{\ddot\cP}}
\def\ssD{\ddot{\bfD}}
\def\ssdd{\ddot{\bfdd}}
\def\phik{\phi_{\fkk}}
\def\phikp{\phi_{\fkk'}}
%\def\bbfK{\breve{\bfK}}
\def\bbfK{\wtbfK}
\def\brrho{\breve{\rho}}

\def\whAX{\widehat{A_X}}
\def\mktvvp{\varsigma_{{\bf V},{\bf V}'}}

\def\Piunip{\Pi^{\mathrm{unip}}}
\def\cf{\emph{cf.} }
\def\Groth{\mathrm{Groth}}
\def\Irr{\mathrm{Irr}}

\def\edrc{\mathrm{DRC}^{\mathrm e}}
\def\drc{\mathrm{DRC}}
\def\drcs{\mathrm{DRC}^{s}}
\def\drcns{\mathrm{DRC}^{ns}}
\def\LS{\mathrm{LS}}
\def\LLS{\mathrm{{}^{\ell} LS}}
\def\LSaod{\mathrm{LS^{aod}}}
\def\Unip{\mathrm{Unip}}
\def\lUnip{\mathrm{{}^{\ell}Unip}}
\def\tbfxx{\tilde{\bfxx}}


\newcommand{\noticed}{noticed }
\newcommand{\ess}{essential }

% Ytableau tweak
\makeatletter
\pgfkeys{/ytableau/options,
  noframe/.default = false,
  noframe/.is choice,
  noframe/true/.code = {%
    \global\let\vrule@YT=\vrule@none@YT
    \global\let\hrule@YT=\hrule@none@YT
  },
  noframe/false/.code = {%
    \global\let\vrule@YT=\vrule@normal@YT
    \global\let\hrule@YT=\hrule@normal@YT
  },
  noframe/on/.style = {noframe/true},
  noframe/off/.style = {noframe/false},
}
\makeatother
%\ytableausetup{noframe=on,smalltableaux}
\ytableausetup{noframe=on,boxsize=1.3em}
\let\ytb=\ytableaushort

\newcommand{\tytb}[1]{{\tiny\ytb{#1}}}

\makeatletter
\newcommand{\dotminus}{\mathbin{\text{\@dotminus}}}

\newcommand{\@dotminus}{%
  \ooalign{\hidewidth\raise1ex\hbox{.}\hidewidth\cr$\m@th-$\cr}%
}
\makeatother


\def\cOp{\cO^{\prime}}
\def\cOpp{\cO^{\prime\prime}}
\def\cLpp{\cL^{\prime\prime}}
\def\cLppp{\cL^{\prime\prime\prime}}
\def\pUpsilon{\Upsilon^+}
\def\nUpsilon{\Upsilon^-}
\def\pcL{\cL^+}
\def\ncL{\cL^-}
\def\pcE{\cE^+}
\def\ncE{\cE^-}
\def\pcC{\cC^+}
\def\ncC{\cC^-}
\def\pcLp{\cL^{\prime+}}
\def\ncLp{\cL^{\prime-}}
\def\pcLpp{\cL^{\prime\prime+}}
\def\ncLpp{\cL^{\prime\prime-}}
\def\pcB{\cB^+}
\def\ncB{\cB^-}
\def\uptaup{\uptau^{\prime}}
\def\uptaupp{\uptau^{\prime\prime}}
\def\uptauppp{\uptau^{\prime\prime\prime}}
\def\bdelta{{\bar{\delta}}}
\def\tcO{\tilde{\cO}}
\def\tcOp{\tcO^{\prime}}
\def\tcOpp{\tcO^{\prime\prime}}
\def\tuptau{\tilde{\uptau}}
\def\tuptaup{\tuptau^{\prime}}
\def\tuptaupp{\tuptau^{\prime\prime}}
\def\tuptauppp{\tuptau^{\prime\prime\prime}}
\def\taup{\tau^{\prime}}
\def\taupp{\tau^{\prime\prime}}
\def\tauppp{\tau^{\prime\prime\prime}}
\def\cpT{\cT^+}
\def\cnT{\cT^-}
\def\cpB{\cB^+}
\def\cnB{\cB^-}

\def\deltas{\delta^s}
\def\deltans{\delta^{ns}}


\def\uum{{\dotminus}}
\def\uup{\divideontimes}

\def\umm{{=}}
\def\upp{{\ast}}
\def\upp{
  {{\setbox0\hbox{$\times$}
      \rlap{\hbox to \wd0{\hss$+$\hss}}\box0
    }}
}

\begin{document}


\title[]{Combinatorics for unipotent representations}

\author [D. Barbasch] {Dan M. Barbasch}
\address{the Department of Mathematics\\
  310 Malott Hall, Cornell University, Ithaca, New York 14853 }
\email{dmb14@cornell.edu}

\author [J.-J. Ma] {Jia-jun Ma}
\address{School of Mathematical Sciences\\
  Shanghai Jiao Tong University\\
  800 Dongchuan Road, Shanghai, 200240, China} \email{hoxide@sjtu.edu.cn}

\author [B. Sun] {Binyong Sun}
% MCM, HCMS, HLM, CEMS, UCAS,
\address{Academy of Mathematics and Systems Science\\
  Chinese Academy of Sciences\\
  Beijing, 100190, China} \email{sun@math.ac.cn}

\author [C.-B. Zhu] {Chen-Bo Zhu}
\address{Department of Mathematics\\
  National University of Singapore\\
  10 Lower Kent Ridge Road, Singapore 119076} \email{matzhucb@nus.edu.sg}




\subjclass[2000]{22E45, 22E46} \keywords{orbit method, unitary dual, unipotent
  representation, classical group, theta lifting, moment map}

% \thanks{Supported by NSFC Grant 11222101}
\maketitle


\tableofcontents


\delete{
\section{Relevent nilpotent orbits}

Let
\[\cO=(C_{2a},C_{2a-1}, \cdots, C_1,C_0,C_{-1}=0)\]
be a special orbit.  The dual orbit
\[ \ckcO = [R_{2a}\geq R_{2a-1}\geq \cdots \geq R_1\geq R_0].\] The $\ckcO$ is
given by add a box on the longest row of $\cO^T$ and then do $B$-collapse.


\begin{defn}
  Suppose $\bfG$ is type $C$.  An special orbit $\cO$ is called purely even if
  $\ckcO$ only has odd rows.
  This is equivalent to the condition on columns of $\cO$ such that (see \Cref{lem:C.even})
  \begin{enumC}
  \item $C_{2i} \equiv C_{2i-1} \pmod{2}$; 
  \item 
   If $C_{2i-1}$ is even,  $C_{2i-1} > C_{2i-2}$; 
  \item
    If $C_{2i}$ is odd, $C_{2i}=C_{2i-1}$. 
  \end{enumC}
  We call a purely even orbit $\cO$ is stablelly trivial when only even columns
  occures. This is equivalent to  
  \[
    C_{2i} \equiv C_{2i-1} \equiv 0 \pmod{2}  \text{ and } C_{2i-1}>
    C_{2i-2} \quad \forall i.
  \]
  {\color{red} In this case all LS in the unipotent representations are
    irreducible. }
  This is also equivalent to
  \[
    R_{2i}>R_{2i-1} \quad \forall i
  \]

  We call a purely even orbit $\cO$ is theta-admissible when each row columns
  appears at most $2$-times.
  {\color{red} All unipotent representations of these orbits are expected to
    constructed by theta correspondence. Moreover, they are parameterized by the
    their local systems. 
  }
  This is also equivalent to each row lenght appears at most $3$-times in
  $\ckcO$. 
\end{defn}

Suppose $\cO$ is purely even. 
The dual orbit 
$\ckcO = [R_{2a}\geq R_{2a-1}\geq \cdots \geq R_1\geq R_0]$
is calculated in the following way: 
\[
  [R_{2i},R_{2i-1}] = \begin{cases}
    [C_0+1, 0] & i=0\\
    [C_{2i}+1, C_{2i-1}-1] & i>0, C_{2i} \text{ is even} \\
    [C_{2i},C_{2i-1}] & i>0, C_{2i} \text{ is odd}
  \end{cases}
\]


\begin{defn}
  Suppose $\bfG$ is type $D$.  An special orbit
  \[
    \cO=(C_{2a-1}\ge\dots \ge C_0\geq C_{-1}=0) \quad C_{2i}\equiv
    C_{2i-1}\pmod{2}, C_{2a-1} \equiv 0 \pmod{2}
  \]
  is called purley even if
  $\ckcO$ only has odd rows.
  This is equivalent to the condition on columns of $\cO$ such that (see \Cref{lem:C.even})
  \begin{enumC}
  \item $C_{2i} \equiv C_{2i-1} \pmod{2}$; 
  \item 
   If $C_{2i-1}$ is even,  $C_{2i-1} > C_{2i-2}$; 
  \item
    If $C_{2i}$ is odd, $C_{2i}=C_{2i-1}$. 
  \end{enumC}
  Note that this is the same condition as the type C case. 
\end{defn}


Suppose $\cO$ is purely even. 
The dual orbit 
$\ckcO = [R_{2a-1}\geq \cdots \geq R_1\geq R_0, R_{-1}=0]$
is calculated in the following way: 
\[
  [R_{2i},R_{2i-1}] = \begin{cases}
    [C_0+1, 0] & i=0\\
    [0, C_{2a-1}-1] & i = a \\
    [C_{2i}+1, C_{2i-1}-1] & a> i>0, C_{2i} \text{ is even} \\
    [C_{2i},C_{2i-1}] & a>i>0, C_{2i} \text{ is odd}
  \end{cases}
\]
}


\section{Statement of the main result}
\subsection{Nilpotent Orbits}
Let $\bfG = \Sp,  \rO$, and $\cO\in \Nil(\bfG)$ is a complex nilpotent
orbit.
``Type M'' means that $\wtG$ is a metaplectic group. its dual group $\ckbfG$ is a symplectic
group with the same rank, and its duality means the metaplectic dual.
Let $\drc(\cO)$ be the set of dot-r-c diagrams parameterizing unipotent
representations of the real forms of $\bfG$.

We say an orbit $\ckcO\in \Nil(\ckbfG)$ is purely even (good parity in
Moeglin-Renard's notation) if
\begin{des}
 \item[] all row lengths of $\ckcO$ are even when $\ckbfG$ is a symplectic group.
 \item[]  all row lengths of $\ckcO$ are odd when $\ckbfG$ is an orthogonal group.
\end{des}

%The following set of orbit are the purely even case:

\begin{defn} \label{def:noticed}
  Now we describe the nilpotent orbit $\cO\in \Nil(\bfG)$ who is the
  Barbasch-Vogan dual of a purely even orbit $\ckcO$. If $G$ is
  \begin{des}
    \item [Type C] Then
          $\cO$ is special and it has columns
          \[
            ( C_{2k},C_{2k-1},\cdots, C_{1},C_{0}, C_{-1}=0 )
            \quad \text{such that } C_{2i+1}>C_{2i} \text{ if } C_{2i} \text{ is even for
            } i=0, \cdots, k-1.
          \]
           Note that, by the
          definition of special orbit,
          we have $C_{2i}\equiv C_{2i-1} \pmod{2}$, $C_{2i} = C_{2i-1}$ if
          $C_{2i}$ is odd for $i = 1, \cdots k$.
    \item [Type M] Then
          $\cO$ is metaplecitic special and it has columns
          \[
            ( C_{2k},C_{2k-1},\cdots, C_{1},C_{0}=0 )
            \quad \text{such that } C_{2i+1}>C_{2i} \text{ if } C_{2i} \text{ is odd for } i=0, \cdots, k-1.
          \]
          Note that, by the
          definition of metaplectic special,
          we have $C_{2i}\equiv C_{2i-1} \pmod{2}$, $C_{2i} = C_{2i-1}$ if
          $C_{2i}$ is even
          for $i = 1, \cdots k$.
    \item [Type D] and $\cO$ is special and it has columns
          \[
            ( C_{2k+1},C_{2k},C_{2k-1},\cdots, C_{1},C_{0}, C_{-1}=0 )
            \quad \text{such that } C_{2i+1}>C_{2i} \text{ if } C_{2i} \text{ is
              even for } i=0, \cdots, k.
          \]
          Note that, the
          condition is equivalent to $\DD(\cO)$ is purely even of type C and
          $C_{2k+1}$ is always even.
    \item [Type B] and $\cO$ is special and it has columns
          \[
            ( C_{2k+1},C_{2k},C_{2k-1},\cdots, C_{1},C_{0} =0 )
            \quad \text{such that } C_{2i+1}>C_{2i} \text{ if } C_{2i} \text{ is
              odd for } i=0, \cdots, k.
          \]
          Note that, the condition is equivalent to $\DD(\cO)$ is purely even of type M and
          $C_{2k+1}$ is always odd.
  \end{des}
  Let $\dpeNil(\star)$ the set of duals of purely even
  orbits for type $\star=$ B,C,D,M respectively.

  % Under the above notation, $\cO$ is called \emph{\ess} if
  % $C_{2i}=C_{2i-1}$ for all $i$.
\end{defn}


\subsection{Descent of orbits in the purely even case}
  For $\set{\star,\star'} = \set{C,D}, \set{B,M}$, we define the notion of
  descent of orbits. The descent map is defined in the following way
  (under the above notation):

  \[
    %\begin{tikzpicture}
    %\matrix[matrix of math nodes, column sep=0em, row sep=0em]{
    \begin{tikzcd}
      \eDD \colon & \dpeNil(\star) \ar[rr] & \hspace{3em} &
      \dpeNil(\star') \\
      & \cO \ar[rr, maps to] & & \cO'
    \end{tikzcd}
    % };
    %\end{tikzpicture}
  \]
  such that
  \begin{itemize}
    \item when $\star$ is B or D, $\cO':= \DD(\cO)$;
    \item when $\star$ is C or M,
          \[
          \cO':= \begin{cases}
             & \text{if $C_{2k}$ is even and $\star=B$}\\[-0.8em]
            \DD(\cO) & \\[-0.8em]
             & \text{or $C_{2k}$ is odd and $\star=M$};\\
            ( C_{2k-1}+1, C_{2k-2},C_{2k-3},\cdots, C_{0} )  & \text{otherwise}.
          \end{cases}
          \]
          In the second case,  $C_{2k}=C_{{2k-1}}$ and
          $\cO \mapsto \eDD(\cO)$ is a good generalized descent.
  \end{itemize}


The following set of orbit are essential to us:
\begin{defn} \label{def:noticed}
  A nilpotent orbit $\cO\in \Nil(\bfG)$ is called \emph{\noticed} if $G$ is
  \begin{des}
    \item [Type D] and $\cO$ is special and it has columns
          \[
            ( C_{2k+1},C_{2k},C_{2k-1},\cdots, C_{1},C_{0}, C_{-1}=0 )
            \quad \text{such that } C_{2i+1}>C_{2i} \text{ for } i=0, \cdots, k.
          \]
          % Note that, the
          % condition is equivalent to $\DD(\cO)$ is \noticed of type C, i.e.
          %  $C_{2i}\equiv C_{2i-1} \pmod{2}$ and $C_{2i} = C_{2i-1}$ if $C_{2i}$
          %  is odd for $i = 1, \cdots k$.
    \item [Type B] and $\cO$ is special and it has columns
          \[
            ( C_{2k+1},C_{2k},C_{2k-1},\cdots, C_{1},C_{0} =0 )
            \quad \text{such that } C_{2i+1}>C_{2i} \text{ for } i=0, \cdots, k.
          \]
          % Note that, the
          % condition is equivalent to $\DD(\cO)$ is \noticed of type M, i.e.
          %  $C_{2i}\equiv C_{2i-1} \pmod{2}$ and $C_{2i} = C_{2i-1}$ if $C_{2i}$
          %  is even for $i=1, \cdots, k$.
    \item [Type C] and $\eDD(\cO)$ is noticed of type $D$.

          % $\cO$ is special and it has columns
          % \[
          %   [C_{2k},C_{2k-1},\cdots, C_{1},C_{0}, C_{-1}=0]
          %   \quad \text{such that } C_{2i+1}>C_{2i} \text{ for } i=0, \cdots, k-1.
          % \]
          %  Note that, by the
          % definition of special orbit,
          % we have $C_{2i}\equiv C_{2i-1} \pmod{2}$, $C_{2i} = C_{2i-1}$ if
          % $C_{2i}$ is odd for $i = 1, \cdots k$.
    \item [Type M] and $\eDD(\cO)$ is notice of type $B$.
          % $\cO$ is metaplecitic special and it has columns
          % \[
          %   [C_{2k},C_{2k-1},\cdots, C_{1},C_{0}=0]
          %   \quad \text{such that } C_{2i+1}>C_{2i} \text{ for } i=0, \cdots, k-1.
          % \]
          % Note that, by the
          % definition of metaplectic special,
          % we have $C_{2i}\equiv C_{2i-1} \pmod{2}$, $C_{2i} = C_{2i-1}$ if
          % $C_{2i}$ is even
          % for $i = 1, \cdots k$.
  \end{des}
  Furthermore, We say a noticed orbit $\cO$ of type $D$ or $B$ is
  \emph{+noticed}
  if $C_{2k+1}-C_{2k}\geq 2$.
  % Let $\nNil(\star)$ be the set of noticed %and $\eNil(\star)$ be the sets of \noticed and \ess
  % orbits for type $\star=$ B,C,D,M respectively.
\end{defn}

The following inductive definition of noticed orbit is easy to verify.
\begin{lem}
  An orbit $\cO$ of type D (resp. type B) is noticed if and only if
  \begin{enumI}
    \item $\eDD^{2}(\cO)$ is noticed, when
    $C_{2k}$ is even (resp. odd), or
    \item $\eDD^{2}(\cO)$ is +noticed, when $C_{2k}$ is odd (resp. even). \qedhere
  \end{enumI}
\end{lem}

\subsection{Relavent Weyl group representations and dot-r-c diagrams}
Retain the notation  in \Cref{def:noticed}.
Now we describe all relevent representations of the Weyl group appear in the
counting algorithm.


\trivial[h]{
  We use ``$s$'' and ``$d$'' to denote ``$r'$'' and ``$c'$'' in Dan's notes respectively.
}

\begin{des}
  \item[Type D] We define $c_{i}$ as the following
  \[
    \begin{split}
      c_{2i} & := \floor{\frac{C_{2i}}{2}} \quad \text{and}\quad c_{2i-1} =\floor{\frac{C_{2i-1}+1}{2}}
      \quad \forall i=0,1, \cdots, k,\\
      c_{2k+1} & := \frac{C_{2k+1}}{2}
    \end{split}
  \]
  The (special) Weyl group representation associated to the special orbit $\cO$
  has columns
  \[
    \tau = \tau_L\times\tau_R=(c_{2k+1},c_{2k-1},\dots , c_1)\times (c_{2k},\dots ,c_0).
  \]
  The other representations is obtained from the special one by interchanging
  following pairs when $c_{2i-1}\neq c_{2i}+1$ and $i=1, \cdots, k$ (these are
  pairs coming from even length columns)
  \[
    (c_{2i-1},c_{2i} )\longleftrightarrow (c_{2i}+1,c_{2i-1}-1).
  \]
  \trivial[h]{ Note that we always have $c_{2i-1}-1 \geq c_{2i-2}$ by our
    assumptions of the shape of the orbit. }

  The dot-r-c diagrams of type D: For each representation
  $\tau = \tau_L\times \tau_R$, dots are filled in both sides forming the same
  shape of Young diagram, a column of ``$s$'' is added next to dots on the left
  side diagram, a column of ``$r$'' is added next to dots on the left side
  diagram, add a row of ``$c$''s and finally add a row of ``$d$''s on the left
  side of the diagram; Through this procedure, one must make sure that the shape
  of the diagram is a valid Young diagram after each step. \trivial[h]{At most
    one $r$/$s$ is added in each row of the existing dots diagram and at most
    one $c$/$d$ is added to each column.}

  In summary, $s/r/c/d$ are all added on the left diagram.
  Fixing the representation $\tau$, the left diagram $\uptau_{L}$ determin
  $\uptau_{R}$ completely.

  \item[Type C] Set $C_{2k+1}=0$. The Weyl group representations are obtained by
  the same formula of Type D.

  The dot-r-c diagrams of type C are obtained in almost the same way as that of
  type D, except that $s$ are added on the right side of the diagram.

  \item[Type B] We define $c_{i}$ as the following
  \[
    \begin{split}
      c_{2i} & := \floor{\frac{C_{2i}+1}{2}} \quad \text{and}\quad c_{2i-1} =\floor{\frac{C_{2i-1}}{2}}
      \quad \forall i=1,2, \cdots, k,\\
      c_{2k+1} & := \frac{C_{2k+1}-1}{2}
    \end{split}
  \]
 % \trivial[h]{Note that $C_{2k+1}$ is always odd and $C_{0}=0$}
  Note that $C_{2k+1}$ is always odd and $C_{0}=0$.
  The (special)
  Weyl group representation associated to the special orbit $\cO$ has columns
  \[
    \tau = \tau_L\times\tau_R= (c_{2k},\dots ,c_2) \times (c_{2k+1},c_{2k-1},\dots , c_1).
  \]
  The other representations is obtained from the special one by interchanging
  following pairs when $c_{2i-1}\neq c_{2i}+1$ and $i=1, \cdots, k$ (these are
  pairs coming from odd length columns)
  \[
    (c_{2i}, c_{2i-1})\longleftrightarrow (c_{2i-1}, c_{2i}).
  \]
  \trivial[h]{ Note that we always have $c_{2i-1} \geq c_{2i-2}$ by our
    assumptions of the shape of the orbit. }

  The dot-r-c diagrams of type B: For each representation
  $\tau = \tau_L\times \tau_R$, dots are filled in both sides forming the same
  shape of Young diagram, a column of ``$s$'' is added next to dots on the right
  side of the diagram, a column of ``$r$'' is added next to dots on the right
  side diagram, add a row of ``$d$''s on the right side of the diagram, and
  finally add a row of ``$c$''s on the left side of the diagram. Through this
  procedure, one must make sure that the shape of the diagram is a valid Young
  diagram after each step. \trivial[h]{At most one $r$/$s$ is added in each row
    of the existing dots diagram and at most one $c$/$d$ is added to each
    column.}

  In summary, $s/r/d$ are added on the right and $c$ is added on the left.

  Each dot-r-c diagram of type B is extended into two diagrams by adding an
  extra label ``$a$'' or ``$b$''. This extension helps us to determine the
  signature of special orthogonal groups.


  Fixing the representation $\tau$, the right diagram $\uptau_{R}$ determin
  $\uptau_{L}$ completely.

  \item[Type M] Set $C_{2k+1}=0$. The Weyl group representations are obtained by
  the same formula of Type B.

  The dot-r-c diagrams of type M are obtained in almost the same way as the type
  B except that $s$ are added on the left side of the diagram and we do not add
  labels a/b.
\end{des}

We let $\drc(\cO)$ denote the set of dot-r-c diagrams defined above.
Note that for type B, there is an additional mark $a$/$b$ attached to the
diagram.



In addition, for $\star=B,C,D,M$, let
\[
  \drc(\star) := \bigsqcup_{\cO\in \dpeNil(\star)} \drc(\cO).
\]


% By the counting argument established before, it is suffice to establish \Cref{thm:count} for
% \ess orbits.

\subsection{Main theorem}
Let $\LS(\cO)$ denote the Grothendieck group of $\wtbfK$-equivariant local
systems on $\cO$ for various real forms of $\bfG$.
The following is our main theorem on the counting of unipotent representations.

Let $\ckcO$ be a purely even orbit and $\cO$ be its dual orbit.

\begin{thm} \label{thm:count}
  We can define an injective map using convergence range theta lifting:
  \[
    \begin{tikzcd}[column sep=0em, row sep=0em]
      \uppi:& \drc(\cO) \ar[rr] &\hspace{2em} & \Unip(\cO)\\
      & \uptau \ar[rr,maps to] & & \uppi_{\uptau}\\
    \end{tikzcd}
  \]
\begin{enumS}
  \item
  When $\cO$ is of type C or type M, $\uppi$ is a bijection.
  %we have a bijective map
  % \[
  %   \begin{tikzcd}[column sep=0em, row sep=0em]
  %     \uppi:& \drc(\cO) \ar[rr] &\hspace{2em} & \Unip(\cO)\\
  %     & \uptau \ar[rr,maps to] & & \uppi_{\uptau}\\
  %   \end{tikzcd}
  % \]
%  \item
  \item
  When $\cO$ is of type D or type B, the following map is a bijection
  %Suppose $\cO$ is of type D or type B. we have a bijective map
  \[
    \begin{tikzcd}[column sep=0em, row sep=0em]
       \drc(\cO) \times \bZ/2\bZ \ar[rr] &\hspace{2em} & \Unip(\cO)\\
       (\uptau, \varepsilon) \ar[rr,maps to] & & \uppi_{\uptau}\otimes (\det)^{\varepsilon}\\
    \end{tikzcd}
  \]

\item
  All unipotent representations attached to $\cO$ are unitarizable.
\item
  Suppose $\cO$ is noticed. The associated character map is an injective map:
  \[
    \begin{tikzcd}[column sep=0em, row sep=0em]
      \Ch_{\cO}\colon& \Unip(\cO)  \ar[rr,hook] &\hspace{2em} & \LS(\cO)\\
  %    & \uppi \ar[rr,maps to] & & \cL_{\uppi}\\
    \end{tikzcd}
  \]
%\end{thm}
\item
Suppose $\cO$ is quasi-distinguished. The image of associated character map is
  exactly the set of admissible orbit data. In other words, we have a bijection
  \[
    \begin{tikzcd}[column sep=0em, row sep=0em]
      \Ch_{\cO}\colon& \Unip(\cO)  \ar[rr,hook,two heads] &\hspace{2em} & \LSaod(\cO)\\
  %    & \uppi \ar[rr,maps to] & & \cL_{\uppi}\\
    \end{tikzcd}
    %\Ch_{\cO}\colon \Unip(\cO)\longrightarrow \LS^{aod}(\cO).
  \]
\end{enumS}
\end{thm}

% \begin{thm} \label{cor:count}
%   When $\cO$ is dual purely even,
%   \[
%     \Unip(\cO) = \set{\Thetab_{\chi}\neq 0},
%   \]
%   i.e.  all unipotent representations attached to $\cO$ can be
%   constructed by iterated convergent range theta lifting and therefore unitary.

%   Suppose $\cO$ is noticed. Then the map
%   \[
%     \begin{tikzcd}
%       \Ch\colon \Unip(\cO) \ar[r] & \LS(\cO)
%     \end{tikzcd}
%   \]
%   is injective.
%   %% Moreover, every element in $\Unip(\cO)$ can be obtained by iterated theta lifting.
% \end{thm}

{\color{red} According to Moeglin-Renard's work, Arthur's unipotent Arthur packet
  (defined by endoscopic transfer) is constructed by iterated theta lifting.
  The theorem implies that ABV's unipotent Arthur packet coincide with Arthur's.}


Two unipotent representation $\pi_{1}$ and $\pi_{2}$ are called in the same
LS packet if $\cL_{\pi_{1}} = \cL_{\pi_{2}}$.
For a local system $\cL\in \LS(\cO)$, let
\[
  [\cL] = \set{\pi |\Ch_{\cO}(\pi) = \cL}
\]
denote the set of all unipotent representations attached to $\cL$. We call
$[\cL]$ a  LS packet.

\begin{remark}
  \begin{enumR}
    \item
    % The proof is by establish an explicit injection
    % $\drc(\cO)\longrightarrow \set{\Thetab_{\chi}\neq 0}$ inductively.
    We use convergence range theta lift to construct $\uppi$ inductively with
    respect to the number of columns in $\cO$.
    \item
    When $\cO$ is not noticed, a LS packet may not be a singleton
    and we use the injectivity of theta lifting for a fixed dual pair in an essential way.
    \item
    The size of LS packet $[\cL]$ could be arbitrary large  when the size of $\cO$
    groups.
    For example, $\cO$ is the regular orbit of $\Sp(2n,\bC)$, there is only 5 possible
    local systems when $n\geq 3$. But there are $2n+1$ unipotent
    representations. See \Cref{sec:C.reg}.
    \item One will see easily from the proof that the associated character
    $\Ch_{\cO}(\pi)$ for every $\pi\in \Unip(\cO)$ is always multiplicity free.
    Even when $\Ch_{\cO}(\pi)$ is not irreducible, the character of the local
    system attached the same row length in different irreducible components are
    always the same.
  \end{enumR}
\end{remark}

\section{Matching doc-r-c diagrams, local systems, and unipotent representations}



\subsection{Descent maps of dot-r-c diagram in the algorithm}

Suppose $\bfG$ is of type B or D, we define a map
\[
  \begin{tikzcd}[column sep=0em, row sep=0em]
    \Sign \colon&   \drc(B)\sqcup \drc(D) \ar[rr]& \hspace{3em} & \bN\times \bN \\
    & \uptau \ar[rr, maps to]& & (p,q)
  \end{tikzcd}
\]
Here $(p,q)$ is the signature of the orthogonal group $\rO(p,q)$ corresponding
to the dot-r-c diagram $\uptau$, which is calculated by the following formula
\[
  \begin{split}
    p &= \# \bullet+ 2 \# r + \# c + \# d + \# A\\
    q &= \# \bullet+ 2 \# s + \# c + \# d + \# B\\
  \end{split}
\]



Suppose $\cO$ is the trivial orbit of $\Sp(2c_{0},\bR)$ ($c_{0}\geq 0$) or
$\Mp(0,\bR)$ ($c_{0}=0$).
Let
\begin{equation}\label{eq:uptau0}
  \uptau_{\cO}=\ytb{\emptyset,\none,\none}\times\ytb{s,\vdots,s}
\end{equation}
such that there are $c_{0}$-entries marked by ``$s$'' in the right diagram.
Then $\drc(\cO) = \set{\uptau_{\cO}}$ is a singleton.
Define
$\uppi_{\uptau_{\cO}} = \bfone$ the trivial representation of $\Sp(2c_{0},\bR)$ or $\Mp(0,\bR) = \bfone$.
That by convention, $\Sp(0,\bR)=\Mp(0,\bR)$ is the trivial group.
Let
\[
  \drc_{0}(C) = \set{\uptau_{\cO}| \cO = (2c_{0}), c_{0}=0,1,2,\cdots}
  \text{ and }
  \drc_{0}(M) = \set{\uptau_{(0)}=\emptyset \times \emptyset}.
\]


The algorithm of matching dot-r-c diagrams is encoded in the descent map
 $\eDD$ of dot-r-c diagrams defined below.
%For $\set{\star,\star'} = \set{C,D}, \set{B,M}$,
For $(\star,\star') = (D,C), (B,M)$,
we define maps
\[
  \begin{tikzcd}[column sep=0em, row sep=0em]
    \eDD \colon & \drc(\star)\sqcup (\drc(\star')-\drc_{0}(\star')) \ar[rr] & \hspace{2em} &
    \left(\drc(\star')\sqcup \drc(\star) \right)\times \bZ/2\bZ\\
    & \uptau \ar[rr, maps to] &  & (\uptau', \upepsilon).
  \end{tikzcd}
\]
Here
\begin{enumI}
  \item $\uptau$ is in $\drc(\cO)$ where $\cO\in \dpeNil(\star)\sqcup \dpeNil(\star')$;
  \item $\uptau'\in \drc(\cO')$ where $\cO': = \eDD(\cO)$;
  \item i$\upepsilon = 0$ when $\cO\mapsto \cO'$ is a generalized descent from type
  C or M to type B or D.
\end{enumI}
To ease the notations, we define
\[
    \eDDo \colon  \drc(\star)\sqcup (\drc(\star')-\drc_{0}(\star')) \longrightarrow
    \left(\drc(\star')\sqcup \drc(\star) \right)
\]
such that $\eDDo(\uptau)$ is the first component of $\eDD(\uptau)$.

\medskip

We define $\cL$ and $\uppi$ inductively with respect to the number of
columns of $\cO$.
\begin{enumS}
  \item % Suppose $\cO=[C_{0}=0]$ and $\star =  C$ or  $M$. Then
  % $\drc(\cO) = \set{\uptau_{\emptyset}}$ is a singleton with $\uptau_{\emptyset}=(\emptyset,\emptyset)$.
  Our reduction terminates when $\cO$ is the trivial orbit of $\Sp(2c_{0},\bR)$ or
  $\Mp(0,\bR)$.
  In this case,
  $\drc(\cO) = \set{\uptau_{\cO}}$ which is attached to the trivial
  representation.
  We remark that the theta lift of the trivial representation
  $\uppi_{\uptau_{\cO}}=\bfone$ of $\Sp(0,\bR)$ or $\Mp(0,\bR)$ to $\rO(p,q)$ for
  any signature $p,q$ is the trivial representation $\bfone_{p,q}^{+,+}$ of
  $\rO(p,q)$.\footnote{When $c_{0}=0$, this is an assignment by convention.}
  \item Suppose $\cO\in \Nil(\star)\sqcup \Nil(\star')$ is a nontrivial orbit.
  Let $\uptau \in \drc(\cO)$ and
  \[
    (\uptau', \upepsilon) = \eDD(\tau)\in \drc(\cO')\times \bZ/2\bZ \text{ where }\cO' = \eDD(\cO).
  \]
  \begin{enumS}
    \item Suppose $\uptau\in \drc(C)$ or $\drc(M)$. Now $\wtG = \Sp(2n,\bR)$ or
    $\Mp(2n,\bR)$ with $n=\abs{\uptau}$. By induction, $\uptau'$ is attached to
    a unipotent representation $\uppi_{\uptau'}$ of $\wtG' = \rO(p,q)$ where
    $(p,q)=\Sign(\uptau')$. We define
    \[
      \uppi_{\uptau} := \Thetab_{\wtG',\wtG}(\uppi_{\uptau'}\otimes (\det)^{\upepsilon}).
    \]
    Here $\det$ is the determinant character of $\rO(p,q)$.
    \item Suppose $\uptau\in \drc(D)$ or $\drc(B)$. Now $\wtG = \rO(p,q)$ with
    $(p,q) = \Sign(\uptau)$. By induction, $\uptau'$ is attached to a
    unipotent representation of $\wtG'$ where $\wtG' = \Sp(2n,\bR)$ or
    $\Mp(2n,\bR)$ and $n=\abs{\uptau'}$. We define
    \[
      \uppi_{\uptau} := \Thetab_{\wtG',\wtG}(\uppi_{\uptau'})\otimes (\bfone_{p,q}^{+,-})^{\upepsilon}.
    \]
  \end{enumS}
\end{enumS}

We let
\[
  \begin{split}
    \sL_{\uptau} & := \Ch_{\cO}(\uppi_{\uptau}), \text{ and } \\
    % \lUnip(\cO) &= \set{\uppi_{\uptau}|\uptau \in \drc(\cO)} \\
    \LLS(\cO) & := \set{\cL_{\uptau}|\uptau \in \drc(\cO)}
  \end{split}
\]
to be the set of local systems obtained from our algorithm.

We summarize our definitions in the following diagram with $\star=\cO/C/D/B/M$.
% \[
%   \begin{tikzcd}[column sep=0em, row sep=0em]
%  &   \edrc(\star) \ar[rrrr,"\uppi"] \ar[dddd, "\cL"']  & & & & \Unip(\star) \ar[dddd, "\Ch"] \\
%  &   &  \uptau \ar[rr,maps to] \ar[dd,maps to]& \hspace{4em} & \pi_{\uptau}  \ar[lldd,maps to]&  \\[2em]
% \edrc(\star') \times \bZ/2\bZ \ar[ruu,leftarrow,"\eDD"] \ar[ddd,"\cL\times \id"']&   & & &   & \\[2em]
%  &   &\cL_{\uptau} & &   & \\
%  &  \LLS(\star)\ar[rrrr,hook]\ar[ld,"\eDD"] & & & & \LS(\star) \\[3.5em]
%  \LLS(\star')\times \bZ/2\bZ \ar[rrrr,hook]&   & & &\LS(\star')\ar[ru,dashed,"\vartheta"']  & \\
% \end{tikzcd}
% \]
% \[
%   \begin{tikzcd}[column sep=0em, row sep=0em]
%     \drc(\star) \ar[rrrr,"\uppi"] \ar[dddd, "\cL"']  & & & & \Unip(\star) \ar[dddd, "\Ch"] \\
%     &  \uptau \ar[rr,maps to] \ar[dd,maps to]& \hspace{4em} & \pi_{\uptau}  \ar[lldd,maps to]&  \\[2em]
%     &   & & &   & \\[2em]
%     &\cL_{\uptau} & &   & \\
%    \LLS(\star)\ar[rrrr,hook] & & & & \LS(\star) \\
% \end{tikzcd}
% \]
% Here
% \begin{enumC}
%   \item $\uptau$ is in $\drc(\cO)$ where $\cO$ is essential;
%   \item $\uppi_{\uptau}$ is a unipotent representation of $\wtG$ attached to $\cO$;
%   \item $\wtG$ is $\Mp(2n,\bR)$ or $\Sp(2n,\bR)$ if $\star = $ $C$ or $M$;
%   \item $\wtG$ is $\rO(p,q)$ if $\star=$  $B$ or $C$ and $(p,q)=\Sign(\uptau)$;
%   \item
%   % \item For each local system $\cL\in \LLS(\star)$, the descent map $\cL$
%   % \item the left commutative diagram says the descent map $\eDD$ is compatible
%   % with the local system, i.e. it sends
% \end{enumC}

% Let
% \[ \color{red}
% \]
%

\[
  \begin{tikzcd}[column sep=0em, row sep=0em]
    \drc(\star) \ar[rrrr,"\uppi"] \ar[dddd, "\cL"']  & & & & \Unip(\star) \ar[dddd, "\Ch"] \\
    &  \uptau \ar[rr,maps to] \ar[dd,maps to]& \hspace{4em} & \pi_{\uptau}  \ar[dd,maps to]&  \\[2em]
    &   & & &   & \\[2em]
    &\cL_{\uptau} \ar[rr,<->]& &  \sL_{\uptau} & \\
   \KM(\star)\ar[rrrr,hook,two heads] & & & & \LS(\star) \\
\end{tikzcd}
\]

Here $\KM(\star)$ is a combinatiorially defined monoid parameterizing the local
systems on the relevent nilpotent orbits, see.

\section{The definition of $\eDD$ }
\def\taur{\uptau_{R}}
\def\taul{\uptau_{L}}
\def\taulf{\uptau_{L,0}}
\def\tauls{\uptau_{L,1}}
\def\taurf{\uptau_{R,0}}
\def\taurs{\uptau_{R,1}}

\def\tauplf{\uptau'_{L,0}}
\def\taupls{\uptau'_{L,1}}
\def\tauprf{\uptau'_{R,0}}
\def\tauprs{\uptau'_{R,1}}
\def\tail{\mathrm{tail}}




From now on, we assume
\[
  \uptau = (\taulf,\tauls,\cdots)\times(\taurf,\taurs,\cdots).
\]

% The definition of sign $\upepsilon$
% \begin{enumS}
%   \item Suppose $\uptau$ is type D or B, $\upepsilon=0$ if and only if
%   $\# d (\uptau)> \#d(\uptau')$. This means:
%   \begin{enumS}
%     \item When $\uptau$ is type D,  $\tail(\taulf) = d$.
%     \item When $\uptau$ is type B,   $\tail(\taurf) = d$ or
%     $(\tail(\taulf),\tail(\taurf),\tail(\taurs)) = (c,r,d)$.
%   \end{enumS}
%   \item Suppose $\uptau$ is type C or M, $\upepsilon=0$ if and only if
%   $\abs{\taulf}\geq \abs{\taurf}$.
% \end{enumS}

For any $\uptau$ let $\uptau^{\bullet}$ be the subdiagram consisting of
$\bullet$ and $s$ entries only.


\subsection{dot-s switching algorithm}
\def\bipartl{\mathrm{bi\cP_L}}
\def\bipartr{\mathrm{bi\cP_R}}
\def\dsdiagl{\mathrm{DS_L}}
\def\dsdiagr{\mathrm{DS_R}}
\def\DDl{\eDD_\mathrm{L}}
\def\DDr{\eDD_\mathrm{R}}

We say a bipartition
$\tau = \tau_{L}\times \tau_{R}$
is interlaced with lager left part if  $\tau$ has columns
$(\tau_{L,0}\geq \tau_{L,1}, \cdots, )\times (\tau_{R,0},\tau_{R,1}, \cdots,)$
such that
\[
\tau_{L,0}\geq \tau_{R,0}\geq \tau_{L,1}\geq \tau_{R,1}\geq \tau_{L,2} \geq \cdots.
\]
$\bipartl$ denote the set of all interlaced bipartitions with larger left parts.
Similarly, define $\bipartr$ be the set of bipartitoins with larger right parts:
\[
  \bipartr = \set{\tau=(\tau_{L,0}\geq \tau_{L,1}, \cdots, )\times (\tau_{R,0},\tau_{R,1}, \cdots,)|
\tau_{L,0}\geq \tau_{R,0}\geq \tau_{L,1}\geq \tau_{R,1}\geq \tau_{L,2} \geq \cdots}
\]

Let $\dsdiagl$ (resp. $\dsdiagr$) be the set of of filled diagrams with only
``$s$'' entries on the left (resp. right) diagram and the rest are all ``$\bullet$''.

Note that for each $\uptau\in \dsdiagl$, its shape $\tau$ is in $\bipartl$.
On the other hand, for each $\tau\in \bipartl$, we can fill ``$\bullet$'' in
all entries on the right diagram and corresponding entries on the left, and than fill ``$s$''
in the rest entries on the left. This procedure yields a valid dot-s diagram.
In summary, $\bipartl$ and $\dsdiagl$ are naturally bijective to each other.

Let $\DDl\colon \bipartl\rightarrow \bipartr $ (resp.
$\DDr\colon \bipartr\rightarrow \bipartl$) be the operation
of deleting the longest column on the left (resp. on the right).
Then the operation gives a well defined maps between $\dsdiagl$ and $\dsdiagr$ making
the following diagrams commute (the vertical maps are the natural identification
discussed above):
\[
\begin{tikzcd}
  \bipartl \ar[r,"\DDl"] \ar[d,equal]& \bipartr \ar[d,equal] &
  \bipartr \ar[r,"\DDr"] \ar[d,equal]& \bipartl \ar[d,equal] \\
  \dsdiagl\ar[r,dashed,"\DDl"] & \dsdiagr & \dsdiagr\ar[r,dashed,"\DDr"] & \dsdiagl\\
\end{tikzcd}
\]
By abuse of notation, we also call the dashed arrow above $\DDl$ and $\DDr$.


\subsection{Bijection between ``special'' and ``non-special'' diagrams for type C}
We define a bijection between ``special'' diagram $\uptau^{s}$ and
``non-special'' diagrams $\uptau$.

\begin{defn}\label{def:sp-nsp.C.sp}
  Let $\cO = (C_{2k},C_{2k-1}, \cdots, C_{1},C_{0}, C_{-1}=0)\in \dpeNil(C)$
  with $k\geq 0$.
  Let
  \[
  \tau = (\tau_{2k-1}, \cdots, \tau_{1},\tau_{-1}=0) \times (\tau_{2k}, \cdots, \tau_{0}).
  \]
  be a Weyl group representation attached to $\cO\in \dpeNil(C)$.
  We say $\tau$ has
  \begin{itemize}
    \item \idxemph{special shape} if $\tau_{2k-1}\leq \tau_{2k}+1$;
    \item \idxemph{non-special shape} if $\tau_{2k-1}> \tau_{2k}+1$.
  \end{itemize}
  Clearly, this gives a partition of the set of Weyl group representations
  attached to $\cO$.

  Suppose $k\geq 1$, $C_{2k}=2c_{2k}$ and $C_{2k-1}=2c_{2k-1}$.
  The special shape representation $\tau^{s}$
  \[
    \tau^{s} = \tau_{L}^{s}\times \tau_{R}^{s}=(c_{2k-1},\tau_{2k-3},\cdots, \tau_{1} )
    \times (c_{2k},\tau_{2k-2},\cdots, \tau_{0})
  \]
  is paired with the non-special shape representation
  \[
    \tau^{ns} = \tau_{L}^{ns}\times \tau_{R}^{ns} = (c_{2k}+1,\tau_{2k-3},\cdots, \tau_{1} )
    \times (c_{2k-1}-1,\tau_{2k-2},\cdots, \tau_{0})
  \]
  In the paring $\tau^{s}\leftrightarrow \tau^{ns}$, $\tau_{j}$  is unchanged for $j\leq 2k-2$.
  We call $\tau^{s}$ the \emph{special shape} and $\tau^{ns}$ the
  corresponding \emph{non-special shape}
\end{defn}



\begin{defn}\label{def:sp-nsp.C}
  Suppose $C_{2k}$ is even.
We retain the notation in \Cref{def:sp-nsp.C.sp} where the shapes $\tau^{s}$ and $\tau^{ns}$
correspond with each other.
We define a bijection between $\drc(\tau^{s})$ and $\drc(\tau^{ns})$:
\[
  \begin{tikzcd}[row sep=0em]
    \drc(\tau^{s}) \ar[rr,<->]&\hspace{2em} & \drc(\tau^{ns})\\
    \uptau^{s} \ar[rr,<->]& & \uptau^{ns}
  \end{tikzcd}
\]

Without loss of generality, we can assume
that $\uptau^{s}\in \drc(\tau^{s})$ and $\uptau^{ns}\in \drc(\tau^{ns})$ have the following shapes:
\begin{equation}\label{eq:sp-nsp.C}
  \uptau^{s}:\tytb{\cdots\cdots\cdots,{x_{0}}{\ast}{\cdots},{x_{1}}{x_{2}}{\cdots},\ ,\ ,\ ,\ }
  \times\tytb{\cdots\cdots\cdots,{v_{0}}{\cdots},{x_{3}},{*(srcol)s},{*(srcol)\vdots},{*(srcol)s},\ } \longleftrightarrow \hspace{2em}
  \uptau^{ns}:\tytb{\cdots\cdots\cdots,{y_{0}}{\ast}{\cdots},{*(srcol)r}{y_{2}}\cdots,
    {*(srcol)\vdots},{*(srcol)r},{y_{1}},{y_{3}} }
  \times\tytb{\cdots\cdots\cdots,{w_{0}}{\cdots},\ ,\ ,\ ,\ ,\ }
\end{equation}
(Here the grey parts have length $(C_{2k}-C_{2k-1})/2$ (could be zero length), the row contains
$x_{0}$ and $y_{0}$ may be empty, and $*$/$\cdots$ are
arbitrary entries. We remark that the value of $v_{0}$ and $w_{0}$ are
completely determined by $x_{0}$ and $y_{0}$. In particular, $v_{0}$ and $w_{0}$
are non-empty if and only if
$x_{0}/y_{0}\neq \emptyset$, i.e. $C_{2k-1}\geq 4$.)

% We marks the entries of $\uptau^{s}$ and $\uptau^{ns}$ as in
% \eqref{eq:sp-nsp.C}.
The correspondence between $(x_{0},x_{1}, x_{2}, x_3)$, with
$(y_{0},y_{1},y_{2},y_{3})$ is given by \Cref{tab:nonsp.C}.
The rest part of the diagrams are unchanged.

We define
\[
  \drcs(\cO) := \bigsqcup_{\tau^{s}} \drc(\tau^{s}) \quad \text{ and }\quad
  \drcns(\cO) := \bigsqcup_{\tau^{ns}} \drc(\tau^{ns})
\]
where $\tau^{s}$ (resp. $\tau^{ns}$) runs over all specail (resp. non-special) shape representations attached to $\cO$.
Clearly $\drc(\cO) = \drcs(\cO)\sqcup \drcns(\cO)$.
\end{defn}
It is easy to check the bijectivity in the above definition, which we leave it
to the reader.



\subsection{Special and non-special shapes of type D}
Now we define the notion of special and non-special shape of type D.
\begin{defn}\label{def:sp-nsp.D.sp}
  Let
  \[
  \tau = (\tau_{2k+1},\tau_{2k-1}, \cdots, \tau_{1})\times (\tau_{2k}, \cdots, \tau_{0})
  \]
  be a Weyl group representation attached to $\cO\in \dpeNil(D)$.
  We say $\tau$ has
  \begin{itemize}
    \item \idxemph{special shape} if $\tau_{2k-1}\leq \tau_{2k}+1$;
    \item \idxemph{non-special shape} if $\tau_{2k-1}> \tau_{2k}+1$.
  \end{itemize}

  Suppose
  $\cO=(C_{2k+1}=2c_{2k+1},C_{2k}=2c_{2k},C_{2k-1}=2c_{2k-1},\cdots,C_{0}\geq 0)$
  with $k\geq 1$.\footnote{Note that $C_{2k}$ and $C_{2k-1}$ are non-zero
    even integers.}
  A representation $\tau^{s}$ attached to $\cO$ has the shape
  \[
    \tau^{s} = \tau_{L}^{s}\times \tau_{R}^{s}=(c_{2k+1},c_{2k-1},\tau_{2k-3},\cdots, \tau_{1} )
    \times (c_{2k},\tau_{2k-2},\cdots, \tau_{0})
  \]
  is paired with
  \[
    \tau^{ns} = \tau_{L}^{ns}\times \tau_{R}^{ns} = (c_{2k+1},c_{2k}+1,\tau_{2k-3},\cdots, \tau_{1} )
    \times (c_{2k-1}-1,\tau_{2k-2},\cdots, \tau_{0}).
  \]
  In the paring $\tau^{s}\leftrightarrow \tau^{ns}$, $\tau_{j}$  is unchanged for $j\leq 2k-2$.
  % We call $\tau^{s}$ the \emph{special shape} and $\tau^{ns}$ the
  % corresponding \emph{non-special shape}
\end{defn}

\begin{table}[hpb]
\[\tiny
\begin{array}{c|c|c:c|c}
  & \uptaup & \uptau^{s} & \uptau^{ns} \\
  \hline
           & \ytb{{x'_{0}}{\ast},{x'_{1}}{x'_{2}},\none,\none,\none,\none}
             \times \ytb{\ast,\none,\none,\none,\none,\none}
           &\ytb{{x_{0}}{\ast},{x_{1}}{x_{2}},\ ,\ ,\ ,\ }
    \times\ytb{{\ast}{\ast},{x_{3}},{*(srcol)s},{*(srcol)\vdots},{*(srcol)s},\ }&
  \ytb{{y_{0}}{\ast},{*(srcol)r}{y_{2}},{*(srcol)\vdots},{*(srcol)r},{y_{1}},{y_{3}} }
    \times\ytb{{\ast}\ast,\ ,\ ,\ ,\ ,\ }\\
  \hline
  \ytb{{x'_{1}}{=}{s},{\text{then}},{z_{0}}{=}{\emptyset/}{s},{x_{0}}{=}{\emptyset/}{\bullet},{x_{1}}{=}{\bullet},
  {x_{2}}{=}{x'_{2}}}
  &
            \ytb{{x'_{0}}{\ast},{s}{x_{2}},\none,\none,\none,\none}
             \times \ytb{\ast,\none,\none,\none,\none,\none}
                &
                \ytb{{x_{0}}{\ast},{\bullet}{x_{2}},\ ,\ ,\ ,\ }
    \times\ytb{{\ast}\ast,{\bullet},{*(srcol)s},{*(srcol)\vdots},{*(srcol)s},\ }&
  \ytb{{x_{0}}{\ast},{*(srcol)r}{x_{2}},{*(srcol)\vdots},{*(srcol)r},{c},{d} }
                          \times\ytb{{\ast}{\ast},\ ,\ ,\ ,\ ,\ }
          &x_{2}\neq r
  \\
  \cline{3-5} & &
                  \ytb{{x_{0}}{\ast},{\bullet}{r},\ ,\ ,\ ,\ }
    \times\ytb{{\ast}\ast,{\bullet},{*(srcol)s},{*(srcol)\vdots},{*(srcol)s},\ }&
  \ytb{{x_{0}}{\ast},{*(srcol)r}{c},{*(srcol)\vdots},{*(srcol)r},{r},{d} }
    \times\ytb{{\ast}\ast,\ ,\ ,\ ,\ ,\ } & x_{2}=r \\
  \hline
\ytb{{x'_{1}}{\neq}{s},{\text{then}},{x_{1}}{=}{x'_{1}},{x_{2}}{=}{x'_{2}}}
  &
            \ytb{{x'_{0}}{\ast},{x'_{1}}{x_{2}},\none,\none,\none,\none}
             \times \ytb{\ast,\none,\none,\none,\none,\none}
   &
    \ytb{{x_{0}}{\ast},{x_{1}}{x_{2}},\ ,\ ,\ ,\ }
    \times\ytb{{\ast}\ast,{s},{*(srcol)s},{*(srcol)\vdots},{*(srcol)s},\ }&
  \ytb{{x_{0}}{\ast},{*(srcol)r}{x_{2}},{*(srcol)\vdots},{*(srcol)r},{r},{x_{1}} }
       \times\ytb{{\ast}\ast,\ ,\ ,\ ,\ ,\ } &
     \ytb{{x'_{0}}{\neq}{c},{\text{then}},{x'_{0}}{=}{\emptyset/}{s/}{r},{x_{0}}{=}{\emptyset/}{\bullet/}{r}}
  \\
  %\hline
  \cline{3-5}
   & &
    \ytb{{c}{\ast},{d}{x_{2}},\ ,\ ,\ ,\ }
    \times\ytb{{\ast}\ast,{s},{*(srcol)s},{*(srcol)\vdots},{*(srcol)s},\ }&
  \ytb{{r}{\ast},{*(srcol)r}{x_{2}},{*(srcol)\vdots},{*(srcol)r},{c},{d} }
  \times\ytb{{\ast}\ast,\ ,\ ,\ ,\ ,\ }&
  \ytb{{x'_{0}}{=}{c}\none\none,{\text{then}},{x_{0}}{=}{x'_{0}}{=}{c},{x_{1}}{=}{x'_{1}}{=}{d},{x_{2}}{=}{x'_{2}}{=}{\emptyset/d}}
  \\
  \hline
  \hline
\end{array}
\]
\caption{``special-non-special'' switch}
\label{tab:nonsp.C}
\end{table}


\subsection{Definition of $\eDD$ of Type D and C} \label{sec:alg.CD}

The induction starts with type C: For the trivial orbit $\cO$ of
$\wtG = \Sp(2c_{0},\bR)$, $\drc(\cO) = \set{\uptau_{\cO}}$ and
$\uppi_{\uptau_{\cO}}=\bfone$ is the trivial representation, see \eqref{eq:uptau0}.

\subsubsection{The defintion of $\upepsilon$.} \label{sec:upepsilon}
\begin{enumerate}[label=(\arabic*).,series=alg1]
  \item When $\uptau\in \drc(D)$, $\upepsilon$ is determined by the ``basal
  disk'' of $\uptau$ (see \eqref{eq:x.uptau} for the definition of $x_{\uptau}$):%escent from type D to type C,
  \[
    \upepsilon_{\uptau}:=
    \begin{cases}
      0, & \text{if $x_{\uptau}=d$;} \\
      1, & \text{otherwise.}
    \end{cases}
  \]
  \item When $\uptau\in \drc(C)$, the  $\upepsilon$  is determined by
  the lengths of $\taulf$ and $\taurf$:
  \[
    \upepsilon_{\uptau} :=
    \begin{cases}
      0, & \text{if $\uptau$ has special shape, i.e. } \abs{\taulf}-\abs{\taurf} \leq  1;\\
      1, & \text{if $\uptau$ has non-special shape, i.e. }\abs{\taulf} - \abs{\taurf}> 1.
    \end{cases}
  \]
\end{enumerate}
%To indicate the relation with $\uptau$, we will also w

\medskip

%{\bf Definition of the descent of diagram $\uptaup$.}

\subsubsection{Initial cases}
%For type D, %when $\cO$ must have odd number of column.
Let $\cO$ be a nilpotent orbit of type D with at most 2 columns.
\begin{enumerate}[resume*=alg1]
  \item Suppose $\cO = (2c_{1},2c_{0})$.
        Then
        $\cOp:=\eDD(\cO)$ is the trivial orbit of $\Sp(2c_{0},\bR)$. $\drc(\cO)$
        consists of diagrams of shape
        $\tau_{L}\times \tau_{R} =(c_{1},)\times (c_{0},)$.
        The set $\drc(\cOp)$ is a singleton $\set{\uptau_{\cOp}}$, and every element
        $\uptau \in \drc(\cO)$ maps to $\uptau_{\cOp}$ (see \eqref{eq:uptau0}):
        \[\tiny
          \drc(\cO)\ni \uptau: \hspace{1em} \ytb{\bullet,\vdots,\bullet,{x_{1}},\vdots,{x_{n}}}
          \times \ytb{\bullet,\vdots,\bullet,\none,\none,\none}
          \mapsto \uptau_{\cOp}:
          \ytb{\emptyset,\none,\none,\none,\none,\none}
          \times \ytb{s,\vdots,s,\none,\none,\none}
        \]
        We define
        \[
          \bfpp_{\uptau}:=\bfxx_{\uptau} := \tytb{{x_{1}},\vdots,{x_{n}}} \text{ and }x_{\uptau}:=x_{n}
        \]
        We call $\bfpp_{\uptau}=\bfxx_{\uptau}$ the ``peduncle'' part of $\uptau$ and
        $x_{\uptau}$ the
        ``basal disk'' of $\uptau$.
        Note that $\bfxx_{\uptau}$ consists of entries marked by $s/r/c/d$.
\end{enumerate}


\subsubsection{The descent from C to D}
The descent of a special shape diagram is simple, the descent of a non-special shape
reduces to the corresponding special one:
\begin{enumerate}[resume*=alg1]
  \item Suppose $\abs{\taulf} - \abs{\taurf} < 2$, i.e. $\uptau$ has special
        shape. Keep $r,c,d$ unchanged, delete $\taurf$ and fill the remaining
        part with ``$\bullet$'' and ``$s$'' by $\DDr(\uptau^{\bullet})$ using
        the dot-s switching algorithm.
  \item Suppose $\abs{\taulf} - \abs{\taurf} \geq 2$. We define
        \[\uptaup=\eDDo(\uptau):=\eDDo(\uptau^{s})\] where $\uptau^{s}$ is the
        special diagram corresponding to $\uptau$ defined in
        \Cref{def:sp-nsp.C}.
\end{enumerate}

\begin{lem}\label{lem:ds.CD}
  Suppose $\cO = (C_{2k},C_{2k-1}, \cdots, C_{0})$ with $k\geq 1$ and $C_{2k}$ even.
  Let $\cOp := \eDD(\cO)=(C_{2k-1},\cdots, C_{0})$.
  Then
  \[
    \begin{split}
      \drcs(\cO) &\xrightarrow{\hspace{2em}\eDDo\hspace{2em}} \drc(\cOp) \\
      \drcns(\cO)&
      \xrightarrow{\hspace{2em}\eDDo\hspace{2em}} \drc(\cOp)
    \end{split}
  \]
  are a bijections.
\end{lem}
\begin{proof}
  The claim for $\drcs(\cO)$ is clear by the definition of descent.
  The claim for $\drcns(\cO)$ reduces to that of $\drcs(\cO)$ using \Cref{def:sp-nsp.C}.
\end{proof}

\begin{lem}\label{lem:gd.CD}
  Suppose $\cO = (C_{2k},C_{2k-1}, \cdots, C_{0})$ with $k\geq 1$ and $C_{2k}$ odd.
  Let $\cOp := \eDD(\cO)=(C_{2k-1}+1,\cdots, C_{0})$.
  Then the following map is a bijection
  \[
      \drc(\cO) \xrightarrow{\hspace{2em}\eDDo\hspace{2em}} \set{\uptaup\in \drc(\cOp)| x_{\uptau}\neq s}. %\subsetneq \drc(\cOp).
  \]
\end{lem}
\begin{proof}
  This is clear by the descent algorithm.
\end{proof}
\subsubsection{Descent from $D$ to $C$} Now we consider the general case of the
descent from type $D$ to type $C$. So we assume $\cO$ has at least 3 columns.
First note that the shape of $\uptau' = \eDD(\uptau)$ is the shape of $\uptau$ deleting the
longest column on the left.

% We assume $\uptau$ and $\uptau'$ has the following shape where $(y_{1},y_{2})$
% could be empty and $w_{0}$ is non-empty if $\uptau_{R}\neq \emptyset$.
The definition splits in cases below. In all these cases we define
\begin{equation}\label{eq:x.uptau}
\bfxx_{\uptau} := x_{1}\cdots x_{n} \text{ and } x_{\uptau} := x_{n}
\end{equation}
which is marked by $s/r/c/d$.
For the part marked by $*/\cdots$ , $\eDD$ keeps $r,c,d$ and maps the rest part consisting of $\bullet$ and $s$ by dot-s switching algorithm.
\begin{enumerate}[resume*=alg1]
  \item When $C_{2k}=C_{2k-1}$ is odd, we could assume $\uptau$ and $\uptaup$ hase
        the following forms with $n = (C_{2k+1}-C_{2k}+1)/2$.
      \[
        \uptau: \hspace{1em} \tytb{\cdots\cdots\cdots,{\ast}{\ast}{\cdots},{x_{1}}{y_{1}}\cdots,{\vdots},{x_{n}}}
        \times \tytb{\cdots\cdots\cdots,{w_{0}}{\ast}{\cdots},\none,\none,\none}
        \mapsto
        \uptaup:  \tytb{\none\cdots\cdots,\none{\ast}{\cdots},\none{z_{1}}\cdots,\none,\none}
        \times \tytb{\cdots\cdots\cdots,{w_{0}}{\ast}{\cdots},\none,\none,\none}
      \]
        Suppose $y_{1} = c$ and $(x_{1}, \cdots, x_{n}) = (r, \cdots, r)$ or
        $(r, \cdots, r,d)$, then let $z_{1}= r$; otherwise, set $z_{1}:= y_{1}$.
        We define
        \[
        \bfpp_{\uptau}:=\tytb{{x_{1}}{y_{1}},\vdots,{x_{n}}}.
        \]
  \item When $C_{2k}$ is even and $\uptau$ has special shape, the descent is given by the following diagram
      \[\tiny
        \uptau: \hspace{1em}
        \ytb{
        \cdots\cdots\cdots,
        {\ast}{\ast}{\cdots},
        {*(srcol)\bullet},
        {*(srcol)\vdots},{*(srcol)\bullet},{x_1},\vdots,{x_n}}
        \times \ytb{\cdots\cdots\cdots,{\ast}{\ast}{\cdots},{*(srcol)\bullet},{*(srcol)\vdots},{*(srcol)\bullet},\none,\none,\none}
        \mapsto
       \uptaup: \ytb{\none\cdots\cdots,\none{\ast}{\cdots},\none,\none,\none,\none,\none,\none}
        \times \ytb{\cdots\cdots\cdots,{\ast}{\ast}{\cdots},{*(srcol)s},{*(srcol)\vdots},{*(srcol)s},\none,\none,\none}
      \]
  \item When $C_{2k}$ is even and $\uptau$ has non-special shape, then
        \[
        \uptau:\tytb{{\cdots}\cdots\cdots\cdots,{\ast}{\ast}{\cdots}{\cdots},{*(srcol)s}{*(srcol)r}{\cdots}\cdots,
        {*(srcol)\vdots}{*(srcol)\vdots},{*(srcol)s}{*(srcol)r},{x_{0}}{y_{1}},{x_{1}}{y_{2}},{\vdots},{x_{n}}}
        \times\tytb{{\cdots}\cdots\cdots,{\ast}{\cdots}\cdots,\ ,\ ,\ ,\ ,\ ,\ ,\ }
        \mapsto \uptau':
        \tytb{\cdots\cdots\cdots,{\ast}{\ast}{\cdots},{*(srcol)r}{\ast}\cdots,{*(srcol)\vdots},{*(srcol)r},{z_{1}},{z_{2}},\ ,\ }
        \times\tytb{{\cdots}\cdots\cdots,{\ast}{\cdots}\cdots,\ ,\ ,\ ,\ ,\ ,\ ,\ }
        \]
      where $y_{1},y_{2}$ are non empty.
      In most of the case, we just delete the longest column on the left and set
      $(z_{1},z_{2})=(y_{1},y_{2})$. The exceptional cases are listed below:
      \begin{itemize}
        \item When
 $x_{0}=r$, we have $(y_{1},y_{2})=(c,d)$. We let
      \[
          \tytb{{x_{0}}{y_{1}}\none{r}{c},{x_{1}}{y_{2}}{=}{x_{1}}{d},\vdots\none\none\vdots,{x_{n}}\none\none{x_{n}}}
          \mapsto \tytb{{z_{1}}\none{r},{z_{2}}{:=}r,\none,\none}
        \]

        \item When $x_{0}=c$, we have $n=1$ and $(x_{0},x_{1})=(y_{1},y_{2})=(c,d)$. We let
        \[
          \tytb{{x_{0}}{y_{1}},{x_{1}}{y_{2}}} = \tytb{cc,dd}
          \mapsto \tytb{{z_{1}},{z_{2}}}:=\tytb{r,c}.
      \]
      \item
      We remark that $x_{0}$ never equals to $d$.
  \end{itemize}
\end{enumerate}

\subsubsection{A key property in the descent case}
We retain the notation in \Cref{def:sp-nsp.D.sp}, where
\[
\cO=(C_{2k+1}=2c_{2k+1},C_{2k}=2c_{2k},C_{2k-1}=2c_{2k-1},\cdots, C_{0}) \text{
  such that } k\geq 1.
\]
Let $\cOp=\eDD(\cO)$ and $\cOpp=\eDD(\cOp)$.
The following lemma is the key property satisfied by our definition
\begin{lem}\label{lem:sp-nsp.D}
  Let $\tau^{s}$ and $\tau^{ns}$ are two representations attached to $\cO$ as in
  \Cref{def:sp-nsp.D.sp}.
  Then
  \begin{enumS}
    \item \label{lem:sp-nsp.D.1} For every $\uptau\in \drc(\tau^{s})\sqcup \drc(\tau^{ns})$, the shape
    of $\eDD^{2}(\uptau)$ is
    \[
      \taupp = (c_{2k-1},\tau_{2k-3},\tau_{1})\times (\tau_{2k-2},\cdots, \tau_{0})
    \]
    \item \label{lem:sp-nsp.D.2} Let $\cO_{1}= (2(c_{2k+1}-c_{2k}),)\in \Nil(D)$ be the trivial orbit
    of $\rO(2(c_{2k-1}-c_{2k}),\bC)$.
    We define $\delta\colon \drc(\cO)\rightarrow \drc(\cOpp)\times \drc(\cO_{1})$ by $\delta(\uptau) = (\eDD^{2}(\uptau),\bfxx_{\uptau})$.
    The following maps are bijections
    \[
      \begin{tikzcd}[row sep=0em]
        \drc(\tau^{s})\ar[r,"\delta"] & \drc(\taupp)\times \drc(\cO_{1}) &\ar[l,"\delta"'] \drc(\tau^{ns})\\
        \uptau^{s}\ar[r,maps to] & (\eDD^{2}(\uptau^{s}), \bfxx_{\uptau^{s}})&\\
        & (\eDD^{2}(\uptau^{ns}), \bfxx_{\uptau^{ns}})& \uptau^{ns}\ar[l,maps to]\\
      \end{tikzcd}
    \]
    In particular, we obtain an one-one correspondence
    $\uptau^{s}\leftrightarrow \uptau^{ns}$ such that $\delta(\uptau^{s})=\delta(\uptau^{ns})$.
    \item\label{lem:sp-nsp.D.3}
    Suppose $\uptau^{s}$ and $\uptau^{ns}$ correspond as the above such that
    $\delta(\uptau^{s})=\delta(\uptau^{ns})=(\uptaupp,\bfxx)$. Then
    \begin{equation} \label{eq:sp-nsp-sig}
      \ssign(\uptau^{s})=\ssign(\uptau^{ns})=(C_{2k},C_{2k})+\ssign(\uptaupp)+\ssign(\bfxx).
    \end{equation}
  \end{enumS}
\end{lem}
\begin{proof}
  Suppose $\uptau^{s}$ has special shape.
  the the behavior of $\uptau^{s}$ under the descent map $\eDD$ is illostrated
  as the following:
  \[\tiny
    \uptau^{s}: \hspace{1em}
    \ytb{
      {*(srcol)\cdots}\cdots\cdots,
      {*(srcol)\ast}{\ast}{\cdots},
      {*(srcol)\bullet},
      {*(srcol)\vdots},{*(srcol)\bullet},{x_1},\vdots,{x_n}}
    \times
    \ytb{{*(srcol)\cdots}\cdots\cdots,
      {*(srcol)\ast}{\ast}{\cdots},{*(srcol)\bullet},{*(srcol)\vdots},{*(srcol)\bullet},\none,\none,\none}
    \mapsto
    \uptau'^{s}
     \ytb{\none\cdots\cdots,\none{\ast}{\cdots},\none,\none,\none,\none,\none,\none}
    \times \ytb{{*(srcol)\cdots}\cdots\cdots,{*(srcol)\ast}{\ast}{\cdots},{*(srcol)s},{*(srcol)\vdots},{*(srcol)s},\none,\none,\none}
    \mapsto
    \uptaupp:
    \ytb{\none\cdots\cdots,\none{\ast}{\cdots},\none,\none,\none,\none,\none,\none}
    \times \ytb{\none\cdots\cdots,{\none}{\ast}{\cdots},\none,\none,\none,\none,\none,\none}
  \]
  Note that $\uptau^{s}$ is obtained from $\taupp$ by attaching the grey part
  consisting totally $2c_{2k}$ dots and $\bfxx$. Now the claims for $\uptau^{s}$ is
  clear.


  Now consider the descent of a non-special diagram $\uptau^{ns}$:
  \[\tiny
    \uptau^{ns}:\ytb{{*(srcol)\cdots}\cdots\cdots\cdots,{*(srcol)\ast}{y_{0}}{\ast}{\cdots},{*(srcol)s}{*(srcol)r}{y_{2}}\cdots,
      {*(srcol)\vdots}{*(srcol)\vdots},{*(srcol)s}{*(srcol)r},{x_{0}}{y_{1}},{x_{1}}{y_{3}},{\vdots},{x_{n}} }
    \times\ytb{{*(srcol)\cdots}\cdots\cdots,{*(srcol)u_{0}}{\cdots},\ ,\ ,\ ,\ ,\ ,\ ,\ }
    \mapsto
    \uptau'^{ns}:\ytb{\cdots\cdots\cdots,{y'_{0}}{\ast}{\cdots},{*(srcol)r}{y'_{2}}\cdots,
      {*(srcol)\vdots},{*(srcol)r},{y'_{1}},{y'_{3}},\ ,\ }
    \times\ytb{{*(srcol)\cdots}\cdots\cdots,{*(srcol)w_{0}}{\cdots},\ ,\ ,\ ,\ ,\ ,\ ,\ }
    \mapsto
    \uptaupp:\hspace{1em}
            \ytb{{\cdots}{\cdots}{\cdots},{x'_{0}}{\ast}{\cdots},{x'_{1}}{x'_{2}}\cdots,\none,\none,\none,\none,\none,\none}
            \times
            \ytb{{\cdots}{\cdots},{\cdots}\none,\none,\none,\none,\none,\none,\none,\none}
            % \ytb{\emptyset,\none,\none,\none,\none,\none}
  \]
  The bijection follows from the observation that 1. $\uptau^{ns}$ is obtained by
  attaching the grey part $x_{0},\cdots,x_{n}$ and $y_{0},y_{1},y_{2},y_{3}$
  to the $\ast/\cdots$ part of $\uptaupp$;
  2. the value of $(y_{0},y_{1},y_{2},y_{3})$ is completely determined by
  $(x'_{0},x'_{1},x'_{2})$ and $\bfxx=x_{1}\cdots x_{n}$.

  We leave it to the reader for checking case by case that the grey part of $\uptau^{ns}$ has signature
  $(C_{2k}-2,C_{2k}-2)$ and
  \[\ssign(x_{0}y_{0}y_{1}y_{2}y_{3})-\ssign(x'_{0}x'_{1}x'_{2})=(2,2).
  \]
 Therefore \eqref{eq:sp-nsp-sig} follows.

 \trivial[]{
   First assume that $C_{2k}=2$. Note that we can not/(or now?) assume $k=1$, consider the
   orbit $\cOpp=(2,1,1,1,1)$.
   \begin{enumPF}
     \item
     $x'_{1}=s$, now $y'_{0} = \emptyset/\bullet$ when $x'_{0}=\emptyset/s$
     \begin{enumPF}
       \item Suppose $x'_{2}\neq r$. Then $(y'_{2},y'_{1},y'_{3}) = (x'_{2},r,d)$,
       $(x_{0},y_{0},y_{2},y_{1},y_{3}) = (s,x'_{0}, x'_{2},c,d)$.
       Therefore, the sign difference is $\ssign(s,c,d)-\ssign(s)=(2,2)$.
       \item Suppose $x'_{2} = r$. Then $(y'_{2},y'_{1},y'_{3}) = (c,r,d)$.
       $(x_{0},y_{0},y_{2},y_{1},y_{3}) = (s,x'_{0},c,r,d)$.
       Therefore, the sign difference is $\ssign(s,c,r,d)-\ssign(s,r)=(2,2)$.
     \end{enumPF}
     \item $x'_{1}\neq s$.
     \begin{enumPF}
       \item Suppose $x'_{0}\neq c$.
       Then $x'_{0}=\emptyset/s/r$, $x'_{1}=r/c/d$, $(y'_{0}, y'_{2},y'_{1},y'_{3}) = (\emptyset/\bullet/r,x'_{2},r,x'_{1})$.
       \begin{enumPF}
         \item $x'_{1}=r$. Then
         $(x_{0},y_{0},y_{2},y_{1},y_{3}) = (r,x'_{0},x'_{2},c,d)$.
         Therefore, the sign difference is $\ssign(r,c,d)-\ssign(r)=(2,2)$.
         \item $x'_{1}=c$. Then  $x_{1}=d$.
         $(x_{0},y_{0},y_{2},y_{1},y_{3}) = (c,x'_{0},x'_{2},c,d)$.
         Therefore, the sign difference is $\ssign(c,c,d)-\ssign(c)=(2,2)$.
         \item $x'_{1}=d$. Then
         $(x_{0},y_{0},y_{2},y_{1},y_{3}) = (s,x'_{0},y'_{2},y'_{1},y'_{3})=(s,x'_{0},x'_{2},r,d)$.
         Therefore, the sign difference is $\ssign(s,r,d)-\ssign(d)=(2,2)$.
       \end{enumPF}
       $(x_{0},y_{0},y_{2},y_{1},y_{3}) = (s,r,x'_{2},c,d)$.
       Therefore, the sign difference is $\ssign(s,r,x'_{2},c,d)-\ssign(c,d,x'_{2})=(2,2)$.
       \item Suppose $x'_{0} = c$.
       Then $x'_{1}=d$, $(y'_{0}, y'_{2},y'_{1},y'_{3}) = (r,x'_{2},c,d)$.
       $(x_{0},y_{0},y_{2},y_{1},y_{3}) = (s,r,x'_{2},c,d)$.
       Therefore, the sign difference is $\ssign(s,r,x'_{2},c,d)-\ssign(c,d,x'_{2})=(2,2)$.
     \end{enumPF}
   \end{enumPF}
   }
 \end{proof}

\subsubsection{A key proposition in the generalized descent case}\label{sec:gd2.CD}

We now assume that $k\geq 1$ and $C_{2k}$ is odd, i.e. $\cOp\leadsto \cOpp$ is a
generalized descent.

Without loss of generality, we can assume the dot-r-c diagrams have the
following shapes with $n = (C_{2k+1}-C_{2k}+1)/2$:
\begin{equation}\label{eq:gd2.drc}
  \uptau:\tytb{
    {*(srcol)\cdots}\cdots\cdots\cdots,
    {*(srcol)\ast}{\ast}{\ast}{\ast},
    {x_{1}}{y_{1}}{\cdots}\cdots,
    \vdots,
    {x_{n}}
  }
  \times
\tytb{{*(srcol)\cdots}\cdots\cdots,
    {*(srcol)\ast}{\ast}{\cdots},
    \none,\none,\none}
  \xmapsto{\hspace{1em}\eDDo\hspace{1em}}
  \uptaup:\tytb{
    \cdots\cdots\cdots,
    {\ast}{\ast}{\ast},
    {x_{\uptaupp}}{\cdots}\cdots,
    \none,
    \none
  }
  \times
\tytb{{*(srcol)\cdots}\cdots\cdots,
    {*(srcol)\ast}{\ast}{\cdots},
    \none,\none,\none}
  \xmapsto{\hspace{1em}\eDDo\hspace{1em}}
  \uptaupp:\tytb{
    \cdots\cdots\cdots,
    {\ast}{\ast}{\ast},
    {x_{\uptaupp}}{\cdots}\cdots,
    \none,
    \none
  }
  \times
\tytb{\cdots\cdots,
    {\ast}{\cdots},
    \none,\none,\none}
\end{equation}

The diagram $\uptau$ is obtained from the diagram of $\uptau$ by adding
$\bullet$ at the
grey parts, changing $x_{\uptaupp}$ to $y_{1}$ and attaching $x_{1}\cdots x_{n}$.


We define
\[
  \bfpp_{\uptau}:=\tytb{{x_{1}}{y_{1}},\vdots,{x_{n}}}
\]
and call $\bfpp_{\uptau}$ the peduncle part of $\uptau$. Let
$\cO_{1} = (C_{2k+1}-C_{2k}+1,)\in \dpeNil(D)$. We define
$\bfuu_{\uptau}\in \drc(\cO_{1})$ by the following formula:
\begin{equation}\label{eq:def.u}
  \bfuu_{\uptau}:=u_{1} \cdots u_{n} =
  \begin{cases}
    r\cdots r c, & \text{when } \bfpp_{\uptau} = \tytb{rc,\vdots,r}\\
    r\cdots c d, & \text{when } \bfpp_{\uptau} = \tytb{rc,\vdots,d}\\
    x_{1}\cdots, x_{n} & \text{otherwise}.
  \end{cases}
\end{equation}


\begin{lem}\label{lem:gd.inj}
  The map $\delta\colon  \drc(\cO)\rightarrow \drc(\cOpp) \times \drc(\cO_{1})$
  given by $\uptau\mapsto (\eDDo^{2}(\uptau),\bfuu_{\uptau})$ is injective.
  Moreover,
  \[
  \ssign(\uptau) =\ssign(\uptaupp) + (C_{2k}-1, C_{2k}-1)+\ssign(\bfuu_{\uptau}).
  \]
  The map $\tdelta\colon \drc(\cO)\rightarrow \drc(\cOpp) \times \bN^{2}\times \bZ/2\bZ$
  given by $\uptau\mapsto (\eDDo^{2}(\uptau), \ssign(\uptau),\upepsilon_{\uptau})$ is injective.
\end{lem}
\begin{proof}
  The the injectivity of $\delta$ and the siginature formula follows directly from our algorithm.
  Now the injectivity of $\tdelta$ follows from
  $\bfxx_{\uptau}\mapsto (\ssign(\bfxx_{\uptau}), \upepsilon)$ is injective by
  \Cref{c:init.CD}.
  % Let $\uptaupp:=\eDDo^{2}(\uptau)$ and
  % Then $\uptau\mapsto (\uptaupp,\bfuu_{\uptau})$ is injective and
  % \[
  % %\ssign(\uptau) =\ssign(\uptaupp) + 2(n_1,n_{1})+\ssign(\bfuu_{\uptau}) \text{ where }2n_{1}+1 = C_{2k}.
  % \ssign(\uptau) =\ssign(\uptaupp) + (C_{2k}-1,C_{2k}-1)+\ssign(\bfuu_{\uptau}) \text{ where }2n_{1}+1 = C_{2k}.
  % \]
  % Now the lemma follows from
  % $\bfuu_{\uptau}\mapsto (\ssign(\bfuu_{\uptau}), \upepsilon)$ is injective by \Cref{c:init.CD}.
\end{proof}



\section{Convention on the local system}
An irreducible local system is represented by a {\emph{marked Young
    diagram}\index{Young Diagram}}. A general local system is understand as the
union of irreducible ones. By our computation later, it would be clear that the
local systems appeared in our cases are always \emph{multiplicity one}.


\subsection{Commutative free monoid}
Given a  set $\sfA$, the commutative free monoid $\sfC(\bA)$ on $\bfA$ consists all finite multisets with
elements drawn from $\sfA$.
It is equipped with the binary operation, say $\bullet$ such that
\[
  \set{a_{1},\cdots, a_{k_{1}}}\bullet \set{b_{1},\cdots, b_{k_{2}}} :=
  \set{a_{1},\cdots, a_{k_{1}},b_{1},\cdots, b_{k_{2}}}.
\]
In the following, we always identify $\sfA$ with the subset of $\sfC(\sfA)$
consists of singletons.

Suppose $\sfM$ is a commutative monoid, a map $f\colon \sfA\rightarrow \sfM$ is
uniquelly extends to a morphism $\sfC(\sfA)\rightarrow \sfM$ which is still
called $f$ by abuse of notations.

In particular a map $f\colon \sfA\rightarrow \sfB$ between sets uniquely extends to an morphism
$f\colon \sfC(\sfA)\rightarrow \sfC(\sfB)$ by
\[
  f(\set{a_{1},\cdots, a_{k}}) = \set{f(a_{1}), \cdots, f(a_{k})}.
\]

\subsection{Marked Young diagram}
We now explain the combinatiorial objects parameterizing local systems.
Basically ,we use $+/-$ (resp. $\upp/\umm$) to denote the associated character on the
corresponding factor of the isotropic group.
Let $\star=B,C,D,M$.


\begin{itemize}
  \item {\bf Young diagram } Let $\YD$ be the commutative free monoid
        $\sfC(\bN^{+})$ on the set of positive integers $\bZ_{\geq 1}$. An
        element $\cO = \set{R_{1}, \cdots, R_{k}}$ is identified with the Young
        diagram whose set of row lengths is $\cO$. Therefore, we identify $\YD$
        with the set of Young diagrams. Let
        \[
        \abs{ \ \ }\colon \YD \longrightarrow \bN
        \]
        be the map sending $\set{R_{1},\cdots R_{k}}$ to $\sum_{i=1}^{k} R_{i}$.

        Let $\YD(\star)$ to denote the Young diagrams corresponding to
        partitions of type $\star$. The set $\YD(\star)$ parameterizes the set
        of nilpotent orbits of type $\star$. Note that $\YD(C)=\YD(M)$.



  \item {\bf Signed Young diagram } Let $\sfS$ be the set of non-empty finite
        length strings of alternating signs of the form $+-+\cdots$ or
        $-+-\cdots$. Let $\SYD :=\sfC(\sfS)$ be the monoid on $\sfS$.

        The monoid $\SYD$ is identified with the set of all signed Young
        diagrams: $\sO=\set{\bfrr_{1}, \cdots, \bfrr_{k}}\in \SYD$ is identified
        with the signed Young diagram whose rows are filled by $\bfrr_{j}$. The
        map $ \sfS\rightarrow \bZ_{\geq 1}$ sending a string to its length
        extends to the morphism
        \[
        \begin{tikzcd}[row sep=0em]
          \cY\colon \SYD \ar[r] & \YD\\
          \set{\bfrr_{1},\cdots, \bfrr_{k}} \ar[r,maps to] & \set{\abs{\bfrr_{1}},\cdots, \abs{\bfrr_{k}}}
        \end{tikzcd}
        \]
        where $\abs{\bfrr_{j}}$ denote the length of the string $\bfrr_{j}$. By
        abused of notation, we also denote $\abs{\ \ }\circ \cY$ by
        $\abs{\ \ }$.

        We define $\SYD(\star)$ to be a subset of the preimage of $\YD(\star)$
        under the above map: When $\star=C,M$ (resp. $\star=D,B$),
        \[
        \SYD(\star) = \Set{ \sO\in \SYD| \begin{minipage}{15em}
            $\cY(\sO)\in \YD(\star)$\\
            For each positive odd number (resp. even number) $r$, the strings $+-\cdots$ and
            $-+\cdots$ of the length $r$ appears in $\sO$ with the same
            multiplicity (could be 0).
          \end{minipage}
        }
        \]

        The set $\SYD(\star)$ parameterizes the real nilpotent orbits of type
        $\star$ and $\bfK$-orbits in $\fpp$.

        We define two signature maps $\sfS\rightarrow \bN\times \bN$ as the
        following. For $\bfrr\in \sfS$, define
        \[
        \begin{split}
          \ssign(\bfrr)& := (\#+(\bfrr), \#-(\bfrr)) \quad \text{and}\\
          \lsign(\bfrr)& := \begin{cases}
            (1,0) & \text{if } \bfrr = +\cdots\\
            (0,1) & \text{if } \bfrr = -\cdots\\
          \end{cases}.
        \end{split}
        \]
        The above defined signature maps naturally extends to signature maps
        $\ssign\colon \SYD\rightarrow \bN\times \bN$ and
        $\ssign\colon \SYD\rightarrow \bN\times \bN$.



  \item {\bf Marked Young diagram} Let $\sfM$ be the set of finite length
        strings of alternating marks of the form $+-+\cdots$,$-+-\cdots$,
        $\upp\umm\upp\cdots$, or $\umm\upp\umm\cdots$.

        Let $\MYD:=\sfC(\sfM)$ which is identified with the set of Young
        diagrams marked with alternating marks $+/-$, $\upp/\umm$. Define
        $\sF\colon \sfM \rightarrow \sfS$ by the formula
        \[
        \sF(\bfrr) := \begin{cases}
          +-+\cdots & \text{ if } \bfrr = +-+\cdots \text{ or } \upp\umm\upp\cdots\\
          -+-\cdots & \text{ if } \bfrr = -+-\cdots \text{ or } \umm\upp\umm\cdots\\
        \end{cases} \quad \forall \bfrr\in \sfM
        \]
        such that $\sF(\bfrr)$ and $\bfrr$ have the same length. $\sF$ extends
        to the map $\sF: \MYD \rightarrow \SYD$.

        By abused of notation, we also denote $\abs{\ \ }\circ \cY$ by
        $\abs{\ \ }$.

        For $\cL\in\MYD$, we define
        \[
        \ssign(\cL) = \ssign(\sF(\cL)) \quad \text{and} \quad \lsign(\cL) = \lsign(\sF(\cL)).
        \]
        We define $\MYD(\star)$ to be a subset of the preimage of the above map:
        \[
        \MYD(\star) =\Set{\cL\in \MYD(\star)|\begin{minipage}{18em}
            $\sF(\cL)\in \SYD(\star)$;\\
            $\bfrr_{1}=\bfrr_{2}$ if $\bfrr_1, \bfrr_2\in \cL$ and
            $\sF(\bfrr_{1})=\sF(\bfrr_{2})$;\\
            $\bfrr = +-+\cdots $ or $-+-\cdots $ if the length of $\bfrr$ is odd
            (resp. even ) when $\star \in \set{C,M}$ (resp.
            $\star\in \set{B,D}$).
          \end{minipage}
        }
        \]
        We will use $\MYD(\star)$ to parameterize irreducible local systems
        attached to the nilpotent orbits of type $\star$.
\end{itemize}
For monoid  $\YD$, $\SYD$ or $\MYD$, the identity element is the empty set
$\emptyset$, and the monoid binary operator is denoted by
``$\cdot$''.
The  $\emptyset$ is  always in
$\MYD(\star)$ by convention.
According to the definition of $\MYD(\star)$.  it could happen that $\cL_{1}\cdot \cL_{2}\notin \MYD(\star)$ for
$\cL_{1},\cL_{2}\in \MYD(\star)$.

% We will represent an element $\cL$ in $\SYD$ or $\MYD$ by the Young diagram whose
% rows are filled with the strings in $\cL$.
%
%Let $\bZ/2\bZ\times \sfM\rightarrow \sfM$ by

We define an involution
\begin{equation}\label{eq:def.bar}
 \overline{\phantom{+}}\colon \sfM\rightarrow \sfM
\end{equation}
on $\sfM$  by
switching the symbols $+$ with $\upp$  and $-$ with $\umm$.
For example $\overline{+-+}= \upp\umm\upp$ and $\overline{ \umm\upp } = -+$.
% \[
%   \meltese(\bfrr) =\begin{cases}
%     +-\cdots & \text{if } \bfrr= \upp\umm\cdots
%     -+\cdots & \text{if } \bfrr= \upp\umm\cdots
%     \upp-\cdots & \text{if } \bfrr= \upp\umm\cdots
%   \end{cases}
% \]

We define the map
\begin{equation}\label{eq:def.dagger}
  \dagger\colon \sfM\rightarrow \sfM
\end{equation}
by extending the length of
the strings to the left by one.
For example $\dagger(+-+) = -+-+$ and $\dagger(\umm\upp)=\upp\umm\upp$.

For  $\cL,\cC\in \MYD$, the expression $\cL\succ\cC$ means that there is $\cB\in \MYD$
such that the
factorization $\cL = \cB\cdot \cC$ holds.

In the following, we will represent an element $\cL\in \MYD$ by the Young diagram whose rows are
filled with the strings in $\cL$. The terminology ``$l$-row'' means a row of length $l$.
If we do not care whether the mark is $\upp$ or $+$ (resp.  $\umm$ or $=$), we will use $\uup$ (resp. $\uum$)
to mark the corresponding position.

\begin{eg}
  The following two diagrams $\ddagger_{p,q}$ and $\dagger_{p,q}$ are in
  $\MYD(B)\cup \MYD(D)$.
  \[
    \ddagger_{p,q} := \tytb{+,\vdots,+,=,\vdots,=}
  \]
  represents the local system on the trivial orbit of $\rO(p,q)$ which has the
  trivial character on $\rO(p)$ and $\det$ on $\rO(q)$. Similarly,
  \[
    \dagger_{p,q}:= \tytb{+,\vdots,+,-,\vdots,-}
  \]
  represents the trivial local system on the trivial orbit.
\end{eg}



\subsubsection{Local systems}
We define  $\KM$ to be the free commutative monoid generated by $\MYD$.
The binary operation in $\KM$ is denoted by ``$+$''.

The operation ``$\cdot$'' naturally extends to $\KM$:
\[
  (\sum_{i}  m_{i}\cL_{i})\cdot (\sum_{j} m_{j}\cL'_{j}) := \sum_{i,j} m_{i}m_{j}(\cL_{i}\cdot \cL'_{j}).
\]
where $\cL_{i},\cL'_{j}\in \MYD$, $m_{i},m_{j}\in \bZ_{\geq 0}$.

Now $(\KM, +, \cdot)$ is in fact a commutative semiring. We let $0$ be the
additive identity element in $\KM$ and $\emptyset$ is the multiplicative
identify. Note that $\cL\cdot 0=0$ for any
 $\cL\in \KM$ by definition.

For $\cL,\cD\in \KM$, $\cL\succ \cD$ means that there exists $\cB\in \KM$ such that
$\cL=\cB\cdot \cD$.
In addition, we write $\cL\supset \cD$ if there is $\cL_{1},\cL_{2}\in \KM$ such
that $\cL_{1}\succ \cD$ and $\cL_{1}+\cL_{2}=\cL$.

Let $\KM(\star)$ be the sub-monoid of $\KM$ generated by $\MYD(\star)$.
Note that $\KM(\star)$ is not a sub semi-ring of $\KM$.

For $\cL = \sum_{i=1}^{k}\cL_{i}\in \KM$ with $\cL_{i}\in \MYD$, we let
\[
  \ssign(\cL)= \Set{\ssign(\cL_{i})|i=1,\cdots,k}\text{ and }
  \lsign(\cL)= \Set{\lsign(\cL_{i})|i=1,\cdots,k}
\]

\subsection{Operations on $\KM(\star)$}
\def\npp{\#_{+}}
\def\nmm{\#_{-}}
% We use $\npp(\cS)$ to denote the number of $+$ and $\upp$ in the object $\cS$.
% For example $\npp(\tytb{\upp\umm\upp,+})=3$.

% The involution $\overline{\phantom{+}}$ defined in \eqref{eq:def.bar} extends to
% an in

The operation $\dagger$ defined in \eqref{eq:def.dagger} extends to an operation
\[
  \dagger \colon \KM\longrightarrow \KM
\]
on $\KM$ which adding a column on the left of the original marked Young diagram.


\subsubsection{Character twists on $\KM(D)$ and $\KM(B)$}
\label{sec:tchar.DB}
Recall that  $\bZ/2\bZ\times \bZ/2\bZ$ paramterizes
the characters of a indefinite orthogonal group $\rO(p,q)$:
\[
  \chi=(\chi^{+},\chi^{-})\longleftrightarrow (\bfone^{-,+})^{\chi^{+}}\otimes (\bfone^{+,-})^{\chi^{-}}.
\]

We define the action of $\bZ/2\bZ \times \bZ/2\bZ$ action on $\sfM$ by:
\[
  \bfrr \otimes \chi := \begin{cases}
    \overline{\bfrr} & \text{if } p+q \equiv 1 \text{ and }\chi^{+}p + \chi^{-} q\equiv 1 \pmod{2}\\
    \bfrr & \text{otherwise}%\text{if } \chi^{+}p + \chi^{-}q\equiv 0 \pmod{2} \\
  \end{cases}
\]
where $\bfrr\in \sfM, \chi=(\chi^{+},\chi^{-}) \in \bZ/2\bZ\times \bZ/2\bZ$ and
$(p,q)=\ssign(\bfrr)$.

The $\bZ/2\bZ\times \bZ/2\bZ$ action extends to actions on $\KM(D)$ and $\KM(B)$.
% Now $\bZ/2\bZ\times \bZ/2\bZ$ has an action on $\KM(D)$ and $\KM(B)$ defined as
% the following:

% Let $\chi\in \bZ/2\bZ\times \bZ/2\bZ$ and $\set{}$


% We use $\cB\cdot\cC$ to denote attach the local system $\cC$ below $\cB$ and the
% operation is distributive. And use $\cup$ or $+$ to denote the union of local systems
% systems.

\subsubsection{Twists on $\KM(C)$ and $\KM(M)$}
We define an involution $\maltese_{C} \colon \sfM\rightarrow \sfM$ on $\sfM$ by
\[
  \maltese(\bfrr):= \begin{cases}
    \overline{\bfrr} & \text{if } \abs{\bfrr} \equiv 2 \pmod{4}\\
    \bfrr & \text{otherwise}.
  \end{cases}
\]
The involution $\maltese$ extends to an involution on $\KM$.%$(C)$.
We will only apply $\maltese$ on $\cL\in \KM(C)$ or $\KM(M)$.

% Now fix a signature $(p,q)\in \bN\times \bN$ such that $p+q$ is even.
% For $\cL\in \KM(C)$, we define
% \[
%   \maltese_{(p,q)}(\cL) := \maltese^{\frac{p-q}{2}}(\cL)= \begin{cases}
%     \cL & \text{if } p-q \equiv 0 \pmod{4}\\
%     \maltese(\cL) & \text{if } p-q \equiv 2 \pmod{4}\\
%   \end{cases}
% \]
% For $\cL\in \KM(M)$, let
% \[
%   \maltese_{(p,q)}:=\maltese^{\frac{p-q-1}/2} =
%   \begin{case}
%     \cL & \text{if } p-q\equiv 1 \pmod{4}\\
%     \maltese{\cL} & \text{if } p-q\equiv 3 \pmod{4}
%   \end{case}
% \]

% \[
%   \maltese^{(p,q)}:=\maltese^{\frac{p-q+1}/2} =
%   \begin{case}
%     \cL & \text{if } p-q\equiv 3 \pmod{4}\\
%     \maltese{\cL} & \text{if } p-q\equiv 1 \pmod{4}
%   \end{case}
% \]
% We define the action of $\bZ/2\bZ \times \bZ/2\bZ$ action on $\sfM$ by:
% \[
%   \bfrr \otimes \chi := \begin{cases}
%     \overline{\bfrr} & \text{if } p+q \equiv 1 \text{ and }\chi^{+}p + \chi^{-} q\equiv 1 \pmod{2}\\
%     \bfrr & \text{otherwise}%\text{if } \chi^{+}p + \chi^{-}q\equiv 0 \pmod{2} \\
%   \end{cases}
% \]
% where $\bfrr\in \sfM, \chi=(\chi^{+},\chi^{-}) \in \bZ/2\bZ\times \bZ/2\bZ$ and
% $(p,q)=\ssign(\bfrr)$.

% The $\bZ/2\bZ\times \bZ/2\bZ$ action extends to actions on $\KM(D)$ and $\KM(B)$.

\subsubsection{Truncations}% on $\KM(D)$ and $\KM(B)$}.

\def\Tpp{\sfT^+}
\def\Tmm{\sfT^-}
\def\Tpm{\sfT^{\pm}}

We define the Truncation maps %$\Tpp$ and $\Tmm$
\[
  \Tpp \colon \KM\longrightarrow \KM
\]
such that for $\cL\in \MYD$
\[
\Tpp(\cL) = \begin{cases}
  \cL_{1} & \text{if there is $\cL_{1}\in \MYD$ such that $\cL_{1}\cdot + = \cL $}\\
  0 & \text{otherwise}
\end{cases}
\]
We define $\Tmm\colon \KM \longrightarrow \KM$ similarly such that
$\cL=\Tmm(\cL)\cdot - $ if $\cL\succ -$ and $\Tmm(\cL)=0$
otherwise for $\cL\in \MYD$.

To symplify the notation, we write $\sL^{\pm} := \Tpm(\sL)$ for $\cL\in \KM$.

\subsection{Theta lifts of local systems}


\subsubsection{Lift from type C to D}

For each signature $(p,q)\in \bN\times \bN$ such that $p+q$ is even, we define a map
\[
  \vartheta_{CD,(p,q)} \colon \KM(C)\rightarrow \KM(D)
\]
as the following:
Let $\cL\in \MYD(C)$. Let $(p_{1},q_{1})=\lsign(\cL)$ and
\[
(p_{0}, q_{0})  = (p,q) -\ssign(\cL)-(q_{1},p_{1}).
\]
We define
\[
  \vartheta_{CD,(p,q)}(\cL) =
  \begin{cases}
    (\dagger \maltese^{\frac{p-q}{2}}(\cL))\cdot \dagger_{p_{0}, q_{0}} &
    \text{if } p_{0}\geq 0 \text{ and } q_{0} \geq 0 \\
    0 & \text{otherwise}
  \end{cases}
\]


\subsubsection{Lift from type D to type C}

For each non-zero integer $n\in \bN$, we define a map
\[
  \vartheta_{DC,n} \colon \KM(D)\rightarrow \KM(C)
\]
as the following:
Let $\cL\in \MYD(D)$. Let $(p_{1},q_{1})=\lsign(\cL)$ and
$(p,q) = \ssign(\cL)$.
Let
\[
  n_{0} = n - (p+q)/2 - (p_{1}+q_{1})/2
\]
We define
\[
  \vartheta_{DC,n}(\cL) =
  \begin{cases}
    \maltese^{\frac{p-q}{2}}((\dagger \cL)\cdot \dagger_{n_{0}, n_{0}}) &
    \text{if } n_{0}\in \bZ_{\geq 0}\\
    \maltese^{\frac{p-q}{2}}(\dagger \cL^{+}  + \dagger \cL^{-})& \text{if } n_{0}=-1\\
    0&\text{otherwise}
  \end{cases}
\]
We remark that
$\maltese^{\frac{p-q}{2}}((\dagger \cL)\cdot \dagger_{n_{0}, n_{0}})= (\maltese^{\frac{p-q}{2}}(\dagger\cL))\cdot \dagger_{n_{0},n_{0}}$.

\trivial{
  We take a splitting $\rO(p,q)\times \Sp(n,\bR)$ in to the big metaplectic
  group $\Mp$ such that $\rO(p,\bC)\times\rO(q,\bC)$ acts on the Fock model linearly.
  The maximal compact $K_{\Sp(n,\bR)}= \rU(2n)$ acts on the Fock model by the
  character $\zeta=\det^{(p-q)/2}$.
  For the component of the rational nilpotent orbit $\Sp(2n,\bR)$,
  the factor of the component group is $\Sp$ if the corresponding row has odd
  length.
  Otherwise the component group is $\rO(p_{2k})\times\rO(q_{2k})$ where
  $(p_{2k},q_{2k})$ is the signature corresponding to the $2k$-rows.
  $\zeta|_{\rO(p_{2k})}= \det^{(p-q)k/2}_{\rO(p_{2k})}$ which is nontrivial if
  and only if $p-q\equiv 2\pmod{4}$ and $2k\equiv 2\pmod{4}$.
  This gives the formula above.
}


{
  \color{red}
  In the proof later, we don't care above the marks for rows with length longer
  than 1 in the most of the case. So we will omit $\maltese_{(p,q)}$.
  When there is no confusion, we will simply write $\vartheta$ for
  $\vartheta_{CD,(p,q)}$ and $\vartheta_{CD,(p,q)}$.
}


\subsubsection{Lift from type M to B}

For each signature $(p,q)\in \bN\times \bN$ such that $p+q$ is odd, we define a map
\[
  \vartheta_{MB,(p,q)} \colon \KM(M)\rightarrow \KM(B)
\]
as the following:
Let $\cL\in \MYD(M)$. Let $(p_{1},q_{1})=\lsign(\cL)$ and
\[
(p_{0}, q_{0})  = (p,q) -\ssign(\cL)-(q_{1},p_{1}).
\]
We define
\[
  \vartheta_{MB,(p,q)}(\cL) =
  \begin{cases}
    (\dagger \maltese^{\frac{p-q+1}{2}}(\cL))\cdot \dagger_{p_{0}, q_{0}} &
    \text{if } p_{0}\geq 0 \text{ and } q_{0} \geq 0\\
    0 & \text{otherwise}
  \end{cases}
\]


\subsubsection{Lift from type B to type M}

For each non-zero integer $n\in \bN$, we define a map
\[
  \vartheta_{BM,n} \colon \KM(B)\rightarrow \KM(M)
\]
as the following:
Let $\cL\in \MYD(D)$. Let $(p_{1},q_{1})=\lsign(\cL)$ and
$(p,q) = \ssign(\cL)$.
Let
\[
  n_{0} = n - (p+q)/2 - (p_{1}+q_{1})/2
\]
We define
\[
  \vartheta_{BM,n}(\cL) =
  \begin{cases}
    \maltese^{\frac{p-q-1}{2}}((\dagger \cL)\cdot \dagger_{n_{0}, n_{0}}) &
    \text{if } n_{0}\in \bZ_{\geq 0}\\
    \maltese^{\frac{p-q-1}{2}}(\dagger \cL^{+}  + \dagger \cL^{-})& \text{if } n_{0}=-1\\
    0&\text{otherwise}
  \end{cases}
\]
We remark that
$\maltese^{\frac{p-q-1}{2}}((\dagger \cL)\cdot \dagger_{n_{0}, n_{0}})= (\maltese^{\frac{p-q-1}{2}}(\dagger\cL))\cdot \dagger_{n_{0},n_{0}}$.

\trivial{
  For odd orthogonal-metaplectic group case, we still take the splitting such
  that $\rO(p,q)$ acts linearly.

  The maximal compact $\wtK_{\Sp(n,\bR)}= \widetilde{\rU(2n)}$ acts on the Fock model by the
  character $\zeta=\det^{(p-q)/2}$.

  Then
  $\det^{(p-q)/2}|_{\rO(p_{2k})\times \rO(q_{2k})}=\det^{\frac{(p-q)k}{2}}_{\rO(p_{2k})}\boxtimes
  \det^{\frac{(p-q)k}{2}}_{\rO(q_{2k})}. $
  Note that $p-q$ is an odd number.
  When $k$ is even the character on $\widetilde{\rO(p_{2k})}/\widetilde{\rO(q_{2k})}$ are
  $\bfone$ or $\det$. When $k$ is odd the character would be $\det^{\half}$ or
  $\det^{\half+1}$.

  We assume the default character on $2k$-rows is $\det^{\frac{k}{2}}$, and
  we use $+/-$ to mark the row. Otherwise, we use $\upp/\umm$ to mark the rows.
}

\subsection{Examples and remarks}

\subsubsection{Example}
\begin{eg}
  Let
  $\cT :=\dagger\dagger (\dagger_{2,1}) + \dagger\dagger(\dagger_{{1,2}})$ and $\cP= \ddagger_{1,3}$.
  Then
  \[
    \cT\cdot \cP = \tytb{\uup\uum\uup,\uup\uum\uup,\uum\uup\uum,+,=,=,=}
    \cup
    \tytb{\uup\uum\uup,\uum\uup\uum,\uum\uup\uum,+,=,=,=}.
  \]
\end{eg}


% For each local system $\cL$ of type B/D, define $\pcL$ (resp. $\ncL$) be the
% part of $\cL$ obtained by deleting a ``$+$''-symbol (resp. ``$-$''=symbol) among 1-rows.
% $\pcL:=0$ (resp. $\ncL:=0$) if there is no ``$+$''-symbol (resp.
% ``$-$''-symbol).

% For an local system $\cL$, $\cL\succeq\cC$ means there is a
% factorization of $\cL = \cB\cdot \cC$;
% $\cL\supset \cC$ means there is an irreducible component $\cL_{1}$ of $\cL$ such
% that $\cL_{1}\succeq \cC$.


% \subsubsection{}

%\subsection{Definition of $\eDD$}
\subsubsection{Signs of the local systems obtained by iterated lifting}


Suppose $\cL\in \KM$ is obtained by iterated theta lifting and character
twisting in \Cref{sec:tchar.DB}.
Then the set $\lsign(\cL)$ has restricted possibilities.
First, $\ssign(\cL)$ is always a singleton.

When $\cO$ is of type B/D, $\lsign(\cL)$ is a singleton $\set{(p_{1},q_{1})}$
where
\[
  (p_{1},q_{1})= \ssign(\cL) - (n_{0},n_{0}) \quad \text{and} \quad 2n_{0} = \abs{\eDD(\cO)}.
\]

When $\cO$ is of type C/M,
$\lsign( \cL ) =\set{ (p_{1},q_{1})}$ is a singleton in the most of the case.
The $\lsign(\cL)$
may have two elements if $\cL$ is obtained by good generalized lifting, and it has the form
$\lsign( \cL ) = \set{(p_{1}+1,q_{1}),(p_{1},q_{1}+1)}$.

% Another important invariant is $\bsign( \cL )$ which indicates the signature of
% the local system of length one rows in $\cL$.


\section{Proof of main theorem in the type C/D case}
We do induction on the number of columns of the nilpotent orbits.
\subsection{Notation}
In this section, $\cO$ is always an nilpotent orbit of type D in $\dpeNil(D)$.
We always let
\[
  \begin{split}
    \cO &= (C_{2k+1}, C_{2k}, C_{2k-1}, \cdots,C_{1},C_{0}),\\
    \cOp &=\eDD(\cO)=(C_{2k}, C_{2k-1},\cdots,C_{1},C_{0} ).
    \end{split}
\]
When  $k\geq 1$, we let
\[
    \cOpp  = \eDD^{2}(\cO) =
    \begin{cases}
      (C_{2k-1},\cdots,C_{1},C_{0} ) & \text{when  $C_{2k}$ is even},\\
      (C_{2k-1}+1,\cdots,C_{1},C_{0} ) & \text{when  $C_{2k}$ is odd}.\\
    \end{cases}
\]

For a $\uptau\in \drc(\cO)$, we let
\[
(\uptaup,\upepsilon)=\eDD(\uptau) \quad\text{ and }\quad
   (\uptaupp,\upepsilon')=\eDD(\uptaup) \text{ (when $k\geq 1$)}.
 \]

 \subsubsection{Main proposition}
The aim of this section is to prove the following main proposition holds for
$\cO\in \dpeNil(D)$.
% with only one column.

%Inductively, we will assume the following claims holds for $\cOpp$.

\begin{prop}\label{prop:CD}
   We have:
  \begin{enumS}
    \item \label{p:drcls.0} $\cL_{\uptau}$ is non-zero. In particular,
    $\uppi_{\uptau}\neq 0$.
    \item %\label{p:drcls.4}
    Suppose $\uptaup_{1}\neq \uptaup_{2}\in\drc(\cOp)$, then $\uppi_{\uptaup_{1}}\neq \uppi_{\uptaup_{2}}$.
    \item \label{p:drcls.4}
    Suppose $\uptau_{1}\neq \uptau_{2}\in\drc(\cO)$, then $\uppi_{\uptau_{1}}\neq \uppi_{\uptau_{2}}$.
    \item \label{p:drcls.1}
    The local system $\cL_{\uptau}$ is disjoint with their determinant twist:
    \begin{equation}\label{eq:LS.dis}
      \set{\cL_{\uptau} | \uptau \in \drc(\cO)} \cap \set{\cL_{\uptau}\otimes \det| \uptau\in \drc(\cO)} = \emptyset.
    \end{equation}
    % \item \label{p:drcls.4}
    % Suppose $\uptau_{1}\neq \uptau_{2}$, then $\uppi_{\uptau_{1}}\neq \uppi_{\uptau_{2}}$.
    % \item \label{p:drcls.2}
    % Compatible with LS lifting: Suppose $\uptaup_{1}\in [\cL_{\uptaup}]$ and
    % $\uptaup_{1}\neq \uptaup$, then
    % there is (a unique) $\uptau_{1}\in \cL_{\uptau}$ such that
    % $\eDD(\uptau_{1}) = (\uptaup_{1},\upepsilon)$ and $\uptau_{1}\neq \uptau$.

    % \item  \label{p:drcls.3}
    % Suppose $\upepsilon=0$.
    % $\cL_{\uptau}\neq \cL_{\uptau}\otimes \bfonepn$.
    %Therefore $\lUnip(\cO)=\Unip(\cO)$.
    \item  $x_{\uptau}$ determine the factorization of local system described in
    \Cref{tab:ls.factor}:
    \[
      \cL_{\uptau} = \sum_{i=-k}^{k} \cB_{\uptau,i}\cdot \cD_{\uptau,i}.
    \]
    %Here $i$ running over a finite index set.
    The key properties of the factorization are
    \begin{equation}\label{eq:bd.prop}
      \begin{split}
        (\cB_{\uptau,i}\cdot\cD_{\uptau,i})^{+} &=
        \begin{cases}
          \cB_{\uptau,i}\cdot (\cD_{\uptau,i})^{+} & \text{for } i \geq 0\\
          0 & \text{for }i < 0\\
        \end{cases}
        ,\\
        (\cB_{\uptau,i}\cdot\cD_{\uptau,i})^{-} &=
        \begin{cases}
          0 &\text{for } i > 0\\
          \cB_{\uptau,i}\cdot (\cD_{\uptau,i})^{-} &\text{for } i \leq 0\\
        \end{cases}\\
        \text{and } \quad \lsign(\cD_{\uptau,i}) & = \ssign(\bfxx_{\uptau}).
      \end{split}
    \end{equation}

    % \begin{equation}\label{eq:ls.factor}
  %   \begin{array}{c|c|c|c}
  %     x_{\uptau} & \cL_{\uptau} & \cD_{\uptau} &\\
  %     \hline
  %     s & \displaystyle\sum_{i=1}^{2}\cB_{\uptau,i}\cdot\cD_{\uptau,i} & \cD_{\uptau,1}=\tytb{=,=},
  %                                                                        \cD_{\uptau,2}= \tytb{\uum\cdots,=}
  %     &\begin{minipage}{10em}exactly one of $\cB_{\uptau,1}$ and $\cB_{\uptau,2}$ is non-zero \end{minipage} \\
  %     \hline
  %     r & \displaystyle\sum_{i=1}^{2}\cB_{\uptau,i}\cdot\cD_{\uptau,i} & \cD_{\uptau,1}=\tytb{+,+},
  %                                                                        \cD_{\uptau,2} = \tytb{\uup\cdots,+}
  %     & \begin{minipage}{10em}exactly one of $\cB_{\uptau,1}$ and $\cB_{\uptau,2}$ is non-zero \end{minipage} \\
  %     \hline
  %     %\ytb{sd\cdots,\vdots,s,c} & \displaystyle\sum_{i=1}^{2}\cB_{\uptau,i}\cD_{\uptau,i} &  \cD_{\uptau,1} = \ytb{\uum\cdots,+},
  %     %                                                                                      \cD_{\uptau,2}=\ytb{\uup\cdots,=}
  %     %                                         &  \#s \geq 0\\
  %     %\hdashline
  %     %\ytb{\star\cdots,\vdots, c} & \cB_{\uptau,1}\cD_{\uptau,1} &  \cD_{\uptau,1} = \ytb{\uum\cdots,+} & \\
  %     %& & & \cB_{\uptau,2}\neq \emptyset \text{only if
  %     %      } \bfxx_{\uptau} = (s\cdots sc)^{t}\\
  %    c
  %                    & \displaystyle\sum_{i=1}^{2} \cB_{\uptau,i}\cdot\cD_{\uptau,i}
  %                                   &  \cD_{\uptau,1} = \tytb{\uum\cdots,+},
  %                                     \cD_{\uptau,2}= \tytb{\uup\cdots,=}
  %                                                  & \cB_{\uptau,1}\neq \emptyset.
  %     \\
  %     %& & & \cB_{\uptau,2}\neq \emptyset \text{only if
  %     %      } \bfxx_{\uptau} = (s\cdots sc)^{t}\\
  %    \hline
  %    d & \displaystyle\sum_{i=1}^{2}\cB_{\uptau,i}\cdot\cD_{\uptau,i} &  \cD_{\uptau,1} = \tytb{\uum\cdots,+},
  %                                                                                            \cD_{\uptau,2}=\tytb{\uup\cdots,-}
  %                                              &\cB_{\uptau,1}\text{ and }\cB_{\uptau,2}\neq \emptyset\\
  %   \end{array}
  % \end{equation}
  \begin{table}[pb]
    \[
      \begin{array}{c|c|c|c}
        x_{\uptau} & \cL_{\uptau} & \cD_{\uptau} &\\
        \hline
        s & \displaystyle\sum_{i=0}^{k}\cB_{\uptau,-i}\cdot\cD_{\uptau,-i}
                                  & %\cD_{\uptau,1}=\tytb{=,=},
                                    \cD_{\uptau,i}=\begin{cases}
                                      0 & \text{when } i>0\\
                                    \tytb{\uum\cdots\uum,=}  & \text{when
                                    } i\leq 0
                                  \end{cases}
          & \exists i\geq 0 \text{ s.t. } \cB_{\uptau,-i}\neq 0 \\
        \hline
        r & \displaystyle\sum_{i=0}^{k}\cB_{\uptau,i}\cdot\cD_{\uptau,i}
                                  & \cD_{\uptau,i}=
                                    \begin{cases}
                                      \tytb{\uup\cdots\uup,+} & \text{when } i\geq 0\\
                                      0&  \text{when } i<0
                                    \end{cases}
                                  & \exists i\geq 0 \text{ s.t. } \cB_{\uptau,i}\neq 0 \\
        \hline
        c
                   & \displaystyle\sum_{i=-k}^{k} \cB_{\uptau,i}\cdot\cD_{\uptau,i}
                                  &  \cD_{\uptau,i} =\begin{cases} \tytb{\uum\cdots\uum,+} & \text{when
                                    } i\geq 0,\\
                                    \tytb{\uup\cdots\uup,=} & \text{when }i <0
                                  \end{cases}
                                  &  \exists i\geq 0 \text{ s.t. } \cB_{\uptau,i}\neq 0.
        \\
        \hline
        d
                   & \displaystyle\sum_{i=-k}^{k} \cB_{\uptau,i}\cdot\cD_{\uptau,i}
                                  &  \cD_{\uptau,i} =\begin{cases} \tytb{\uum\cdots\uum,+} & \text{when
                                    } i\geq 0,\\
                                    \tytb{\uup\cdots\uup,-} & \text{when }i <0
                                  \end{cases}
                                  &  \exists i,j\geq 0 \text{ s.t. }
                                    \begin{cases}
                                      \cB_{\uptau,i}\neq 0,\\% &\\ %\text{and} \\
                                      \cB_{\uptau,-j}\neq 0.% &.  \\
                                    \end{cases}

      \end{array}
    \]
  In the above table $\uup\cdots\uup$ or $\uum\cdots\uum$ has length $2i+1$
  where we do not specify the associated character of the local system on these rows when $i\neq 0$.

  (The associated character depends on $\uptau$ and can be
  specified inductively.)

  The local system of $\cD_{\uptau,0}$ is specified in the following
  table:
  \begin{equation}\label{eq:ls.srcd}
   \begin{array}{c|c|c|c|c}
      x_{\uptau} &  s & r & c &d \\
      \hline
      \cD_{\uptau,0} & \tiny \tytb{=,=} &\tiny \tytb{+,+} & \tiny \tytb{+,=} &\tiny \tytb{+,-}
    \end{array}
  \end{equation}
    \caption{Factorization of the local system}
    \label{tab:ls.factor}
    \label{eq:ls.factor}
  \end{table}
  \item
  In particular, we have:
  \begin{enumT}
    \item $x_{\uptau} = s$, then $\pcL_{\uptau} = \ncL_{\uptau} = 0$.
    \item $x_{\uptau} = r/c$, then $\pcL_{\uptau} \neq 0$ and $\ncL_{\uptau}=0$.
    \item $x_{\uptau} = d$, then $\pcL_{\uptau}\neq 0$ and $\ncL_{\uptau}\neq 0$.
  \end{enumT}
  % \item
  % When $x_{\uptau}=c$, $\cD_{\uptau,2}\neq \emptyset$ only if
  % $\bftt_{\uptau} = \ytb{sd\cdots, \vdots,s,c}$.
  \item If $\cOp \leadsto \cOpp$ is a usual descent, we have a simpler
  factorization:
  \begin{equation}\label{eq:d.factor}
    \cL_{\uptau} = \cB_{\uptau,0}\cdot \cD_{\uptau,0}
  \end{equation}
  where $\cD_{\uptau,0}$ is given by \eqref{eq:ls.srcd}.

  In particular, $\cD_{\uptau,0}=\tytb{=,=}$ (resp.
  $\cD_{\uptau,0}=\tytb{+,+}$) if and only if all 1-rows
  of $\cL_{\uptau}$ have ``$=$'' marks (resp. ``+'' marks), i.e. $x_{\uptau}=s$
  (resp. $x_{\uptau}=r$).
  \item When $\cOp$ is noticed, the map
  $\cL\colon \drc( \cOp )\longrightarrow \LLS(\cOp)$ is a bijection.
  \item When $\cO$ is noticed, the map
  $\cL\colon \drc(\cO)\longrightarrow \LLS(\cO)$ is a bijection.
    \item When $\cO$ is +noticed, the maps
    % $\pUpsilon_{1}:=\pr_{1}\circ \Upsilon$
    \[
      \begin{tikzcd}[row sep=0em]
        \pUpsilon_{\cO} \colon \set{\cL_{\uptau}|\pcL_{\uptau}\neq 0} \ar[r] & \set{\pcL_{\uptau}}\\
       \cL_{\uptau} \ar[r,maps to] & \pcL_{\uptau}& \text{and}\\
        \nUpsilon_{\cO} \colon \set{\cL_{\uptau}|\ncL_{\uptau}\neq 0} \ar[r] & \set{\ncL_{\uptau}}&\\
       \cL_{\uptau} \ar[r,maps to] & \ncL_{\uptau}
      \end{tikzcd}
    \]
    are injective.
    % \item If $C_{2k+1}-C_{2k}\geq 2$ and $\bfxx_{\uptau}$ ends with $d$, $\cL_{\uptau}$ has an irreducible
    % component $\succeq \dagger_{(1,1)}$. In particular, if $\cO$ is +noticed and
    % $\bfxx_{\uptau}$ ends with $d$, we have
    % $\cL_{\uptau}\supset \dagger_{(1,1)}$.
    \item When $\cO$ is noticed, the map
    \begin{equation}\label{eq:up}
      \begin{tikzcd}[row sep=0em]
        \Upsilon_{\cO} \colon \set{\cL_{\uptau}|\pcL_{\uptau}\neq 0} \ar[r] & \set{(\pcL_{\uptau},\ncL_{\uptau})|\pcL_{\uptau}\neq 0} \\
        \cL_{\uptau} \ar[r,maps to] & (\pcL_{\uptau},\ncL_{\uptau})
      \end{tikzcd}
    \end{equation}
    is injective.\footnote{In fact, we only need to know whether $\ncL_{\uptau}$ is
      non-zero or not.}
  \end{enumS}
\end{prop}

Note that, $\drc(\cO)$ counts the unipotent representations of special
orthogonal groups a priori. But \Cref{prop:CD}~\eqref{p:drcls.1}, implies that the representation
$\uppi_{\uptau}$ and $\uppi_{\uptau}\otimes \det$ are two different
representations of the orthogonal groups. Therefore $\drc(\cO)\times \bZ/2\bZ$
counts the set of unipotent representations of orthogonal groups.

The factorization \eqref{eq:ls.srcd} also holds in more general cases, we
indicate when it holds in the proof of \Cref{prop:CD}.

The main theorem would be a consequence of \Cref{prop:CD}.

\subsubsection{How to think about the factorization of local systems}
When % $C_{2k}$ is even, i.e.
$\cOp\leadsto \cOpp$ is the usual descent,
the local system $\cL_{\uptau}$ could be compared to a water hydra, see in
\Cref{fig:wh}.
\begin{figure}[pb]
\includestandalone[width=.5\textwidth]{hydra}
 % \includegraphics[scale=0.5]{abc.pdf}
  \caption{Freshwater hydra}
  \label{fig:wh}
\end{figure}

The tentacles part $\cT_{\uptau} = \dagger \cL_{\uptaup}$
could be reducible. The peduncle part $\cP_{\uptau} := \cL_{\bfxx_{\uptau}}$ and
$\cL_{\uptau} = \cT_{\uptau}\cdot \cP_{\uptau}$.

On the other hand, $\cL_{\uptau}$ also could be factorized into the body part
and the basal
disk part:
\[
  \cL_{\uptau} = \cB_{\uptau,0}\cdot \cD_{\uptau,0}.
\]
Here $\cD_{\uptau,0}$ is given in \eqref{eq:ls.srcd}.
% ) which will be used in the
% generalized descent case.

\medskip

When $\cOp\leadsto \cOpp$ is a generalized descent, the situation is more
complicated as the local system $\cL_{\uptau}$ may consists of several ``hydras''.

To ease the notation, we will set $\cB_{\uptau} :=\cB_{\uptau,0}$  and
$\cD_{\uptau} :=\cD_{\uptau,0}$
in the following proof.
% %On the other hand, $x_{\uptau}$ could been detected from the following list of 1-rows:
% \[
% \begin{array}{c|c|c|c|c}
%   x_{\uptau} &  s & r & c &d \\
%   \hline
%   \cD_{\uptau} & \tiny \ytb{=,=} &\tiny \ytb{+,+} & \tiny \ytb{+,=} &\tiny \ytb{+,-}
% \end{array}
% \]

% Here $x_{\uptau}=s$ (resp. $x_{n}=r$) is equivalent to that only $=$ (resp. $+$)
% occurs in 1-rows of $\cL_{\uptau}$.


% Therefore
% \[
% \cL_{\uptau} = \vartheta(\cL_{\uptaup})\otimes (\bfone^{+,-})^{\upepsilon}\mapsto (\cL_{\uptaup}, \upepsilon)
% \]
% is a bijection on $\LLS(\cO,\rO(p,q))$
% In particular, \Cref{prop:CD.1} \ref{p:drcls.1}% ,  \ref{p:drcls.2}
% and \ref{p:drcls.3} can be read
% off from the character twist action on the local systems in the above
% table.

% \Cref{prop:CD.1}~\ref{p:drcls.4} follows from the fact that $\lUnip(\cO')=\Unip(\cO')$.
% Note that in this case $\uptaupp$ is the subdiagram of $\uptau$ and $*$'s above
% $x_{1}$ must be $\bullet$.

% \subsubsection{descent of nonspecial dot-r-c diagram} \label{sec:dd.nonsep}
% In the most of the case the $\uptaup$ is obtained by deleting the most left
% column of $\uptau$ and then apply the dot-s switch algorithm.

% Now we list the exceptional case:
% \[
% \begin{array}{c|c|c}
%   \ytb{{x_{0}}{y_{1}},{x_{1}}{y_{2}},\vdots,{x_{n}}}= & \ytb{rc,{x_{1}}d,\vdots,{x_{n}}} & \ytb{cc,dd,\ ,\ }\\
%   \hline
%   \ytb{{x_{\uptaupp}},{z_{2}}}= & \ytb{r,r} & \ytb{r,c}\\
% \end{array}
% \]

% \subsubsection{specail-nonspecial twist} \label{sec:nsp.twist}
% Let $\uptau$ and $\uptaup$ be two diagram in $\drc(\cO)$ with the following
% shape. We assume $\cOp\leadsto \cOpp$ is a usual descent.
% Fix the leg $\bfxx = x_{1}\cdots x_{n}$.
% \begin{claim}
%   There is an bijection between diagrams
% \[
%   \uptau_{1} = \ytb{\bullet{w_{0}}\cdots,\bullet{w_{1}}{w_{2}}\cdots,{x_{1}},{x_{2}},\vdots,{x_{n}}}\times
%   \ytb{\bullet\cdots,\bullet,\ ,\ ,\ ,\ }
%   \longleftrightarrow
%   \uptau_{2} = \ytb{\bullet{y_{0}}\cdots\cdots,{x_{0}}{y_{1}}{y_{2}}\cdots,{x_{1}}{y_{3}},{x_{2}},\vdots,{x_{n}}}\times
%   \ytb{\bullet\cdots,\ ,\ ,\ ,\ ,\ }
% \]
% such that
% $\uptaup_{2} = \eDD(\uptau_{2})$ is the non-special twist of $\uptaup_{1} = \eDD(\uptau_{1})$ (see \eqref{eq:nonsp.twist}).
% Moreover, $\ssign(\uptau_{1}) = \ssign(\uptau_{2})$.
% \end{claim}
% \begin{proof}
%   This is by \eqref{eq:nonsp.twist} and \eqref{eq:d.nonsp.e}.
%   {\color{red}Detail is needed to be written.}
% \end{proof}

%\subsection{Proof of \Cref{thm:count} for type C/D}

\subsection{The initial case:}\label{sec:pfDC.init}
When $k=0$, $\cO = (C_{1}=2c_{1},C_{0}=2c_{0})$ has  at most
two columns.
Now $\uppi_{\uptaup}$ is the trivial representation of $\Sp(2c_{0},\bR)$.
The lift $\uppi_{\uptaup} \mapsto \Thetab(\uppi_{\uptaup})$ is in fact a stable
range theta lift.%, and therefore $\uppi_{\uptau}\neq 0$.
For $\uptau\in \drc(\cO)$, let $(p_{1},q_{1}) = \ssign(\bfxx_{\uptau})$ and
$x_{\uptau}$ be the foot of $\uptau$. It is easy to see that  (using associated
character formula)
\[
  \cL_{\uppi_{\uptau}} = \cT_{\uptau}\cdot \cP_{\uptau},
  \text{ where }\cT_{\uptau}:=\dagger \dagger_{c_{0},c_{0}},
  \text{ and } \cP_{\uptau}:=\begin{cases}
    \ddagger_{p_{1},q_{1}}&  x_{\uptau} \neq d,\\
    \dagger_{p_{1},q_{1}}& x_{\uptau} = d.\\
  \end{cases}
\]
Note that $\cL_{\uppi_{\uptau}}$ is irreducible.

Let $\cO_{1} = (2(c_{1}-c_{0}))\in \dpeNil(D)$. Using the above formula,
one can check that the following maps are bijections
\[
  \begin{tikzcd}[row sep=0em]
    \drc(\cO_{1}) &\ar[l] \drc(\cO) \ar[r] & \LLS(\cO)\\
    \bfxx_{\uptau} & \ar[l,maps to] \uptau \ar[r,mapsto] & \cL_{\uptau}.
  \end{tikzcd}
\]
To see that $\drc(\cO)\ni\uptau\mapsto \cL_{\uptau}$ is an injection, one check the
following claim holds:
\begin{claim}\label{c:init.CD}
  The map $\drc(\cO_{1})\longrightarrow \bN^{2}\times \bZ/2\bZ$ given by
  $\uptau\mapsto (\ssign(\uptau),\upepsilon_{\uptau})$ is injective (see
  \Cref{sec:upepsilon} for the definition of $\upepsilon_{\uptau}$). Moreover,
  $\upepsilon_{\uptau}=0$ only if $\ssign(\uptau)\geq (0,1)$. \qedhere
\end{claim}


Moreover, $\cL_{\uptau}\otimes \det \notin \LLS(\cO)$ for any $\uptau\in \drc(\cO)$.
In fact, if $\ssign(\bfxx_{\uptau})\succeq (1,0)$, $\cL_{\uptau}$
on the 1-rows with $+$-signs is trivial and $\cL_{\uptau}\otimes \det$ has
non-trivial restriction. When $\ssign(\bfxx_{\uptau})\nsucceq (1,0)$,
$\bfxx_{\uptau}=s\cdots s$, all 1-rows of $\cL_{\uptau}$ are marked by ``$=$'' and
all 1-rows of $\cL_{\uptau}\otimes \det$ are  marked by ``$-$''.

Obviously we have a factoriation of the peduncle:
\begin{equation}\label{eq:ped.factor1}
  \cP_{\uptau} = \cL_{\bfxx_{\uptau}} = \cB_{\bfxx_{\uptau}}\cdot \cD_{\bfxx_{\uptau}}
  \text{ with }
  \cB_{\bfxx_{\uptau}}:=\begin{cases}
    \ddagger_{\ssign(x_{1}\cdots x_{n-1})} & \text{when }x_{\uptau} \neq d\\
    \dagger_{\ssign(x_{1}\cdots x_{n-1})} & \text{when }x_{\uptau} = d\\
  \end{cases}
\end{equation}
and $\cD_{\bfxx_{\uptau}}$ given in \eqref{eq:ls.srcd}.
Now
\begin{equation}\label{eq:bd.factor2}
  \cL_{\uptau} = \cB_{\uptau}\cdot \cD_{\uptau} \text{ with
  } \cB_{\uptau}:=\cT_{\uptau} \cdot \cB_{\bfxx_{\uptau}} \text{ and }
  \cD_{\uptau} := \cD_{\bfxx_{\uptau}}.
\end{equation}
and so the factorization \eqref{eq:d.factor} holds.
% For the above discussion, one can immediately deduce the factorization of local
% system of $\cL_{\uptau}$ in \eqref{eq:ls.factor}.



Therefore, our main theorem, \Cref{thm:count}, holds for $\cO$.


\subsection{The descent case}\label{sec:pf.ds.CD}
Now we assume $k\geq 1$ and $C_{2k}$ is even. In this case $\cOp\leadsto \cOpp$
is a usual descent.

\subsubsection{Unipotent representations attached to $\cOp$}
Therefore, $\LS(\cOpp)\longrightarrow \LS(\cOp)$ given by
$\cL\mapsto \vartheta(\cL)$ is an injection between abelian groups.


For $\cLpp\in \LLS(\cOpp)\sqcup \LLS(\cOpp)\otimes \det$, let $(p_{1},q_{1}) = \lsign(\cL)$ and
$(p_2,q_2)=(C_{2k}-q_{1},C_{2k}-p_{1})$. Then
\[
  \vartheta(\cLpp) = \dagger\cLpp \cdot \dagger_{p_{2},q_{2}}.
\]

By \eqref{eq:LS.dis}, we have a injection
\begin{equation}\label{eq:LS.CD.inj}
  \begin{tikzcd}[row sep=0em]
    \LLS(\cOpp)\times \bZ/2\bZ \ar[r] &\LLS(\cOp)\\
    (\cLpp,\upepsilon') \ar[r,maps to] & \vartheta(\cLpp\otimes \det^{\upepsilon'}).
  \end{tikzcd}
\end{equation}
In particular, we have $\cL_{\uptaup}\neq 0$ and $\uppi_{\uptaup}\neq 0$ for every
$\uptaup\in \drc(\cOp)$.



\subsubsection{Unipotent representations attached to $\cO$}
Now suppose $\uptaup_{1}\neq \uptaup_{2}\in \drc(\cOp)$ and
$\cL_{\uptaup_{1}}=\cL_{\uptaup_{2}}$. By \eqref{eq:LS.CD.inj},
$\upepsilon'_{1}=\upepsilon'_{2}$ and $\cL_{\uptaupp_{1}}=\cL_{\uptaupp_{2}}$.
In particular,  $\uptaupp_{1}$ and $\uptaupp_{2}$  have the same signature. %, say $(p'',q'')$.% are dot-r-c diagrams for the
% same real orthogonal group.
By \Cref{lem:ds.CD} and the induction hypothesis, $\uptaupp_{1}\neq \uptaupp_{2}$ and so
$\uppi_{\uptaupp_{1}}\otimes {\det}^{\upepsilon'_{1}} \neq \uppi_{\uptaupp_{2}}\otimes {\det}^{\upepsilon'_{2}}$.
Now the injectivity of theta lifting yields
\[
  \uppi_{\uptaup_{2}} = \Thetab(\uppi_{\uptaupp_{1}}\otimes {\det}^{\upepsilon'_{1}})
  \neq \Thetab(\uppi_{\uptaupp_{2}}\otimes {\det}^{\upepsilon'_{2}}) = \uppi_{\uptaup_{2}}
\]
Suppose $\cOp$ is noticed. Then $\cOpp = \eDD(\cOp)$ is noticed by definition
and
$(\uptaupp,\upepsilon')\mapsto \Ch(\uppi_{\uptaupp}\otimes{\det}^{\upepsilon'})$
is an injection into $\LS(\cOpp)$.
Now \eqref{eq:LS.CD.inj} implies $\uptaup\mapsto\cL_{\uptaup}$ is also an injection.

Suppose $\cOp$ is quasi-distingushed. Then $\cOpp = \eDD(\cOp)$ is
quasi-distingushed by definition
and
$(\uptaupp,\upepsilon')\mapsto \Ch(\uppi_{\uptaupp}\otimes{\det}^{\upepsilon'})$
is  a bijection with $\LSaod(\cOpp)$.
Since the component group of each K-nilpotent orbit in $\cOp$ is naturally
isomorphic to the component group of its descent. So we deduce that $\uptaup \mapsto \cL_{\uptaup}$
is a bijection onto $\LSaod(\cOp)$.

This proves the main proposition for $\cOp$.


Now let $\uptau\in \drc(\cO)$. % and $(p_{0},q_{0}):=\ssign(\bfxx_{\uptau})$.
By \eqref{eq:sp-nsp-sig}, we have
\begin{equation}\label{eq:LS.D.ds}
  \cL_{\uptau} =  \cT_{\uptau}\cdot \cP_{\uptau}\neq 0, \text{ where
  }\cT_{\uptau} := \dagger\cL_{\uptaup}, \text{ and } \cP_{\uptau}:=\cL_{\bfxx_{\uptau}}.
\end{equation}
In particular, $\uppi_{\uptau}\neq 0$ and we could determine $\bfxx_{\uptau}$ from
the marks on the 1-rows of $\cL_{\uptau}$.

The factorization of $\cL_{\uptau}$ in the fashion of \eqref{eq:ls.factor} is also given by
\eqref{eq:ped.factor1} and \eqref{eq:bd.factor2}.
% When $\upepsilon=0$, $(p_{0},q_{0})\succeq \ssign(d)=(1,1)$ and so
% $\cL_{\uptau}\neq \cL_{\uptau}\otimes \bfone_{\ssign(\uptau)}^{+,-}$.

Now suppose $\uptau_{1} \neq \uptau_{2}$ and
$\cL_{\uptau_{1}}=\cL_{\uptau_{2}}$. By \eqref{eq:LS.D.ds}, the result in
\Cref{sec:pfDC.init} and
\Cref{lem:sp-nsp.D}~\eqref{lem:sp-nsp.D.2},
we have
$\bfxx_{\uptau_{1}}=\bfxx_{\uptau_{2}}$,  $\upepsilon_1=\upepsilon_{2}$ and
$\uptaupp_{1}\neq \uptaupp_{2}$.
By \eqref{eq:LS.D.ds}, we have
$\cL_{\uptaup_{1}}=\cL_{\uptaup_{2}}$.
Applying the main proposition of $\cOp$, we have
$\upepsilon'_{1}=\upepsilon'_{2}$ and $\uptaup_{1}\neq \uptaup_{2}$.
Now by the injectivity of theta lifing, we conclude that
\[
  \uppi_{\uptau_{2}} = \Thetab(\uppi_{\uptaup_{1}})\otimes (\bfone^{+,-})^{\upepsilon_{1}}
  \neq \Thetab(\uppi_{\uptaup_{2}})\otimes (\bfone^{+,-})^{\upepsilon_{2}} = \uppi_{\uptaup_{2}}
\]

\subsubsection{}

Note that $\cL_{\uptau}$ is obtained from $\cL_{\uptaup}$ by attaching the peduncle
determined by $\bfxx_{\uptau}$. The claims about noticed and
quasi-distinguished orbit are easy to verify. We leave them to the reader.


%\subsection{}

% For a dot-r-c diagram $\uptau\in \drc(D)$, we have the following three cases:
% (The top rows are either empty or $w_{0}\neq \emptyset$.)


% {\bf Case I: } $x_{1}\neq \emptyset$. In this case $\cOpp\leadsto \cOp$ is usual
% lift and $\upepsilon'=0$.
% \[
% \begin{tikzpicture}
%   \matrix[matrix of math nodes, row sep={1.5em,between origins},column sep={1.5em,between origins}]{
%   \cdots & \cdots & \cdots & \cdots &                          & \cdots & \cdots & \cdots & \cdots & \\
%   *      & *      & *      & *      & \times                   & w_{0}  & *      & *      & *      & \\
%   x_{1}  &        &        &            &                          &        &        &        &        & \\
%   \vdots &        &        &        &                          &        &        &        &        & \\
%   x_{n}   &        &        &        &                          &        &        &        &        & \\
%   };
% \end{tikzpicture}
% \]
% We define $\bfxx_{\uptau}=\bftt_{\uptau} = \ytb{{x_{1}},\vdots,{x_{n}}}$.


% {\bf Case II:} $y_{2}\neq \emptyset$. In this case $\cOpp\leadsto \cOp$ is usual
% lift and $\upepsilon'=1$.
% \[
% \begin{tikzpicture}
%   \matrix[matrix of math nodes, row sep={1.5em,between origins},column sep={1.5em,between origins}]{
%   \cdots & \cdots & \cdots & \cdots &                          & \cdots & \cdots & \cdots & \cdots & \\
%   *      & *      & *      & *      & \times                   & w_{0}  & *      & *      & *      & \\
%   x_{0}  & y_{1}  &  *     &  \cdots    &                          &        &        &        &        & \\
%   x_{1}  & y_{2}  &        &        &                          &        &        &        &        & \\
%   \vdots &        &        &        &                          &        &        &        &        & \\
%   x_{n}  &    &        &        &                          &        &        &        &        & \\
%   };
% \end{tikzpicture}
% \]
% We define $\bfxx_{\uptau}=\bftt_{\uptau} = \ytb{{x_{1}},\vdots,{x_{n}}}$.


% {\bf Case III:} $y_{1}\neq \emptyset$. In this case $\cOpp\leadsto \cOp$ is a
% generalized lift and $\upepsilon'=0$.
% \[
% \begin{tikzpicture}
%   \matrix[matrix of math nodes, row sep={1.5em,between origins},column sep={1.5em,between origins}]{
%   \cdots & \cdots & \cdots & \cdots &                          & \cdots & \cdots & \cdots & \cdots & \\
%   *      & *      & *      & *      & \times                   & w_{0}  & *      & *      & *      & \\
%   x_{1}  & y_{1}  &  *     &  \cdots    &                          &        &        &        &        & \\
%   \vdots &        &        &        &                          &        &        &        &        & \\
%   x_{n}  &    &        &        &                          &        &        &        &        & \\
%   };
% \end{tikzpicture}
% \]
% We define $\bfxx_{\uptau}:= \ytb{{x_{1}},\vdots,{x_{n}}}$ and $\bftt_{\uptau}:= \ytb{{x_{1}}{y_{1}},\vdots,{x_{n}}}$.

% We call $\bftt_{\uptau}$ the ``leg'' of $\uptau$ and $x_{\uptau}:=x_{n}$ the ``foot'' of $\uptau$.

% In case I/II, the sign of length-1 row in $\cL_{\uptau}$ is
% \[
% \bsign{\uptau} = \Sign{\bfxx_{\uptau}}
% \]
% We factorize $\cL_{\uptau} = \cB_{\uptau}\cdot \cC_{\uptau}$. Here
% $\lsign{\cT_{\uptau}}$ is a constant on each irreducible components.
% $\lsign{\cB_{\uptau}}=\bsign{\uptau}$.


% In case III, the sign of lenght-1 row with a plus length row  is determined by
% \[
% \bsign{\uptau} = \Sign{\bfxx_{\uptau}}
% \]

% $\cL_{\uptau} = \cpT_{\uptau}\cdot \cpB_{\uptau} + \cnT_{\uptau}\cdot \cnB_{\uptau}$.
% We allow $\cnB=\emptyset$, and $\cpB$ is assumed to have
% more length-1 row marked by $+$ than that of $\cnB$.
% We still have sign of $\cpT$ and $\cnT$ are the same. More importantly,
% $\lsign{\cpB} = \Sign{\bfxx_{\uptau}}$, and $\lsign{\cnB}=\Sign{\bfxx_{\uptau}}$
% if $\cnB\neq \emptyset$.
% We call $x_{n}$ the tail of $\uptau$.

% The key property is that the lifting of local system is controlled by the leg $\bftt_{\uptau}$.

% Moreover, we claim that all $\uptau$ in a LS-packet $[\cL]$ has the same
% tail $x_{n}$.

%We will establish this by induction.


% \subsection{Descent case}
% We assume now $\cOp\leadsto \cOpp$ is a usual descent.

% When $\upepsilon' = 0$, the dot-r-c diagrams have the following shape:
% \[
%   \begin{tikzcd}[row sep={1.5em,between origins},column sep={1.5em,between origins}]
%              & \cdots & \cdots & \cdots & \cdots &                          & \cdots & \cdots & \cdots & \cdots & \\
%     \uptau:  & *      & *      & *      & *      & \times                   &   *     & *      & *      & *      & \\
%              & x_{1}  &        &        &            &                          &        &        &        &        & \\
%              & \vdots &        &        &        &                          &        &        &        &        & \\
%              & x_{n}  &        &        &        & \  \ar[d,maps to,"\eDD"] &        &        &        &        & \\[1em]
%              &        &        &        &        & \                        &        &        &        &        & \\
%              &        & \cdots & \cdots & \cdots & \                        & \cdots & \cdots & \cdots & \cdots & \\
%     \uptau': &        & *      & *      & *      & \times                   & *      & *      & *      & *      & \\
%              &        &        &        &        & \ \ar[d,maps to,"\eDD"]   &        &        &        &        & \\[1em]
%              &        & \cdots & \cdots & \cdots & \                        &        & \cdots & \cdots & \cdots & \\
%     \uptaupp:
%              &        & *      & *      & *      & \times                   &        & *      & *      & *      & \\
%              &        &        &        &        &                          &        &        &        &        & \\
%   \end{tikzcd}
% \]


% When $\upepsilon'=1$, the dot-r-c diagrams have the following shape:
% \[
%   \begin{tikzcd}[row sep={1.5em,between origins},column sep={1.5em,between origins}]
%              & \cdots & \cdots & \cdots & \cdots &                          & \cdots & \cdots & \cdots & \cdots \\
%     \uptau:  & *      & *      & *      & *      & \times                   & *      & *      & *      & *      \\
%              & x_{0}  & y_{1}  &        &        &                          &        &        &        &        \\
%              & x_{1}   & y_{2}  &        &        &                          &        &        &        &        \\
%              & \vdots  &        &        &        &                          &        &        &        &        \\
%              & x_{n}  &        &        &        & \  \ar[d,maps to,"\eDD"] &        &        &        &        \\[1em]
%              &        &        &        &        & \                        &        &        &        &        \\
%              &        & \cdots & \cdots & \cdots & \                        & \cdots & \cdots & \cdots & \cdots \\
%     \uptau': &        & *      & *      & *      & \times                   & *      & *      & *      & *      \\
%              &        & x_{\uptaupp}  &        &        & \                        &        &        &        &        \\
%              &        & z_{2}  &        &        & \ \ar[d,maps to,"\eDD"]  &        &        &        &        \\[1em]
%     \uptaupp:
%              &        & \cdots & \cdots & \cdots & \                        &        & \cdots & \cdots & \cdots & \\
%              &        & *      & *      & *      & \times                   &        & *      & *      & *      \\
%              &        & w_{1}  &        &        &                          &        &        &        &        \\
%   \end{tikzcd}
% \]
% The actual definition of $\uptau\mapsto \uptaup$ is in \Cref{sec:alg.CD}. %\Cref{sec:dd.nonsep}.

% Consider the following map of local system
% \[
%   \begin{tikzcd}[row sep=0em]
%     \LLS(\cOpp) \times \bZ/2\bZ \ar[r]& \LS(\cO')\\
%     (\cLpp, \upepsilon') \ar[r,maps to] & \vartheta(\cLpp\otimes {\det}^{\upepsilon'}).
%   \end{tikzcd}
% \]
% Its image is always non-zero and it is an injection by
% \Cref{prop:CD.1}~\ref{p:drcls.1} for $\cOpp$.

% Therefore the theta lift $ \uppi_{\uptaup}:= \Thetab(\uppi_{\uptaupp}\otimes {\det}^{\upepsilon'})$ is always
% non-vanishing and $\uppi_{\uptaup}$ are distinct with each other.
% %Hence $\lUnip(\cO') = \Unip(\cO')$.
% Hence $\Unip(\cO') = \set{\uppi_{\uptaup}}$.

% We now prove that we have a factorization of local system:
% \[
%   \cL_{\uptau} = \cB_{\uptau}\cdot \cC_{\uptau},
% \]
% here $\cC_{\uptau}$ is determined by $\bfxx_{\uptau} = x_{1}\cdots x_{n}$.

% Since $C_{2k}=C_{2k-1}$ is even,  $\cOpp\leadsto \cOp$ is a usual lifting and
% $\det\otimes \cL_{\uptaupp}\notin \LLS(\cOpp)$,
% \[
% \cL_{\uptaup} = \vartheta(\cL_{\uptaupp}\otimes (\det)^{\upepsilon'}) \mapsto (\cL_{\uptaupp}, \upepsilon')
% \]
% is a well defined map and restricted on $\LLS(\cO, \rO(p,q))$ is a bijection.



% Using our algorithm, we compute
% $\Sign(\uptau)-\Sign(\uptaupp) = (2a+p_{1},2a+q_{1})$. Here $a$ is the number of
% entries in $\taulf$ above $x_{1}$ and
% $(p_{1},q_{1})=\Sign((x_{1}, \cdots, x_{n})^{T})$ (see \Cref{sec:nsp.twist} for
% the non-special representation case).

% Therefore, the components in the local system $\cL_{\uptau}$ is given by:
% Attach two columns on the left of the irreducible LS of $\cL_{\uptaupp}$,
% and $2n$ 1-rows with siginature $(p_{1},q_{1})$.

% The tail $x_{n}$ determine the twisting character $(\bfonepn)^{\upepsilon}$.
% When $x_{n} = r$ (resp. $x_{n}=s$), $q_{1}=0$ (resp. $p_{1}=0$).

%\subsection{The general case}
% We will prove two aspects: 1. the liftings $\uppi_{\uptau}\neq 0$; 2. they are
% non-isomorphic.

% Here $\cB$ stands for ``body'', $\cD$ stands for ``foot'.
% Later $\cC$ stands for ``leg'' (crus in Latin)

% We always assume $\uptau$ is a type D dot-r-c diagram.
% Inductively we establish the following properties.

% \begin{claim}
% \begin{enumT}
%   \item  $x_{\uptau}$ determine the factorization of local system:
%   \begin{equation}\label{eq:ls.factor}
%     \begin{array}{c|c|c|c}
%       \bftt_{\uptau} & \cL_{\uptau} & \cF_{\uptau} &\\
%       \hline
%       \ytb{s\cdots,\vdots,s} & \displaystyle\sum_{i=1}^{2}\cB_{\uptau,i}\cdot\cF_{\uptau,i} & \cF_{\uptau,1}=\ytb{=,=},
%                                                                                          \cF_{\uptau,2}= \ytb{\uum\cdots,=} &\\
%       \hline
%       \ytb{\star\cdots,\vdots,r} & \displaystyle\sum_{i=1}^{2}\cB_{\uptau,i}\cdot\cF_{\uptau,i} & \cF_{\uptau,1}=\ytb{+,+}, \cF_{\uptau,2} = \ytb{\uup\cdots,+}\\
%       \hline
%       %\ytb{sd\cdots,\vdots,s,c} & \displaystyle\sum_{i=1}^{2}\cB_{\uptau,i}\cF_{\uptau,i} &  \cF_{\uptau,1} = \ytb{\uum\cdots,+},
%       %                                                                                      \cF_{\uptau,2}=\ytb{\uup\cdots,=}
%       %                                         &  \#s \geq 0\\
%       %\hdashline
%       %\ytb{\star\cdots,\vdots, c} & \cB_{\uptau,1}\cF_{\uptau,1} &  \cF_{\uptau,1} = \ytb{\uum\cdots,+} & \\
%       %& & & \cB_{\uptau,2}\neq \emptyset \text{only if
%       %      } \bfxx_{\uptau} = (s\cdots sc)^{t}\\
%       \ytb{\star\cdots,\vdots, c}
%                      & \displaystyle\sum_{i=1}^{2} \cB_{\uptau,i}\cdot\cF_{\uptau,i}
%                                     &  \cF_{\uptau,1} = \ytb{\uum\cdots,+},
%                                       \cF_{\uptau,2}= \ytb{\uup\cdots,=}
%                                                    & \cB_{\uptau,1}\neq \emptyset.
%       \\
%       %& & & \cB_{\uptau,2}\neq \emptyset \text{only if
%       %      } \bfxx_{\uptau} = (s\cdots sc)^{t}\\
%      \hline
%       \ytb{\star\cdots,\vdots,d} & \displaystyle\sum_{i=1}^{2}\cB_{\uptau,i}\cdot\cF_{\uptau,i} &  \cF_{\uptau,1} = \ytb{\uum\cdots,+},
%                                                                                              \cF_{\uptau,2}=\ytb{\uup\cdots,-}
%                                                &\cB_{\uptau,1}\text{ and }\cB_{\uptau,2}\neq \emptyset\\
%     \end{array}
%   \end{equation}
%   When $x_{\uptau}=c$, $\cF_{\uptau,2}\neq \emptyset$ only if
%   $\bftt_{\uptau} = \ytb{sd\cdots, \vdots,s,c}$.
%   \item If $\cOp \leadsto \cOpp$ is a usual descent, we have a simpler
%   factorization:
%   $\cL_{\uptau} = \cB_{\uptau} \cF_{\uptau}$.
%   \begin{equation}\label{eq:ls.srcd}
%     \begin{array}{c|c} x_{\uptau} & \cF_{\uptau} \\
%       \hline
%       s &  \ytb{=,=}\\
%       \hline
%       r & \ytb{+,+}\\
%       \hline
%       c & \ytb{+,=}\\
%       \hline
%       d & \ytb{+,-}\\
%       \hline
%     \end{array}
%   \end{equation}
%   \item
%   From the above table, we have:
%   \begin{enumT}
%     \item $x_{\uptau} = s$, then $\pcL = \ncL = 0$.
%     \item $x_{\uptau} = r/c$, then $\pcL \neq 0$ and $\ncL=0$.
%     \item $x_{\uptau} = d$, then $\pcL$ and $\ncL$ are both non-zero.
%   \end{enumT}
%   \item
%     The set of lifted local system are and disjoint with their determinant twist:
%     \[
%       \set{\cL_{\uptau} | \uptau \in \drc(\cO)} \cap \set{\cL_{\uptau}\otimes \det| \uptau\in \drc(\cO)} = \emptyset.
%     \]
% %  \item
% % \ref{p:drcls.1} follows from the above properties.
% %   \[
% %   \begin{array}{c|c|c|c|c}
% %     x_{\uptau} & s & r &c &d\\
% %     \hline
% %     \pcL_{\uptau} & \emptyset &  \sum_{i=1}^{2} \cB_{\uptau_{i}} \cT_{\uptau_{i}} & \\
% %     \ncL_{\uptau} & \emptyset & \emptyset & \emptyset &
% %   \end{array}
% % \]
%   % \item We have
%   % $\set{\cL_{\uptau}|\pcL_{\uptau}\neq 0} \rightarrow \set{\pcL_{\uptau}}$ given
%   % by $\cL_{\uptau}\mapsto \pcL_{\uptau}$ is injective if $n\geq 2$ or
%   % $\cOp\leadsto \cOpp$ is a usual descent..

%     \item If $\cO$ is noticed,
%     \[
%       \begin{tikzcd}[row sep=0em]
%         \set{\cL_{\uptau}|\pcL_{\uptau}\neq 0} \ar[r] & \set{\pcL_{\uptau}}\times \set{T,F} &\\
%         & (\pcL_{\uptau},F)
%         \arrow[dd, start anchor=north west, end anchor=south west, no head, xshift=-1em, decorate, decoration={brace,mirror}]
%         & \ncL_{\uptau} = \emptyset
%         \\[-1em]
%         \cL_{\uptau} \ar[r,mapsto, end anchor={[xshift=-2em]west}] & \phantom{(\pcL_{\uptau},F)} & \\[-1em]
%         &(\pcL_{\uptau},T) &  \ncL_{\uptau} \neq \emptyset
%       \end{tikzcd}
%     \]
%     is injection.
%     \item If $\cO$ is +noticed, the composition of the above map with $\pr_{1}$
%     is injective.
%     \item If $C_{2k+1}-C_{2k}\geq 2$ and $\bfxx_{\uptau}$ ends with $d$, $\cL_{\uptau}$ has an irreducible
%     component $\succ \dagger_{(1,1)}$. In particular, if $\cO$ is +noticed and
%     $\bfxx_{\uptau}$ ends with $d$, we have
%     $\cL_{\uptau}\supset \dagger_{(1,1)}$.
% \end{enumT}

% \end{claim}




% \subsubsection{Usual descent case}
% When $\cOp \rightarrow \cOpp$ is a usual descent, the non-vanishing is clear and
% the above properties are easily follows.
% Moreover, the signature of $1$-rows in $\cL_{\uptau}$ is given by
% $\ssign(\bfxx_{\uptau})$. Note that, (1). the lift of $\LS(\cOpp)\longrightarrow\LS(\cOp)$ is
% an injection, we could read the twisting character $\upepsilon_{\uptaup}$ from
% $\cL_{\uptaup}$.

% On the other hand, the twisting character $\upepsilon_{\uptau}$ could be read
% from the restriction on 1-rows of $\cL_{\uptau}$.  In fact, in the most of the
% case $\upepsilon_{\uptau}=1$. $\upepsilon_{\uptau}=0$ if and only if $x_{\uptau}=d$
% where $\cL_{\uptau}$ must contain an irreducible component  whose 1-rows contain a
%  $-$ sign.

% \subsubsection{Proof of Generalized descent case}



% Non-vanishing of the lifting:

% The dot-r-c diagrams have the following shape:
% \[
%   \begin{tikzcd}[row sep={1.5em,between origins},column sep={1.5em,between origins}]
%              & \cdots & \cdots & \cdots & \cdots &                          & \cdots & \cdots & \cdots & \cdots & \\
%     \uptau:  & *      & *      & *      & *      & \times                   &   *     & *      & *      & *      & \\
%              & x_{0}  & y_{1}   & w_{1}   & \cdots &                          &        &        &        &        & \\
%              & \vdots &        &        &        &                          &        &        &        &        & \\
%              & x_{n}  &        &        &        & \  \ar[d,maps to,"\eDD"] &        &        &        &        & \\[1em]
%              &        &        &        &        & \                        &        &        &        &        & \\
%              &        & \cdots & \cdots & \cdots & \                        & \cdots & \cdots & \cdots & \cdots & \\
%     \uptau': &        & *      & *      & *      & \times                   & *      & *      & *      & *      & \\
%              &        & x_{\uptaupp}  & w_{1}   &  \cdots   & \ \ar[d,maps to,"\eDD"]   &        &        &        &        & \\[1em]
%              &        & \cdots & \cdots & \cdots & \                        &        & \cdots & \cdots & \cdots & \\
%     \uptaupp:
%              &        & *      & *      & *      & \times                   &        & *      & *      & *      & \\
%              &        & x_{\uptaupp}   & w_{1}  &        &                          &        &        &        &        & \\
%   \end{tikzcd}
% \]


\subsection{The general descent case}\label{sec:pf.gd.CD}
%We now starts with induction:
We  assume $k\geq 1$ and all the properties are satisfied by $\uptaupp\in \drc(\cOpp)$.
We retain the notation in \Cref{sec:gd2.CD}.

\subsubsection{Local systems}
We now describe the local systems $\cL_{\uptaup}$ and $\cL_{\uptau}$ more
precisely using our formula of the associated character and the induction
hypothesis. Note that these local systems are always non-zero which implies that
$\uppi_{\uptaup}$ and $\uppi_{\uptau}$ are unipotent representations attached to
$\cOp$ and $\cO$ respectively.

First note that in $\uptaup$ and $\uptaupp$, $x_{\uptaupp}\neq s$ by the definition of
dot-r-c diagram.
\footnote{
  Suppose $\uptaupp\in \drc(\cOpp)$ has the shape in \eqref{eq:gd2.drc} and
  $x_{\uptaupp}=s$. By the induction hypothesis,   $\pcL_{\uptaupp}=\ncL_{\uptaupp}=0$ and so
  the theta lift of $\uppi_{\uptaupp}$ vanishes.
}
% in generalized descent.
By the induction hypothesis, $\cL_{\uptaupp}^{+}\neq 0$.
% , and so
% \[
%   \cL_{\uptaup} = \vartheta(\cL_{\uptaupp}) = \dagger \cL_{\uptaupp}^{+}\cup
%   \dagger \cL_{\uptaupp}^{-} \neq 0.
% \]

% Now for each $\uptaup\in \drc(\cOp)$, $\uptaupp$


Let $(p''_{1},q''_{1}) := \lsign(\cL_{\uptaupp})$.
Then
\begin{equation}\label{eq:lsign.1}
\lsign(\pcL_{\uptaupp}) = (p_{0}-1,q_{0}), \text{ and }
\lsign(\ncL_{\uptaupp})=(p_{0},q_{0}-1) \text{ if } \ncL_{\uptaupp}\neq \emptyset.
\end{equation}
\begin{equation}\label{eq:LS.taup}
  \cL_{\uptaup} = \vartheta(\cL_{\uptaupp}) = \dagger \pcL_{\uptaupp} + \dagger \ncL_{\uptaupp} \neq 0
\end{equation}
where
$\lsign(\dagger\pcL_{\uptaupp}) = (q''_{1}, p''_{1}-1)$ and
$\lsign(\dagger\ncL_{\uptaupp})=(q''_{1}-1,p''_{1})$ (if
$\dagger\ncL_{\uptaupp}\neq 0$).

Let $(p_{1},q_{1}) := \lsign(\cL_{\uptau})$ and
$(e,f):=(p_{1}-p''_{1}+1, q_{1}-q''_{1}+1)$. Using \eqref{eq:bd.prop} and
\eqref{eq:def.u}, one can show that
\[
  (e,f)=\ssign(\bfuu_{\uptau}).
\]
\trivial[]{ It suffice to consider the most left three columns of the peduncle
  part: this part has signature
  $\lsign(\cP_{\uptau}) +(1,1) = \ssign(\bfuu_{\uptau}x_{\uptaupp})= \ssign(\bfuu_{\uptau})+\lsign(\cD_{\uptaupp})$.
  Therfore,
  \[\ssign(\bfuu_{\uptau}) = \lsign(\cP_{\uptau})-\lsign(\cD_{\uptaupp})
  + (1,1) = \lsign(\cL_{\uptau})-\lsign(\cL_{\uptaupp})+(1,1).
  \]
}

Now
\begin{equation}\label{eq:gd.ls}
  \cL_{\uptau} = \dagger\dagger \pcL_{\uptaupp} \cdot \pcE_{\uptau}   + \dagger\dagger \ncL_{\uptaupp}  \cdot \ncE_{\uptau}
\end{equation}
where
\[
  \begin{split}
    \pcE_{\uptau}& = \begin{cases} \ddagger_{e,f-1} & \text{when
      } f\geq 1 \text{ and } x_{\uptau}\neq d \\
      \dagger_{e,f-1} & \text{when
      } f\geq 1 \text{ and } x_{\uptau}= d \\
      0 & \text{otherwise}
    \end{cases}\\
    \ncE_{\uptau}& = \begin{cases} \ddagger_{e-1,f} & \text{when
      } e\geq 1 \text{ and } x_{\uptau}\neq d \\
      \dagger_{e-1,f} & \text{when
      } e\geq 1 \text{ and } x_{\uptau}= d \\
      0 & \text{otherwise}
    \end{cases}.\\
  \end{split}
\]
%
% $\pcE$ and $\ncE$ are 1-rows of sign $(p_{1}-p''_{1}+1,q_{1}-q''_{1})$ and
% $(p-p'',q-q''+1)$ respectively.

\medskip

By induction hypothesis, we have
a factorization of $\cL_{\uptaupp}$ as in \eqref{eq:ls.factor}
\[
  \cL_{\uptaupp} = \sum_{i=-(k-1)}^{k-1}\cB_{\uptaupp,i}\cdot \cD_{\uptaupp,i}.
\]
%where $\abs{i}\leq k-1$ in the above formula.
Then we have the following factorization of $\cL_{\uptau}$:
\[
\cL_{\uptau}=\sum_{i\neq 0} \cT_{\uptau,i}\cdot \cP_{\uptau,i}, \text{ with
} \cT_{\uptau,i}:=
\begin{cases}
\dagger\dagger \cB_{\uptaupp,i-1} & i\geq 1 \\
\dagger\dagger \cB_{\uptaupp,i+1} & i\leq -1.
\end{cases}
\]
Now we describle $\cP_{\uptau,i}$ according to $x_{\uptaupp}$ case by case. In the
following discussion,
\[
  \uup\cdots\uup\quad \text{ and }\quad  \uum\cdots\uum
\]
always denote a row has length
$2i+1$ with unspecified associated character.
%We have the following cases
%\subsubsection{}
%First note that $x_{\uptaupp}\neq s$ in generalized descent.

 \subsubsection{Case $x_{\uptaupp}=r$}\label{sec:z.r}
 We will see that the factorization \eqref{eq:d.factor} always holds
 when $x_{\uptaupp}=r$ and $n\geq 2$.
 % according to the about list.
  %\item $x_{\uptaupp}=r$. %In this case, all length $3$ rows ends with $+$ sign!
  % So
  % $\bsign{\bfxx_{\uptaupp}} \succ (2,0)$, $\pcL_{\uptaupp}\neq 0$ has tail with
  % sign $+$, and $\ncL_{\uptaupp}=0$.
  We have three sub-cases:
  \begin{enumT}
  \item $\bfpp_{\uptau} = \tytb{rc\cdots,\vdots,r}$. Then
  $\cL_{\uptau}=\sum_{i=1}^{k} (\dagger\dagger \cB_{\uptaupp,i-1})\cdot \cP_{\uptau,i}$, where
  \begin{equation}\label{eq:rr.c}
  \cP_{\uptau,i}  = \tytb{\uup\cdots\uup,+,\vdots,+} \quad \forall i\geq 1.
  \end{equation}

 When $n\geq 2$, we have $\cP_{\uptau,i}\succeq \tytb{+,+}$ and the
 factorization \eqref{eq:d.factor} holds.

 When $n=1$, we set
 \begin{equation} \label{eq:gd.rr}
   \cB_{\uptau,0} = 0,
   \cB_{\uptau,i} = \cT_{\uptau,i} \text{ and
   } \cD_{\uptau,i} = \cP_{\uptau,i} \text{ for } i\geq 1.
 \end{equation}
 Now $\cL_{\uptau} = \sum_{i=1}^{k} \cB_{\uptau,i}\cdot \cD_{\uptau,i}$.
%where $i$ is the index such that $\cB_{\uptaupp,i}\neq 0$.

 \item $\bfpp_{\uptau} = \tytb{rc\cdots,\vdots,r,d}$.
 In this case, $n\geq 2$.
  \begin{equation}\label{eq:rd.c}
  \cP_{\uptau,i}  = \tytb{\uup\cdots\uup,-,+,\vdots,+} \succ \tytb{-,+} \quad  \forall i\geq 1
  % \cP_{\uptau,1}  = \tytb{\uup\uum\uup,-,+,\vdots,+}\quad
  % \cP_{\uptau,2}  = \tytb{\uup\uum\uup\cdots,-,+,\vdots,+}
  \end{equation}
In particular, \eqref{eq:d.factor} holds.
% $\cP_{\uptau,i}\succeq \ddagger_{(2,1)}\succ \dagger_{(1,1)}$ and the
% factorization
% $\cL_{\uptau} = \cB_{\uptau,1}\cdot \cD_{\uptau,1}$ such that $\cD_{\uptau,1}=\tytb{+,-}$.
 \item $\bfpp_{\uptau} = \tytb{sr\cdots,\vdots,s,{x_{j}},\vdots,{x_{n}}}$.
% $y_{1}=x_{\uptaupp}=r$ and $\bfxx_{\uptau} = s\cdots sx_{j}\cdots x_{n}$ with
 In this case $\#s(\bfxx_{\uptau})\geq 1$.

 When $n\geq 2$, the factorization \eqref{eq:d.factor} holds.
 More precisely, we have four cases according to the mark of $x_{n}$.
 \begin{enumT}
   \item $x_{n}=s$, i.e. $\bfxx_{\uptau} = s\cdots s$. Then
   \[
  \cP_{\uptau,i}  = \tytb{\uup\cdots\uup,=,\vdots,=} \quad \forall i \geq 1
   \]
   When $n= 1$,
   $\cL_{\uptau} = \sum_{i=-1}^{-k} \cB_{\uptau,i}\cdot \cD_{\uptau,i}$
   with $\cB_{\uptau,i} = \cT_{\uptau,i}$ and $\cD_{\uptau,i} = \cP_{\uptau,i}$.
   % and the
   % factorization is also given by the formula \eqref{eq:gd.rr}.
   \item $x_{n}=r$. Then $n\geq 2$ and
   \[
  \cP_{\uptau,i}  = \tytb{\uup\cdots\uup,+,\vdots,+,=,\vdots,=} \succeq \tytb{+,+,=}\succ \tytb{+,+,\none}
  \quad \forall i\geq 1
  \]
   \item $x_{n}=c$. Then $n\geq 2$ and
   \[
  \cP_{\uptau,i}  = \tytb{\uup\cdots\uup,+,\vdots,+,=,\vdots,=} \succeq \tytb{+,=,=}\succ \tytb{+,=,\none}
  \quad \forall i\geq 1
  %\cP_{\uptau,2}  = \tytb{\uup\uum\uup\cdots,+,\vdots,+,=,\vdots,=} \succeq \tytb{+,=,=}\succ \tytb{+,=,\none}
   \]
   \item $x_{n}=d$. Then $n\geq 2$ and
   \[
  \cP_{\uptau,i}  = \tytb{\uup\cdots\uup,+,\vdots,+,-,\vdots,-}  \succeq \tytb{+,-,-}\succ \tytb{+,-,\none}
  \quad \forall i\geq 1
  %\cP_{\uptau,2}  = \tytb{\uup\uum\uup\cdots,+,\vdots,+,-,\vdots,-} \succeq \tytb{+,-,-}\succ \tytb{+,-,\none}
   \]
 \end{enumT}
 % Suppose $\Sign(\bfxx_{\uptau})=(p_{1},q_{1})$ with $p_{1}+q_{1}=2n$
  % \begin{equation}%\cL_{\uptau} = \dagger\dagger \pcL_{\uptaupp} \cdot \ytableaushort{+,\vdots,+,{-/=},\cdots,{-/=}}
  %   \cB_{\uptau}  =
  %   \tytb{+-+,+,\vdots,+,-,\vdots,-}  \text{ when } x_{n} = d \text{ or  } \tytb{+-+,+,\vdots,+,=,\vdots,=}
  %   \text{ when } x_{n} = r/c
  % \end{equation}
  % \[
  %   \cB_{\uptau} = \tytb{+-+,=,\vdots,=} \text{ when } x_{n}=s.
  % \]
  % Note that, we have at least one $-$ sign in 1-rows.
  % At least two $-$ signs if $x_{n}=c/d$.

  %  We have  $\cB_{\uptau}\succeq \ddagger_{(0,2)}$  when $x_{n}=s$;
  %  $\cB_{\uptau}\succeq \ddagger_{(2,1)}$  when $x_{n}=r$;
  % $\cB_{\uptau}\succeq \ddagger_{(2,1)}$ when $x_{n}=c$;
  % $\cB_{\uptau}\succeq \dagger_{(2,1)}$ when $x_{n}=d$.
  \end{enumT}


  \subsubsection{Case $x_{\uptaupp}=c$}\label{sec:z.c}
  % In this case, the length $3$-rows will contain at least one $-$ sign.
  %$\pcL_{\uptaupp}\neq 0$ has tail sign $=$, and $\ncL_{\uptaupp}=0$.
  %We assume $\Sign(\bfxx_{\uptau})=(p_{1},q_{1})$.
  We always have $\cP_{\uptau,2}=\emptyset$.
  \begin{enumT}
    \item  $x_{n} =  s$. In this case $x_{\uptau}=s\cdots s$.
    \begin{equation}\label{eq:ss.c}
      \cP_{\uptau,i} = \tytb{\uum\cdots\uum,=,\vdots,=} \quad \forall i\geq 1.
    \end{equation}
    When $n\geq 2$, the  factorization \eqref{eq:d.factor} holds.

    When $n=1$, we have
    \begin{equation}\label{eq:gd.ss}
      \cL_{\uptau} = \sum_{i=-1}^{-k}\cB_{\uptau,i}\cdot \cD_{\uptau,i}
      \text{ where } \cB_{\uptau,i}=\cT_{\uptau,-i}
      \text{ and }\cD_{\uptau,i}=\cP_{\uptau,-i}.
  \end{equation}
  \item $x_{n} = r$.
    In this case, $\bfxx_{\uptau}= s\cdots sr\cdots r$ where
    $\#s(\bfxx_{\uptau})>1$ and $n\geq 2$.
   \[
     \cP_{\uptau,i}  = \tytb{\uum\cdots\uum,=,\vdots,=,+,\vdots,+,} \succeq \tytb{=,+,+}\succ\tytb{\none,+,+}
     \quad \forall i\geq 1.
   \]
   In particular, \eqref{eq:d.factor} holds.
   \item $x_{n} = c$.
   % In this case, $\bfxx_{\uptau}= s\cdots s r\cdots r$ where number of
   \begin{enumT}
     \item
     When $\#s(\bfxx_{\uptau})>1$ and so $n\geq 2$.
     \[
       \cP_{\uptau,i}  = \tytb{\uum\cdots\uum,+,\vdots,+,=,\vdots,=} \succ \tytb{+,=}
     \quad \forall i\geq 1.
     \]
     % The
     % Otherwise, $\cP_{\uptau,1}\succ \ddagger_{1,1}$.
     The factorization \eqref{eq:d.factor} holds.
     \item
     When $\bfxx_{\uptau}=r\cdots rc$ with $n_{r}:=\#r(\bfxx_{\uptau}) \geq 0$.
     \[
       \cP_{\uptau,i}  = \tytb{\uum\cdots\uum,+,\vdots,+} %\quad \forall i\geq 1. \succ \tytb{+}.
       \quad \forall i\geq 1.
     \]
     We have $\cL_{\uptau} = \sum_{i=1}^{k}\cB_{\uptau,i}\cdot \cD_{\uptau,i}$  where $\cD_{\uptau,i} = \tytb{\uum\cdots\uum,+}$ and
     $\cB_{\uptau,i} = \cT_{\uptau,i}\cdot \ddagger_{2n_{r},0}$.
   \end{enumT}
   \item $x_{n} = d$. In this case, $n\geq 2$ and
   $\#s(\bfxx_{\uptau})+\#c(\bfxx_{\uptau})\geq 1$. Therefore,
    \begin{equation}\label{eq:ss.c}
      \cP_{\uptau,i} = \tytb{\uum\uup\uum\cdots,-,\vdots,-,+,\vdots,+}\succ \tytb{-,+}, \quad \forall i\geq 1
    \end{equation}
    and the factorization \eqref{eq:d.factor} holds.
  \end{enumT}

  \subsubsection{Case $x_{\uptaupp}=d$}\label{sec:z.d}
  In this case, $\bfxx_{\uptau}$ could be anything.
%  Then
  %$\pcL_{\uptaupp} = \cT_{\uptaupp} \cdot \pcB_{\uptaupp}$ and $\ncL_{\uptaupp}=\cT_{\uptaupp} \cdot \ncB_{\uptaupp}$.
 % We assume $\bsign{\pcL_{\uptaupp}}=(p-1,q)$ and $\bsign{\ncL_{\uptaupp}}=(p,q-1)$.
  % \[
  %   \cL_{\uptau} = \dagger \dagger \cT_{\uptaupp}\cdot \cB_{\uptau}.
  % \]
  \begin{enumT}
    \item $\bfxx_{\uptau}=s\cdots s$. We have
    \[
      \cP_{\uptau,i} = \tytb{\uum\cdots\uum,=,\vdots,=} \quad \forall i\geq 1.
    \]
    When $n\geq 2$, the factorization \eqref{eq:d.factor} holds.
% Clearly $\pcL_{\uptau}=\emptyset$.
    When $n=1$, the factorizaton is given by \eqref{eq:gd.ss}.
    \item $\bfxx_{\uptau}=r\cdots r$.
    \[
      \cP_{\uptau,i} = \tytb{\uup\cdots\uup,+,\vdots,+} \quad \forall i\leq -1.
    \]
    When $n\geq 2$, the factorization \eqref{eq:d.factor} holds.

    When $n=1$,
    \[
    \cL_{\uptau} = \sum_{i=1}^{k}\cB_{\uptau,i}\cdot \cD_{\uptau,i} \text{
      where } \cB_{\uptau,i} = \cT_{\uptau,-i} \text{ and } \cD_{\uptau,i} = \cP_{\uptau,-i}.
    \]
    \item $\bfxx_{\uptau} = s\cdots sr\cdots r$ with $n_{s}:=\#s(\bfxx_{\uptau})\geq 1$
    and $n_{r}:=\#r(\bfxx_{\uptau})\geq 1$
    \[
      \cP_{\uptau,i} = \tytb{\uum\cdots\uum,=,+,+,+,\vdots,+,=,\vdots,=}
      \quad \cP_{\uptau,-i} = \tytb{\uup\cdots\uup,+,=,=,+,\vdots,+,=,\vdots,=}
      \quad \forall i\geq 1
    \]
    %Clearly $\cP_{\uptau,i}\succ \ddagger_{(1,1)}$ and $n\geq 2$.
    We have
    \[
      \cL_{\uptau} = \sum_{i=0}^{k}\cB_{\uptau,i}\cdot \cD_{\uptau,i}.
    \]
    Here
      \begin{align*}
        \cB_{\uptau,0} &= \sum_{i=1}^{k} \cT_{\uptau,i}\cdot \tytb{\uum\cdots\uum,=} \cdot\ddagger_{2n_{r}-2,2n_{s}-2}, &
        \cD_{\uptau,0} &= \tytb{+,+},\\
        \cB_{\uptau,i} &= \cT_{\uptau,-i}\cdot \ddagger_{2n_{r}-2,2n_{s}},  &
        \cD_{\uptau,i} &= \tytb{\uup\cdots\uup,+}, \quad \forall i\geq 1\\
      \end{align*}
    \item $\bfxx_{\uptau} = s\cdots sc$ where $n_{s}:=\#s(\bfxx_{\uptau})\geq 0$.
    \begin{equation}\label{eq:ped.ssc}
      \cP_{\uptau,i} = \tytb{\uum\cdots\uum,+,=,\vdots,=} \quad \cP_{\uptau,-i} = \tytb{\uup\cdots\uup,=,=,\vdots,=} \quad \forall i\geq 1
    \end{equation}
    We have
    \[
      \cL_{\uptau} = \sum_{i=1}^{k}\cB_{\uptau,i}\cdot \cD_{\uptau,i}+ \sum_{i=1}^{k}\cB_{\uptau,-i}\cdot \cD_{\uptau,-i}.
    \]
    For $i\geq 1$,
      \begin{align*}
        \cB_{\uptau,i} &= \cT_{\uptau,i}\cdot \ddagger_{0,2n_{s}}, &
        \cD_{\uptau,i} &= \tytb{\uum\cdots\uum,+},\\
        \cB_{\uptau,-i} &= \cT_{\uptau,-i}\cdot \ddagger_{0,2n_{s}},  &
        \cD_{\uptau,-i} &= \tytb{\uup\cdots\uup,=}. %, \quad \forall i\geq 1\\
      \end{align*}
    \item $\bfxx_{\uptau} = r\cdots rc$ where $n_{r} = \#r(\bfxx_{\uptau})\geq 1$.
    \[
      \cP_{\uptau,i} = \tytb{\uum\cdots\uum,+,+,\vdots,+} \quad \cP_{\uptau,-i} = \tytb{\uup\cdots\uup,=,+,\vdots,+}
      \quad \forall i\geq 1
    \]
    We have
    \[
      \cL_{\uptau} = \sum_{i=0}^{k}\cB_{\uptau,i}\cdot \cD_{\uptau,i}.
    \]
    where
      \begin{align*}
        \cB_{\uptau,0} &= \sum_{i=1}^{k}\cT_{\uptau,-i}\cdot \tytb{\uup\cdots\uup,+} \cdot \ddagger_{2n_{r}-2,0}, &
        \cD_{\uptau,0} &= \tytb{=,+},\\
        \cB_{\uptau,i} &= \cT_{\uptau,i}\cdot \ddagger_{2n_{r},0},  &
        \cD_{\uptau,i} &= \tytb{\uum\cdots\uum,+}  \quad \forall i\geq 1.\\
      \end{align*}
    \item $\bfxx_{\uptau} = s\cdots sr\cdots rc$ where
    $n_{s} := \#s(\bfxx_{\uptau})\geq 1, n_{r}:=\#r(\bfxx_{\uptau})\geq 1$.
    \[
      \cP_{\uptau,i} = \tytb{\uum\cdots\uum,+,+,\vdots,+,=,\vdots,=}\succ \tytb{+,=},
      \quad \cP_{\uptau,-i} = \tytb{\uup\cdots\uup,=,+,\vdots,+,=,\vdots,=}\succ \tytb{+,=} \quad \forall i\geq 1
    \]
    %Note that $\cP_{\uptau,i}\succ \ddagger_{(1,1)}$.
    Now the factorization \eqref{eq:d.factor} holds.
    \item $\bfxx_{\uptau} = s\cdots sd$ where $\#s(\bfxx_{\uptau})\geq 1$.
    \begin{equation}\label{eq:ped.ssd}
      \cP_{\uptau,i} = \tytb{\uum\cdots\uum,+,-,\vdots,-} \quad \cP_{\uptau,-i} = \tytb{\uup\cdots\uup,-,-,\vdots,-}
      \quad \forall i\geq 1
    \end{equation}
    We have
    \[
      \cL_{\uptau} = \sum_{i=0}^{k}\cB_{\uptau,-i}\cdot \cD_{\uptau,-i}.
    \]
    where
      \begin{align*}
        \cB_{\uptau,0} &= \sum_{i=1}^{k}\cT_{\uptau,i}\cdot \tytb{\uum\cdots\uum,-} \cdot \dagger_{0,2n_{s}-2}, &
        \cD_{\uptau,0} &= \tytb{+,-},\\
        \cB_{\uptau,-i} &= \cT_{\uptau,-i}\cdot \ddagger_{0,2n_{s}},  &
        \cD_{\uptau,-i} &= \tytb{\uup\cdots\uup,-}  \quad \forall i\geq  1.\\
      \end{align*}
    \item $\bfxx_{\uptau} = r\cdots rd$ where $n_{r}=\#r(\bfxx_{\uptau})\geq 1$.
    \[
      \cP_{\uptau,i} = \tytb{\uum\cdots\uum,+,+,\vdots,+} \quad \cP_{\uptau,-i} = \tytb{\uup\cdots\uup,-,+,\vdots,+}
      \quad \forall i\geq 1
    \]
    We have
    \[
      \cL_{\uptau} = \sum_{i=0}^{k}\cB_{\uptau,i}\cdot \cD_{\uptau,i}.
    \]
    where
      \begin{align*}
        \cB_{\uptau,0} &= \sum_{i=1}^{k}\cT_{\uptau,-i}\cdot \tytb{\uup\cdots\uup,+} \cdot \dagger_{2n_{r}-2,0}, &
        \cD_{\uptau,0} &= \tytb{-,+},\\
        \cB_{\uptau,i} &= \cT_{\uptau,i}\cdot \dagger_{2n_{r},0},  &
        \cD_{\uptau,i} &= \tytb{\uum\cdots\uum,+}  \quad \forall i\geq  1.\\
      \end{align*}
    \item $\bfxx_{\uptau} = s\cdots sr\cdots rd$ where
    $\#s(\bfxx_{\uptau})\geq 1, \#r(\bfxx_{\uptau})\geq 1$.
    \[
      \cP_{\uptau,i} = \tytb{\uum\cdots\uum,+,+,\vdots,+,-,\vdots,-} \succ\tytb{+,-},
      \quad \cP_{\uptau,-i} = \tytb{\uup\cdots\uup,-,+,\vdots,+,-,\vdots,-} \succ\tytb{+,-}
      \quad \forall i\geq 1.
    \]
    Now the factorization \eqref{eq:d.factor} holds.
    \item $\bfxx_{\uptau} = s\cdots sr\cdots rcd$ where
    $\#s(\bfxx_{\uptau})\geq 0, \#r(\bfxx_{\uptau})\geq 0$.
    \[
      \cP_{\uptau,i} = \tytb{\uum\cdots\uum,+,+,\vdots,+,-,\vdots,-}\succ\tytb{+,-},
      \quad \cP_{\uptau,-i} = \tytb{\uup\cdots\uup,-,+,\vdots,+,-,\vdots,-}\succ\tytb{+,-}
      \quad \forall i\geq 1.
    \]
    Now the factorization \eqref{eq:d.factor} holds.
    \item $\bfxx_{\uptau}=d$. In this case, $\bfpp_{\uptau} = dd\cdots$
    \[
      \cP_{\uptau,i} = \tytb{\uum\cdots\uum,+} \quad \cP_{\uptau,-i} = \tytb{\uup\cdots\uup,-}
      \quad \forall i \geq 1
    \]
    We have
    \[
      \cL_{\uptau} = \sum_{i=0}^{k}\cB_{\uptau,i}\cdot \cD_{\uptau,i}+\sum_{i=0}^{k}\cB_{\uptau,-i}\cdot \cD_{\uptau,-i}.
    \]
    where
    \[
        \cB_{\uptau,i} = \cT_{\uptau,i}, \quad
        \cD_{\uptau,i} = \cP_{\utau,i}  \quad \forall i = -k, \cdots, -1,1,\cdots, k.\\
    \]
  \end{enumT}

\subsubsection{Summary of the factorization}
We have the non-vanishing of the lifted local system $\cL_{\uptau}$ and the
factorization described in
\Cref{eq:ls.factor} holds.

\subsubsection{Unipotent representations attached to $\cOp$}
In this section, we let $\uptaup\in \drc(\cOp)$ and
$\uptaupp  = \eDDo(\uptaup)\in \drc(\cOpp)$. First note that
$x_{\uptaupp}\neq s$ and so $\pcL_{\uptaupp}$ always non-zero.

Recall \eqref{eq:LS.taup}.
We claim that we can recover $\pcL_{\uptaupp}$ and
$\ncL_{\uptaupp}$ from $\cL_{\uptaup}$:
\begin{claim}\label{c:gd.C1}
  The map $\Omega_{\cOp}\colon \cL_{\uptaup}\mapsto (\pcL_{\uptaupp},\ncL_{\uptaupp})$ is a well
  defined map.
\end{claim}
\begin{proof}
  When $\lsign(\cL_{\uptaup})$ has two elements, $x_{\uptaupp}=d$ and
  $\lsign(\cL_{\uptaup}) = \set{(q''_{1},p''_{1}-1),(q''_{1}-1,p''_{1})}$. In
  \eqref{eq:LS.taup}, $\dagger\pcL_{\uptaupp}$ (resp. $\dagger\ncL_{\uptaupp}$)
  consists of components whose first column has siginature $(q''_{1},p''_{1}-1)$
  (resp. $(q''_{1}-1,p''_{1})$). When $\lsign(\cL_{\uptaup})$ has only one
  elements, $x_{\uptau}=r/c$,
  $\lsign(\cL_{\uptaup}) = \set{(q''_{1},p''_{1}-1)}$,
  $\cL_{\uptaup} = \dagger\pcL_{\uptaupp}$ and $\ncL_{\uptaupp}=0$
  % \trivial[h]{
  %   Note that, $\lsign( \dagger\pcL_{\uptaupp} ) = (q_{0},p_{0}-1)$ and
  %   $\lsign( \dagger\ncL_{\uptaupp} ) = (q_{0}-1,p_{0})$ (if it is none-zero) by
  %   \eqref{eq:lsign.1}. }

  In any case, we get
  % recover $\lsign(\cL_{\uptaupp})=(p_{0},q_{0})$ and
  \begin{equation}\label{eq:uptaupp.sign}
    \ssign(\uptaupp) = (n_{0},n_{0})+(p_{0},q_{0})\quad \text{where }
    2n_{0} = \abs{\DD(\cOpp)}.
  \end{equation}
% $+\half \abs{\DD(\cOpp)} \cdot(1,1)$.
  Using the signature $\ssign(\uptaupp)$, we could recover the twisting
  characters in the theta lifting of local system. Therefore
  $\pcL_{\uptaupp}$ and $\ncL_{\uptaupp}$ can be recovered from  $\cL_{\uptaup}$.
\end{proof}


\begin{claim}\label{c:gd.C3}
  Suppose $\uptaup_{1}\neq \uptaup_{2}\in \drc(\cOp)$. Then
  $\uppi_{\uptaup_{1}}\neq \uppi_{\uptaup_{2}}$.
\end{claim}
\begin{proof}
  It suffice to consider the case when $\cL_{\uptaup_{1}} = \cL_{\uptaup_{2}}$.
 By \eqref{eq:uptaupp.sign} in the proof of \Cref{c:gd.C1},
 $\ssign(\uptaupp_{1}) = \ssign(\uptaupp_{2})$.
 On the other hand, $\upepsilon_{1}=\upepsilon_{2}=0$ and
 $\uptaupp_{1}\neq \uptaupp_{2}$ by \Cref{lem:gd.CD}. So
 \[
   \uppi_{\uptaup_{1}} = \Thetab(\uppi_{\uptaupp_{1}})\neq\Thetab(\uppi_{\uptaupp_{2}}) = \uppi_{\uptaup_{2}}
 \]
  by the injectivity of theta lift.
\end{proof}


%\medskip

% When $\bfpp_{\uptau} = \tytb{sd\cdots,\vdots,s,{c/d}}$,
% $\pcL_{\uptau} = (\dagger\dagger \cB_{\uptaupp,1} \cP_{\uptau,1})^{+}$
% and this is the only case that $\pcL_{\uptau}\neq \emptyset$ has less irreducible
% components than that of $\cL_{\uptau}$.

% {
%   \color{red}
%   The above computation also tells the shape of legs of $\cL_{\uptau}$ according
%   to $\bfpp_{\uptau}$.
% }

% , $\cOpp$ is noticed when
% $C_{2k}$ is even or $\cOpp$ is +notice when $\cO_{2k}$ is odd.

\subsubsection{Unipotent representations attached to $\cO$}

\begin{claim}\label{c:gd.D1}
  Suppose $\uptau_{1}\neq \uptau_{2}\in \drc(\cO)$. Then $\uppi_{\uptau_{1}}\neq \uppi_{\uptau_{2}}$.
\end{claim}
\begin{proof}
 It suffice to consider the case that $\cL_{\uptau_{1}}=\cL_{\uptau_{2}}$.
 Clearly, $\ssign(\uptau_{1})=\ssign(\uptau_{2})$. On the other hand,
 the twisting $\upepsilon_{\uptau_{1}}=\upepsilon_{\uptau_{2}}$ since it is
 determined by the  local system:
 $\upepsilon_{\uptau_{i}}=0$ if and only if $\cL_{\uptau_{i}}\supset \tytb{-}$.
 %(see \eqref{eq:gd.ls}).

 We conclude that $\uptaupp_{1}\neq \uptaupp_{2}$ and $\uptaup_{1}\neq \uptaup_{2}$
 by \Cref{lem:gd.inj} and \Cref{lem:gd.CD}.
 By \Cref{c:gd.C3} and the injectivity of theta lift,
 \[
   \uppi_{\uptau_{1}} = \Thetab(\pi_{\uptaup_{1}})\otimes (\bfone^{+,-})^{\upepsilon_{\uptau_{1}}}
  \neq \Thetab(\pi_{\uptaup_{2}})\otimes (\bfone^{+,-})^{\upepsilon_{\uptau_{2}}} = \uppi_{\uptau_{2}}
 \]
\end{proof}


\subsubsection{Noticed orbits}

% \begin{claim}
% Suppose $\uptau_{1}\neq \uptau_{2}$ and
% $\cL_{\uptau_{1}}=\cL_{\uptau_{2}}$. Then, $\uptaup_{1}\neq \uptaup_{2}$ and
% $\upepsilon_{1}=\upepsilon_{2}$.
% \end{claim}
% \begin{proof}
% First note taht $\upepsilon_{i}$ is completely determined by the shape of local
% system $\cL_{\uptau_{i}}$.
% Observe that
% $\abs{\ssign(\uptau)} - \abs{\ssign(\uptaupp)} = \abs{\ssign(\bfxx_{\uptau}y_{1})} - \abs{\ssign(z_{1})}$.
% This implies that the signature of $\uptau$ is completely determined by
% $\bfxx_{\uptau}y_{1}$ and $\uptaupp$.
% Also observe that $\uptaupp$ is determined by $\uptaup$.

% By our algorithm we see that, for every fixing $\uptaup, \upepsilon$,
% the map
% \[
% \set{\uptau | \eDD(\uptau)= (\uptaup,\upepsilon)} \rightarrow \bZ
% \]
% defined by $\uptau\mapsto \abs{\ssign(\bfxx_{\uptau}y_{1})} - \abs{\ssign(z_{1})}$ is
% injective.

% Now $\cL_{\uptau_{1}}=\cL_{\uptau_{2}}$ implies
% $\ssign(\uptau_{1})=\ssign(\uptau_{2})$.

% Suppose $\uptaup:=\uptaup_{1}=\uptaup_{2}$ and $\upepsilon=\upepsilon_{1}=\upepsilon_{2}$, $\uptau_{1}$ and $\uptau_{2}$ are two
% elements in the above set with the same image in $\bZ$. This yields an
% contradiction.
% \end{proof}


% Now fix signature $(p,q)$. We will show that
% \[
%   \set{\cL_{\uptau}| \lsign(\cL_{\uptau}) = (p,q)}\longrightarrow \set{\cL_{\uptaup}}
% \]
% is injective when $\cO$ is noticed.


In this section, we prove the claims about noticed and +noticed orbits.

% \begin{claim}\label{c:gd.C2}
%   Suppose $\cOpp$ is +noticed. Then
%   \begin{enumT}
%     \item \label{c:gd.C2.1}The map
% $\set{\cL_{\uptaupp}| x_{\uptaupp}\neq s}\rightarrow \set{\cL_{\uptaup}}$ given by
% $\cL_{\uptaupp}\mapsto \cL_{\uptaup}=\vartheta(\cL_{\uptaupp})$ is a bijection.
% \item
% The map $\drc(\cOp)\rightarrow \LLS(\cOp)$ given by $\uptaup\mapsto \cL_{\uptaup}$ is a bijection.
% \end{enumT}
% \end{claim}
% \begin{proof} %Recall \eqref{eq:LS.taup}.
%   Let $\uptaup\in \drc(\cOp)$.
%   By \Cref{c:gd.C1}, we have a well defined map
%   $\cL_{\uptaup}\mapsto \pcL_{\uptaupp}$ with $\pcL_{\uptaupp}\neq 0$.
%   Since $\cOpp$ is +noticed, $\pcL_{\uptaupp}$ uniquely determines the local
%   system $\cL_{\uptaupp}$ by inductive hypothesis. Therefore,
%   $\cL_{\uptaup}\mapsto \pcL_{\uptaupp}\mapsto \cL_{\uptaupp}$ gives the inverse
%   map in the claim.

%   Now suppose $\uptaup_{1}\neq \uptaup_{2}\in \drc(\cOp)$. By \Cref{lem:gd.CD},
%   $\uptaupp_{1}\neq \uptaupp_{2}$. Hence
%   $\cL_{\uptaupp_{1}}\neq\cL_{\uptaupp_{2}}$ by the induction hypothesis.
%   Therefore, $\cL_{\uptaup_{1}}\neq \cL_{\uptaup_{2}}$ by part \eqref{c:gd.C2.1}
%   of the claim.
% % Case $x_{\uptaupp}=d$:
% %   When $\lsign(\cL_{\uptaupp}) = \set{(q_{0},p_{0}-1),(q_{0}-1,p_{0})}$ has two
% %   elements,
% %   we let $\pcL_{\uptaup}$ (resp. $\ncL_{\uptaup}$) be the parts whose
% %   $\lsign = (q_{0},p_{0}-1)$ (resp. $=(q_{0}-1,p_{0})$).
% %   If $\bsign(\cL_{\uptaup})= \set{(0,0)}$,
% %   $\cL_{\uptaupp} = \DD(\pcL_{\uptaup})\cdot + \cup \DD(\ncL_{\uptaup})\cdot - $.
% %   Otherwise, if every component of $\cL_{\uptaupp}$ $\succ \dagger_{(1,0)}$,
% %   we have $\cL_{\uptaupp} = \DD(\pcL_{\uptaup})\cdot +$; if
% %   $\cL_{\uptaupp} \succ \dagger_{(0,1)}$,
% %   $\cL_{\uptaupp} = \DD(\ncL_{\uptaup})\cdot -$.

% % Case $x_{\uptapp}=r/c$:
% % Since $\cOpp$ is +noticed, we have
% % $\cL_{\uptaup} = \dagger\pcL_{\uptaupp}\mapsto \pcL_{\uptaupp}\mapsto \cL_{\uptaupp}$
% % is well defined, and bijective with $\set{\cL_{\uptaupp}|\pcL_{\uptaupp}\neq \emptyset}$.
% % $\cL_{\uptaupp}\mapsto \pcL_{\uptaupp}$ is injective. Hence
% % $\cL_{\uptaup}\mapsto \cL_{\uptaupp}$ is injective.
% \end{proof}

% \begin{prop}\label{c:gd.pnoticed}
%     Suppose $\cO$ is +noticed.
%     The map $\Upsilon \colon \set{\uptau\in\drc{\cO}|x_{\uptau}}
%     \longrightarrow \set{\cL_{\uptau}|\pcL_{\uptau}\neq 0}\longrightarrow \set{\pcL_{\uptau}\neq 0}$
%     is injective.
%   %   \item
%   %   Suppose $\cO$ is noticed the map $\Upsilon$ defined in \eqref{eq:up} is injective.
%   % \end{enumC}
% \end{prop}

\begin{claim}\label{c:gd.C2}
  Suppose $\cOp$ is noticed. Then
  \begin{enumT}
    \item \label{c:gd.C2.1}The map
$\set{\cL_{\uptaupp}| x_{\uptaupp}\neq s}\longrightarrow \set{\cL_{\uptaup}}=\LLS(\cOp)$ given by
$\cL_{\uptaupp}\mapsto \cL_{\uptaup}=\vartheta(\cL_{\uptaupp})$ is a bijection.
\item
The map $\drc(\cOp)\rightarrow \LLS(\cOp)$ given by $\uptaup\mapsto \cL_{\uptaup}$ is a bijection.
\end{enumT}
\end{claim}
\begin{proof} %Recall \eqref{eq:LS.taup}.
  \begin{enumPF}
    \item
  Note that $\cOpp$ is noticed by definition. Hence $\Upsilon_{\cOpp}$ is invertible.
  % Let $\uptaup\in \drc(\cOp)$.
  % By  and the induction hypothesis , we have a well defined map
  % $\cL_{\uptaup}\mapsto \pcL_{\uptaupp}$ with $\pcL_{\uptaupp}\neq 0$.
  % Since $\cOpp$ is +noticed, $\pcL_{\uptaupp}$ uniquely determines the local
  % system $\cL_{\uptaupp}$ by inductive hypothesis. Therefore,
  %
  Recall \Cref{c:gd.C1}, we see that
  \[
    (\Upsilon_{\cOpp})^{-1}\circ \Omega_{\cOp}\colon
    \cL_{\uptaup}\mapsto (\pcL_{\uptaupp},\ncL_{\uptaupp})\mapsto \cL_{\uptaupp}
  \]
  gives the inverse of the map in the claim.
  \item
  Suppose $\uptaup_{1}\neq \uptaup_{2}\in \drc(\cOp)$. By \Cref{lem:gd.CD},
  $\uptaupp_{1}\neq \uptaupp_{2}$. Hence
  $\cL_{\uptaupp_{1}}\neq\cL_{\uptaupp_{2}}$ by the induction hypothesis.
  Therefore, $\cL_{\uptaup_{1}}\neq \cL_{\uptaup_{2}}$ by \ref{c:gd.C2.1}
  of the claim.
\end{enumPF}
\end{proof}

\begin{claim}\label{c:noticed.bij}
 Suppose $\cO$ is noticed, $\cL \colon \drc(\cO)\mapsto \LLS(\cO)$ given by
 $\uptau\mapsto \cL_{\uptau}$ is bijective.
% Moreover, $\cL_{\uptau_{1}}=\cL_{\uptau_{2}}$ implies $\cL_{\uptaup_{1}}=\cL_{\uptaup_{2}}$.
\end{claim}
\begin{proof}
We prove the claim by contradiction. We assume $\uptau_{1}\neq \uptau_{2}$ such
that $\cL_{\uptau_{1}}=\cL_{\uptau_{2}}$.


Recall \eqref{eq:gd.ls}: there is an pair of non-negative integers
$(e_{i},f_{i})=\ssign(\bfuu_{\uptau})$ such that
\[
  \cL_{\uptau_{i}} = \dagger\dagger \pcL_{\uptaupp_{i}}\cdot \pcE_{\uptau_{i}}
  + \dagger\dagger \ncL_{\uptaupp} \cdot \ncE_{\uptau_{i}}.
\]

\begin{enumPF}
  \item
Suppose that $\cL_{\uptau_{i}}$ contains a 1-row marked by $-$ or $=$. Then we
have
%\begin{enumI}
%  \item
$\upepsilon_{\uptau_{1}}=\upepsilon_{\uptau_{2}}$, which can be read from the mark $-/=$.
Moreover, the term
$\dagger\dagger\pcL_{\uptaupp_{i}}\cdot \pcE_{\uptau_{i}}\neq 0$ and we can
recover it from $\cL_{\uptau_{i}}$. So $\pcL_{\uptaupp_{1}} = \pcL_{\uptaupp_{2}}$.
Since $\cOpp$ is +noticed, $\uptaupp_{1}=\uptaupp_{2}$ by
\eqref{c:gd.pnoticed} and the induction hypothesis.
Note that $\ssign(\uptau_{1})=\ssign(\uptau_{2})$, we get
$\uptau_{1}=\uptau_{2}$ by \Cref{lem:gd.inj} which is contradict to our assumption.

% as the parts.. must contains the term
 % where $\pcE_{i}$ and $\ncE_{i}$ are columns of sign $(p_{i},q_{i}-1)$ and
 % $(p_{i}-1,q_{i})$ respectively.
 % Here we adapt the convention that $\pcE_{i} = \emptyset$ (resp. $\ncE_{i}$) if
 % $q_{i}-1<0$ (resp. $p_{i}-1<0$).

% Therefore, we have $\cL_{\uptaup_{i}}= \DD(\cL_{\uptau_{i}}) $ if $\cL_{\uptau}$ contains a 1-row
% of $+$ sign and a 1-row of $-$ sign.
\item
Now we assume that $\cL_{\uptau_{i}}$ does not contains a 1-row marked by $-/=$.
By checking the list in \Cref{sec:z.r,sec:z.c,sec:z.d}, we conclude that
$\bfpp_{\uptau_{i}}$ must have one of the following form:
% The exceptioinal cases are:
% \begin{enumT}
%   \item $(p_{i},q_{i}) = (2a,0)$ and so $\pcE_{i}=\emptyset$.
%   In this case, $\set{\bfpp_{\uptau_{i}}}$ must consists of the following three elements
  \[
    \bfpp_{1} =\tytb{rc,\vdots,r,r}, \quad \bfpp_{2}=\tytb{rc,\vdots,r,c}, \quad\text{and}
    \quad\bfpp_{3}= \tytb{rd,\vdots,r,r}\\
  \]
  We consider case by case according to the set $\set{\bfpp_{\uptau_{1}},\bfpp_{\uptau_{2}}}$:
  \begin{enumPF}
    \item Suppose
    $\set{\bfpp_{\uptau_{1}},\bfpp_{\uptau_{2}}}=\set{\bfpp_{1}}$ or $ \set{\bfpp_{2}}$.
    In these cases, $\upepsilon_{\uptau_{1}}=\upepsilon_{\uptau_{2}}$ and
    $\uptaupp_{1}\neq \uptaupp_{2}$ by \Cref{lem:gd.inj}.
    On the other hand, $\pcL_{\uptaupp_{1}}=\pcL_{\uptaupp_{2}}$ by \eqref{eq:gd.ls}.
    Since $\cOpp$ is +noticed, we get $\uptaupp_{1}=\uptaupp_{2}$ a contradiction.
    \item Suppose
    $\set{\bfpp_{\uptau_{1}},\bfpp_{\uptau_{2}}}=\set{\bfpp_{3}}$.
    We still have $\upepsilon_{\uptau_{1}}=\upepsilon_{\uptau_{2}}$ and
    $\uptaupp_{1}\neq \uptaupp_{2}$ by \Cref{lem:gd.inj}.
    On the other hand, $\ncL_{\uptaupp_{1}}=\ncL_{\uptaupp_{2}}$ by \eqref{eq:gd.ls}.
    This again contradict to the injectivity of $\cL_{\uptaupp}\mapsto \ncL_{\uptaupp}$.
    %Since $\cOpp$ is +noticed, we get $\uptaupp_{1}=\uptaupp_{2}$ a contradiction.
    \item  Suppose
    $\set{\bfpp_{\uptau_{1}},\bfpp_{\uptau_{2}}} = \set{\bfpp_{1},\bfpp_{2}}$.
    The the argument above, we have $\pcL_{\uptaupp_{1}}=\pcL_{\uptaupp_{2}}$
    Now $\uptaupp_{1}\neq \uptaupp_{2}$ since $\set{x_{\uptaupp_{1}},x_{\uptaupp_{2}}}=\set{r,c}$.
    This contradict to that $\cOpp$ is +noticed again.
    \item\label{it:c:noticed.bij.4} Suppose $\bfpp_{\uptau_{1}} = \bfpp_{1} \text{ or } \bfpp_{2}, \bfpp_{\uptau_{2}}=\bfpp_{3}$.
    Then
    $\cL_{\uptau_{1}} = \dagger\dagger\pcL_{\uptaupp_{1}}\cdot \pcE_{\uptaupp_{1}}$
    and
    $\cL_{\uptau_{2}} = \dagger\dagger\ncL_{\uptaupp_{2}}\cdot \ncE_{\uptaupp_{2}}$.
    This implies
    $\dagger \pcL_{\uptaupp_{1}}= \dagger\ncL_{\uptaupp_{2}}$ and
    $\abs{\ssign(\uptaupp_{1})} \equiv \abs{\ssign(\uptaupp_{2})}+2 \pmod{4}$.

    By the list in \Cref{sec:z.r,sec:z.c,sec:z.d}, we see that
    $\cL_{\uptaupp_{2}}\supset \tytb{-,+}$ and
    $\ncL_{\uptaupp_{2}}\supset \tytb{+}$.

    This leads to a contradiction: Thanks to the character twist in the theta
    lifting formula of the local system, we see that the the associated
    character restricted on the 2-row $\tytb{-+}$ of $\pcL_{\uptau_{1}}$ and
    $\ncL_{\uptau_{2}}$ must be are different.

    % Note that  $\cL_{\uptau_{2}}\succ \dagger_{(1,1)}$ implies that
    % $\ncL_{\uptau_{2}}\succ \dagger_{(1,0)}$. So $\dagger\ncL_{\uptau_{2}}$ has
    % a 2-row of sign $\tytb{-+}$.
    % On the other hand, $\cL_{\uptau_{1}}=\cL_{\uptau_{2}}$ implies
    % $\dagger \pcL_{\uptau_{1}}=\pcL_{\uptaup_{1}} = \ncL_{\uptaup_{2}}= \dagger\ncL_{\uptau_{2}}$. So
    % $\abs{\ssign(\uptaupp_{1})} \equiv \abs{\ssign(\uptaupp_{2})}+2 \pmod{4}$.
    % , i.e.
    % \[\cL_{\uptau_{1}}=\dagger \pcL_{\uptaupp_{1}} \cdot \dagger_{(2a-1,0)}
    %   \neq \ddagger\ncL_{\uptaup_{2}}\cdot \ddagger_{(2a-1,0)} = \cL_{\uptau_{2}}.
    % \]
    % \item $\bfpp_{\uptau_{1}} = \bfpp_{2},\bfpp_{\uptau_{2}}=\bfpp_{3}$.
    % On argue by the same way as the case \eqref{} and yield a contRadiction.
  \end{enumPF}
%   \item $(p_{i},q_{i}) = (0,2a)$ and so $\ncE_{i}=\emptyset$.
%   In this case, $\set{\bfpp_{\uptau_{i}}}$ must consists of two the following three elements
%   \[
%     \bfpp_{1} =\tytb{sc,\vdots,s}, \quad \bfpp_{2}=\tytb{sr,\vdots,s}, \quad\text{and}
%     \quad\bfpp_{3}= \tytb{sd,\vdots,s}\\
%   \]
%   This case is also easy.
%   $ \uptau_{1}\neq \uptau_{2} $ implies $\uptaupp_{1}\neq \uptaupp_{2}$.
%   But $\cL_{\uptau_{1}}=\cL_{\uptau_{2}}$ implies
%   $\cL_{\uptaupp_{1}} = \DD^{2}(\cL_{\uptau_{1}})\cdot + \neq \DD^{2}(\cL_{\uptau_{2}})\cdot + = \cL_{\uptaupp_{2}}$.
%   This contradict to the bijection between local systems with dot-r-c diagrams
%   for the +noticed orbit $\cOpp$.
%   %$\uptaupp_{1} = \DD^{2}(\uptau_{1})\neq \DD^{2}(\uptau_{2})  = \uptaupp_{2}$
% \end{enumT}
% \end{enumI}
\end{enumPF}
  We finished the proof of the claim.
\end{proof}

\subsubsection{The maps $\pUpsilon_{\cO}$, $\nUpsilon_{\cO}$ and $\Upsilon_{\cO}$}

\begin{claim}\label{c:gd.pnoticed}\label{c:gd.pnoticed.p}
    Suppose $\cO$ is +noticed.
    The map
    $\pUpsilon_{\cO} \colon \set{\cL_{\uptau}|\pcL_{\uptau}\neq 0}\rightarrow \set{\pcL_{\uptau}\neq 0}$
    is injective.
\end{claim}
\begin{proof}
    %First assume $\cO$ is +noticed.
    We have two cases:
    \begin{enumPF}
      \item $\cL_{\uptau}$ contain an irreducible component $\succ \tytb{+,+}$.
      This is equivalent to $\pcL_{\uptau}$ has an irreducible component
      $\succ\tytb{+}$. Now all irreducible components of
      $\cL_{\uptau} \succ \tytb{+}$ and $\cL_{\uptau}\mapsto \pcL_{\uptau}$ will
      not kill any irreducible components. Hence we could recover $\cL_{\uptau}$
      from $\pcL_{\uptau}$.%, and the map $\Upsilon_{1}$.
      \item Suppose that $\cL_{\uptau_{1}}\neq \cL_{\uptau_{2}}$ do not contain
      a component $\succ \tytb{+,+}$.

      By \Cref{c:noticed.bij}
      $\uptau_{1}\neq \uptau_{2}$.\footnote{$\cL_{\uptau_{1}}\neq \cL_{\uptau_{2}}$
        clearly implies $\uptau_{1}\neq \uptau_{2}$.} We now show that
      $\pcL_{\uptau_{1}}=\pcL_{\uptau_{2}}\neq 0$ leads to a contradiction.
      Clearly, $\ssign(\uptau_{1})=\ssign(\uptau_{2})$ By checking the list in
      \Cref{sec:z.r,sec:z.c,sec:z.d}, for $i=1,2$, we have
      \[
        \bfpp_{\uptau_{i}} = \tytb{s{x_{\uptaupp}},\vdots,s,{x_{\uptau_{i}}}} \quad \text{
          where }x_{\uptaupp_{i}}=c/d, x_{\uptau_{i}}=c/d.
      \]
      By looking at \eqref{eq:ped.ssc} and \eqref{eq:ped.ssd}, we see that
      $\pcL_{\uptau_{1}}=\pcL_{\uptau_{2}}$ implies
      \begin{equation}\label{eq:pnoticed.1}
        \pcL_{\uptaupp_{1}}=\pcL_{\uptaupp_{2}}.
      \end{equation}

      Moreover, we see that every components of $\cL_{\uptau_{i}}$ has a 1-row
      of mark ``$-/=$'' and we can determine $\upepsilon_{\uptau_{i}}$ by the
      mark $-/=$ of the 1-row appeared in $\pcL_{\uptau_{i}}$. In particular, we
      have $\upepsilon_{1}=\upepsilon_{2}$. Now \Cref{lem:gd.inj} implies
      $\uptaupp_{1}\neq \uptaupp_{2}$. This is contradict to
      \eqref{eq:pnoticed.1} and our induction hypothesis since $\cOpp$ is also
      +noticed.

      % On the other hand, $\cOpp$ is also +noticed. But \eqref{eq:ped.ssc} and
      % \eqref{eq:ped.ssd} also imply that
      % $\pcL_{\uptaupp_{1}}=\pcL_{\uptaupp_{2}}$ which is a contradiction to
      % our
      % induction hypothesis.
      % % By our descent algorithm $\eDD$, we have
      % % $\uptaupp_{1}\neq \uptaupp_{2}$.
      % % Note that the collision implies
      % % $\pcL_{\uptaupp_{1}} =\pcL_{\uptaupp_{2}}$,
      % % which is impossible by induction.
    \end{enumPF}
% \item
%  Suppose $\cO$ is noticed, we many need to consider the additional case where
%  $\bfpp_{\uptau} = \tytb{{x_{1}} y}$ where $x_{1}= c/d$. But $x_{1}=d$ is
%  equivalent to the non-vanishing of $\ncL_{\uptau}$.
% \end{enumPF}
\end{proof}


\begin{claim}\label{c:gd.noticed.inj}
  Suppose $\cO$ is noticed. Then the map $\Upsilon_{\cO}\colon \cL_{\uptau}\mapsto (\pcL_{\uptau},\ncL_{\uptau})$ is injective.
\end{claim}
\begin{proof}
  It suffice to consider the case where $\cO$ is noticed but not +noticed, i.e.
  $C_{2k+1}=C_{2k}+1$. In this case, $n=1$.
  The peduncle $\bfpp_{\uptau}$ have four possible cases:
  \[
    \tytb{rc}, \quad \tytb{cc}, \quad \tytb{cd},\quad \text{or}\quad  \tytb{dd}.
  \]
  By \eqref{eq:gd.ls}, $\pcL_{\uptau} = \dagger\dagger\pcL_{\uptaupp}$.
  % \[
  %   \bfpp_{1} = \tytb{rc}, \bfpp_{2}=\tytb{cc}, \bfpp_{3} = \tytb{cd}, \bfpp_{4} = \tytb{dd}.
  % \]

  Now suppose $\uptau_{1}\neq \uptau_{2}$ such that
  $\pcL_{\uptau_{1}}=\pcL_{\uptau_{2}}$.
  Since $\cOpp$ is +noticed, we have $\uptaupp_{1}=\uptaupp_{2}$ by the induction
  hypothesis (see \Cref{c:gd.pnoticed.p}).
  By \Cref{lem:gd.inj}, this only happens when
  $\set{\bfpp_{\uptau_{1}},\bfpp_{\uptau_{2}}}=\set{\tytb{cd},\tytb{dd}}$.
  Since one of $\ncL_{\uptau_{i}}$ is zero and the other is non-zero, we
  conclude that $\Upsilon_{\cO}(\cL_{\uptau_{1}})\neq \Upsilon_{\cO}(\cL_{\uptau_{2}})$.
\end{proof}

\begin{claim}\label{c:gd.pnoticed.n}
    Suppose $\cO$ is +noticed.
    The map $\nUpsilon_{\cO}\colon \set{\cL_{\uptau}|\ncL_{\uptau}\neq 0}\rightarrow \set{\ncL_{\uptau}\neq 0}$
    is injective.
\end{claim}
\begin{proof}
  The proof is similar to that of \Cref{c:gd.pnoticed.p}.
    We have two cases:
    \begin{enumPF}
      \item $\cL_{\uptau}$ contain an irreducible component $\succ \tytb{-,-}$.
      This is equivalent to $\ncL_{\uptau}$ has an irreducible component
      $\succ\tytb{-}$ and $\cL_{\uptau}\mapsto \ncL_{\uptau}$ will
      not kill any irreducible components. Hence we could recover $\cL_{\uptau}$
      from $\ncL_{\uptau}$.%, and the map $\Upsilon_{1}$.

      \item Now we make a weaker assumption that $\uptau_{1}\neq \uptau_{2}\in \drc(\cO)$ such that
      $\cL_{\uptau_{1}}$ and $\cL_{\uptau_{2}}$ do not contain a component
      $\succ \tytb{-,-}$.
      % We now show that  $\ncL_{\uptau_{1}}=\ncL_{\uptau_{2}}\neq 0$ leads to a
      % contradiction.
      % Clearly,
      % $\ssign(\uptau_{1})=\ssign(\uptau_{2})$

      By checking the list in \Cref{sec:z.r,sec:z.c,sec:z.d}, for $i=1,2$, we
      have $\bfpp_{\uptau_{i}}$ must be one of the following
      \[
        \bfpp_{1}=  \tytb{rc,\vdots,r,d} \quad \text{ or } \quad \bfpp_{2}= \tytb{rd,\vdots,r,d}.
      \]
      Without of loss of generality, we have the following possibilities:
      \begin{enumPF}
        \item $\bfpp_{\uptau_{1}}=\bfpp_{\uptau_{2}}=\bfpp_{1}$. We have
        $\pcL_{\uptaupp_{1}}=\pcL_{\uptaupp_{2}}$ and obtain the contradiction.
        \item $\bfpp_{\uptau_{1}}=\bfpp_{\uptau_{2}}=\bfpp_{2}$. We have
        $\ncL_{\uptaupp_{1}}=\ncL_{\uptaupp_{2}}$ and obtain the contradiction.
        \item $\bfpp_{\uptau_{1}}=\bfpp_{1}$ and $\bfpp_{\uptau_{2}}=\bfpp_{2}$.
        We get $\dagger\pcL_{\uptaupp_{1}}=\dagger \ncL_{\uptaupp_{2}}$.
        Note that $x_{\uptaupp_{1}}=r$.
        We inspect the list \Cref{sec:z.r,sec:z.c,sec:z.d} for the generalized descent case and
        the factorization \eqref{eq:ls.srcd} for the descent case. We conclude
        that $\pcL_{\uptaupp_{1}}$ must contain a 1-row with
        $+$ mark. We obtain the contradiction to
        $\dagger \pcL_{\uptaupp_{1}}=\dagger\ncL_{\uptaupp_{2}}$
        by looking at the associated
        character on the $\tytb{-+}$ row by the same argument in the
        case~\ref{it:c:noticed.bij.4} of the proof of \Cref{c:noticed.bij}.
      \end{enumPF}
    \end{enumPF}
    This finished the proof.
\end{proof}

\section{Type B/M case}


\subsection{``Special'' and ``non-special'' diagrams}
\subsubsection{type M}

\begin{defn}\label{def:sp-nsp.M.sp}
  Let $\cO = (C_{2k},C_{2k-1}, \cdots, C_{1},C_{0}=0)\in \dpeNil(M)$
  with $k\geq 1$.
  Let
  \[
  \tau =  (\tau_{2k}, \cdots, \tau_{2})\times  (\tau_{2k-1}, \cdots, \tau_{1}).
  \]
  be a Weyl group representation attached to $\cO\in \dpeNil(M)$.
  We say $\tau$ has
  \begin{itemize}
    \item \idxemph{special shape} if $\tau_{2k}\geq \tau_{2k-1}$;
    \item \idxemph{non-special shape} if $\tau_{2k}< \tau_{2k-1}$.
  \end{itemize}
  Clearly, this gives a partition of the set of Weyl group representations
  attached to $\cO$.

  Suppose $k\geq 1$, $C_{2k}=2c_{2k}-1$ and $C_{2k-1}=2c_{2k-1}+1$.
  The special shape representation $\tau^{s}$
  \[
    \tau^{s} = \tau_{L}^{s}\times \tau_{R}^{s}
    = (c_{2k},\tau_{2k-2}, \cdots, \tau_{2})\times (c_{2k-1},\tau_{2k-3} \cdots, \tau_{1}).
  \]
  is paired with the non-special shape representation
  \[
    \tau^{ns} = \tau_{L}^{ns}\times \tau_{R}^{ns} =
     (c_{2k-1},\tau_{2k-2},\cdots, \tau_{2})\times (c_{2k},\tau_{2k-3},\cdots, \tau_{1} )
  \]
  In the paring $\tau^{s}\leftrightarrow \tau^{ns}$, $\tau_{j}$  is unchanged for $j\leq 2k-2$.
  We call $\tau^{s}$ the \emph{special shape} and $\tau^{ns}$ the
  corresponding \emph{non-special shape}
\end{defn}
% We define a bijection between ``special'' diagram $\uptau^{s}$ and
% ``non-special'' diagrams $\uptau^{ns}$.

% \begin{defn}\label{def:sp-nsp.M.sp}

%   Let $\cO\in \Nil(M)$ such that $C_{2k}=2c_{2k}-1$ and $C_{2k-1}=2c_{2k-1}+1$.
%   A representation $\tau^{s}$ attached to $\cO$ has the shape
%   \[
%     \tau^{s} = \tau_{L}^{s}\times \tau_{R}^{s}=
%     (c_{2k},\tau_{2k-2},\cdots, \tau_{2}) \times (c_{2k-1},\tau_{2k-3},\cdots, \tau_{1})
%   \]
%   is paired with
%   \[
%     \tau^{ns} = \tau_{L}^{ns}\times \tau_{R}^{ns} =
%     (c_{2k-1},\tau_{2k-2},\cdots, \tau_{2}) \times (c_{2k},\tau_{2k-3},\cdots, \tau_{1})
%   \]
%   where the $c_{j}$ are unchanged for $j\leq 2k-2$.
%   We call $\tau^{s}$ the \emph{special shape} and $\tau^{ns}$ the
%   corresponding \emph{non-special shape}
% \end{defn}

% Note that the shape
% $\tau = (\tau_{2k},\cdots, \tau_{2},0)\times (\tau_{2k-1},\cdots, \tau_{1})$
% is a special shape if and only if $\tau_{2k}>\tau_{2k-1}$.



\begin{defn}\label{def:sp-nsp.M}
  Suppose $C_{2k}$ is odd.
We retain the notation in \Cref{def:sp-nsp.M.sp} where the shapes $\tau^{s}$ and $\tau^{ns}$
correspond with each other.
We define a bijection between $\drc(\tau^{s})$ and $\drc(\tau^{ns})$:
\[
  \begin{tikzcd}[row sep=0em]
    \drc(\tau^{s}) \ar[rr,<->]&\hspace{2em} & \drc(\tau^{ns})\\
    \uptau^{s} \ar[rr,<->]& & \uptau^{ns}
  \end{tikzcd}
\]
Without loss of generality, we can assume
that $\uptau^{s}\in \drc(\tau^{s})$ and $\uptau^{ns}\in \drc(\tau^{ns})$ have the following shapes:
\begin{equation}\label{eq:sp-nsp.M}
  \uptau^{s}:
  \tytb{\cdots\cdots\cdots,{y_{1}}\cdots\cdots,{*(srcol)s},{*(srcol)\vdots},{*(srcol)s},{y_{2}}}
  \times
  \tytb{\cdots\cdots\cdots,{x_{0}}{\cdots}{\cdots},\ ,\ ,\ ,\ }
  \longleftrightarrow \hspace{2em}
  \uptau^{ns}:
  \tytb{\cdots\cdots\cdots,{y_{0}}\cdots\cdots,\ ,\ ,\ ,\ }
  \times
  \tytb{\cdots\cdots\cdots,{x_{1}}{\cdots}\cdots,{*(srcol)r},
    {*(srcol)\vdots},{*(srcol)r},{x_{2}} }
\end{equation}
Here the grey parts have length $(C_{2k}-C_{2k-1})/2$ (could be zero length).
The entries $x_{0}$, $x_{1}$, $y_{0}$ and $y_{1}$ are either all empty or all
non-empty.
% and the rows contains them may be empty. % We remark that the
% value of $v_{0}$ and $w_{0}$ are completely determined by $x_{0}$ and $y_{0}$.
% In particular, $v_{0}$ and $w_{0}$ are non-empty if and only if
% $x_{0}/y_{0}\neq \emptyset$, i.e. $C_{2k-1}\geq 4$.
% We marks the entries of $\uptau^{s}$ and $\uptau^{ns}$ as in
% \eqref{eq:sp-nsp.C}.
The correspondence between $(x_{0},x_{1}, x_{2})$, with
$(y_{0},y_{1},y_{2})$ is given by switching $r$ and $s$,  $d$ and $c$. i.e.
\[
  \begin{cases}
    y_{i} = s \Leftrightarrow x_{i} = r\\
    y_{i} = c \Leftrightarrow x_{i} = d\\
  \end{cases}
  \quad \text{for } i=0,1,2.
\]

We define
\[
  \drcs(\cO) := \bigsqcup_{\tau^{s}} \drc(\tau^{s}) \quad \text{ and }\quad
  \drcns(\cO) := \bigsqcup_{\tau^{ns}} \drc(\tau^{ns})
\]
where $\tau^{s}$ (resp. $\tau^{ns}$)runs over all specail (resp. non-special) shape representations attached to $\cO$.
Clearly $\drc(\cO) = \drcs(\cO)\sqcup \drcns(\cO)$.
\end{defn}
It is easy to check the bijectivity in the above definition, which we leave it
to the reader.



\subsubsection{Special and non-special shapes of type B}
Now we define the notion of special and non-special shape of type B.
\begin{defn}\label{def:sp-nsp.B.sp}
  Let
  \[
    \tau = (\tau_{2k}, \cdots, \tau_{2}) \times (\tau_{2k+1},\tau_{2k-1}, \cdots, \tau_{1})
  \]
  be a Weyl group representation attached to $\cO\in \dpeNil(B)$. We say $\tau$
  has
  \begin{itemize}
    \item \idxemph{special shape} if $\tau_{2k}\geq \tau_{2k-1}$;
    \item \idxemph{non-special shape} if $\tau_{2k} < \tau_{2k-1}$.
  \end{itemize}

  Suppose $k\geq 1$ and
  \begin{equation} \label{eq:B.orb.ds}
    \cO=(C_{2k+1}=2c_{2k+1}+1,C_{2k}=2c_{2k}-1,C_{2k-1}=2c_{2k-1}+1,\cdots,C_{1}>0, C_{0}=0).
  \end{equation}
  A representation $\tau^{s}$ attached to $\cO$ has the shape
  \[
    \tau^{s} = \tau_{L}^{s}\times \tau_{R}^{s}= (c_{2k},\tau_{2k-2},\cdots, \tau_{2}) \times (c_{2k+1},c_{2k-1},\tau_{2k-3},\cdots, \tau_{1})
  \]
  is paired with
  \[
    \tau^{ns} = \tau_{L}^{ns}\times \tau_{R}^{ns} = (c_{2k-1},\tau_{2k-2},\cdots, \tau_{2}) \times (c_{2k+1},c_{2k},\tau_{2k-3},\cdots, \tau_{1}).
  \]
  In the paring $\tau^{s}\leftrightarrow \tau^{ns}$ described as the above,
  $\tau_{j}$ are unchanged for $j\leq 2k-2$.
\end{defn}




\subsection{Definition of $\eDD$ for type B and M}


\subsubsection{The defintion of $\upepsilon$.} \label{sec:upepsilon}
\begin{enumerate}[label=(\arabic*).,series=alg2]
  \item When $\uptau\in \drc(B)$, $\upepsilon$ is determined by the ``basal
  disk'' of $\uptau$ (see \eqref{eq:x.uptau} for the definition of $x_{\uptau}$):
  \[
    \upepsilon_{\uptau}:=
    \begin{cases}
      0, & \text{if $x_{\uptau}=d$;} \\
      1, & \text{otherwise.}
    \end{cases}
  \]
  \item When $\uptau\in \drc(M)$, the  $\upepsilon$  is determined by
  the lengths of $\taulf$ and $\taurf$:
  \[
    \upepsilon_{\uptau} :=
    \begin{cases}
      0, & \text{if }\abs{\taulf} - \abs{\taurf} \geq  0;\\
      1, & \text{if }\abs{\taulf} - \abs{\taurf}<0.
    \end{cases}
  \]
\end{enumerate}


\subsubsection{Initial cases}

\begin{enumerate}[resume*=alg2]
  \item Suppose $\cO = (2c_{1}+1)$ has only one column. Then
        $\cOp_{0}:=\eDD(\cO)$ is the trivial orbit of $\Mp(0,\bR)$. $\drc(\cO)$
        consists of diagrams of shape
        $\tau_{L}\times \tau_{R} =\emptyset\times (c_{1},)$.
        The set $\drc(\cOp_{0})$ is a singleton $\set{\uptaup_{0}=\emptyset\times \emptyset}$, and every element
        $\uptau \in \drc(\cO)$ maps to $\uptau_{0}$ (see \eqref{eq:uptau0}):
        \[\tiny
          \drc(\cO)\ni \uptau:= \hspace{1em} \ytb{\emptyset,\none,\none,\none}
          \times \ytb{{x_{1}},\vdots,{x_{n}},{m_{\uptau}}}
          \mapsto \uptaup_{0}:=\emptyset\times \emptyset
        \]
        Here $m_{\uptau}=a$ or $b$ is the mark of $\uptau$.
        We define
        \[
          \bfpp_{\uptau}:=\bfxx_{\uptau} := x_{1}\cdots x_{n} m_{\uptau}.%  \text{ and }x_{\uptau}:=x_{n}
        \]
        We call $\bfpp_{\uptau}=\bfxx_{\uptau}$ the ``peduncle'' part of
        $\uptau$.
\end{enumerate}

\subsubsection{The descent from M to B}
The descent of a special shape diagram is simple, the descent of a non-special shape
reduces to the corresponding special one:
\begin{enumerate}[resume*=alg2]
  \item Suppose $\abs{\taulf} \geq \abs{\taurf} $, i.e. $\uptau$ has special
        shape. Keep $r,d$ unchanged, delete $\taulf$ and fill the remaining part
        with ``$\bullet$'' and ``$s$'' by $\DDr(\uptau^{\bullet})$ using the
        dot-s switching algorithm. The mark $m_{\uptaup}$ of
        $\uptaup:=\eDDo(\uptau)$ is given by the following formula:
        % \footnote{Note that $\taulf$ ends with $s$ or $c$ since
        % $\abs{\taulf}>\abs{\taurf}$}
        \[
        m_{\uptaup}:= \begin{cases}
          a & \text{if  $\taulf$ ends with $s$ or $\bullet$},\\
          b & \text{if $\taulf$ ends with $c$}.
        \end{cases}
        \]

  \item Suppose $\abs{\taulf} < \abs{\taurf}$, i.e. $\uptau$ has non-special
        shape. We define
        \[\eDDo(\uptau):=\eDDo(\uptau^{s})\]
        where $\uptau^{s}$ is the special diagram corresponding to $\uptau$
        defined in \Cref{def:sp-nsp.M}.
\end{enumerate}

\begin{lem}\label{lem:ds.BM}
  Suppose $\cO = (C_{2k},C_{2k-1}, \cdots, C_{0})$ with $k\geq 1$ and $C_{2k}$
  is odd.
  Let $\cOp := \eDD(\cO)=(C_{2k-1},\cdots, C_{0})$.
  Then
  \[
    \begin{split}
      \drcs(\cO) &\xrightarrow{\hspace{2em}\eDDo\hspace{2em}} \drc(\cOp) \\
      \drcns(\cO)&
      \xrightarrow{\hspace{2em}\eDDo\hspace{2em}} \drc(\cOp)
    \end{split}
  \]
  are a bijections.
\end{lem}
\begin{proof}
  The claim for $\drcs(\cO)$ is clear by the definition of descent.
  The claim for $\drcns(\cO)$ reduces to that of $\drcs(\cO)$ using \Cref{def:sp-nsp.M}.
\end{proof}

\begin{lem}\label{lem:gd.BM}
  Suppose $\cO = (C_{2k},C_{2k-1}, \cdots, C_{0})$ with $k\geq 1$ and $C_{2k}$ even.
  Let $\cOp := \eDD(\cO)=(C_{2k-1}+1,\cdots, C_{0})$.
  Then the following map is a bijection
  \[
    \drc(\cO) \xrightarrow{\hspace{2em}\eDDo\hspace{2em}}
    \Set{\uptaup\in \drc(\cOp)| \begin{minipage}{11em}
        $m_{\uptaup}\neq b$ and\\
         $\uptaup_{R,0}$ dose not ends with $s$.
      \end{minipage}
    }. %\subsetneq \drc(\cOp).
  \]
\end{lem}
\begin{proof}
  This is clear by the descent algorithm.
\end{proof}


\subsubsection{Descent from $B$ to $M$}
Now we define the general case of the descent from type $B$ to type $M$.
We assume that $k\geq 1$ i.e. $\cO$ has at least 3 columns.
First note that the shape of $\uptau' = \eDD(\uptau)$ is the shape of $\uptau$ deleting the
most left column on the right diagram.

% We assume $\uptau$ and $\uptau'$ has the following shape where $(y_{1},y_{2})$
% could be empty and $w_{0}$ is non-empty if $\uptau_{R}\neq \emptyset$.
The definition splits in cases below. In all these cases we define
\begin{equation}\label{eq:x.uptau}
 % \bfpp_{\uptau}:=
 \tbfxx_{\uptau} := x_{1}\cdots x_{n}m_{\uptau} \text{ and } \txx_{\uptau} := x_{n}
\end{equation}
which is marked by $s/r/c/d$.
For the part marked by $*/\cdots$ , $\eDD$ keeps $r,c,d$ and maps the rest part consisting of $\bullet$ and $s$ by dot-s switching algorithm.
\begin{enumerate}[resume*=alg2]
  \item When $C_{2k}$ is odd and $\uptau$ has special shape, the descent is given by the following diagram
      \[
        \uptau: \hspace{1em}
        \tytb{
        \cdots\cdots\cdots,
        {\ast}{\cdots}{\cdots},
        {*(srcol)\bullet},
        {*(srcol)\vdots},{*(srcol)\bullet},{y_0},\none,\none,\none}
      \times
      \tytb{\cdots\cdots\cdots,
        {\ast}{\ast}{\cdots},
        {*(srcol)\bullet},
        {*(srcol)\vdots},
        {*(srcol)\bullet},
        {x_{1}},{\vdots},{x_{n}},{m_{\uptau}}}
        \mapsto
       \uptaup: \tytb{\cdots\cdots\cdots,{\ast}{\cdots}{\cdots},{*(srcol)s},{*(srcol)\vdots},{*(srcol)s},{y'_{0}},\none,\none,\none}
        \times \tytb{\none\cdots\cdots,\none{\ast}{\cdots},\none,\none,\none,\none,\none,\none,\none}
      \]
      Here the grey columns has length $(C_{2k}-C_{2k-1})/2$ and $n = (C_{2k+1}-C_{2k})/2$.
      The  ``$\ast/\cdots$'' part of $\uptaup$ is given by keeping the corresponding
      entries marked by $r/d/c$ in
       $\uptau$ unchange and filling $s/\bullet$ accordingly in the rest of the
       entries.
       The entry $y'_{0}$ of $\uptaup$ is given by the following formula:
       % \footnote{When $x_{1}\cdots x_{n}=r\cdots r$, we have $y_{0}=c$.
       %   Otherwise $x_{1}=s$ or $x_{1}\cdots x_{n} = r\cdots r d$.}
       % \[
       %   y'_{0} := \begin{cases}
       %     s & \text{if $x_{1}=\bullet$ or
       %       $x_{1}\cdots x_{n} m_{\uptau}= r\cdots ra$ or $r\cdots rda$}\\
       %     c & \text{if $x_{1}=s$ or
       %       $x_{1}\cdots x_{n} m_{\uptau}= r\cdots rb$ or $r\cdots rdb$}.
       %   \end{cases}
       % \]
       \[
         y'_{0} := \begin{cases}
           s & \text{if $x_{1}=\bullet$ or
             $(x_{1},m_{\uptau})=(r/d, a)$}\\
           c & \text{if $x_{1}=s$ or
             $(x_{1},m_{\uptau})=(r/d, b)$}.
         \end{cases}
       \]
       % We define $\tbfxx_{\uptau}$ be the diagram by replacing $\bullet$ to $s$
       % in $\bfxx_{\uptau}$.
       %  We define $\tfbfxx_{\uptau}\in \DRC()$
  \item When $C_{2k}$ is odd and $\uptau$ has non-special shape, then
        \[
        \uptau: \hspace{1em}
        \tytb{
        {\cdots}\cdots\cdots,
        {\ast}{\cdots}{\cdots},
        \none,\none,\none,\none,\none,\none,\none}
        \times
        \tytb{\cdots\cdots\cdots,
        {\ast}{\ast}{\cdots},
        {*(srcol)s}{*(srcol)r},
        {*(srcol)\vdots}{*(srcol)\vdots},
        {*(srcol)s}{*(srcol)r},
        {x_{1}}{x_{0}},{\vdots},{x_{n}},{m_{\uptau}}}
        \mapsto
        \uptaup: \tytb{\cdots\cdots\cdots,{\ast}{\cdots}{\cdots},\none,\none,\none,\none,\none,\none,\none}
        \times \tytb{
        \none\cdots\cdots,\none{\ast}{\cdots},
        \none{*(srcol)r},
        \none{*(srcol)\vdots},
        \none{*(srcol)r},
        \none{x'_{0}},\none,\none,\none}
        \]
        Here the grey columns has length $(C_{2k}-C_{2k-1})/2$ and $n = (C_{2k+1}-C_{2k})/2$.
        The  ``$\ast/\cdots$'' part of $\uptaup$ is given by keeping the corresponding
        entries marked by $r/d/c$ in
        $\uptau$ unchange and filling $s/\bullet$ accordingly in the rest of the
        entries.
        The entry $x'_{0}$ of $\uptaup$ is given by the following formula:
        % \footnote{When $x_{1}\cdots x_{n}=r\cdots r$, we have $x_{0}=d$.}
        \[
        x'_{0} := \begin{cases}
          r & \text{if $(x_{1},m_{\uptau})=(r/d, a)$,}\\
          d & \text{if $(x_{1},m_{\uptau})=(r/d, b)$,}\\
          x_{0} & \text{if $x_{1}=s$.}
        \end{cases}
        \]
        Note that
        % \[
        %   x'_{0} := \begin{cases}
        %     r & \text{if $x_{1}\cdots x_{n}= r\cdots r$}\\
        %     x_{0} & \text{otherwise}
        %   \end{cases}
        % \]
        % We define $\tbfxx_{\uptau}:=\bfxx_{\uptau}$.
        %
  \item When $C_{2k}=C_{2k-1}$ is even, we could assume $\uptau$ and $\uptaup$ hase
        the following forms with $n = (C_{2k+1}-C_{2k}+1)/2$.
      \[
        \uptau: \hspace{1em}
        \tytb{
        {\cdots}\cdots\cdots,
        {\ast}{\cdots}{\cdots},
        {y_{0}}\cdots,\none,\none,\none}
      \times
      \tytb{\cdots\cdots\cdots,
        {\ast}{\ast}{\cdots},
        {x_{1}}{x_{0}},{\vdots},{x_{n}},{m_{\uptau}}}
        \mapsto
       \uptaup: \tytb{\cdots\cdots\cdots,{\ast}{\cdots}{\cdots},{y'_{0}}\cdots,\none,\none,\none}
        \times \tytb{\none\cdots\cdots,\none{\ast}{\cdots},\none{x'_{0}},\none,\none,\none}
      \]
      In the most of the case, $\uptau'$ is obtained from $\uptau$ by deleting
      the first column of $\uptau_{R}$, keeping $c/r/d$ and filling $s/\bullet$ accordingly.
      The exceptional cases are when $x_{1}=r/d$. In the exceptional case we
      always have $x_{0}=d$.  We define
      \[
        \begin{split}
          x'_{0} & := d, \\
          y'_{0} & := \begin{cases}
            s & \text{if $m_{\uptau}=a$},\\
            c & \text{if $m_{\uptau}=b$}.
          \end{cases}
        \end{split}
      \]
      Note that this definition is similar to that of the special
      shape diagrams in the descent case.
      We define the ``peduncle'' of $\uptau$ to be
      \[
        \bfpp_{\uptau} = \tytb{{x_{1}}{x_{0}}, \vdots, {x_{n}},{m_{\uptau}}}.
      \]
\end{enumerate}

In all the cases, let
$\cO_{1} = (2n,)$ is the trivial nilpotent of type D.
For each $\uptau\in \drc(\cO)$, we define $\bfxx_{\uptau} \in \drc(\cO_{1})$ by
require the following identity of multisets: %the entires of $\bfxx_{\uptau}$ forms the following multiset:
\[
\set{\text{entry of }\bfxx_{\uptau}} =
\begin{cases}
  \set{c, x_{2},\cdots, x_{n}} & \text{if $(x_1,m_{\uptau}) = (\bullet/s,a)$},\\
  \set{s, x_{2},\cdots, x_{n}} & \text{if $(x_1,m_{\uptau}) = (\bullet/s,b)$},\\
  \set{x_{1}, x_{2},\cdots, x_{n}} & \text{otherwise ($x_{1}=r/d$)}.
\end{cases}
\]
In the above definition we include the initial case where $\cO$ is a trivial
orbit of type B. We define $\bfxx_{\uptau}=\emptyset$ when $\cO$ is the trivial
orbit of $\rO(1,\bC)$.

We define the  ``basal disk'' of $\uptau$ as the following:
\[
  x_{\uptau} :=\begin{cases}
    % \emptyset & \text{if $\cO = (1)$}\\
    \bfxx_{\uptau} & \text{if $C_{2k+1}=C_{2k}+1$}.\\
    c & \text{if $x_{n}m_{\uptau} =sa$ or $\bullet a$}\\
    x_{n} & \text{otherwise}. % \text{if $C_{2k+1}-C_{2k}\geq 1$ and $x_{n}m_{\uptau}\neq sa$}\\
  \end{cases}
\]


\subsubsection{A key proposition for descent case}
Now we assume $\cO$ is given by \eqref{eq:B.orb.ds}
Let $\cOp=\eDD(\cO)$ and $\cOpp=\eDD(\cOp)$.



The following lemma is the key property satisfied by our definition
\begin{lem}\label{lem:sp-nsp.B}
  Suppose $k\geq 1$ and
  $\tau^{s}$ and $\tau^{ns}$ are two representations attached to $\cO$ as in
  \Cref{def:sp-nsp.B.sp}.
  Then
  \begin{enumS}
    \item \label{lem:sp-nsp.B.1} For every $\uptau\in \drc(\tau^{s})\sqcup \drc(\tau^{ns})$, the shape
    of $\eDD^{2}(\uptau)$ is
    \[
      \taupp = (\tau_{2k-2},\cdots, \tau_{2})\times  (c_{2k-1},\tau_{2k-3},\tau_{1})
    \]
    \item \label{lem:sp-nsp.B.2} % Let $\cO_{1}= (C_{2k+1}-C_{2k}+1,)\in \Nil(B)$ be the trivial orbit
    % of $\rO(C_{2k+1}-C_{2k}+1,\bC)$.
    We define $\delta\colon \drc(\cO)\rightarrow \drc(\cOpp)\times\drc(\cO_{1})$
    by $\delta(\uptau) = (\eDD^{2}(\uptau),\bfxx_{\uptau})$.
    The following maps are bijective%with the same image
    \[
      \begin{tikzcd}[row sep=0em]
        \drc(\tau^{s})\ar[r,"\delta"] & \drc(\taupp)\times \drc(\cO_{1}) &\ar[l,"\delta"'] \drc(\tau^{ns})\\
        \uptau^{s}\ar[r,maps to] & (\eDD^{2}(\uptau^{s}), \bfxx_{\uptau^{s}})&\\
        & (\eDD^{2}(\uptau^{ns}), \bfxx_{\uptau^{ns}})& \uptau^{ns}\ar[l,maps to]\\
      \end{tikzcd}
    \]
    In particular, we obtain an one-one correspondence
    $\uptau^{s}\leftrightarrow \uptau^{ns}$ such that $\delta(\uptau^{s})=\delta(\uptau^{ns})$.
    \item\label{lem:sp-nsp.B.3}
    Suppose $\uptau^{s}$ and $\uptau^{ns}$ correspond as the above such that
    $\delta(\uptau^{s})=\delta(\uptau^{ns})=(\uptaupp,\bfxx)$. Then
    \begin{equation} \label{eq:sp-nsp-sig}
      \ssign(\uptau^{s})=\ssign(\uptau^{ns})=(C_{2k},C_{2k})+\ssign(\uptaupp)+\ssign(\bfxx).
    \end{equation}
    % with
    % \[
    %   \tsign(\tbfxx) = \begin{cases}
    %     \ssign(x_{1}\cdots x_{n})  & \text{if $x_{1}=r/d$}\\
    %     \ssign(\tbfxx) - (0,1) & \text{otherwise.}\\
    %   \end{cases}
    % \]
    Note that $\ssign(\bfxx)\in \bN\times \bN$.
  \end{enumS}
\end{lem}
\begin{proof}
  By our assumption, we have $c_{2k+1}\geq c_{2k}>c_{2k-1}$ and
  \[
    \tau^{s} = (c_{2k},\tau_{2k-2},\cdots, \tau_{2})\times (c_{2k+1}, c_{2k-1}).
  \]
  The the behavior of $\uptau^{s}$ under the descent map $\eDD$ is illostrated
  as the following:
  \begin{equation}\label{eq:ds2.B.sp}
    \uptau^{s}: \hspace{1em}
    \tytb{
      {*(srcol)\cdots}\cdots\cdots,
      {*(srcol)\ast}{\ast}{\cdots},
      {*(srcol)\bullet},
      {*(srcol)\vdots},{*(srcol)\bullet},{x_0},\none,\none,\none}
    \times
    \tytb{{*(srcol)\cdots}\cdots\cdots,
      {*(srcol)\ast}{\ast}{\cdots},{*(srcol)\bullet},{*(srcol)\vdots},{*(srcol)\bullet},{x_{1}},\vdots,{x_{n}},{m_{\uptau}}}
    \mapsto
    \uptau'^{s}: \hspace{1em}
    \tytb{{*(srcol)\cdots}\cdots\cdots,{*(srcol)\ast}{\ast}{\cdots},{*(srcol)s},{*(srcol)\vdots},{*(srcol)s},{x'_{0}},
      \none,\none,\none}
    \times \tytb{\none\cdots\cdots,\none{\ast}{\cdots},\none,\none,\none,\none,\none,\none,\none}
    \mapsto
    \uptaupp:
    \tytb{\none\cdots\cdots,\none{\ast}{\cdots},\none,\none,\none,\none,\none,\none,\none}
    \times \tytb{\none\cdots\cdots,{\none}{\ast}{\cdots},\none{\phantom{m}\mathclap{m_{\uptaupp}}},\none,\none,\none,\none,\none,\none}
  \end{equation}
  The grey part consists of totally $2(c_{2k}-1)=C_{2k}-1$ dots.
  Hence the total signature of the gray part is $(C_{2k}-1,C_{2k}-1)$.
  Note that $\uptau^{s}$ is obtained from the ``$\ast\cdots$'' part of $\taupp$ by attaching the grey part
  and $x_{0},\cdots, x_{n},m_{\uptau}$.

  The descent of a non-special diagram $\uptau^{ns}$:
  \begin{equation}\label{eq:ds2.B.nsp}
        \uptau^{ns}: \hspace{1em}
        \tytb{
        {*(srcol)\cdots}\cdots\cdots,
        {*(srcol)\ast}{\ast}{\cdots},
        \none,\none,\none,\none,\none,\none,\none}
      \times
      \tytb{{*(srcol)\cdots}\cdots\cdots,
        {*(srcol)\ast}{\ast}{\cdots},
        {*(srcol)s}{*(srcol)r},
        {*(srcol)\vdots}{*(srcol)\vdots},
        {*(srcol)s}{*(srcol)r},
        {x_{1}}{x_{0}},{\vdots},{x_{n}},{m_{\uptau}}}
        \mapsto
       {\uptaup}^{ns}: \tytb{{*(srcol)\cdots}\cdots\cdots,{*(srcol)\ast}{\ast}{\cdots},\none,\none,\none,\none,\none,\none,\none}
       \times
       \tytb{\none\cdots\cdots,\none{\ast}{\cdots},
         \none{*(srcol)r},
         \none{*(srcol)\vdots},
         \none{*(srcol)r},
         \none{x'_{0}},\none,\none,\none}
       \mapsto
       \uptaupp:
       \tytb{\none\cdots\cdots,\none{\ast}{\cdots},\none,\none,\none,\none,\none,\none,\none}
       \times \tytb{\none\cdots\cdots,{\none}{\ast}{\cdots},\none{\phantom{m}\mathclap{m_{\uptaupp}}},\none,\none,\none,\none,\none,\none}
     \end{equation}
     Here the gray parts in $\uptau$ consists of $C_{2k}-1$ marks of $\bullet$, $s$, or
     $r$ and $s$ and $r$ occur with the same multiplicity. Hence the total
     signature of the gray part is $(C_{2k}-1,C_{2k}-1)$.

     To prove the lemma, it suffice to verify the following claims which we leave to the reader:
     \begin{itemize}
       \item In both the special shape and non-special shape cases, the
       following map is bijective:
       \[
       x_{0}x_{1}\cdots x_{n}m_{\uptau^{s/ns}}\mapsto (\bfxx_{\uptau^{s/ns}},m_{\uptaupp}).
       \]
       \item The following equation of signatures holds
       \[
         \ssign( x_{0}x_{1}\cdots x_{n}m_{\uptau} ) - \ssign(m_{\uptaupp})  =\ssign(\bfxx_{\uptau}).
       \]
     \end{itemize}
     In the verification, one can use the fact that $m_{\uptau}=m_{\uptaupp}$
     when $x_{1} = r/d$.

     \trivial[]{
       Suppose $x_{1}=r/d$. Then $x_{0} = c/d$ and
       \[\ssign(x_{0}\cdots x_{n}m_{\uptau})-\ssign(m_{\uptaupp}) = (1,1)+\ssign(x_{1}\cdots x_{n}).\]

       Now we consider the generic cases, i.e. $x_{1}=\bullet/s$:
       In this case, we clain that $\ssign(x_{0}x_{1}) - \ssign(m_{\uptaupp}) = (1,2)$.
       Now
       \[
         \begin{split}
           & \ssign(x_{0}x_{1}\cdots x_{n}m_{\uptau})-\ssign(m_{\uptaupp}) \\
           =& (1,1)+\ssign(x_{2}\cdots x_{n}m_{\uptau}) +(0,1) \\
           = &\begin{cases} (1,1) +\ssign(x_{2}\cdots x_{n}) + \ssign(c)
             & \text{if } m_{\uptau} = a\\
             (1,1) +\ssign(x_{2}\cdots x_{n}) + \ssign(s) & \text{if } m_{\uptau} = b\\
           \end{cases}\\
         \end{split}
       \]

       First consider the special shape diagram $\uptau=\uptau^{s}$.
       \begin{enumPF}
         \item Suppose $m_{\uptaupp}=a$. Then $x_{0}\times x_{1}=\bullet\times \bullet$.
           \item Suppose $m_{\uptaupp}= b$. Then $x_{0}\times x_{1}=c\times s$.
       \end{enumPF}
       Now consider the non-special shape diagram $\uptau=\uptau^{ns}$.
       \begin{enumPF}
         \item Suppose $m_{\uptaupp}=a$. Then
         $\emptyset \times x_{1}x_{0}=\emptyset \times sr$.
         \item Suppose $m_{\uptaupp}= b$. Then
         $\emptyset \times x_{1}x_{0}=\emptyset \times sd$.
       \end{enumPF}
       The macthing between $\uptau^{s}$ and $\uptau^{ns}$ is also clear by the
       above listing of cases.
     }
 \end{proof}



\subsubsection{A key proposition for the generalized descent case}

Now assume $C_{2k}$ is even. By our assumption, we have
$c_{2k+1}\geq c_{2k}=c_{2k-1}$.

\begin{equation}\label{eq:gd.drc.B}
  \uptau: \hspace{1em}
  \tytb{
    {*(srcol)\cdots}\cdots\cdots,
    {y_{0}}{\cdots}{\cdots},
    \none,\none,\none,\none}
  \times
  \tytb{{*(srcol)\cdots}\cdots\cdots,
    {x_{1}}{x_{0}}{\cdots},
    {x_{2}},{\vdots},{x_{n}},{m_{\uptau}}}
  \mapsto
  \uptaup: \tytb{{*(srcol)\cdots}\cdots\cdots,{y'_{0}}{\cdots}{\cdots},\none,\none,\none,\none}
  \times \tytb{\none\cdots\cdots,\none{x'_{0}}{\cdots},\none,\none,\none,\none}
  \mapsto
  \uptaupp: \tytb{\none\cdots\cdots,{\none}{\cdots}{\cdots},\none,\none,\none,\none}
  \times \tytb{\none\cdots\cdots,\none{x''}{\cdots},\none{\phantom{m}\mathclap{m_{\uptaupp}}},\none,\none,\none}
\end{equation}
Here the gray parts in $\uptau$ are two columns of $\bullet$ of length
$C_{2k}/2-1$.

\begin{lem}\label{lem:gd.inj.M}
  In \eqref{eq:gd.drc.B}, $x''=x_{0}$.
  The map $\delta\colon  \drc(\cO)\rightarrow \drc(\cOpp) \times \drc(\cO_{1})$
  given by $\uptau\mapsto (\eDDo^{2}(\uptau),\bfxx_{\uptau})$ is injective.
  Moreover,
  \[
  \ssign(\uptau) =\ssign(\uptaupp) + (C_{2k}-1, C_{2k}-1)+\ssign(\bfxx_{\uptau}).
  \]
  The map $\tdelta\colon \drc(\cO)\rightarrow \drc(\cOpp) \times \bN^{2}\times \bZ/2\bZ$
  given by $\uptau\mapsto (\eDDo^{2}(\uptau), \ssign(\uptau),\upepsilon_{\uptau})$ is injective.
\end{lem}
\begin{proof}
  The claim $x''=x_{0}$, the injectivity of $\delta$ and the siginature formula follows directly from our algorithm.

  Now the injectivity of $\tdelta$ follows from
  $\bfxx_{\uptau}\mapsto (\ssign(\bfxx_{\uptau}), \upepsilon)$ is injective by
  \Cref{c:init.CD}.
  \trivial{
    We can fully recover $\uptau$ from $\bfxx_{\uptau}$ and $\uptaupp$.

    Suppose $\bfxx_{\uptau}$ dose not contains $s$ or $c$.
    Then $x_{1}=r/d$,  $m_{\uptau}=m_{\uptaupp}$, $y_{0}=c$.
    Hence the claim holds.
    \[
      \ssign(\uptau) =\ssign(\uptaupp) + (C_{2k}-2,C_{2k}-2) + \ssign(c) +\ssign(\bfxx_{\uptau})
    \]

    Now assume $\bfxx_{\uptau}$ contain at least one $s$ or $c$.
    Now
    \[
      m_{\uptau} = \begin{cases}
        a & \text{if $\bfxx_{\uptau}$ contains $c$}\\
        b & \text{otherwise}
      \end{cases}
    \]
    Therefore, $\ssign(x_{2}\cdots x_{n}m_{\uptau}) = \ssign(\bfxx_{\uptau}) - (0,1)$.


    Clearly, we can recover $x_{2}\cdots x_{n}$ from $\bfxx_{\uptau}$ by
    deleting the $c$ (if it exists) or a $s$.
    On the other hand, $(y_{0}, x_{1})$ is completely determined by
    $m_{\uptaupp}$:
    \[
      (y_{0},x_{1}) = \begin{cases}
        (\bullet, \bullet) & \text{if $m_{\uptaupp}=a$}\\
        (c, s) & \text{if $m_{\uptaupp}=b$}
      \end{cases}
    \]
    Hence $\ssign(y_{0}x_{1}) = \ssign(m_{\uptaupp}) + (1,2)$.

    Now
    $\ssign(y_{0}x_{0}x_{1}\cdots x_{n}m_{\uptau})  =  \ssign(\bfxx_{\uptau}) +\ssign(x'' m_{\uptaupp}) + (1,2)-(0,1)$.
    This yields the signature identity.
  }
\end{proof}


\subsection{Proof of the type BM case}
We will reduce the proof to the type CD case.

In this section, $k\geq 1$ and
$\cO = (C_{2k+1}, C_{2k}, \cdots,C_{1}, C_{0}=0)\in \dpeNil(B)$.
$\cOp:=\eDD(\cO)$ and $\cOpp := \eDD(\cOp)$.

Take $\uptau\in \drc(\cO)$, we marks the entries in $\uptau$ as in  \cref{eq:ds2.B.sp,eq:ds2.B.nsp,eq:gd.drc.B}.
\subsubsection{Usual decent case}
Thanks to \Cref{lem:sp-nsp.B}, the proof in the descent case is exactly the same
as that in \Cref{sec:pf.ds.CD}.

\subsubsection{Reduction to type CD case in the general descent case.}

Suppose $C_{2k}$ is even.
We define $\tcO = (C_{2k+1}-C_{2k}+1,1,1)\in\dpeNil(D)$ and
$\tcOpp := \eDDo(\tcO)= (2)$.

The key proposition of reduction.
\begin{prop}
 We have  the following properties:
  \begin{enumC}
    \item When $\uptaupp = \eDDo^{2}(\uptau)$ for certain $\uptau\in \drc(\cO)$,
    we have $x_{\uptaupp}\neq s$;
    \item For each $\uptaupp$ such that $x_{\uptaupp}\neq s$, we have a
    bijection
    \[
      \begin{tikzcd}[row sep=0em]
       \bdelta \colon  \set{\uptau\in \drc(\cO)| \eDDo^{2}(\uptau)=\uptaupp} \ar[r] &
        \set{(x_{\tuptau},\bfuu_{\tuptau})| \tuptau\in \drc(\tcO) \text{ s.t.
          } \eDDo^{2}(\tuptau)=x_{\uptaupp}}\\
        \uptau \ar[r,mapsto] & (x_{\uptaupp}, \bfxx_{\uptau})
      \end{tikzcd}
  \]
  where $\bfuu_{\tuptau}$ is defined in \eqref{eq:def.u}.
  \end{enumC}
\end{prop}


The situation could be summarized in the following commutative diagram:
  \[
    \begin{tikzcd}
     & \uptau \ar[r,"\delta",mapsto] \ar[d,mapsto]& (\uptaupp, \bfxx_{\uptau})\ar[d,mapsto]\ar[r,mapsto] & \uptaupp\ar[d,mapsto]\\
     x_{\uptau} \ar[d,leftrightarrow,dashed] &\ar[l,mapsto]  \bfpp_{\uptau}\ar[r,mapsto]\ar[d,leftrightarrow]
     & (x_{\uptaupp}, \bfxx_{\uptau}) \ar[d,equal]\ar[r,mapsto]& x_{\uptaupp} \ar[d,equal]\\
    x_{\tuptau} & \ar[l,mapsto] \tuptau \ar[r,"\delta",mapsto]& (\tuptaupp,\bfuu_{\tuptau})\ar[r,mapsto]& \tuptaupp
    \end{tikzcd}
  \]
  We remark that $x_{\uptau}$ and $x_{\tuptau}$ are equal in the most of the
  case, especially when $n=1$. But there are exceptional cases.

  \begin{proof}
    This follows from a case by case verification according to our algorithm.
    We list all possible cases below:

    When $n=1$, see \Cref{tb:rd1}.% We have the following table:
    \begin{table}[p]
      \[
        \begin{array}{c|c|c|c|c}
          \hline
          \hline
          x_{\uptau} & \tytb{{\txx_{\uptaupp}},{m_{\uptaupp}}} & \bfpp_{\uptau} & (x_{\uptaupp},\bfxx_{\uptau}) & \tuptau_{L} \\
          \hline
          r & \tytb{r,a} &  \tytb{\bullet r,b} & (r,s) & \tytb{sr} \\
          \cline{3-5}
                     &            &  \tytb{\bullet r,a} & (r,c) & \tytb{rc} \\
          \hline
          r & \tytb{r,b} &  \tytb{sr,b} & (r,s) & \tytb{sr} \\
          \cline{3-5}
                     &            &  \tytb{sr,a} & (r,c) & \tytb{rc} \\
          \hline
          c & \tytb{s,a} &  \tytb{\bullet s,b} & (c,s) & \tytb{sc} \\
          \cline{3-5}
                     &            &  \tytb{\bullet s,a} & (c,c) & \tytb{cc} \\
          \hline
          d & \tytb{d,a} & \tytb{\bullet d,b} & (d,s) & \tytb{sd} \\
          \cline{3-5}
                     &            & \tytb{\bullet d,a} & (d,c) & \tytb{cd} \\
          \cline{3-5}
                     &            & \tytb{r d,a} & (d,r) & \tytb{rd} \\
          \cline{3-5}
                     &            & \tytb{d d,a} & (d,d) & \tytb{dd} \\
          \cline{2-5}
                     & \tytb{d,b} & \tytb{s d,b} & (d,s) & \tytb{sd} \\
          \cline{3-5}
                     &            & \tytb{s d,a} & (d,c) & \tytb{cd} \\
          \cline{3-5}
                     &            & \tytb{r d,b} & (d,r) & \tytb{rd} \\
          \cline{3-5}
                     &            & \tytb{d d,b} & (d,d) & \tytb{dd} \\
          \hline
          \hline
        \end{array}
      \]
      \caption{Reduction when $n=1$}
      \label{tb:rd1}
    \end{table}

    Now assume $n\geq 1$.
    We let $\bfxx = x_{1}x_{2}\cdots x_{n}$ and define
    \[
      \bfxx' = \begin{cases}
        \bfxx \text{ deleteing ``$c$''}  & \text{if } c\in \bfxx\\
        \bfxx \text{ deleteing ``$s$''}  & \text{if } c\notin \bfxx, s\in \bfxx\\
        \text{undefined} & \text{otherwise.}
      \end{cases}
    \]
    Now all the cases are listed in \Cref{tb:rd2}.
    % If $c\in \bfxx$, let $\bfxx'$ be the string obtained by deleting  $c$ from
    % $\bfxx$.
    % In the following table, if $c$ occures in $\bfxx$ or
    % $\bfxx'$, we  will view $c$ as $s$ and
    % arrange the order of entries when place them in $\bfpp_{\uptau}$.

    \begin{table}[p]
      \[
        \begin{array}{c|c|c|c|c|c:c}
          \hline
          \hline
          x_{\uptau} & \tytb{{\txx_{\uptaupp}},{m_{\uptaupp}}} & \bfpp_{\uptau} & x_{\uptaupp}, \bfxx_{\uptau}
          & \tuptau_{L} & \multicolumn{2}{ c}{\text{conditions}}\\
          \hline
          r & \tytb{r,a} &  \tytb{\bullet r,{\bfxx'},a} & \tytb{r{,}{\bfxx}} & \tytb{{\bfxx}r,}& x_{1}\neq r& c\in \bfxx  \\
          \cline{5-6}
                     & & & &\tytb{rc,{\bfxx'}} & x_{1}=r& \\
          \cline{3-3}\cline{4-7}
                     &            &  \tytb{\bullet r,{\bfxx'},b} & \tytb{r{,}{\bfxx}} & \tytb{{\bfxx}r}
                        & \multicolumn{2}{c}{c\notin \bfxx,s\in \bfxx} \\
          \hline
          r & \tytb{r,b} &  \tytb{sr,{\bfxx'},a} & \tytb{r{,}{\bfxx}} & \tytb{{\bfxx}r,}& x_{1}\neq r& c\in \bfxx  \\
          \cline{5-6}
                     & & & &\tytb{rc,{\bfxx'}} & x_{1}=r& \\
          \cline{3-3}\cline{4-7}
                     &            &  \tytb{sr,{\bfxx'},b} & \tytb{r{,}{\bfxx}} & \tytb{{\bfxx}r}
                        & \multicolumn{2}{c}{c\notin \bfxx,s\in \bfxx} \\
          \hline
          c & \tytb{s,a} &\tytb{\bullet s,{\bfxx'},a} & \tytb{c{,}{\bfxx}} & \tytb{{\bfxx}c}
                        & c\in \bfxx   & s\in \bfxx \text{ is automatic} \\%\multicolumn{2}{c}{c\in \bfxx}\\
          \cline{3-3}\cline{6-6}
                     &            &\tytb{\bullet s,{\bfxx'},b} &                    &
                        & c\notin \bfxx   &\\%\multicolumn{2}{c}{c\in \bfxx}\\
          \hline
          d & \tytb{d,a} &\tytb{\bullet d,{\bfxx'},a} & \tytb{d{,}{\bfxx}} & \tytb{{\bfxx}d}& \multicolumn{2}{c}{c\in \bfxx}\\
          \cline{3-3}\cline{6-7}
                     &            &\tytb{\bullet d,{\bfxx'},b} &                    &
                        & x_{1}=s & c\notin \bfxx   \\%\multicolumn{2}{c}{c\in \bfxx}\\
          \cline{3-3}\cline{6-7}
                     &            &\tytb{{\bfxx}d,a} &                    &
                        & x_{1}\neq s & c\notin \bfxx   \\%\multicolumn{2}{c}{c\in \bfxx}\\
          \hline
          d & \tytb{d,b} &\tytb{s d,{\bfxx'},a} & \tytb{d{,}{\bfxx}} & \tytb{{\bfxx}d}
                        & \multicolumn{2}{c}{c\in \bfxx}\\
          \cline{3-3}\cline{6-7}
                     &            &\tytb{s d,{\bfxx'},b} &                    &
                        & x_{1}=s & c\notin \bfxx   \\%\multicolumn{2}{c}{c\in \bfxx}\\
          \cline{3-3}\cline{6-7}
                     &            &\tytb{{\bfxx}d,b} &                    &
                        & x_{1}\neq s & c\notin \bfxx   \\%\multicolumn{2}{c}{c\in \bfxx}\\
          \hline
          \hline
        \end{array}
      \]
      \caption{Reduction when $n\geq 1$}
      \label{tb:rd2}
    \end{table}
  \end{proof}

  Thanks to $\bdelta$ defined in  the above lemma, we can use \eqref{eq:gd.ls}.
  The rest of the proof is similar to that in \Cref{sec:pf.gd.CD}.
  We leave the details to the reader.

% \begin{prop}
%   Inductively, we can establish the following properties:
%   \begin{enumS}
%     \item  The local system $\cL_{\uptau}$ has the factorization according to
%     $\txx_{\uptau}$.
%   \end{enumS}
% \end{prop}

\delete{
\section{An example}\label{sec:C.reg}
Consider the case that
$\cO = (2,\underbrace{1, \cdots, 1}_{(t+1)\text{-terms}})\in \Nil(D)$.
%$[2k+2,2k+1, \cdots, 2k+1, 2l, 2l,\cdots ]$.
%here we have $2k+1$-columns $2t$.

Then $\cOp$ is a regular orbit of the corresponding symplectic group. $\cOp$ is
the opposite case compared to the noticed oribts. It is very ``degenerate''.

By the calculation above, we have
\[
  \cL_{\uptaupp} = \cB_{\uptaupp} \cdot \cP_{\bfpp_{\uptaupp}}.
\]

Now the


\begin{equation} \label{eq:B1-2}
  \begin{array}{c|c|c}
   \bfpp_{\uptaup}=\bfpp_{\uptau}=x_{1}y_{1}  &   \cP_{\uptaupp} & \cP_{\uptaup}\\
    \hline
    sr & \tytb{+-+,=} & \\
    \hline
    sc & \tytb{-+-,=} & \\
    \hline
    sd & \tytb{=\ast=,=} &\\
    \hline
    rc & \tytb{\ast=\ast,+} & \tytb{=\ast=\ast}\\
    \hline
    rd & \tytb{+-+,+} & \tytb{-+-+}\\
    \hline
    cc & \tytb{=\ast=,+} &  \tytb{\ast=\ast=} \\
    \hline
    cd & \tytb{-+-,+} \cup \tytb{\ast=\ast,=} &\tytb{+-+-} \\
    \hline
    dd & \tytb{-+-,+}\cup\tytb{+-+,-} &\tytb{+-+-} \cup \tytb{-+-+}\\
  \end{array}
\end{equation}


\begin{equation} \label{eq:B1-2}
\begin{array}{cc|c|cc}
  \bfxx_{\uptaupp} & \cP_{\uptaupp}         & \cP_{\uptaup} & \bfxx_{\uptau} & \cP_{\uptaupp} \\
  \hline
  src              & \tytb{\ast\cdots ,=} &                                                 \\
  \hline
  srd              & \tytb{+\cdots,=}          &                                                 \\
  \hline
  scc              & \tytb{-\cdots,=}          &                                                 \\
  \hline
  scd              & \tytb{=\cdots,=} &                                                 \\
  sdd              & %\tytb{=\ast=\ast=,=}
                                            &                                                 \\
  \hline
  rcc              & \tytb{\ast\cdots,+} & \tytb{-\cdots}                                                \\
  \hline
  rcd              & \tytb{+\cdots,+} & \tytb{=\cdots}                             \\
  rdd              & %\tytb{+-+-+,+} & \tytb{=\ast=\ast=\ast}                             \\
                                & \\
  \hline
  ccc              & \tytb{-\cdots,+}          & \tytb{+\cdots}                                               \\
  \hline
  ccd              & \tytb{=\cdots,+}          & \tytb{\ast\cdots}                                             \\
  cdd              & \tytb{=\cdots,+}\cup \tytb{+\cdots,=} & \\
  \hline
  ddd              & \tytb{-\cdots,+}\cup \tytb{+\cdots,-}  &
                                                            \tytb{+\cdots} \cup\tytb{-\cdots}                     \\
\end{array}
\end{equation}

When $t$ is even.
In the following
%$\cA_{8}\cap\cA_{9}=\emptyset$.
$\set{\cA_{1},\cA_{2}, \cA_{3},\cA_{4}}$ are disjoint, $\set{\cA_{5},\cA_{6}}$ are disjoint,
and $\set{\cA_{7},\cA_{8},\cA_{9}}$ % and $\set{\cA_{9},\cA_{10}}$
are disjoint.
$\cA_{9}=\set{d\cdots d}$
\begin{equation} \label{eq:B1-2}
\begin{array}{cc|c|cc}
  \bfxx_{\uptaupp} & \cP_{\uptaupp}         & \cP_{\uptaup} & \bfxx_{\uptau} & \cP_{\uptau} \\
  \hline
  s\cA_{1}              & \tytb{\ast\cdots ,=} &                                                 \\
  \hline
  s\cA_{2}              & \tytb{+\cdots,=}          &                                                 \\
  \hline
  s\cA_{3}              & \tytb{-\cdots,=}          &                                                 \\
  \hline
  s\cA_{4}              & \tytb{=\cdots,=} &                                                 \\
  \hline
  r\cA_{5}              & \tytb{\ast\cdots,+} & \tytb{-\cdots}  & sr\cA_{5} & \tytb{\ast\cdots, =}                            \\
                        &                    &                & rc\cA_{5} &\tytb{+\cdots,+}                                   \\
  \hline
  r\cA_{6}              & \tytb{+\cdots,+} & \tytb{=\cdots}    & sr\cA_{6} &  \tytb{+\cdots,=}                        \\
                        &                    &                & rc\cA_{6} &\tytb{\ast\cdots,+}                                   \\
  \hline
  c\cA_{7}              & \tytb{-\cdots,+}          & \tytb{+\cdots} & sc\cA_{7} & \tytb{=\cdots,=}                                        \\
                        &                    &                & cc\cA_{7} &\tytb{-\cdots,+}                                   \\
  \hline
  c\cA_{8}              & \tytb{=\cdots,+}          & \tytb{\ast\cdots} & sc(\cA_{8}\cup \cA_{9}) & \tytb{-\cdots,=}                                   \\
   c\cA_{9}           &   \tytb{=\cdots,+}\cup \tytb{+\cdots,=}     &                & cc(\cA_{8}\cup \cA_{9}) &\tytb{=\cdots,+}                                   \\
  \hline
  d\cA_{9}              & \tytb{-\cdots,+}\cup \tytb{+\cdots,-}  &
                                        \tytb{+\cdots} \cup\tytb{-\cdots}    & sd\cA_{9} & \tytb{=\cdots,=}                 \\
                        &                    &                & rd\cA_{9} &\tytb{+\cdots,+}                                   \\
                        &                    &                & cd\cA_{9} &\tytb{-\cdots,+}  \cup \tytb{\ast\cdots,=}                \\
                        &                    &                & dd\cA_{9} &\tytb{-\cdots,+} \cup \tytb{+\cdots,-}                       \\
\end{array}
\end{equation}

We observe that there are two coincidences:
\begin{enumT}
  \item $\tytb{+\cdots,+}$: clearly $rc\cA_{5}$ and $rd\cA_{9}$ are disjoint.
  \item $\tytb{=\cdots,=}$: clearly $sc\cA_{7}$ and $sd\cA_{9}$ are also disjoint.
\end{enumT}


When $t$ is odd.
In the following
%$\cA_{8}\cap\cA_{9}=\emptyset$.
$\set{\cA_{1},\cA_{2}, \cA_{3},\cA_{4}}$ are disjoint, $\set{\cA_{5},\cA_{6}}$ are disjoint,
and $\set{\cA_{7},\cA_{8},\cA_{9}}$ % and $\set{\cA_{9},\cA_{10}}$
are disjoint.
$\cA_{9}=\set{d\cdots d}$
\begin{equation} \label{eq:B1-2}
\begin{array}{cc|c|cc}
  \bfxx_{\uptaupp} & \cP_{\uptaupp}         & \cP_{\uptaup} & \bfxx_{\uptau} & \cP_{\uptau} \\
  \hline
  s\cA_{1}              & \tytb{\ast\cdots ,=} &                                      \\
  \hline
  s\cA_{2}              & \tytb{+\cdots,=}          &                                     \\
  \hline
  s\cA_{3}              & \tytb{-\cdots,=}          &                                 \\
  \hline
  s\cA_{4}              & \tytb{=\cdots,=} &                                      \\
  \hline
  r\cA_{5}              & \tytb{\ast\cdots,+} & \tytb{=\cdots}  & sr\cA_{5} & \tytb{\ast\cdots,=}                      \\
                        &                    &                & rc\cA_{5} &\tytb{\ast\cdots,+}                         \\
  \hline
  r\cA_{6}              & \tytb{+\cdots,+} & \tytb{-\cdots}    & sr\cA_{6} &  \tytb{+\cdots,=}                    \\
                        &                    &                & rc\cA_{6} &\tytb{+\cdots,+}                          \\
  \hline
  c\cA_{7}              & \tytb{=\cdots,+}          & \tytb{\ast\cdots} & sc\cA_{7} & \tytb{-\cdots,=}                                        \\
                        &                    &                & cc\cA_{6} &\tytb{-\cdots,+}                                \\
  \hline
  c\cA_{8}              & \tytb{-\cdots,+}          & \tytb{+\cdots} & sc(\cA_{8}\cup \cA_{9}) &\tytb{=\cdots,=}             \\
                        &                    &                & cc(\cA_{8}\cup \cA_{9}) &\tytb{=\cdots,+}                                   \\
  c\cA_{9}              & \tytb{-\cdots,+}\cup \tytb{\ast\cdots,=} & \\
  \hline
  d\cA_{9}              & \tytb{-\cdots,+}\cup \tytb{+\cdots,-}  &
                                                            \tytb{+\cdots} \cup\tytb{-\cdots}    & sd\cA_{9} & \tytb{=\cdots,=}                 \\
                        &                    &                & rd\cA_{9} &\tytb{+\cdots,+}                                   \\
                        &                    &                & cd\cA_{9} &\tytb{=\cdots,+}  \cup \tytb{+\cdots,=}                \\
                        &                    &                & dd\cA_{9} &\tytb{-\cdots,+} \cup \tytb{+\cdots,-}                       \\
\end{array}
\end{equation}
}

\appendix

\begin{bibdiv}
  \begin{biblist}
% \bib{AB}{article}{
%   title={Genuine representations of the metaplectic group},
%   author={Adams, Jeffrey},
%   author = {Barbasch, Dan},
%   journal={Compositio Mathematica},
%   volume={113},
%   number={01},
%   pages={23--66},
%   year={1998},
% }

\bib{Ad83}{article}{
  author = {Adams, J.},
  title = {Discrete spectrum of the reductive dual pair $(O(p,q),Sp(2m))$ },
  journal = {Invent. Math.},
  number = {3},
 pages = {449--475},
 volume = {74},
 year = {1983}
}

%\bib{Ad07}{article}{
%  author = {Adams, J.},
%  title = {The theta correspondence over R},
%  journal = {Harmonic analysis, group representations, automorphic forms and invariant theory,  Lect. Notes Ser. Inst. Math. Sci. Natl. Univ. Singap., 12},
% pages = {1--39},
% year = {2007}
% publisher={World Sci. Publ.}
%}


\bib{ABV}{book}{
  title={The Langlands classification and irreducible characters for real reductive groups},
  author={Adams, J.},
  author={Barbasch, B.},
  author={Vogan, D. A.},
  series={Progress in Math.},
  volume={104},
  year={1991},
  publisher={Birkhauser}
}

\bib{AC}{article}{
  title={Algorithms for representation theory of
    real reductive groups},
  volume={8},
  DOI={10.1017/S1474748008000352},
  number={2},
  journal={Journal of the Institute of Mathematics of Jussieu},
  publisher={Cambridge University Press},
  author={Adams, Jeffrey}
  author={du Cloux,
    Fokko},
  year={2009},
  pages={209-259}
}

\bib{ArPro}{article}{
  author = {Arthur, J.},
  title = {On some problems suggested by the trace formula},
  journal = {Lie group representations, II (College Park, Md.), Lecture Notes in Math. 1041},
 pages = {1--49},
 year = {1984}
}


\bib{ArUni}{article}{
  author = {Arthur, J.},
  title = {Unipotent automorphic representations: conjectures},
  %booktitle = {Orbites unipotentes et repr\'esentations, II},
  journal = {Orbites unipotentes et repr\'esentations, II, Ast\'erisque},
 pages = {13--71},
 volume = {171-172},
 year = {1989}
}

\bib{AK}{article}{
  author = {Auslander, L.},
  author = {Kostant, B.},
  title = {Polarizations and unitary representations of solvable Lie groups},
  journal = {Invent. Math.},
 pages = {255--354},
 volume = {14},
 year = {1971}
}

\bib{B.Uni}{article}{
  author = {Barbasch, D.},
  title = {Unipotent representations for real reductive groups},
 %booktitle = {Proceedings of ICM, Kyoto 1990},
 journal = {Proceedings of ICM (1990), Kyoto},
   % series = {Proc. Sympos. Pure Math.},
 %   volume = {68},
     pages = {769--777},
 publisher = {Springer-Verlag, The Mathematical Society of Japan},
      year = {2000},
}

\bib{B.W}{article}{
  author={Barbasch, Dan},
  author={Vogan, David},
  editor={Trombi, P. C.},
  title={Weyl Group Representations and Nilpotent Orbits},
  bookTitle={Representation Theory of Reductive Groups:
    Proceedings of the University of Utah Conference 1982},
  year={1983},
  publisher={Birkh{\"a}user Boston},
  address={Boston, MA},
  pages={21--33},
  %doi={10.1007/978-1-4684-6730-7_2},
}



\bib{B.Orbit}{article}{
  author = {Barbasch, D.},
  title = {Orbital integrals of nilpotent orbits},
 %booktitle = {The mathematical legacy of {H}arish-{C}handra ({B}altimore,{MD}, 1998)},
    journal = {The mathematical legacy of {H}arish-{C}handra, Proc. Sympos. Pure Math.},
    %series={The mathematical legacy of {H}arish-{C}handra, Proc. Sympos. Pure Math},
    volume = {68},
     pages = {97--110},
 publisher = {Amer. Math. Soc., Providence, RI},
      year = {2000},
}



\bib{B10}{article}{
  author = {Barbasch, D.},
  title = {The unitary spherical spectrum for split classical groups},
  journal = {J. Inst. Math. Jussieu},
% number = {9},
 pages = {265--356},
 volume = {9},
 year = {2010}
}



\bib{B17}{article}{
  author = {Barbasch, D.},
  title = {Unipotent representations and the dual pair correspondence},
  journal = {J. Cogdell et al. (eds.), Representation Theory, Number Theory, and Invariant Theory, In Honor of Roger Howe. Progress in Math.}
  %series ={Progress in Math.},
  volume = {323},
  pages = {47--85},
  year = {2017},
}

\bib{BVUni}{article}{
 author = {Barbasch, D.},
 author = {Vogan, D. A.},
 journal = {Annals of Math.},
 number = {1},
 pages = {41--110},
 title = {Unipotent representations of complex semisimple groups},
 volume = {121},
 year = {1985}
}

\bib{Br}{article}{
  author = {Brylinski, R.},
  title = {Dixmier algebras for classical complex nilpotent orbits via Kraft-Procesi models. I},
  journal = {The orbit method in geometry and physics (Marseille, 2000). Progress in Math.}
  volume = {213},
  pages = {49--67},
  year = {2003},
}

\bib{Carter}{book}{
   author={Carter, Roger W.},
   title={Finite groups of Lie type},
   series={Wiley Classics Library},
   %note={Conjugacy classes and complex characters;
   %Reprint of the 1985 original;
   %A Wiley-Interscience Publication},
   publisher={John Wiley \& Sons, Ltd., Chichester},
   date={1993},
   pages={xii+544},
   isbn={0-471-94109-3},
   %review={\MR{1266626}},
}
\bib{Ca89}{article}{
 author = {Casselman, W.},
 journal = {Canad. J. Math.},
 pages = {385--438},
 title = {Canonical extensions of Harish-Chandra modules to representations of $G$},
 volume = {41},
 year = {1989}
}



\bib{Cl}{article}{
  author = {Du Cloux, F.},
  journal = {Ann. Sci. \'Ecole Norm. Sup.},
  number = {3},
  pages = {257--318},
  title = {Sur les repr\'esentations diff\'erentiables des groupes de Lie alg\'ebriques},
  url = {http://eudml.org/doc/82297},
  volume = {24},
  year = {1991},
}

\bib{CM}{book}{
  title = {Nilpotent orbits in semisimple Lie algebra: an introduction},
  author = {Collingwood, D. H.},
  author = {McGovern, W. M.},
  year = {1993}
  publisher = {Van Nostrand Reinhold Co.},
}


% \bib{Dieu}{book}{
%    title={La g\'{e}om\'{e}trie des groupes classiques},
%    author={Dieudonn\'{e}, Jean},
%    year={1963},
% 	publisher={Springer},
%  }

\bib{DKPC}{article}{
title = {Nilpotent orbits and complex dual pairs},
journal = {J. Algebra},
volume = {190},
number = {2},
pages = {518 - 539},
year = {1997},
author = {Daszkiewicz, A.},
author = {Kra\'skiewicz, W.},
author = {Przebinda, T.},
}

\bib{DKP2}{article}{
  author = {Daszkiewicz, A.},
  author = {Kra\'skiewicz, W.},
  author = {Przebinda, T.},
  title = {Dual pairs and Kostant-Sekiguchi correspondence. II. Classification
	of nilpotent elements},
  journal = {Central European J. Math.},
  year = {2005},
  volume = {3},
  pages = {430--474},
}


\bib{DM}{article}{
  author = {Dixmier, J.},
  author = {Malliavin, P.},
  title = {Factorisations de fonctions et de vecteurs ind\'efiniment diff\'erentiables},
  journal = {Bull. Sci. Math. (2)},
  year = {1978},
  volume = {102},
  pages = {307--330},
}

%\bibitem[DM]{DM}
%J. Dixmier and P. Malliavin, \textit{Factorisations de fonctions et de vecteurs ind\'efiniment diff\'erentiables}, Bull. Sci. Math. (2), 102 (4),  307-330 (1978).



%\bib{Du77}{article}{
% author = {Duflo, M.},
% journal = {Annals of Math.},
% number = {1},
% pages = {107-120},
% title = {Sur la Classification des Ideaux Primitifs Dans
%   L'algebre Enveloppante d'une Algebre de Lie Semi-Simple},
% volume = {105},
% year = {1977}
%}

\bib{Du82}{article}{
 author = {Duflo, M.},
 journal = {Acta Math.},
  volume = {149},
 number = {3-4},
 pages = {153--213},
 title = {Th\'eorie de Mackey pour les groupes de Lie alg\'ebriques},
 year = {1982}
}



\bib{GZ}{article}{
author={Gomez, R.},
author={Zhu, C.-B.},
title={Local theta lifting of generalized Whittaker models associated to nilpotent orbits},
journal={Geom. Funct. Anal.},
year={2014},
volume={24},
number={3},
pages={796--853},
}

\bib{EGAIV2}{article}{
  title = {\'El\'ements de g\'eom\'etrie alg\'brique IV: \'Etude locale des
    sch\'emas et des morphismes de sch\'emas. II},
  author = {Grothendieck, A.},
  author = {Dieudonn\'e, J.},
  journal  = {Inst. Hautes \'Etudes Sci. Publ. Math.},
  volume = {24},
  year = {1965},
}


\bib{EGAIV3}{article}{
  title = {\'El\'ements de g\'eom\'etrie alg\'brique IV: \'Etude locale des
    sch\'emas et des morphismes de sch\'emas. III},
  author = {Grothendieck, A.},
  author = {Dieudonn\'e, J.},
  journal  = {Inst. Hautes \'Etudes Sci. Publ. Math.},
  volume = {28},
  year = {1966},
}


\bib{HLS}{article}{
    author = {Harris, M.},
    author = {Li, J.-S.},
    author = {Sun, B.},
     title = {Theta correspondences for close unitary groups},
 %booktitle = {Arithmetic Geometry and Automorphic Forms},
    %series = {Adv. Lect. Math. (ALM)},
    journal = {Arithmetic Geometry and Automorphic Forms, Adv. Lect. Math. (ALM)},
    volume = {19},
     pages = {265--307},
 publisher = {Int. Press, Somerville, MA},
      year = {2011},
}

\bib{HS}{book}{
 author = {Hartshorne, R.},
 title = {Algebraic Geometry},
publisher={Graduate Texts in Mathematics, 52. New York-Heidelberg-Berlin: Springer-Verlag},
year={1983},
}

\bib{He}{article}{
author={He, H.},
title={Unipotent representations and quantum induction},
journal={arXiv:math/0210372},
year = {2002},
}

\bib{HL}{article}{
author={Huang, J.-S.},
author={Li, J.-S.},
title={Unipotent representations attached to spherical nilpotent orbits},
journal={Amer. J. Math.},
volume={121},
number = {3},
pages={497--517},
year={1999},
}


\bib{HZ}{article}{
author={Huang, J.-S.},
author={Zhu, C.-B.},
title={On certain small representations of indefinite orthogonal groups},
journal={Represent. Theory},
volume={1},
pages={190--206},
year={1997},
}



\bib{Howe79}{article}{
  title={$\theta$-series and invariant theory},
  author={Howe, R.},
  book = {
    title={Automorphic Forms, Representations and $L$-functions},
    series={Proc. Sympos. Pure Math},
    volume={33},
    year={1979},
  },
  pages={275-285},
}

\bib{HoweRank}{article}{
author={Howe, R.},
title={On a notion of rank for unitary representations of the classical groups},
journal={Harmonic analysis and group representations, Liguori, Naples},
pages={223-331},
year={1982},
}

\bib{Howe89}{article}{
author={Howe, R.},
title={Transcending classical invariant theory},
journal={J. Amer. Math. Soc.},
volume={2},
pages={535--552},
year={1989},
}

\bib{Howe95}{article}{,
  author = {Howe, R.},
  title = {Perspectives on invariant theory: Schur duality, multiplicity-free actions and beyond},
  journal = {Piatetski-Shapiro, I. et al. (eds.), The Schur lectures (1992). Ramat-Gan: Bar-Ilan University, Isr. Math. Conf. Proc. 8,},
  year = {1995},
  pages = {1-182},
}

\bib{JLS}{article}{
author={Jiang, D.},
author={Liu, B.},
author={Savin, G.},
title={Raising nilpotent orbits in wave-front sets},
journal={Represent. Theory},
volume={20},
pages={419--450},
year={2016},
}

\bib{Ki62}{article}{
author={Kirillov, A. A.},
title={Unitary representations of nilpotent Lie groups},
journal={Uspehi Mat. Nauk},
volume={17},
issue ={4},
pages={57--110},
year={1962},
}


\bib{Ko70}{article}{
author={Kostant, B.},
title={Quantization and unitary representations},
journal={Lectures in Modern Analysis and Applications III, Lecture Notes in Math.},
volume={170},
pages={87--208},
year={1970},
}


\bib{KP}{article}{
author={Kraft, H.},
author={Procesi, C.},
title={On the geometry of conjugacy classes in classical groups},
journal={Comment. Math. Helv.},
volume={57},
pages={539--602},
year={1982},
}

\bib{KR}{article}{
author={Kudla, S. S.},
author={Rallis, S.},
title={Degenerate principal series and invariant distributions},
journal={Israel J. Math.},
volume={69},
pages={25--45},
year={1990},
}


\bib{Ku}{article}{
author={Kudla, S. S.},
title={Some extensions of the Siegel-Weil formula},
journal={In: Gan W., Kudla S., Tschinkel Y. (eds) Eisenstein Series and Applications. Progress in Mathematics, vol 258. Birkh\"auser Boston},
%volume={69},
pages={205--237},
year={2008},
}





\bib{LZ1}{article}{
author={Lee, S. T.},
author={Zhu, C.-B.},
title={Degenerate principal series and local theta correspondence II},
journal={Israel J. Math.},
volume={100},
pages={29--59},
year={1997},
}

\bib{LZ2}{article}{
author={Lee, S. T.},
author={Zhu, C.-B.},
title={Degenerate principal series of metaplectic groups and Howe correspondence},
journal = {D. Prasad at al. (eds.), Automorphic Representations and L-Functions, Tata Institute of Fundamental Research, India,},
year = {2013},
pages = {379--408},
}

\bib{Li89}{article}{
author={Li, J.-S.},
title={Singular unitary representations of classical groups},
journal={Invent. Math.},
volume={97},
number = {2},
pages={237--255},
year={1989},
}

\bib{LiuAG}{book}{
  title={Algebraic Geometry and Arithmetic Curves},
  author = {Liu, Q.},
  year = {2006},
  publisher={Oxford University Press},
}

\bib{LM}{article}{
   author = {Loke, H. Y.},
   author = {Ma, J.},
    title = {Invariants and $K$-spectrums of local theta lifts},
    journal = {Compositio Math.},
    volume = {151},
    issue = {01},
    year = {2015},
    pages ={179--206},
}

\bib{LS}{article}{
   author = {Lusztig, G.},
   author = {Spaltenstein, N.},
    title = {Induced unipotent classes},
    journal = {j. London Math. Soc.},
    volume = {19},
    year = {1979},
    pages ={41--52},
}

\bib{Lu.I}{article}{
   author={Lusztig, G.},
   title={Intersection cohomology complexes on a reductive group},
   journal={Invent. Math.},
   volume={75},
   date={1984},
   number={2},
   pages={205--272},
   issn={0020-9910},
   review={\MR{732546}},
   doi={10.1007/BF01388564},
}
	

\bib{Ma}{article}{
   author = {Mackey, G. W.},
    title = {Unitary representations of group extentions},
    journal = {Acta Math.},
    volume = {99},
    year = {1958},
    pages ={265--311},
}


\bib{Mc}{article}{
   author = {McGovern, W. M},
    title = {Cells of Harish-Chandra modules for real classical groups},
    journal = {Amer. J.  of Math.},
    volume = {120},
    issue = {01},
    year = {1998},
    pages ={211--228},
}

\bib{Mo96}{article}{
 author={M{\oe}glin, C.},
    title = {Front d'onde des repr\'esentations des groupes classiques $p$-adiques},
    journal = {Amer. J. Math.},
    volume = {118},
    issue = {06},
    year = {1996},
    pages ={1313--1346},
}

\bib{Mo17}{article}{
  author={M{\oe}glin, C.},
  title = {Paquets d'Arthur Sp\'eciaux Unipotents aux Places Archim\'ediennes et Correspondance de Howe},
  journal = {J. Cogdell et al. (eds.), Representation Theory, Number Theory, and Invariant Theory, In Honor of Roger Howe. Progress in Math.}
  %series ={Progress in Math.},
  volume = {323},
  pages = {469--502}
  year = {2017}
}


\bib{MVW}{book}{
  volume={1291},
  title={Correspondances de Howe sur un corps $p$-adique},
  author={M{\oe}glin, C.},
  author={Vign\'eras, M.-F.},
  author={Waldspurger, J.-L.},
  series={Lecture Notes in Mathematics},
  publisher={Springer}
  ISBN={978-3-540-18699-1},
  date={1987},
}

\bib{NOTYK}{article}{
   author = {Nishiyama, K.},
   author = {Ochiai, H.},
   author = {Taniguchi, K.},
   author = {Yamashita, H.},
   author = {Kato, S.},
    title = {Nilpotent orbits, associated cycles and Whittaker models for highest weight representations},
    journal = {Ast\'erisque},
    volume = {273},
    year = {2001},
   pages ={1--163},
}

\bib{NOZ}{article}{
  author = {Nishiyama, K.},
  author = {Ochiai, H.},
  author = {Zhu, C.-B.},
  journal = {Trans. Amer. Math. Soc.},
  title = {Theta lifting of nilpotent orbits for symmetric pairs},
  volume = {358},
  year = {2006},
  pages = {2713--2734},
}


\bib{NZ}{article}{
   author = {Nishiyama, K.},
   author = {Zhu, C.-B.},
    title = {Theta lifting of unitary lowest weight modules and their associated cycles},
    journal = {Duke Math. J.},
    volume = {125},
    issue = {03},
    year = {2004},
   pages ={415--465},
}



\bib{Ohta}{article}{
  author = {Ohta, T.},
  %doi = {10.2748/tmj/1178227492},
  journal = {Tohoku Math. J.},
  number = {2},
  pages = {161--211},
  publisher = {Tohoku University, Mathematical Institute},
  title = {The closures of nilpotent orbits in the classical symmetric
    pairs and their singularities},
  volume = {43},
  year = {1991}
}

\bib{Ohta2}{article}{
  author = {Ohta, T.},
  journal = {Hiroshima Math. J.},
  number = {2},
  pages = {347--360},
  title = {Induction of nilpotent orbits for real reductive groups and associated varieties of standard representations},
  volume = {29},
  year = {1999}
}

\bib{Ohta4}{article}{
  title={Nilpotent orbits of $\mathbb{Z}_4$-graded Lie algebra and geometry of
    moment maps associated to the dual pair $(\mathrm{U} (p, q), \mathrm{U} (r, s))$},
  author={Ohta, T.},
  journal={Publ. RIMS},
  volume={41},
  number={3},
  pages={723--756},
  year={2005}
}

\bib{PT}{article}{
  title={Some small unipotent representations of indefinite orthogonal groups and the theta correspondence},
  author={Paul, A.},
  author={Trapa, P.},
  journal={University of Aarhus Publ. Series},
  volume={48},
  pages={103--125},
  year={2007}
}


\bib{PV}{article}{
  title={Invariant Theory},
  author={Popov, V. L.},
  author={Vinberg, E. B.},
  book={
  title={Algebraic Geometry IV: Linear Algebraic Groups, Invariant Theory},
  series={Encyclopedia of Mathematical Sciences},
  volume={55},
  year={1994},
  publisher={Springer},}
}




%\bib{PPz}{article}{
%author={Protsak, V.} ,
%author={Przebinda, T.},
%title={On the occurrence of admissible representations in the real Howe
%    correspondence in stable range},
%journal={Manuscr. Math.},
%volume={126},
%number={2},
%pages={135--141},
%year={2008}
%}


\bib{PrzInf}{article}{
      author={Przebinda, T.},
       title={The duality correspondence of infinitesimal characters},
        date={1996},
     journal={Colloq. Math.},
      volume={70},
       pages={93--102},
}


\bib{Pz}{article}{
author={Przebinda, T.},
title={Characters, dual pairs, and unitary representations},
journal={Duke Math. J. },
volume={69},
number={3},
pages={547--592},
year={1993}
}

\bib{Ra}{article}{
author={Rallis, S.},
title={On the Howe duality conjecture},
journal={Compositio Math.},
volume={51},
pages={333--399},
year={1984}
}

\bib{Sa}{article}{
author={Sahi, S.},
title={Explicit Hilbert spaces for certain unipotent representations},
journal={Invent. Math.},
volume={110},
number = {2},
pages={409--418},
year={1992}
}

\bib{Se}{article}{
author={Sekiguchi, J.},
title={Remarks on real nilpotent orbits of a symmetric pair},
journal={J. Math. Soc. Japan},
%publisher={The Mathematical Society of Japan},
year={1987},
volume={39},
number={1},
pages={127--138},
}

\bib{SV}{article}{
  author = {Schmid, W.},
  author = {Vilonen, K.},
  journal = {Annals of Math.},
  number = {3},
  pages = {1071--1118},
  %publisher = {Princeton University, Mathematics Department, Princeton, NJ; Mathematical Sciences Publishers, Berkeley},
  title = {Characteristic cycles and wave front cycles of representations of reductive Lie groups},
  volume = {151},
year = {2000},
}

\bib{So}{article}{
author = {Sommers, E.},
title = {Lusztig's canonical quotient and generalized duality},
journal = {J. Algebra},
volume = {243},
number = {2},
pages = {790--812},
year = {2001},
}

\bib{SS}{book}{
  author = {Springer, T. A.},
  author = {Steinberg, R.},
  title = {Seminar on algebraic groups and related finite groups; Conjugate classes},
  series = {Lecture Notes in Math.}
  volume = {131}
publisher={Springer},
year={1970},
}

\bib{SZ1}{article}{
title={A general form of Gelfand-Kazhdan criterion},
author={Sun, B.},
author={Zhu, C.-B.},
journal={Manuscripta Math.},
pages = {185--197},
volume = {136},
year={2011}
}


%\bib{SZ2}{article}{
%  title={Conservation relations for local theta correspondence},
%  author={Sun, B.},
%  author={Zhu, C.-B.},
%  journal={J. Amer. Math. Soc.},
%  pages = {939--983},
%  volume = {28},
%  year={2015}
%}



\bib{Tr}{article}{
  title={Special unipotent representations and the Howe correspondence},
  author={Trapa, P.},
  year = {2004},
  journal={University of Aarhus Publication Series},
  volume = {47},
  pages= {210--230}
}

% \bib{Wa}{article}{
%    author = {Waldspurger, J.-L.},
%     title = {D\'{e}monstration d'une conjecture de dualit\'{e} de Howe dans le cas $p$-adique, $p \neq 2$ in Festschrift in honor of I. I. Piatetski-Shapiro on the occasion of his sixtieth birthday},
%   journal = {Israel Math. Conf. Proc., 2, Weizmann, Jerusalem},
%  year = {1990},
% pages = {267-324},
% }

\bib{V4}{article}{
   author={Vogan, D. A. },
   title={Irreducible characters of semisimple Lie groups. IV.
   Character-multiplicity duality},
   journal={Duke Math. J.},
   volume={49},
   date={1982},
   number={4},
   pages={943--1073},
   issn={0012-7094},
   review={\MR{683010}},
}
\bib{VoBook}{book}{
author = {Vogan, D. A. },
  title={Unitary representations of reductive Lie groups},
  year={1987},
  series = {Ann. of Math. Stud.},
 volume={118},
  publisher={Princeton University Press}
}


\bib{Vo89}{article}{
  author = {Vogan, D. A. },
  title = {Associated varieties and unipotent representations},
 %booktitle ={Harmonic analysis on reductive groups, Proc. Conf., Brunswick/ME (USA) 1989,},
  journal = {Harmonic analysis on reductive groups, Proc. Conf., Brunswick/ME
    (USA) 1989, Prog. Math.},
 volume={101},
  publisher = {Birkh\"{a}user, Boston-Basel-Berlin},
  year = {1991},
pages={315--388},
  editor = {W. Barker and P. Sally},
}

\bib{Vo98}{article}{
  author = {Vogan, D. A. },
  title = {The method of coadjoint orbits for real reductive groups},
 %booktitle ={Representation theory of Lie groups (Park City, UT, 1998)},
 journal = {Representation theory of Lie groups (Park City, UT, 1998). IAS/Park City Math. Ser.},
  volume={8},
  publisher = {Amer. Math. Soc.},
  year = {2000},
pages={179--238},
}

\bib{Vo00}{article}{
  author = {Vogan, D. A. },
  title = {Unitary representations of reductive Lie groups},
 %booktitle ={Mathematics towards the Third Millennium (Rome, 1999)},
 journal ={Mathematics towards the Third Millennium (Rome, 1999). Accademia Nazionale dei Lincei, (2000)},
  %series = {Accademia Nazionale dei Lincei, 2000},
 %volume={9},
pages={147--167},
}


\bib{Wa1}{book}{
  title={Real reductive groups I},
  author={Wallach, N. R.},
  year={1988},
  publisher={Academic Press Inc. }
}

\bib{Wa2}{book}{
  title={Real reductive groups II},
  author={Wallach, N. R.},
  year={1992},
  publisher={Academic Press Inc. }
}


\bib{Weyl}{book}{
  title={The classical groups: their invariants and representations},
  author={Weyl, H.},
  year={1947},
  publisher={Princeton University Press}
}

\bib{Ya}{article}{
  title={Degenerate principal series representations for quaternionic unitary groups},
  author={Yamana, S.},
  year = {2011},
  journal={Israel J. Math.},
  volume = {185},
  pages= {77--124}
}



% \bib{EGAIV4}{article}{
%   title = {\'El\'ements de g\'eom\'etrie alg\'brique IV 4: \'Etude locale des
%     sch\'emas et des morphismes de sch\'emas},
%   author = {Grothendieck, Alexandre},
%   author = {Dieudonn\'e, Jean},
%   journal  = {Inst. Hautes \'Etudes Sci. Publ. Math.},
%   volume = {32},
%   year = {1967},
%   pages = {5--361}
% }



\end{biblist}
\end{bibdiv}


\printindex


\end{document}


%%% Local Variables:
%%% coding: utf-8
%%% mode: latex
%%% TeX-engine: xetex
%%% ispell-local-dictionary: "en_US"
%%% End:
