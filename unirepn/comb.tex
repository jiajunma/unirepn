\documentclass[ssunip]{subfiles}

\begin{document}
 \section{The descents of painted bipartitions}\label{sec:comb}

As before, let  $\star\in \{ B, C,  D, \widetilde{C},  C^*, D^*\}$ and let $\check \CO$ be a Young diagram that has $\star$-good parity. Put
\begin{equation}\label{lstarco}
  l:=l_{\star, \check \CO}:=\begin{cases}
 \frac{\bfrr_1(\ckcO)}{2}; & \quad \textrm{if } \star\in \{B, \widetilde C\};\smallskip\\
 \frac{\bfrr_1(\ckcO)-1}{2}, &\quad \textrm{if } \star\in \{C, C^* \};\smallskip\\
 \frac{\bfrr_1(\ckcO)+1}{2}, &\quad \textrm{if } \star\in \{ D, D^*\}.\\
\end{cases}
\end{equation}
This is the length of the leading column of every element of $\mathrm{PBP}_\star(\check \CO)$. 

 
 In various context, we use $\emptyset$ to denote the empty set, the empty Young diagram or the painted Young diagram whose underlying Young diagram is empty. For every Young diagram $\imath$, its descent, which is denoted by $\nabla(\jmath)$, is defined to be the Young diagram obtained from $\jmath$ by removing the first column. By convention, $\nabla(\emptyset)=\emptyset$. 
 
 In the rest of this section, we assume that $\check \CO\neq \emptyset$, and write $\check \CO'$ for its dual descent. Write $\star'$ for the Howe dual of $\star$ so that $\check \CO'$ has $\star'$-good parity. Put
\[
l':=l_{\star', \check \CO'}
\]
    
 \subsection{Naive descents of painted bipartitions }
\def\bipartl{\mathrm{bi\cP_L}}
\def\bipartr{\mathrm{bi\cP_R}}
\def\dsdiagl{\mathrm{DS_L}}
\def\dsdiagr{\mathrm{DS_R}}
\def\DDl{\eDD_\mathrm{L}}
\def\DDr{\eDD_\mathrm{R}}


In this subsection, let $\tau=(\imath,\cP)\times (\jmath,\cQ)\times \alpha$ be a  painted bipartition such that $\star_\tau=\star$. Write $\star'$ for the Howe dual of $\star$ and put
\begin{equation} \label{eq:def.alphap}
\alpha'=\begin{cases} B^+,
& \textrm{if $\alpha = \wtC$ and $\cP_\tau(l_{\star,\ckcO},1),1) \neq c$;}\\
B^-,
& \textrm{if $\alpha = \wtC$ and $\cP_\tau(l_{\star,\ckcO},1),1)  = c$;}\\
\star', & \textrm{if $\alpha\neq \widetilde C$}. 
\end{cases}
\end{equation}
\trivial{
  \begin{equation} \label{eq:def.alphap}
    \alpha'=\begin{cases} B^+,
  & \textrm{if $\alpha=\widetilde{C}$ and $c$ does not occur in the leading column of $\tau$}; \smallskip \\
  B^-,
  & \textrm{if $\alpha=\widetilde{C}$ and  $c$ occurs in the leading column of $\tau$}; \smallskip \\
  \star', & \textrm{if $\alpha\neq \widetilde C$}. 
  \end{cases}
  \end{equation}
  }
\begin{lem}\label{lemDDn1}
  If $\star \in \set{B,C,C^*}$, then there is a unique painted bipartition of the form $\tau'= (\imath',\cP')\times (\jmath',\cQ')\times \alpha'$ with the following properties:
  \begin{itemize}
        \item $
   (\imath',\jmath')= (\imath,\DD(\jmath)); \smallskip
   $
   \item for all $(i,j)\in \BOX(\imath')$,
   \[
     \cP'(i,j)=\begin{cases}   
    \bullet \textrm{ or } s,&\textrm{ if  $\ \cP(i,j)\in \{\bullet, s\}$;} \smallskip \\
  \cP(i,j),& \textrm{ if $\ \cP(i,j)\notin \{\bullet, s\}$};\end{cases}
   \]
   \item for all $(i,j)\in \BOX(\jmath')$,
   \[
     \cQ'(i,j)=\begin{cases}   
    \bullet \textrm{ or } s,&\textrm{ if  $\ \cQ(i,j+1)\in \{\bullet, s\}$;} \smallskip \\
  \cQ(i,j+1), & \textrm{ if $\ \cQ(i,j+1)\notin \{\bullet, s\}$}.  \end{cases}
   \]
    \end{itemize} 
    \end{lem}
    
    


   \begin{proof}
    First assume that the images of $\cP$ and $\cQ$ are both contained in $\{\bullet, s\}$. Then  the image of $\cP$  is in fact contained in $\{\bullet\}$, and $(\imath, \jmath)$ is  right interlaced in the sense that 
 \[
 \mathbf{c}_1(\jmath)\geq \mathbf{c}_1(\imath)\geq \mathbf{c}_2(\jmath)\geq \mathbf{c}_2(\imath)\geq \mathbf{c}_3(\jmath)\geq \mathbf{c}_3(\imath) \geq \cdots.
 \]
 Hence $ (\imath',\jmath'):= (\imath,\DD(\jmath))$ is left interlaced in the sense that 
 \[
 \mathbf{c}_1(\imath')\geq \mathbf{c}_1(\jmath')\geq \mathbf{c}_2(\imath')\geq \mathbf{c}_2(\jmath')\geq \mathbf{c}_3(\imath')\geq \mathbf{c}_3(\jmath') \geq \cdots.
 \]
 Then it is clear that there is  unique painted bipartition of the form  $\tau'=(\imath',\cP')\times (\jmath',\cQ')\times \alpha'$ such that images of $\cP'$ and $\cQ'$ are both contained in $\{\bullet, s\}$. This proves the lemma in the special case when the images of $\cP$ and $\cQ$ are both contained in $\{\bullet, s\}$. 
 
 The proof of the lemma in the general case is easily reduced to this special case. 
   \end{proof}
    \begin{lem}\label{lemDDn2}
    If $\star \in \set{ \widetilde C, D,D^*}$, then there is a unique painted bipartition of the form $\tau'= (\imath',\cP')\times (\jmath',\cQ')\times \alpha'$ with the following properties:
  \begin{itemize}
        \item $
   (\imath',\jmath')= (\DD(\imath),\jmath); \smallskip
   $
   \item for all $(i,j)\in \BOX(\imath')$,
   \[
     \cP'(i,j)=\begin{cases}   
    \bullet \textrm{ or } s,&\textrm{ if  $\ \cP(i,j+1)\in \{\bullet, s\}$;} \smallskip \\
  \cP(i,j+1),& \textrm{ if $\ \cP(i,j+1)\notin \{\bullet, s\}$};\end{cases}
   \]
   \item for all $(i,j)\in \BOX(\jmath')$,
   \[
     \cQ'(i,j)=\begin{cases}   
    \bullet \textrm{ or } s,&\textrm{ if  $\ \cP(i,j)\in \{\bullet, s\}$;} \smallskip \\
  \cQ(i,j), & \textrm{ if $\ \cQ(i,j)\notin \{\bullet, s\}$}.  \end{cases}
   \]
  
    \end{itemize}
\end{lem}
\begin{proof}
  The proof is similar to that of \Cref{lemDDn1}. 
  
\end{proof}

\begin{defn}
 In the notation of \Cref{lemDDn1,lemDDn2}, we call $\tau'$ the naive descent of $\tau$, to be denoted by $\DDn(\tau)$.  
\end{defn} 

  

  
 \begin{Example} If
    \[
     \tau = \ytb{\bullet\bullet\bullet {c},\bullet {s} {c},{s},{c}}
    \times \ytb{\bullet\bullet\bullet ,\bullet {r} {d},{d}{d}, \none}
    \times \widetilde C, \]
   then 
   \[
    \nabla_{\mathrm{naive}}(\tau) =\ytb{\bullet\bullet{c} ,\bullet{c},\none }
    \times  \ytb{\bullet\bullet {s} ,\bullet {r} {d},{d}{d}}\times B^-.
    \]
    
\end{Example}
 
  \subsection{Descents of painted bipartitions}\label{sec:desc}
 

Suppose that 
$
\tau=(\imath,\cP)\times(\jmath,\cQ)\times \alpha \in  \mathrm{PBP}_\star(\check \CO)
$
and write 
\[
  \tau'_{\mathrm{naive}}=(\imath', \cP'_{\mathrm{naive}})\times (\jmath', \cQ'_{\mathrm{naive}})\times \alpha'
\]
for the naive descent of $\tau$. This is clearly an element of $  \mathrm{PBP}_{\star'}(\check \CO')$. 
%Put
%\begin{equation}\label{lstarco}
%  l:=l_{\star, \check \CO}:=\begin{cases}
% \frac{\bfrr_2(\ckcO)}{2}; & \quad \textrm{if } \star\in \{B, \widetilde C\};\\
% \frac{\bfrr_2(\ckcO)+1}{2}, &\quad \textrm{if } \star\in \{C, C^* \};\\
% \frac{\bfrr_2(\ckcO)-1}{2}, &\quad \textrm{if } \star\in \{ D, D^*\}.\\
%\end{cases}
%\end{equation}

The following two lemmas are easily verified and we omit the proofs. We will give an example for each of them. 
\delete{
\begin{lem}\label{descb}
Suppose that 
\[ 
\begin{cases}
\alpha = B^+; & \\
(2,3)\in \wp;\quad  &\\
\cQ(l',1)\in \set{r,d}.
\end{cases}
\]
Then there is a unique element in $\mathrm{PBP}_{\star'}(\check \CO',\wp')$ of the form
  \[
      \tau'=(\imath', \cP')\times (\jmath', \cQ')\times \alpha'
  \]
such that 
     $
     \cP' = \cP'_{\mathrm{naive}}
     $
     and 
     for all $(i,j)\in \BOX(\jmath')$, 
\[
\cQ'(i,j) = \begin{cases}
  r, & \ \text{ if  $(i,j) = (l',1)$;}\\
  \cQ'_{\mathrm{naive}}(i,j), & \ \text{ otherwise}.
\end{cases}
\]
\end{lem}


\begin{Example}
 If 
 \[
 \tau= \ytb{\bullet\bullet,\none} \times \ytb{\bullet \bullet, dd}\times 
  B^+,
 \]
 then 
\[
 \tau'_{\mathrm{naive}}= \ytb{\bullet s,\none} \times \ytb{\bullet, d}\times 
  \widetilde C\qquad\textrm{and}\qquad \tau'= \ytb{\bullet s,\none} \times \ytb{\bullet, r}\times 
  \widetilde C.
 \]
 Note that in this case, the nonzero row lengths of $\check \CO$ are $4,4,2,2$, $\wp=\{(2,3)\}$ and $l'=2$.
\end{Example}
\delete{\begin{proof}
 We only need to check that the triple $\tau'$ defined in the lemma is a painted bipartition. 
 
 Note that 
 \[
  \bar \Lambda_{l-1,2}(\imath', \cP')=\bar \Lambda_{l-1,2}(\imath'_{\mathrm{naive}}, \cP'_{\mathrm{naive}})
 \]
 and 
 \[
 \begin{array}{ccc}
 
      \Lambda_{l-1,1}(\cP_\tau)\times \Lambda_{l-1,2}(\cQ_\tau)
     &  &
        \Lambda_{l-1,1}(\cP_{\tau'})\times \Lambda_{l-1,2}(\cQ_{\tau'})\\
     \hline 
     \hspace{1em}\\
       \emptyset
      \times
      \ytb{ {x_{1}}{x_0},{\enon{\vdots}},{\enon{\vdots}},{x_{n}}}
      &
        \mapsto  &
        \emptyset 
        \times
      \ytb{ {\none}{r},{\none},{\none},\none}
      \end{array}
    \]
  
\end{proof}

Lemma \ref{descb} is easy to check and we omit the details. Note that $(\frac{\bfrr_2(\ckcO)}{2},1) \in \BOX(\jmath')$ under the first two conditions  of Lemma \ref{descb}. Similarly, we also have the following three lemmas. 
}
 }
\begin{lem}\label{descb2}
  Suppose that 
\[  \begin{cases}
 \alpha = B^+; & \\
 \bfrr_2(\ckcO)>0; & \\
 \cQ(l,1)\in \set{r,d}.
\end{cases}
\]
 Then there is a unique element in $\mathrm{PBP}_{\star'}(\check \CO')$ of the form
  \[
      \tau'=(\imath', \cP')\times (\jmath', \cQ')\times \alpha'
  \]
 such that 
     $
     \cQ' = \cQ'_{\mathrm{naive}}
     $
     and
     for all $(i,j)\in \BOX(\imath')$, 
\[
\cP'(i,j) = \begin{cases}
  s, & \ \text{ if $(i,j) = (l',1)$;}\\
  \cP'_{\mathrm{naive}}(i,j), & \ \text{ otherwise}.
\end{cases}
\]
\end{lem}

\begin{Example}
 If 
 \[
 \tau= \ytb{\bullet c, c} \times \ytb{\bullet r, rd}\times 
  B^+,
 \]
 then 
\[
 \tau'_{\mathrm{naive}}= \ytb{s c, c} \times \ytb{r, d}\times 
  \widetilde C\qquad\textrm{and}\qquad \tau'= \ytb{s c, s} \times \ytb{r, d}\times 
  \widetilde C.
 \]
 Note that in this case, the nonzero row lengths of $\check \CO$ are $4,4,4,2$, and $l'=2$.
\end{Example}

\delete{
\begin{lem}\label{descd1}
  Suppose that 
  \[  \begin{cases}
 \alpha = D; & \\
 (2,3)\in \wp;\quad  &\\
 \cP(l',1) \in \set{r,c}.
\end{cases}
\]
 Then there is a unique element in $\mathrm{PBP}_{\star'}(\check \CO',\wp')$ of the form
  \[
      \tau'=(\imath', \cP')\times (\jmath', \cQ')\times \alpha'
  \]
  such that $\cQ'=\cQ'_{\mathrm{naive}}$ and  for all $(i,j)\in \BOX(\imath')$, 
  \[
\cP'(i,j) = \begin{cases}
  r, & \ \text{ if } (i,j) = (l',1); \\
  \cP(l',1), &\  \text{ if } (i,j) = (l'+1,1);\\
  \cP'_{\mathrm{naive}}(i,j), & \ \text{ otherwise}.
\end{cases}
\]
   
\end{lem}




\begin{Example}
 If 
 \[
 \tau= \ytb{\bullet s,  c c, d d} \times \ytb{\bullet,\none, \none }\times 
  D,
 \]
 then 
\[
 \tau'_{\mathrm{naive}}=  \ytb{\bullet,  c,  d}  \times  \ytb{\bullet,\none, \none }\times 
  C,\qquad\textrm{and}\qquad \tau'= \ytb{\bullet, r, c}  \times  \ytb{\bullet,\none, \none }\times
  C.
 \]
 Note that in this case, the nonzero row lengths of $\check \CO$ are $5,5,3,1$,  $\wp=\{(2,3)\}$ and $l'=2$.
\end{Example}
}
\begin{lem}\label{descd2}
  Suppose that 
  \[  \begin{cases}
 \alpha = D; & \\
\mathbf r_2(\check \CO)=\mathbf r_3(\check \CO)>0;  &\\
\cP(l'+1,1)=r; &\\
\cP(l'+1,2)=c; &\\
 \cP(l,1)\in \set{r,d}.
\end{cases}
\]
 Then there is a unique element in $\mathrm{PBP}_{\star'}(\check \CO')$ of the form
  \[
      \tau'=(\imath', \cP')\times (\jmath', \cQ')\times \alpha'
  \]
  such that $\cQ'=\cQ'_{\mathrm{naive}}$ and  for all $(i,j)\in \BOX(\imath')$, 
  \[
\cP'(i,j) = \begin{cases}
  r, & \ \text{ if } (i,j) = (l'+1,1); \\
  \cP'_{\mathrm{naive}}(i,j), & \ \text{ otherwise}.
\end{cases}
\]
   
\end{lem}


\begin{Example}
 If 
 \[
 \tau= \ytb{\bullet\bullet, \bullet s, \bullet s, r c} \times \ytb{\bullet\bullet,\bullet,\bullet, \none }\times 
  D,
 \]
 then 
\[
 \tau'_{\mathrm{naive}}=\ytb{\bullet, \bullet , \bullet ,  c} \times \ytb{\bullet s,\bullet,\bullet, \none } \times 
  C,\qquad\textrm{and}\qquad \tau'=\ytb{\bullet, \bullet , \bullet ,  r} \times \ytb{\bullet s,\bullet,\bullet, \none } \times
  C.
 \]
 Note that in this case, the nonzero row lengths of $\check \CO$ are $7,7,7,3$,   and $l'=3$.
\end{Example}

\begin{defn}
We define the descent of $\tau$ to be 
\[
  \nabla(\tau):= \begin{cases}
  \tau', & \ \text{ if the condition of Lemma \ref{descb2}  or \ref{descd2} holds}; \\
  \nabla_{\mathrm{naive}}( \tau), & \ \text{ otherwise},
\end{cases}
\]
which is an element of $  \mathrm{PBP}_{\star'}(\check \CO')$. 
Here $\tau'$ is as in Lemmas  \ref{descb2} and \ref{descd2}. 
\end{defn}
In conclusion, we have defined the descent map
\[
\nabla: \mathrm{PBP}_{\star}(\check \CO)\rightarrow \mathrm{PBP}_{\star'}(\check \CO').
\]



\end{document}

%%% Local Variables:
%%% coding: utf-8
%%% mode: latex
%%% TeX-master: "ssunip.tex"
%%% TeX-engine: xetex
%%% ispell-local-dictionary: "en_US"
%%% End:
