\documentclass[ssunip]{subfiles}

\begin{document}

\def\oAC#1{\AC(#1)}
\def\pAC#1{\Lambda_{+}(#1)}
\def\nAC#1{\Lambda_{-}(#1)}

\section{Painted bipartitions}


\subsection{Vanishing propotition}
Because of the following proposition, we assume in the rest of this paper that $\check \CO$ is quasi-distinghuished  when $\star\in \{C^*, D^*\}$.


\begin{prop}
  Suppose that $\star\in \{C^*, D^*\}$. If the set $\mathrm{PBP}_\star(\check \CO)$ is nonempty, then $\check \CO$ is quasi-distinguished.
\end{prop}
\begin{proof}
  Suppose that $\tau=(\imath,\cP)\times(\jmath,\cQ)\times \alpha \in  \mathrm{PBP}_\star(\check \CO)$. If  $\star=C^*$, then  the definition of painted bipartitions implies that
 \[
 \bfcc_i(\imath)\leq \bfcc_i(\jmath) \qquad \textrm{for all } i=1,2,3, \cdots.
 \]
This forces that $\check \CO$ is quasi-distinguished.

 If  $\star=D^*$, then  the definition of painted bipartitions implies that
 \[
 \bfcc_{i+1}(\imath)\leq \bfcc_i(\jmath) \qquad \textrm{for all } i=1,2,3, \cdots.
 \]
This  also forces that   $\check \CO$ is quasi-distinguished.
 \end{proof}


\subsection{Tails of painted bipartitions}

In the rest of this subsection, we assume that $\star\in\{B, D, C^*\}$.
Let $(\imath,\jmath) = (\imath_{\star}(\ckcO),\jmath_{\star}(\ckcO))$.
%Note that $l\geq l'$ if $\star\in \{B,C^*\}$,   and $l\geq l'+1$ if $\star=D$.
Put
\[
  \star_{\mathbf t}:= \begin{cases}
  D, & \ \text{ if $\star\in \{B,D\}$}; \\
  C^*, &\  \text{ if $\star=C^*$}.
\end{cases}
\quad
\text{and}
\quad
k := \begin{cases}
  \frac{\bfrr_{1}(\ckcO)-\bfrr_{2}(\ckcO)}{2} + 1 &
    \text{if $\star\in \{B,D\}$}; \\
\frac{\bfrr_{1}(\ckcO)-\bfrr_{2}(\ckcO)}{2} -1 &  \text{if $\star=C^*$}.
  \end{cases}
\]
Here $k = \bfcc_{1}(\jmath)-\bfcc_{1}(\imath)+1$,
$\bfcc_{1}(\jmath)-\bfcc_{1}(\imath)$,
and $\bfcc_{1}(\imath)-\bfcc_{1}(\jmath)$
when $\star = B,C^{*},D^{*}$, respectively.

Let $\ckcO_{\bftt}$ be the following Young diagram that is
determined by the pair $(\star, \ckcO)$.
\begin{itemize}
    \item If $\star \in \set{B,D}$,
then $\ckcO_{\bftt}$  consists of two rows with lengths $2k-1$ and $1$.
\item
If $\star =C^*$, then $\ckcO_{\bftt}$ consists of one row
with length  $2k+1$.
\end{itemize}
Note that  in all these three cases
 $\check \CO_{\mathbf t}$ has $\star_{\mathbf t}$-good parity and every element in $\PBP_{\star_\bftt}(\ckcO_\bftt)$ has the form
 \begin{equation}
 \label{tail0}
  \ytb{{x_1} , {x_2} , {\enon\vdots},{\enon{\vdots}},{x_k}  } \times \emptyset \times
  D,\qquad \qquad  \ytb{{x_1} , {x_2} , {\enon\vdots},{\enon{\vdots}},{x_k}  } \times \emptyset \times
  D\qquad\textrm{or}\qquad \emptyset \times  \ytb{{x_1} , {x_2} , {\enon\vdots},{\enon{\vdots}},{x_k}  } \times
 C^*,
\end{equation}
respectively if $\star=B, D$ or $C^*$. %Here $k=l-l'+1, l-l'$ or $l-l'$ respectively.

%\subsubsection{The case when $\star = B$}
Let
$
\tau=(\imath,\cP)\times(\jmath,\cQ)\times \alpha \in  \mathrm{PBP}_\star(\check \CO)
$ be as before.


\noindent {\bf The case when $\star = B$.}
In this case, we define the tail $\tau_\bftt$ of $\tau$ to be the first painted bipartition in \eqref{tail0} such that the multiset $\{x_1, x_2, \cdots, x_k\}$ is the
union of the multiset
\[
  \set{\cQ(j,1)| \bfcc_{1}(\imath)+1 \leq j \leq  \bfcc_{1}(\jmath) }
    % \cQ(\bfcc_{1}(\imath)+1,1),\cQ(\bfcc_{1}(\imath)+2,1),\cdots, \cQ(\bfcc_{1}(\jmath),1)}
\]
with the set
\[
  \begin{cases}
 \set{c}, &
 \qquad
  \text{if $\alpha = B^+$, and either $\bfcc_{1}(\imath)=0$ or $\cQ(\bfcc_{1}(\imath),1)\in \set{\bullet,s}$};  \\
 \set{s},&
  \qquad \text{if $\alpha = B^-$, and either $\bfcc_{1}(\imath)=0$ or $\cQ(\bfcc_{1}(\imath),1)\in \set{\bullet,s}$}; \\
%  \qquad\text{when } \alpha_\tau = B^-, \text{ and, } l'=0 \textrm{ or } \cQ_\tau(l',1)\in \set{\bullet,s},  \\
\set{\cQ(\bfcc_{1}(\imath),1)},&
\qquad  \text{if $\bfcc_{1}(\imath)>0$ and $\cQ(\bfcc_{1}(\imath),1)\in \{r,d\}$.}
\end{cases}
\]


\smallskip

 \smallskip






\noindent {\bf The case when $\star = D$.}
In this case, we define the tail $\tau_\bftt$ of $\tau$ to be the second painted bipartition in \eqref{tail0} such that the multiset $\{x_1, x_2, \cdots, x_k\}$ is the
union of the multiset
\[
\set{\cP(j,1)| \bfcc_{1}(\jmath)+2 \leq j \leq \bfcc_{1}(\imath)}
\]
with the set
\[
  \begin{cases}
  &
 \ \text{if $\bfrr_2(\ckcO)=\bfrr_3(\ckcO)$,}\\
 \set{c},& \quad \text{$(\cP(\bfcc_{1}(\jmath)+1,1) ,\cP(\bfcc_{1}(\jmath)+1,2)) = (r,c)$ }\\
 & \quad \text{ and $\cP(l,1)\in \set{r,d}$};  \\
\set{\cP(\bfcc_{1}(\jmath)+1,1)},&
\    \text{otherwise.}
\end{cases}
\]

 \smallskip

 \smallskip

\noindent {\bf The case $\star = C^*$.}
When $k=0$, we define the tail $\tau_{\bftt}$ of $\tau$ to be
$\emptyset\times \emptyset \times C^{*}$.
When $k> 0$, we define the tail $\tau_\bftt$ of $\tau$ to be the third painted bipartition in \eqref{tail0} such that
\[
  (x_1, x_2, \cdots, x_k)= (\cQ(\bfcc_{1}(\imath)+1,1),\cQ(\bfcc_{1}(\imath)+2,1),\cdots, \cQ(\bfcc_{1}(\jmath),1)).
\]


 When $\star \in \set{B,D}$, the symbol in the last box of the tail $\tau_\bftt\in \PBP_{\star_\bftt}(\ckcO_\bftt)$ will be impotent for us. We write $x_\tau$ for it, namely
\[
x_\tau := \cP_{\tau_\bftt}(k,1).
\]
 The following lemma is easy to check.

\begin{lem}\label{tailtip}
If $\star=B$, then
\[
x_\tau=s\Longleftrightarrow
\begin{cases}
  \alpha=B^-;\\
  \cQ(\bfcc_{1}(\jmath),1) \in\{\bullet, s\},
  \end{cases}
%\quad \textrm{if and only if}\quad \alpha=B^- \ \textrm{ and }\  \cQ(l,1) = s,
\]
and
\[
x_\tau=d \Longleftrightarrow
%\quad \textrm{if and only if}\quad
\cQ(\bfcc_{1}(\jmath),1) =d.
\]
If $\star=D$, then
\[
x_\tau=s\Longleftrightarrow \cP(\bfcc_{1}(\imath),1) = s,
\]
and
\[
x_\tau=d\Longleftrightarrow \cP(\bfcc_{1}(\imath),1) =d.
\]
\qed
\end{lem}







\subsection{Some properties of the descent maps}



%We state the key properties of the descent map in this section and the proofs will be given in \Cref{sec:DD.proof}.


The key properties of the descent map when $\star\in \set{C,\wtC,D^*}$ are summarized in the following proposition.

\begin{prop}\label{prop:CC.bij}
Suppose that $\star \in \set{C,\wtC,D^*}$ and cosider the
descent map
\begin{equation}\label{eq:DD.CC}
\nabla: \PBP_\star(\ckcO)\longrightarrow  \PBP_{\star'}(\ckcOp).
\end{equation}
\begin{enuma}
\item If
$\star=D^*$ or $\bfrr_1(\ckcO)>\bfrr_2(\ckcO)$, then
the map \eqref{eq:DD.CC}  is bijective.

\item If  $\star\in \{C,\widetilde C\}$ and $\bfrr_1(\ckcO)=\bfrr_2(\ckcO)$, then the  map \eqref{eq:DD.CC} is injective and its image equals
\[
\Set{\tau'\in \PBP_{\star'}(\ckcOp)| x_{\tau'}\neq s}.
\]
\end{enuma}
\end{prop}

\begin{proof}
  We give the detailed proof of part (b) when $\star=\wtC$.
  The proofs in the other  cases are similar and are left to the reader.

%We assume that $\star = \wtC$ and $\bfrr_1(\ckcO)=\bfrr_2(\ckcO)$.
By the definition of descent map (see \eqref{eq:def.alphap}), we have a well
defined map% , the map \eqref{eq:DD.CC} induces a map
\begin{equation}\label{eq:DD.CC1}
\DD: \Set{\tau\in \PBP_{\star}(\ckcO)| \cP_\tau(\bfcc_{1}(\imath),1)\neq c}\rightarrow \Set{\tau'\in \PBP_{\star'}(\ckcOp)|  \alpha_{\tau'}=B^+}.
\end{equation}
Suppose that $\tau'$ is an element  in the codomain of the map \eqref{eq:DD.CC1}. Similar to the proof of Lemma \ref{lemDDn1}, there is a unique element in $\tau:=\nabla^{-1}(\tau')\in \PBP_{\star}(\ckcO)$ such that
for all $i=1,2, \cdots,\bfcc_{1}(\imath)$,
\[
  \cP_{\tau}(i,1)\in \{\bullet, s\},
\]
and
for all $(i,j)\in \BOX(\tau')$,
\[
%\begin{equation}
     \cP_\tau(i,j+1)=\begin{cases}
    \bullet \textrm{ or } s,&\textrm{ if  $\ \cP_{\tau'}(i,j)\in \{\bullet, s\}$;} \smallskip \\
  \cP_{\tau'}(i,j),& \textrm{ if $\ \cP_{\tau'}(i,j)\notin \{\bullet, s\}$},\end{cases}
%   \end{equation}
\]
 and
 \[
%   \begin{equation}
     \cQ_\tau(i,j)=\begin{cases}
    \bullet \textrm{ or } s,&\textrm{ if  $\ \cQ_{\tau'}(i,j)\in \{\bullet, s\}$;} \smallskip \\
  \cQ_{\tau'}(i,j), & \textrm{ if $\ \cQ_{\tau'}(i,j)\notin \{\bullet, s\}$}.  \end{cases}
%   \end{equation}
\]
Note that $\tau$ is in the domain of \eqref{eq:DD.CC1}. It is then routine to check that the map
\[
  \nabla^{-1}: \Set{\tau'\in \PBP_{\star'}(\ckcOp)|  \alpha_{\tau'}=B^+}
  \rightarrow  \Set{\tau\in \PBP_{\star}(\ckcO)| \cP_\tau(\bfcc_{1}(\imath),1)\neq c}
\]
and the map \eqref{eq:DD.CC1} are inverse to each other. Hence the map \eqref{eq:DD.CC1} is bijective.

\smallskip

Similarly, the map
\[
  \nabla: \Set{\tau\in \PBP_{\star}(\ckcO)| \cP_\tau(\bfcc_{1}(\imath),1)= c}
  \rightarrow \Set{\tau'\in \PBP_{\star'}(\ckcOp)| \alpha_{\tau'}=B^-, \cQ_{\tau'}(\bfcc_{1}(\imath),1)\in\{r,d\}}
\]
is well-defined, and we show that it is bijective by explicitly  constructing its inverse. In view of Lemma \ref{tailtip}, this proves the proposition in the case we are considering.

\trivial{
  Suppose $\star = \wtC$ and $\bfrr_{1}(\ckcO) >\bfrr_{2}(\ckcO)$. Then
  $\bfcc_{1}(\imath)>\bfcc_{1}(\jmath)$ and the ``inverse'' of descent is
  constructed in an obvious way such that
  $\cP(\bfcc_{1}(\imath),1) = c\Leftrightarrow \alpha = B^{-}$.


  Suppose $\star = C$ and $\bfrr_{1}(\ckcO) -\bfrr_{2}(\ckcO)$. Then
  $\bfcc_{1}(\imath)=\bfcc_{1}(\jmath)+1$. So
  $\cP_{\tau}(\bfcc_{1}(\imath),1)\neq s$. Hence
  $\cP_{\tau'}(\bfcc_{1}(\imath),1)= \cP_{\tau}(\bfcc_{1}(\imath),1) \neq s$.
  The ``inverse'' of descent is
  constructed in an obvious way.

  Suppose $\star = D^{*}$. Then
  $\bfcc_{1}(\imath)\geq \bfcc_{1}(\jmath)+1$.
  The ``inverse'' of descent is
  constructed in an obvious way.
}
\end{proof}

\medskip

The key properties of the descent map when $\star\in \set{D,B,C^*}$ are summarized in the following two propositions.


%The following equation of signatures will be crucial in our computation of the local system in the next section.

The following proposition is easy to verify:
\begin{prop}\label{prop:D.sign}
  \begin{enuma}
    \item
Suppose $k\in \bN^{+}$, $\star = D$ and $\ckcO$ consists of two rows with lengths $2k-1$ and $1$.
The following map is bijective:
\[
  \PBP_{\star}(\ckcO) \longrightarrow \Set{
    (p,q,\varepsilon)|
    \begin{minipage}{13em}
      $p,q \in \bN$ such that $p+q=2k$,\\
      % $\varepsilon \in \bZ/2\bZ$
      $\varepsilon=1$ if $pq=0$, and \\
      $\varepsilon\in \set{0,1}$ otherwise.
    \end{minipage}
  } \quad \tau \mapsto (p_{\tau},q_{\tau}, \varepsilon_{\tau}).
\]
\item
Suppose $k\in \bN$, $\star = C^{*}$ and $\ckcO$ consists of one row with length $2k+1$.
The following map is bijective:
\[
  \PBP_{\star}(\ckcO) \longrightarrow \set{(2p,2q)|p,q \in \bN \text{ such that } p+q=k} \quad \tau \mapsto (p_{\tau},q_{\tau}).
\]
\end{enuma}
\qed
\end{prop}



For every painted bipartition $\tau$, write
\[
  \ssign(\tau):=(p_\tau, q_\tau).
\]

When $\bfrr_2(\check \CO)>0$, the double descent
$\DD^2(\tau):=\DD(\DD(\tau))$ is well-defined whenever
$\tau\in \PBP_{\star}(\ckcO)$. As in the Introduction, $\CO$ denotes
the Barbasch-Vogan dual of $\check \CO$. We also consider it as a Young diagram.

\begin{prop}\label{prop:delta}
Assume that $\star \in \set{D,B,C^*}$ and $\bfrr_2(\ckcO)>0$. Write $\ckcOpp := \ckDD(\ckcO')$ and consider the map
\begin{equation}\label{eq:delta}
  \delta  \colon \PBP_\star(\ckcO)\longrightarrow
    \PBP_\star(\ckcOpp)\times \PBP_{\star_\bftt}(\ckcO_\bftt),
    \qquad \tau \mapsto (\DD^2(\tau),\tau_\bftt).
\end{equation}
\begin{enuma}
\item Suppose that
$\star = C^*$ or $\bfrr_2(\ckcO)>\bfrr_3(\ckcO)$. Then the map \eqref{eq:delta} is a bijective, and for every $\tau\in  \PBP_\star(\ckcO) $,
    % We have the following equation of signatures.
\begin{equation}\label{eq:sign.D}
\ssign(\tau)
=(\bfcc_2(\cO),\bfcc_2(\cO))+\ssign(\DD^2(\tau))+\ssign(\tau_\bftt).
\end{equation}

\item Suppose that  $\star \in \set{B,D}$ and $\bfrr_2(\ckcO)=\bfrr_3(\ckcO)$. Then the map \eqref{eq:delta} is  injection and its  image equals
\begin{equation}\label{eq:delta.I}
    \Set{ (\tau'',\tau_0)  \in \PBP_\star(\ckcOpp)\times \PBP_D(\ckcO_\bftt)  |
      \begin{minipage}{15em}
    $x_{\tau''} = d$ or \\
    $x_{\tau''}\in \set{r,c}$  and
    $\cP_{\tau_0}^{-1}(\set{s,c})\neq \emptyset$
  \end{minipage}
}.
\end{equation}
Moreover,  for every $\tau\in  \PBP_\star(\ckcO) $,
\begin{equation}\label{eq:sign.GD}
\ssign(\tau)
=(\bfcc_2(\cO)-1,\bfcc_2(\cO)-1)+\ssign(\DD^2(\tau))+\ssign(\tau_\bftt).
\end{equation}
\end{enuma}
\end{prop}

\begin{proof}
  We give the detailed proof of part (b) when $\star=B$.
  The proofs in the other  cases are similar and are left to the reader.
  Let $\tau = (\imath,\cP)\times (\jmath,\cQ)\times \alpha$ and $\tau'' = \DD^{2}(\tau)$.
  % $\tau' = \DD(\tau)  = (\imath',\cP')\times (\jmath',\cQ')\times \alpha$.
  % $\tau'' = \DD(\tau')  = (\imath'',\cP'')\times (\jmath'',\cQ')\times \alpha$.

  According to the descent algorithm, we have
  \begin{equation}\label{eq:delta.1}
    \begin{split}
    \cP(i,j) &= \begin{cases}
      \bullet & \text{if } i< \bfcc_{1}(\imath), j=1;\\
      \cP_{\tau''}(i,j-1) & \text{if } i< \bfcc_{1}(\imath) \text{ or } j>1;\\
      \end{cases}\\
    \cQ(i,j) &= \begin{cases}
      \bullet & \text{if } i< \bfcc_{1}(\imath), j=1;\\
      \cQ_{\tau''}(i,j-1) & \text{if } i< \bfcc_{1}(\imath) \text{ or }  j>2;\\
      \end{cases}
    \end{split}
  \end{equation}

  We label the boxes avoided in \eqref{eq:delta.1} as the following:
  \begin{equation}\label{eq:ydelta}
  \tau: \hspace{1em}
  \ytb{
    %{\none[1]},
    %{\none[\bfcc_{1}(\jmath)]}
    {x_{0}},
    \none,\none,\none,\none}
  \times
  \ytb{
    %{\none[1]}{\none[2]},
    {x_{1}}{x''},
    {x_{2}},{\enon{\vdots}},{\enon{\vdots}},{x_{k}}}
  \times
  \alpha
   % \mapsto
   % \tau': \ytb{{y'_{0}},\none,\none,\none,\none}
   % \times \ytb{\none{x'_{0}},\none,\none,\none,\none}
   \quad
  \mapsto
  \quad
  \tau'': \ytb{{\none[\emptyset]},\none,\none,\none,\none}
  \times \ytb{\none{y''},\none,\none,\none,\none}
  \times \alpha''
\end{equation}
where $k := \bfcc_{1}(\jmath)-\bfcc_{1}(\imath)+1$, $x_{0}$, $x_{1}$ and $y''$ has
coordinate $(\bfcc_{1}(\imath),1)$ in the corresponding painted Young diagram.
By the descent algorithm, $x'' = y''$.
Also note that $\bfcc_{2}(\cO) = 2\bfcc_{1}(\imath)$.
% We have the following two cases, $x_{1}\in \set{r,d}$ or
% $x_{1}\in \set{\bullet,s}$.

We now discuss case by case according to the paint of $x_{1}$.
When $x_{1} = s$, we have $(x_{0}, \alpha'') = (c,B^{-})$.
Now %(the meaning of $\Sign$ are defined in the same way of that in \eqref{defpbp0})
\begin{equation}\label{eq:sign.GD2}
  \begin{split}
    & \Sign(\tau) - \Sign(\tau'') \\
    &= (2\bfcc_{1}(\imath)-2,2\bfcc_{1}(\imath)-2)
    + (1,1) + (0,2) - (0,1) + \Sign(\alpha) + \Sign(x_{2}\cdots x_{k})\\
    & = (\bfcc_{2}(\cO)-1, \bfcc_{2}(\cO)-1)+ \Sign(\tau_{\bftt}).
  \end{split}
\end{equation}
In second line of the above formula, the terms count the signatures of extra
bullets, $x_{0}$, $x_{1}$, $\alpha'$, $\alpha$ and $x_{1}\cdots x_{k}$ respectively.

When $x_{1} = \bullet$, we have $(x_{0}, \alpha'') = (\bullet,B^{+})$.
When $x_{1} \in \set{r,d}$, we have $x'' = y''= d$ $(x_{0}, \alpha'') = (c,\alpha)$.
In these two cases, the signature formula can be checked similarly.% to \eqref{eq:sign.GD2}.
It also clear that the image of $\delta$ is in \eqref{eq:delta.I}.

%One can check case by case that
It is not hard to construct the map from \eqref{eq:delta.I} to
$\PBP_{\star}(\ckcO)$, and then the proposition follows. To illustrate the idea,
we give the detail construction of $\delta^{-1}(\tau'',\tau_{0})$
when $\cP_{\tau_{0}}(k,1)=c$: most of the entries in $\cP$ and $\cQ$ are defined
by \eqref{eq:delta.1}; we refer to \eqref{eq:ydelta} for the labeling of the
rest of the entries, and set $\alpha:=B^{-}$, $x'':= \cQ_{\tau''}(\bfcc_{1}(\imath),1)$,
\[
(x_{0},x_{1}) := \begin{cases}
  (\bullet,\bullet) & \text{if } \alpha''= B^{+},\\
  (c,s) & \text{if } \alpha''= B^{-},
\end{cases}
\quad \text{and}\quad
x_{i} := \cP_{\tau_{0}}(i-1,1) \quad \text{for } i = 2,3,\cdots,k.
\]

\trivial{
In fact, it suffice to check the proposition for the orbit $\ckcO$
consists of three rows with lengths $2k$, $2$, and $2$.
}



  \trivial{
    Suppose $\star =  C^{*}$.
    Then $\tau$ is obtained by attaching a column of $\bfcc_{1}(\imath)$
    bullets on the left of $\cP$, and a column of  $\bfcc_{1}(\imath)$
    bullets concatenated with $\tau_{\bftt}$ on the left of $\cQ$. The bijectivity is clear. The
    signature formula follows from $2\bfcc_{1}(\imath)  = \bfcc_{2}(\cO)$.


    Suppose $\star =  D$ and $\bfrr_{2}(\ckcO)>\bfrr_{3}(\ckcO)$.
    Then $\tau$ is obtained by attaching a column of $\bfcc_{1}(\jmath)$
    bullets concatenated with $\tau_{\bftt}$ on the left of $\cP$, and a column of  $\bfcc_{1}(\jmath)$
    bullets on the left of $\cQ$. The bijectivity is clear. The
    signature formula follows from $2\bfcc_{1}(\jmath)  = \bfcc_{2}(\cO)$.

    % Suppose $\star =  D$ and $\bfrr_{2}(\ckcO)=\bfrr_{3}(\ckcO)$.
    % Then $\tau$ is obtained by attaching a column of $\bfcc_{1}(\jmath)$
    % bullets concatenated with $\tau_{\bftt}$ on the left of $\cP$, and a column of  $\bfcc_{1}(\jmath)$
    % bullets on the left of $\cQ$. The bijectivity is clear. The
    % signature formula follows from $2\bfcc_{1}(\jmath)  = \bfcc_{2}(\cO)$.
  }
\end{proof}


\begin{prop}\label{cor:D.inj1}
Suppose that $\star \in \set{D,B,C^*}$ and $\bfrr_{2}(\ckcO)>0$.
Then the map
\begin{equation}\label{eq:D.BD}
  \delta' \colon \PBP_{\star}(\ckcO)\longrightarrow
   \PBP_{\star'}(\ckcOp)\times \PBP_{\star_\bftt}(\ckcO_\bftt)
   \qquad \tau \mapsto (\DD(\tau), \tau_\bftt)
\end{equation}
% \begin{equation}\label{eq:D.BD}
%   \delta' \colon \PBPes(\ckcO)\longrightarrow
%    \PBPesp(\ckcOp)\times \PBP_{\star_\bftt}(\ckcO_\bftt)
%    \qquad \tau \mapsto (\DD(\tau), \tau_\bftt)
% \end{equation}
is injective. Moreover, \eqref{eq:D.BD} is bijective
unless $\star\in \set{B,D}$ and $\bfrr_2(\ckcO)=\bfrr_3(\ckcO)>0$.
% when $\star = C^*$
\end{prop}
\begin{proof}
  % Suppose $\tau_{1}, \tau_{2} \in \PBP_{\star}(\ckcO)$ such that
  % $ \delta' (\tau_{1}) = \delta'(\tau_{2})$. Now $\DD(\tau_{1}) = \DD(\tau_{2})$ follows from
  % the injectivity results in \Cref{prop:delta} and \Cref{prop:CC.bij}.
It follows from the injectivity results in \Cref{prop:delta} and \Cref{prop:CC.bij}.
  % Suppose $\uptau_{1}, \uptau_{2} \in \PBPes(\ckcO)$ such that
  % $ \delta' (\uptau_{1}) = \delta'(\uptau_{2})$. By the definition of descent,
  % $\wp_{\uptau_{1}} = \wp_{\uptau_{2}}$. Now $\tau_{1} = \tau_{2}$ follows from
  % the injectivity results in \Cref{prop:delta} and \Cref{prop:CC.bij}.
\end{proof}

Combining the above propotition with \Cref{prop:D.sign}, we get the following.
\begin{cor}\label{cor:dpinj}
If $\star \in \set{B, D,C^*}$, then the map
\begin{equation}
  \begin{array}{rcl}
   \PBP_{\star}(\ckcO)&\rightarrow&
   \PBP_{\star'}(\ckcOp)\times \BN\times \bN\times \Z/2\Z, \smallskip\\
   \tau& \mapsto & (\DD(\tau), p_\tau, q_\tau, \varepsilon_\tau)
   \end{array}
\end{equation}
is injective. \qed
\end{cor}

\section{Proof of theorem on counting}

We do induction on the number of rows of the nilpotent orbits $\ckcO$.

\subsection{Properties of $\AC(\tau)$}

\begin{prop}
  \begin{enuma}
    \item For each $\uptau\in \PBPes(\ckcOp)$, $\oAC{\uptaup}\neq 0$.
  \end{enuma}
\end{prop}

In this section, we let $\star \in \set{B,D,C^{*}}$. Now
$\star' = \wtC, C, D^{*}$  respectively.

We assume all the claim had been proven when $\cO$ has at most two columns, that
is
\begin{itemize}
  \item When $\star = D$, $\bfrr_{3}(\ckcO) = 0$;
  \item When $\star = B$, $\bfrr_{2}(\ckcO)=0$;% is the zero orbit.
  \item When $\star = C^{*}$, $\bfrr_{2}(\ckcO)=0$.% is the zero orbit.
\end{itemize}


\begin{prop}\label{prop:LLS}
   We have:
  \begin{enumS}
    \item For each $\uptau'\in \PBPesp(\ckcOp)$, $\oAC{\uptaup}\neq 0$.
    \item For each $\uptau\in \PBPes(\ckcO)$, $\oAC{\uptau}\neq 0$.
    \item Let $\uptau'_{1},\uptau'_{2}$ be two distinct elements in
    $\PBPesp(\ckcOp)$. Then either
    \begin{itemize}
      \item      $\oAC{\uptau'_{1}}\neq \oAC{\uptau'_{2}}$; or
      \item $(\tau''_{1},\varepsilon'_{1})\neq (\tau''_{2},\varepsilon'_{2})$
      and $\Sign(\tau''_{1})=\Sign(\tau''_{2})$ where
      $(\tau''_{i},\varepsilon'_{i}) := \DD(\uptaup_{i})$ for $i=1,2$.
  \end{itemize}
  \item Let $\uptau_{1},\uptau_{2}$ be two distinct elements in
    $\PBPes(\ckcO)$. Then either
    \begin{itemize}
      \item $\oAC{\uptau_{1}}\neq \oAC{\uptau_{2}}$; or
      \item $\uptau'_{1}\neq \uptau'_{2}$ and $\varepsilon_{1} = \varepsilon_{2}$
       where $(\uptau'_{i},\varepsilon_{i}) := \DD(\uptau_{i})$ for $i=1,2$.
  \end{itemize}
    \item \label{p:drcls.1}
    When $\star \in \set{B,D}$
    The local system $\oAC{\uptau}$ is disjoint with their determinant twist:
    \begin{equation}\label{eq:LS.dis}
      \set{\oAC{\uptau} | \uptau \in \PBPes(\ckcO)} \cap
      \set{\oAC{\uptau}\otimes \det| \uptau\in \PBPes(\ckcO)} = \emptyset.
    \end{equation}
  \item \label{it:LLS.4}
  We have
  \begin{enumT}
    \item If $x_{\uptau} = s$, then $\pAC{\uptau} = \nAC{\uptau} = 0$.
    \item If $x_{\uptau} \in \set{r,c}$, then $\pAC{\uptau} \neq 0$ and $\nAC{\uptau}=0$.
    \item If $x_{\uptau} = d$, then $\pAC{\uptau}\neq 0$ and $\nAC{\uptau}\neq 0$.
  \end{enumT}
  \item When $\ckcOp$ is weakly-distinguished, the map
  $\AC\colon \PBPesp( \cOp )\longrightarrow K_{\star'}(\cOp)$ is injective.
  \item When $\ckcO$ is weakly-distinguished, the map
  $\AC\colon \PBPes(\cO)\longrightarrow K_{\star}(\cO)$ is a injective.
    \item When $\ckcO$ is +weakly-distinguished, the maps
    % $\pUpsilon_{1}:=\pr_{1}\circ \Upsilon$
    \[
      \begin{tikzcd}[row sep=0em]
        \pUpsilon_{\cO} \colon \set{\oAC{\uptau}|\pAC{\uptau}\neq 0} \ar[r] & \set{\pAC{\uptau}}\\
       \oAC{\uptau} \ar[r,maps to] & \pAC{\uptau}& \text{and}\\
        \nUpsilon_{\cO} \colon \set{\oAC{\uptau}|\nAC{\uptau}\neq 0} \ar[r] & \set{\nAC{\uptau}}&\\
       \oAC{\uptau} \ar[r,maps to] & \nAC{\uptau}
      \end{tikzcd}
    \]
    are injective.
    \item When $\cO$ is noticed, the map
    \begin{equation}\label{eq:up}
      \begin{tikzcd}[row sep=0em]
        \Upsilon_{\cO} \colon \set{\oAC{\uptau}|\pAC{\uptau}\neq 0} \ar[r] & \set{(\pAC{\uptau},\nAC{\uptau})|\pAC{\uptau}\neq 0} \\
        \oAC{\uptau} \ar[r,maps to] & (\pAC{\uptau},\nAC{\uptau})
      \end{tikzcd}
    \end{equation}
    is injective.\footnote{In fact, we only need to know whether $\nAC{\uptau}$ is
      non-zero or not.}
  \end{enumS}
\end{prop}




\subsection{The initial case:}\label{sec:pfDC.init}
When $k=0$, $\cO = (C_{1}=2c_{1},C_{0}=2c_{0})$ has  at most
two columns.
Now $\uppi_{\uptaup}$ is the trivial representation of $\Sp(2c_{0},\bR)$.
The lift $\uppi_{\uptaup} \mapsto \Thetab(\uppi_{\uptaup})$ is in fact a stable
range theta lift.%, and therefore $\uppi_{\uptau}\neq 0$.
For $\uptau\in \drc(\cO)$, let $(p_{1},q_{1}) = \ssign(\bfxx_{\uptau})$ and
$x_{\uptau}$ be the foot of $\uptau$. It is easy to see that  (using associated
character formula)
\[
  \oAC{\uppi_{\uptau}} = \cT_{\uptau}\cdot \cP_{\uptau},
  \text{ where }\cT_{\uptau}:=\dagger \dagger_{c_{0},c_{0}},
  \text{ and } \cP_{\uptau}:=\begin{cases}
    \ddagger_{p_{1},q_{1}}&  x_{\uptau} \neq d,\\
    \dagger_{p_{1},q_{1}}& x_{\uptau} = d.\\
  \end{cases}
\]
Note that $\oAC{\uppi_{\uptau}}$ is irreducible.

Let $\cO_{1} = (2(c_{1}-c_{0}))\in \dpeNil(D)$. Using the above formula,
one can check that the following maps are bijections
\[
  \begin{tikzcd}[row sep=0em]
    \drc(\cO_{1}) &\ar[l] \drc(\cO) \ar[r] & \LLS(\cO)\\
    \bfxx_{\uptau} & \ar[l,maps to] \uptau \ar[r,mapsto] & \oAC{\uptau}.
  \end{tikzcd}
\]
To see that $\drc(\cO)\ni\uptau\mapsto \oAC{\uptau}$ is an injection, one check the
following lemma holds:
\begin{lem}\label{c:init.CD}
  The map $\drc(\cO_{1})\longrightarrow \bN^{2}\times \bZ/2\bZ$ given by
  $\uptau\mapsto (\ssign(\uptau),\upepsilon_{\uptau})$ is injective (see
  \Cref{sec:upepsilon} for the definition of $\upepsilon_{\uptau}$). Moreover,
  $\upepsilon_{\uptau}=0$ only if $\ssign(\uptau)\geq (0,1)$. \qedhere
\end{lem}


Moreover, $\oAC{\uptau}\otimes \det \notin \LLS(\cO)$ for any $\uptau\in \drc(\cO)$.
In fact, if $\ssign(\bfxx_{\uptau})\succeq (1,0)$, $\oAC{\uptau}$
on the 1-rows with $+$-signs is trivial and $\oAC{\uptau}\otimes \det$ has
non-trivial restriction. When $\ssign(\bfxx_{\uptau})\nsucceq (1,0)$,
$\bfxx_{\uptau}=s\cdots s$, all 1-rows of $\oAC{\uptau}$ are marked by ``$=$'' and
all 1-rows of $\oAC{\uptau}\otimes \det$ are  marked by ``$-$''.

Therefore our main \Cref{thm:count} and main proposition \Cref{prop:CD} holds for $\cO$.



Let $C_{i} = \bfcc_{i}(\ckcO)$ for $i =1,2, \cdots$.

\subsection{The descent case}\label{sec:pf.ds.CD}
Now we assume
$(2,3)\in \PP_{\star}(\ckcO)$.


% \medskip
\subsubsection{}
We first calculate the local system attached to $\ckcOp$.

Let $\uptau'\in \PBPesp(\ckcO')$.

For $\cLpp\in \AOD(\cOpp)$, let $(p_{1},q_{1}) = \lsign(\cLpp)$ and
$(p_2,q_2)=(C_{2}-q_{1},C_{2}-p_{1})$. Then
\[
  \vartheta(\cLpp) = (\maltese^{y}\dagger\cLpp) \cdot \dagger_{(n_{0},n_{0})}
\]
where $y$ is an integer determined by $\Sign(\uptaupp)$ and $n_{0} = (C_{2}-C_{3})/2$.
%\sqcup \LLS(\cOpp)\otimes \det

Therefore, $\LS(\cOpp)\longrightarrow \LS(\cOp)$ given by
$\cL\mapsto \vartheta(\cL)$ is an injection between abelian groups.

By \eqref{eq:LS.dis}, we have a injection
\begin{equation}\label{eq:LS.CD.inj}
  \begin{tikzcd}[row sep=0em]
    \LLS(\cOpp)\times \bZ/2\bZ \ar[r] &\LLS(\cOp)\\
    (\cLpp,\upepsilon') \ar[r,maps to] & \vartheta(\cLpp\otimes \det^{\upepsilon'}).
  \end{tikzcd}
\end{equation}
In particular, we have $\oAC{\uptaup}\neq 0$ for every
$\uptaup\in \drc(\cOp)$.

Now suppose $\uptaup_{1}\neq \uptaup_{2}$.
Suppose
$\oAC{\uptaup_{1}} = \oAC{\uptaup_{2}}$. By \eqref{eq:LS.CD.inj},
\begin{equation}\label{eq:eps.CD1}
  \varepsilon_{\uptaup_{1}} = \varepsilon_{\uptaup_{2}}
\end{equation}
and
$\oAC{\uptaupp_{1}} = \oAC{\uptaupp_{2}}$.
In particular, we have
$\Sign(\uptaupp_{1}) = \Sign(\oAC{\uptaupp_{1}}) = \Sign(\oAC{\uptaupp_{2}}) = \Sign(\uptaupp_{1})$.
We now claim that $\uptaupp_{1}\neq \uptaupp_{2}$.
It suffice to consider the case where $\wp_{\uptaupp_{1}} = \wp_{\uptaupp_{2}}$.
Then $\wp_{\uptaup_{1}} = \wp_{\uptaup_{2}}$ by \eqref{eq:eps.CD1} and so
$\taup_{1}\neq \taup_{2}$.
By \Cref{prop:CC.bij}, $\taupp_{1}\neq \taupp_{2}$ and this proves the claim.

\trivial{
Now suppose $\uptaup_{1}\neq \uptaup_{2}\in \drc(\cOp)$ and
$\oAC{\uptaup_{1}}=\oAC{\uptaup_{2}}$. By \eqref{eq:LS.CD.inj},
$\upepsilon'_{1}=\upepsilon'_{2}$ and $\oAC{\uptaupp_{1}}=\oAC{\uptaupp_{2}}$.
In particular,  $\uptaupp_{1}$ and $\uptaupp_{2}$  have the same signature. %, say $(p'',q'')$.% are dot-r-c diagrams for the
% same real orthogonal group.
By \Cref{lem:ds.CD} and the induction hypothesis, $\uptaupp_{1}\neq \uptaupp_{2}$ and so
$\uppi_{\uptaupp_{1}}\otimes {\det}^{\upepsilon'_{1}} \neq \uppi_{\uptaupp_{2}}\otimes {\det}^{\upepsilon'_{2}}$.
Now the injectivity of theta lifting yields
\[
  \uppi_{\uptaup_{2}} = \Thetab(\uppi_{\uptaupp_{1}}\otimes {\det}^{\upepsilon'_{1}})
  \neq \Thetab(\uppi_{\uptaupp_{2}}\otimes {\det}^{\upepsilon'_{2}}) = \uppi_{\uptaup_{2}}
\]
Suppose $\cOp$ is noticed. Then $\cOpp = \eDD(\cOp)$ is noticed by definition
and
$(\uptaupp,\upepsilon')\mapsto \Ch(\uppi_{\uptaupp}\otimes{\det}^{\upepsilon'})$
is an injection into $\LS(\cOpp)$.
Now \eqref{eq:LS.CD.inj} implies $\uptaup\mapsto\oAC{\uptaup}$ is also an injection.

Suppose $\cOp$ is quasi-distingushed. Then $\cOpp = \eDD(\cOp)$ is
quasi-distingushed by definition
and
$(\uptaupp,\upepsilon')\mapsto \Ch(\uppi_{\uptaupp}\otimes{\det}^{\upepsilon'})$
is  a bijection with $\LSaod(\cOpp)$.
Since the component group of each K-nilpotent orbit in $\cOp$ is naturally
isomorphic to the component group of its descent. So we deduce that $\uptaup \mapsto \oAC{\uptaup}$
is a bijection onto $\LSaod(\cOp)$.

This proves the main proposition for $\cOp$.
}

\subsubsection{}

Now we consider the local system attached to $\ckcO$.
Let $\uptau\in \PBPes(\ckcO)$.
By the signature formula \eqref{eq:sign.D}
\begin{equation}\label{eq:LS.D.ds}
  \begin{split}
  \AC(\uptau)  & =\left( \left(\maltese^{y'}
    (\AC(\uptaupp)\otimes (\det)^{\varepsilon_{\wp'}})
    \cdot 1^{(n_{0},n_{0})}\right) \cdot
   1^{\Sign(\tau_{\bftt})} \right) \otimes (1^{+,-})^{\varepsilon_{\tau}}\\
  & =\left(\maltese^{y'} (\AC(\uptau'')\otimes (\det)^{\varepsilon_{\wp'}}) \otimes (1^{+,-})^{\varepsilon_{\tau}}
    \right) \cdot \AC(\tau_{\bftt})\\
  & =\left(\maltese^{y} \AC(\uptau') \otimes (1^{+,-})^{\varepsilon_{\tau}}
    \right) \cdot \AC(\tau_{\bftt})
  \end{split}
\end{equation}
where $y' = (p_{\tau}-q_{\tau}-p_{\tau''}+q_{\tau''})/2$ and $y$ is determined
by $\Sign(\uptau)$ when
$\star\in \set{B,D}$.
Whne $\star\in \set{C^{*}}$, $y'=y''=0$.



% Now let $\uptau\in \drc(\cO)$. % and $(p_{0},q_{0}):=\ssign(\bfxx_{\uptau})$.
% By \eqref{eq:sp-nsp-sig}, we have
% \begin{equation}\label{eq:LS.D.ds}
%   \oAC{\uptau} =  \cT_{\uptau}\cdot \cP_{\uptau}\neq 0, \text{ where
%   }\cT_{\uptau} := \dagger\oAC{\uptaup}, \text{ and } \cP_{\uptau}:=\oAC{\bfxx_{\uptau}}.
% \end{equation}
% In particular, $\uppi_{\uptau}\neq 0$ and we could determine $\bfxx_{\uptau}$ from
% the marks on the 1-rows of $\oAC{\uptau}$.

% The factorization of $\oAC{\uptau}$ in the fashion of \Cref{eq:ls.factor} is also given by
% \eqref{eq:ped.factor1} and \eqref{eq:bd.factor2}.
% When $\upepsilon=0$, $(p_{0},q_{0})\succeq \ssign(d)=(1,1)$ and so
% $\oAC{\uptau}\neq \oAC{\uptau}\otimes \bfone_{\ssign(\uptau)}^{+,-}$.

Now suppose $\uptau_{1} \neq \uptau_{2} \in \PBPes(\ckcO)$ and
$\AC({\uptau_{1}})=\AC( \uptau_{2} )$. By \eqref{eq:LS.D.ds},

By \eqref{eq:LS.D.ds}, we have
$\AC({{\tau_{1}}_{\bftt}}) = \AC({{\tau_{2}}_{\bftt}})$ and .
Therefore
${\tau_{1}}_{\bftt}={\tau_{2}}_{\bftt}$, $\varepsilon_{\uptau_{1}} =\varepsilon_{\uptau_{2}}$ and
$\AC( \uptaup_{1} )=\AC(\uptaup_{2} )$.
Thanks to \Cref{cor:D.inj1}, $\uptaup_{1} \neq \uptaup_{2}$.

% $\uptaupp_{1}\neq \uptaupp_{2}$.
% Hence
% Applying the main proposition of $\cOp$, we have
% $\upepsilon'_{1}=\upepsilon'_{2}$ and $\uptaup_{1}\neq \uptaup_{2}$.
% Now by the injectivity of theta lifing, we conclude that
% \[
%   \uppi_{\uptau_{2}} = \Thetab(\uppi_{\uptaup_{1}})\otimes (\bfone^{+,-})^{\upepsilon_{1}}
%   \neq \Thetab(\uppi_{\uptaup_{2}})\otimes (\bfone^{+,-})^{\upepsilon_{2}} = \uppi_{\uptaup_{2}}
% \]

The claims in item \ref{it:LLS.4} follows from \eqref{eq:LS.D.ds} and the initial
case for $\tau_{\bftt}$.

\subsubsection{}

Note that $\oAC{\uptau}$ is obtained from $\oAC{\uptaup}$ by attaching the peduncle
determined by $\bfxx_{\uptau}$. The claims about noticed and
quasi-distinguished orbit are easy to verify. We leave them to the reader.



\subsection{The general descent case}\label{sec:pf.gd.CD}
%We  assume $k\geq 1$ and all the properties are satisfied by $\uptaupp\in \drc(\cOpp)$.
%We retain the notation in \Cref{sec:gd2.CD}.
We assume $\bfrr_{2}(\ckcO) = \bfrr_{3}(\ckcO)>0$.
Note that in this case, $\star\in \set{B,D}$.

\subsubsection{}
We now describe the local systems $\AC( \uptaup )$.

\begin{equation}\label{eq:LS.taup}
  \AC( \uptaup ) = \vartheta_{\sfss'}^{\sfss}(\AC(\uptaupp)) = \maltese^{y'}
  (\Lambda_{+}( \AC(\uptaupp)) + \Lambda_{-}(\AC(\uptaupp)))\cdot 1^{(0,0)} \neq 0
\end{equation}
Here $y'$ is dentermined by $\Sign(\taupp)$.


According to \Cref{prop:CC.bij},  $x_{\taupp}\neq s$.
By induction hypothesis, $\Lambda_{+}(\AC(\uptaupp)) \neq 0$. Hence
$\AC(\uptaup)\neq 0$.

% \begin{equation}\label{eq:LS.taup}
%   \oAC{\uptaup} = \vartheta(\oAC{\uptaupp}) = \maltese^{\frac{\abs{\ssign(\uptaupp)}}{2}}
%   (\dagger \pAC{\uptaupp} + \dagger \nAC{\uptaupp}) \neq 0
% \end{equation}
% where
% $\lsign(\dagger\pAC{\uptaupp}) = (q''_{1}, p''_{1}-1)$ and
% $\lsign(\dagger\nAC{\uptaupp})=(q''_{1}-1,p''_{1})$ (if
% $\dagger\nAC{\uptaupp}\neq 0$).


% \subsubsection{}
% We now describe the associated cycle $\AC( \uptaup )$ and $\AC( \uptau )$ more
% precisely using our formula of the associated character and the induction
% hypothesis. Note that these local systems are always non-zero which implies that
% $\uppi_{\uptaup}$ and $\uppi_{\uptau}$ are unipotent representations attached to
% $\cOp$ and $\cO$ respectively.

% First note that in $\uptaup$ and $\uptaupp$, $x_{\uptaupp}\neq s$ by the definition of
% dot-r-c diagram.
% \footnote{
%   Suppose $\uptaupp\in \drc(\cOpp)$ has the shape in \eqref{eq:gd2.drc} and
%   $x_{\uptaupp}=s$. By the induction hypothesis,   $\pAC{\uptaupp}=\nAC{\uptaupp}=0$ and so
%   the theta lift of $\uppi_{\uptaupp}$ vanishes.
% }
% % in generalized descent.
% By the induction hypothesis, $\oAC{\uptaupp}^{+}\neq 0$.
% % , and so
% % \[
% %   \oAC{\uptaup} = \vartheta(\oAC{\uptaupp}) = \dagger \oAC{\uptaupp}^{+}\cup
% %   \dagger \oAC{\uptaupp}^{-} \neq 0.
% % \]

% % Now for each $\uptaup\in \drc(\cOp)$, $\uptaupp$


Let $(p''_{1},q''_{1}) := \lsign(\oAC{\uptaupp})$.
Then
\begin{equation}\label{eq:lsign.1}
\lsign(\pAC{\uptaupp}) = (p_{0}-1,q_{0}), \text{ and }
\lsign(\nAC{\uptaupp})=(p_{0},q_{0}-1) \text{ if } \nAC{\uptaupp}\neq \emptyset.
\end{equation}
\begin{equation}\label{eq:LS.taup}
  \oAC{\uptaup} = \vartheta(\oAC{\uptaupp}) = \maltese^{\frac{\abs{\ssign(\uptaupp)}}{2}}
  (\dagger \pAC{\uptaupp} + \dagger \nAC{\uptaupp}) \neq 0
\end{equation}
where
$\lsign(\dagger\pAC{\uptaupp}) = (q''_{1}, p''_{1}-1)$ and
$\lsign(\dagger\nAC{\uptaupp})=(q''_{1}-1,p''_{1})$ (if
$\dagger\nAC{\uptaupp}\neq 0$).

Let $(p_{1},q_{1}) := \lsign(\oAC{\uptau})$ and
$(e,f):=(p_{1}-p''_{1}+1, q_{1}-q''_{1}+1)$. Using \Cref{lem:gd.inj}
% \eqref{eq:bd.prop} and
% \eqref{eq:def.u},
one can show that
\[
  (e,f)=\ssign(\tau_{\bftt}).
\]

\trivial[]{ It suffice to consider the most left three columns of the peduncle
  part: this part has signature
  $\lsign(\cT_{\uptau}) +(1,1) = \ssign(\tau_{\bftt}x_{\uptaupp})= \ssign(\tau_{\bftt})+\lsign(\cD_{\uptaupp})$.
  Therfore,
  \[\ssign(\tau_{\bftt}) = \lsign(\cT_{\uptau})-\lsign(\cD_{\uptaupp})
  + (1,1) = \lsign(\oAC{\uptau})-\lsign(\oAC{\uptaupp})+(1,1).
  \]
}

Now
\begin{equation}\label{eq:gd.ls}
  \oAC{\uptau} =
  \begin{cases}
    (\bfone^{+,-}\otimes \maltese^t (\Lambda_{+}\AC( \uptaupp )\cdot 1^{(0,0)}) \cdot \pcP_{\uptau}
    + (\bfone^{+,-}\otimes (\Lambda_{-}\AC( \uptaupp )\cdot 1^{(0,0)})  \cdot \ncP_{\uptau}
    & \text{if } x_{\uptau} \neq d\\
    (\dagger\maltese^t \dagger \pAC{\uptaupp}) \cdot \pcP_{\uptau}
    + (\dagger\maltese^{t} \dagger \nAC{\uptaupp})  \cdot \ncP_{\uptau}
    & \text{if } x_{\uptau} =d\\
  \end{cases}
\end{equation}
% \begin{equation}\label{eq:gd.ls}
%   \oAC{\uptau} =
%   \begin{cases}
%     (\bfone^{+,-}\otimes \dagger\maltese^t \dagger \pAC{\uptaupp}) \cdot \pcP_{\uptau}
%     + (\bfone^{+,-}\otimes \dagger\maltese^{t} \dagger \nAC{\uptaupp})  \cdot \ncP_{\uptau}
%     & \text{if } x_{\uptau} \neq d\\
%     (\dagger\maltese^t \dagger \pAC{\uptaupp}) \cdot \pcP_{\uptau}
%     + (\dagger\maltese^{t} \dagger \nAC{\uptaupp})  \cdot \ncP_{\uptau}
%     & \text{if } x_{\uptau} =d\\
%   \end{cases}
% \end{equation}
where
\[
  \begin{split}
    t & = \frac{\abs{\ssign(\uptau)-\ssign(\uptaupp)}}{2} = \frac{\abs{\ssign(\tau_{\bftt})}}{2}\\
    \pcP_{\uptau}& = \begin{cases} 1^{(e,(-1)^{\varepsilon_{\tau}}(f-1))} & \text{when
      } f\geq 1 \text{ and } x_{\uptau}\neq d \\
      \dagger_{e,f-1} & \text{when
      } f\geq 1 \text{ and } x_{\uptau}= d \\
      0 & \text{otherwise}
    \end{cases}\\
    \ncP_{\uptau}& = \begin{cases} \ddagger_{e-1,f} & \text{when
      } e\geq 1 \text{ and } x_{\uptau}\neq d \\
      \dagger_{e-1,f} & \text{when
      } e\geq 1 \text{ and } x_{\uptau}= d \\
      0 & \text{otherwise}
    \end{cases}.\\
  \end{split}
\]
%
% $\pcP$ and $\ncP$ are 1-rows of sign $(p_{1}-p''_{1}+1,q_{1}-q''_{1})$ and
% $(p-p'',q-q''+1)$ respectively.

\medskip

\delete{
By induction hypothesis, we have
a factorization of $\oAC{\uptaupp}$ as in \eqref{eq:ls.factor}
\[
  \oAC{\uptaupp} = \sum_{i=-(k-1)}^{k-1}\cB_{\uptaupp,i}\cdot \cD_{\uptaupp,i}.
\]
%where $\abs{i}\leq k-1$ in the above formula.
Then we have the following factorization of $\oAC{\uptau}$:
\[
\oAC{\uptau}=\sum_{i\neq 0} \cT_{\uptau,i}\cdot \cT_{\uptau,i}, \text{ with
} \cT_{\uptau,i}:=
\begin{cases}
\dagger\dagger \cB_{\uptaupp,i-1} & i\geq 1 \\
\dagger\dagger \cB_{\uptaupp,i+1} & i\leq -1.
\end{cases}
\]
Now we describle $\cT_{\uptau,i}$ according to $x_{\uptaupp}$ case by case. In the
following discussion,
\[
  \uup\cdots\uup\quad \text{ and }\quad  \uum\cdots\uum
\]
always denote a row has length
$2i+1$ with unspecified associated character.
%We have the following cases
%\subsubsection{}
%First note that $x_{\uptaupp}\neq s$ in generalized descent.

 \subsubsection{Case $x_{\uptaupp}=r$}\label{sec:z.r}
 We will see that the factorization \eqref{eq:d.factor} always holds
 when $x_{\uptaupp}=r$ and $n\geq 2$.
 % according to the about list.
  %\item $x_{\uptaupp}=r$. %In this case, all length $3$ rows ends with $+$ sign!
  % So
  % $\bsign{\bfxx_{\uptaupp}} \succ (2,0)$, $\pAC{\uptaupp}\neq 0$ has tail with
  % sign $+$, and $\nAC{\uptaupp}=0$.
  We have three sub-cases:
  \begin{enumT}
  \item $\bfpp_{\uptau} = \tytb{rc\cdots,\vdots,r}$. Then
  $\oAC{\uptau}=\sum_{i=1}^{k} (\dagger\dagger \cB_{\uptaupp,i-1})\cdot \cT_{\uptau,i}$, where
  \begin{equation}\label{eq:rr.c}
  \cT_{\uptau,i}  = \tytb{\uup\cdots\uup,+,\vdots,+} \quad \forall i\geq 1.
  \end{equation}

 When $n\geq 2$, we have $\cT_{\uptau,i}\succeq \tytb{+,+}$ and the
 factorization \eqref{eq:d.factor} holds.

 When $n=1$, we set
 \begin{equation} \label{eq:gd.rr}
   \cB_{\uptau,0} = 0,
   \cB_{\uptau,i} = \cT_{\uptau,i} \text{ and
   } \cD_{\uptau,i} = \cT_{\uptau,i} \text{ for } i\geq 1.
 \end{equation}
 Now $\oAC{\uptau} = \sum_{i=1}^{k} \cB_{\uptau,i}\cdot \cD_{\uptau,i}$.
%where $i$ is the index such that $\cB_{\uptaupp,i}\neq 0$.

 \item $\bfpp_{\uptau} = \tytb{rc\cdots,\vdots,r,d}$.
 In this case, $n\geq 2$.
  \begin{equation}\label{eq:rd.c}
  \cT_{\uptau,i}  = \tytb{\uup\cdots\uup,-,+,\vdots,+} \succ \tytb{-,+} \quad  \forall i\geq 1
  % \cT_{\uptau,1}  = \tytb{\uup\uum\uup,-,+,\vdots,+}\quad
  % \cT_{\uptau,2}  = \tytb{\uup\uum\uup\cdots,-,+,\vdots,+}
  \end{equation}
In particular, \eqref{eq:d.factor} holds.
% $\cT_{\uptau,i}\succeq \ddagger_{(2,1)}\succ \dagger_{(1,1)}$ and the
% factorization
% $\oAC{\uptau} = \cB_{\uptau,1}\cdot \cD_{\uptau,1}$ such that $\cD_{\uptau,1}=\tytb{+,-}$.
 \item $\bfpp_{\uptau} = \tytb{sr\cdots,\vdots,s,{x_{j}},\vdots,{x_{n}}}$.
% $y_{1}=x_{\uptaupp}=r$ and $\bfxx_{\uptau} = s\cdots sx_{j}\cdots x_{n}$ with
 In this case $\#s(\bfxx_{\uptau})\geq 1$.

 When $n\geq 2$, the factorization \eqref{eq:d.factor} holds.
 More precisely, we have four cases according to the mark of $x_{n}$.
 \begin{enumT}
   \item $x_{n}=s$, i.e. $\bfxx_{\uptau} = s\cdots s$. Then
   \[
  \cT_{\uptau,i}  = \tytb{\uup\cdots\uup,=,\vdots,=} \quad \forall i \geq 1
   \]
   When $n= 1$,
   $\oAC{\uptau} = \sum_{i=-1}^{-k} \cB_{\uptau,i}\cdot \cD_{\uptau,i}$
   with $\cB_{\uptau,i} = \cT_{\uptau,i}$ and $\cD_{\uptau,i} = \cT_{\uptau,i}$.
   % and the
   % factorization is also given by the formula \eqref{eq:gd.rr}.
   \item $x_{n}=r$. Then $n\geq 2$ and
   \[
  \cT_{\uptau,i}  = \tytb{\uup\cdots\uup,+,\vdots,+,=,\vdots,=} \succeq \tytb{+,+,=}\succ \tytb{+,+,\none}
  \quad \forall i\geq 1
  \]
   \item $x_{n}=c$. Then $n\geq 2$ and
   \[
  \cT_{\uptau,i}  = \tytb{\uup\cdots\uup,+,\vdots,+,=,\vdots,=} \succeq \tytb{+,=,=}\succ \tytb{+,=,\none}
  \quad \forall i\geq 1
  %\cT_{\uptau,2}  = \tytb{\uup\uum\uup\cdots,+,\vdots,+,=,\vdots,=} \succeq \tytb{+,=,=}\succ \tytb{+,=,\none}
   \]
   \item $x_{n}=d$. Then $n\geq 2$ and
   \[
  \cT_{\uptau,i}  = \tytb{\uup\cdots\uup,+,\vdots,+,-,\vdots,-}  \succeq \tytb{+,-,-}\succ \tytb{+,-,\none}
  \quad \forall i\geq 1
  %\cT_{\uptau,2}  = \tytb{\uup\uum\uup\cdots,+,\vdots,+,-,\vdots,-} \succeq \tytb{+,-,-}\succ \tytb{+,-,\none}
   \]
 \end{enumT}
 % Suppose $\Sign(\bfxx_{\uptau})=(p_{1},q_{1})$ with $p_{1}+q_{1}=2n$
  % \begin{equation}%\oAC{\uptau} = \dagger\dagger \pAC{\uptaupp} \cdot \ytableaushort{+,\vdots,+,{-/=},\cdots,{-/=}}
  %   \cB_{\uptau}  =
  %   \tytb{+-+,+,\vdots,+,-,\vdots,-}  \text{ when } x_{n} = d \text{ or  } \tytb{+-+,+,\vdots,+,=,\vdots,=}
  %   \text{ when } x_{n} = r/c
  % \end{equation}
  % \[
  %   \cB_{\uptau} = \tytb{+-+,=,\vdots,=} \text{ when } x_{n}=s.
  % \]
  % Note that, we have at least one $-$ sign in 1-rows.
  % At least two $-$ signs if $x_{n}=c/d$.

  %  We have  $\cB_{\uptau}\succeq \ddagger_{(0,2)}$  when $x_{n}=s$;
  %  $\cB_{\uptau}\succeq \ddagger_{(2,1)}$  when $x_{n}=r$;
  % $\cB_{\uptau}\succeq \ddagger_{(2,1)}$ when $x_{n}=c$;
  % $\cB_{\uptau}\succeq \dagger_{(2,1)}$ when $x_{n}=d$.
  \end{enumT}


  \subsubsection{Case $x_{\uptaupp}=c$}\label{sec:z.c}
  % In this case, the length $3$-rows will contain at least one $-$ sign.
  %$\pAC{\uptaupp}\neq 0$ has tail sign $=$, and $\nAC{\uptaupp}=0$.
  %We assume $\Sign(\bfxx_{\uptau})=(p_{1},q_{1})$.
  We always have $\cT_{\uptau,2}=\emptyset$.
  \begin{enumT}
    \item  $x_{n} =  s$. In this case $x_{\uptau}=s\cdots s$.
    \begin{equation}\label{eq:ss.c}
      \cT_{\uptau,i} = \tytb{\uum\cdots\uum,=,\vdots,=} \quad \forall i\geq 1.
    \end{equation}
    When $n\geq 2$, the  factorization \eqref{eq:d.factor} holds.

    When $n=1$, we have
    \begin{equation}\label{eq:gd.ss}
      \oAC{\uptau} = \sum_{i=-1}^{-k}\cB_{\uptau,i}\cdot \cD_{\uptau,i}
      \text{ where } \cB_{\uptau,i}=\cT_{\uptau,-i}
      \text{ and }\cD_{\uptau,i}=\cT_{\uptau,-i}.
  \end{equation}
  \item $x_{n} = r$.
    In this case, $\bfxx_{\uptau}= s\cdots sr\cdots r$ where
    $\#s(\bfxx_{\uptau})>1$ and $n\geq 2$.
   \[
     \cT_{\uptau,i}  = \tytb{\uum\cdots\uum,=,\vdots,=,+,\vdots,+,} \succeq \tytb{=,+,+}\succ\tytb{\none,+,+}
     \quad \forall i\geq 1.
   \]
   In particular, \eqref{eq:d.factor} holds.
   \item $x_{n} = c$.
   % In this case, $\bfxx_{\uptau}= s\cdots s r\cdots r$ where number of
   \begin{enumT}
     \item
     When $\#s(\bfxx_{\uptau})>1$ and so $n\geq 2$.
     \[
       \cT_{\uptau,i}  = \tytb{\uum\cdots\uum,+,\vdots,+,=,\vdots,=} \succ \tytb{+,=}
     \quad \forall i\geq 1.
     \]
     % The
     % Otherwise, $\cT_{\uptau,1}\succ \ddagger_{1,1}$.
     The factorization \eqref{eq:d.factor} holds.
     \item
     When $\bfxx_{\uptau}=r\cdots rc$ with $n_{r}:=\#r(\bfxx_{\uptau}) \geq 0$.
     \[
       \cT_{\uptau,i}  = \tytb{\uum\cdots\uum,+,\vdots,+} %\quad \forall i\geq 1. \succ \tytb{+}.
       \quad \forall i\geq 1.
     \]
     We have $\oAC{\uptau} = \sum_{i=1}^{k}\cB_{\uptau,i}\cdot \cD_{\uptau,i}$  where $\cD_{\uptau,i} = \tytb{\uum\cdots\uum,+}$ and
     $\cB_{\uptau,i} = \cT_{\uptau,i}\cdot \ddagger_{2n_{r},0}$.
   \end{enumT}
   \item $x_{n} = d$. In this case, $n\geq 2$ and
   $\#s(\bfxx_{\uptau})+\#c(\bfxx_{\uptau})\geq 1$. Therefore,
    \begin{equation}\label{eq:ss.c}
      \cT_{\uptau,i} = \tytb{\uum\uup\uum\cdots,-,\vdots,-,+,\vdots,+}\succ \tytb{-,+}, \quad \forall i\geq 1
    \end{equation}
    and the factorization \eqref{eq:d.factor} holds.
  \end{enumT}

  \subsubsection{Case $x_{\uptaupp}=d$}\label{sec:z.d}
  In this case, $\bfxx_{\uptau}$ could be anything.
%  Then
  %$\pAC{\uptaupp} = \cT_{\uptaupp} \cdot \pcB_{\uptaupp}$ and $\nAC{\uptaupp}=\cT_{\uptaupp} \cdot \ncB_{\uptaupp}$.
 % We assume $\bsign{\pAC{\uptaupp}}=(p-1,q)$ and $\bsign{\nAC{\uptaupp}}=(p,q-1)$.
  % \[
  %   \oAC{\uptau} = \dagger \dagger \cT_{\uptaupp}\cdot \cB_{\uptau}.
  % \]
  \begin{enumT}
    \item $\bfxx_{\uptau}=s\cdots s$. We have
    \[
      \cT_{\uptau,i} = \tytb{\uum\cdots\uum,=,\vdots,=} \quad \forall i\geq 1.
    \]
    When $n\geq 2$, the factorization \eqref{eq:d.factor} holds.
% Clearly $\pAC{\uptau}=\emptyset$.
    When $n=1$, the factorizaton is given by \eqref{eq:gd.ss}.
    \item $\bfxx_{\uptau}=r\cdots r$.
    \[
      \cT_{\uptau,i} = \tytb{\uup\cdots\uup,+,\vdots,+} \quad \forall i\leq -1.
    \]
    When $n\geq 2$, the factorization \eqref{eq:d.factor} holds.

    When $n=1$,
    \[
    \oAC{\uptau} = \sum_{i=1}^{k}\cB_{\uptau,i}\cdot \cD_{\uptau,i} \text{
      where } \cB_{\uptau,i} = \cT_{\uptau,-i} \text{ and } \cD_{\uptau,i} = \cT_{\uptau,-i}.
    \]
    \item $\bfxx_{\uptau} = s\cdots sr\cdots r$ with $n_{s}:=\#s(\bfxx_{\uptau})\geq 1$
    and $n_{r}:=\#r(\bfxx_{\uptau})\geq 1$
    \[
      \cT_{\uptau,i} = \tytb{\uum\cdots\uum,=,+,+,+,\vdots,+,=,\vdots,=}
      \quad \cT_{\uptau,-i} = \tytb{\uup\cdots\uup,+,=,=,+,\vdots,+,=,\vdots,=}
      \quad \forall i\geq 1
    \]
    %Clearly $\cT_{\uptau,i}\succ \ddagger_{(1,1)}$ and $n\geq 2$.
    We have
    \[
      \oAC{\uptau} = \sum_{i=0}^{k}\cB_{\uptau,i}\cdot \cD_{\uptau,i}.
    \]
    Here
      \begin{align*}
        \cB_{\uptau,0} &= \sum_{i=1}^{k} \cT_{\uptau,i}\cdot \tytb{\uum\cdots\uum,=} \cdot\ddagger_{2n_{r}-2,2n_{s}-2}, &
        \cD_{\uptau,0} &= \tytb{+,+},\\
        \cB_{\uptau,i} &= \cT_{\uptau,-i}\cdot \ddagger_{2n_{r}-2,2n_{s}},  &
        \cD_{\uptau,i} &= \tytb{\uup\cdots\uup,+}, \quad \forall i\geq 1\\
      \end{align*}
    \item $\bfxx_{\uptau} = s\cdots sc$ where $n_{s}:=\#s(\bfxx_{\uptau})\geq 0$.
    \begin{equation}\label{eq:ped.ssc}
      \cT_{\uptau,i} = \tytb{\uum\cdots\uum,+,=,\vdots,=} \quad \cT_{\uptau,-i} = \tytb{\uup\cdots\uup,=,=,\vdots,=} \quad \forall i\geq 1
    \end{equation}
    We have
    \[
      \oAC{\uptau} = \sum_{i=1}^{k}\cB_{\uptau,i}\cdot \cD_{\uptau,i}+ \sum_{i=1}^{k}\cB_{\uptau,-i}\cdot \cD_{\uptau,-i}.
    \]
    For $i\geq 1$,
      \begin{align*}
        \cB_{\uptau,i} &= \cT_{\uptau,i}\cdot \ddagger_{0,2n_{s}}, &
        \cD_{\uptau,i} &= \tytb{\uum\cdots\uum,+},\\
        \cB_{\uptau,-i} &= \cT_{\uptau,-i}\cdot \ddagger_{0,2n_{s}},  &
        \cD_{\uptau,-i} &= \tytb{\uup\cdots\uup,=}. %, \quad \forall i\geq 1\\
      \end{align*}
    \item $\bfxx_{\uptau} = r\cdots rc$ where $n_{r} = \#r(\bfxx_{\uptau})\geq 1$.
    \[
      \cT_{\uptau,i} = \tytb{\uum\cdots\uum,+,+,\vdots,+} \quad \cT_{\uptau,-i} = \tytb{\uup\cdots\uup,=,+,\vdots,+}
      \quad \forall i\geq 1
    \]
    We have
    \[
      \oAC{\uptau} = \sum_{i=0}^{k}\cB_{\uptau,i}\cdot \cD_{\uptau,i}.
    \]
    where
      \begin{align*}
        \cB_{\uptau,0} &= \sum_{i=1}^{k}\cT_{\uptau,-i}\cdot \tytb{\uup\cdots\uup,+} \cdot \ddagger_{2n_{r}-2,0}, &
        \cD_{\uptau,0} &= \tytb{=,+},\\
        \cB_{\uptau,i} &= \cT_{\uptau,i}\cdot \ddagger_{2n_{r},0},  &
        \cD_{\uptau,i} &= \tytb{\uum\cdots\uum,+}  \quad \forall i\geq 1.\\
      \end{align*}
    \item $\bfxx_{\uptau} = s\cdots sr\cdots rc$ where
    $n_{s} := \#s(\bfxx_{\uptau})\geq 1, n_{r}:=\#r(\bfxx_{\uptau})\geq 1$.
    \[
      \cT_{\uptau,i} = \tytb{\uum\cdots\uum,+,+,\vdots,+,=,\vdots,=}\succ \tytb{+,=},
      \quad \cT_{\uptau,-i} = \tytb{\uup\cdots\uup,=,+,\vdots,+,=,\vdots,=}\succ \tytb{+,=} \quad \forall i\geq 1
    \]
    %Note that $\cT_{\uptau,i}\succ \ddagger_{(1,1)}$.
    Now the factorization \eqref{eq:d.factor} holds.
    \item $\bfxx_{\uptau} = s\cdots sd$ where $\#s(\bfxx_{\uptau})\geq 1$.
    \begin{equation}\label{eq:ped.ssd}
      \cT_{\uptau,i} = \tytb{\uum\cdots\uum,+,-,\vdots,-} \quad \cT_{\uptau,-i} = \tytb{\uup\cdots\uup,-,-,\vdots,-}
      \quad \forall i\geq 1
    \end{equation}
    We have
    \[
      \oAC{\uptau} = \sum_{i=0}^{k}\cB_{\uptau,-i}\cdot \cD_{\uptau,-i}.
    \]
    where
      \begin{align*}
        \cB_{\uptau,0} &= \sum_{i=1}^{k}\cT_{\uptau,i}\cdot \tytb{\uum\cdots\uum,-} \cdot \dagger_{0,2n_{s}-2}, &
        \cD_{\uptau,0} &= \tytb{+,-},\\
        \cB_{\uptau,-i} &= \cT_{\uptau,-i}\cdot \ddagger_{0,2n_{s}},  &
        \cD_{\uptau,-i} &= \tytb{\uup\cdots\uup,-}  \quad \forall i\geq  1.\\
      \end{align*}
    \item $\bfxx_{\uptau} = r\cdots rd$ where $n_{r}=\#r(\bfxx_{\uptau})\geq 1$.
    \[
      \cT_{\uptau,i} = \tytb{\uum\cdots\uum,+,+,\vdots,+} \quad \cT_{\uptau,-i} = \tytb{\uup\cdots\uup,-,+,\vdots,+}
      \quad \forall i\geq 1
    \]
    We have
    \[
      \oAC{\uptau} = \sum_{i=0}^{k}\cB_{\uptau,i}\cdot \cD_{\uptau,i}.
    \]
    where
      \begin{align*}
        \cB_{\uptau,0} &= \sum_{i=1}^{k}\cT_{\uptau,-i}\cdot \tytb{\uup\cdots\uup,+} \cdot \dagger_{2n_{r}-2,0}, &
        \cD_{\uptau,0} &= \tytb{-,+},\\
        \cB_{\uptau,i} &= \cT_{\uptau,i}\cdot \dagger_{2n_{r},0},  &
        \cD_{\uptau,i} &= \tytb{\uum\cdots\uum,+}  \quad \forall i\geq  1.\\
      \end{align*}
    \item $\bfxx_{\uptau} = s\cdots sr\cdots rd$ where
    $\#s(\bfxx_{\uptau})\geq 1, \#r(\bfxx_{\uptau})\geq 1$.
    \[
      \cT_{\uptau,i} = \tytb{\uum\cdots\uum,+,+,\vdots,+,-,\vdots,-} \succ\tytb{+,-},
      \quad \cT_{\uptau,-i} = \tytb{\uup\cdots\uup,-,+,\vdots,+,-,\vdots,-} \succ\tytb{+,-}
      \quad \forall i\geq 1.
    \]
    Now the factorization \eqref{eq:d.factor} holds.
    \item $\bfxx_{\uptau} = s\cdots sr\cdots rcd$ where
    $\#s(\bfxx_{\uptau})\geq 0, \#r(\bfxx_{\uptau})\geq 0$.
    \[
      \cT_{\uptau,i} = \tytb{\uum\cdots\uum,+,+,\vdots,+,-,\vdots,-}\succ\tytb{+,-},
      \quad \cT_{\uptau,-i} = \tytb{\uup\cdots\uup,-,+,\vdots,+,-,\vdots,-}\succ\tytb{+,-}
      \quad \forall i\geq 1.
    \]
    Now the factorization \eqref{eq:d.factor} holds.
    \item $\bfxx_{\uptau}=d$. In this case, $\bfpp_{\uptau} = dd\cdots$
    \[
      \cT_{\uptau,i} = \tytb{\uum\cdots\uum,+} \quad \cT_{\uptau,-i} = \tytb{\uup\cdots\uup,-}
      \quad \forall i \geq 1
    \]
    We have
    \[
      \oAC{\uptau} = \sum_{i=0}^{k}\cB_{\uptau,i}\cdot \cD_{\uptau,i}+\sum_{i=0}^{k}\cB_{\uptau,-i}\cdot \cD_{\uptau,-i}.
    \]
    where
    \[
        \cB_{\uptau,i} = \cT_{\uptau,i}, \quad
        \cD_{\uptau,i} = \cT_{\utau,i}  \quad \forall i = -k, \cdots, -1,1,\cdots, k.\\
    \]
  \end{enumT}
}

%\subsubsection{Summary of the factorization}
% We have the non-vanishing of the lifted local system $\oAC{\uptau}$ and the
% factorization described in
% \Cref{eq:ls.factor} holds.
\begin{lem}
  The local system $\AC(\uptau)\neq 0$.
\end{lem}
\begin{proof}
  Since $x_{\uptaupp}\neq s$, $\AC(\uptaupp)\neq 0$. When
  $\Sign(\tau_{\bftt})\succeq (0,1)$, the first term in \eqref{eq:gd.ls} dose
  not vanish. Now suppose $\Sign(\tau_{\bftt})\nsucc(0,1)$. By \Cref{prop:delta},
  $x_{\uptaupp}=d$, and so $\Lambda_{-}\AC(\uptaupp)\neq 0$ by induction
  hypothesis.
  Since $\ssign(\tau_{\bftt})\succ(2,0)$, the second term in \eqref{eq:gd.ls}
  dose not vanish.
  Hence $\AC( \uptau )\neq 0$ in all
  cases.
\end{proof}

\begin{lem}
  The local system $\AC( \uptau )$ satisfies the following properties:
  \begin{enumT}
    \item When $x_{\uptau} = s$, then $\Lambda_+(\AC(\uptau)) = \Lambda_{-}(\AC(\uptau)) = 0$.
    \item When $x_{\uptau} = r/c$, then $\Lambda_+(\AC(\uptau)) \neq 0$ and $\Lambda_{-}(\AC(\uptau))=0$.
    \item When $x_{\uptau} = d$, then $\Lambda_+(\AC(\uptau))\neq 0$ and $\Lambda_{-}(\AC(\uptau))\neq 0$.
  \end{enumT}
\end{lem}
\begin{proof}
  If $x_{\uptau}=s/r/c$, %  ``$-$'' mark dose not appear as a 1-row in
  % $\oAC{\uptau}$
  $\AC(\uptau)\nsupset 1^{(0,1)}$
  by \eqref{eq:gd.ls}. So $\Lambda_{-}(\AC(\uptau))=0$ in these cases.

  \begin{enumPF}
    \item  Suppose $x_{\uptau}=s$. We have $\Sign(\tau_{\bftt})=(0,2n_{0})$
    where $n_{0}= \abs{\tau_{\bftt}}$.
    Therefore, only the first term in  \eqref{eq:gd.ls} is non-zero and
    $\pcP_{\uptau} = 1^{(0,-(2n_{0}-1))}$. Hence $\Lambda_{+}(\AC(\uptau)) =0$.
    \item Suppose $x_{\uptau} \in \set{r,c}$. If $\ssign(\tau_{\bftt})\succ(1,1)$, the
    first term in \eqref{eq:gd.ls} is non-zero and $\Lambda_+(\AC(\uptau))\succ 1^{(1,0)}$.
    So $\Lambda_+(\AC(\uptau))\neq 0$.
    Now suppose $\ssign(\tau_{\bftt})\nsucc (1,1)$. Then
    $\tau_{\bftt}= r\cdots r$ and $x_{\uptaupp}=d$ by \Cref{prop:CC.bij}.
    Now the second term of \eqref{eq:gd.ls} is non-zero and
    $\ncP_{\uptau}\succ  1^{(1,0)}$. So $\Lambda_{+}\AC(\uptau)\neq 0$.
    \item Suppose $x_{\uptau}=d$. Suppose $\ssign(\tau_{\bftt})\succ (1,2)$.
    Then the first term in \eqref{eq:gd.ls} is non-zero and
    $\Lambda_+(\AC(\uptau))\succ 1^{(1,1)}$. So $\Lambda_+(\AC(\uptau))\neq 0$ and
    $\Lambda_-(\AC(\uptau))\neq 0$.
    Now suppose $\ssign(\tau_{\bftt})\nsucc(1,2)$. Then
    $\tau_{\bftt} = r\cdots rd$
    \footnote{$\tau_{\bftt}=d$ if it has length 1.} and
    $x_{\uptaupp}=d$ by \Cref{prop:CC.bij}.
    So the both terms in \eqref{eq:gd.ls} are non-zero. Since
    $\pcP_{\uptau}\succ 1^{(1,0)}$  and
    $\ncP_{\uptau}\succ 1^{(0,1)}$, we get the conclusion.
  \end{enumPF}
\end{proof}

\subsubsection{Unipotent representations attached to $\cOp$}
In this section, we let $\uptaup\in \PBPesp(\cOp)$ and
$\uptaupp  = \DD(\uptaup)\in \PBPes(\cOpp)$. First note that
$x_{\uptaupp}\neq s$ and so $\Lambda_+(\AC( \uptaupp ))$ always non-zero
by induction hypothesis.

Recall \eqref{eq:LS.taup}.
We claim that we can recover $\Lambda_+(\AC{\uptaupp}$ and
$\nAC{\uptaupp}$ from $\oAC{\uptaup}$:
\begin{lem}\label{c:gd.C1}
  The map $\Omega_{\cOp}\colon \oAC{\uptaup}\mapsto (\Lambda_+(\AC( \uptaupp)),\Lambda_-(\AC( \uptaupp)))$ is a well
  defined map.
\end{lem}
\begin{proof}
  When $\lsign(\oAC{\uptaup})$ has two elements, $x_{\uptaupp}=d$ and
  $\lsign(\oAC{\uptaup}) = \set{(q''_{1},p''_{1}-1),(q''_{1}-1,p''_{1})}$. In
  \eqref{eq:LS.taup}, $\dagger\Lambda_+(\AC{\uptaupp}$ (resp. $\dagger\nAC{\uptaupp}$)
  consists of components whose first column has siginature $(q''_{1},p''_{1}-1)$
  (resp. $(q''_{1}-1,p''_{1})$). When $\lsign(\oAC{\uptaup})$ has only one
  elements, $x_{\uptau}=r/c$,
  $\lsign(\oAC{\uptaup}) = \set{(q''_{1},p''_{1}-1)}$,
  $\oAC{\uptaup} = \dagger\Lambda_+(\AC{\uptaupp}$ and $\nAC{\uptaupp}=0$
  % \trivial[h]{
  %   Note that, $\lsign( \dagger\Lambda_+(\AC{\uptaupp} ) = (q_{0},p_{0}-1)$ and
  %   $\lsign( \dagger\nAC{\uptaupp} ) = (q_{0}-1,p_{0})$ (if it is none-zero) by
  %   \eqref{eq:lsign.1}. }

  In any case, we get
  % recover $\lsign(\oAC{\uptaupp})=(p_{0},q_{0})$ and
  \begin{equation}\label{eq:uptaupp.sign}
    \ssign(\uptaupp) = (n_{0},n_{0})+(p''_{1},q''_{1})\quad \text{where }
    2n_{0} = \abs{\DD(\cOpp)}.
  \end{equation}
% $+\half \abs{\DD(\cOpp)} \cdot(1,1)$.
  Using the signature $\ssign(\uptaupp)$, we could recover the twisting
  characters in the theta lifting of local system. Therefore
  $\Lambda_+(\AC{\uptaupp}$ and $\nAC{\uptaupp}$ can be recovered from  $\oAC{\uptaup}$.
\end{proof}


\begin{lem}\label{c:gd.C3}
  Suppose $\uptaup_{1}\neq \uptaup_{2}\in \drc(\cOp)$. Then
  $\uppi_{\uptaup_{1}}\neq \uppi_{\uptaup_{2}}$.
\end{lem}
\begin{proof}
  It suffice to consider the case when $\oAC{\uptaup_{1}} = \oAC{\uptaup_{2}}$.
 By \eqref{eq:uptaupp.sign} in the proof of \Cref{c:gd.C1},
 $\ssign(\uptaupp_{1}) = \ssign(\uptaupp_{2})$.
 On the other hand, $\upepsilon_{1}=\upepsilon_{2}=0$ and
 $\uptaupp_{1}\neq \uptaupp_{2}$ by \Cref{lem:gd.CD}. So
 \[
   \uppi_{\uptaup_{1}} = \Thetab(\uppi_{\uptaupp_{1}})\neq\Thetab(\uppi_{\uptaupp_{2}}) = \uppi_{\uptaup_{2}}
 \]
  by the injectivity of theta lift.
\end{proof}


%\medskip

% When $\bfpp_{\uptau} = \tytb{sd\cdots,\vdots,s,{c/d}}$,
% $\Lambda_+(\AC(\uptau)) = (\dagger\dagger \cB_{\uptaupp,1} \cT_{\uptau,1})^{+}$
% and this is the only case that $\Lambda_+(\AC(\uptau))\neq \emptyset$ has less irreducible
% components than that of $\oAC{\uptau}$.

% {
%   \color{red}
%   The above computation also tells the shape of legs of $\oAC{\uptau}$ according
%   to $\bfpp_{\uptau}$.
% }

% , $\cOpp$ is noticed when
% $C_{2k}$ is even or $\cOpp$ is +notice when $\cO_{2k}$ is odd.

\subsubsection{Unipotent representations attached to $\cO$}

\begin{lem}\label{c:gd.D1}
  Suppose $\uptau_{1}\neq \uptau_{2}\in \drc(\cO)$. Then $\uppi_{\uptau_{1}}\neq \uppi_{\uptau_{2}}$.
\end{lem}
\begin{proof}
 It suffice to consider the case that $\oAC{\uptau_{1}}=\oAC{\uptau_{2}}$.
 Clearly, $\ssign(\uptau_{1})=\ssign(\uptau_{2})$. On the other hand,
 the twisting $\upepsilon_{\uptau_{1}}=\upepsilon_{\uptau_{2}}$ since it is
 determined by the  local system:
 $\upepsilon_{\uptau_{i}}=0$ if and only if $\oAC{\uptau_{i}}\supset \tytb{-}$.
 %(see \eqref{eq:gd.ls}).

 We conclude that $\uptaupp_{1}\neq \uptaupp_{2}$ and $\uptaup_{1}\neq \uptaup_{2}$
 by \Cref{lem:gd.inj} and \Cref{lem:gd.CD}.
 By \Cref{c:gd.C3} and the injectivity of theta lift,
 \[
   \uppi_{\uptau_{1}} = \Thetab(\pi_{\uptaup_{1}})\otimes (\bfone^{+,-})^{\upepsilon_{\uptau_{1}}}
  \neq \Thetab(\pi_{\uptaup_{2}})\otimes (\bfone^{+,-})^{\upepsilon_{\uptau_{2}}} = \uppi_{\uptau_{2}}
 \]
\end{proof}


\subsubsection{Noticed orbits}



In this section, we prove the claims about noticed and +noticed orbits.

\begin{lem}\label{c:gd.C2}
  Suppose $\cOp$ is noticed. Then
  \begin{enumT}
    \item \label{c:gd.C2.1}The map
$\set{\oAC{\uptaupp}| x_{\uptaupp}\neq s}\longrightarrow \set{\oAC{\uptaup}}=\LLS(\cOp)$ given by
$\oAC{\uptaupp}\mapsto \oAC{\uptaup}=\vartheta(\oAC{\uptaupp})$ is a bijection.
\item
The map $\drc(\cOp)\rightarrow \LLS(\cOp)$ given by $\uptaup\mapsto \oAC{\uptaup}$ is a bijection.
\end{enumT}
\end{lem}
\begin{proof} %Recall \eqref{eq:LS.taup}.
  \begin{enumPF}
    \item
  Note that $\cOpp$ is noticed by definition. Hence $\Upsilon_{\cOpp}$ is invertible.
  % Let $\uptaup\in \drc(\cOp)$.
  % By  and the induction hypothesis , we have a well defined map
  % $\oAC{\uptaup}\mapsto \Lambda_+(\AC{\uptaupp}$ with $\Lambda_+(\AC{\uptaupp}\neq 0$.
  % Since $\cOpp$ is +noticed, $\Lambda_+(\AC{\uptaupp}$ uniquely determines the local
  % system $\oAC{\uptaupp}$ by inductive hypothesis. Therefore,
  %
  Recall \Cref{c:gd.C1}, we see that
  \[
    (\Upsilon_{\cOpp})^{-1}\circ \Omega_{\cOp}\colon
    \oAC{\uptaup}\mapsto (\Lambda_+(\AC{\uptaupp},\nAC{\uptaupp})\mapsto \oAC{\uptaupp}
  \]
  gives the inverse of the map in the claim.
  \item
  Suppose $\uptaup_{1}\neq \uptaup_{2}\in \drc(\cOp)$. By \Cref{lem:gd.CD},
  $\uptaupp_{1}\neq \uptaupp_{2}$. Hence
  $\oAC{\uptaupp_{1}}\neq\oAC{\uptaupp_{2}}$ by the induction hypothesis.
  Therefore, $\oAC{\uptaup_{1}}\neq \oAC{\uptaup_{2}}$ by \ref{c:gd.C2.1}
  of the claim.
\end{enumPF}
\end{proof}

\begin{lem}\label{c:noticed.bij}
 Suppose $\cO$ is noticed, $\cL \colon \drc(\cO)\mapsto \LLS(\cO)$ given by
 $\uptau\mapsto \oAC{\uptau}$ is bijective.
% Moreover, $\oAC{\uptau_{1}}=\oAC{\uptau_{2}}$ implies $\oAC{\uptaup_{1}}=\oAC{\uptaup_{2}}$.
\end{lem}
\begin{proof}
We prove the claim by contradiction. We assume $\uptau_{1}\neq \uptau_{2}$ such
that $\oAC{\uptau_{1}}=\oAC{\uptau_{2}}$.


Recall \eqref{eq:gd.ls}: there is an pair of non-negative integers
$(e_{i},f_{i})=\ssign(\bfuu_{\uptau_{i}})$ such that
\[
  \oAC{\uptau_{i}} = \dagger\dagger \Lambda_+(\AC{\uptaupp_{i}}\cdot \pcP_{\uptau_{i}}
  + \dagger\dagger \nAC{\uptaupp} \cdot \ncP_{\uptau_{i}}.
\]

\begin{enumPF}
  \item
Suppose that $\oAC{\uptau_{i}}$ contains a 1-row marked by $-$ or $=$. Then we
have
%\begin{enumI}
%  \item
$\upepsilon_{\uptau_{1}}=\upepsilon_{\uptau_{2}}$, which can be read from the mark $-/=$.
Moreover, the term
$\dagger\dagger\Lambda_+(\AC{\uptaupp_{i}}\cdot \pcP_{\uptau_{i}}\neq 0$ and we can
recover it from $\oAC{\uptau_{i}}$. So $\Lambda_+(\AC{\uptaupp_{1}} = \Lambda_+(\AC{\uptaupp_{2}}$.
Since $\cOpp$ is +noticed, $\uptaupp_{1}=\uptaupp_{2}$ by
\eqref{c:gd.pnoticed} and the induction hypothesis.
Note that $\ssign(\uptau_{1})=\ssign(\uptau_{2})$, we get
$\uptau_{1}=\uptau_{2}$ by \Cref{lem:gd.inj} which is contradict to our assumption.

% as the parts.. must contains the term
 % where $\pcP_{i}$ and $\ncP_{i}$ are columns of sign $(p_{i},q_{i}-1)$ and
 % $(p_{i}-1,q_{i})$ respectively.
 % Here we adapt the convention that $\pcP_{i} = \emptyset$ (resp. $\ncP_{i}$) if
 % $q_{i}-1<0$ (resp. $p_{i}-1<0$).

% Therefore, we have $\oAC{\uptaup_{i}}= \DD(\oAC{\uptau_{i}}) $ if $\oAC{\uptau}$ contains a 1-row
% of $+$ sign and a 1-row of $-$ sign.
\item
Now we assume that $\oAC{\uptau_{i}}$ does not contains a 1-row marked by $-/=$.
By \eqref{eq:gd.ls}, we conclude that $x_{\uptau}\neq d$ and
$\tau_{\bftt} = r\cdots r$ or $ r\cdots rc$. Therefore $\bfpp_{\uptau_{i}}$
must be one of the following form by \Cref{lem:u}:
% The exceptioinal cases are:
% \begin{enumT}
%   \item $(p_{i},q_{i}) = (2a,0)$ and so $\pcP_{i}=\emptyset$.
%   In this case, $\set{\bfpp_{\uptau_{i}}}$ must consists of the following three elements
  \[
    \bfpp_{1} =\tytb{rc,\vdots,r,r}, \quad \bfpp_{2}=\tytb{rc,\vdots,r,c}, \quad\text{and}
    \quad\bfpp_{3}= \tytb{rd,\vdots,r,r}\\
  \]
  We consider case by case according to the set $\set{\bfpp_{\uptau_{1}},\bfpp_{\uptau_{2}}}$:
  \begin{enumPF}
    \item Suppose
    $\set{\bfpp_{\uptau_{1}},\bfpp_{\uptau_{2}}}=\set{\bfpp_{1}}$ or $ \set{\bfpp_{2}}$.
    In these cases, $\upepsilon_{\uptau_{1}}=\upepsilon_{\uptau_{2}}$ and
    $\uptaupp_{1}\neq \uptaupp_{2}$ by \Cref{lem:gd.inj}.
    On the other hand, $\Lambda_+(\AC{\uptaupp_{1}}=\Lambda_+(\AC{\uptaupp_{2}}$ by \eqref{eq:gd.ls}.
    Since $\cOpp$ is +noticed, we get $\uptaupp_{1}=\uptaupp_{2}$ a contradiction.
    \item Suppose
    $\set{\bfpp_{\uptau_{1}},\bfpp_{\uptau_{2}}}=\set{\bfpp_{3}}$.
    We still have $\upepsilon_{\uptau_{1}}=\upepsilon_{\uptau_{2}}$ and
    $\uptaupp_{1}\neq \uptaupp_{2}$ by \Cref{lem:gd.inj}.
    On the other hand, $\nAC{\uptaupp_{1}}=\nAC{\uptaupp_{2}}$ by \eqref{eq:gd.ls}.
    This again contradict to the injectivity of $\oAC{\uptaupp}\mapsto \nAC{\uptaupp}$.
    %Since $\cOpp$ is +noticed, we get $\uptaupp_{1}=\uptaupp_{2}$ a contradiction.
    \item  Suppose
    $\set{\bfpp_{\uptau_{1}},\bfpp_{\uptau_{2}}} = \set{\bfpp_{1},\bfpp_{2}}$.
    By the argument above, we have $\Lambda_+(\AC{\uptaupp_{1}}=\Lambda_+(\AC{\uptaupp_{2}}$
    Now $\uptaupp_{1}\neq \uptaupp_{2}$ since $\set{x_{\uptaupp_{1}},x_{\uptaupp_{2}}}=\set{r,c}$.
    This contradict to that $\cOpp$ is +noticed again.
    \item\label{it:c:noticed.bij.4} Suppose $\bfpp_{\uptau_{1}} = \bfpp_{1} \text{ or } \bfpp_{2}, \bfpp_{\uptau_{2}}=\bfpp_{3}$.
    Then
   \[
    \oAC{\uptau_{1}} = \dagger\maltese^{\frac{\abs{\bfuu_{\uptau_{1}}}}{2}}\dagger\Lambda_+(\AC{\uptaupp_{1}}\cdot \pcP_{\uptaupp_{1}}
    \quad \text{and} \quad
    \oAC{\uptau_{2}} = \dagger\maltese^{\frac{\abs{\bfuu_{\uptau_{2}}}}{2}}\dagger\nAC{\uptaupp_{2}}\cdot \ncP_{\uptaupp_{2}}.
    \]
    This implies
    $\abs{\ssign(\uptaupp_{1})} \equiv \abs{\ssign(\uptaupp_{2})}+2 \pmod{4}$
    and
    $\maltese \dagger \Lambda_+(\AC{\uptaupp_{1}}= \dagger\nAC{\uptaupp_{2}}$.


    \begin{claim}\label{c:d+}
      We have $\nAC{\uptaupp_{2}}\supset \tytb{+}$.
    \end{claim}
    \begin{proof}
      If $\cOpp$ is obtained by the usual descent, we have
      $\oAC{\uptau}\succeq \tytb{-,+}$ and we are done.

      Now assume $\cOpp$ is obtained by generalized descent and $\cOpp$ is
      +noticed, i.e. $\bfuu_{\uptaupp}$ has at least length $2$.
      Suppose $\ssign( \bfuu_{\uptaupp} )\succ (1,2)$. Applying \eqref{eq:gd.ls} to $\uptaupp$,
      we see that the first term is non-zero and
      $\oAC{\uptaupp}\supseteq \dagger_{(1,1)}$.

      Otherwise, $\ssign( \bfuu_{\uptaupp} ) = (2n_{0}-1,1)\succeq (3,1)$ where $n_{0}$ is the
      length of $\bfuu_{\uptaupp}$. By \eqref{eq:gd.ls}, the second term is
      non-zero and $\oAC{\uptaupp}\succ \dagger_{(2n_{0}-1,1)}\succ +$.
    \end{proof}
    % Note that $\cOpp$ is a +noticed orbit. Then
    % By the list in \Cref{sec:z.r,sec:z.c,sec:z.d}, we see that
    % $\oAC{\uptaupp_{2}}\supset \tytb{-,+}$ and
    % $\nAC{\uptaupp_{2}}\supset \tytb{+}$.

    The claim leads to a contradiction: Thanks to the character twist in the theta
    lifting formula of the local system, we see that the the associated
    character restricted on the 2-row $\tytb{-+}$ of $\maltese \dagger \Lambda_+(\AC{\uptaupp_{1}}$ and
    $\dagger \nAC{\uptaupp_{2}}$ must be are different, a contradiction.

    % Note that  $\oAC{\uptau_{2}}\succ \dagger_{(1,1)}$ implies that
    % $\nAC{\uptau_{2}}\succ \dagger_{(1,0)}$. So $\dagger\nAC{\uptau_{2}}$ has
    % a 2-row of sign $\tytb{-+}$.
    % On the other hand, $\oAC{\uptau_{1}}=\oAC{\uptau_{2}}$ implies
    % $\dagger \Lambda_+(\AC{\uptau_{1}}=\Lambda_+(\AC{\uptaup_{1}} = \nAC{\uptaup_{2}}= \dagger\nAC{\uptau_{2}}$. So
    % $\abs{\ssign(\uptaupp_{1})} \equiv \abs{\ssign(\uptaupp_{2})}+2 \pmod{4}$.
    % , i.e.
    % \[\oAC{\uptau_{1}}=\dagger \Lambda_+(\AC{\uptaupp_{1}} \cdot \dagger_{(2a-1,0)}
    %   \neq \ddagger\nAC{\uptaup_{2}}\cdot \ddagger_{(2a-1,0)} = \oAC{\uptau_{2}}.
    % \]
    % \item $\bfpp_{\uptau_{1}} = \bfpp_{2},\bfpp_{\uptau_{2}}=\bfpp_{3}$.
    % On argue by the same way as the case \eqref{} and yield a contRadiction.
  \end{enumPF}
%   \item $(p_{i},q_{i}) = (0,2a)$ and so $\ncP_{i}=\emptyset$.
%   In this case, $\set{\bfpp_{\uptau_{i}}}$ must consists of two the following three elements
%   \[
%     \bfpp_{1} =\tytb{sc,\vdots,s}, \quad \bfpp_{2}=\tytb{sr,\vdots,s}, \quad\text{and}
%     \quad\bfpp_{3}= \tytb{sd,\vdots,s}\\
%   \]
%   This case is also easy.
%   $ \uptau_{1}\neq \uptau_{2} $ implies $\uptaupp_{1}\neq \uptaupp_{2}$.
%   But $\oAC{\uptau_{1}}=\oAC{\uptau_{2}}$ implies
%   $\oAC{\uptaupp_{1}} = \DD^{2}(\oAC{\uptau_{1}})\cdot + \neq \DD^{2}(\oAC{\uptau_{2}})\cdot + = \oAC{\uptaupp_{2}}$.
%   This contradict to the bijection between local systems with dot-r-c diagrams
%   for the +noticed orbit $\cOpp$.
%   %$\uptaupp_{1} = \DD^{2}(\uptau_{1})\neq \DD^{2}(\uptau_{2})  = \uptaupp_{2}$
% \end{enumT}
% \end{enumI}
\end{enumPF}
  We finished the proof of the lemma.
\end{proof}

\subsubsection{The maps $\pUpsilon_{\cO}$, $\nUpsilon_{\cO}$ and $\Upsilon_{\cO}$}

\begin{lem}\label{c:gd.pnoticed}\label{c:gd.pnoticed.p}
    Suppose $\cO$ is +noticed.
    The map
    $\pUpsilon_{\cO} \colon \set{\oAC{\uptau}|\Lambda_+(\AC(\uptau))\neq 0}\rightarrow \set{\Lambda_+(\AC(\uptau))\neq 0}$
    is injective.
\end{lem}
\begin{proof}
    %First assume $\cO$ is +noticed.
    We have two cases:
    \begin{enumPF}
      \item $\oAC{\uptau}$ contain an irreducible component $\succ \tytb{+,+}$.
      This is equivalent to $\Lambda_+(\AC(\uptau))$ has an irreducible component
      $\succ\tytb{+}$. Now all irreducible components of
      $\oAC{\uptau} \succ \tytb{+}$ and $\oAC{\uptau}\mapsto \Lambda_+(\AC(\uptau))$ will
      not kill any irreducible components. Hence we could recover $\oAC{\uptau}$
      from $\Lambda_+(\AC(\uptau))$.%, and the map $\Upsilon_{1}$.
      \item Suppose that $\oAC{\uptau_{1}}\neq \oAC{\uptau_{2}}$ do not contain
      a component $\succ \tytb{+,+}$.

      By \Cref{c:noticed.bij}
      $\uptau_{1}\neq \uptau_{2}$.\footnote{$\oAC{\uptau_{1}}\neq \oAC{\uptau_{2}}$
        clearly implies $\uptau_{1}\neq \uptau_{2}$.} We now show that
      $\Lambda_+(\AC{\uptau_{1}}=\Lambda_+(\AC{\uptau_{2}}\neq 0$ leads to a contradiction.
      Clearly, $\ssign(\uptau_{1})=\ssign(\uptau_{2})$.
      By \eqref{eq:gd.ls} and checking the properties in \Cref{lem:u}, we have
      \[
        \bfpp_{\uptau_{i}} = \tytb{s{x_{\uptaupp}},\vdots,s,{x_{\uptau_{i}}}} \quad \text{
          where }x_{\uptaupp_{i}}=c/d, x_{\uptau_{i}}=c/d.
      \]
      % By looking at \eqref{eq:ped.ssc} and \eqref{eq:ped.ssd},
      On the other hand, when $\bfpp_{\uptau_{i}}$ have the above form, the
      first term of \eqref{eq:gd.ls} is non-zero. So we conclude that
      $\Lambda_+(\AC{\uptau_{1}}=\Lambda_+(\AC{\uptau_{2}}$ implies
      \begin{equation}\label{eq:pnoticed.1}
        \Lambda_+(\AC{\uptaupp_{1}}=\Lambda_+(\AC{\uptaupp_{2}}.
      \end{equation}

      Moreover, we see that every components of $\oAC{\uptau_{i}}$ has a 1-row
      of mark ``$-/=$'' and we can determine $\upepsilon_{\uptau_{i}}$ by the
      mark $-/=$ of the 1-row appeared in $\Lambda_+(\AC{\uptau_{i}}$. In particular, we
      have $\upepsilon_{1}=\upepsilon_{2}$. Now \Cref{lem:gd.inj} implies
      $\uptaupp_{1}\neq \uptaupp_{2}$. This is contradict to
      \eqref{eq:pnoticed.1} and our induction hypothesis since $\cOpp$ is also
      +noticed.

      % On the other hand, $\cOpp$ is also +noticed. But \eqref{eq:ped.ssc} and
      % \eqref{eq:ped.ssd} also imply that
      % $\Lambda_+(\AC{\uptaupp_{1}}=\Lambda_+(\AC{\uptaupp_{2}}$ which is a contradiction to
      % our
      % induction hypothesis.
      % % By our descent algorithm $\eDD$, we have
      % % $\uptaupp_{1}\neq \uptaupp_{2}$.
      % % Note that the collision implies
      % % $\Lambda_+(\AC{\uptaupp_{1}} =\Lambda_+(\AC{\uptaupp_{2}}$,
      % % which is impossible by induction.
    \end{enumPF}
% \item
%  Suppose $\cO$ is noticed, we many need to consider the additional case where
%  $\bfpp_{\uptau} = \tytb{{x_{1}} y}$ where $x_{1}= c/d$. But $x_{1}=d$ is
%  equivalent to the non-zero of $\nAC{\uptau}$.
% \end{enumPF}
\end{proof}


\begin{lem}\label{c:gd.noticed.inj}
  Suppose $\cO$ is noticed. Then the map $\Upsilon_{\cO}\colon \oAC{\uptau}\mapsto (\Lambda_+(\AC(\uptau)),\nAC{\uptau})$ is injective.
\end{lem}
\begin{proof}
  It suffice to consider the case where $\cO$ is noticed but not +noticed, i.e.
  $C_{2k+1}=C_{2k}+1$. In this case, $n=1$.
  The peduncle $\bfpp_{\uptau}$ have four possible cases:
  \[
    \tytb{rc}, \quad \tytb{cc}, \quad \tytb{cd},\quad \text{or}\quad  \tytb{dd}.
  \]
  By \eqref{eq:gd.ls}, $\pAC{\uptau} = \dagger\dagger\pAC{\uptaupp}$.
  % \[
  %   \bfpp_{1} = \tytb{rc}, \bfpp_{2}=\tytb{cc}, \bfpp_{3} = \tytb{cd}, \bfpp_{4} = \tytb{dd}.
  % \]

  Now suppose $\uptau_{1}\neq \uptau_{2}$ such that
  $\pAC{\uptau_{1}}=\pAC{\uptau_{2}}$.
  Since $\cOpp$ is +noticed, we have $\uptaupp_{1}=\uptaupp_{2}$ by the induction
  hypothesis (see \Cref{c:gd.pnoticed.p}).
  By \Cref{lem:gd.inj}, this only happens when
  $\set{\bfpp_{\uptau_{1}},\bfpp_{\uptau_{2}}}=\set{\tytb{cd},\tytb{dd}}$.
  Since one of $\nAC{\uptau_{i}}$ is zero and the other is non-zero, we
  conclude that $\Upsilon_{\cO}(\oAC{\uptau_{1}})\neq \Upsilon_{\cO}(\oAC{\uptau_{2}})$.
\end{proof}

\begin{lem}\label{c:gd.pnoticed.n}
    Suppose $\cO$ is +noticed.
    The map $\nUpsilon_{\cO}\colon \set{\oAC{\uptau}|\nAC{\uptau}\neq 0}\rightarrow \set{\nAC{\uptau}\neq 0}$
    is injective.
\end{lem}
\begin{proof}
  The proof is similar to that of \Cref{c:gd.pnoticed.p}.
    We have two cases:
    \begin{enumPF}
      \item $\oAC{\uptau}$ contain an irreducible component $\succ \tytb{-,-}$.
      This is equivalent to $\nAC{\uptau}$ has an irreducible component
      $\succ\tytb{-}$ and $\oAC{\uptau}\mapsto \nAC{\uptau}$ will
      not kill any irreducible components. Hence we could recover $\oAC{\uptau}$
      from $\nAC{\uptau}$.%, and the map $\Upsilon_{1}$.

      \item Now we make a weaker assumption that $\uptau_{1}\neq \uptau_{2}\in \drc(\cO)$ such that
      $\oAC{\uptau_{1}}$ and $\oAC{\uptau_{2}}$ do not contain a component
      $\succ \tytb{-,-}$.
      % We now show that  $\nAC{\uptau_{1}}=\nAC{\uptau_{2}}\neq 0$ leads to a
      % contradiction.
      % Clearly,
      % $\ssign(\uptau_{1})=\ssign(\uptau_{2})$

      By \eqref{eq:gd.ls} and the properties in \Cref{lem:u}, we
      see that $\bfpp_{\uptau_{i}}$ must be one of the following
      \[
        \bfpp_{1}=  \tytb{rc,\vdots,r,d} \quad \text{ or } \quad \bfpp_{2}= \tytb{rd,\vdots,r,d}.
      \]
      Without of loss of generality, we have the following possibilities:
      \begin{enumPF}
        \item $\bfpp_{\uptau_{1}}=\bfpp_{\uptau_{2}}=\bfpp_{1}$. We have
        $\pAC{\uptaupp_{1}}=\pAC{\uptaupp_{2}}$ and obtain the contradiction.
        \item $\bfpp_{\uptau_{1}}=\bfpp_{\uptau_{2}}=\bfpp_{2}$. We have
        $\nAC{\uptaupp_{1}}=\nAC{\uptaupp_{2}}$ and obtain the contradiction.
        \item $\bfpp_{\uptau_{1}}=\bfpp_{1}$ and $\bfpp_{\uptau_{2}}=\bfpp_{2}$.
        Now we apply the same argument in the case~\ref{it:c:noticed.bij.4} of
        the proof of \Cref{c:noticed.bij}. We get
        $\maltese \dagger\pAC{\uptaupp_{1}}=\dagger \nAC{\uptaupp_{2}}$,
        $\nAC{\uptaupp_{2}} \supset +$ and a contradiction by looking at the
        associated character on the 2-row $\tytb{-+}$ .
      \end{enumPF}
    \end{enumPF}
    This finished the proof.
\end{proof}


\end{document}


%%% Local Variables:
%%% coding: utf-8
%%% mode: latex
%%% TeX-master: "ssunip.tex"
%%% TeX-engine: xetex
%%% ispell-local-dictionary: "en_US"
%%% End:
