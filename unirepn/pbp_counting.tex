\documentclass[unipcounting]{subfiles}

\begin{document}

\section{}

\begin{table}[hpb]
\[
\begin{array}{c|c|cc|c}
  & \tau & \tau^{s} & \tau^{ns} \\
  \hline
           & \ytb{{x'_{0}}{\ast},{x'_{1}}{x'_{2}},\none,\none,\none,\none}
             \times \ytb{\ast,\none,\none,\none,\none,\none}
           &\ytb{{x_{0}}{\ast},{x_{1}}{x_{2}},\none,\none,\none,\none}
    \times\ytb{{\ast}{\ast},{x_{3}},{*(srcol)s},{*(srcol)\vdots},{*(srcol)s},\none}&
  \ytb{{y_{0}}{\ast},{*(srcol)r}{y_{2}},{*(srcol)\vdots},{*(srcol)r},{y_{1}},{y_{3}} }
    \times\ytb{{\ast}\ast,\none,\none,\none,\none,\none}\\
  \hline
  \ytb{{x'_{1}}{=}{s},{\text{then}},{z_{0}}{=}{\emptyset/}{s},{x_{0}}{=}{\emptyset/}{\bullet},{x_{1}}{=}{\bullet},
  {x_{2}}{=}{x'_{2}}}
  &
            \ytb{{x'_{0}}{\ast},{s}{x_{2}},\none,\none,\none,\none}
             \times \ytb{\ast,\none,\none,\none,\none,\none}
                &
                \ytb{{x_{0}}{\ast},{\bullet}{x_{2}},\none,\none,\none,\none}
    \times\ytb{{\ast}\ast,{\bullet},{*(srcol)s},{*(srcol)\vdots},{*(srcol)s},\none}&
  \ytb{{x_{0}}{\ast},{*(srcol)r}{x_{2}},{*(srcol)\vdots},{*(srcol)r},{c},{d} }
                          \times\ytb{{\ast}{\ast},\none,\none,\none,\none,\none}
          &x_{2}\neq r
  \\
  \cline{3-5} & &
                  \ytb{{x_{0}}{\ast},{\bullet}{r},\none,\none,\none,\none}
    \times\ytb{{\ast}\ast,{\bullet},{*(srcol)s},{*(srcol)\vdots},{*(srcol)s},\none}&
  \ytb{{x_{0}}{\ast},{*(srcol)r}{c},{*(srcol)\vdots},{*(srcol)r},{r},{d} }
    \times\ytb{{\ast}\ast,\none,\none,\none,\none,\none} & x_{2}=r \\
  \hline
\ytb{{x'_{1}}{\neq}{s},{\text{then}},{x_{1}}{=}{x'_{1}},{x_{2}}{=}{x'_{2}}}
  &
            \ytb{{x'_{0}}{\ast},{x'_{1}}{x_{2}},\none,\none,\none,\none}
             \times \ytb{\ast,\none,\none,\none,\none,\none}
   &
    \ytb{{x_{0}}{\ast},{x_{1}}{x_{2}},\none,\none,\none,\none}
    \times\ytb{{\ast}\ast,{s},{*(srcol)s},{*(srcol)\vdots},{*(srcol)s},\none}&
  \ytb{{x_{0}}{\ast},{*(srcol)r}{x_{2}},{*(srcol)\vdots},{*(srcol)r},{r},{x_{1}} }
       \times\ytb{{\ast}\ast,\none,\none,\none,\none,\none} &
     \ytb{{x'_{0}}{\neq}{c},{\text{then}},{x'_{0}}{=}{\emptyset/}{s/}{r},{x_{0}}{=}{\emptyset/}{\bullet/}{r}}
  \\
  %\hline
  \cline{3-5}
   & &
    \ytb{{c}{\ast},{d}{x_{2}},\none,\none,\none,\none}
    \times\ytb{{\ast}\ast,{s},{*(srcol)s},{*(srcol)\vdots},{*(srcol)s},\none}&
  \ytb{{r}{\ast},{*(srcol)r}{x_{2}},{*(srcol)\vdots},{*(srcol)r},{c},{d} }
  \times\ytb{{\ast}\ast,\none,\none,\none,\none,\none}&
  \ytb{{x'_{0}}{=}{c}\none\none,{\text{then}},{x_{0}}{=}{x'_{0}}{=}{c},{x_{1}}{=}{x'_{1}}{=}{d},{x_{2}}{=}{x'_{2}}{=}{\emptyset/d}}
  \\
  \hline
  \hline
\end{array}
\]
\caption{``special-non-special'' switch}
\label{tab:nonsp.C}
\end{table}


\def\PBPs{\PBP_\star}

\section{Special non-special switch}

\subsubsection{The case of $\star=\wtC$}
Suppose that $\star=\wtC$.
\def\dsrcd{\set{\bullet,s,r,c,d}}
We define an involution $\natural$ on the set $\dsrcd$  by 
sending $\bullet,s,r,c,d$ to $\bullet, r,s,d,c$ respectively. 

Let $\wp\subset \PP_\star(\ckcO)$ such that $\wp\neq \emptyset$. 
Pick a pair $(2k-1,2k)\in \wp$, and let $\wp' = \wp - \set{(2k-1,2k)}$.
\begin{lem}\label{lem:sp-nsp.C}
For each $\tau'\in \PBPs(\ckcO,\wp')$, there is a unique $\tau\in\PBPs(\ckcO,\wp)$
such that 
\begin{equation}\label{eq:sp-nsp.tC.def}
\begin{split}
 \cP_\tau(i,j) & := \begin{cases}
    \natural(\cQ_{\tau'}(i,j) ) & \text{if } i = k \\
    \cP_{\tau'}(i,j) & \text{otherwise} 
 \end{cases} \\ %\quad \text{and}\\
 \cQ_\tau(i,j)&  := \begin{cases}
    \natural(\cP_{\tau'}(i,j) ) & \text{if } i = k \\
    \cQ_{\tau'}(i,j) & \text{otherwise} 
 \end{cases}
 \end{split}
\end{equation}
Moreover, the map $\tau'\mapsto \tau$ yields a bijection
\begin{equation}\label{eq:sp-nsp.tC.bij}
    \PBP_\star(\ckcO,\wp') \longrightarrow \PBPs(\ckcO,\wp). 
\end{equation}
\end{lem}
\begin{proof}
One first variety that $(\imath(\ckcO,\wp), \cP_\tau)
\times (\jmath(\ckcO,\wp),\cQ_\tau)\times \wtC$ is a valid painted bi-partition. 
We can define the inverse of \eqref{eq:sp-nsp.tC.bij} by \eqref{eq:sp-nsp.tC.def} (switching the role of $\tau$ and $\tau'$). 
This proves the bijectivity of \eqref{eq:sp-nsp.tC.bij}. 
\end{proof}
\trivial{
We now verify $\tau$ is a valid painted bipartition. 

The relevant portion of $\cP$ and $\cQ$ are boxes with indexes $(i,j)$ such that 
$k-1\leq i \leq k$ and 
$\frac{\bfrr_{2k}(\ckcO)}{2}\leq j\leq \frac{\bfrr_{2k-1}(\ckcO)}{2}$
\begin{equation}\label{legtau11}
\tau': \ \ \ 
 \nytb{{w_1}{x_2},{\ast}\none,{\ast}\none ,{\ast}\none ,{\ast}\none ,{w_2}\none}
    \times
  \nytb{{w_3}{x_{0}},{\ast}{*(srcol)r},{\ast}{*(srcol)\vdots},{\ast}{*(srcol){\vdots}},
  {\ast}{*(srcol)r},{w_4}{x_{1}} }
    \hspace{8em}
    \bartau: \ \ \ 
 \nytb{{w_5}{y_{0}},{\ast}{*(srcol)s},{\ast}{*(srcol)\vdots},{\ast}{*(srcol){\vdots}},
  {\ast}{*(srcol)s},{w_6}{y_{1}} } 
    \times
    \nytb{{w_7}{y_2},{\ast}\none,{\ast}\none ,{\ast}\none ,{\ast}\none ,{w_8}\none},
\end{equation}
 Either all $w_i$  are $\emptyset$ ($\star$ are all $\emptyset$) 
or all none-empty ($\star$ are all  $\bullet$). 
When $w_i$ are all $\emptyset$ or $\bullet$, the validity of $\tau$ is easy.  

For the rest cases, $w_1,W_3, w_5,w_7$ must be painted by $\bullet$. 
Therefore, 
it suffice to consider the switch 
\[
\nytb{{w_2}\none}\times \nytb{{w_4}{x_1}}
\longleftrightarrow
\nytb{{w_6}{y_1}}\times \nytb{{w_8}{\none}}.
\]

We can list all the cases:
\[
\begin{array}{c|c}
\hline
\hline
\tau' & \tau\\
\hline
   \nytb{s\none} \times \nytb{rd}  &  
   \nytb{ sc} \times \nytb{r\none}    \\
   \hline
   \nytb{s\none} \times \nytb{dd}  &  
   \nytb{ sc} \times \nytb{d\none}    \\
   \hline
   \nytb{c\none} \times \nytb{rd}  &  
   \nytb{ cc} \times \nytb{r\none}    \\
   \hline
   \nytb{c\none} \times \nytb{dd}  &  
   \nytb{ cc} \times \nytb{d\none}    \\
   \hline
   \hline
\end{array}
\]
}

\begin{lem}\label{lem:DD.bij.BD}
Suppose $\star\in \set{B,D}$ and $\wp \in \PP_\star(\ckcO)$. 
Let $\ckcO_1$ be the $\star'$-good orbit having the rows 
\[
\bfrr_1(\ckcO_1) := \bfrr_1(\ckcO)+2, \quad
\text{and} \quad \bfrr_{i+1}(\ckcO_1):= \bfrr_i(\ckcO)
\forall i=1,2, \cdots
\]
Let 
\[
\wp_1 = \set{(i+1,i+2)| (i,i+1)\in \wp}. 
\]
Then the naive descent map yields a bijection
\[
\PBP_{\star'}(\ckcO_1,\wp_1) \xrightarrow{\ \ \DDn\ \ }
\PBP_{\star'}(\ckcO,\wp)
\]
\end{lem}
\begin{proof}
Clear by the definition of $\DDn$. 
\end{proof}


\begin{prop}
Suppose $\star \in \set{B,C,\wtC,D}$. 
For $\wp\subset \PP_\star(\ckcO)$, 
$\abs{\PBPs(\ckcO,\wp)} = \abs{\PBPs(\ckcO,\emptyset)}$
\end{prop}
\begin{proof}
Suppose $\star \in \set{C,\wtC}$. 
The proposition following 
by applying \Cref{lem:sp-nsp.C} finitely many times to reduce $\wp$ to $\emptyset$. 

Suppose $\star \in \set{B,D}$. By \Cref{lem:DD.bij.BD},  the problem translate to $\star'\in \set{C,\wtC}$ which we already proved.   
\end{proof}

Then $\bfrr_1(\ckcO)>\bfrr_2(\ckcO) \geq 0$. 
%Write
%\[
%       l:= l_{\star,\ckcO} :=  \begin{cases} \frac{\bfrr_2(\ckcO)-2}{2},
%  & \textrm{if $\bfrr_2(\ckcO)\geq 2$}; \smallskip \\
%  0, & \textrm{if $\bfrr_2(\ckcO)=0$}. 
%  \end{cases}
%  \]


For every  $\tau= (\imath,\cP)\times(\jmath,\cQ)\times \star\in \pbpst(\ckcO)$, its leg is defined to be the  pair
\[
\LEG(\tau) := \Lambda_{\max(l'-1,0),1}(\imath,\cP)\times \Lambda_{\max(l'-1,0),1}(\jmath,\cQ),
\]
and its  body  is defined to be the pair 
\[\BODY(\tau) := 
\barLambda_{\max(l'-1,0),1}(\imath,\cP)\times \barLambda_{\max(l'-1,0),1}(\jmath,\cQ).\]
  
 Note that if $\tau\in \mathrm{PBP}_\star(\ckcO,\wp ) $,  then $\LEG(\tau)$ is represented by the first pair  in \eqref{legtau11} where  
\[
x_1\neq \eee,\qquad x_0=\emptyset\Leftrightarrow x_2=\emptyset\Leftrightarrow \mathbf{r}_2(\check \CO)=0,
\]
and the grey part consisting of $l-l'-1$ boxes with label $r$.
 
\begin{equation}\label{legtau11}
 \LEG(\tau):\ \   \ytb{{x_2},\none,\none ,\none ,\none ,\none}
    \times
  \ytb{{x_{0}},{*(srcol)r},{\enon[srcol]\vdots},{\enon[srcol]{\vdots}},
  {*(srcol)r},{x_{1}} }
    \hspace{8em}
    \LEG(\bartau):\ \ 
 \ytb{{y_{0}},{*(srcol)s},{\enon[srcol]\vdots},{\enon[srcol]{\vdots}},
  {*(srcol)s},{y_{1}} } 
    \times
    \ytb{{y_2},\none,\none ,\none ,\none ,\none},
\end{equation}


 Likewise, for every $\bartau\in \mathrm{PBP}_\star(\ckcO,\bar \wp ) $,  $\LEG(\bartau)$ is represented by the second pair of \eqref{legtau11}  
where
\[
y_1\neq \eee,\qquad y_0=\emptyset\Leftrightarrow y_2=\emptyset\Leftrightarrow \mathbf{r}_2(\check \CO)=0,
\]
and the grey part consisting of $l-l'-1$ boxes with label $s$.
 
 \delete{
 If $\tau\in \mathrm{PBP}_\star(\ckcO,\bar \wp ) $,  then $\LEG(\tau)$ is represented by a pair of the form
\be\label{tildecps}
  \ytb{{x_{0}},{*(srcol)r},{\enon[srcol]\vdots},{\enon[srcol]{\vdots}},
  {*(srcol)r},{x_{1}} } 
    \times
    \ytb{{x_2},\none,\none ,\none ,\none ,\none},
\ee
where
\[
x_1\neq \eee,\qquad x_0=\emptyset\Leftrightarrow x_2=\emptyset\Leftrightarrow \mathbf{r}_2(\check \CO)=0,
\]
and the grey part consisting of $m$ boxes with label $r$.
}

The following proposition is much easier to check than Proposition \ref{propswithc}. We omit the details. 
\begin{prop}\label{propswithc2}
Suppose that $\star=\widetilde C$ and $(1,2)\in \wp$. For every $\tau\in \mathrm{PBP}_\star(\ckcO,\wp ) $ such that $\LEG(\tau)$ is represented by the first pair  in \eqref{legtau11}, there is a unique element $\bartau\in \mathrm{PBP}_\star(\ckcO,\bar \wp ) $ such that 
 \[
    \BODY(\bartau) = \BODY(\tau) \qquad 
\]
and $\LEG(\bartau)$ is represented by the second pair  in \eqref{legtau11} with
\[
y_i = \bullet,r,s,d,c, \text{ or }\eee\qquad (i=0,1,2),
\]
respectively if 
\[
x_i = \bullet,s,r,c,d, \text{ or }\eee.  %\text{ respectively} 
\]
Moreover, the map 
\[
 \mathrm{PBP}_\star(\ckcO,\wp)\rightarrow \mathrm{PBP}_\star(\ckcO,\bar \wp), \qquad
 \tau \mapsto \bartau. 
 \]
 is bijective.
\end{prop}


\section{Counting the multiplicity of non-special representations}
In this section, we assume $\star\in \set{B,C,\wtC,D}$. 
Let $\ckcO$ be a $\star$-good parity orbit.

For $\star\in \set{B,D}$, we write 
\[
\begin{split}
\PBP_\star^d(\ckcO, \wp) & := 
\set{\tau\in \PBP_{\star}(\ckcO,\wp) | x_\tau = d}.\\
\PBP_\star^{rc}(\ckcO, \wp) & := 
\set{\tau\in \PBP_{\star}(\ckcO,\wp) | x_\tau \in \set{r,c}}.\\
\PBP_\star^{s}(\ckcO, \wp) & := 
\set{\tau\in \PBP_{\star}(\ckcO,\wp) | x_\tau = s}.\\
\PBP_\star^{-s}(\ckcO, \wp) & := 
\set{\tau\in \PBP_{\star}(\ckcO,\wp) | x_\tau \neq s}.\\
\end{split}
\]
Note that $\PBP_\star^{-s}(\ckcO, \wp)  
= \PBP_\star^d(\ckcO, \wp)\sqcup \PBP_\star^{rc}(\ckcO, \wp)$.
We write
\[
\PBP_{D,sc}(\ckcO_0) := 
\set{\tau_0\in \PBP_D(\ckcO_0)| \cP_{\tau_0}^{-1}(\set{s,c}) \neq \emptyset}. 
\]
We define $\PBP_{D,sc}^\sharp$ similarly. 

We will prove the following counting lemma:

\begin{prop}
Let $\wp\in \PP(\ckcO)$, we have $\abs{\PBP_\star(\ckcO,\wp)} = \abs{\PBP_\star(\ckcO,\emptyset)}$. 
When $\star  \in \set{B,D}$, 
for $\sharp \in \set{ -s, s, d, rc}$,
we have  
\begin{equation}\label{eq:pbp.sh}
\abs{\PBP_\star^\sharp(\ckcO, \wp)}=\abs{\PBP_\star^\sharp(\ckcO, \emptyset)}
\end{equation}
\end{prop}
\begin{proof}
We prove the proposition by induction. 
Assume for each good parity orbit $\ckcO'$ such that $\bfrr_k(\ckcO')=0$ the propitiation holds.   

Suppose $\star\in \set{C,\wtC}$.

When $(1,2)\in \PP_\star(\ckcO)$, and $(1,2)\in \wp$.  
Let $\bar\wp = \wp  - \set{(1,2)}$.  
By the switching algorithm, we have 
\[
\abs{\PBP_\star(\cO,\wp)} = \abs{\PBP_\star(\ckcO,\bar\wp)}.
\]
Therefore, we can assume $(1,2)\notin \wp$ without of loss of generality. 

Suppose $(1,2)\in \PP_\star(\ckcO)$, we have a bijection
\[
\PBP_\star(\ckcO,\wp)\xrightarrow{\ \ \ \ \DD\ \ \ \ }\PBP_{\star'}(\ckcO',\wp')
\]
for each $(1,2)\notin \wp\subset \PP_\star(\ckcO)$. 
By the induction hypothesis, 
\[
\abs{\PBP_\star(\ckcO,\wp)} = \abs{\PBP_{\star'}(\ckcO',\wp')}
= \abs{\PBP_{\star'}(\ckcO',\emptyset)} = 
\abs{\PBP_\star(\ckcO,\emptyset)}.
\]

Suppose $(1,2)\notin \PP_\star(\ckcO)$, we have a bijection
\[
\PBP_\star(\ckcO,\wp)\xrightarrow{\ \ \ \ \DD\ \ \ \ } \PBP_{\star'}^{-s}(\ckcO',\wp')
\]
for each $\wp \subset \PP_\star(\ckcO)$. 
Therefore, by induction hypothesis, we have 
\[
\abs{\PBP_\star(\ckcO,\wp)} = \abs{\PBP_{\star'}^{-s}(\ckcO',\wp')}
= \abs{\PBP_{\star'}^{-s}(\ckcO',\emptyset)}
= \abs{\PBP_\star(\ckcO,\emptyset)}.
\]

Note that under the  
\[
\PBP_{\star'}(\ckcO',\wp')\longrightarrow \PBP_{\star}(\ckcO'', \wp'')\times 
\PBP_{D}(\ckcO'_0)
\]

Suppose $\star \in \set{B,D}$. 
Assume $(2,3)\in \PBP_\star(\ckcO)$. 
Then
\[
\delta \colon \PBP_{\star}(\ckcO,\wp)\longrightarrow \PBP_{\star}(\ckcO', \wp')\times \PBP_{D}(\ckcO_0)
\]
is a bijection for each $\wp \in \PP_\star(\ckcO)$. 
Therefore
\[
\abs{\PBP_\star(\ckcO,\wp)} =
\abs{\PBP_{\star'}(\ckcO',\wp')}\abs{\PBP_{D}(\ckcO_0)}
= \abs{\PBP_{\star'}(\ckcO',\emptyset)}\abs{\PBP_{D}(\ckcO_0)}
= \abs{\PBP_\star(\ckcO,\emptyset)}.
\]
Since 
$x_\tau = d\Leftrightarrow x_{\tau_0} =d$
and 
$x_\tau = s\Leftrightarrow x_{\tau_0} = s$, 
$\delta$ induces bijections 
\[
\PBP_{\star}^\sharp (\ckcO,\wp) \longrightarrow  \PBP_{\star}(\ckcO', \wp')\times
\PBP_{D}^\sharp(\ckcO_0)% = \abs{\PBP_{\star}^\sharp (\ckcO,\emptyset)}. 
\]
for  $\sharp \in \set{s,-s,d,rc}$. 
Now \eqref{eq:pbp.sh} follows. 

Assume $(2,3)\notin \PBP_\star(\ckcO)$. 
Note that the image of $\delta$ as the following
\[
%\delta \colon \PBP_{\star}(\ckcO,\wp)\longrightarrow \PBP_{\star}(\ckcO', \wp')\times \PBP_{D}(\ckcO_0)
\begin{split}
\Im (\delta) & =
\Set{ (\tau'',\tau_0)  \in \PBP_\star(\ckcO'',\wp'')\times \PBP_D(\ckcO_\bftt,\emptyset)  |  x_{\tau''} = d, \text{ or }  \cP_{\tau_0}^{-1}(\set{s,c})\neq \emptyset } \\
& =  \PBP_\star^{d}(\ckcO'',\wp'')\times \PBP_D(\ckcO_0) 
\sqcup
\PBP_\star^{rc}(\ckcO'',\wp'')\times \PBP_{D,sc}(\ckcO_0) .
\end{split}
\]
Now 
\[
\begin{split}
 & \abs{\PBP_{\star}(\ckcO,\wp)}   \\
 & = \abs{\PBP_{\star}^d(\ckcO,\wp'')} \abs{\PBP_{d}(\ckcO_0)} 
 + \abs{\PBP_{\star}^{rc}(\ckcO,\wp'')} \abs{\PBP_{\star}^{sc}(\ckcO_0)} \\
 & = \abs{\PBP_{\star}(\ckcO,\wp)}   \\
\end{split}
\]

The equation \eqref{eq:pbp.sh} follows from the bijection:
\[
\PBP_\star^{\sharp}(\ckcO,\wp)
 \xrightarrow{\ \ \ \delta \ \ \ }  \PBP_\star^{d}(\ckcO'',\wp'')\times \PBP_D^\sharp(\ckcO_0) 
\sqcup
\PBP_\star^{rc}(\ckcO'',\wp'')\times \PBP_{D,sc}^\sharp(\ckcO_0) .
\]

This finished the proof of the proposition.
\end{proof}

In the rest of this section we assume that $\check \CO\neq \emptyset$.  Let $\check \CO'$ be the dual descent of $\check \CO$ as defined in the Introduction. Then $\check \CO'$ has $\star'$-good parity, where $\star'$ is the  Howe dual of $\star$. Put
\[
  l':=l_{\star', \check \CO'}.
\]







The following equation of signatures will be crucial in our computation of the local system in the next section. 


\begin{prop}\label{prop:delta}
Suppose $\bfrr_2(\ckcO)>0$. Let $\ckcOpp := \ckDD^2(\ckcO)$ and  $\wp'' = \ckDD^2(\wp)$. Consider the map 
\begin{equation}\label{eq:delta}
    \delta \colon \PBP_\star(\ckcO)\longrightarrow 
    \PBP_\star(\ckcOpp,\wp'')\times \PBP_{\star_\bftt}(\ckcO_\bftt,\emptyset),
    \qquad \tau \mapsto (\DD^2(\tau),\tau_\bftt)
\end{equation}
Then $\delta$ is an injection. 
\begin{itemize}
\item
    When $\star = C^*$ or $\bfrr_2(\ckcO)>\bfrr_3(\ckcO)$, the map $\delta$ is a bijection. 
    Moreover,
    % We have the following equation of signatures.
\[
\ssign(\tau)
=(\bfcc_2(\cO),\bfcc_2(\cO))+\ssign(\DD^2(\tau))+\ssign(\tau_\bftt).
\]
\item
    When  $\star \in \set{B,D}$ and $\bfrr_2(\ckcO)=\bfrr_3(\ckcO)$, the map $\delta$ is an injection,
    whose image equals 
    \[
    \Set{ (\tau'',\tau_0)  \in \PBP_\star(\ckcO,\wp'')\times \PBP_D(\ckcO_\bftt,\emptyset)  | 
    x_{\tau''} = d, \text{ or } 
    \cP_{\tau_0}^{-1}(\set{s,c})\neq \emptyset }.
    \]
    %We have the following equation of signatures.
    Moreover,
\[
\ssign(\tau)
=(\bfcc_2(\cO)-1,\bfcc_2(\cO)-1)+\ssign(\DD^2(\tau))+\ssign(\tau_\bftt).
\]
\end{itemize}
\end{prop}




\end{document}