\documentclass[12pt,a4paper]{amsart}
\usepackage[margin=2.5cm,marginpar=2cm]{geometry}
\usepackage{amscd}

\usepackage[bookmarksopen,bookmarksdepth=2,hidelinks,colorlinks=false]{hyperref}
\usepackage[nameinlink]{cleveref}

% \usepackage[color]{showkeys}
% \makeatletter
%   \SK@def\Cref#1{\SK@\SK@@ref{#1}\SK@Cref{#1}}%
% \makeatother

\usepackage{array}
%% FONTS
\usepackage{amssymb}
\usepackage{amsmath}
\usepackage{mathrsfs}
\usepackage{mathbbol,mathabx}
\usepackage{amsthm}
\usepackage{graphicx}
\usepackage{braket}
\usepackage{mathtools}

\usepackage{amsrefs}

\usepackage[all,cmtip]{xy}
\usepackage{rotating}
\usepackage{leftidx}
%\usepackage{arydshln}

% circled number
\usepackage{pifont}
\makeatletter
\newcommand*{\circnuma}[1]{%
  \ifnum#1<1 %
    \@ctrerr
  \else
    \ifnum#1>20 %
      \@ctrerr
    \else
      \mbox{\ding{\numexpr 171+(#1)\relax}}%
     \fi
  \fi
}
\makeatother


\DeclareSymbolFont{bbold}{U}{bbold}{m}{n}
\DeclareSymbolFontAlphabet{\mathbbold}{bbold}


%\usepackage[dvipdfx,rgb,table]{xcolor}
\usepackage[rgb,table,dvipsnames]{xcolor}
%\usepackage{color}
%\usepackage{mathrsfs}

\setcounter{tocdepth}{1}
\setcounter{secnumdepth}{2}

%\usepackage[abbrev,shortalphabetic]{amsrefs}


\usepackage{imakeidx}
\def\idxemph#1{\emph{#1}\index{#1}}
\makeindex


\usepackage[normalem]{ulem}


\usepackage[centertableaux]{ytableau}

%\usepackage[mathlines,pagewise]{lineno}
%\linenumbers

\usepackage{enumitem}
%% Enumitem
\newlist{enumC}{enumerate}{1} % Conditions in Lemma/Theorem/Prop
\setlist[enumC,1]{label=(\alph*),wide,ref=(\alph*)}
\crefname{enumCi}{condition}{conditions}
\Crefname{enumCi}{Condition}{Conditions}
\newlist{enumT}{enumerate}{3} % "Theorem"=conclusions in Lemma/Theorem/Prop
\setlist[enumT]{label=(\roman*),wide}
\setlist[enumT,1]{label=(\roman*),wide}
\setlist[enumT,2]{label=(\alph*),ref ={(\roman{enumTi}.\alph*)},left=2em}
\setlist[enumT,3]{label*=.(\arabic*), ref ={(\roman{enumTi}.\alph{enumTii}.\alph*)}}
\crefname{enumTi}{}{}
\Crefname{enumTi}{Item}{Items}
\crefname{enumTii}{}{}
\Crefname{enumTii}{Item}{Items}
\crefname{enumTiii}{}{}
\Crefname{enumTiii}{Item}{Items}
\newlist{enumPF}{enumerate}{3}
%\setlist[enumPF]{label=(\alph*),wide}
\setlist[enumPF,1]{label=(\roman*),wide}
\setlist[enumPF,2]{label=(\alph*),left=2em}
\setlist[enumPF,3]{label=\arabic*).,left=1em}
\newlist{enumS}{enumerate}{3} % Statement outside Lemma/Theorem/Prop
\setlist[enumS]{label=\roman*)}
\setlist[enumS,1]{label=\roman*)}
\setlist[enumS,2]{label=\alph*)}
\setlist[enumS,3]{label=\arabic*.}
\newlist{enumI}{enumerate}{3} % items
\setlist[enumI,1]{label=\roman*),leftmargin=*}
\setlist[enumI,2]{label=\alph*), leftmargin=*}
\setlist[enumI,3]{label=\arabic*), leftmargin=*}
\newlist{enumIL}{enumerate*}{1} %inline enum
\setlist*[enumIL]{label=\roman*)}
\newlist{enumR}{enumerate}{1} % remarks
\setlist[enumR]{label=\arabic*.,wide,labelwidth=!, labelindent=0pt}
\crefname{enumRi}{remark}{remarks}


\newlist{enuma}{enumerate}{1} % Statement in Lemma/Theorem/Prop
\setlist[enuma]{label=(\alph*),nosep,leftmargin=*}

%\definecolor{srcol}{RGB}{255,255,51}

%\definecolor{srcol}{RGB}{255,255,51}
\colorlet{srcol}{black!15}

\crefname{equation}{}{}
\Crefname{equation}{Equation}{Equations}
\Crefname{lem}{Lemma}{Lemma}
\Crefname{thm}{Theorem}{Theorem}

\newlist{des}{enumerate}{1}
\setlist[des]{font=\upshape\sffamily\bfseries, label={}}
%\setlist[des]{before={\renewcommand\makelabel[1]{\sffamily \bfseries ##1 }}}

% editing macros.
%\blendcolors{!80!black}
\long\def\okay#1{\ifcsname highlightokay\endcsname
{\color{red} #1}
\else
{#1}
\fi
}
\long\def\editc#1{{\color{red} #1}}
\long\def\mjj#1{{{\color{blue}#1}}}
\long\def\mjjr#1{{\color{red} (#1)}}
\long\def\mjjd#1#2{{\color{blue} #1 \sout{#2}}}
\def\mjjb{\color{blue}}
\def\mjje{\color{black}}
\def\mjjcb{\color{green!50!black}}
\def\mjjce{\color{black}}

\long\def\sun#1{{{\color{cyan}#1}}}
\long\def\sund#1#2{{\color{cyan}#1  \sout{#2}}}
\long\def\mv#1{{{\color{red} {\bf move to a proper place:} #1}}}
\long\def\delete#1{}

%\reversemarginpar
\newcommand{\lokec}[1]{\marginpar{\color{blue}\tiny #1 \mbox{--loke}}}
\newcommand{\mjjc}[1]{\marginpar{\color{green}\tiny #1 \mbox{--ma}}}


%\def\showtrivial{\relax}

\newcommand{\trivial}[2][]{\if\relax\detokenize{#1}\relax
  {%\hfill\break
   % \begin{minipage}{\textwidth}
      \color{orange} \vspace{0em} $[$  #2 $]$
  %\end{minipage}
  %\break
      \color{black}
  }
  \else
\ifx#1h
\ifcsname showtrivial\endcsname
{%\hfill\break
 % \begin{minipage}{\textwidth}
    \color{orange} \vspace{0em}  $[$ #2 $]$
%\end{minipage}
%\break
    \color{black}
}
\fi
\else {\red Wrong argument!} \fi
\fi
}

\newcommand{\byhide}[2][]{\if\relax\detokenize{#1}\relax
{\color{orange} \vspace{0em} Plan to delete:  #2}
\else
\ifx#1h\relax\fi
\fi
}



\newcommand{\Rank}{\mathrm{rk}}
\newcommand{\cqq}{\mathscr{D}}
\newcommand{\rsym}{\mathrm{sym}}
\newcommand{\rskew}{\mathrm{skew}}
\newcommand{\fraksp}{\mathfrak{sp}}
\newcommand{\frakso}{\mathfrak{so}}
\newcommand{\frakm}{\mathfrak{m}}
\newcommand{\frakp}{\mathfrak{p}}
\newcommand{\pr}{\mathrm{pr}}
\newcommand{\rhopst}{\rho'^*}
\newcommand{\Rad}{\mathrm{Rad}}
\newcommand{\Res}{\mathrm{Res}}
\newcommand{\Hol}{\mathrm{Hol}}
\newcommand{\AC}{\mathrm{AC}}
%\newcommand{\AS}{\mathrm{AS}}
\newcommand{\WF}{\mathrm{WF}}
\newcommand{\AV}{\mathrm{AV}}
\newcommand{\AVC}{\mathrm{AV}_\bC}
\newcommand{\VC}{\mathrm{V}_\bC}
\newcommand{\bfv}{\mathbf{v}}
\newcommand{\depth}{\mathrm{depth}}
\newcommand{\wtM}{\widetilde{M}}
\newcommand{\wtMone}{{\widetilde{M}^{(1,1)}}}

\newcommand{\nullpp}{N(\fpp'^*)}
\newcommand{\nullp}{N(\fpp^*)}
%\newcommand{\Aut}{\mathrm{Aut}}
%\usepackage{mnsymbol}


\def\YD{{\mathsf{YD}}}
\def\SYD{{\mathsf{SYD}}}
\def\MYD{{\mathsf{MYD}}}

\def\KM{{\mathcal{K_{\mathsf{M}}}}}

\newcommand{\bfonenp}{\mathbf{1}^{-,+}}
\newcommand{\bfonepn}{\mathbf{1}^{+,-}}
\newcommand{\bfone}{\mathbf{1}}
\newcommand{\piSigma}{\pi_\Sigma}
\newcommand{\piSigmap}{\pi'_\Sigma}


\newcommand{\sfVprime}{\mathsf{V}^\prime}
\newcommand{\sfVdprime}{\mathsf{V}^{\prime \prime}}
\newcommand{\gminusone}{\mathfrak{g}_{-\frac{1}{m}}}

\newcommand{\eva}{\mathrm{eva}}

% \newcommand\iso{\xrightarrow{
%    \,\smash{\raisebox{-0.65ex}{\ensuremath{\scriptstyle\sim}}}\,}}

\def\Ueven{{U_{\rm{even}}}}
\def\Uodd{{U_{\rm{odd}}}}
\def\ttau{\tilde{\tau}}
\def\Wcp{W}
\def\Kur{{K^{\mathrm{u}}}}

\def\Im{\operatorname{Im}}


\providecommand{\bcN}{{\overline{\cN}}}



\makeatletter

\def\gen#1{\left\langle
    #1
      \right\rangle}
\makeatother

\makeatletter
\def\inn#1#2{\left\langle
      \def\ta{#1}\def\tb{#2}
      \ifx\ta\@empty{\;} \else {\ta}\fi ,
      \ifx\tb\@empty{\;} \else {\tb}\fi
      \right\rangle}
\def\binn#1#2{\left\lAngle
      \def\ta{#1}\def\tb{#2}
      \ifx\ta\@empty{\;} \else {\ta}\fi ,
      \ifx\tb\@empty{\;} \else {\tb}\fi
      \right\rAngle}
\makeatother

\makeatletter
\def\binn#1#2{\overline{\inn{#1}{#2}}}
\makeatother


\def\innwi#1#2{\inn{#1}{#2}_{W_i}}
\def\innw#1#2{\inn{#1}{#2}_{\bfW}}
\def\innv#1#2{\inn{#1}{#2}_{\bfV}}
\def\innbfv#1#2{\inn{#1}{#2}_{\bfV}}
\def\innvi#1#2{\inn{#1}{#2}_{V_i}}
\def\innvp#1#2{\inn{#1}{#2}_{\bfV'}}
\def\innp#1#2{\inn{#1}{#2}'}

% choose one of then
\def\simrightarrow{\iso}
\def\surj{\twoheadrightarrow}
%\def\simrightarrow{\xrightarrow{\sim}}

\newcommand\iso{\xrightarrow{
   \,\smash{\raisebox{-0.65ex}{\ensuremath{\scriptstyle\sim}}}\,}}

\newcommand\riso{\xleftarrow{
   \,\smash{\raisebox{-0.65ex}{\ensuremath{\scriptstyle\sim}}}\,}}









\usepackage{xparse}
\def\usecsname#1{\csname #1\endcsname}
\def\useLetter#1{#1}
\def\usedbletter#1{#1#1}

% \def\useCSf#1{\csname f#1\endcsname}

\ExplSyntaxOn

\def\mydefcirc#1#2#3{\expandafter\def\csname
  circ#3{#1}\endcsname{{}^\circ {#2{#1}}}}
\def\mydefvec#1#2#3{\expandafter\def\csname
  vec#3{#1}\endcsname{\vec{#2{#1}}}}
\def\mydefdot#1#2#3{\expandafter\def\csname
  dot#3{#1}\endcsname{\dot{#2{#1}}}}

\def\mydefacute#1#2#3{\expandafter\def\csname a#3{#1}\endcsname{\acute{#2{#1}}}}
\def\mydefbr#1#2#3{\expandafter\def\csname br#3{#1}\endcsname{\breve{#2{#1}}}}
\def\mydefbar#1#2#3{\expandafter\def\csname bar#3{#1}\endcsname{\bar{#2{#1}}}}
\def\mydefhat#1#2#3{\expandafter\def\csname hat#3{#1}\endcsname{\hat{#2{#1}}}}
\def\mydefwh#1#2#3{\expandafter\def\csname wh#3{#1}\endcsname{\widehat{#2{#1}}}}
\def\mydeft#1#2#3{\expandafter\def\csname t#3{#1}\endcsname{\tilde{#2{#1}}}}
\def\mydefu#1#2#3{\expandafter\def\csname u#3{#1}\endcsname{\underline{#2{#1}}}}
\def\mydefr#1#2#3{\expandafter\def\csname r#3{#1}\endcsname{\mathrm{#2{#1}}}}
\def\mydefb#1#2#3{\expandafter\def\csname b#3{#1}\endcsname{\mathbb{#2{#1}}}}
\def\mydefwt#1#2#3{\expandafter\def\csname wt#3{#1}\endcsname{\widetilde{#2{#1}}}}
%\def\mydeff#1#2#3{\expandafter\def\csname f#3{#1}\endcsname{\mathfrak{#2{#1}}}}
\def\mydefbf#1#2#3{\expandafter\def\csname bf#3{#1}\endcsname{\mathbf{#2{#1}}}}
\def\mydefc#1#2#3{\expandafter\def\csname c#3{#1}\endcsname{\mathcal{#2{#1}}}}
\def\mydefsf#1#2#3{\expandafter\def\csname sf#3{#1}\endcsname{\mathsf{#2{#1}}}}
\def\mydefs#1#2#3{\expandafter\def\csname s#3{#1}\endcsname{\mathscr{#2{#1}}}}
\def\mydefcks#1#2#3{\expandafter\def\csname cks#3{#1}\endcsname{{\check{
        \csname s#2{#1}\endcsname}}}}
\def\mydefckc#1#2#3{\expandafter\def\csname ckc#3{#1}\endcsname{{\check{
      \csname c#2{#1}\endcsname}}}}
\def\mydefck#1#2#3{\expandafter\def\csname ck#3{#1}\endcsname{{\check{#2{#1}}}}}

\cs_new:Npn \mydeff #1#2#3 {\cs_new:cpn {f#3{#1}} {\mathfrak{#2{#1}}}}

\cs_new:Npn \doGreek #1
{
  \clist_map_inline:nn {alpha,beta,gamma,Gamma,delta,Delta,epsilon,varepsilon,zeta,eta,theta,vartheta,Theta,iota,kappa,lambda,Lambda,mu,nu,xi,Xi,pi,Pi,rho,sigma,varsigma,Sigma,tau,upsilon,Upsilon,phi,varphi,Phi,chi,psi,Psi,omega,Omega,tG} {#1{##1}{\usecsname}{\useLetter}}
}

\cs_new:Npn \doSymbols #1
{
  \clist_map_inline:nn {otimes,boxtimes} {#1{##1}{\usecsname}{\useLetter}}
}

\cs_new:Npn \doAtZ #1
{
  \clist_map_inline:nn {A,B,C,D,E,F,G,H,I,J,K,L,M,N,O,P,Q,R,S,T,U,V,W,X,Y,Z} {#1{##1}{\useLetter}{\useLetter}}
}

\cs_new:Npn \doatz #1
{
  \clist_map_inline:nn {a,b,c,d,e,f,g,h,i,j,k,l,m,n,o,p,q,r,s,t,u,v,w,x,y,z} {#1{##1}{\useLetter}{\usedbletter}}
}

\cs_new:Npn \doallAtZ
{
\clist_map_inline:nn {mydefsf,mydeft,mydefu,mydefwh,mydefhat,mydefr,mydefwt,mydeff,mydefb,mydefbf,mydefc,mydefs,mydefck,mydefcks,mydefckc,mydefbar,mydefvec,mydefcirc,mydefdot,mydefbr,mydefacute} {\doAtZ{\csname ##1\endcsname}}
}

\cs_new:Npn \doallatz
{
\clist_map_inline:nn {mydefsf,mydeft,mydefu,mydefwh,mydefhat,mydefr,mydefwt,mydeff,mydefb,mydefbf,mydefc,mydefs,mydefck,mydefbar,mydefvec,mydefdot,mydefbr,mydefacute} {\doatz{\csname ##1\endcsname}}
}

\cs_new:Npn \doallGreek
{
\clist_map_inline:nn {mydefck,mydefwt,mydeft,mydefwh,mydefbar,mydefu,mydefvec,mydefcirc,mydefdot,mydefbr,mydefacute} {\doGreek{\csname ##1\endcsname}}
}

\cs_new:Npn \doallSymbols
{
\clist_map_inline:nn {mydefck,mydefwt,mydeft,mydefwh,mydefbar,mydefu,mydefvec,mydefcirc,mydefdot} {\doSymbols{\csname ##1\endcsname}}
}



\cs_new:Npn \doGroups #1
{
  \clist_map_inline:nn {GL,Sp,rO,rU,fgl,fsp,foo,fuu,fkk,fuu,ufkk,uK} {#1{##1}{\usecsname}{\useLetter}}
}

\cs_new:Npn \doallGroups
{
\clist_map_inline:nn {mydeft,mydefu,mydefwh,mydefhat,mydefwt,mydefck,mydefbar} {\doGroups{\csname ##1\endcsname}}
}


\cs_new:Npn \decsyms #1
{
\clist_map_inline:nn {#1} {\expandafter\DeclareMathOperator\csname ##1\endcsname{##1}}
}

\decsyms{Mp,id,SL,Sp,SU,SO,GO,GSO,GU,GSp,PGL,Pic,Lie,Mat,Ker,Hom,Ext,Ind,reg,res,inv,Isom,Det,Tr,Norm,Sym,Span,Stab,Spec,PGSp,PSL,tr,Ad,Br,Ch,Cent,End,Aut,Dvi,Frob,Gal,GL,Gr,DO,ur,vol,ab,Nil,Supp,rank,Sign}

\def\abs#1{\left|{#1}\right|}
\def\norm#1{{\left\|{#1}\right\|}}


% \NewDocumentCommand\inn{m m}{
% \left\langle
% \IfValueTF{#1}{#1}{000}
% ,
% \IfValueTF{#2}{#2}{000}
% \right\rangle
% }
\NewDocumentCommand\cent{o m }{
  \IfValueTF{#1}{
    \mathop{Z}_{#1}{(#2)}}
  {\mathop{Z}{(#2)}}
}


\def\fsl{\mathfrak{sl}}
\def\fsp{\mathfrak{sp}}


%\def\cent#1#2{{\mathrm{Z}_{#1}({#2})}}


\doallAtZ
\doallatz
\doallGreek
\doallGroups
\doallSymbols
\ExplSyntaxOff


% \usepackage{geometry,amsthm,graphics,tabularx,amssymb,shapepar}
% \usepackage{amscd}
% \usepackage{mathrsfs}


\usepackage{diagbox}
% Update the information and uncomment if AMS is not the copyright
% holder.
%\copyrightinfo{2006}{American Mathematical Society}
%\usepackage{nicematrix}
\usepackage{arydshln}
\usepackage[mode=buildnew]{standalone}% requires -shell-escape

\usepackage{tikz,etoolbox}
\usetikzlibrary{matrix,arrows,positioning,backgrounds}
\usetikzlibrary{decorations.pathmorphing,decorations.pathreplacing}
\usetikzlibrary{cd}
% \usetikzlibrary{external}
%   \tikzexternalize
% \usetikzlibrary{cd}

%  \AtBeginEnvironment{tikzcd}{\tikzexternaldisable}
%  \AtEndEnvironment{tikzcd}{\tikzexternalenable}

%  \usetikzlibrary{matrix,arrows,positioning,backgrounds}
%  \usetikzlibrary{decorations.pathmorphing,decorations.pathreplacing}

% % externalization not work properly
% % \usetikzlibrary{external}
% \tikzexternalize[prefix=figures/]
% % % activate the following such that you can check the macro expansion in
% % % *-figure0.md5 manually
% %\tikzset{external/up to date check=diff}
% \usepackage{environ}

% \def\temp{&} \catcode`&=\active \let&=\temp

% \newcommand{\mytikzcdcontext}[2]{
%   \begin{tikzpicture}[baseline=(maintikzcdnode.base)]
%     \node (maintikzcdnode) [inner sep=0, outer sep=0] {\begin{tikzcd}[#2]
%         #1
%     \end{tikzcd}};
%   \end{tikzpicture}}

% \NewEnviron{mytikzcd}[1][]{%
% % In the following, we need \BODY to expanded before \mytikzcdcontext
% % such that the md5 function gets the tikzcd content with \BODY expanded.
% % Howerver, expand it only once, because the \tikz-macros aren't
% % defined at this point yet. The same thing holds for the arguments to
% % the tikzcd-environment.
% \def\myargs{#1}%
% \edef\mydiagram{\noexpand\mytikzcdcontext{\expandonce\BODY}{\expandonce\myargs}}%
% \mydiagram%
% }

\usepackage{upgreek}

\usepackage{listings}
\lstset{
    basicstyle=\ttfamily\tiny,
    keywordstyle=\color{black},
    commentstyle=\color{white}, % white comments
    stringstyle=\ttfamily, % typewriter type for strings
    showstringspaces=false,
    breaklines=true,
    emph={Output},emphstyle=\color{blue},
}

\newcommand{\BA}{{\mathbb{A}}}
%\newcommand{\BB}{{\mathbb {B}}}
\newcommand{\BC}{{\mathbb {C}}}
\newcommand{\BD}{{\mathbb {D}}}
\newcommand{\BE}{{\mathbb {E}}}
\newcommand{\BF}{{\mathbb {F}}}
\newcommand{\BG}{{\mathbb {G}}}
\newcommand{\BH}{{\mathbb {H}}}
\newcommand{\BI}{{\mathbb {I}}}
\newcommand{\BJ}{{\mathbb {J}}}
\newcommand{\BK}{{\mathbb {U}}}
\newcommand{\BL}{{\mathbb {L}}}
\newcommand{\BM}{{\mathbb {M}}}
\newcommand{\BN}{{\mathbb {N}}}
\newcommand{\BO}{{\mathbb {O}}}
\newcommand{\BP}{{\mathbb {P}}}
\newcommand{\BQ}{{\mathbb {Q}}}
\newcommand{\BR}{{\mathbb {R}}}
\newcommand{\BS}{{\mathbb {S}}}
\newcommand{\BT}{{\mathbb {T}}}
\newcommand{\BU}{{\mathbb {U}}}
\newcommand{\BV}{{\mathbb {V}}}
\newcommand{\BW}{{\mathbb {W}}}
\newcommand{\BX}{{\mathbb {X}}}
\newcommand{\BY}{{\mathbb {Y}}}
\newcommand{\BZ}{{\mathbb {Z}}}
\newcommand{\Bk}{{\mathbf {k}}}

\newcommand{\CA}{{\mathcal {A}}}
\newcommand{\CB}{{\mathcal {B}}}
\newcommand{\CC}{{\mathcal {C}}}

\newcommand{\CE}{{\mathcal {E}}}
\newcommand{\CF}{{\mathcal {F}}}
\newcommand{\CG}{{\mathcal {G}}}
\newcommand{\CH}{{\mathcal {H}}}
\newcommand{\CI}{{\mathcal {I}}}
\newcommand{\CJ}{{\mathcal {J}}}
\newcommand{\CK}{{\mathcal {K}}}
\newcommand{\CL}{{\mathcal {L}}}
\newcommand{\CM}{{\mathcal {M}}}
\newcommand{\CN}{{\mathcal {N}}}
\newcommand{\CO}{{\mathcal {O}}}
\newcommand{\CP}{{\mathcal {P}}}
\newcommand{\CQ}{{\mathcal {Q}}}
\newcommand{\CR}{{\mathcal {R}}}
\newcommand{\CS}{{\mathcal {S}}}
\newcommand{\CT}{{\mathcal {T}}}
\newcommand{\CU}{{\mathcal {U}}}
\newcommand{\CV}{{\mathcal {V}}}
\newcommand{\CW}{{\mathcal {W}}}
\newcommand{\CX}{{\mathcal {X}}}
\newcommand{\CY}{{\mathcal {Y}}}
\newcommand{\CZ}{{\mathcal {Z}}}


\newcommand{\RA}{{\mathrm {A}}}
\newcommand{\RB}{{\mathrm {B}}}
\newcommand{\RC}{{\mathrm {C}}}
\newcommand{\RD}{{\mathrm {D}}}
\newcommand{\RE}{{\mathrm {E}}}
\newcommand{\RF}{{\mathrm {F}}}
\newcommand{\RG}{{\mathrm {G}}}
\newcommand{\RH}{{\mathrm {H}}}
\newcommand{\RI}{{\mathrm {I}}}
\newcommand{\RJ}{{\mathrm {J}}}
\newcommand{\RK}{{\mathrm {K}}}
\newcommand{\RL}{{\mathrm {L}}}
\newcommand{\RM}{{\mathrm {M}}}
\newcommand{\RN}{{\mathrm {N}}}
\newcommand{\RO}{{\mathrm {O}}}
\newcommand{\RP}{{\mathrm {P}}}
\newcommand{\RQ}{{\mathrm {Q}}}
%\newcommand{\RR}{{\mathrm {R}}}
\newcommand{\RS}{{\mathrm {S}}}
\newcommand{\RT}{{\mathrm {T}}}
\newcommand{\RU}{{\mathrm {U}}}
\newcommand{\RV}{{\mathrm {V}}}
\newcommand{\RW}{{\mathrm {W}}}
\newcommand{\RX}{{\mathrm {X}}}
\newcommand{\RY}{{\mathrm {Y}}}
\newcommand{\RZ}{{\mathrm {Z}}}

\DeclareMathOperator{\absNorm}{\mathfrak{N}}
\DeclareMathOperator{\Ann}{Ann}
\DeclareMathOperator{\LAnn}{L-Ann}
\DeclareMathOperator{\RAnn}{R-Ann}
\DeclareMathOperator{\ind}{ind}
%\DeclareMathOperator{\Ind}{Ind}



\def\ckbfG{\check{\bfG}}

\newcommand{\cod}{{\mathrm{cod}}}
\newcommand{\cont}{{\mathrm{cont}}}
\newcommand{\cl}{{\mathrm{cl}}}
\newcommand{\cusp}{{\mathrm{cusp}}}

\newcommand{\disc}{{\mathrm{disc}}}



\newcommand{\Gm}{{\mathbb{G}_m}}



\newcommand{\I}{{\mathrm{I}}}

\newcommand{\Jac}{{\mathrm{Jac}}}
\newcommand{\PM}{{\mathrm{PM}}}


\newcommand{\new}{{\mathrm{new}}}
\newcommand{\NS}{{\mathrm{NS}}}
\newcommand{\N}{{\mathrm{N}}}

\newcommand{\ord}{{\mathrm{ord}}}

%\newcommand{\rank}{{\mathrm{rank}}}

\newcommand{\rk}{{\mathrm{k}}}
\newcommand{\rr}{{\mathrm{r}}}
\newcommand{\rh}{{\mathrm{h}}}

\newcommand{\Sel}{{\mathrm{Sel}}}
\newcommand{\Sim}{{\mathrm{Sim}}}

\newcommand{\wt}{\widetilde}
\newcommand{\wh}{\widehat}
\newcommand{\pp}{\frac{\partial\bar\partial}{\pi i}}
\newcommand{\pair}[1]{\langle {#1} \rangle}
\newcommand{\wpair}[1]{\left\{{#1}\right\}}
\newcommand{\intn}[1]{\left( {#1} \right)}
\newcommand{\sfrac}[2]{\left( \frac {#1}{#2}\right)}
\newcommand{\ds}{\displaystyle}
\newcommand{\ov}{\overline}
\newcommand{\incl}{\hookrightarrow}
\newcommand{\lra}{\longrightarrow}
\newcommand{\imp}{\Longrightarrow}
%\newcommand{\lto}{\longmapsto}
\newcommand{\bs}{\backslash}

\newcommand{\cover}[1]{\widetilde{#1}}

\renewcommand{\vsp}{{\vspace{0.2in}}}

\newcommand{\Norma}{\operatorname{N}}
\newcommand{\Ima}{\operatorname{Im}}
\newcommand{\con}{\textit{C}}
\newcommand{\gr}{\operatorname{gr}}
\newcommand{\ad}{\operatorname{ad}}
\newcommand{\der}{\operatorname{der}}
\newcommand{\dif}{\operatorname{d}\!}
\newcommand{\pro}{\operatorname{pro}}
\newcommand{\Ev}{\operatorname{Ev}}
% \renewcommand{\span}{\operatorname{span}} \span is an innernal command.
%\newcommand{\degree}{\operatorname{deg}}
\newcommand{\Invf}{\operatorname{Invf}}
\newcommand{\Inv}{\operatorname{Inv}}
\newcommand{\slt}{\operatorname{SL}_2(\mathbb{R})}
%\newcommand{\temp}{\operatorname{temp}}
%\newcommand{\otop}{\operatorname{top}}
%\renewcommand{\small}{\operatorname{small}}
\newcommand{\HC}{\operatorname{HC}}
\newcommand{\lef}{\operatorname{left}}
\newcommand{\righ}{\operatorname{right}}
\newcommand{\Diff}{\operatorname{DO}}
\newcommand{\diag}{\operatorname{diag}}
\newcommand{\sh}{\varsigma}
\newcommand{\sch}{\operatorname{sch}}
%\newcommand{\oleft}{\operatorname{left}}
%\newcommand{\oright}{\operatorname{right}}
\newcommand{\open}{\operatorname{open}}
\newcommand{\sgn}{\operatorname{sgn}}
\newcommand{\triv}{\operatorname{triv}}
\newcommand{\Sh}{\operatorname{Sh}}
\newcommand{\oN}{\operatorname{N}}

\newcommand{\oc}{\operatorname{c}}
\newcommand{\od}{\operatorname{d}}
\newcommand{\os}{\operatorname{s}}
\newcommand{\ol}{\operatorname{l}}
\newcommand{\oL}{\operatorname{L}}
\newcommand{\oJ}{\operatorname{J}}
\newcommand{\oH}{\operatorname{H}}
\newcommand{\oO}{\operatorname{O}}
\newcommand{\oS}{\operatorname{S}}
\newcommand{\oR}{\operatorname{R}}
\newcommand{\oT}{\operatorname{T}}
%\newcommand{\rU}{\operatorname{U}}
\newcommand{\oZ}{\operatorname{Z}}
\newcommand{\oD}{\textit{D}}
\newcommand{\oW}{\textit{W}}
\newcommand{\oE}{\operatorname{E}}
\newcommand{\oP}{\operatorname{P}}
\newcommand{\PD}{\operatorname{PD}}
\newcommand{\oU}{\operatorname{U}}

\newcommand{\gC}{{\mathfrak g}_{\C}}
%\renewcommand{\sl}{\mathfrak s \mathfrak l}
\newcommand{\gl}{\mathfrak g \mathfrak l}


\newcommand{\re}{\mathrm e}

\renewcommand{\rk}{\mathrm k}

\newcommand{\g}{\mathfrak g}
\newcommand{\h}{\mathfrak h}
\newcommand{\p}{\mathfrak p}
\newcommand{\Z}{\mathbb{Z}}
\DeclareDocumentCommand{\C}{}{\mathbb{C}}
\newcommand{\R}{\mathbb R}
\newcommand{\Q}{\mathbb Q}
\renewcommand{\H}{\mathbb{H}}
%\newcommand{\N}{\mathbb{N}}
\newcommand{\K}{\mathbb{K}}
%\renewcommand{\S}{\mathbf S}
\newcommand{\M}{\mathbf{M}}
\newcommand{\A}{\mathbb{A}}
\newcommand{\B}{\mathbf{B}}
%\renewcommand{\G}{\mathbf{G}}
\newcommand{\V}{\mathbf{V}}
\newcommand{\W}{\mathbf{W}}
\newcommand{\F}{\mathbf{F}}
\newcommand{\E}{\mathbf{E}}
%\newcommand{\J}{\mathbf{J}}
\renewcommand{\H}{\mathbf{H}}
\newcommand{\X}{\mathbf{X}}
\newcommand{\Y}{\mathbf{Y}}
%\newcommand{\RR}{\mathcal R}
\newcommand{\FF}{\mathcal F}
%\newcommand{\BB}{\mathcal B}
\newcommand{\HH}{\mathcal H}
%\newcommand{\UU}{\mathcal U}
%\newcommand{\MM}{\mathcal M}
%\newcommand{\CC}{\mathcal C}
%\newcommand{\DD}{\mathcal D}
%\def\eDD{\mathrm{d}^{e}}
%\def\eDD{\bigtriangledown}
\def\eDD{\overline{\nabla}}
\def\eDDo{\overline{\nabla}_1}
%\def\eDD{\mathrm{d}}
\def\DD{\nabla}
\def\DDc{\boldsymbol{\nabla}}
\def\gDD{\nabla^{\mathrm{gen}}}
\def\gDDc{\boldsymbol{\nabla}^{\mathrm{gen}}}
%\newcommand{\OO}{\mathcal O}
%\newcommand{\ZZ}{\mathcal Z}
\newcommand{\ve}{{\vee}}
\newcommand{\aut}{\mathcal A}
\newcommand{\ii}{\mathbf{i}}
\newcommand{\jj}{\mathbf{j}}
\newcommand{\kk}{\mathbf{k}}

\newcommand{\la}{\langle}
\newcommand{\ra}{\rangle}
\newcommand{\bp}{\bigskip}
\newcommand{\be}{\begin {equation}}
\newcommand{\ee}{\end {equation}}

\newcommand{\LRleq}{\stackrel{LR}{\leq}}

\numberwithin{equation}{section}


\def\flushl#1{\ifmmode\makebox[0pt][l]{${#1}$}\else\makebox[0pt][l]{#1}\fi}
\def\flushr#1{\ifmmode\makebox[0pt][r]{${#1}$}\else\makebox[0pt][r]{#1}\fi}
\def\flushmr#1{\makebox[0pt][r]{${#1}$}}


%\theoremstyle{Theorem}
% \newtheorem*{thmM}{Main Theorem}
% \crefformat{thmM}{main theorem}
% \Crefformat{thmM}{Main Theorem}
\newtheorem*{thm*}{Theorem}
\newtheorem{thm}{Theorem}[section]
\newtheorem{thml}[thm]{Theorem}
\newtheorem{lem}[thm]{Lemma}
\newtheorem{obs}[thm]{Observation}
\newtheorem{lemt}[thm]{Lemma}
\newtheorem*{lem*}{Lemma}
\newtheorem{whyp}[thm]{Working Hypothesis}
\newtheorem{prop}[thm]{Proposition}
\newtheorem{prpt}[thm]{Proposition}
\newtheorem{prpl}[thm]{Proposition}
\newtheorem{cor}[thm]{Corollary}
%\newtheorem*{prop*}{Proposition}
\newtheorem{claim}[thm]{Claim}
\newtheorem*{claim*}{Claim}
%\theoremstyle{definition}
\newtheorem{defn}[thm]{Definition}
\newtheorem{dfnl}[thm]{Definition}
\newtheorem*{IndH}{Induction Hypothesis}

\newtheorem*{eg*}{Example}
\newtheorem{eg}[thm]{Example}

\theoremstyle{remark}
\newtheorem*{remark}{Remark}
\newtheorem*{remarks}{Remarks}
\newtheorem*{Example}{Example}

\def\cpc{\sigma}
\def\ccJ{\epsilon\dotepsilon}
\def\ccL{c_L}

\def\wtbfK{\widetilde{\bfK}}
%\def\abfV{\acute{\bfV}}
\def\AbfV{\acute{\bfV}}
%\def\afgg{\acute{\fgg}}
%\def\abfG{\acute{\bfG}}
\def\abfV{\bfV'}
\def\afgg{\fgg'}
\def\abfG{\bfG'}

\def\half{{\tfrac{1}{2}}}
\def\ihalf{{\tfrac{\mathbf i}{2}}}
\def\slt{\fsl_2(\bC)}
\def\sltr{\fsl_2(\bR)}

% \def\Jslt{{J_{\fslt}}}
% \def\Lslt{{L_{\fslt}}}
\def\slee{{
\begin{pmatrix}
 0 & 1\\
 0 & 0
\end{pmatrix}
}}
\def\slff{{
\begin{pmatrix}
 0 & 0\\
 1 & 0
\end{pmatrix}
}}\def\slhh{{
\begin{pmatrix}
 1 & 0\\
 0 & -1
\end{pmatrix}
}}
\def\sleei{{
\begin{pmatrix}
 0 & i\\
 0 & 0
\end{pmatrix}
}}
\def\slxx{{\begin{pmatrix}
-\ihalf & \half\\
\phantom{-}\half & \ihalf
\end{pmatrix}}}
% \def\slxx{{\begin{pmatrix}
% -\sqrt{-1}/2 & 1/2\\
% 1/2 & \sqrt{-1}/2
% \end{pmatrix}}}
\def\slyy{{\begin{pmatrix}
\ihalf & \half\\
\half & -\ihalf
\end{pmatrix}}}
\def\slxxi{{\begin{pmatrix}
+\half & -\ihalf\\
-\ihalf & -\half
\end{pmatrix}}}
\def\slH{{\begin{pmatrix}
   0   & -\mathbf i\\
\mathbf i & 0
\end{pmatrix}}
}

\ExplSyntaxOn
\clist_map_inline:nn {J,L,C,X,Y,H,c,e,f,h,}{
  \expandafter\def\csname #1slt\endcsname{{\mathring{#1}}}}
\ExplSyntaxOff


\def\Mop{\fT}

\def\fggJ{\fgg_J}
\def\fggJp{\fgg'_{J'}}

\def\NilGC{\Nil_{\bfG}(\fgg)}
\def\NilGCp{\Nil_{\bfG'}(\fgg')}
\def\Nilgp{\Nil_{\fgg'_{J'}}}
\def\Nilg{\Nil_{\fgg_{J}}}
%\def\NilP'{\Nil_{\fpp'}}
\def\peNil{\Nil^{\mathrm{pe}}}
\def\dpeNil{\Nil^{\mathrm{dpe}}}
\def\nNil{\Nil^{\mathrm n}}
\def\eNil{\Nil^{\mathrm e}}


\NewDocumentCommand{\NilP}{t'}{
\IfBooleanTF{#1}{\Nil_{\fpp'}}{\Nil_\fpp}
}

\def\KS{\mathsf{KS}}
\def\MM{\bfM}
\def\MMP{M}

\NewDocumentCommand{\KTW}{o g}{
  \IfValueTF{#2}{
    \left.\varsigma_{\IfValueT{#1}{#1}}\right|_{#2}}{
    \varsigma_{\IfValueT{#1}{#1}}}
}
\def\IST{\rho}
\def\tIST{\trho}

\NewDocumentCommand{\CHI}{o g}{
  \IfValueTF{#1}{
    {\chi}_{\left[#1\right]}}{
    \IfValueTF{#2}{
      {\chi}_{\left(#2\right)}}{
      {\chi}}
  }
}
\NewDocumentCommand{\PR}{g}{
  \IfValueTF{#1}{
    \mathop{\pr}_{\left(#1\right)}}{
    \mathop{\pr}}
}
\NewDocumentCommand{\XX}{g}{
  \IfValueTF{#1}{
    {\cX}_{\left(#1\right)}}{
    {\cX}}
}
\NewDocumentCommand{\PP}{g}{
  \IfValueTF{#1}{
    {\fpp}_{\left(#1\right)}}{
    {\fpp}}
}
\NewDocumentCommand{\LL}{g}{
  \IfValueTF{#1}{
    {\bfL}_{\left(#1\right)}}{
    {\bfL}}
}
\NewDocumentCommand{\ZZ}{g}{
  \IfValueTF{#1}{
    {\cZ}_{\left(#1\right)}}{
    {\cZ}}
}

\NewDocumentCommand{\WW}{g}{
  \IfValueTF{#1}{
    {\bfW}_{\left(#1\right)}}{
    {\bfW}}
}




\def\gpi{\wp}
\NewDocumentCommand\KK{g}{
\IfValueTF{#1}{K_{(#1)}}{K}}
% \NewDocumentCommand\OO{g}{
% \IfValueTF{#1}{\cO_{(#1)}}{K}}
\NewDocumentCommand\XXo{d()}{
\IfValueTF{#1}{\cX^\circ_{(#1)}}{\cX^\circ}}
\def\bfWo{\bfW^\circ}
\def\bfWoo{\bfW^{\circ \circ}}
\def\bfWg{\bfW^{\mathrm{gen}}}
\def\Xg{\cX^{\mathrm{gen}}}
\def\Xo{\cX^\circ}
\def\Xoo{\cX^{\circ \circ}}
\def\fppo{\fpp^\circ}
\def\fggo{\fgg^\circ}
\NewDocumentCommand\ZZo{g}{
\IfValueTF{#1}{\cZ^\circ_{(#1)}}{\cZ^\circ}}

% \ExplSyntaxOn
% \NewDocumentCommand{\bcO}{t' E{^_}{{}{}}}{
%   \overline{\cO\sb{\use_ii:nn#2}\IfBooleanTF{#1}{^{'\use_i:nn#2}}{^{\use_i:nn#2}}
%   }
% }
% \ExplSyntaxOff

\NewDocumentCommand{\bcO}{t'}{
  \overline{\cO\IfBooleanT{#1}{'}}}

\NewDocumentCommand{\oliftc}{g}{
\IfValueTF{#1}{\boldsymbol{\vartheta} (#1)}{\boldsymbol{\vartheta}}
}
\NewDocumentCommand{\oliftr}{g}{
\IfValueTF{#1}{\vartheta_\bR(#1)}{\vartheta_\bR}
}
\NewDocumentCommand{\olift}{g}{
\IfValueTF{#1}{\vartheta(#1)}{\vartheta}
}
% \NewDocumentCommand{\dliftv}{g}{
% \IfValueTF{#1}{\ckvartheta(#1)}{\ckvartheta}
% }
%\def\dliftv{\vartheta}
\NewDocumentCommand{\tlift}{g}{
\IfValueTF{#1}{\wtvartheta(#1)}{\wtvartheta}
}

\def\slift{\cL}

\def\BB{\bB}


\def\thetaO#1{\vartheta\left(#1\right)}

\def\bbThetav{\check{\mathbbold{\Theta}}}
\def\Thetav{\check{\Theta}}
\def\thetav{\check{\theta}}

\DeclareDocumentCommand{\NN}{g}{
\IfValueTF{#1}{\fN(#1)}{\fN}
}
\DeclareDocumentCommand{\RR}{m m}{
\fR({#1},{#2})
}

%\DeclareMathOperator*{\sign}{Sign}

% \NewDocumentCommand{\lsign}{m}{
% {}^l\mathrm{Sign}(#1)
% }

% \NewDocumentCommand{\bsign}{m}{
% {}^b\mathrm{Sign}(#1)
% }
%
\def\tsign{{}^t\mathrm{Sign}}
\def\lsign{{}^l\mathrm{Sign}}
\def\bsign{{}^b\mathrm{Sign}}
\def\ssign{\mathrm{Sign}}
\NewDocumentCommand{\sign}{m}{
  \mathrm{Sign}(#1)
}

\NewDocumentCommand\lnn{t+ t- g}{
  \IfBooleanTF{#1}{{}^l n^+\IfValueT{#3}{(#3)}}{
    \IfBooleanTF{#2}{{}^l n^-\IfValueT{#3}{(#3)}}{}
  }
}


% Fancy bcO, support feature \bcO'^a_b = \overline{\cO'^a_b}
\makeatletter
\def\bcO{\def\O@@{\cO}\@ifnextchar'\@Op\@Onp}
\def\@Opnext{\@ifnextchar^\@Opsp\@Opnsp}
\def\@Op{\afterassignment\@Opnext\let\scratch=}
\def\@Opnsp{\def\O@@{\cO'}\@Otsb}
\def\@Onp{\@ifnextchar^\@Onpsp\@Otsb}
\def\@Opsp^#1{\def\O@@{\cO'^{#1}}\@Otsb}
\def\@Onpsp^#1{\def\O@@{\cO^{#1}}\@Otsb}
\def\@Otsb{\@ifnextchar_\@Osb{\@Ofinalnsb}}
\def\@Osb_#1{\overline{\O@@_{#1}}}
\def\@Ofinalnsb{\overline{\O@@}}

% Fancy \command: \command`#1 will translate to {}^{#1}\bfV, i.e. superscript on the
% lift conner.

\def\defpcmd#1{
  \def\nn@tmp{#1}
  \def\nn@np@tmp{@np@#1}
  \expandafter\let\csname\nn@np@tmp\expandafter\endcsname \csname\nn@tmp\endcsname
  \expandafter\def\csname @pp@#1\endcsname`##1{{}^{##1}{\csname @np@#1\endcsname}}
  \expandafter\def\csname #1\endcsname{\,\@ifnextchar`{\csname
      @pp@#1\endcsname}{\csname @np@#1\endcsname}}
}

% \def\defppcmd#1{
% \expandafter\NewDocumentCommand{\csname #1\endcsname}{##1 }{}
% }



\defpcmd{bfV}
\def\KK{\bfK}\defpcmd{KK}
\defpcmd{bfG}
\def\A{\!A}\defpcmd{A}
\def\K{\!K}\defpcmd{K}
\def\G{G}\defpcmd{G}
\def\J{\!J}\defpcmd{J}
\def\L{\!L}\defpcmd{L}
\def\eps{\epsilon}\defpcmd{eps}
\def\pp{p}\defpcmd{pp}
\defpcmd{wtK}
\makeatother

\def\fggR{\fgg_\bR}
\def\rmtop{{\mathrm{top}}}
\def\dimo{\dim^\circ}

\NewDocumentCommand\LW{g}{
\IfValueTF{#1}{L_{W_{#1}}}{L_{W}}}
%\def\LW#1{L_{W_{#1}}}
\def\JW#1{J_{W_{#1}}}

\def\floor#1{{\lfloor #1 \rfloor}}

\def\KSP{K}
\def\UU{\rU}
\def\UUC{\rU_\bC}
\def\tUUC{\widetilde{\rU}_\bC}
\def\OmegabfW{\Omega_{\bfW}}


\def\BB{\bB}


\def\thetaO#1{\vartheta\left(#1\right)}

\def\Thetav{\check{\Theta}}
\def\thetav{\check{\theta}}

\def\Thetab{\bar{\Theta}}

\def\cKaod{\cK^{\mathrm{aod}}}

%G_V's or G
%%%%%%%%%%%%%%%%%%%%%%%%%%%
% \def\GVr{G_{\bfV}}
% \def\tGVr{\wtG_{\bfV}}
% \def\GVpr{G_{\bfV'}}
% \def\tGVpr{\wtG_{\bfV'}}
% \def\GVpr{G_{\abfV}}
% \def\tGVar{\wtG_{\abfV}}
% \def\GV{\bfG_{\bfV}}
% \def\GVp{\bfG_{\bfV'}}
% \def\KVr{K_{\bfV}}
% \def\tKVr{\wtK_{\bfV}}
% \def\KV{\bfK_{\bfV}}
% \def\KaV{\bfK_{\acute{V}}}

% \def\KV{\bfK}
% \def\KaV{\acute{\bfK}}
% \def\acO{\acute{\cO}}
% \def\asO{\acute{\sO}}
%%%%%%%%%%%%%%%%%%%%%%%%%%%
%%%%%%%%%%%%%%%%%%%%%%%%%%%


\def\mstar{{\star}}

\def\GVr{G}
\def\tGVr{\wtG}
\def\GVpr{G'}
\def\tGVpr{\widetilde{G'}}
\def\GVar{G'}
\def\tGVar{\wtG'}
\def\GV{\bfG}
\def\GVp{\bfG'}
\def\KVr{K_{\bfV}}
\def\tKVr{\wtK_{\bfV}}
\def\KV{\bfK_{\bfV}}
\def\KaV{\bfK_{\acute{V}}}

\def\KV{\bfK}
\def\KaV{\acute{\bfK}}
\def\acO{{\cO'}}
\def\asO{{\sO'}}

\DeclareMathOperator{\sspan}{span}

%%%%%%%%%%%%%%%%%%%%%%%%%%%%

\def\sp{{\mathrm{sp}}}

\def\bfLz{\bfL_0}
\def\sOpe{\sO^\perp}
\def\sOpeR{\sO^\perp_\bR}
\def\sOR{\sO_\bR}

\def\ZX{\cZ_{X}}
\def\gdliftv{\vartheta}
\def\gdlift{\vartheta^{\mathrm{gen}}}
\def\bcOp{\overline{\cO'}}
\def\bsO{\overline{\sO}}
\def\bsOp{\overline{\sO'}}
\def\bfVpe{\bfV^\perp}
\def\bfEz{\bfE_0}
\def\bfVn{\bfV^-}
\def\bfEzp{\bfE'_0}

\def\totimes{\widehat{\otimes}}
\def\dotbfV{\dot{\bfV}}

\def\aod{\mathrm{aod}}
\def\unip{\mathrm{unip}}


\def\ssP{{\ddot\cP}}
\def\ssD{\ddot{\bfD}}
\def\ssdd{\ddot{\bfdd}}
\def\phik{\phi_{\fkk}}
\def\phikp{\phi_{\fkk'}}
%\def\bbfK{\breve{\bfK}}
\def\bbfK{\wtbfK}
\def\brrho{\breve{\rho}}

\def\whAX{\widehat{A_X}}
\def\mktvvp{\varsigma_{{\bf V},{\bf V}'}}

\def\Piunip{\Pi^{\mathrm{unip}}}
\def\cf{\emph{cf.} }
\def\Groth{\mathrm{Groth}}
\def\Irr{\mathrm{Irr}}

\def\edrc{\mathrm{DRC}^{\mathrm e}}
\def\drc{\mathrm{DRC}}
\def\drcs{\mathrm{DRC}^{s}}
\def\drcns{\mathrm{DRC}^{ns}}
\def\LS{\mathrm{LS}}
\def\LLS{\mathrm{{}^{\ell} LS}}
\def\LSaod{\mathrm{LS^{aod}}}
\def\Unip{\mathrm{Unip}}
\def\lUnip{\mathrm{{}^{\ell}Unip}}
\def\tbfxx{\tilde{\bfxx}}
\def\PBPe{\mathrm{PBP}^{\mathrm{ext}}}
\def\PBPes{\mathrm{PBP}^{\mathrm{ext}}_{\star}}
\def\PBPesp{\mathrm{PBP}^{\mathrm{ext}}_{\star'}}
\def\pbp{\mathrm{PBP}}
\def\pbpst{\mathrm{PBP}_{\star}}
\def\pbpssp{\pbp_{\star}^{\mathrm{ps}}}
\def\pbpsns{\pbp_{\star}^{\mathrm{ns}}}
\def\pbpsp{\pbp^{\mathrm{ps}}}
\def\pbpns{\pbp^{\mathrm{ns}}}
\def\DDn{\DD_{\mathrm{naive}}}
\newcommand{\noticed}{noticed }
\newcommand{\ess}{essential }

\def\dsrcd{\set{\bullet,s,r,c,d}}
\def\taupna{{\tau^{\prime}_{\mathrm{naive}}}}
\def\tauna{{\tau_{\mathrm{naive}}}}

% Ytableau tweak
\makeatletter
\pgfkeys{/ytableau/options,
  noframe/.default = false,
  noframe/.is choice,
  noframe/true/.code = {%
    \global\let\vrule@YT=\vrule@none@YT
    \global\let\hrule@YT=\hrule@none@YT
  },
  noframe/false/.code = {%
    \global\let\vrule@YT=\vrule@normal@YT
    \global\let\hrule@YT=\hrule@normal@YT
  },
  noframe/on/.style = {noframe/true},
  noframe/off/.style = {noframe/false},
}

\def\hrule@enon@YT{%
  \hrule width  \dimexpr \boxdim@YT + \fboxrule *2 \relax
  height 0pt
}
\def\vrule@enon@YT{%
  \vrule height \dimexpr  \boxdim@YT + \fboxrule\relax
     width \fboxrule
}

\def\enon{\omit\enon@YT}
\newcommand{\enon@YT}[2][clear]{%
  \def\thisboxcolor@YT{#1}%
  \let\hrule@YT=\hrule@enon@YT
  \let\vrule@YT=\vrule@enon@YT
  \startbox@@YT#2\endbox@YT
  \nullfont
}

\makeatother
%\ytableausetup{noframe=on,smalltableaux}
\ytableausetup{noframe=off,boxsize=1.3em}
\let\ytb=\ytableaushort

\newcommand{\tytb}[1]{{\tiny\ytb{#1}}}

\makeatletter
\newcommand{\dotminus}{\mathbin{\text{\@dotminus}}}

\newcommand{\@dotminus}{%
  \ooalign{\hidewidth\raise1ex\hbox{.}\hidewidth\cr$\m@th-$\cr}%
}
\makeatother


\def\ckcOp{\ckcO^{\prime}}
\def\ckcOpp{\ckcO^{\prime\prime}}

\def\cOp{\cO^{\prime}}
\def\cOpp{\cO^{\prime\prime}}
\def\cLpp{\cL^{\prime\prime}}
\def\cLppp{\cL^{\prime\prime\prime}}
\def\pUpsilon{\Upsilon^+}
\def\nUpsilon{\Upsilon^-}
\def\pcL{\cL^+}
\def\ncL{\cL^-}
\def\pcP{\cP^+}
\def\ncP{\cP^-}
% \def\pcE{\cE^+}
% \def\ncE{\cE^-}
\def\pcC{\cC^+}
\def\ncC{\cC^-}
\def\pcLp{\cL^{\prime+}}
\def\ncLp{\cL^{\prime-}}
\def\pcLpp{\cL^{\prime\prime+}}
\def\ncLpp{\cL^{\prime\prime-}}
\def\pcB{\cB^+}
\def\ncB{\cB^-}
\def\uptaup{\uptau^{\prime}}
\def\uptaupp{\uptau^{\prime\prime}}
\def\uptauppp{\uptau^{\prime\prime\prime}}
\def\bdelta{{\bar{\delta}}}
\def\tcO{\tilde{\cO}}
\def\tcOp{\tcO^{\prime}}
\def\tcOpp{\tcO^{\prime\prime}}
\def\tuptau{\tilde{\uptau}}
\def\tuptaup{\tuptau^{\prime}}
\def\tuptaupp{\tuptau^{\prime\prime}}
\def\tuptauppp{\tuptau^{\prime\prime\prime}}
\def\taup{\tau^{\prime}}
\def\taupp{\tau^{\prime\prime}}
\def\tauppp{\tau^{\prime\prime\prime}}
\def\cpT{\cT^+}
\def\cnT{\cT^-}
\def\cpB{\cB^+}
\def\cnB{\cB^-}
\def\BOX{\mathrm{Box}}
\def\ckDD{{\check\DD}}
\def\deltas{\delta^s}
\def\deltans{\delta^{ns}}

\def\PP{\mathrm{PP}}

\def\uum{{\dotminus}}
\def\uup{\divideontimes}
\def\LEG{\mathrm{Leg}}
\def\PBP{\mathrm{PBP}}
\def\BODY{\mathrm{Body}}
\def\eee{\emptyset}


\def\upp{{\rotatebox[origin=c]{45}{$+$}}}
\def\umm{{\rotatebox[origin=c]{45}{$-$}}}

\usepackage{subfiles}



\title[]{Special unipotent representations : orthogonal and symplectic groups}

\author [D. Barbasch] {Dan M. Barbasch}
\address{the Department of Mathematics\\
  310 Malott Hall, Cornell University, Ithaca, New York 14853 }
\email{dmb14@cornell.edu}

\author [J.-J. Ma] {Jia-jun Ma}
\address{School of Mathematical Sciences\\
  Shanghai Jiao Tong University\\
  800 Dongchuan Road, Shanghai, 200240, China} \email{hoxide@sjtu.edu.cn}


\author [B. Sun] {Binyong Sun}
% MCM, HCMS, HLM, CEMS, UCAS,
%\address{Academy of Mathematics and Systems Science\\
%  Chinese Academy of Sciences\\
 % Beijing, 100190, China} \email{sun@math.ac.cn}

\address{Institute for Advanced Study in Mathematics\\
  Zhejiang University\\
  Hangzhou, 310058, China} \email{sunbinyong@zju.edu.cn}


\author [C.-B. Zhu] {Chen-Bo Zhu}
\address{Department of Mathematics\\
  National University of Singapore\\
  10 Lower Kent Ridge Road, Singapore 119076} \email{matzhucb@nus.edu.sg}




\subjclass[2000]{22E45, 22E46} \keywords{orbit method, unitary dual, special unipotent
  representation, classical group, theta lifting, moment map}

\begin{document}
% \thanks{Supported by NSFC Grant 11222101}

\begin{abstract} Let $G$ be a real classical group of type $B$, $C$, $D$ (including the real metaplectic group). We consider a nilpotent adjoint orbit $\check \CO$ of $\check G$, the Langlands dual of $G$ (or the metaplectic dual of $G$ when $G$ is a real metaplectic group). We classify all special unipotent representations of $G$ attached to $\check \CO$, in the sense of Barbasch and Vogan. When $\check \CO$ is of good parity, we construct all such representations of $G$ via the method of theta lifting. As a consequence of the construction and the classification, we conclude that all special unipotent representations of $G$ are unitarizable, as predicted by the Arthur-Barbasch-Vogan conjecture.
\end{abstract}



\maketitle


\tableofcontents


\section{Introduction and the main results}\label{sec:intro}


\subsection{Unitary representations and the orbit method}
A fundamental problem in representation theory is to determine the unitary dual
of a given Lie group $G$, namely the set of equivalent classes of irreducible
unitary representations of $G$. A principal idea, due to Kirillov and Kostant,
is that there is a close connection between irreducible unitary representations
of $G$ and the orbits of $G$ on the dual of its Lie algebra \cite{Ki62,Ko70}.
This is known as orbit method (or the method of coadjoint orbits). Due to its
resemblance with the process of attaching a quantum mechanical system to a
classical mechanical system, the process of attaching a unitary representation
to a coadjoint orbit is also referred to as quantization in the representation
theory literature.

As it is well-known, the orbit method has achieved tremendous success in the
context of nilpotent and solvable Lie groups \cite{Ki62,AK}. For more general
Lie groups, work of Mackey and Duflo \cite{Ma,Du82} suggest that one should
focus attention on reductive Lie groups. As expounded by Vogan in his writings
(see for example \cite{VoBook,Vo98,Vo00}), the problem finally is to quantize
nilpotent coadjoint orbits in reductive Lie groups. The ``corresponding''
unitary representations are called unipotent representations.

Significant developments on the problem of unipotent representations occurred in
the 1980's. We highlight two. Motivated by Arthur's conjectures on unipotent
representations in the context of automorphic forms \cite{ArPro,ArUni}, Adams,
Barbasch and Vogan established some important local consequences for the unitary
representation theory of the group $G$ of real points of a connected reductive
algebraic group defined over $\R$. See \cite{ABV}. The problem of classifying
(integral) special unipotent representations for complex semisimple groups was
solved earlier by Barbasch and Vogan \cite{BVUni} and the unitarity of these representations
was established by Barbasch for complex classical groups \cite[Section 10]{B.Class}. Shortly after,
Barbasch outlined a proof of the unitarity of special unipotent
representations for real classical groups in his 1990 ICM talk \cite{B.Uni}. The
second major development is Vogan's theory of associated varieties \cite{Vo89}
in which Vogan pursues the method of coadjoint orbits by investigating the
relationship between a Harish-Chandra module and its associate variety. Roughly
speaking, the Harish-Chandra module of a representation ``attached'' to a nilpotent
coadjoint orbit should have a simple structure after taking the ``classical
limit'', and it should have a specified support dictated by the nilpotent
coadjoint orbit via the Kostant-Sekiguchi correspondence.

Simultaneously but in an entirely different direction, there were significant
developments in Howe's theory of (local) theta lifting and it was clear by the
end of 1980's that the theory has much relevance for unitary representations of
classical groups. The relevant works include the notion of rank by Howe
\cite{HoweRank}, the description of discrete spectrum by Adams \cite{Ad83} and Li \cite{Li90},
and the preservation of unitarity in stable range theta lifting by Li \cite{Li89}.
Therefore it was natural, and there were many attempts, to link the orbit method
with Howe's theory, and in particular to construct unipotent representations in
this formalism. See for example \cite{Sa,Pz,HZ,HL,Br,He,Tr,PT,B17}. We would also like
to mention the work of Przebinda \cite{Pz} in which a double
fiberation of moment maps made its appearance in the context of theta lifting,
and the work of He \cite{He} in which an innovative technique called quantum
induction was devised to show the non-vanishing of the lifted representations.
More recently the double fiberation of moment maps was successfully used by a
number of authors to understand refined (nilpotent) invariants of
representations such as associated cycles and generalized Whittaker models
\cite{NOTYK, NZ, GZ, LM}, which among other things demonstrate the tight link
between the orbit method and Howe's theory.

As mentioned earlier, there have been extensive investigations of unipotent representations for real reductive
groups, by Vogan and his collaborators (see e.g. \cite{VoBook,Vo89,ABV}). Nevertheless, the subject remains extraordinarily mysterious thus far.

%in particular on the related problems of classification of the primitive ideals and Fourier inversion of unipotent orbit integrals \cite{BVPri1, BVPri2},
%it is only for unitary groups complete results are known (\emph{cf}. \cite{BV83, Tr}), in which case all such representations may also be described in terms of cohomological induction.

In the present article we will demonstrate that Howe's (constructive) theory has immense implications for the orbit method, and in particular
has near perfect synergy with special unipotent representations (in the case of classical groups). (Barbasch, M{\oe}glin, He and Trapa pursued a similar theme. See
\cite{B17,Mo17,He,Tr}.) We will restrict our attention to a real classical group
$G$ of type $B$, $C$ or $D$ (which in our terminology includes a real metaplectic group),
and we will classify all special unipotent representations of $G$ attached to $\check \CO$, in the sense of Barbasch and
Vogan. Here $\check \CO$ is a nilpotent adjoint orbit of $\check G$, the Langlands dual of $G$ (or the metaplectic dual of $G$ when $G$ is a real metaplectic group \cite{BMSZ1}).
When $\check \CO$ is of good parity \cite{MR}, we will construct all special unipotent representations of $G$ attached to $\check \CO$ via the method of theta lifting.
As a direct consequence of the construction and the classification, we conclude that all special unipotent representations of $G$ are unitarizable, as predicted by the Arthur-Barbasch-Vogan conjecture (\cite[Introduction]{ABV}).
%in particular on the related problems of classification of the primitive ideals and Fourier inversion of unipotent orbit integrals \cite{BVPri1, BVPri2},
%it is only for unitary groups complete results are known (\emph{cf}. \cite{BV83, Tr}), in which case all such representations may also be described in terms of cohomological induction.


\subsection{Special unipotent representations of classical groups of type $B$, $C$ or $D$}\label{secsu}
In this article, we aim to classify special unipotent representations of classical groups of type $B$, $C$ or $D$. As the cases of complex orthogonal groups  and complex symplectic groups are well-understood (see
\cite{BVUni} and \cite{B17}), we will focus on the following groups:
\be\label{typebcd}
  \oO(p,q), \Sp_{2n}(\R), \  \widetilde \Sp_{2n}(\R), \ \Sp(p,q), \  \oO^*(2n),
  \ee
  where $p,q, n\in \BN:=\{0,1,2, \cdots\}$. Here $ \widetilde \Sp_{2n}(\R)$ denotes the real metaplectic group, namely the double cover of the symplectic group  $\Sp_{2n}(\R)$ that does not split unless $n=0$.

Let $G$ be one of the groups in   \eqref{typebcd}.  As usual, we view $G$ as  a real form of $G_\C$ (or a double cover of a real form of $G_\C$ in the metaplecitic case), where
\[
  G_\C:=
  \left\{
    \begin{array}{ll}
      \oO_{p+q}(\C) , & \hbox{if $G=\oO(p,q)$;} \smallskip\\
    \Sp_{2n}(\C) , & \hbox{if $G=\Sp_{2n}(\R)$ or $\widetilde \Sp_{2n}(\R)$;} \smallskip \\
  \Sp_{2p+2q}(\C) , & \hbox{if $G=\Sp(p,q)$;} \smallskip \\
 \oO_{2n}(\C) , & \hbox{if $G=\oO^*(2n)$.} \\
    \end{array}
  \right.
\]
Write $\fgg_\R$ and $\g$ for the Lie algebras of $G$ and $G_\C$, respectively, and view $\g_\R$ as a real form of $\g$.

Denote $r_\g$ the rank of $\fgg$. Let $W_{r_\g}$ be the subgroup of $\GL_{r_\g}(\C)$ generated
by the permutation matrices and the diagonal matrices with diagonal entries
$\pm 1$.
 %We identify a Cartan subalgebra of $\g_V$ with $\C^{r_V}$, using the standard coordinates.
 Then as usual, Harish-Chandra
isomorphism yields an identification
\be\label{ugz}
  \oU(\g)^{G_\C}=\left(\oS(\C^{r_\g})\right)^{W_{r_\g}}.
\ee
Here and henceforth,  ``$\oU$'' indicates the universal enveloping algebra of a Lie algebra,  a superscript group indicate the
space of invariant vectors under the group action, and  ``$\oS$'' indicates
the symmetric algebra. Unless $G_\C$ is an even orthogonal group,
$\oU(\g)^{G_\C}$ equals the center $\oZ(\g)$ of $\oU(\g)$.
By \eqref{ugz}, we have the following parameterization of  characters of $\oU(\g)^{G_\C}$:
\[
  \Hom_{\mathrm{alg}}(\oU(\g)^{G_\C}, \C)=W_{r_\g} \backslash (\C^{r_\g})^*= W_{r_\g}\backslash \C^{r_\g}
\]
Here ``$ \Hom_{\mathrm{alg}}$" indicates the set of $\C$-algebra homomorphisms, and a superscript ``$\,^*\,$" over a vector space indicates the dual space.

We define the Langlands dual of $G$ to be the complex group
\[
  \check G:=
  \left\{
    \begin{array}{ll}
      \Sp_{p+q-1}(\C) , & \hbox{if $G=\oO(p,q)$ and $p+q$ is odd;} \smallskip\\
    \oO_{p+q}(\C) , & \hbox{if $G=\oO(p,q)$ and $p+q$ is even;} \smallskip \\
    \oO_{2n+1}(\C) , & \hbox{if $G=\Sp_{2n}(\R)$;} \smallskip \\
   \Sp_{2n}(\C) , & \hbox{if $G=\widetilde \Sp_{2n}(\R)$;} \smallskip \\
 \oO_{2p+2q+1}(\C) , & \hbox{if $G=\Sp(p,q)$;} \smallskip \\
 \oO_{2n}(\C) , & \hbox{if $G=\oO^*(2n)$.} \\
    \end{array}
  \right.
\]
 Write $\check \fgg$ for the Lie algebra of $\check G$.

\begin{remark} The authors have defined the notion of metaplectic dual for a real metaplectic group \cite{BMSZ1}. In this article, we have chosen to use the uniform terminology of Langlands dual (rather than metaplectic dual in the case of a real metaplectic group).
\end{remark}

Denote by $\Nil(\check \fgg)$ the set of   $\check G$-orbits  of nilpotent matrices in $\check \fgg$. 
%Likewise,  $\Nil(\fgg)$ denotes the set of   $G_\C$-orbits  of nilpotent matrices in $\fgg$.
%namely the set of all  orbits of nilpotent matrices under the adjoint action of $\check G$ in $\check \g$.
When no confusion is possible, we will not distinguish a nilpotent orbit for $\GL_n(\C)$, $\oO_n(\C)$ or $\Sp_{2n}(\C)$ with its corresponding Young diagram. In particular, the zero orbit is represented by the Young diagram consisting of one nonempty column at most. 

Let  $\check \CO \in \Nil(\check \fgg) $.  It determines a character $\chi(\check \CO): \oU(\g)^{G_\C}\rightarrow \C$ as in what follows. For every  $a\in \bN$, write
\[
  \rho(a):=\left\{ \begin{array}{ll}
                  (1, 2, \cdots, \frac{a-1}{2}), \quad &\textrm{if $a$ is odd;}\\
                    (\frac{1}{2}, \frac{3}{2}, \cdots, \frac{a-1}{2}), \quad &\textrm{if $a$ is even;}\\
                    \end{array}
                 \right.
\]
By convention, $\rho(1)$ and $\rho(0)$ are  the empty sequence.
Write $a_1\geq  a_2\geq \cdots\geq a_s>0$ ($s\geq 0$)  for the row lengths of  $\check \CO$. Define
\be\label{chico}
 \chi(\check \CO):= (\rho( a_1), \rho(a_2),  \cdots, \rho(a_s), 0, 0, \cdots, 0 ),
\ee
to be viewed as a character $\chi(\check \CO): \oU(\g)^{G_\C}\rightarrow \C$.
Here the number of $0$'s is
\[
 \left\lfloor\frac{\textrm{the number of odd rows of the Young diagram of $\check \CO$}}{2}\right\rfloor.
\]


Recall the following well-known result of Dixmier (\cite[Section 3]{Bor}): for every algebraic character $\chi$ of $\oZ(\g)$, there exists a unique maximal ideal of $\oU(\g)$ that contains the kernel of $\chi$. %, to be called the maximal ideal of $\oU(\g)$ with infinitesimal character $\chi$.
As an easy consequence, there is a unique maximal $G_\C$-stable ideal of $\oU(\g)$ that contains the kernel of $\chi(\check \CO)$, for $\check \CO \in \Nil(\check \fgg)$. Write $I_{\check \CO}$ for this ideal.

Recall that a  smooth Fr\'echet representation of  moderate growth  of a real reductive group is called a Casselman-Wallach representation (\cite{Ca89,Wa2}) if its Harish-Chandra module has  finite length. When $G=\widetilde \Sp_{2n}(\R)$ is a metaplectic group, write $\varepsilon_G$ for the non-trivial element in the kernel of the covering map $G\rightarrow \Sp_{2n}(\R)$. Then a representation of $G$ is said to be genuine if $\varepsilon_G$ acts via the scalar multiplication by $-1$. The notion of ``genuine" will be used in similar situations without further explanation.
Following Barbasch and Vogan (\cite{ABV,BVUni}), we make the following definition.

\begin{defn}
%Let $\check \CO\in \Nil(\check \g)$.
Let $\check \CO\in \Nil(\check \g)$. An irreducible Casselman-Wallach representation $\pi$ of $G$  is attached to $\check \CO$ if
\begin{itemize}
\item  $I_{\check \CO}$ annihilates $\pi$; and
\item $\pi $ is genuine if $G$ is a metaplectic group.
\end{itemize}
\end{defn}

Write  $\Unip_{\check \cO}(G)$ for the  set of isomorphism classes of irreducible Casselman-Wallach  representations of $G$ that are attached to $\check \CO$.
We say that an irreducible Casselman-Wallach representation  of $G$ is special unipotent if it is attached to $\check \CO$, for some $\check \CO\in \Nil(\check \g)$.
As mentioned earlier, we will construct all special unipotent representations of $G$, and will show that all of them are unitarizable, as predicted by  the Arthur-Barbasch-Vogan conjecture (\cite[Introduction]{ABV}).



\subsection{Combinatorial construct: painted bipartitions}\label{secbip}

We introduce a symbol $\star$, taking values in $\{B,C,D,\widetilde {C}, C^*, D^*\}$, to specify the type of the groups that we are considering as in \eqref{typebcd}, namely odd real orthogonal groups, real symplectic groups, even real orthogonal groups, real metaplectic groups, quaternionic symplectic groups and quaternionic orthogonal groups, respectively.
%Let $\star\in \{B,C,D,\widetilde {C}, C^*, D^*\}$.

For a Young diagram $\imath$, write
\[
 \mathbf r_1(\imath)\geq \mathbf r_2(\imath)\geq \mathbf r_3(\imath)\geq \cdots
\]
for its row lengths, and similarly,
write
\[
 \mathbf c_1(\imath)\geq \mathbf c_2(\imath)\geq \mathbf c_3(\imath)\geq \cdots
\]
for its column lengths.
Denote by $\abs{\imath}:=\sum_{i=1}^\infty \mathbf r_i(\imath)$ the total size of $\imath$.



For any Young diagram $\imath$, we introduce the set $\BOX(\imath)$ of boxes of $\imath$ as the following subset
of $\bN^+\times \bN^+$ ($\bN^+$ denotes the set of positive integers):
\begin{equation}\label{eq:BOX}
\BOX(\imath):=\Set{(i,j)\in\bN^+\times \bN^+| j\leq \bfrr_i(\imath)}.
\end{equation}
%We will also call a subset of $\bN^+\times \bN^+$  of the form \eqref{eq:BOX} a Young diagram.

%We say that a Young diagram $\imath'$ is contained
%in $\imath$ (and write $\imath'\subset \imath$) if
%\[
%  \mathbf r_i(\imath')\leq \mathbf r_i(\imath)\qquad \textrm{for all } i=1,2, 3, \cdots.
%\]
%When  this is the case, $\mathrm{Box}(\imath')$ is viewed as a subset of $\mathrm{Box}(\imath)$ concentrating on the upper-left corner.
%We say that a subset of $\mathrm{Box}(\imath)$ is a Young subdiagram if it equals $\mathrm{Box}(\imath')$ for a Young diagram $\imath'\subset \imath$.
%  In this case, we call $\imath'$ the Young diagram corresponding to this Young subdiagram.

We also introduce five symbols $\bullet$, $s$, $r$, $c$ and $d$, and make the following definition.
\begin{defn}
A painting on a Young diagram $\imath$ is a map
\[
  \CP: \mathrm{Box}(\imath) \rightarrow \{\bullet, s, r, c, d \}
\]
with the following properties:
\begin{itemize}
\item
 $\CP^{-1}(S)$ is the set of boxes of a Young diagram when $S=\{\bullet\}, \{\bullet, s \}, \{\bullet, s, r\}$ or $\{\bullet, s, r, c \} $;
 \item
 when $S=\{s\}$ or $ \{r\}$, every row of $\imath$ has at most one  box in $\CP^{-1}(S)$;
   \item
 when $S=\{c\}$ or $ \{d \}$, every column of $\imath$ has at most one  box in $\CP^{-1}(S)$.
 \end{itemize}
A painted Young diagram is then a pair $(\imath, \CP)$, consisting of a Young diagram $\imath$ and a painting $\CP$ on $\imath$.

\end{defn}


\begin{Example} Suppose that $\imath=\tytb{\ \ ,\  }$, then there are $25+12+6+2=45$ paintings on $\imath$ in total as listed below.
\begin{equation*}\label{eq:sp-nsp.C}
\begin{array}{ll}
   \tytb{\bullet \alpha ,\beta } \quad \alpha, \beta\in \{\bullet, s,r,c,d\} \qquad \qquad  \qquad  & \tytb{s \alpha ,\beta } \quad \alpha\in \{r,c,d\}, \beta\in \{s,r,c,d\} \medskip \medskip \\
     \tytb{r \alpha ,\beta } \quad \alpha\in \{c,d\}, \beta\in \{r,c,d\} \qquad \qquad  \qquad
   &  \tytb{c \alpha , d } \quad  \alpha\in \{c,d\}
     \end{array}
  \end{equation*}

  \end{Example}


When no confusion is possible, we write $\alpha\times \beta\times \gamma$ for a triple $(\alpha, \beta, \gamma)$.   We introduce two more symbols $B^+$ and $B^-$, and make the following definition.
 \begin{defn}\label{defpbp0}
 A painted bipartition is a triple $\tau=(\imath, \CP)\times (\jmath, \CQ)\times \alpha$, where $(\imath, \CP)$ and $ (\jmath, \CQ)$ are painted Young diagrams, and $\alpha\in \{B^+,B^-, C,D,\widetilde {C}, C^*, D^*\}$, subject to the following conditions:
 \begin{itemize}
  \delete{\item
 $(\imath, \jmath)\in \mathrm{BP}_\alpha$ if $\alpha\notin\{B^+,B^-\}$, and  $(\imath, \jmath)\in \mathrm{BP}_{B}$ if $\alpha\in\{B^+,B^-\}$;}

 \item
 $\CP^{-1}(\bullet)=\CQ^{-1}(\bullet)$;
 \item
 the image of $\CP$ is contained in
 \[
 \left\{
     \begin{array}{ll}
         \{\bullet, c\}, &\hbox{if $\alpha=B^+$ or $B^-$}; \smallskip\\
            \{\bullet,  r, c,d\}, &\hbox{if $\alpha=C$}; \smallskip\\
          \{\bullet, s, r, c,d\}, &\hbox{if $\alpha=D$}; \smallskip\\
            \{\bullet, s, c\}, &\hbox{if $\alpha=\widetilde{ C}$}; \smallskip \\
        \{\bullet\}, &\hbox{if $\alpha=C^*$}; \smallskip \\
          \{\bullet, s\}, &\hbox{if $\alpha=D^*$},\\
            \end{array}
   \right.
 \]
 \item
 the image of $\CQ$ is contained in
 \[
 \left\{
     \begin{array}{ll}
         \{\bullet, s, r, d\}, &\hbox{if $\alpha=B^+$ or $B^-$}; \smallskip\\
           \{\bullet, s\}, &\hbox{if $\alpha=C$}; \smallskip\\
           \{\bullet\}, &\hbox{if $\alpha=D$}; \smallskip\\
             \{\bullet, r, d\}, &\hbox{if $\alpha=\widetilde{ C}$}; \smallskip\\
        \{\bullet, s,r\}, &\hbox{if $\alpha=C^*$}; \smallskip \\
          \{\bullet, r\}, &\hbox{if $\alpha=D^*$}.
            \end{array}
   \right.
 \]

 \end{itemize}
 \end{defn}

  %\begin{remark}
 %The set of painted bipartition counts the multiplicities of an irreducible representation of $W_{r_{\fgg}}$ occurs in the coherent continuation representation at the infinitesimal character of the trivial representation.
%For the relationship between painted bipartitions and the coherent continuation representations of Harish-Chandra modules, see \cite{Mc}.
%\end{remark}

For any painted bipartition $\tau$ as in Definition \ref{defpbp0}, we write
\[
  \imath_\tau:=\imath,\ \cP_\tau:=\cP,\  \jmath_\tau:=\jmath,\  \cQ_\tau:=\cQ,\ \alpha_\tau:=\alpha,
\]
and
\[
  \star_\tau:= \left\{
     \begin{array}{ll}
         B, &\hbox{if $\alpha=B^+$ or $B^-$}; \smallskip\\
            \alpha, & \hbox{otherwise}.           \end{array}
   \right.
  \]
%Its leading column is then defined to be the first column of $(\jmath, \CQ)$ when $\star_\tau\in \{B, C,C^*\}$,
%and the first column of  $(\imath, \CP)$ when $\star_\tau\in \{\widetilde C, D, D^*\}$.

We further attach some objects to $\tau$ in what follows:
 \[
  \abs{\tau}, \    (p_\tau, \ q_\tau), \ G_\tau, \ \dim \tau, \ \varepsilon_\tau .
 \]

%   \noindent $\check \CO_\tau$ and $ \wp_\tau$: This is the unique pair such that
%   \begin{itemize}
%   \item
%   $\check \CO_\tau$ is a Young diagram that has $\star_\tau$-good parity;
%   \item $\wp_\tau\subset \mathrm{PP}_{\star_\tau}(\check \CO_\tau)$; and
%   \item
%   $
%   (\imath, \jmath)=(\imath_{\star_\tau}(\check \CO_\tau, \wp_\tau), \jmath_{\star_\tau}(\check \CO_\tau, \wp_\tau)).
%$
%\end{itemize}
%The existence of such a pair is a consequence of the explicit description of the coherent continuation representations. See   \cite[Theorems 6, 7, 10, 11]{Mc} and
%\cite[Proposition 6.9]{RT}


\smallskip

%   \noindent $\check \CO_\tau$ and $ \wp_\tau$: This is the unique pair such that
%   \begin{itemize}
%   \item
%   $\check \CO_\tau$ is a Young diagram that has $\star_\tau$-good parity;
%   \item $\wp_\tau\subset \mathrm{PP}_{\star_\tau}(\check \CO_\tau)$; and
%   \item
%   $
%   (\imath, \jmath)=(\imath_{\star_\tau}(\check \CO_\tau, \wp_\tau), \jmath_{\star_\tau}(\check \CO_\tau, \wp_\tau)).
%$
%\end{itemize}
%The existence of such a pair is a consequence of the explicit description of the coherent continuation representations. See   \cite[Theorems 6, 7, 10, 11]{Mc} and 
%\cite[Proposition 6.9]{RT} 

 \noindent $\abs{\tau}$ : This is the natural number \[
  \abs{\tau}:=\abs{\imath}+\abs{\jmath}.
\]
\delete{Note that
\[
 \abs{\tau}= \left\{
     \begin{array}{ll}
        \frac{\abs{\check \CO}-1}{2}, &\hbox{if $\star_\tau=C$ or $C^*$}; \smallskip\\
          \frac{\abs{\check \CO}}{2}, &\hbox{otherwise}. \smallskip\\
                      \end{array}
   \right.
\]
}


 

 \smallskip


 \smallskip



  \noindent $(p_{\tau}, q_{\tau})$ : If $\star_\tau\in \{B, D, C^*\}$, this is a pair of natural numbers given by counting  the various symbols appearing in $(\imath, \CP)$, $(\jmath, \CQ)$ and $\{\alpha\}$ :
  \begin{equation}\label{ptqt}
  \left\{
     \begin{array}{l}
    p_\tau :=( \# \bullet)+ 2 (\# r) +(\# c )+ (\# d) + (\# B^+);\smallskip\\
    q_\tau :=( \# \bullet)+ 2 (\# s) + (\# c) + (\# d) + (\# B^-).\\
    \end{array}
    \right.
\end{equation}
Here 
\[
\#\bullet:=\#(\cP^{-1}(\bullet))+\#(\cQ^{-1}(\bullet))\qquad (\textrm{$\#$ indicates the cardinality of a finite set}),
\] 
and the other terms are similarly defined. 
If $\star_\tau\in \{C, \widetilde C, D^*\}$,  $p_\tau:=q_\tau:=\abs{\tau}$. 

\smallskip


 \smallskip


  \noindent $G_{\tau}$: This is a classical group given by
  \[
 G_\tau:= \left\{
     \begin{array}{ll}
         \oO(p_\tau, q_\tau), &\hbox{if $\star_\tau=B$ or $D$}; \smallskip\\
            \Sp_{2\abs{\tau}}(\R), &\hbox{if $\star_\tau=C$}; \smallskip\\
           \widetilde{\Sp}_{2\abs{\tau}}(\R), &\hbox{if $\star_\tau=\widetilde{ C}$}; \smallskip \\
        \Sp(\frac{p_\tau}{2}, \frac{q_\tau}{2}), &\hbox{if $\star_\tau=C^*$}; \smallskip \\
          \oO^*(2\abs{\tau}), &\hbox{if $\star_\tau=D^*$}.\\
            \end{array}
   \right.
\]

\smallskip


 \smallskip

  \noindent $\dim \tau$:
This is the dimension of the standard representation of the complexification of $G_\tau$, or equivalently, 
 \[
 \dim \tau:= \left\{
     \begin{array}{ll}
          2\abs{ \tau}+1, &\hbox{if $\star_\tau=B$}; \medskip\\
         2 \abs{\tau}, &\hbox{otherwise}.
            \end{array}
   \right.
 \]

\smallskip


 \smallskip

  \noindent $\varepsilon_\tau$:
This is the element in $\Z/2\Z$ such that 
\[
  \varepsilon_\tau=0\Leftrightarrow  \textrm{the symbol $d$ occurs in the first column of $( \imath, \cP)$ or $(\jmath, \cQ)$}. 
\]

 \smallskip


\delete{
\begin{description}
\item [$\abs{\tau}$] This is the natural number \[
  \abs{\tau}:=\abs{\imath}+\abs{\jmath}.
\]
\delete{Note that
\[
 \abs{\tau}= \left\{
     \begin{array}{ll}
        \frac{\abs{\check \CO}-1}{2}, &\hbox{if $\star_\tau=C$ or $C^*$}; \smallskip\\
          \frac{\abs{\check \CO}}{2}, &\hbox{otherwise}. \smallskip\\
                      \end{array}
   \right.
\]
}

\smallskip
\smallskip

\item [$(p_{\tau}, q_{\tau})$] If $\star_\tau\in \{B, D, C^*\}$, this is a pair of natural numbers given by counting the various symbols appearing in $(\imath, \CP)$, $(\jmath, \CQ)$ and $\{\alpha\}$ :
  \[
  \left\{
     \begin{array}{l}
    p_\tau := \# \bullet+ 2 \# r + \# c + \# d + \# B^+;\smallskip\\
    q_\tau := \# \bullet+ 2 \# s + \# c + \# d + \# B^-.\\
    \end{array}
    \right.
\]
If $\star_\tau\in \{C, \widetilde C, D^*\}$,  we let $p_\tau:=q_\tau:=\abs{\tau}$.

\smallskip
\smallskip

\item [$G_{\tau}$] This is a classical group given by
  \[
 G_\tau:= \left\{
     \begin{array}{ll}
         \oO(p_\tau, q_\tau), &\hbox{if $\star_\tau=B$ or $D$}; \smallskip\\
            \Sp_{2\abs{\tau}}(\R), &\hbox{if $\star_\tau=C$}; \smallskip\\
           \widetilde{\Sp}_{2\abs{\tau}}(\R), &\hbox{if $\star_\tau=\widetilde{ C}$}; \smallskip \\
        \Sp(\frac{p_\tau}{2}, \frac{q_\tau}{2}), &\hbox{if $\star_\tau=C^*$}; \smallskip \\
          \oO^*(2\abs{\tau}), &\hbox{if $\star_\tau=D^*$}.\\
            \end{array}
   \right.
\]

\smallskip
\smallskip

\item [$\dim \tau$]
This is the dimension of the standard representation of the complexification of $G_\tau$, or equivalently,
 \[
 \dim \tau:= \left\{
     \begin{array}{ll}
          2\abs{ \tau}+1, &\hbox{if $\star_\tau=B$}; \medskip\\
         2 \abs{\tau}, &\hbox{otherwise}.
            \end{array}
   \right.
 \]

\smallskip
\smallskip

\item [$\varepsilon_\tau$]
This is the element in $\Z/2\Z$ such that
\[
  \varepsilon_\tau=0\Leftrightarrow  \textrm{the symbol $d$ occurs in the first column of $( \imath, \cP)$ or $(\jmath, \cQ)$}.
\]

\end{description}
}

The triple $\mathsf s_{\tau}=(\star_{\tau}, p_{\tau},q_{\tau})\in  \{B,C,D, \widetilde C, C^*, D^*\}\times \bN\times \bN$ will also be referred to as the classical signature attached to $\tau$.

\smallskip

\begin{Example} Suppose that
\[
\tau= \tytb{\bullet c ,\bullet , c,\none } \times \tytb{\bullet s r ,\bullet ,r, r }\times B^+.
\]
Then
\[
\begin{cases}
\abs{\tau}=10;\\
 p_\tau=4+6+2+0+1=13; \quad\\
 q_\tau=4+2+2+0+0=8; \quad \\
 G_\tau=\oO(13,8);\quad \\
 \dim \tau=21;\\
 \varepsilon_\tau=1.
 \end{cases}
 \]

\end{Example}


  \subsection{Counting special unipotent representations by painted bipartitions}\label{secbip}

We fix a classical group $G$ which has type $\star$.
  %Then $\check G$ is a complex symplectic group if $\star=B$ or $\widetilde{C}$, and is a complex orthogonal group otherwise.
Following \cite[Definition 4.1]{MR}, we say that
    $\check \CO\in \Nil(\check \g)$ has  $\star$-good parity if
\[
  \left\{ \begin{array}{l}
               \textrm{all nonzero row lengths of $\check \CO$ are even if $\check G$ is a complex symplectic group; and }\\
                     \textrm{all nonzero row lengths of $\check \CO$ are odd if $\check G$ is a complex orthogonal group.}
                       % \textrm{the total size $\abs{\check \CO}$ is  odd if and only if $\star\in \{C, C^*\}$}.
                          \end{array}
                 \right.
\]
%We will also not distinguish a Young diagram with the corresponding partition of a natural number.
%Let $\mathrm{YD}$ denote the set of all Young diagrams.
In general, the study of the special unipotent representations attached to $\check \CO$ will be reduced to the case when $\check \CO$ has $\star$-good parity. We refer the reader to \cite{BMSZ2}.
%Appendix \ref{secapp}.

For the rest of this subsection, assume that $\check \CO$ has  $\star$-good parity. Equivalently we consider $\check \CO$ as a Young diagram that has $\star$-good parity in the following sense:
\[
  \left\{ \begin{array}{l}
               \textrm{all nonzero row lengths of $\check \CO$ are even if $\star\in \{B, \widetilde{C}\}$;}\\
                     \textrm{all nonzero row lengths of $\check \CO$ are odd if $\star\in \{C, D, C^*, D^*\}$; and}\\
                        \textrm{the total size $\abs{\check \CO}$ is  odd if and only if $\star\in \{C, C^*\}$}.                   \end{array}
                 \right.
\]
%In particular, the empty Young diagram $\emptyset$ is $\star$-even.

%We call an ordered pair of Young diagrams a  bipartition. In what follows we will define a  set   $\mathrm{BP}_\star(\check \CO)$ of bipartitions.

\begin{defn}
 A $\star$-pair is a pair  $(i,i+1)$ of consecutive positive integers such that
\[
   \left\{
     \begin{array}{ll}
      i\textrm{ is odd}, \quad &\textrm{if $\star\in\{C, \widetilde{C}, C^*\}$};  \\
      i \textrm{ is even}, \quad &\textrm{if $\star\in\{B, D, D^*\}$}. \\
       \end{array}
   \right.
\]
A $\star$-pair   $(i,i+1)$ is said to be
\begin{itemize}
\item
vacant in $\check \CO$, if $\mathbf r_i(\check \CO)=\mathbf r_{i+1}(\check \CO)=0$;
\item
balanced in $\check \CO$,  if  $\mathbf r_i(\check \CO)=\mathbf r_{i+1}(\check \CO)>0$;
\item
tailed in $\check \CO$,  if  $\mathbf r_i(\check \CO)-\mathbf r_{i+1}(\check \CO)$ is positive and odd;
\item
primitive in $\check \CO$, if    $\mathbf r_i(\check \CO)-\mathbf r_{i+1}(\check \CO)$ is positive and even.
\end{itemize}
Denote $\mathrm{PP}_\star(\check \CO)$ the  set of all $\star$-pairs that are primitive in $\check \CO$.
\end{defn}

We attach to $\check \CO$ a pair of Young diagrams
\[
(\imath_{\check \CO}, \jmath_{\check \CO}):=(\imath_\star(\check \CO), \jmath_\star(\check \CO)),
\]
 as follows.

\medskip

\noindent
{\bf The case when $\star=B$.} In this case,
 \[
   \mathbf c_{1}(\jmath_{\check \CO})=\frac{\mathbf r_1(\check \CO)}{2},
\]
and for all $i\geq 1$,
\[
\left (\mathbf c_{i}(\imath_{\check \CO}), \mathbf c_{i+1}(\jmath_{\check \CO})\right )=
            \left (\frac{\mathbf r_{2i}(\check \CO)}{2},  \frac{\mathbf r_{2i+1}(\check \CO)}{2}\right ).
\]

\medskip

\noindent
{\bf The case when $\star=\widetilde C$.} In this case, for all $i\geq 1$,
\[
(\mathbf c_{i}(\imath_{\check \CO}), \mathbf c_{i}(\jmath_{\check \CO}))=
           \left (\frac{\mathbf r_{2i-1}(\check \CO)}{2},  \frac{\mathbf r_{2i}(\check \CO)}{2}\right).
\]

\medskip

\noindent
{\bf The case when $\star\in \{C,C^*\}$.} In this case, for all $i\geq 1$,
\[
(\mathbf c_{i}(\jmath_{\check \CO}), \mathbf c_{i}(\imath_{\check \CO}))=
   \left\{
     \begin{array}{ll}
        (0,  0), &\hbox{if $(2i-1, 2i)$ is vacant  in $\check \CO$};\smallskip\\
        (\frac{\mathbf r_{2i-1}(\check \CO)-1}{2},  0), & \hbox{if $(2i-1, 2i)$ is tailed in $\check \CO$};\smallskip\\
                  (\frac{\mathbf r_{2i-1}(\check \CO)-1}{2},  \frac{\mathbf r_{2i}(\check \CO)+1}{2}), &\hbox{otherwise}.\\
            \end{array}
   \right.
\]
\medskip

\noindent
{\bf The case when $\star\in \{D,D^*\}$.} In this case,
 \[
   \mathbf c_{1}(\imath_{\check \CO})= \left\{
     \begin{array}{ll}
      0,  &\hbox{if $\mathbf r_1(\check \CO)=0$}; \smallskip\\
       \frac{\mathbf r_1(\check \CO)+1}{2},   &\hbox{if $\mathbf r_1(\check \CO)>0$},\\
            \end{array}
   \right.
 \]
and for all $i\geq 1$,
\[
(\mathbf c_{i}(\jmath_{\check \CO}), \mathbf c_{i+1}(\imath_{\check \CO}))=
   \left\{
     \begin{array}{ll}
        (0,  0), &\hbox{if $(2i, 2i+1)$ is vacant in $\check \CO$};\smallskip\\
      \left  (\frac{\mathbf r_{2i}(\check \CO)-1}{2},  0\right ), & \hbox{if $(2i, 2i+1)$ is tailed in $\check \CO$};\smallskip\\
                \left  (\frac{\mathbf r_{2i}(\check \CO)-1}{2},  \frac{\mathbf r_{2i+1}(\check \CO)+1}{2}\right ), &\hbox{otherwise}.\\
            \end{array}
   \right.
\]




\delete{
In all cases, it is known that (see \cite[Section13.2]{Carter})
\be\label{eqbp}
(\imath_\star(\check \CO, \wp), \jmath_\star(\check \CO, \wp))=(\imath_\star(\check \CO', \wp'), \jmath_\star(\check \CO', \wp'))\quad \textrm{if and only if}\quad
 \left\{
     \begin{array}{l}
           \check \CO=\check \CO';\\
           \wp=\wp',
            \end{array}
   \right.
\ee
where $\check \CO'$ is another  Young diagram that has $\star$-good parity, and $\wp'\subset \mathrm{PP}_\star(\check \CO')$.
}
\delete{Put
\[
\mathrm{BP}_\star(\check \CO) :=\{(\imath_\wp, \jmath_\wp)\mid \wp\subset \mathrm{PP}_\star(\check \CO)\}
\]
and
\[
  \mathrm{BP}_\star:=\bigcup_{\check \CO \textrm{ is Young diagram that has $\star$-good parity} }\mathrm{BP}_\star(\check \CO).
\]
The latter is a disjoint union by \eqref{eqbp}, and its elements are called $\star$-bipartitions.
Recall that  the set $\mathrm{BP}_\star(\check \CO)$ parametrizes certain Weyl group representations (\cite[Section 11.4]{Carter}).
}

%Recall that  the pair $\mathrm{BP}_\star(\check \CO)$ parametrizes certain Weyl group representations (\cite[Section 11.4]{Carter}).

\delete{
\begin{Example} Suppose that $\star=C$, and $\check \CO$ is the following Young diagram which has $\star$-good parity.
\begin{equation*}\label{eq:sp-nsp.C}
  \tytb{\ \ \ \ \  , \ \ \  , \ \ \ , \ \ \  , \ \ \ , \  ,\  }
   \end{equation*}
   Then
\[
  \mathrm{PP}_\star(\check \CO)=\{(1,2), (5,6)\}.
\]
and $(\imath_\wp, \jmath_\wp)$ %\in \mathrm{BP}_\star(\check \CO)$
has the  following form.

\begin{equation*}\label{eq:sp-nsp.C}
\begin{array}{rclcrcl}
  \wp=\emptyset & : & \tytb{\ \ \ ,\ \  } \times \tytb{\ \ \ , \  }  & \qquad \quad &  \wp=\{(1,2)\}& : & \tytb{\ \ \  , \ \ , \   } \times \tytb{\ \ \  } \medskip \medskip \medskip \\
    \wp=\{(5,6)\} & : & \tytb{\ \ \ ,\ \ \ } \times \tytb{\ \ , \   }  & \qquad \quad &  \wp=\{(1,2), (5,6)\}  & : & \tytb{\ \ \  , \ \ \ ,  \ } \times \tytb{\ \   } \\
  \end{array}
  \end{equation*}

\end{Example}


Here and henceforth, when no confusion is possible, we write $\alpha\times \beta$ for a pair $(\alpha, \beta)$.  We will also write $\alpha\times \beta\times \gamma$ for a triple $(\alpha, \beta, \gamma)$.


\subsection{Combinatorics : painted bipartitions}

}



%Write $\mathrm{PBP}$ for the set of all painted bipartitions.


Define
\[
\mathrm{PBP}_\star(\check \CO):=\Set{
\tau\textrm{ is a painted bipartition}  \mid  \star_\tau = \star \text{ and } (\imath_\tau,\jmath_\tau) = (\imath_\star(\check \CO), \jmath_\star(\check \CO))
}.
\]
We also define the following extended parameter set:
\[
\PBPes(\check \CO):=\begin{cases}
\mathrm{PBP}_\star(\check \CO)\times \{\wp\subset \mathrm{PP}_\star(\check \CO)\},\quad& \textrm{if }\star\in \{B,C, D, \widetilde C\};\\
\mathrm{PBP}_\star(\check \CO)\times \{\emptyset \},\quad& \textrm{if }\star\in \{C^*, D^*\}.
\end{cases}
\]
We use $\uptau=(\tau,\wp)$ to denote an element in $\PBPes(\ckcO)$.

Put
\[
  \mathrm{Unip}_{\star}(\check \CO):=\left\{
     \begin{array}{ll}
         \bigsqcup_{p,q\in \BN, p+q=\abs{\check \CO}+1} \Unip_{\check \cO}(\oO(p,q)), &\hbox{if $\star=B$}; \medskip\\
           \Unip_{\check \CO}(\Sp_{\abs{\check \CO}-1}(\R)), &\hbox{if $\star=C$}; \medskip\\
           \bigsqcup_{p,q\in \BN, p+q=\abs{\check \CO}} \Unip_{\check \cO}(\oO(p,q)), &\hbox{if $\star=D$}; \medskip\\
          \Unip_{\check \CO}(\widetilde \Sp_{\abs{\check \CO}}(\R)), &\hbox{if $\star=\widetilde{ C}$}; \medskip \\
     \bigsqcup_{p,q\in \BN, 2p+2q=\abs{\check \CO}-1} \Unip_{\check \cO}(\Sp(p,q)), &\hbox{if $\star=C^*$}; \medskip \\
          \Unip_{\check \CO} (\oO^*(\abs{\check \CO})), &\hbox{if $\star=D^*$}.\\
            \end{array}
   \right.
\]

The main result of \cite{BMSZ2}, on counting of special unipotent representations, is as follows.

\begin{thm}\label{thmcount}
Suppose that $\ckcO\in \Nil(\check\fgg)$ has $\star$-good parity. Then
\[
 \#(\Unip_{\star}({\ckcO}))= \begin{cases}
2  \#(\PBPes(\ckcO)),&\quad  \text{if  $\star\in \{B,D\}$ and $(\star, \check \CO)\neq (D, \emptyset)$;}\\
 \#(\PBPes(\ckcO)),  &\quad  \text{otherwise }.\\
\end{cases}
\]
\end{thm}
% We will use the set $\PBPes(\ckcO)$ to count the set $\mathrm{Unip}_{\star}(\check \CO)$.



\delete{ It equals $0$ in the following cases:
   \begin{itemize}
   \item[(a)]
   $\star_\tau=B$, $\abs{\tau}>0$ and  the symbol $d$ occurs in the first column of  the painted Young diagram $(\jmath_\tau, \CQ_\tau)$;
    \item[(b)]
   $\star_\tau=D$, $\abs{\tau}>0$ and  the symbol $d$ occurs in the first column of  the painted Young diagram $(\imath_\tau, \CP_\tau)$;
   \item[(c)]
   $\star_\tau\in \{C, \widetilde C\}$ and  $(1,2)\not\in \wp_\tau$.
  \end{itemize}
In all other case, $\varepsilon_{\tau}=1$.
}


\subsection{Descending painted bipartitions and constructing representations by induction}
\label{subsec:comTOrep}

In the sequel, we will use $\emptyset$ to denote the empty set (in the usual way), as well as the empty Young diagram or the painted Young diagram whose underlying Young diagram is empty.

Let $\star$ and $\check \CO$ be as before so that $\check \CO$ has $\star$-good parity. Define a symbol
\[
\star':=\widetilde{C}, \ D, \  C, \ B, \ D^*\  \textrm{ or } \ C^*
\]
respectively if
\[
\star=B,\  C, \ D, \ \widetilde{C}, \ C^* \ \textrm{ or }\  D^*.
\]
We call $\star'$
 the Howe dual of $\star$. Define the naive dual descent of $\check \CO$ to be
 \[
  \check \nabla_{\mathrm{naive}}(\check \CO):= \textrm{the Young diagram obtained from $\check \CO$ by removing the first row}.
  \]
   By convention, $\check \nabla_{\mathrm{naive}}(\emptyset)=\emptyset$. The dual descent of $\check \CO$ is defined to be 
  \[
   \check \CO':=\check \nabla(\check \CO):=\check \nabla_\star( \check \CO):=\begin{cases}
   \tytb{\   }\, , \quad& \textrm{if $\star\in \{D, D^*\}$ and $\check \CO=\emptyset$};\smallskip\\
   \check \nabla_{\mathrm{naive}}(\check \CO), \quad & \textrm{otherwise},
    \end{cases}
  \]
  where $\tytb{\   }$ denotes the Young diagram that has total size $1$. 
  
  % Here $\emptyset$ stands for the empty Young diagram, and $\tytb{\ }$ stands for the Young diagram of total size $1$.
 Note that $\check \CO'$ has $\star'$-good parity, and
\[
 \mathrm{PP}_{\star'}(\check \CO')=\{(i,i+1)\mid i\in \bN^+, \, (i+1, i+2)\in \mathrm{PP}_{\star}(\check \CO) \}.
\]
Define the dual descent of a subset $\wp\subset \mathrm{PP}_{\star}(\check \CO)$ to be the subset 
\begin{equation}\label{eq:DD.wp}
  \wp':=\ckDD(\wp):=\{(i,i+1)\mid i\in \bN^+, \, (i+1, i+2)\in \wp \}\subset \mathrm{PP}_{\star'}(\check \CO').
\end{equation}
 In Section \ref{sec:comb}, we will define the descent map
 \[
   \nabla:  \mathrm{PBP}_\star(\check \CO)\rightarrow  \mathrm{PBP}_{\star'}(\check \CO').
 \]
Let  $\uptau = (\tau,\wp)\in \mathrm{\PBPes}(\check \CO)$.  We define its  descent be the element
 \[
  \uptau' := (\tau',\wp'):=\nabla(\uptau):= (\DD(\tau), \ckDD(\wp))\in \mathrm{\PBP}_{\star'}(\check \CO').
 \]

%For $\uptau = (\tau,\wp)$, p
Let $(W_{\tau, \tau'}, \la \,\cdot\,,\,\cdot\,\ra_{\tau, \tau'})$ be a real symplectic space of  dimension $\dim \tau\cdot \dim \tau'$. As usual, there are continuous homomorphisms $G_\tau\rightarrow \Sp(W_{\tau, \tau'})$ and $G_{\tau'}\rightarrow \Sp(W_{\tau, \tau'})$ whose images form a reductive dual pair in $\Sp(W_{\tau, \tau'})$. We form the semidirect product
   \[
   J_{\tau, \tau'}:=(G_\tau\times G_{\tau'})\ltimes \oH(W_{\tau, \tau'}),
   \]
   where
  \[
  \oH(W_{\tau, \tau'}):=W_{\tau, \tau'}\times \R
  \]
  is the Heisenberg group with group multiplication
  \[
  (w,t)(w',t'):=(w+w', t+t'+\la w,w'\ra_{\tau, \tau'}), \quad
  w,w'\in W_{\tau, \tau'},\,\,t,t'\in \R.
 \]

 Let $\omega_{\tau, \tau'}$ be a suitably normalized smooth oscillator representation of $J_{\tau, \tau'}$ such that every $t\in \R\subset J_{\tau, \tau'}$ acts on it through the scalar multiplication by $e^{\sqrt{-1}\, t}$. % (the letter $\pi$  often denotes a representation, but here it stands for the circumference ratio).  
 See Section \ref{secoscil} for details.
 For any Casselman-Wallach representation $\pi'$ of $G_{\tau'}$, write
 \[
   \check \Theta_{\tau'}^{\tau}(\pi'):=(\omega_{\tau, \tau'}\widehat \otimes \pi')_{G_{\tau'}} \qquad (\textrm{the Hausdorff coinvariant space}),
 \]
 where $\widehat \otimes$ indicates the complete projective tensor product. This representation is clearly Fr\'echet, smooth, and of moderate growth, and so a Casselman-Wallach representation by the fundamental result of Howe \cite{Howe89}.


 In what follows we will construct a representation $\pi_{\uptau}$ of $G_\tau$ by the method of theta lifting. For the initial case when $\check \CO$ is the empty Young diagram, define
 \[
 \pi_{\uptau}:= \left\{
     \begin{array}{ll}
                    \textrm{the determinant character} , &\hbox{ if $\alpha_\tau=B^-$ so that $G_\tau=\rO(0,1)$;} \smallskip\\
                    \textrm{the one dimensional genuine representation} , &\hbox{ if $\star=\widetilde C$ so that $G_\tau=\widetilde{\Sp}_0(\R)$;} \smallskip\\
 \textrm{the one dimensional trivial representation} , &\hbox{ otherwise.}            \end{array}
   \right.
\]
When $\check \CO\neq\emptyset$,   we define the representation $\pi_{\uptau}$ of $G_\tau$ by induction on the number of nonempty rows of $\check \CO$:
 \begin{equation}\label{eq:def-pi}
    \pi_{\uptau}:=\left\{
     \begin{array}{ll}
          %\textrm{the trivial representation $\C$}, &\hbox{if $\abs{\check \CO_\tau}\leq 1$}; \medskip\\
         \check \Theta_{\tau'}^{\tau}(\pi_{\uptau'})\otimes (1_{p_\tau, q_\tau}^{+,-})^{\varepsilon_{\tau}}, &\hbox{if  $\star=B$ or $D$}; \smallskip \\
         \check \Theta_{\tau'}^{\tau}(\pi_{\uptau'}\otimes \det^{\varepsilon_{\wp}}), &\hbox{if $\star=C$ or $\widetilde C$};  \smallskip\\
              \check \Theta_{\tau'}^{\tau}(\pi_{\uptau'}), &\hbox{if $\star=C^*$ or $D^*$}. \\
            \end{array}
   \right.
 \end{equation}
 Here $1_{p_\tau, q_\tau}^{+,-}$ denotes the character of $\oO(p_\tau, q_\tau)$ whose restriction to $\oO(p_\tau)\times \oO(q_\tau)$ equals $1\otimes \det$ ($1$ stands for the trivial character), and
$\varepsilon_{\wp}$ denote the element in $\Z/2\Z$ such that  \[
 \varepsilon_{\wp}=1\Leftrightarrow  (1,2)\in \wp.
\]
\delete{\begin{align*}
 \varepsilon_{\wp} &:=
 \begin{cases}
 1, & \quad \text{if } (1,2)\in \wp;\\
 0, & \quad \text{if }(1,2)\notin \wp.\\
 \end{cases}
 \end{align*}



 \varepsilon_{\tau} & :=
 \begin{cases}
 0 & \text{if }(\star,\cP_\tau(\bfcc_1(\imath_\tau),1))  = (D,d)\\
 0 & \text{if }(\star,\cQ_\tau(\bfcc_1(\jmath_\tau),1)) = (B,d)\\
 1 & \text{otherwise}
 \end{cases}\\

}
%We define
%\[
%\varepsilon_\uptau := \begin{cases}
%\varepsilon_\wp & \text{if } \star \in \set{C,\wtC,D^*}\\
%\varepsilon_\tau& \text{if } \star \in \set{B,D,C^*}
%\end{cases}
%\]

%As always, we let $\star\in \{B, C,D,\widetilde {C}, C^*, D^*\}$.

We are now ready to state our first main theorem.

\begin{thm}\label{thm1} Suppose that $\check \CO\in \Nil(\check \g)$ has $\star$-good parity.

\noindent (a) For every $\uptau = (\tau,\wp)\in \PBPes(\check \CO)$, the representation $\pi_{\uptau}$ of $G_\tau$ in \eqref{eq:def-pi} is irreducible and attached to $\ckcO$.

\noindent  (b) If $\star\in \{B,D\}$ and $(\star, \check \CO)\neq (D, \emptyset)$, then the map
\[
\begin{array}{rcl}
\PBPes(\check \CO)\times \Z/2\Z&\rightarrow &\mathrm{Unip}_{\star}(\check \CO),\\
  (\uptau, \epsilon)&\mapsto& \pi_{\uptau}\otimes \det^\epsilon
  \end{array}
\]
is bijective.

\noindent
(c) In all other cases, the map
\[
\begin{array}{rcl}
\PBPes(\check \CO)&\rightarrow &\mathrm{Unip}_{\star}(\check \CO),\\
  \uptau &\mapsto& \pi_{\uptau}
  \end{array}
\]
is bijective.
\end{thm}

By the above theorem, we have explicitly constructed all special unipotent representations in $\mathrm{Unip}_{\star}(\check \CO)$, when $\check \CO$ has $\star$-good parity.
The method of matrix coefficient integrals (Section \ref{sec:Integrals}) will imply the following unitarity result. 

\begin{cor}\label{cor1}
Suppose that $\check \CO\in \Nil(\check \g)$ has $\star$-good parity. Then all special unipotent representations in $\mathrm{Unip}_{\star}(\check \CO)$ are unitarizable.
\end{cor}

The reduction of \cite{BMSZ2} allows us to classify all special unipotent presentations attached to a general $\check \CO\in \Nil(\check \g)$ from those which have $\star$-good parity, by using irreducible unitary parabolic inductions. We thus conclude the following generalization of Corollary \ref{cor1}. 

\begin{thm}
All special unipotent representations of the classical groups in \eqref{typebcd} are unitarizable.
\end{thm}



\subsection{Computing associated cycles of the constructed representations}
Let $G,  G_\C$, $\g_\R, \g, \check \g$ and $\check \CO \in \mathrm{Nil}(\check \g)$ be as in Section \ref{secsu}.  As in Section \ref{secbip}, we assume that $G$ has type $\star$, and $\check \CO$ has $\star$-good parity.  Fix a maximal compact subgroup $K$ of $G$. We equipping $\g$ with the trace form. Then we have an orthogonal decomposition
\[
  \g=\frak k\oplus \p,
\]
where $\frak k$ is the complexified Lie algebra of $K$.  By taking the dual spaces, we also have a
decomposition
\[
  \g^*=\frak k^*\oplus \p^*.
\]
Write $K_\C$ for the complexification of the compact group $K$. It is a complex algebraic group with an obvious algebraic action on $\p^*$.
We identify $\g^*$ with $\g$ by using the trace form, and denote by $\mathrm{Nil}(\g)=\mathrm{Nil}(\g^*)$ the set of $G_\C$-orbits of nilpotent matrices in $\g=\g^*$.  

For any $\check \CO\in \Nil(\check \g)$, write $\CO\in \mathrm{Nil}(\g^*)$ for the Barbarsch-Vogan dual of $\check \CO$ so that its Zariski closure in $\g^*$ equals the associated variety of the ideal $I_{\check \CO}\subset \oU(\g)$. See \cite{BVUni,BMSZ1} and Section \ref{subsecass}. The algebraic variety
$
  \CO\cap \p^*
$
is a finite union of $K_\C$-orbits.  Given any such orbit $\sO$, write
$\cK_{K_\C}(\sO)$ for the  Grothendieck group of the category of $K_\C$-equivariant algebraic vector bundles on $\sO$. Put
\[
\cK_{K_\C}(\CO):=\bigoplus_{\sO\textrm{ is a $K_\C$-orbit in
      $\CO\cap \p^*$}} \cK_{K_\C}(\sO).
\]

%Recall the group $\cK^{\mathbb p}_{\cO}(\wtbfK)$ from  \Cref{sec:LVB}.


We say that a Casselman-Wallach representation of $G$ is  $\CO$-bounded  if
the associated variety  of its annihilator ideal
is contained in the Zariski closure of $\CO$. Note that all representations in $\mathrm{Unip}_{\check \CO}(G)$ are $\CO$-bounded. Write $\cK(G)_{\CO-\textrm{bounded}}$ for the  Grothendieck group of the category of all such representations.
From \cite[Theorem 2.13]{Vo89},  we have a canonical homomorphism
\[
\xymatrix{
  \mathrm{AC}_\cO\colon   \cK(G)_{\CO-\textrm{bounded}} \ar[r]& \cK_{K_\C}(\CO).
}
\]
We call $ \mathrm{AC}_\cO(\pi)$ the associated cycle of $\pi$, where $\pi$ is an $\CO$-bounded Casselman-Wallach representation of $G$. This is a fundamental invariant attached to $\pi$.


Following Vogan \cite[Section 8]{Vo89}, we make the following definition.

\begin{defn}\label{defaod}
  Let $\sO$ be a $K_\C$-orbit in $\CO\cap \p^*$. An admissible orbit datum over
  $\sO$ is an irreducible $K_\C$-equivariant algebraic vector bundle $\CE$
  on $\sO$ such that
  \begin{itemize}
    \item $\CE_\mathbf e$ is isomorphic to a multiple of
    $(\bigwedge^{\mathrm{top}} \fkk_\mathbf e )^{\frac{1}{2}}$ as a representation of
    $\fkk_\mathbf e$;
    \item $\CE$ is genuine if $\star=\widetilde C$.
  \end{itemize}
  Here $\mathbf e\in \sO$, $\CE_\mathbf e$ is the fibre of $\cE$ at $\mathbf e$, $\fkk_\mathbf e$
  denotes the Lie algebra of the stabilizer of $\mathbf e$ in $K_\C$, and
  $(\bigwedge^{\mathrm{top}} \fkk_\mathbf e)^{\frac{1}{2}}$ is a one-dimensional
  representation of $\fkk_\mathbf e$ whose tensor square is the top degree wedge
  product $\bigwedge^{\mathrm{top}} \fkk_\mathbf e$.
\end{defn}

Note that in the situation of the classical groups we consider in this article, all admissible
orbit data are line bundles.  Denote by $\mathrm{AOD}_{K_\C}(\sO)$ the
set of isomorphism classes of admissible orbit data over $\sO$, to be viewed as
a subset of $\cK_{K_\C}(\CO)$. Put
\be\label{kgroupaod}
  \mathrm{AOD}_{K_\C}(\CO):=\bigsqcup_{\sO\textrm{ is a $K_\C$-orbit in
      $\CO\cap \p^*$}} \mathrm{AOD}_{K_\C}(\sO)\subset
  \cK_{K_\C}(\CO).
\ee


%For the rest of this subsection, we suppose that $G$ has type $\star\in \{B, C,D,\widetilde {C}, C^*, D^*\}$, and $\check \CO$ has $\star$-good parity.
Recall that a nilpotent orbit in $\check \g$ is said to be distinguished if it is has empty intersection with every proper Levi subalgebra of $\check \g$. Combinatorially, this is equivalent to saying that no pair of rows of the Young diagram  have equal nonzero length. Note that all distinguished nilpotent orbits in $\check \g$ has $\star$-good parity.

\begin{defn}
\noindent
(a) The orbit $\check \CO$ (which has $\star$-good parity) is said to be quasi-distinguished if there is no $\star$-pair that is balanced in $\check \CO$.

\noindent
(b) If  $\star\in \{B, D, D^*\}$, then $\check \CO$ is said to be weakly-distinguished if there is no positive even integer $i$ such that $\mathbf{r}_i(\check \CO)=\mathbf{r}_{i+1}(\check \CO)= \mathbf{r}_{i+2}(\check \CO)=\mathbf{r}_{i+3}(\check \CO)>0$.


\noindent
(c) If  $\star\in \{C, \widetilde C, C^*\}$ so that its Howe dual $\star'\in  \{B, D, D^*\}$, then $\check \CO$ is said to be weakly-distinguished if its dual descent $\check \CO'$
% (which is a Young diagram that has $\star'$-good parity) 
is weakly-distinguished.

\end{defn}


We will compute the associated cycle of $\pi_\uptau$ for every painted bipartition $\tau$ associated to $\check \CO$ which has $\star$-good parity. The computation will be a key ingredient in the proof of Theorem \ref{thm1}. Our second main theorem is the following result concerning associated cycles of the special unipotent representations.

\begin{thm}\label{thmac0} Suppose that $\check \CO\in \mathrm{Nil}(\check \g)$  has $\star$-good parity.

\noindent (a) For every $\pi\in \mathrm{Unip}_{\check \CO}(G)$, the associated cycle $\mathrm{AC}_{\CO}(\pi)\in \cK_{K_\C}(\CO) $ is a nonzero  sum of  admissible orbit data 
over pairwise distinct $K_\C$-orbits in $\CO\cap \p^*$. % elements of $\mathrm{AOD}_{K_\C}(\CO)$ 

\noindent  (b) If $\check \CO$ is weakly-distinguished, then the map
\[
\mathrm{AC}_\CO: \mathrm{Unip}_{\check \CO}(G)\rightarrow  \cK_{K_\C}(\CO)
\]
is injective.

\noindent  (c) If $\check \CO$ is quasi-distinguished, then the map $\mathrm{AC}_{\CO}$ induces a bijection
\[
\mathrm{Unip}_{\check \CO}(G)\rightarrow  \mathrm{AOD}_{K_\C}(\CO).
\]

\end{thm}

\begin{remark}
Suppose that $\star\in \{C^*, D^*\}$ so that $G$ is quaternionic.
Then there is  precisely one admissible orbit datum over  $\sO$ for each $K_\C$-orbit $\sO\subset \CO\cap \p^*$.
Thus
\[
 \mathrm{AOD}_{K_\C}(\CO)=K_\C\backslash  (\CO\cap \p^*).
\]
If $\check \CO$ is not quasi-distinghuished, then
$\CO\cap \p^*$ is empty (see \cite[Theorems 9.3.4 and 9.3.5]{CM}), and hence $\mathrm{Unip}_{\check \CO}(G)$ is also empty.

\end{remark}



\section{Descents of painted bipartitions: combinatorics}\label{sec:comb}

In this section, we define the descent of a painted bipartition, as alluded to in Section \ref{subsec:comTOrep}. As before, let  $\star\in \{ B, C,  D, \widetilde{C},  C^*, D^*\}$ and let $\check \CO$ be a Young diagram that has $\star$-good parity. Recall the dual descent $\check \CO'$ of $\CO$ which has $\star'$-good parity, where $\star'$ is the Howe dual of $\star$.

\delete{Put
\begin{equation}\label{lstarco}
  l:=l_{\star, \check \CO}:=\begin{cases}
 \frac{\bfrr_1(\ckcO)}{2}; & \quad \textrm{if } \star\in \{B, \widetilde C\};\smallskip\\
 \frac{\bfrr_1(\ckcO)-1}{2}, &\quad \textrm{if } \star\in \{C, C^* \};\smallskip\\
 \frac{\bfrr_1(\ckcO)+1}{2}, &\quad \textrm{if } \star\in \{ D, D^*\}.\\
\end{cases}
\end{equation}
This is the length of the leading column of every element of $\mathrm{PBP}_\star(\check \CO)$.
}


For a Young diagram $\imath$, its naive descent, which is denoted by $\nabla_\mathrm{naive}(\imath)$, is defined to be the Young diagram obtained from $\imath$ by removing the first column. By convention, $\nabla_\mathrm{naive}(\emptyset)=\emptyset$.

%In the rest of this section, we assume that $\check \CO\neq \emptyset$. 

 %Put\[
%l':=l_{\star', \check \CO'}
%\]

 \subsection{Naive descent of a painted bipartition }
\def\bipartl{\mathrm{bi\cP_L}}
\def\bipartr{\mathrm{bi\cP_R}}
\def\dsdiagl{\mathrm{DS_L}}
\def\dsdiagr{\mathrm{DS_R}}
\def\DDl{\eDD_\mathrm{L}}
\def\DDr{\eDD_\mathrm{R}}


In this subsection, let $\tau=(\imath,\cP)\times (\jmath,\cQ)\times \alpha$ be a  painted bipartition such that $\star_\tau=\star$. Put
\delete{\begin{equation} \label{eq:def.alphap}
\alpha'=\begin{cases} B^+,
& \textrm{if $\alpha = \wtC$ and $\cP_\tau(l_{\star,\ckcO},1),1) \neq c$;}\\
B^-,
& \textrm{if $\alpha = \wtC$ and $\cP_\tau(l_{\star,\ckcO},1),1)  = c$;}\\
\star', & \textrm{if $\alpha\neq \widetilde C$}.
\end{cases}
\end{equation}
}
  \begin{equation} \label{eq:def.alphap}
    \alpha'=\begin{cases} B^+,
  & \textrm{if $\alpha=\widetilde{C}$ and $c$ does not occur in the first column of $(\imath,\cP)$}; \smallskip \\
  B^-,
  & \textrm{if $\alpha=\widetilde{C}$ and  $c$ occurs in the first column of $(\imath,\cP)$}; \smallskip \\
  \star', & \textrm{if $\alpha\neq \widetilde C$}.
  \end{cases}
  \end{equation}

\begin{lem}\label{lemDDn1}
  If $\star \in \set{B,C,C^*}$, then there is a unique painted bipartition of the form $\tau'= (\imath',\cP')\times (\jmath',\cQ')\times \alpha'$ with the following properties:
  \begin{itemize}
        \item $
   (\imath',\jmath')= (\imath,\DD_\mathrm{naive}(\jmath)); \smallskip
   $
   \item for all $(i,j)\in \BOX(\imath')$,
   \[
     \cP'(i,j)=\begin{cases}
    \bullet \textrm{ or } s,&\textrm{ if  $\ \cP(i,j)\in \{\bullet, s\}$;} \smallskip \\
  \cP(i,j),& \textrm{ if $\ \cP(i,j)\notin \{\bullet, s\}$};\end{cases}
   \]
   \item for all $(i,j)\in \BOX(\jmath')$,
   \[
     \cQ'(i,j)=\begin{cases}
    \bullet \textrm{ or } s,&\textrm{ if  $\ \cQ(i,j+1)\in \{\bullet, s\}$;} \smallskip \\
  \cQ(i,j+1), & \textrm{ if $\ \cQ(i,j+1)\notin \{\bullet, s\}$}.  \end{cases}
   \]
    \end{itemize}
    \end{lem}




   \begin{proof}
    First assume that the images of $\cP$ and $\cQ$ are both contained in $\{\bullet, s\}$. Then  the image of $\cP$  is in fact contained in $\{\bullet\}$, and $(\imath, \jmath)$ is  right interlaced in the sense that
 \[
 \mathbf{c}_1(\jmath)\geq \mathbf{c}_1(\imath)\geq \mathbf{c}_2(\jmath)\geq \mathbf{c}_2(\imath)\geq \mathbf{c}_3(\jmath)\geq \mathbf{c}_3(\imath) \geq \cdots.
 \]
 Hence $ (\imath',\jmath'):= (\imath,\DD(\jmath))$ is left interlaced in the sense that
 \[
 \mathbf{c}_1(\imath')\geq \mathbf{c}_1(\jmath')\geq \mathbf{c}_2(\imath')\geq \mathbf{c}_2(\jmath')\geq \mathbf{c}_3(\imath')\geq \mathbf{c}_3(\jmath') \geq \cdots.
 \]
 Then it is clear that there is a unique painted bipartition of the form  $\tau'=(\imath',\cP')\times (\jmath',\cQ')\times \alpha'$ such that images of $\cP'$ and $\cQ'$ are both contained in $\{\bullet, s\}$. This proves the lemma in the special case when the images of $\cP$ and $\cQ$ are both contained in $\{\bullet, s\}$.

 The proof of the lemma in the general case is easily reduced to this special case.
   \end{proof}
    \begin{lem}\label{lemDDn2}
    If $\star \in \set{ \widetilde C, D,D^*}$, then there is a unique painted bipartition of the form $\tau'= (\imath',\cP')\times (\jmath',\cQ')\times \alpha'$ with the following properties:
  \begin{itemize}
        \item $
   (\imath',\jmath')= (\DD_\mathrm{naive}(\imath),\jmath); \smallskip
   $
   \item for all $(i,j)\in \BOX(\imath')$,
   \[
     \cP'(i,j)=\begin{cases}
    \bullet \textrm{ or } s,&\textrm{ if  $\ \cP(i,j+1)\in \{\bullet, s\}$;} \smallskip \\
  \cP(i,j+1),& \textrm{ if $\ \cP(i,j+1)\notin \{\bullet, s\}$};\end{cases}
   \]
   \item for all $(i,j)\in \BOX(\jmath')$,
   \[
     \cQ'(i,j)=\begin{cases}
    \bullet \textrm{ or } s,&\textrm{ if  $\ \cQ(i,j)\in \{\bullet, s\}$;} \smallskip \\
  \cQ(i,j), & \textrm{ if $\ \cQ(i,j)\notin \{\bullet, s\}$}.  \end{cases}
   \]

    \end{itemize}
\end{lem}
\begin{proof}
  The proof is similar to that of \Cref{lemDDn1}.
\end{proof}

\begin{defn}
 In the notation of \Cref{lemDDn1,lemDDn2}, we call $\tau'$ the naive descent of $\tau$, to be denoted by $\DDn(\tau)$.
\end{defn}




 \begin{Example} If
    \[
     \tau = \ytb{\bullet\bullet\bullet {c},\bullet {s} {c},{s},{c}}
    \times \ytb{\bullet\bullet\bullet ,\bullet {r} {d},{d}{d}, \none}
    \times \widetilde C, \]
   then
   \[
    \nabla_{\mathrm{naive}}(\tau) =\ytb{\bullet\bullet{c} ,\bullet{c},\none }
    \times  \ytb{\bullet\bullet {s} ,\bullet {r} {d},{d}{d}}\times B^-.
    \]

\end{Example}

  \subsection{Descents of painted bipartitions}\label{sec:desc}


Suppose that
$
\tau=(\imath,\cP)\times(\jmath,\cQ)\times \alpha \in  \mathrm{PBP}_\star(\check \CO)
$
and write
\[
  \tau'_{\mathrm{naive}}=(\imath', \cP'_{\mathrm{naive}})\times (\jmath', \cQ'_{\mathrm{naive}})\times \alpha'
\]
for the naive descent of $\tau$. This is clearly an element of $  \mathrm{PBP}_{\star'}(\check \CO')$.
%Put
%\begin{equation}\label{lstarco}
%  l:=l_{\star, \check \CO}:=\begin{cases}
% \frac{\bfrr_2(\ckcO)}{2}; & \quad \textrm{if } \star\in \{B, \widetilde C\};\\
% \frac{\bfrr_2(\ckcO)+1}{2}, &\quad \textrm{if } \star\in \{C, C^* \};\\
% \frac{\bfrr_2(\ckcO)-1}{2}, &\quad \textrm{if } \star\in \{ D, D^*\}.\\
%\end{cases}
%\end{equation}

The following two lemmas are easily verified and we omit the proofs. We will give an example for each case.
\delete{
\begin{lem}\label{descb}
Suppose that
\[
\begin{cases}
\alpha = B^+; & \\
(2,3)\in \wp;\quad  &\\
\cQ(l',1)\in \set{r,d}.
\end{cases}
\]
Then there is a unique element in $\mathrm{PBP}_{\star'}(\check \CO',\wp')$ of the form
  \[
      \tau'=(\imath', \cP')\times (\jmath', \cQ')\times \alpha'
  \]
such that
     $
     \cP' = \cP'_{\mathrm{naive}}
     $
     and
     for all $(i,j)\in \BOX(\jmath')$,
\[
\cQ'(i,j) = \begin{cases}
  r, & \ \text{ if  $(i,j) = (l',1)$;}\\
  \cQ'_{\mathrm{naive}}(i,j), & \ \text{ otherwise}.
\end{cases}
\]
\end{lem}


\begin{Example}
 If
 \[
 \tau= \ytb{\bullet\bullet,\none} \times \ytb{\bullet \bullet, dd}\times
  B^+,
 \]
 then
\[
 \tau'_{\mathrm{naive}}= \ytb{\bullet s,\none} \times \ytb{\bullet, d}\times
  \widetilde C\qquad\textrm{and}\qquad \tau'= \ytb{\bullet s,\none} \times \ytb{\bullet, r}\times
  \widetilde C.
 \]
 Note that in this case, the nonzero row lengths of $\check \CO$ are $4,4,2,2$, $\wp=\{(2,3)\}$ and $l'=2$.
\end{Example}
\delete{\begin{proof}
 We only need to check that the triple $\tau'$ defined in the lemma is a painted bipartition.

 Note that
 \[
  \bar \Lambda_{l-1,2}(\imath', \cP')=\bar \Lambda_{l-1,2}(\imath'_{\mathrm{naive}}, \cP'_{\mathrm{naive}})
 \]
 and
 \[
 \begin{array}{ccc}

      \Lambda_{l-1,1}(\cP_\tau)\times \Lambda_{l-1,2}(\cQ_\tau)
     &  &
        \Lambda_{l-1,1}(\cP_{\tau'})\times \Lambda_{l-1,2}(\cQ_{\tau'})\\
     \hline
     \hspace{1em}\\
       \emptyset
      \times
      \ytb{ {x_{1}}{x_0},{\enon{\vdots}},{\enon{\vdots}},{x_{n}}}
      &
        \mapsto  &
        \emptyset
        \times
      \ytb{ {\none}{r},{\none},{\none},\none}
      \end{array}
    \]

\end{proof}

Lemma \ref{descb} is easy to check and we omit the details. Note that $(\frac{\bfrr_2(\ckcO)}{2},1) \in \BOX(\jmath')$ under the first two conditions  of Lemma \ref{descb}. Similarly, we also have the following three lemmas.
}
 }
\begin{lem}\label{descb2}
  Suppose that
\[  \begin{cases}
 \alpha = B^+; & \\
 \bfrr_2(\ckcO)>0; & \\
 \cQ(\mathbf c_1(\jmath),1)\in \set{r,d}.
\end{cases}
\]
 Then there is a unique element in $\mathrm{PBP}_{\star'}(\check \CO')$ of the form
  \[
      \tau'=(\imath', \cP')\times (\jmath', \cQ')\times \alpha'
  \]
 such that
     $
     \cQ' = \cQ'_{\mathrm{naive}}
     $
     and
     for all $(i,j)\in \BOX(\imath')$,
\[
\cP'(i,j) = \begin{cases}
  s, & \ \text{ if $(i,j) = (\mathbf c_1(\imath'),1)$;}\\
  \cP'_{\mathrm{naive}}(i,j), & \ \text{ otherwise}.
\end{cases}
\]
\end{lem}

\begin{Example}
 If
 \[
 \tau= \ytb{\bullet c, c} \times \ytb{\bullet r, rd}\times
  B^+,
 \]
 then
\[
 \tau'_{\mathrm{naive}}= \ytb{s c, c} \times \ytb{r, d}\times
  \widetilde C\qquad\textrm{and}\qquad \tau'= \ytb{s c, s} \times \ytb{r, d}\times
  \widetilde C.
 \]
 Note that in this case, the nonzero row lengths of $\check \CO$ are $4,4,4,2$.
\end{Example}

\delete{
\begin{lem}\label{descd1}
  Suppose that
  \[  \begin{cases}
 \alpha = D; & \\
 (2,3)\in \wp;\quad  &\\
 \cP(\mathbf c_1(\imath'),1) \in \set{r,c}.
\end{cases}
\]
 Then there is a unique element in $\mathrm{PBP}_{\star'}(\check \CO',\wp')$ of the form
  \[
      \tau'=(\imath', \cP')\times (\jmath', \cQ')\times \alpha'
  \]
  such that $\cQ'=\cQ'_{\mathrm{naive}}$ and  for all $(i,j)\in \BOX(\imath')$,
  \[
\cP'(i,j) = \begin{cases}
  r, & \ \text{ if } (i,j) = (\mathbf c_1(\imath'),1); \\
  \cP(\mathbf c_1(\imath'),1), &\  \text{ if } (i,j) = (\mathbf c_1(\imath')+1,1);\\
  \cP'_{\mathrm{naive}}(i,j), & \ \text{ otherwise}.
\end{cases}
\]

\end{lem}




\begin{Example}
 If
 \[
 \tau= \ytb{\bullet s,  c c, d d} \times \ytb{\bullet,\none, \none }\times
  D,
 \]
 then
\[
 \tau'_{\mathrm{naive}}=  \ytb{\bullet,  c,  d}  \times  \ytb{\bullet,\none, \none }\times
  C,\qquad\textrm{and}\qquad \tau'= \ytb{\bullet, r, c}  \times  \ytb{\bullet,\none, \none }\times
  C.
 \]
 Note that in this case, the nonzero row lengths of $\check \CO$ are $5,5,3,1$,  $\wp=\{(2,3)\}$.
\end{Example}
}
\begin{lem}\label{descd2}
  Suppose that
  \[  \begin{cases}
 \alpha = D; & \smallskip \\
\mathbf r_2(\check \CO)=\mathbf r_3(\check \CO)>0;  &\smallskip\\
\left(\cP(\mathbf c_2(\imath),1), \textrm{\vrule width 0pt  height 0.9em  \relax} \cP(\mathbf c_2(\imath),2)\right)=(r,c); &\smallskip\\
 \cP(\mathbf c_1(\imath),1)\in \set{r,d}.
\end{cases}
\]
 Then there is a unique element in $\mathrm{PBP}_{\star'}(\check \CO')$ of the form
  \[
      \tau'=(\imath', \cP')\times (\jmath', \cQ')\times \alpha'
  \]
  such that $\cQ'=\cQ'_{\mathrm{naive}}$ and  for all $(i,j)\in \BOX(\imath')$,
  \[
\cP'(i,j) = \begin{cases}
  r, & \ \text{ if } \ (i,j) = (\mathbf c_1(\imath'),1); \\
  \cP'_{\mathrm{naive}}(i,j), & \ \text{ otherwise}.
\end{cases}
\]

\end{lem}


\begin{Example}
 If
 \[
 \tau= \ytb{\bullet\bullet, \bullet s, \bullet s, r c} \times \ytb{\bullet\bullet,\bullet,\bullet, \none }\times
  D,
 \]
 then
\[
 \tau'_{\mathrm{naive}}=\ytb{\bullet, \bullet , \bullet ,  c} \times \ytb{\bullet s,\bullet,\bullet, \none } \times
  C,\qquad\textrm{and}\qquad \tau'=\ytb{\bullet, \bullet , \bullet ,  r} \times \ytb{\bullet s,\bullet,\bullet, \none } \times
  C.
 \]
 Note that in this case, the nonzero row lengths of $\check \CO$ are $7,7,7,3$.
\end{Example}

\begin{defn}
We define the descent of $\tau$ to be
\[
  \nabla(\tau):= \begin{cases}
  \tau', & \ \text{ if either of the condition of Lemma \ref{descb2}  or \ref{descd2} holds}; \\
  \nabla_{\mathrm{naive}}( \tau), & \ \text{ otherwise},
\end{cases}
\]
which is an element of $  \mathrm{PBP}_{\star'}(\check \CO')$.
Here $\tau'$ is as in Lemmas  \ref{descb2} and \ref{descd2}.
\end{defn}
In conclusion, we have by now a well-defined descent map
\[
\nabla: \mathrm{PBP}_{\star}(\check \CO)\rightarrow \mathrm{PBP}_{\star'}(\check \CO').
\]


The following injectivity result will be  proved in Section \ref{secppb}. %will be important for us. 
%The key property that we will need  of the descent map when $\star\in \set{D,B,C^*}$ are summarized in the following two propositions.

\begin{prop}\label{prop:DD.BDinj}
If $\star \in \set{B, D,C^*}$, then the map
\begin{equation}
  \begin{array}{rcl}
   \PBP_\star(\ckcO)&\rightarrow&
   \PBP_{\star'}(\ckcOp)\times \BN\times \bN\times \Z/2\Z, \smallskip\\
   \tau& \mapsto & (\DD(\tau), p_\tau, q_\tau, \varepsilon_\tau)
   \end{array}
\end{equation}
is injective. If $\star \in  \set{C,\wtC,D^*}$, then the map
\begin{equation}
  \begin{array}{rcl}
  \DD:  \PBP_\star(\ckcO)&\rightarrow&
   \PBP_{\star'}(\ckcOp)
   \end{array}
\end{equation}
is injective.

\end{prop}


\section{From painted bipartitions to associated cycles} \label{sec:Nil}


%\section{Classical groups and their nilpotent orbits} \label{sec:Nil}

% \subsection{Classical groups}
% \subsubsection{complex orthogonal and symplectic groups}
\subsection{Classical spaces}
Let $\star\in \{B,C,D, \widetilde C, C^*, D^*\}$ as before. Put
\be\label{epsilond}
(\epsilon, \dot \epsilon):=(\epsilon_\star, \dot \epsilon_\star):=\begin{cases}
  (1,1),&\quad \textrm{ if  $\star\in \{B,D\}$;} \smallskip \\
 (-1,-1),&\quad \textrm{ if  $\star\in \{C,\widetilde C\}$;} \smallskip \\
 (-1,1),&\quad \textrm{ if  $\star=C^*$;} \smallskip \\
 (1,-1),&\quad \textrm{ if  $\star=D^*$.}
 \end{cases}
\ee
A classical signature is defined to be a triple  $\mathsf s=(\star, p,q)\in  \{B,C,D, \widetilde C, C^*, D^*\}\times \bN\times \bN$ such that
\[
\begin{cases}
  p+q\textrm{ is odd },&\quad \textrm{ if  $\star=B$;} \smallskip \\
   p+q\textrm{ is even },&\quad \textrm{ if  $\star=D$;} \smallskip \\
 p=q,&\quad \textrm{ if  $\star\in \{C,\widetilde C, D^*\}$;} \smallskip \\
\textrm{both $p$ and $q$ are even} ,&\quad \textrm{ if  $\star=C^*$.} \smallskip \\
 \end{cases}
\]
 Suppose that $\mathsf s=(\star, p,q)$ is a classical  signature in the rest of this section.
Set 
\[
  \abs{\sfss}:=p+q,
\]
and we call $\star$ the type of $\sfss$. 

%\subsubsection{Cartan involution}
%We recall a standard result on (global) Cartan involution.
We omit the proof of the following lemma (\cf \cite[Section~1.3]{Ohta}).
\begin{lem}\label{lem:cartan}
  There is quadruple  $(V, \la\,,\,\ra, J,L)$ satisfying the following conditions:
  \begin{itemize}
%  \item \label{it:cartan.1} $\inn{Lu}{Lv}_{\bfV} = \inn{u}{v}_{\bfV} $ and $L^2 = \dotepsilon$;
   % and $\epsilon\ccL\ccJ = 1$;
   \item $V$ is a complex vector space of dimension $\abs{\sfss}$;
   \item $\la\,,\,\ra$ is an $\epsilon$-symmetric non-degenerate bilinear form on $V$;
   \item $J: V\rightarrow V$ is a conjugate linear automorphism of $V$ such that $J^2=\epsilon\cdot \dot \epsilon$;
  \item $L: V\rightarrow V$ is a  linear automorphism of $V$ such that $L^2=\dot \epsilon$; %the earlier $L$ satisfies $L^2=\dot \epsilon$.
\item   $\inn{J u}{Jv}=
  \overline{\inn{u}{v}},\quad$  for all $u,v\in V$;
\item  $ \inn{Lu}{Lv}=
  \inn{u}{v}$, $\quad$ for all  $ u,v\in V$;
  \item $LJ =  JL$;
  \item  the Hermitian form $(u,v)\mapsto \inn{Lu}{Jv}$ on
    $V$ is positive definite;
    \item if $\dot \epsilon=1$, then $\dim\{ v\in V\mid Lv=v\}=p$ and $\dim\{ v\in V\mid Lv=-v\}=q$.
  \end{itemize}
 Moreover, such a quadruple is  unique in the following sense: if  $(V', \la\,,\,\ra', J',L')$ is another quadruple satisfying the analogous conditions, then there is a linear isomorphism $\phi: V\rightarrow V'$ that respectively transforms $ \la\,,\,\ra$, $J$ and $L$ to  $ \la\,,\,\ra'$, $J'$ and $L'$.
 \end{lem}


In the notation of Lemma \ref{lem:cartan}, we  call $(V, \la\,,\,\ra, J,L)$ a classical space of signature $\mathsf s$, and denote it by
 $ (V_{\mathsf s}, \la\,,\,\ra_{\mathsf s}, J_{\mathsf s},L_{\mathsf s})$.

Denote by $G_{\mathsf s, \C}$ the isometry group of $ (V_{\mathsf s}, \la\,,\,\ra_{\mathsf s})$, which is an complex orthogonal group if $\epsilon=1$ and a complex symplectic group if $\epsilon=-1$. Respectively denote by  $G_{\mathsf s, \C}^{J_\mathsf s}$  and  $G_{\mathsf s, \C}^{L_\mathsf s}$ the centralizes of $J_\mathsf s$ and $L_\mathsf s$ in  $G_{\mathsf s, \C}$. Then $G_{\mathsf s, \C}^{J_\mathsf s}$ is a real classical group isomorphic with
\[
 \left\{
     \begin{array}{ll}
         \oO(p, q), &\hbox{if $\star=B$ or $D$}; \smallskip\\
            \Sp_{2p}(\R), &\hbox{if $\star\in \{C,\widetilde C\}$}; \smallskip\\
                   \Sp(\frac{p}{2}, \frac{q}{2}), &\hbox{if $\star=C^*$}; \smallskip \\
          \oO^*(2p), &\hbox{if $\star=D^*$}.\\
            \end{array}
   \right.
\]
Put
\[
G_\mathsf s:= \left\{
     \begin{array}{ll}
       \textrm{the metaplectic double cover of $G_{\mathsf s, \C}^{J_\mathsf s}$}, \quad &\hbox{if $\star=\widetilde C$}; \smallskip\\
            G_{\mathsf s, \C}^{J_\mathsf s},  \quad  &\hbox{otherwise}.\\
            \end{array}
   \right.
\]
Denote by $K_{\mathsf s}$ the inverse image of  $G_{\mathsf s, \C}^{L_\mathsf s}$  under the natural homomorphism $G_\mathsf s\rightarrow G_{\mathsf s,\C}$, which is a maximal compact subgroup of $G_{\mathsf s}$. Write $K_{\mathsf s, \C}$ for the complexification of  $K_{\mathsf s}$, which is a reductive complex linear algebraic group.

For every $\lambda\in \C$, write $V_{\mathsf s, \lambda}$ for the eigenspace of $L_{\mathsf s}$ with eigenvalue $\lambda$. Write
\[
{\det}_\lambda : K_{\mathsf s, \C}\rightarrow \C^\times
\]
for the composition of
\be\label{comdet}
   K_{\mathsf s, \C}\xrightarrow{\textrm{the natural homomorphism}} \GL(V_{\mathsf s, \lambda})\xrightarrow{\textrm{the determinant character}}\C^\times.
\ee
If $\star=\widetilde C$, then the natural homomorphism  $ K_{\mathsf s, \C}\rightarrow \GL(V_{\mathsf s, \sqrt{-1}})$ is a double cover, and there is a unique genuine algebraic character
$\det_{\sqrt{-1}}^{\frac{1}{2}}:  K_{\mathsf s, \C}\rightarrow \C^\times $ whose square equals $\det_{\sqrt{-1}}$.

Write
\[
{\det} : K_{\mathsf s, \C}\rightarrow \C^\times
\]
for the composition of
\be\label{comdet}
   K_{\mathsf s, \C}\xrightarrow{\textrm{the natural homomorphism}} \GL(V_{\mathsf s})\xrightarrow{\textrm{the determinant character}}\C^\times.
\ee
This is the trivial character unless $\star=B$ or $D$.
%Note that the first map in \eqref{comdet} is a linear isomorphism if $\star\in \{C,  D^*\}$, and is a two fold covering map if $\star=\widetilde C$.
%In the latter case, there is a unique genuine algebraic character of $ K_{\mathsf s, \C}$ whose square equals $\det$.  Write $\det^{\frac{1}{2}}$ for this character.


Denote by $\g_\mathsf s$ the Lie algebra of $G_{\mathsf s, \C}$. Then we have a decomposition
\[
\g_\mathsf s=\mathfrak k_\mathsf s \oplus \p_\mathsf s,
\]
where $\mathfrak k_\mathsf s$ is the Lie algebra of $K_{\mathsf s, \C}$, and $\p_\mathsf s$ is the orthogonal complement of $\mathfrak k_\mathsf s$ in $\g_\mathsf s$ under the trace form.
Using the trace form, we also identify $\g_{\mathsf s}^*$ with $\g_\mathsf s$. Denote by $\mathrm{Nil}(\g_\mathsf s)$ the set of nilpotent $G_{\mathsf s, \C}$-orbits in $\g_\mathsf s$, and by
$\mathrm{Nil}(\p_\mathsf s)$ the set of nilpotent $K_{\mathsf s, \C}$-orbits in $\p_\mathsf s$.

Given a $K_{\mathsf s,\C}$-orbit $\sO$ in $\p_\mathsf s$, write $\CK_{\mathsf s}(\sO)$ for the Grothendieck group of the categogy of
$K_{\mathsf s,\C}$-equivariant algebraic vector bundles  over $\sO$. Denote by $\CK^+_{\mathsf s}(\sO)\subset \CK_{\mathsf s}(\sO)$ the submonoid generated by all these equivariant algebraic vector bundles. The notion of admissible orbit datum over $\sO$ is defined as in Definition \ref{defaod}. Write
\[
\mathrm{AOD}_\mathsf s(\sO)\subset \CK^+_{\mathsf s}(\sO)
\]
for the set of isomorphism classes of admissible orbit data over $\sO$.

Let $\CO$ be a $G_{\mathsf s,\C}$-orbit in $\g_{\mathsf s,\C}$. It is well-known that $\CO\cap \p_{\mathsf s}$ has only finitely many $K_{\mathsf s,\C}$-orbits. Put
\[
\CK_{\mathsf s}(\CO):=\bigoplus_{\sO\textrm{ is a $K_{\mathsf s,\C}$-orbit in $\CO\cap \p_{\mathsf s}$ }}\CK_{\mathsf s}(\sO),
\]
and
\[
\CK_{\mathsf s}^+(\CO):=\sum_{\sO\textrm{ is a $K_{\mathsf s,\C}$-orbit in $\CO\cap \p_{\mathsf s}$ }}\CK^+_{\mathsf s}(\sO).
\]
Put
\[
\mathrm{AOD}_\mathsf s(\CO):=\bigsqcup_{\sO\textrm{ is a $K_{\mathsf s,\C}$-orbit in $\CO\cap \p_{\mathsf s}$ }}\mathrm{AOD}_\mathsf s(\sO)\subset \CK^+_{\mathsf s}(\CO).
\]


Define a partial order $\preceq $ on $ \CK_{\mathsf s}(\CO)$ such that
\[
  \CE_1\preceq \CE_2\Leftrightarrow \CE_2-\CE_1\in \CK_{\mathsf s}^+(\CO) \qquad (\CE_1, \CE_2\in \CK_{\mathsf s}(\CO)).
\]
For every algebraic character $\chi$ of $K_{\mathsf s,\C}$, the twisting map
\[
\CK_{\mathsf s}(\CO)\rightarrow \CK_{\mathsf s}(\CO), \qquad \CE\mapsto \CE\otimes \chi
\]
is obviously defined.

\subsection{The moment maps}\label{secmmap}
Recall that $\star'$ is the Howe dual of $\star$. Suppose that $\mathsf s'=(\star', p',q')$ is another classical  signature.  Put
\[
  W_{\mathsf s, \mathsf s'} := \Hom_\bC(V_\mathsf s,V_{\mathsf s'}).
\]
Then we have the adjoint map
\[
  W_{\mathsf s, \mathsf s'} \rightarrow W_{\mathsf s', \mathsf s},\qquad \phi\mapsto \phi^*
\]
that is specified by requiring
 \be\label{adjointmap}
    \inn{\phi v}{v'}_{\mathsf s'} = \inn{v}{ \phi^* v'}_{\mathsf s},  \qquad\textrm{for all }v\in
    V_\mathsf s,\, v'\in V_{\mathsf s'}, \, \phi\in   W_{\mathsf s, \mathsf s'}.
  \ee

Define three maps
\[
  \la\,,\,\ra_{\mathsf s, \mathsf s'}:  W_{\mathsf s, \mathsf s'}\times  W_{\mathsf s, \mathsf s'}\rightarrow \C, \quad(\phi_1,\phi_2)\mapsto \tr(\phi_1^* \phi_2),
\]
\[
J_{\mathsf s, \mathsf s'}: W_{\mathsf s, \mathsf s'}\rightarrow W_{\mathsf s, \mathsf s'}, \quad \phi\mapsto  J_{\mathsf s'}\circ \phi \circ J_{\mathsf s}^{-1};
\]
and
\[
L_{\mathsf s, \mathsf s'}: W_{\mathsf s, \mathsf s'}\rightarrow W_{\mathsf s, \mathsf s'}, \quad \phi\mapsto  \dot \epsilon L_{\mathsf s'}\circ \phi \circ L_{\mathsf s}^{-1}.
\]
It is routine to check that
\[
( W_{\mathsf s, \mathsf s'},  \la\,,\,\ra_{\mathsf s, \mathsf s'}, J_{\mathsf s, \mathsf s'}, L_{\mathsf s, \mathsf s'})
\]
is a classical space of signature $(C, \frac{\abs{\sfss}\cdot \abs{\sfss'}}{2}, \frac{\abs{\sfss}\cdot \abs{\sfss'}}{2})$.

Write
\begin{equation}\label{def:Xss'}
   W_{\mathsf s, \mathsf s'}=\cX_{\mathsf s, \mathsf s'}\oplus  \cY_{\mathsf s, \mathsf s'},
\end{equation}
where $\cX_{\mathsf s, \mathsf s'}$ and $ \cY_{\mathsf s, \mathsf s'}$ are  the eigenspaces of $L_{\mathsf s, \mathsf s'}$ with eigenvalues $\sqrt{-1}$ and $-\sqrt{-1}$, respectively.
Then we have the following two well-defined algebraic maps:
  \be\label{momentmap}
    \xymatrix@R=0em@C=4em{
      \fpp_\mathsf s &\ar[l]_{M_\mathsf s:=M_\mathsf s^{\sfss, \sfss'}} \cX_{\mathsf s, \mathsf s'}\ar[r]^{M_{\sfss'}:=M_{\mathsf s'}^{\sfss, \sfss'}}& \fpp_{\mathsf s'},\\
     \phi^* \phi & \ar@{|->}[l] \phi \ar@{|->}[r] & \phi \phi^*.
    }
  \ee
\delete{
Then we have the following two well-defined algebraic maps:
  \be\label{momentmap}
    \xymatrix@R=0em@C=3em{
      \fpp_\mathsf s &\ar[l]_{M_\mathsf s} \cX_{\mathsf s, \mathsf s'}\ar[r]^{M_{\mathsf s'}}& \fpp_{\mathsf s'},\\
     \phi^* \phi & \ar@{|->}[l] \phi \ar@{|->}[r] & \phi \phi^*.
    }
  \ee
}
These two maps $M_\mathsf s$ and $M_{\mathsf s'}$ are called the moment maps. They are both $K_{\mathsf s,\C}\times K_{\mathsf s', \C}$-equivariant. Here  $K_{\mathsf s', \C}$ acts trivially on $\p_\mathsf s$,
 $K_{\mathsf s, \C}$ acts trivially on $\p_\mathsf s'$, and all the other actions are the obvious ones.

Put
\[
  W_{\mathsf s, \mathsf s'}^\circ:=\{\phi\in W_{\mathsf s, \mathsf s'}\mid \textrm{the image of $\phi$ is non-degenerate with respect to $\la\,,\,\ra_{\mathsf s'}$}\}
\]
and
\[
  \cX_{\mathsf s, \mathsf s'}^\circ:=\cX_{\mathsf s, \mathsf s'}\cap W_{\mathsf s, \mathsf s'}^\circ.
\]

\begin{lem}[{\cf \cite[Lemma 13]{Ohta}, \cite[Lemma 3.4]{Zh}}] \label{descko}
Let $\sO$ be a $K_{\mathsf s, \C}$-orbit in $\p_{\mathsf s}$. Suppose that $\sO$ is contained in the image of the moment map $M_\mathsf s$. Then the set
\be\label{kkpo}
  M_{\mathsf s}^{-1}(\sO)\cap \cX_{\mathsf s, \mathsf s'}^\circ
\ee
is a single $K_{\mathsf s,\C}\times K_{\mathsf s', \C}$-orbit. Moreover, for every element $\phi$ in $M_{\mathsf s}^{-1}(\sO)\cap \cX_{\mathsf s, \mathsf s'}^\circ$, there is an exact sequence of algebraic groups:
\[
  1\rightarrow (K_{\mathsf s',\C})_\phi \rightarrow (K_{\mathsf s,\C}\times K_{\mathsf s', \C})_\phi\xrightarrow{\textrm{the projection to the first factor}} (K_{\mathsf s,\C})_{\mathbf e}\rightarrow 1,
\]
where $\mathbf e:=M_\mathsf s(\phi)\in \sO$,  $(K_{\mathsf s',\C})_\phi$ is the stabilizer of $\phi$ in $K_{\mathsf s',\C}$, $(K_{\mathsf s,\C}\times K_{\mathsf s', \C})_\phi$ is the stabilizer of $\phi$ in $K_{\mathsf s,\C}\times K_{\mathsf s', \C}$, and $(K_{\mathsf s,\C})_{\mathbf e}$ is the stabilizer of $\mathbf e$ in $K_{\mathsf s,\C}$.
\end{lem}


In the notation of Lemma \ref{descko}, write
\[
  \nabla^{\mathsf s}_{\mathsf s'}(\sO):=\textrm{the image of the set \eqref{kkpo} under the moment map  $M_{\mathsf s'}$,}
\]
which is a $ K_{\mathsf s', \C}$-orbit in $\p_{\mathsf s'}$. This is called the descent of $\sO$. It is an element of $\mathrm{Nil}(\p_{\mathsf s'})$ if $\sO\in \mathrm{Nil}(\p_{\mathsf s})$.

Similar to \eqref{momentmap}, we have the following two well-defined algebraic maps:
   \be\label{momentmap2}
    \xymatrix@R=0em@C=4em{
      \g_\mathsf s &\ar[l]_{\tilde M_\mathsf s:=\tilde M_\mathsf s^{\sfss, \sfss'}} W_{\mathsf s, \mathsf s'}\ar[r]^{\tilde M_{\mathsf s'}:=\tilde M_{\mathsf s'}^{\sfss, \sfss'}}& \g_{\mathsf s'},\\
     \phi^* \phi & \ar@{|->}[l] \phi \ar@{|->}[r] & \phi \phi^*.
    }
  \ee
These two maps are also called the moment maps. Similar to the maps in \eqref{momentmap},   they  are both $G_{\mathsf s,\C}\times G_{\mathsf s', \C}$-equivariant.


We record the following lemma for later use. 
\begin{lem}\label{imageofmm}
Suppose that $\CO\in \mathrm{Nil}(\g_\sfss)$. Then  $\CO$ is contained in the image of the moment map $\tilde M_{\mathsf s}$ if and only if
\[
 \delta:=  \abs{\sfss'}-\abs{\DD_\mathrm{naive}(\CO)}\geq 0.
\]
When this is the case, the Young diagram of $\DD_{\mathsf s'}^{\mathsf s}(\CO)\in \mathrm{Nil}(\g_{\mathsf s'})$ is obtained from that of $\DD_\mathrm{naive}(\CO)$ by adding $\delta$ boxes in the first column.

\end{lem}
\begin{proof}
This is implied by \cite[Theorem 3.6]{DKPC}.
\end{proof}

Now we suppose that the $G_{\sfss, \C}$-orbit $\CO\subset\g_\sfss$
  is contained in the image of the moment map $\tilde M_{\mathsf s}$. Similar to
the first assertion of Lemma \ref{descko}, the set
\be\label{kkpo2}
  \tilde M_{\mathsf s}^{-1}(\CO)\cap W_{\mathsf s, \mathsf s'}^\circ
\ee
is a single $G_{\mathsf s,\C}\times G_{\mathsf s', \C}$-orbit.
Write
\[
 \CO':= \nabla^{\mathsf s}_{\mathsf s'}(\CO):=\textrm{the image of the set \eqref{kkpo2} under the moment map  $\tilde M_{\mathsf s'}$,}
\]
which is a $ G_{\mathsf s', \C}$-orbit in $\g_{\mathsf s'}$.
This is  called the descent of $\CO$. It is an element of  $\mathrm{Nil}(\g_{\mathsf s'})$ if $\CO\in \mathrm{Nil}(\g_{\mathsf s})$.



\subsection{Geometric theta lift} \label{sec:dlift}
Let $\zeta_{\mathsf s, \mathsf s'}$ denote the algebraic character on $K_{\mathsf s, \C}\times K_{\mathsf s', \C}$  such that
\[
 (\zeta_{\mathsf s, \mathsf s'})|_{K_{\mathsf s, \C}}=
   \begin{cases}
    1, \quad & \textrm{if $\star\in \{B, D, C^*\}$}; \smallskip\\
      ({\det}_{\sqrt{-1}})^{\frac{p'-q'}{2}}, \quad & \textrm{if $\star\in \{C, D^*\}$};\smallskip\\
     ({\det}_{\sqrt{-1}}^{\frac{1}{2}})^{p'-q'}, \quad & \textrm{if $\star=\widetilde C$,}\\
  \end{cases}
\]
and
\[
 (\zeta_{\mathsf s, \mathsf s'})|_{K_{\mathsf s', \C}}=
   \begin{cases}
    1, \quad & \textrm{if $\star\in \{C, \widetilde C, D^*\}$}; \smallskip\\
      ({\det}_{\sqrt{-1}})^{\frac{q-p}{2}}, \quad & \textrm{if $\star\in \{D, C^*\}$};\smallskip\\
     ({\det}_{\sqrt{-1}}^{\frac{1}{2}})^{q-p}, \quad & \textrm{if $\star=B$}.
 \end{cases}
\]
Here and henceforth, when no confusion is possible, we  use $1$ to indicate the trivial representation of a group (we also use $1$ to denote the identity element of a group).
Let  $\sO$ be a $K_{\mathsf s, \C}$-orbit in $\p_{\mathsf s}$ as before. Suppose that $\sO$ is contained in the image of the moment map $M_{\mathsf s}$, and write $\sO':=\nabla^{\mathsf s}_{\mathsf s'}(\sO)$.
Let $\phi,\mathbf e$ be as in Lemma  \ref{descko} and let $\mathbf e':=M_{\mathsf s'}(\phi)$. We have an exact sequence
\[
  1\rightarrow  (K_{\mathsf s',\C})_\phi\rightarrow (K_{\mathsf s,\C}\times K_{\mathsf s', \C})_\phi\rightarrow (K_{\mathsf s,\C})_{\mathbf e}\rightarrow 1
\]
as in Lemma  \ref{descko}.

 Let $\CE'$ be a $K_{\mathsf s',\C}$-equivariant algebraic vector bundle  over $\sO'$. Its fibre
$\CE_\mathbf e'$ at $\mathbf e'$ is an algebraic representation of the stabilizer group $(K_{\mathsf s',\C})_{\mathbf e'}$. We also view it as a representation of the group
$(K_{\mathsf s,\C}\times K_{\mathsf s', \C})_\phi$ by the pull-back through the homomorphism
\[
  (K_{\mathsf s,\C}\times K_{\mathsf s', \C})_\phi\xrightarrow{\textrm{the projection to the second factor}} (K_{\mathsf s',\C})_{\mathbf e'}.
\]
Then $\CE'_{\mathbf e'} \otimes \zeta_{\mathsf s, \mathsf s'}$ is a representation of $ (K_{\mathsf s,\C}\times K_{\mathsf s', \C})_\phi$, and the coinvariant space
\[
(\CE'_{\mathbf e'} \otimes \zeta_{\mathsf s, \mathsf s'})_{ (K_{\mathsf s',\C})_\phi}
\]
 is an algebraic representation of $(K_{\mathsf s,\C})_{\mathbf e}$. Write $\CE:= \check \vartheta_{\sO'}^{\sO}(\mathcal E')$ for the  $K_{\mathsf s,\C}$-equivariant algebraic vector bundle  over $\sO$ whose fibre at $\mathbf e$ equals this coinvariant space representation. In this way, we get an exact functor $  \check \vartheta_{\sO'}^{\sO}$ from the category of
$K_{\mathsf s',\C}$-equivariant algebraic vector bundle  over $\sO'$ to the category of $K_{\mathsf s,\C}$-equivariant algebraic vector bundle  over $\sO$. This exact functor induces a  homomorphism of the  Grothendieck groups:
\[
   \check \vartheta_{\sO'}^{\sO}:  \CK_{\mathsf s'}(\sO')\rightarrow  \CK_{\mathsf s}(\sO).
\]
The above homomorphism is independent of the choice of $\phi$.




%Here  $K_{\mathsf s', \C}$ acts trivially on $\p_\mathsf s$,  $K_{\mathsf s, \C}$ acts trivially on $\p_\mathsf s'$, and all the other actions are the obvious ones.





Recall that $\CO':= \nabla^{\mathsf s}_{\mathsf s'}(\CO)$, and finally we define the geometric theta lift to be the homomorphism
\[
 \check \vartheta_{\CO'}^{\CO}: \CK_{\mathsf s'}(\CO')\rightarrow \CK_{\mathsf s}(\CO)
\]
such that
\[
 \check \vartheta_{\CO'}^{\CO}(\CE')= \sum_{\sO\textrm{ is a $K_{\mathsf s, \C} $-orbit in $\CO\cap \p_{\mathsf s}$,  $\, \nabla^{\mathsf s}_{\mathsf s'}(\sO)=\sO'$}}    \check \vartheta_{\sO'}^{\sO}(\CE'),
\]
for all $K_{\mathsf s', \C} $-orbit $\sO'$ in $\CO'\cap \p_{\mathsf s'}$, and all $\CE'\in \CK_{\mathsf s'}(\sO')$.


\subsection{Associated cycles of painted bipartitions}\label{subsecass}


Let $\check \CO$ be a Young diagram that has $\star$-good parity.

\begin{defn}\label{bvdual0}
The  Barbasch-Vogan dual $\mathrm d_{\mathrm{BV}}^\star(\check \CO)$ of $\check \CO$ is the Young diagram satisfying
\begin{eqnarray*}
 &&\left (\mathbf c_i(\mathrm d_{\mathrm{BV}}^\star(\check \CO)), \mathbf c_{i+1}(\mathrm d_{\mathrm{BV}}^\star(\check \CO))\right )\\
  &=&\begin{cases}
      (\mathbf r_i(\check \CO), \mathbf r_{i+1}(\check \CO)),\quad& \textrm{if  $(i,i+1)$ is a $\star$-pair that is vacant or balanced in $\check \CO$};  \\
        (\mathbf r_i(\check \CO)-1, 0),\quad& \textrm{if  $(i,i+1)$ is a $\star$-pair that is  tailed in $\check \CO$};  \\
         (\mathbf r_i(\check \CO)-1, \mathbf r_{i+1}(\check \CO)+1),\quad& \textrm{if  $(i,i+1)$ is a $\star$-pair that is primitive in $\check \CO$},  
  \end{cases}
  \end{eqnarray*}
and 
\[
    \mathbf c_1(\mathrm d_{\mathrm{BV}}^\star(\check \CO))=\begin{cases}
      0,\qquad& \textrm{if  $\star\in \{D, D^*\}$ and $\check \CO=\emptyset$};  \\
       \mathbf r_1(\check \CO)+1, \qquad& \textrm{if  either $\star=B$, or $\star\in \{D, D^*\}$ and $\check \CO\neq \emptyset$.}  
          \end{cases}
\]

\end{defn}

Assume that 
\[
 \abs{\mathrm d_{\mathrm{BV}}^\star(\check \CO)}=\abs{\sfss}.
\]
Then  $\mathrm d_{\mathrm{BV}}^\star(\check \CO)$ is identified with an element of $\mathrm{Nil}(\g_\sfss)$. 
Recall  the ideal $I_{\check \CO}$ of $\oU(\g_\sfss)$ from Section \ref{secsu}.

% be as in Definition \ref{bvdual0}. Assume that $\abs{\CO}=\abs{\sfss}$. Then $\CO$ is 

\begin{lem}\label{bvdual}
The  associated variety of $I_{\check \CO}$ equals   $\mathrm d_{\mathrm{BV}}^\star(\check \CO)$.
\end{lem}
\begin{proof}
This is implied by \cite[Corollary A3]{BVUni} and \cite[Theorem B]{BMSZ1}. 
\end{proof}

%Then
%\[
%p+q=\abs{\CO}.
%\]

As before, let $\check \CO'$ be the dual descent of $\check \CO$, which has $\star'$-good parity. Assume that
\[
  \abs{\mathrm d_{\mathrm{BV}}^{\star'}(\check \CO')}=\abs{\sfss'}.
\]
Then  $\mathrm d_{\mathrm{BV}}^{\star'}(\check \CO')$ is identified with an element of $\mathrm{Nil}(\g_{\sfss'})$. 

\begin{lem}\label{dualdesc}
The orbit $\mathrm d_{\mathrm{BV}}^{\star}(\check \CO)$ is contained in the image of the moment map $\tilde M_{\mathsf s}$, and
\[
  \nabla_{\mathsf s'}^{\mathsf s}(\mathrm d_{\mathrm{BV}}^{\star}(\check \CO))=\mathrm d_{\mathrm{BV}}^{\star'}(\check \CO').
\]
\end{lem}
\begin{proof}
This is elementary to check by using Lemmas \ref{imageofmm}.
\end{proof}

In view of Lemma \ref{dualdesc}, we suppose that 
\[
\CO=\mathrm d_{\mathrm{BV}}^\star(\check \CO)\in \mathrm{Nil}(\g_\sfss)\quad \textrm{and}\quad \CO'=\mathrm d_{\mathrm{BV}}^{\star'}(\check \CO')\in \mathrm{Nil}(\g_{\sfss'}).
\] 

Put
\[
  \PBPe_\star(\check \CO,\mathsf s):=\{(\tau, \wp)\in  \PBPe_\star(\check \CO)\mid (p_\tau,q_\tau)=(p,q)\}.
\]
Let $\uptau=(\tau, \wp)\in \PBPe_\star(\check \CO)$. 
In what follows we will define the associated cycle $\mathrm{AC}(\uptau)\in\CK_{\mathsf s}(\CO)$ of $\uptau$.

For the initial case when $\check \CO$ is the empty Young diagram so that $\CO\cap \p_{\mathsf s}$ is a singleton, we define  $\mathrm{AC}(\uptau)\in\CK_{\mathsf s}(\CO)$ to be the element that corresponds to the following  representation of $K_{\mathsf s,\C}$: 
 \[
\left\{
     \begin{array}{ll}
                    \textrm{the determinant character} , &\hbox{ if $\alpha_\tau=B^-$ so that $K_{\mathsf s,\C}=\rO(0,1)$;} \smallskip\\
                    \textrm{the one dimensional genuine representation} , &\hbox{ if $\star=\widetilde C$ so that $K_{\mathsf s,\C}=\{\pm 1\}$;} \smallskip\\
 \textrm{the one dimensional trivial representation} , &\hbox{ otherwise.}            \end{array}
   \right.
\]

 Write  $\uptau'\in \PBPe_{\star'}(\check \CO')$ for the descent of $\uptau$, and assume that $\uptau'\in \PBPe_{\star'}(\check \CO',\mathsf s')$.
When $\check \CO\neq\emptyset$,  similar to the definition of $\pi_\uptau$ in the introductory section, we inductively define
$\mathrm{AC}(\uptau)\in \CK_{\mathsf s}(\CO) $ by
 \[
   \mathrm{AC}(\uptau):=\left\{
     \begin{array}{ll}
          %\textrm{the trivial representation $\C$}, &\hbox{if $\abs{\check \CO_\tau}\leq 1$}; \medskip\\
         \check \vartheta_{\CO'}^{\CO}(\mathrm{AC}(\uptau'))\otimes ({\det}_{-1})^{\varepsilon_{\tau}}, &\hbox{if  $\star=B$ or $D$}; \smallskip\\
         \check \vartheta_{\CO'}^{\CO}(\mathrm{AC}(\uptau')\otimes \det^{\varepsilon_{\wp}}), &\hbox{if $\star=C$ or $\widetilde C$};\smallskip \\
              \check \vartheta_{\CO'}^{\CO}(\mathrm{AC}(\uptau')), &\hbox{if $\star=C^*$ or $D^*$}. \\
            \end{array}
   \right.
 \]
 
We have the following three analogous results of Theorem \ref{thmac0}.

\begin{prop}\label{thmac1}
 For every $\uptau\in \PBPe_\star(\check \CO)$,  the associated cycle $\mathrm{AC}(\uptau)\in \CK_{\mathsf s}(\CO)$ is a nonzero  sum of  admissible orbit data 
over pairwise distinct $K_{\sfss,\C}$-orbits in $\CO\cap \p$.
%  a nonzero  sum of  pairwise distinct elements of $\mathrm{AOD}_{\CO}(K_{\mathsf s, \C})$.
\end{prop}

\begin{proof}
  This follows from \Cref{lem:aod} and \Cref{lem:ac0}.
\end{proof}

\begin{prop}\label{thmac2}
Suppose that  $\check \CO$ is weakly-distinguished.  If $\star\in \{B,D\}$ and $(\star, \check \CO)\neq (D, \emptyset)$, then  the map
\[
\mathrm{AC}: \PBPe_\star(\check \CO,\mathsf s)\times \Z/2\Z \rightarrow  \CK_{\mathsf s}(\CO),\quad (\uptau, \epsilon)\mapsto \mathrm{AC}(\uptau)\otimes {\det}^{\epsilon}
\]
is injective. In all other cases, the
map
\[
\mathrm{AC}: \PBPe_\star(\check \CO,\mathsf s)\rightarrow  \CK_{\mathsf s}(\CO)
\]
is injective.
\end{prop}
\begin{proof}
When $\star\in \set{B,D}$, this follows from \Cref{lem:BD}~(b).
When $\star = C^{*}$, this follows from \Cref{lem:C*}.
When $\star\in \set{C,\wtC,D^{*}}$, this follows from \Cref{lem:C}~(b).
\end{proof}



\begin{prop}\label{thmac3}
Suppose that  $\check \CO$ is quasi-distinguished.  If $\star\in \{B,D\}$ and $(\star, \check \CO)\neq (D, \emptyset)$, then  the map
\[
\mathrm{AC}: \PBPe_\star(\check \CO,\mathsf s)\times \Z/2\Z \rightarrow  \mathrm{AOD}_{\sfss}(\CO),\quad (\uptau, \epsilon)\mapsto \mathrm{AC}(\uptau)\otimes {\det}^{\epsilon}
\]
is well-defined and bijective. In all other cases, the
map
\[
\mathrm{AC}: \PBPe_\star(\check \CO,\mathsf s)\rightarrow \mathrm{AOD}_{\sfss}(\CO)
\]
is well-defined and bijective.
\end{prop}
\begin{proof}
When $\star\in \set{B,D}$, this follows from \Cref{lem:BD}~(c).
When $\star = C^{*}$, this follows from \Cref{lem:C*}.
When $\star\in \set{C,\wtC,D^{*}}$, this follows from \Cref{lem:C}~(c).
\end{proof}

The following two propositions will also be important for us.
\begin{prop}\label{thmac4}
Suppose that  $\star\in \{B,D\}$ and $(\star, \check \CO)\neq (D, \emptyset)$. Let $\uptau_i=(\tau_i, \wp_i)\in \PBPe_\star(\check \CO,\mathsf s)$ and $\epsilon_i\in \Z/2\Z$ ($i=1,2$).   If
\[
  \mathrm{AC}(\uptau_1)\otimes {\det}^{\epsilon_1}= \mathrm{AC}(\uptau_2)\otimes {\det}^{\epsilon_2},
\]
then
\[
  \epsilon_1=\epsilon_2\qquad\textrm{and}\qquad \varepsilon_{\tau_1}=\varepsilon_{\tau_2}.
\]
 \end{prop}
\begin{proof}
  This follows from \Cref{lem:BD}~(a).
\end{proof}

\begin{prop}\label{thmac5}
Suppose that  $\star\in \{C,\widetilde C, D^*\}$. Let $\uptau_i=(\tau_i, \wp_i)\in \PBPe_\star(\check \CO,\mathsf s)$, and write $\uptau'_i=(\tau'_i, \wp'_i)$ for its descent ($i=1,2$).   If
\[
  \mathrm{AC}(\uptau_1)= \mathrm{AC}(\uptau_2),
\]
then
\[
 ( p_{\tau'_1}, q_{\tau'_1})=( p_{\tau'_2}, q_{\tau'_2})\qquad\textrm{and}\qquad \varepsilon_{\wp_1}=\varepsilon_{\wp_2}.
\]
 \end{prop}

\begin{proof}
  This follows from \Cref{lem:C}~(a).
\end{proof}


\subsection{Distinguishing the constructed representations}\label{secd}


Let $\uptau\in \PBPe_\star(\check \CO,\mathsf s)$  as before. Recall that $\pi_\uptau$ is the Casselman-Wallach representation of $G_{\mathsf s}$, defined in the introductory section.
We will prove the following theorem in Section \ref{sec:comANDgeo}.

\begin{thm}\label{thmpitau}
The representation $\pi_\uptau$ is irreducible, unitarizable and attached to $\check \CO$. Moreover,
\[
\mathrm{AC}_\CO(\pi_\uptau)=\mathrm{AC}(\tau)\in \CK_{\mathsf s}(\CO).
\]
\end{thm}


Recall from the introductory section, the set $\Unip_{\check \cO}(G_\mathsf s)$ of isomorphism classes of irreducible Casselman-Wallach representations of $G_\mathsf s$ that are attached to $\check \CO$.

\begin{thm}\label{thmac7}
If $\star\in \{B,D\}$ and $(\star, \check \CO)\neq (D, \emptyset)$, then  the map
\be\label{bijthm1}
 \PBPe_\star(\check \CO,\mathsf s)\times \Z/2\Z \rightarrow \Unip_{\check \cO}(G_\mathsf s),\quad (\uptau, \epsilon)\mapsto \pi_\uptau\otimes {\det}^{\epsilon}
\ee
is bijective. In all other cases, the
map
\be\label{bijthm2}
\PBPe_\star(\check \CO,\mathsf s)\rightarrow \Unip_{\check \cO}(G_\mathsf s), \quad \uptau\mapsto \pi_\tau
\ee
is bijective.
\end{thm}
\begin{proof}
In view of Theorem \ref{thmcount}, we only need to show the injectivity of  the two maps in the statement of the theorem. We prove by induction on the number of nonempty rows of $\check \CO$. The theorem is trivially true in the case when  $\check \CO=\emptyset$. So we assume that $\check \CO\neq \emptyset$  and the theorem has been proved for the dual descent $\check \CO'$.
Suppose that  $\uptau_i=(\tau_i, \wp_i)\in \PBPe_\star(\check \CO,\mathsf s)$, $\epsilon_i\in \Z/2\Z$ ($i=1,2$). Write  $\uptau_i'=(\tau_i', \wp_i')$ for the descent of $\uptau_i$.

%\noindent {\bf The case when $\star\in \{B,D\}$}.

First assume that $\star\in \{B,D\}$, and
\[
  \pi_{\uptau_1}\otimes {\det}^{\epsilon_1}\cong \pi_{\uptau_2}\otimes {\det}^{\epsilon_2}.
\]
Theorem \ref{thmpitau} implies that
\[
\mathrm{AC}(\uptau_1)\otimes {\det}^{\epsilon_1}=\mathrm{AC}(\uptau_2)\otimes {\det}^{\epsilon_2}.
\]
By Proposition \ref{thmac4}, we know that
\[
  \epsilon_1=\epsilon_2\qquad\textrm{and}\qquad \varepsilon_{\tau_1}=\varepsilon_{\tau_2}.
\]
Then the definition of $ \pi_{\uptau_1}$ and $ \pi_{\uptau_2}$ implies that
\[
 \check  \Theta_{\tau_1'}^{\tau_1}(\pi_{\uptau_1'})\cong \check \Theta_{\tau_2'}^{\tau_2}(\pi_{\uptau_2'}).
\]
 Consequently,
\[
\pi_{\uptau_1'}\cong \pi_{\uptau_2'}
\]
by the injectivity property of the theta correspondence. Hence $\uptau'_1=\uptau'_2$ by the induction hypothesis, and Proposition \ref{prop:DD.BDinj} finally implies $\uptau_1=\uptau_2$.
This proves that the map \eqref{bijthm1} is injective.

A slightly simplified argument shows that the map \eqref{bijthm2} is injective when $\star=C^*$.


Now assume that $\star\in \{C,\widetilde C, D^*\}$. Suppose that
\[
  \pi_{\uptau_1}\cong \pi_{\uptau_2}.
\]
Theorem \ref{thmpitau} implies that
\[
\mathrm{AC}(\uptau_1)=\mathrm{AC}(\uptau_2).
\]
By Proposition \ref{thmac5}, we know that
\[
 ( p_{\tau'_1}, q_{\tau'_1})=( p_{\tau'_2}, q_{\tau'_2})\qquad\textrm{and}\qquad \varepsilon_{\wp_1}=\varepsilon_{\wp_2}.
\]
Then the definition of $ \pi_{\uptau_1}$ and $ \pi_{\uptau_2}$ implies that
\[
  \check \Theta_{\tau_1'}^{\tau_1}(\pi_{\uptau_1'}\otimes {\det}^{\varepsilon_{\wp_1}})\cong \check \Theta_{\tau_2'}^{\tau_2}(\pi_{\uptau_2'}\otimes {\det}^{\varepsilon_{\wp_1}}).
\]
The injectivity property of the theta correspondence then implies that
\[
\pi_{\uptau_1'}\otimes {\det}^{\varepsilon_{\wp_1}}\cong \pi_{\uptau_2'}\otimes {\det}^{\varepsilon_{\wp_2}},
\]
which further implies that $\pi_{\uptau_1'}\cong \pi_{\uptau_2'}$.
Hence $\uptau'_1=\uptau'_2$ by the induction hypothesis, and Proposition \ref{prop:DD.BDinj} finally implies $\uptau_1=\uptau_2$. This proves that the map \eqref{bijthm2} is injective.

A slightly simplified argument shows that the map \eqref{bijthm2} is injective when $\star=D^*$. This finishes the proof of the theorem.



\end{proof}

Finally, our first main theorem (Theorem \ref{thm1}) follows from Theorems \ref{thmpitau} and \ref{thmac7}. Our second main theorem (Theorem \ref{thmac0}) follows from Propositions \ref{thmac1} and \ref{thmac2}, and Theorems \ref{thmpitau} and \ref{thmac7}.



\section{Theta lifts via matrix coefficient integrals}\label{sec:Integrals}% {and degenerate principal series}

In this section,
we study theta lift via
matrix coefficient integrals. Let $\mathsf s=(\star, p,q)$ and $ \mathsf s'=(\star', p',q')$ be classical signatures such that $\star'$ is the Howe dual of $\star$.



\subsection{The oscillator representation}\label{secoscil}
We use the notation of Section \ref{secmmap}. Write
$
  W_{\mathsf s, \mathsf s'}^{J_{\mathsf s, \mathsf s'}}\subset W_{\mathsf s, \mathsf s'}
$
for the fixed point set of $J_{\mathsf s, \mathsf s'}$. It is a real symplectic space under the restriction of the form $\la\,,\,\ra_{\mathsf s, \mathsf s'}$. Let $H_{\mathsf s, \mathsf s'}:= W_{\mathsf s, \mathsf s'}^{J_{\mathsf s, \mathsf s'}}\times \R$
denote the Heisenberg group attached to $W_{\mathsf s, \mathsf s'}^{J_{\mathsf s, \mathsf s'}}$, with group multiplication
\[
  (\phi ,t)\cdot (\phi ', t'):=(\phi +\phi ', t+t'+\la \phi , \phi '\ra_{\mathsf s, \mathsf s'}), \qquad \phi ,\phi '\in  W_{\mathsf s, \mathsf s'}^{J_{\mathsf s, \mathsf s'}}, \quad t, t'\in \R.
\]
Denote by $\h_{\mathsf s, \mathsf s'}$ the complexified Lie algebra of $H_{\mathsf s, \mathsf s'}$. Then  $\cX_{\mathsf s, \mathsf s'}$ (as in \eqref{def:Xss'}) is  an abelian Lie subalgebra of $\h_{\mathsf s, \mathsf s'}$.

The following is the smooth version of the Stone-von Neumann Theorem.

\begin{lem}[{\cf \cite[Theorem 5.1]{Cl89} and \cite[Section 4]{Ad07}}]\label{vn} 
Up to isomorphism, there exists a unique irreducible smooth Fr\'echet representation $\omega_{\mathsf s, \mathsf s'}$ of $H_{\mathsf s, \mathsf s'}$ of moderate growth with central character
\[
\R\rightarrow \C^\times, \ t\mapsto e^{\sqrt{-1}\, t}.
\]
Moreover, the space
\[
  \omega_{\mathsf s, \mathsf s'}^{\cX_{\mathsf s, \mathsf s'}}:=\{v\in \omega_{\mathsf s, \mathsf s'}\mid x \cdot v=0\quad \textrm{for all $x\in \cX_{\mathsf s, \mathsf s'}$}\}
\]
is one-dimensional.
\end{lem}



The group $G_{\mathsf s}\times G_{\mathsf s'}$ acts on $H_{\mathsf s, \mathsf s'}$ as group automorphisms via the following natural action of $G_{\mathsf s}\times G_{\mathsf s'}$ on  $W_{\mathsf s, \mathsf s'}^{J_{\mathsf s, \mathsf s'}}$:
\[
  (g, g')\cdot \phi:=g'\circ \phi\circ g^{-1}, \qquad (g,g')\in G_{\mathsf s}\times G_{\mathsf s'},\ \phi\in W_{\mathsf s, \mathsf s'}^{J_{\mathsf s, \mathsf s'}}.
\]
From this action, we form the semidirect product $(G_{\mathsf s}\times G_{\mathsf s'})\ltimes H_{\mathsf s, \mathsf s'}$.

Recall the character $\zeta_{\mathsf s, \mathsf s'}$ from Section \ref{sec:dlift}.

%As in   Section \ref{sec:LVB}, we assume that $\mathbb p$ is the parity of $\dim \bfV$ if $\epsilon=1$, and  the parity of $\dim \bfV'$ if $\epsilon '=1$.
\begin{lem}[{\cf \cite[Proposition 7.5]{Ad07}}]\label{deforos}
The representation $\omega_{\mathsf s, \mathsf s'}$ of $H_{\mathsf s, \mathsf s'}$ in Lemma \ref{vn} uniquely extends to a  smooth representation of $(G_{\mathsf s}\times G_{\mathsf s'})\ltimes H_{\mathsf s, \mathsf s'}$ such  that $K_{\mathsf s}\times  K_{\mathsf s'}$ acts on  $ \omega_{\mathsf s, \mathsf s'}^{\cX_{\mathsf s, \mathsf s'}}$ through the scalar multiplication by $\zeta_{\mathsf s, \mathsf s'}$.
\end{lem}

Let $\omega_{\mathsf s, \mathsf s'}$ denote the representation of $(G_{\mathsf s}\times G_{\mathsf s'})\ltimes H_{\mathsf s, \mathsf s'}$ as in Lemma \ref{deforos}, henceforth called the smooth oscillator representation or simply the oscillator representation. As is well-known, this representation is unitarizable \cite{Weil}. Fix an invariant continuous Hermitian inner product $\la\,,\,\ra$ on it, which is unique up to a positive scalar multiplication.
Denote by $\hat \omega_{\mathsf s, \mathsf s'}$ the completion of $\omega_{\mathsf s, \mathsf s'}$ with respect to this inner product, which is a unitary representation of  $(G_{\mathsf s}\times G_{\mathsf s'})\ltimes H_{\mathsf s, \mathsf s'}$.

 Let $\omega_{\mathsf s, \mathsf s'}^\vee$ denote the  contragredient of the  oscillator  representation $\omega_{\mathsf s, \mathsf s'}$. It is  the smooth Fr\'echet representation of $(G_{\mathsf s}\times G_{\mathsf s'})\ltimes H_{\mathsf s, \mathsf s'}$ of moderate growth specified by the following conditions:
 \begin{itemize}
 \item there is given a  $(G_{\mathsf s}\times G_{\mathsf s'})\ltimes H_{\mathsf s, \mathsf s'}$-invariant, non-degenerate, continuous bilinear form
 \[
   \la\,,\,\ra:  \omega_{\mathsf s, \mathsf s'}\times \omega_{\mathsf s, \mathsf s'}^\vee\rightarrow \C;
 \]
 \item  $\omega_{\mathsf s, \mathsf s'}^\vee$  is irreducible as a representation of $H_{\mathsf s, \mathsf s'}$.
 \end{itemize}
Since $\omega_{\mathsf s, \mathsf s'}$ is contained in the unitary representation $\hat \omega_{\mathsf s, \mathsf s'}$, $\omega_{\mathsf s, \mathsf s'}^\vee$ is identified with the complex conjugation of $\omega_{\mathsf s, \mathsf s'}$.

Put
\[%\be\label{sfsspm}
   {\sfss'}^-:=(\star', q', p'),
\]
Fix a linear isomorphism
\be\label{iota0}
 \iota_{\sfss'}:  V_{\sfss'}\rightarrow V_{\sfss'^-}, %\quad v\mapsto v^-
\ee
such that $(-\la\,,\,\ra_{\sfss'}, J_{\sfss'},-L_{\sfss'})$ corresponds to   $(\la\,,\,\ra_{\mathsf s'^-}, J_{\mathsf s'^-}, L_{\mathsf s'^-})$ under this isomorphism.
This induces an isomorphism
\be\label{isoggp}
  G_{\sfss'}\rightarrow G_{\sfss'^-}, \qquad g'\mapsto g'^-.
\ee
Then we have a group isomorphism
\be\label{identifyjg}
\begin{array}{rcl}
  (G_{\mathsf s}\times G_{\mathsf s'})\ltimes H_{\mathsf s, \mathsf s'}&\rightarrow& (G_{\mathsf s'^-}\times G_{\mathsf s})\ltimes H_{\mathsf s'^-, \mathsf s},\smallskip\\
   (g, g', (\phi, t))&\mapsto & (g'^-, g, (\phi^*  \circ \iota_{\sfss'}^{-1}, t)).
   \end{array}
\ee
It is easy to see that the irreducible representation $\omega_{\sfss, \sfss'}$ corresponds to the irreducible representation $\omega_{\sfss'^-, \sfss}$ under this isomorphism.


 For every Casselman-Wallach representation $\pi'$ of $G_{\mathsf s'}$,
put
\[
   \check \Theta_{\mathsf s'}^{\mathsf s}(\pi'):=(\omega_{\mathsf s, \mathsf s'}\widehat \otimes \pi')_{G_{\mathsf s'}} \qquad (\textrm{the  Hausdorff coinvariant space}).
\]
This is a Casselman-Wallach representation of $G_{\mathsf s}$.



\subsection{Growth of Casselman-Wallach representations}
%Let $\mathsf s=(\star, p,q)$  be a classical signature.

Write $\p_\mathsf s^{J_\mathsf s}$ for the centralizer of $J_\mathsf s$ in $\p_\mathsf s$, which is a real form of $\p_\mathsf s$. The Cartan decomposition asserts that
\[
  G_\mathsf s=K_\mathsf s\cdot \exp(\p_\mathsf s^{J_\mathsf s}).
\]
Denote by  $\Psi_\mathsf s$ the function of $G_\mathsf s$ satisfying the following conditions:
\begin{itemize}
\item it is both left and right $K_\mathsf s$-invariant;
\item for all $g\in \exp(\p_\mathsf s^{J_\mathsf s})$,
\[
  \Psi_\mathsf s(g)=\prod_{a} \left(\frac{1+a}{2}\right)^{-\frac{1}{2}},
\]
 where $a$ runs over all eigenvalues of $g: V_\mathsf s\rightarrow V_\mathsf s$, counted with multiplicities.
\end{itemize}
Note that all the  eigenvalues of $g \in \exp(\p_\mathsf s^{J_\mathsf s})$ are positive real numbers, and  $0<\Psi_\mathsf s(g)\leq 1$ for all $g\in G_\mathsf s$.

Set
\[
  \nu_\mathsf s:=\begin{cases}
    \abs{\mathsf s},\quad &\textrm{if }\star\in \{C, \widetilde C\};\\
     \abs{\mathsf s}-1,\quad & \textrm{if }\star= C^*;\\
      \abs{\mathsf s}-2,\quad & \textrm{if }\star\in \{B,D\};\\
       \abs{\mathsf s}-3,\quad & \textrm{if }\star= D^*.
  \end{cases}
\]

Denote by $\Xi_\sfss$ the bi-$K_\sfss$-invariant Harish-Chandra's $\Xi$ function on $G_\sfss$.

\begin{lem}\label{boundpsi}
There exists a real number $C_\sfss>0$ such that
\[
  \Psi_\sfss^{\nu_\sfss}(g)\leq C_\sfss\cdot \Xi_\sfss (g)\quad\textrm{ for all $g\in G_\sfss$}.
\]

\end{lem}
\begin{proof}
This is implied by the well-known  estimate of Harish-Chandra's $\Xi$ function (\cite[Theorem 4.5.3]{Wa1}).
\end{proof}



\begin{lem}\label{int}
Let $f$ be a bi-$K_\sfss$-invariant positive function  on $G_\sfss$  such that
\[
  \p_\mathsf s^{J_\mathsf s}\rightarrow \R, \qquad x\mapsto f(\exp(x))
\]
is a polynomial function. Then for every real number $\nu>0$, the function $f\cdot \Psi_\mathsf s^\nu\cdot \Xi_\sfss^2$ is integrable with respect to a Haar measure on $G_\mathsf s$.
\end{lem}
\begin{proof}
  This follows from the integral formula for $G_\mathsf s$ under the Cartan decomposition (see
 \cite[Lemma 2.4.2]{Wa1}), as well as the  estimate of Harish-Chandra's $\Xi$ function (\cite[Theorem 4.5.3]{Wa1}).
\end{proof}

For every Casselman-Wallach representation $\pi$ of $G_\mathsf s$, write $\pi^\vee$ for its contragredient representation, which is a  Casselman-Wallach representation  of $G_\mathsf s$ equipped with a $G_{\mathsf s}$-invariant, non-degenerate, continuous bilinear form
 \[
   \la\,,\,\ra: \pi \times\pi^\vee\rightarrow \C.
 \]

\begin{defn}
Let $\nu\in \R$. A  positive function $\Psi$ on $G_\mathsf s$ is $\nu$-bounded if there is a  function $f$ as in Lemma \ref{int}  such that
\[
  \Psi(g)\leq f(g)\cdot \Psi_{\sfss}^\nu(g)\cdot \Xi_\sfss(g)\qquad \textrm{for all }g\in G_\sfss.
\]
A Casselman-Wallach representation $\pi$ of $G_\mathsf s$ is said to be $\nu$-bounded if there
exist a $\nu$-bounded positive function $\Psi$ on $G_\sfss$, and continuous seminorms $\abs{\,\cdot\,}_{\pi}$ and $\abs{\,\cdot\,}_{\pi^\vee}$ on  $ \pi$ and $\pi^\vee$ respectively such that
\[
 \abs{ \la g \cdot u, v\ra}\leq \Psi(g)\cdot \abs{u}_{\pi}\cdot \abs{v}_{\pi^\vee}
\]
for all $u\in \pi$, $v\in \pi^\vee$, and $g\in G_{\mathsf s}$.

\end{defn}

Let $X_\sfss$ be a maximal $J_\sfss$-stable totally isotropic subspace of $V_\sfss$, which is unique  up to the action of $K_\sfss$. Put $Y_\sfss:=L_\sfss(X_\sfss)$. Then $X_\sfss\cap Y_\sfss=\{0\}$.
Write
\[
P_\sfss=R_\sfss\ltimes N_\sfss
\]
 for the parabolic subgroup of $G_\sfss$ stabilizing $X_\sfss$, where $R_\sfss$ is the Levi subgroup stabilizing both $X_\sfss$ and $Y_\sfss$, and $N_\sfss$ is the unipotent radical. For every character $\chi: R_\sfss\rightarrow \C^\times$, view it as a character of $P_\sfss$ that is trivial on $N_\sfss$. Write
\[
  I(\chi):=\Ind_{P_\sfss}^{G_\sfss}\chi \qquad(\textrm{normalized smooth induction}),
\]
which is a Casselman-Wallach representation of $G_\sfss$ under the right translations. Note that the representations   $I(\chi)$ and $ I(\chi^{-1})$ are contragredients of each other with the $G_\sfss$-invariant pairing
 \[
  \la\,,\,\ra:  I(\chi)\times I(\chi^{-1})\rightarrow \C, \quad (f, f')\mapsto \int_{K_\sfss} f(g) \cdot f'(g) \od\!g.
 \]
Let $\nu_\chi$ be a real number such that $\abs{\chi}$ equals the composition of
\be\label{nuchi}
   R_\sfss\xrightarrow{\textrm{the natural homomorphism} }\GL(X_\sfss)\xrightarrow{\abs{\det}^{\nu_\chi}} \C^\times.
\ee

\begin{defn}
A classical signature $\sfss$ is split if $V_\sfss=X_\sfss\oplus Y_\sfss$.
\end{defn}

When $\sfss$ is split, $P_\sfss$ is called a Siegel parabolic subgroup of $G_\sfss$.

\begin{lem}\label{growthdp}
Suppose that $\sfss$ is split and let $\chi: P_\sfss\rightarrow \C^\times$ be a character. Then there is a positive function $\Psi$ on $G_\sfss$ with the following properties:
\begin{itemize}
\item
 $\Psi$ is $(1-2\abs{\nu_\chi}-\frac{\abs{\sfss}}{2})$-bounded if $\star\in \{B,C,D, \widetilde C\}$, and  $(2-2\abs{\nu_\chi}-\frac{\abs{\sfss}}{2})$-bounded if $\star\in \{C^*,D^*\}$;
 \item
for all $f\in I(\chi)$, $f'\in I(\chi^{-1})$ and $g\in G_\sfss$,
\be\label{psi123}
\abs{ \la g.f, f'\ra}\leq \Psi(g) \cdot \abs{f|_{K_\sfss}}_\infty \cdot \abs{f'|_{K_\sfss}}_\infty\qquad(\textrm{$\abs{\,\,}_\infty$ stands for the suppernorm}).
\ee
\end{itemize}
\end{lem}
\begin{proof}

 Let $f_0$ denote the element in $I(\abs{\chi})$ such that $(f_0)|_{K_\sfss}=1$, and likewise let $f'_0$ denote the element in $I(\abs{\chi^{-1}})$ such that $(f'_0)|_{K_\sfss}=1$. Put
 \[
   \Psi(g):=\la g \cdot f_0, f_0'\ra, \qquad g\in G_\sfss.
 \]
 Then it is easy to see that \eqref{psi123} holds. Note that $\Psi$ is an elementary spherical function, and the lemma then follows by using the well-known estimate of the elementary spherical functions (\cite[Lemma 3.6.7]{Wa1}).
\end{proof}



\subsection{Matrix coefficient integrals against the oscillator representation}



We begin with the following lemma.



\begin{lem}\label{matrico}
 There exist continuous seminorms $\abs{\,\cdot\,}_{\mathsf s, \mathsf s'}$ and $\abs{\,\cdot\,}_{\mathsf s, \mathsf s'}^\vee$ on  $ \omega_{\mathsf s, \mathsf s'}$ and $\omega_{\mathsf s, \mathsf s'}^\vee$ respectively such that
\[
 \abs{ \la (g,g')\cdot u, v\ra}\leq \Psi_{\mathsf s}^{\abs{\mathsf s'}}(g)\cdot \Psi_{\mathsf s'}^{\abs{\mathsf s}}(g')\cdot \abs{u}_{\mathsf s, \mathsf s'}\cdot \abs{v}_{\mathsf s, \mathsf s'}^\vee
\]
for all $u\in \omega_{\mathsf s, \mathsf s'}$, $v\in \omega_{\mathsf s, \mathsf s'}^\vee$, and $(g,g')\in G_{\mathsf s}\times G_{\mathsf s'}$.
\end{lem}
\begin{proof}
  This follows from the  proof of \cite[Theorem 3.2]{Li89}.
\end{proof}

 \begin{defn}\label{defn:CRcov}
A Casselman-Wallach representation of $G_{\mathsf s'}$ is convergent for $\check \Theta_{\mathsf s'}^{\mathsf s}$
if it is $\nu$-bounded for some $\nu>\nu_{\mathsf s'}-\abs{\mathsf s}$.
\end{defn}


\begin{Example}
Suppose that $\star'\neq \widetilde C$. Then the trivial representation of $G_{\mathsf s'}$ is convergent for $\check \Theta_{\mathsf s'}^{\mathsf s}$ if $\abs{\sfss}>2\nu_{\sfss'}$.

\end{Example}
Let $\pi'$ be a  Casselman-Wallach representation of $G_{\mathsf s'}$ that is convergent for $\check \Theta_{\mathsf s'}^{\mathsf s}$.
Consider the integrals
\be\label{convint00}
\begin{array}{rcl}
 (\pi' \times \omega_{\mathsf s, \mathsf s'})\times (\pi'^\vee \times \omega_{\mathsf s, \mathsf s'}^\vee )&\rightarrow &\C, \smallskip \\
   ((u,v),(u',v')) &\mapsto &\int_{G_{\mathsf s'}} \la g\cdot u, u'\ra\cdot \la g\cdot v, v'\ra \od\! g.
   \end{array}
 \ee
Unless otherwise specified, all the measures on Lie groups occurring  in this article are  Haar measures.

\begin{lem}\label{lemconv}
The integrals in \eqref{convint00} are absolutely convergent and the map \eqref{convint00} is   continuous and multi-linear.
\end{lem}
\begin{proof}
This is a direct consequence of Lemmas \ref{boundpsi}, \ref{int} and \ref{matrico}.
\end{proof}

By Lemma \ref{lemconv}, the integrals in \eqref{convint00} yield a continuous bilinear form
\be\label{convint01}
 (\pi' \widehat \otimes \omega_{\mathsf s, \mathsf s'})\times (\pi'^\vee \widehat \otimes \omega_{\mathsf s, \mathsf s'}^\vee )\rightarrow \C.
 \ee
Put
\begin{equation}\label{thetab0}
  \Thetab_{\mathbf s'}^{\mathbf s}(\pi'):=\frac{\pi' \widehat \otimes \omega_{\mathsf s, \mathsf s'}}{\textrm{the left kernel of \eqref{convint01}}}.
\end{equation}

\begin{prop}\label{boundm}
The representation $\Thetab_{\mathbf s'}^{\mathbf s}(\pi')$ of $G_{\mathsf s}$ is a quotient of  $\check \Theta_{\mathbf s'}^{\mathbf s}(\pi')$, and is  $(\abs{\mathsf s'}-\nu_\sfss)$-bounded.
\end{prop}
\begin{proof}
Note that the bilinear form \eqref{convint01} is  $(G_{\mathsf s'}\times G_{\mathsf s'})$-invariant, as well as $G_{\mathsf s}$-invariant. Thus $\Thetab_{\mathbf s'}^{\mathbf s}(\pi')$ is a quotient of  $\check \Theta_{\mathbf s'}^{\mathbf s}(\pi')$, and is therefore a Casselman-Wallach representation. Its contragedient representation
is identified with
\[
(\Thetab_{\mathbf s'}^{\mathbf s}(\pi'))^\vee:=\frac{\pi'^\vee \widehat \otimes \omega^\vee_{\mathsf s, \mathsf s'}}{\textrm{the right kernel of \eqref{convint01}}}.
\]
 Lemmas \ref{int} and \ref{matrico} implies that there are continuous seminorms $\abs{\,\cdot\,}_{\pi', \mathsf s, \mathsf s'}$ and $\abs{\,\cdot\,}_{\pi'^\vee, \mathsf s, \mathsf s'}$ on  $\pi' \widehat \otimes \omega_{\mathsf s, \mathsf s'}$ and $\pi'^\vee \widehat \otimes \omega^\vee_{\mathsf s, \mathsf s'}$ respectively such that
\[
 \abs{ \la g. u, v\ra}\leq \Psi_{\mathsf s}^{\abs{\mathsf s'}}(g)\cdot \abs{u}_{\pi', \mathsf s, \mathsf s'}\cdot \abs{v}_{\pi'^\vee, \mathsf s, \mathsf s'}
\]
for all $u\in \pi' \widehat \otimes \omega_{\mathsf s, \mathsf s'}$, $v\in \pi'^\vee \widehat \otimes \omega^\vee_{\mathsf s, \mathsf s'}$, and $g\in G_{\mathsf s}$.
The proposition  then easily follows in view of  Lemma \ref{boundpsi}.
 \end{proof}







\subsection{Unitarity}

For the notion of weakly containment of unitary representations, see \cite{CHH} for example.

\begin{lem}\label{weaklycont}
Suppose that $\abs{\mathsf s}\geq  \nu_{\mathsf s'}$. Then as a unitary representation of $G_{\mathsf s'}$, $\hat \omega_{\mathsf s, \mathsf s'}$ is weakly  contained
in the regular representation.
\end{lem}
\begin{proof}
This has been known to experts (see   \cite[Theorem 3.2]{Li89}).  Lemmas \ref{boundpsi}, \ref{int} and \ref{matrico} implies that for a dense subspace of $\hat \omega_{\mathsf s, \mathsf s'}|_{G_{\mathsf s'}}$, the diagonal  matrix coefficients are almost square integrable.
 Thus the lemma follows form \cite[Theorem 1]{CHH}.
\end{proof}


However, if $\star'\in\{B, C^*, D^*\}$, $\abs{\mathsf s}\neq\nu_{\mathsf s'}$ for the parity reason. For this reason, we introduce
\[
   \nu_{\mathsf s'}^\circ:=\begin{cases}
    \nu_{\mathsf s'},\quad &\textrm{if }\star'\in \{C, D, \widetilde C\};\\
     \nu_{\mathsf s'}+1,\quad &\textrm{if }\star'\in \{B, C^*, D^*\}.\\
  \end{cases}
\]

\delete{Set
\[
  \nu_\mathsf s^\circ:=\begin{cases}
    \abs{\mathsf s},\quad &\textrm{if }\star\in \{C, \widetilde C, C^*\};\\
     \abs{\mathsf s}-1,\quad & \textrm{if }\star= B;\\
      \abs{\mathsf s}-2,\quad & \textrm{if }\star\in \{D,D^*\}\\
  \end{cases}
\]
Then $\nu_\mathsf s^\circ=\nu_\mathsf s$ when $\star\in \{C,D, \widetilde C\}$, and $\nu_\mathsf s^\circ=\nu_\mathsf s+1$ when $\star\in \{B,C^*, D^*\}$.
}

The following definition is a slight variation of definition \ref{defn:CRcov}.
 \begin{defn}\label{defn:CR33}
A Casselman-Wallach representation  of $G_{\mathsf s'}$ is overconvergent  for $\check \Theta_{\mathsf s'}^{\mathsf s}$ if
 it is  $\nu$-bounded  for some $\nu>\nu_{\mathsf s'}^\circ -\abs{\mathsf s}$.
\end{defn}

We will prove the  following unitarity result in  the rest of this subsection.
\begin{thm}\label{positivity000}
Assume that $\abs{\mathsf s}\geq \nu^\circ_{\mathsf s'}$. Let $\pi'$ be a Casselman-Wallach representation of $G_{\mathsf s'}$ that is overconvergent  for $\check \Theta_{\mathsf s'}^{\mathsf s}$. If $\pi'$ is unitarizable, then so is $\Thetab_{\mathbf s'}^{\mathbf s}(\pi')$.
\end{thm}



Recall the following positivity result of matrix coefficient integrals, which is a special case of \cite[Theorem A. 5]{HLS}.

\begin{lem}\label{positivity}
Let $G$ be a real reductive group with a maximal compact subgroup $K$. Let $\pi_1$ and $\pi_2$ be two unitary representations of $G$ such that $\pi_2$ is weakly
contained in the regular representation. Let $u_1, u_2, \cdots, u_r$ ($r\in \bN$) be vectors in $\pi_1$ such that for all $i,j=1,2, \cdots, r$,
the integral
\[
  \int_G \la g\cdot u_i, u_j\ra\,\Xi_G (g) \od\!g %\quad (\textrm{$\od\! g$  is a Haar measure  on $G$})
\]
is absolutely convergent, where  $\Xi_G$ is the bi-$K$-invariant Harish-Chandra's $\Xi$ function on $G$.   Let $v_1,v_2,\cdots, v_r$ be  $K$-finite vectors in $\pi_2$.
Put
\[
u:=\sum_{i=1}^r u_i\otimes v_i\in \pi_1\otimes \pi_2.
\]
Then the integral
\[%\begin{equation}\label{geq0}
\int_G \la g \cdot u,u \rangle\,\od\! g
\]%\end{equation}
absolutely converges to a nonnegative real number.
\end{lem}


Now we come to the proof of Theorem \ref{positivity000}.

%We also need the following lemma.

\delete{

\begin{lem}\label{intxi2}
For every real number $\nu>\nu_\mathsf s$, the function $\Psi_\mathsf s^\nu\cdot \Xi_{G_\mathsf s}$ is integrable with respect to a Haar measure on $G_\mathsf s$. Here $\Xi_{G_\mathsf s}$ is the bi-$K_\mathsf s$-invariant Harish-Chandra's $\Xi$ function on $G_\mathsf s$.
\end{lem}

\begin{proof}
This follows easily from the integral formula for $G_\mathsf s$ under the Cartan decomposition (\cite[Lemma 2.4.2]{Wa1}), as well as the estimate of Harish-Chandra's $\Xi$ function (\cite[Theorem 4.5.3]{Wa1}).
\end{proof}}

\begin{proof}[Proof of Theorem \ref{positivity000}]
Fix an invariant continuous Hermitian inner product on $\pi'$, and  write $\hat \pi'$ for the completion of $\pi'$ with respect to this Hermitian inner product.
The space $\pi' \widehat \otimes \omega_{\mathsf s, \mathsf s'}$ is equipped with the  inner product $\la\,,\,\ra$  that is the tensor product of the ones on $\pi'$ and $\omega_{\mathsf s, \mathsf s'} $. It suffices to show that
\[
  \int_{G_{\mathsf s'}}\la g \cdot u,u\ra\od\! g\geq 0
\]
for all $u$ in a dense subspace of $\pi' \widehat \otimes \omega_{\mathsf s, \mathsf s'}$.

If $\nu^\circ_{\mathsf s'}<0$, then $\star\in \{D, D^*\}$ and $\abs{\mathsf s'}=0$. The theorem is trivial in this case.  Thus we assume that $\nu^\circ_{\mathsf s'}\geq 0$.
We also assume that $\star=B$. The proof in the other cases is similar and is omitted.

Note that $\nu^\circ_{\mathsf s'}=\nu_{\mathsf s'}=\abs{\mathsf s'}$ is even.
 Let $\mathsf s_1=(B, p_1, q_1)$ and  $\mathsf s_2=(D, p_2, q_2)$ be two classical signatures such that
\[
  (p_1, q_1)+(p_2, q_2)=(p,q)\quad \textrm{and}\quad \abs{\mathsf s_2}=\nu^\circ_{\mathsf s'}.
\]
View $\hat \omega_{\mathsf s_2, \mathsf s'}$ as a unitary representation of the symplectic group $G_{(C, p',q')}$.
Define $\pi_2$ to be its  pull-back through the covering homomorphism $G_{\mathsf s'}\rightarrow G_{(C, p',q')}$. By Lemma \ref{weaklycont}, the representation $\pi_2$ of $G_{\mathsf s'}$ is weakly contained in the regular representation.

Put
\[
  \pi_1:=\hat \pi'\widehat \otimes_{\mathrm h} (\hat \omega_{\mathsf s_1, \mathsf s'}|_{G_{\mathsf s'}})\qquad (\textrm{$\widehat \otimes_{\mathrm h}$ indicates the Hilbert space tensor product}).
\]
 Lemmas  \ref{int} and \ref{matrico} imply that the integral
\[
 \int_{G_{\mathsf s'}} \la g\cdot u,v\ra \cdot \Xi_{G_{\mathsf s'}}(g)\od\!g
\]
is absolutely convergent for all $u,v\in \pi' \otimes \omega_{\mathsf s_1, \mathsf s'}$. The theorem then follows by Lemma \ref{positivity}.



\end{proof}



\section{Double theta lifts and degenerate principal series}

\def\GLE{\GL(\bfE)^{J_{\bfU}}}
\def\GLEz{\GL_{\bfE_0}}
\def\GLE{{\GL_{\bfE}}}
\def\wtGLE{\widetilde{\GLE}}
\def\wtGLEz{\widetilde{\GLEz}}
\def\wtPE{\widetilde{P_\bfE}}
\def\JU{{J_{\bfU}}}
\def\LU{{L_{\bfU}}}
\def\wtGU{\widetilde{G}_\bfU}
\def\dsfss{{\dot{\mathsf s}}}


In this section, we relate double theta lifts with degenerate principal series representations.

Let
\[
\dot{ \mathsf s}=(\dot \star, \dot p,\dot q),\qquad  \mathsf s'=(\star', p',q')\qquad \textrm{and}\qquad \sfss''=(\star'', p'', q'')
\]
be classical signatures such that
\begin{itemize}
  \item $ (q', p')+(p'', q'')=(\dot p, \dot q)$;\smallskip
  \item if $\dot \star\in \{B,D\}$,  then $\star', \star''\in \{B,D\}$; \smallskip
  \item if $\dot \star\notin \{B,D\}$,  then $\star'=\star''=\dot \star$.
\end{itemize}
%Recall the classical signature $\sfss'$ from \eqref{sfsspm}.
As in Section \ref{secoscil}, put $\sfss'^-:=(\star', q', p')$. We view $V_{\sfss'^-}$ and $V_{\sfss''}$ as subspaces of $V_\dsfss$ such that $(\la\,,\,\ra_{\dsfss}, J_{\dsfss},L_{\dsfss})$  extends both $(\la\,,\,\ra_{\sfss'^-}, J_{\sfss'^-}, L_{\sfss'^-})$ and
$(\la\,,\,\ra_{\mathsf s''}, J_{\mathsf s''}, L_{\mathsf s''})$, and $V_{\sfss'^-}$ and $V_{\sfss''}$ are perpendicular to each other under the form $\la\,,\,\ra_{\dsfss}$. Then we have an orthogonal decomposition
\[
  V_\dsfss=V_{\sfss'^-}\oplus V_{\sfss''},
\]
and both $G_{\sfss'^-}$ and $G_{\sfss''}$ are identified with subgroups of $G_{\dsfss}$.

 Fix a linear isomorphism
\be\label{iota1}
 \iota_{\sfss'}:  V_{\sfss'}\rightarrow V_{\sfss'^-}%\quad v\mapsto v^-
\ee
as in \eqref{iota0}, which  induces an isomorphism
\be\label{isosfss1}
 G_{\sfss'}\rightarrow G_{\sfss'^-}, \quad g\mapsto g^-.
\ee
Let $\pi'$ be a Casselman-Wallach representation of $G_{\sfss'}$. Write $\pi'^-$ for the representation of $G_{\sfss'^-}$ that corresponds to $\pi'$ under the isomorphism \eqref{isosfss1}.



\subsection{Matrix coefficient integrals and double theta lifts}

We begin with the following lemma.

\begin{lem}\label{boundxx}
The function $(\Xi_\dsfss)|_{G_{\sfss'}}$ on $G_{\sfss'}$  is   $\abs{\sfss''}$-bounded.
\end{lem}
\begin{proof}
This follows from the  estimate of Harish-Chandra's $\Xi$ function (\cite[Theorem 4.5.3]{Wa1}).
\end{proof}


 \begin{defn}\label{defn:CR33}
The Casselman-Wallach representation  $\pi'^-$ of $G_{\mathsf s'^-}$ is convergent  for a Casselman-Wallach representation  $\dot \pi$ of $G_{\dsfss}$ if there are real numbers $\nu'$ and $\dot \nu$ such that $\pi'$ is $\nu'$-bounded, $\dot \pi$ is $\dot \nu$-bounded, and
\[
  \nu'+\dot \nu>-\abs{\sfss''}.
\]
\end{defn}

Let $\dot \pi$  be a Casselman-Wallach representation of $G_{\dsfss}$ such that $\pi'^-$  is convergent  for  $\dot \pi$.
Consider the integrals
\be\label{convint0004}
\begin{array}{rcl}
 (\pi'^- \times \dot \pi)\times ((\pi'^-)^\vee \times \dot \pi^\vee )&\rightarrow &\C, \smallskip \\
   ((u,v),(u',v')) &\mapsto &\int_{G_{\mathsf s'}} \la g^-\cdot u, u'\ra\cdot \la g^-\cdot v, v'\ra \od\! g.
   \end{array}
 \ee

\begin{lem}\label{intpi0004}
The integrals in \eqref{convint0004} are absolutely convergent and the map \eqref{convint0004} is   continuous and multi-linear.
\end{lem}
\begin{proof}
Note that
$
  (\Psi_{\dsfss})|_{G_{\sfss'}}=\Psi_{\sfss'}.
$
Thus the lemma follows from Lemmas \ref{int} and \ref{boundxx}.
\end{proof}



By Lemma \ref{intpi0004}, the integrals in \eqref{convint0004} yield a continuous bilinear form
\be\label{convint0011}
 \la\,,\,\ra: (\pi'^- \widehat \otimes{ \dot \pi})\times ((\pi'^-)^\vee \widehat \otimes {\dot \pi}^\vee )\rightarrow \C.
 \ee
Put
\begin{equation}\label{thetab0}
 \pi'^-*\dot \pi:=\frac{\pi'^-\widehat \otimes \dot \pi}{\textrm{the left kernel of \eqref{convint0011}}}.
\end{equation}
This is a smooth Fr\'echet representation of $G_{\sfss''}$ of moderate growth.



%\subsection{Double theta lifts}


For every  classical signature $\sfss=(\star, p,q)$, put
\[
  [\sfss]:=\begin{cases}
    (C, p,q), \quad & \textrm{if $\star=\tilde C$};\\
    \sfss, \quad & \textrm{if $\star\neq \tilde C$}.
  \end{cases}
\]

\begin{prop}\label{doublelift}
Suppose that $\sfss=(\star, p,q)$ is a classical signature such that both  $\star'$ and $\star''$ equals the Howe dual of $\star$,  and $2\nu_\sfss<\abs{\dsfss}$.
Assume that $\pi'$  is $\nu'$-bounded for some $\nu'>\nu_{\sfss'}-\abs{\sfss}$.
Then
\begin{itemize}
\item $\pi'$ is convergent for $\check \Theta^{\sfss}_{\sfss'}$;
 \item $\Thetab^{\sfss}_{\sfss'}(\pi')$  is convergent for $\check \Theta^{\sfss''}_{\sfss}$;
 \item
  the trivial representation
$1$ of $G_{[\sfss]}$ is convergent for $\check \Theta^{\dsfss}_{\bar \sfss}$;
\item
  $\pi'^-$ is convergent for  $\Thetab_{[\sfss]}^\dsfss(1)$;
 \item as representations of $G_{\sfss''}$, 
\begin{equation}
\label{thetabv00}
  \Thetab^{\sfss''}_{\sfss}(\Thetab^{\sfss}_{\sfss'}(\pi'))\cong \pi'^-* \Thetab_{[\sfss]}^\dsfss(1).
\end{equation}
\end{itemize}
\end{prop}
\begin{proof}
The first four claims in the proposition are obvious. We only need to prove the last one.
Note that the integrals in
\begin{equation}\label{intt00}
\begin{array}{rcl}
   (\pi'\totimes \omega_{\sfss,\sfss'}\totimes\omega_{\sfss'', \sfss })\times
    ((\pi')^\vee \totimes\omega_{\sfss,\sfss'}^\vee\totimes \omega_{\sfss'', \sfss }^\vee)&\rightarrow &\C,\\
    (u,v)&\mapsto &\int_{G_{\sfss'}\times G_\sfss} \inn{(g',g)\cdot u}{v}\od\! g'\, \od\! g
    \end{array}\end{equation}
are absolutely convergent and defines a continuous bilinear map. Also note that
\[
  \omega_{\sfss,\sfss'}\totimes\omega_{\sfss_2, \sfss }\cong \omega_{\dsfss,\sfss}.
\]
In view of Fubini's theorem, the lemma follows as both sides of \eqref{thetabv00} are isomorphic to the quotient of $\pi'\totimes \omega_{\sfss,\sfss'}\totimes\omega_{\sfss'', \sfss }$ by the left kernel of the pairing \eqref{intt00}.

\end{proof}

\subsection{Matrix coefficient integrals against degenerate principal series}
\label{sec:DP}

We are particularly interested in the case when $\dot \pi$ is a degenerate principal series representation.
Suppose that $\dsfss$ is split,  and
\[
  p'\leq p''\qquad\textrm{and}\qquad q'\leq q''.
\]
Then $\star'=\star''$, and there is a  split classical signature $\sfss_0=(\star_0, p_0, q_0)$ such that
\begin{itemize}
\item $\star_0=\dot \star$;
  \item $(p', q')+(p_0, q_0)=(p'', q'')$.
\end{itemize}
We view $V_{\sfss'}$  and $V_{\sfss_0}$ as subspaces of $V_{\sfss''}$ such that $(\la\,,\,\ra_{\sfss''}, J_{\sfss''},L_{\sfss''})$  extends  both $(\la\,,\,\ra_{\mathsf s'}, J_{\mathsf s'}, L_{\mathsf s'})$ and $(\la\,,\,\ra_{\sfss_0}, J_{\sfss_0},L_{\sfss_0})$, and  $V_{\sfss'}$  and $V_{\sfss_0}$ are perpendicular to each other under the form $\la\, ,\,\ra_{\sfss_2}$.


Put
\[
   V_{\sfss'}^\triangle:=\{\iota_{\sfss'}(v)+v\in V_\dsfss \mid v\in V_{\sfss'}\}\quad\textrm{and}\quad V_{\sfss'}^\nabla:=\{\iota_{\sfss'}(v)-v\in V_\dsfss \mid v\in V_{\sfss'}\}.
 \]
As before, we have that
\[
  V_{\sfss_0}=X_{\sfss_0}\oplus Y_{\sfss_0},
\]
where $X_{\sfss_0}$ is a maximal $J_{\sfss_0}$-stable totally isotropic subspace of $V_{\sfss_0}$, and $Y_{\sfss_0}:=L_{\sfss_0}(X_{\sfss_0})$.
Suppose that
 \[
 X_{\dsfss}= V_{\sfss'}^\triangle \oplus X_{\sfss_0}\quad\textrm{and}\quad Y_{\dsfss}=V_{\sfss'}^\nabla \oplus Y_{\sfss_0}.
 \]
%They are  maximal $J_{\dsfss}$-stable totally isotropic subspaces of $V_{\dsfss}$ such that $Y_{\dsfss}=L_\dsfss(X_\dsfss)$.
In summary, we have decompositions
\be\label{decomv0}
  V_{\dsfss}=V_{\sfss'^-}\oplus V_{\sfss''}= V_{\sfss'^-}\oplus V_{\sfss'}\oplus  V_{\sfss_0}= (V_{\sfss'}^\triangle \oplus X_{\sfss_0})\oplus (V_{\sfss'}^\nabla \oplus Y_{\sfss_0}).
\ee

As before, $X_\dsfss$ and $X_{\sfss_0}$ yield
 the Siegel parabolic subgroups
 \[
  P_{\dsfss}=R_{\dsfss}\ltimes N_\dsfss\subset G_\dsfss\qquad \textrm{and}\qquad P_{\sfss_0}=R_{\sfss_0}\ltimes N_{\sfss_0}\subset G_{\sfss_0}.
 \]
 Write
\be\label{parabolic} 
  P_{\sfss'',\sfss_0}=R_{\sfss'',\sfss_0}\ltimes N_{\sfss'',\sfss_0}
\ee
for the parabolic subgroup of $G_{\sfss''}$ stabilizing $X_{\sfss_0}$, where $R_{\sfss'',\sfss_0}$ is the Levi subgroup stabilizing both $X_{\sfss_0}$ and $Y_{\sfss_0}$, and $N_{\sfss'',\sfss_0}$ is the unipotent radical. We have an obvious homomorphism
\be  \label{parabolic2}
  G_{\sfss'}\times R_{\sfss_0}\rightarrow R_{\sfss'',\sfss_0},
\ee
which is a two fold covering map when $\dot \star=\widetilde C$, and an isomorphism in the other cases.
For every $g\in G_{\sfss'}$, write $g^\triangle:= g^- g$, which is an element of $R_{\dsfss}$.

Let $\dot \chi: R_{\dsfss}\rightarrow \C^\times $ be a character. It yields a degenerate principal series representation
\[
I(\dot \chi):=\Ind_{P_\dsfss}^{G_\dsfss} \dot \chi
\]
 of $G_\dsfss$.
Write
\be\label{chi0}
  \chi_0:=\dot \chi|_{R_{\sfss_0}},
\ee
and define a character
\be\label{chip}
 \chi':  G_{\sfss'}\rightarrow \C^\times, \quad g\mapsto \dot \chi(g^\triangle) \qquad(\textrm{this is a quadratic character}).
\ee

Recall the representations  $\pi'$  of $G_{\sfss'}$ and
$\pi'^-$ of $G_{\sfss'^-}$. If $\dsfss=\widetilde C$, we assume that both $\pi'$ and $\dot \chi$ are genuine.  Then $(\pi'\otimes \chi') \otimes \chi_0$ descends to a Casselman-Wallach representation of $R_{\sfss'',\sfss_0}$. View it as a representation of $P_{\sfss'',\sfss_0}$ via the trivial action of $N_{\sfss'',\sfss_0}$, and form the representation  $\Ind_{P_{\sfss'',\sfss_0}}^{G_{\sfss''}} ((\pi_1\otimes \chi') \otimes \chi_0)$


Let $\nu_{\dot \chi}\in \R$ be  as in \eqref{nuchi}. The rest of this subsection is devoted to a proof of the following theorem.

\begin{thm}\label{midp}
Assume that the Casselman-Wallach representation $\pi'$ of $G_{\sfss'}$ is $\nu'$-bounded for some
\[
 \nu'>\begin{cases}
  2\abs{\nu_{\dot \chi}}- \frac{\abs{\sfss_0}}{2}-1,\quad &\textrm{if $\dot \star\in \{B,C,D, \widetilde C\}$};\\
   2\abs{\nu_{\dot \chi}}- \frac{\abs{\sfss_0}}{2}-2,\quad & \textrm{if $\dot \star\in \{C^*,D^*\}$}.
 \end{cases}
\]
 Then $\pi'^-$ is convergent for $I_{\dsfss}(\dot \chi)$, and
\[
 \pi'^-*I(\dot \chi)\cong \Ind_{P_{\sfss'',\sfss_0}}^{G_{\sfss''}} ((\pi'\otimes \chi')\otimes \chi_0).
\]
\end{thm}

Let the notation and assumptions be as in Theorem \ref{midp}.
Recall that the representations   $I(\dot \chi)$ and $ I(\dot \chi^{-1})$ are contragredients of each other with the $G_\dsfss$-invariant pairing
 \[
  \la\,,\,\ra:  I(\dot \chi)\times I(\dot \chi^{-1})\rightarrow \C, \quad (f, f')\mapsto \int_{K_\dsfss} f(g) \cdot f'(g) \od\!g,
 \]

The first assertion of Theorem \ref{midp} is a direct consequence of Lemma \ref{growthdp}. As in \eqref{convint0011},  we have a continuous bilinear form
\be\label{convint00112}
\begin{array}{rcl}
(\pi'^- \widehat \otimes{ I(\dot \chi)})\times ((\pi'^-)^\vee \widehat \otimes I(\dot \chi^{-1}) )&\rightarrow &\C,\\
 (u,v)&\mapsto& \int_{G_{\sfss'}} \la g^-\cdot u,v\ra \od\! g
 \end{array}
 \ee
so that
\begin{equation}\label{thetab0}
 \pi'^-*I(\dot \chi):=\frac{\pi'^-\widehat \otimes I(\dot \chi)}{\textrm{the left kernel of \eqref{convint00112}}}.
\end{equation}
%This is a smooth Fr\'echet representation of $G_{\sfss_2}$ of moderate growth.

Note that
\[
\G_\dsfss^\circ:= P_{\dsfss}\cdot G_{\sfss''}
\] is open and dense in $G_\dsfss$, and its complement has measure zero in $G_\dsfss$. Moreover,
\be\label{opencell}
P_{\dsfss}\backslash \G_\dsfss^\circ= (R_{\sfss_0}\ltimes N_{\sfss'', \sfss_0})\backslash G_{\sfss''}.
\ee
Form the normalized Schwartz induction
\[
   I^\circ(\dot \chi):=\ind_{R_{\sfss_0}\ltimes N_{\sfss'', \sfss_0}}^{G_{\sfss''}} \chi_0.
\]
The reader is referred to \cite[Section 6.2]{CS21} for the general notion of Schwartz inductions (in a slightly different unnormalized setting). Similarly, put
\[
   I^\circ(\dot \chi^{-1}):=\ind_{R_{\sfss_0}\ltimes N_{\sfss'', \sfss_0}}^{G_{\sfss''}}\chi_0^{-1}.
\]
Note that
\be\label{modulus}
\textrm{the modulus character of $P_\dsfss$ restricts to the modulus character of  $R_{\sfss_0}\ltimes N_{\sfss'', \sfss_0}$.}
\ee
In view of \eqref{opencell} and \eqref{modulus}, by extension by zero, $I^\circ(\dot \chi)$ is viewed as a closed subspace of $ I(\dot \chi)$, and $ I^\circ(\dot \chi^{-1})$  is viewed as a closed subspace of $ I(\dot \chi^{-1})$. These two closed subspaces are  $(G_{\sfss'}\times G_{\sfss''})$-stable.


Similar to \eqref{convint00112}, we have a continuous bilinear form
\be\label{convint001123}
\begin{array}{rcl}
  (\pi'^- \widehat \otimes{ I^\circ (\dot \chi)})\times ((\pi'^-)^\vee \widehat \otimes I^\circ (\dot \chi^{-1}) )&\rightarrow &\C,\\
 (u,v)&\mapsto& \int_{G_{\sfss'}} \la g^-\cdot u,v\ra \od\! g.
 \end{array}
 \ee
Put
\begin{equation}\label{thetab0}
 \pi'^-*I^\circ (\dot \chi):=\frac{\pi'^- \widehat \otimes I^\circ (\dot \chi)}{\textrm{the left kernel of \eqref{convint001123}}},
\end{equation}
which is still a smooth Fr\'echet representation of $G_{\sfss''}$ of moderate growth.

\begin{lem}\label{isoipi}
 As representations of $G_{\sfss''}$,
 \[
   \pi'^-*I^\circ (\dot \chi)\cong \Ind_{P_{\sfss'',\sfss_0}}^{G_{\sfss''}} ((\pi'\otimes  \chi')\otimes \chi_0).
   \]
\end{lem}
\begin{proof}
As Fr\'echet spaces, $\pi'^-$ is obviously identified with $\pi'$.
Note that the natural map $G_{\sfss'}\rightarrow (R_{\sfss_0}\ltimes N_{\sfss'', \sfss_0})\backslash G_{\sfss''}$ is proper and hence the following integrals are absolutely convergent and yield a $G_{\sfss''}$-equivariant continuous linear map:
\[
\begin{array}{rcl}
\xi:   \pi'^-\widehat \otimes I^\circ (\dot \chi)&\rightarrow & \Ind_{P_{\sfss'',\sfss_0}}^{G_{\sfss''}} ((\pi'\otimes \chi')\otimes \chi_0),\\
   v\otimes f &\mapsto  & \left(h\mapsto\int_{G_{\sfss'}}\chi'(g)\cdot  f(g^{-1}h)(g\cdot v)\od\! g\right).
   \end{array}
\]
Moreover, this map is surjective (\cf \cite[Section 6.2]{CS21}). It is thus open by the open mapping theorem. Similarly, we have a open surjective
$G_{\sfss''}$-equivariant  continuous linear map
\[
\begin{array}{rcl}
\xi':   (\pi'^-)^\vee\widehat \otimes I^\circ (\dot \chi^{-1})&\rightarrow & \Ind_{P_{\sfss'',\sfss_0}}^{G_{\sfss''}} ((\pi'^\vee\otimes \chi')\otimes \chi_0^{-1}),\\
   v'\otimes f' &\mapsto  & \left(h\mapsto\int_{G_{\sfss'}} \chi'(g)\cdot  f'(g^{-1}h)(g\cdot v')\od\! g\right).
   \end{array}
\]

For all $v\otimes f \in \pi'^-\widehat \otimes I^\circ (\dot \chi)$ and $v'\otimes f' \in (\pi'^-)^\vee \widehat \otimes I^\circ (\dot \chi^{-1})$, we have that
\begin{eqnarray*}
 && \int_{G_{\sfss'}}\la g^-\cdot (v\otimes f), v'\otimes f'\ra \od\! g\\
 &=& \int_{G_{\sfss'}}\la g^-\cdot v, v'\ra \cdot \la g^-\cdot f, f'\ra \od\! g\\
 &=& \int_{G_{\sfss'}}\la g^-\cdot v, v'\ra \cdot \int_{K_{\sfss''}} \int_{G_{\sfss'}}   (g^-\cdot f)(hx) \cdot  f'(hx)  \od\! h \od \! x \od \! g\\
 &=& \int_{G_{\sfss'}}\la g^-\cdot v, v'\ra \cdot \int_{K_{\sfss''}} \int_{G_{\sfss'}}  \chi'(g) \cdot f(g^{-1}h x) \cdot  f'(hx)  \od\! h \od \! x \od \! g\\
  &=& \int_{K_{\sfss''}}  \int_{G_{\sfss'}}  \int_{G_{\sfss'}} \la (h^- (g^-)^{-1})\cdot v, v'\ra \cdot \chi'(hg^{-1}) \cdot f(gx) \cdot  f'(hx)  \od\! g \od \! h \od \! x \\
 &=&  \int_{K_{\sfss''}}\la  \xi(v\otimes f)(x),\xi'(v'\otimes f')(x)\ra  \od \! x\\
 &=& \la  \xi(v\otimes f),\xi'(v'\otimes f')\ra.
\end{eqnarray*}
This implies the lemma.
\end{proof}



\begin{lem}\label{imb}
Let $u\in \pi'^- \widehat \otimes{ I(\dot \chi)}$. Assume that
\begin{equation}\label{intguv0}
  \int_{G_{\sfss'}} \la g^-\cdot u, v\ra\od\! g =0
\end{equation}
for all $v\in (\pi'^-)^\vee \widehat \otimes I^\circ (\dot \chi^{-1})$. Then \eqref{intguv0} also holds for all $v\in (\pi'^-)^\vee \widehat \otimes I(\dot \chi^{-1})$.
\end{lem}

\begin{proof}
Take a sequence $(\eta_1, \eta_2, \eta_3, \cdots)$ of real valued smooth functions on $P_{\dsfss}\backslash G_{\dsfss}$ such that
\begin{itemize}
\item
for all $i\geq 1$,  the support of $\eta_i$ is contained in $P_\dsfss \backslash  G_\dsfss^\circ$;
\item for all $i\geq 1$ and  $x\in  P_{\dsfss}\backslash G_{\dsfss}$, $ 0\leq  \eta_i(x)\leq \eta_{i+1}(x)\leq 1$; \smallskip
 \item
 $\bigcup_{i=1}^\infty \eta_i^{-1}(1)= P_{\dsfss}\backslash G_{\dsfss}^\circ$.
\end{itemize}
Let $v\in  (\pi'^-)^\vee \widehat \otimes I^\circ (\dot \chi^{-1})$. Note that $\eta_i  I (\dot \chi^{-1})\subset  I^\circ (\dot \chi^{-1})$. Thus $\eta_i v\in (\pi'^-)^\vee \widehat \otimes  I^\circ (\dot \chi^{-1})$.

Lemma \ref{growthdp} and  Lebesgue's dominated convergence theorem imply that
\[
  \int_{G_{\sfss'}} \la g^-\cdot u, v\ra\od\! g
  =\lim_{i\rightarrow +\infty}  \int_{G_{\sfss'}} \la g^-\cdot u, \eta_i v\ra \od\! g =0.
\]
This proves the lemma.

\end{proof}

Lemma \ref{imb} implies that we have a natural continuous  linear map
\[
 \Ind_{P_{\sfss'',\sfss_0}}^{G_{\sfss''}} ((\pi'\otimes  \chi')\otimes \chi_0)= \pi'^-* I^\circ (\dot \chi)\rightarrow \pi'^-* I(\dot \chi).
\]
This map is clearly  injective and $G_{\sfss''}$-equivariant. Similarly to Lemma \ref{imb}, we know that the natural pairing
\[
 ( \pi'^-* I(\dot \chi))\times( (\pi'^-)^\vee* I^\circ(\dot \chi^{-1}))\rightarrow \C
\]
is well-defined and non-degenerate. Thus Theorem \ref{midp} follows by the following lemma.

%ave a natural  $G_{\sfss_2}$-equivariant injective continuous  linear map
%\[
 %\Ind_{P_{\sfss_2,\sfss_0}}^{G_{\sfss_2}} (((\pi_1^-)^\vee\otimes  \chi_1^-)\otimes \chi_0^{-1})= (\pi_1)^\vee * I^\circ (\dot \chi^{-1})\rightarrow (\pi_1)^\vee* I(\dot \chi^{-1}).
%\]


\begin{lem}\label{imb2}
Let $G$ be a real reductive group. Let $\pi$ be a Casselman-Wallach representation of $G$, and let $\tilde \pi$ be a smooth Fr\'echet representation of $G$ with a $G$-equivariant injective continuous linear map
\[
  \phi: \pi\rightarrow \tilde \pi.
\]
Assume that  there is a non-degenerate $G$-invariant continuous bilinear map
\[
\la\,,\,\ra: \tilde \pi\times \pi^\vee\rightarrow \C,
\]
such that the composition of
\[
 \pi\times \pi^\vee\xrightarrow{(u,v)\mapsto (\phi(u),v)} \tilde \pi\times \pi^\vee\xrightarrow{\la\,,\,\ra} \C
\]
is the natural pairing. Then $\phi$ is a topological isomorphism.
\end{lem}
\begin{proof}
 Let $u\in \tilde \pi$.
 Using the theorem of Dixmier-Malliavin \cite[Theorem 3.3]{DM}, we write
 \[
   u=\sum_{i=1}^{s} \int_G \varphi_i(g) ( g\cdot u_i)\od\! g \qquad (s\in \bN, \ u_i\in\tilde \pi),
 \]
 where $\varphi_i$'s are compactly supported smooth functions on $G$.  As a continuous linear functional on $\pi^\vee$, we have that
  \[
   \la u,\, \cdot\, \ra=\sum_{i=1}^{s} \int_G \varphi_i(g)  \cdot \la g\cdot u_i, \,\cdot\,\ra \od\! g.
 \]
By \cite[Lemma 3.5]{SZ1},  the right-hand side  functional  equals  $\la u_0, \,\cdot\,\ra$ for a unique $u_0\in \pi$. Thus $u=u_0$, and the lemma follows by the open mapping theorem.
\end{proof}



\subsection{Double theta lifts and parabolic induction}\label{doublep}
Let $\star$ be the Howe dual of $\star'=\star''$. Let $k\in \bN$ and we assume that the set
\[
 S_{\star, k}:=\{ \textrm{$\sfss_1$ is a classical signature}\mid \textrm{$\sfss_1$ has type $\star$, and $\abs{\sfss_1}=k$}\}
\]
is non-empty. Thus $k$ is odd if $\star=B$, and $k$ is even in all other cases. 
In this subsection, we further assume that $k\geq 2$ and
\be\label{dotp}
 \dot p=\dot q= \begin{cases}
 k-1,\quad   &\textrm{if } \star\in \{B,D\};\\
k+1,  & \textrm{if $\star\in \{C, \widetilde C\}$;}\\
k & \textrm{if $ \star\in\{C^*,D^*\}$.}
\end{cases}
 \ee
%For technical   reason, we also assume that $k\geq 2$ when $\star\in \{B,D\}$. 

We also assume that the character $\dot \chi: R_{\dsfss}\rightarrow \C^\times $
satisfies  the following conditions:
\be\label{chid}
\begin{cases}
 \dot \chi \textrm{ is genuine and $\dot \chi^4=1$,}& \textrm{if $ \star=B$};\\
 \dot \chi^2=1,  & \textrm{if $\star=D$;}\\
 \dot \chi=1,\quad   &\textrm{if } \star\in \{C,\widetilde C\};\\
    \dot \chi^2\textrm{ equals the composition of $R_\dsfss\xrightarrow{\textrm{natural map}} \GL(X_\dsfss)\xrightarrow{\det}\C^\times$},  & \textrm{if $ \star=C^*$;}\\
 \dot \chi^2\textrm{ equals the composition of $R_\dsfss\xrightarrow{\textrm{natural map}} \GL(Y_\dsfss)\xrightarrow{\det}\C^\times$},  & \textrm{if $ \star=D^*$.}
\end{cases}
\ee
If  $\star\notin \{B,D\}$, the character   $\dot{\chi}$ is uniquely determined by \eqref{chid}.
If  $\dot \star\in \{B,D\}$, there are two characters satisfying \eqref{chid}, and we let $\dot \chi'$ denote the one other than $\dot \chi$.


The relationship between the degenerate principle series representation $I(\dot{\chi})$ and
Rallis quotients is summarized in the following lemma.
\begin{lem}\label{degens}%The following statements hold true.
\noindent
(a) If $\star\in \{B,D\}$,  then
\[
  I(\dot{\chi}) \oplus I(\dot \chi')\cong \bigoplus_{\sfss_1\in S_{\star, k}}
   \check \Theta_{\sfss_1}^{\dsfss}(1).
\]

\noindent
(b) If $\star\in \{C,\widetilde C\}$, then
\[
  I(\dot{\chi})\cong \check \Theta_{\sfss_1}^{\dsfss}(1)\oplus  (\check \Theta_{\sfss_1}^{\dsfss}(1)\otimes \det),\qquad \textrm{where } \ \mathsf s_1=(C, \frac{k}{2}, \frac{k}{2}).
\]


\noindent
(c) If $\star=D^*$, then
\[
  I(\dot{\chi})\cong \check \Theta_{\sfss_1}^{\dsfss}(1),\qquad \textrm{where }\ \mathsf s_1=(D^*,  \frac{k}{2}, \frac{k}{2}).
\]


\noindent
(d) If $\star=C^*$, then there is an exact sequence of representations of $G_{\dsfss}$:
\[
0\rightarrow \bigoplus_{\mathsf s_1\in S_{\star, k}} \check \Theta_{\sfss_1}^{\dsfss}(1) \rightarrow
 I(\dot{\chi})\rightarrow  \bigoplus_{\mathsf s_2\in S_{\star, k-2}} \check \Theta_{\sfss_2}^{\dsfss}(1) \rightarrow 0.
\]

\noindent
(e) All the representations $ \check \Theta_{\sfss_1}^{\dsfss}(1) $and $ \check \Theta_{\sfss_2}^{\dsfss}(1)$ appearing in (a), (b), (c) and (d) are irreducible and unitarizable.


\end{lem}
\begin{proof}
 See \cite[Theorem 2.4]{Ku}, \cite[Introduction]{LZ1}, \cite[Theorem 6.1]{LZ2} and  \cite[Sections 9 and 10]{Ya}.
\end{proof}


\begin{lem}\label{lem:coinv}
For all $\mathsf s_1$ appearing in Lemma \ref{degens}, the trivial representation $1$ of $G_{\mathsf s_1}$ is overconvergent for $\check \Theta_{\mathsf s_1}^{\dsfss}$, and
\[
   \Thetab_{\mathsf s_1}^{\dsfss}(1)=\check \Theta_{\mathsf s_1}^{\dsfss}(1).
\]

\end{lem}
\begin{proof}
Note that the trivial representation $1$ of $G_{\mathsf s_1}$ is $(-\nu_{\mathsf s_1})$-bounded, and $-\nu_{\mathsf s_1}>\nu_{\mathsf s_1}^\circ -\abs{\dsfss}$. This implies the first assertion.
Since $\check \Theta_{\mathsf s_1}^{\dsfss}(1)$ is irreducible and $\Thetab_{\mathsf s_1}^{\dsfss}(1)$ is  a quotient of $\check \Theta_{\mathsf s_1}^{\dsfss}(1)$, for the proof of the second assertion, it suffices to show that $\Thetab_{\mathsf s_1}^{\dsfss}(1)$ is nonzero.

Put
\[
  V_{\sfss_1}(\R):=\begin{cases}
    \textrm{the fixed point set  of  $J_{\sfss_1}$ in $V_{\sfss_1}$},\quad &\textrm{if $\star\in \{B,C, D, \widetilde C\}$};\\
    V_{\sfss_1},\quad & \textrm{if $\star\in \{C^*, D^*\}$}.
  \end{cases}
\]
As usual, realize $\hat \omega_{\dsfss, \sfss_1}$ on the space of square integrable functions on $(V_{\sfss_1}(\R))^{\dot p}$ so that $ \omega_{\dsfss, \sfss_1}$ is identified with the space of the Schwartz functions, and $G_{\sfss_1}$ acts on it through the obvious transformation.

Take a positive valued Schwartz function $\phi$ on
$(V_{\sfss_1}(\R))^{\dot p}$. Then
\[
  \inn{ g\cdot \phi}{ \phi}=\int_{(V_\sfss(\R))^{\dot p}} \phi(g^{-1}\cdot x) \cdot \phi(x)
  \od \! x>0, \quad \textrm{for all }g\in G_{\sfss_1}.
\]
Thus
\[
  \int_{G_{\sfss_1}} \inn{ g \cdot \phi}{ \phi} \od\! g \neq 0,
\]
and the lemma follows.
\end{proof}


\begin{lem}\label{doublelift5}
Assume that $\pi'$  is $\nu'$-bounded for some
%$ \nu'>-\frac{\abs{\sfss_0}}{2}-1$.
\[
  \nu'>
  \begin{cases}
 -\frac{\abs{\sfss_0}}{2}-3,\quad& \textrm{if $\star =C^*$};\\
   -\frac{\abs{\sfss_0}}{2}-1,\quad& \textrm{otherwise}.
   \end{cases}
   \]
\delete{\be\label{bound}
\nu'>\begin{cases}
  \nu_{\sfss'}-2k,\quad& \textrm{if $\dot \star\in \{C, C^*, D^*\}$};\\
  \nu_{\sfss'}-2k+2,\quad& \textrm{if $\dot \star= D$};\\
  \nu_{\sfss'}-2k-1,\quad& \textrm{if $\dot \star=\widetilde C$}.
   \end{cases}
\ee
}
 Then for all $\sfss_1$ appearing in Lemma \ref{degens} and all unitary character $\gamma'$ of $G_{\sfss'}$,  the representation $\pi'\otimes \gamma'$ is convergent for $\check \Theta^{\sfss_1}_{\sfss'}$, and
 $\Thetab^{\sfss_1}_{\sfss'}(\pi'\otimes \gamma')$ is overconvergent for $\check \Theta^{\sfss''}_{\sfss_1}$.
 \end{lem}
\begin{proof}
The first assertion follows by noting that
\be\label{p123}
\nu_{\sfss'}-\abs{\sfss_1}=
  \begin{cases}
 -\frac{\abs{\sfss_0}}{2}-3,\quad& \textrm{if $\star =C^*$};\\
   -\frac{\abs{\sfss_0}}{2}-1,\quad& \textrm{otherwise}.
   \end{cases}
   \ee
The proof of second assertion is similar to the proof of the first assertion of Lemma \ref{lem:coinv}.
\end{proof}

Recall the character
$ \chi_0:=\dot \chi|_{R_{\sfss_0}}$ from \eqref{chi0}. Similarly we define $\chi'_0:=\dot \chi'|_{R_{\sfss_0}}$.
Recall that $\pi'$ is assumed to be genuine when $\dot \star=\widetilde C$.

\begin{thm}\label{doubtt}
Assume that $\pi'$  is $\nu'$-bounded for some
$
  \nu'>
 -\frac{\abs{\sfss_0}}{2}-1.
$

\noindent
(a) If $\star\in \{B,D\}$,  then
\[
    \bigoplus_{\sfss_1\in S_{\star, k}}   \Thetab^{\sfss''}_{\sfss_1}(\Thetab^{\sfss_1}_{\sfss'}(\pi'))\cong \Ind_{P_{\sfss'',\sfss_0}}^{G_{\sfss''}} (\pi'\otimes \chi_0)
    \oplus  \Ind_{P_{\sfss'',\sfss_0}}^{G_{\sfss''}} (\pi' \otimes \chi'_0).
    \]

  \smallskip



    \noindent
(b) If $\star\in \{C,\widetilde C\}$, then
\[
 \Thetab^{\sfss''}_{\sfss_1}(\Thetab^{\sfss_1}_{\sfss'}(\pi'))\oplus\left(  (\Thetab^{\sfss''}_{\sfss_1}(\Thetab^{\sfss_1}_{\sfss'}(\pi'\otimes \det)))\otimes \det \right)\cong \Ind_{P_{\sfss'',\sfss_0}}^{G_{\sfss''}} (\pi'\otimes \chi_0),
\]
where $\sfss_1=(\star, \frac{k}{2},\frac{k}{2})$.
  \smallskip

\noindent
(c) If $\star=D^*$, then
\[
 \Thetab^{\sfss''}_{\sfss_1}(\Thetab^{\sfss_1}_{\sfss'}(\pi'))\cong \Ind_{P_{\sfss'',\sfss_0}}^{G_{\sfss''}} (\pi'\otimes \chi_0),
 \]
where $ \mathsf s_1=(D^*, \frac{k}{2}, \frac{k}{2})$.


\smallskip

\noindent
(d) If $\star=C^*$, then there is an exact sequence
\[
 0\rightarrow   \bigoplus_{\mathsf s_1\in S_{\star, k}}  \Thetab^{\sfss''}_{\sfss_1}(\Thetab^{\sfss_1}_{\sfss'}(\pi'))\rightarrow \Ind_{P_{\sfss'',\sfss_0}}^{G_{\sfss''}} (\pi'\otimes \chi_0)\rightarrow \cJ\rightarrow 0
\]
 of representations of $G_{\sfss''}$ such that $\cJ$ is a quotient of
\[
\bigoplus_{\mathsf s_2\in S_{\star, k-2}} \check \Theta^{\sfss''}_{\sfss_2}(\check \Theta^{\sfss_2}_{\sfss'}(\pi')).
\]
\end{thm}
\begin{proof}
Using Lemma \ref{growthdp}, we know that $\pi'^-$ is convergent for $I(\dot \chi)$ (and convergent for $I(\dot \chi')$ when $\star\in \{B,D\}$). Also note that the character $\chi'$ (see \eqref{chip}) is trivial.

If we are in the situation (a), (b) or (c), by using Theorem \ref{midp}, Lemma \ref{lem:coinv} and Proposition \ref{doublelift}, the theorem follows by applying the operation $\pi'^-*(\,\cdot\,)$ to the isomorphism in Lemma \ref{degens}.

Now assume that we are in the situation (d).
For simplicity, write $0\rightarrow I_1\rightarrow I_2\rightarrow I_3\rightarrow
0$ for the exact sequence in part (e) of Lemma \ref{degens}. Since $\pi'^-$ is nuclear as a Fr\'echet space, the sequence
\[
0\rightarrow \pi'^-\totimes I_1\rightarrow \pi'^-\totimes I_2\rightarrow \pi'^-\totimes I_3\rightarrow
0
\]
is also topologically exact.
Note that the natural map
\[
\pi'^- * I_1\rightarrow \pi'^- * I_2
\]
is injective, and the natural map
\[
  \pi'^-\totimes I_3\cong \frac{\pi'^-\totimes I_2}{\pi'^- \totimes I_1}\longrightarrow\frac{ \pi'^-*I_2}{\pi'^-*I_1}
  \]
descends to a surjective map
\[
  (\pi'^-\totimes  I_3)_{G_{\sfss'^-}}\rightarrow \frac{ \pi'^-*I_2}{\pi'^-*I_1}.
\]
Note that
\[
   (\pi'^-\totimes  I_3)_{G_{\sfss'^-}}\cong \bigoplus_{\mathsf s_2\in S_{\star, k-1}} \check \Theta^{\sfss''}_{\sfss_2}(\check \Theta^{\sfss_2}_{\sfss'}(\pi')).
\]
Theorem \ref{midp} implies that
\[
\pi'^-*I_2\cong \Ind_{P_{\sfss'',\sfss_0}}^{G_{\sfss''}} (\pi'\otimes \chi_0).
\]
Lemma \ref{lem:coinv} and Proposition \ref{doublelift} imply that
\[
\pi'^-*I_1\cong  \bigoplus_{\mathsf s_1\in S_{\star, k} } \Thetab^{\sfss''}_{\sfss_1}(\Thetab^{\sfss_1}_{\sfss'}(\pi')).
\]
Therefore the theorem follows.
\end{proof}

\section{Regular descents and bounding the associated  cycles}\label{sec:AC}

In this section, let $\mathsf s=(\star, p,q)$ and $ \mathsf s'=(\star', p',q')$ be classical signatures such that $\star'$ is the Howe dual of $\star$.
Let $\CO\in \mathrm{Nil}(\g_\mathsf s)$, whose Zariski closure in $\g_\sfss$ is denoted by $\overline \CO$. 


\subsection{Associated cycles}%{Graded $(\oS(\p_{\sfss}), K_{\sfss, \C})$-modules}

In this subsection, we will recall the definition of associated cycles from \cite{Vo89}.
\begin{defn}\label{spkmodule}
 Let $H_\C$ be a complex linear algebraic group. Let $\CA$ be a commutative $\C$-algebra carrying a locally algebraic linear action of $H_\C$ by algebra automorphisms.
  An $(\CA, H_\C)$-module is an  $\CA$-module
$
  \sigma
$
equipped with a locally algebraic linear action of $H_\C$ such that
the module structure map
$
  \CA\otimes \sigma\rightarrow \sigma
$
is $H_{\C}$-equivariant. An $(\CA, H_\C)$-module is said to be finitely generated if it is so as an $\CA$-module.

\end{defn}

%For every complex vector space $E$, let $\oS(E)$ denote the symmetric algebra of $E$.


For every affine complex algebraic variety $Z$, write $\C[Z]$ for the algebra of regular functions on $Z$. By convention, all elements of complex algebraic varieties are closed. 
%Write $\C[\overline \CO\cap \p]$ for the algebra of algebraic functions on $\overline \CO\cap \p$.  The notion of $(\C[\overline \CO\cap \p], K_{\sfss, \C})$-modules is similarly defined as in Definition \ref{spkmodule}. An  $(\oS(\p_{\sfss}), K_{\sfss, \C})$-module is said to be finitely generated if it is so as a $\oS(\p_{\sfss})$-module.
%Similarly, a  $(\C[\overline \CO\cap \p], K_{\sfss, \C})$-module is said to be finitely generated if it is so as a $\C[\overline \CO\cap \p]$-module.
Given a finitely generated $(\C[\overline \CO\cap \p_\sfss], K_{\sfss, \C})$-module $\sigma^\circ$, for each $K_{\sfss, \C}$-orbit $\sO\subset \CO\cap \p_\sfss$, the set
\[
  \bigsqcup_{\mathbf e\in \sO} \C_\mathbf e\otimes_{\C[\overline \CO\cap \p_\sfss]}\sigma^\circ
\]
is naturally a $K_{\sfss, \C}$-equivariant algebraic vector bundle over $\sO$, where $\C_\mathbf e$  denotes the complex number field $\C$ viewing as a $\C[\overline \CO\cap \p_\sfss]$-algebra via the evaluation map at $\mathbf e$. We define $\mathrm{AC}_\sO(\sigma^\circ)\in \CK_\sfss(\sO)$ to be the Grothendieck group element associated to this bundle. Define the  associated cycle of $\sigma^\circ$ to be
\[
  \mathrm{AC}_\CO(\sigma^\circ):=\sum_{\sO\textrm{ is a   $K_{\sfss, \C}$-orbit in $ \CO\cap \p_\sfss$}} \mathrm{AC}_\sO(\sigma^\circ)\in \CK_\sfss(\CO).
\]


Write $I_{\overline \CO\cap \p_\sfss}$ for the radical ideal of $\C[\p_\sfss]$ corresponding to the closed subvariety $\overline \CO\cap \p_\sfss$.
Every $(\C[\overline \CO\cap \p_\sfss], K_{\sfss, \C})$-module is clearly an $(\C[\p_\sfss], K_{\sfss,\C})$-module.
On the other hand, for each  $(\C[\p_{\sfss}], K_{\sfss, \C})$-module $\sigma$,
\[
  \frac{(I_{\overline \CO\cap \p_\sfss})^i\cdot \sigma}{(I_{\overline \CO\cap \p_\sfss})^{i+1}\cdot \sigma}\qquad (i\in \bN)
\]
is naturally a $(\C[\overline \CO\cap \p_\sfss], K_{\sfss, \C})$-module.
We say that $\sigma$ is $\CO$-bounded if
\[
(I_{\overline \CO\cap \p_\sfss})^i \cdot\sigma=0\qquad
\textrm{
for some $i\in \bN$.}
\]
When $\sigma$ is finitely generated, this is equivalent to saying that  the support of $\sigma$ is contained in $\overline{\CO}\cap \p_\sfss$.

 When $\sigma$ is finitely generated and $\CO$-bounded, we define its associated cycle to be
\[
  \mathrm{AC}_\CO(\sigma):=\sum_{i\in \bN}\mathrm{AC}_\CO\left(\frac{(I_{\overline \CO\cap \p_\sfss})^i\cdot \sigma}{(I_{\overline \CO\cap \p_\sfss})^{i+1}\cdot \sigma}\right)\in  \CK_\sfss(\CO).
\]
The assignment $\mathrm{AC}_\CO$ is additive in the following sense: the equality
\[
 \mathrm{AC}_\CO(\sigma)= \mathrm{AC}_\CO(\sigma')+ \mathrm{AC}_\CO(\sigma'')
\]
holds for every exact sequence
$0\rightarrow \sigma'\rightarrow \sigma\rightarrow \sigma''\rightarrow 0$ of finitely generated $\CO$-bounded $(\oS(\p_{\sfss}), K_{\sfss, \C})$-modules.


Given a $(\g_\sfss, K_{\sfss})$-module $\rho$ of finite length, pick a filtration
\be\label{goodf}
\CF:\quad  \cdots\subset \rho_{-1}\subset \rho_0\subset \rho_1\subset \rho_2\subset \cdots
\ee
of $\rho$ that is good in the following sense:
\begin{itemize}
\item for each $i\in \Z$, $\rho_i$ is a finite-dimensional  $K_{\sfss}$-stable subspace of $\rho$;
\item $\g_\sfss \cdot \rho_i\subset \rho_{i+1}$ for all $i\in \Z$, and $\g_\sfss \cdot \rho_i=\rho_{i+1}$ when  $i\in \Z$ is sufficiently large;
\item $\bigcup_{i\in \Z} \rho_i=\rho$, and $\rho_i=0$ for some $i\in \Z$.
\end{itemize}
Then the grading
\[
  \mathrm{Gr}(\rho):= \mathrm{Gr}(\rho,\CF):=\bigoplus_{i\in \Z} \rho_i/\rho_{i+1}
\]
is naturally a finitely generated  $(\oS(\p_{\sfss}), K_{\sfss, \C})$-module, where $\oS(\p_{\sfss})$ denotes the symmetric algebra of $\p_\sfss$. Recall that we have identified $\p_\sfss$ with its dual space $\p_\sfss^*$ by using the trace from. Hence $\oS(\p_{\sfss})$ is identified with $\C[\p_\sfss]$, and $\mathrm{Gr}(\rho)$ is a $(\C[\p_{\sfss}], K_{\sfss, \C})$-module.


Recall that $\rho$ is said to be $\CO$-bounded if the associated variety of its annihilator ideal in $\oU(\g_\mathsf s)$ is contained in  $\overline \CO$.

\begin{lem}\label{l62}
Let $\rho$ be a $(\g_\sfss, K_{\sfss})$-module  of finite length,  and let $\CF$ be a  good filtration of $\rho$. Then $\rho$ is $\CO$-bounded if and only if $\mathrm{Gr}(\rho, \CF)$ is $\CO$-bounded as a $(\C[\p_{\sfss}], K_{\sfss, \C})$-module.
\end{lem}
\begin{proof}
This is a direct consequence of   \cite[Theorem 8.4]{Vo89}.
\end{proof}

When $\rho$ is  $\CO$-bounded, it associated cycle is defined to be
\[
   \mathrm{AC}_\CO(\rho):= \mathrm{AC}_\CO(\mathrm{Gr}(\rho))\in  \CK_\sfss(\CO).
\]
This is independent of the good filtration \eqref{goodf}. Moreover, taking the associated cycles is additive in the following sense: the equality
\[
 \mathrm{AC}_\CO(\rho)= \mathrm{AC}_\CO(\rho')+ \mathrm{AC}_\CO(\rho'')
\]
holds for every exact sequence
$0\rightarrow \rho'\rightarrow \rho\rightarrow \rho''\rightarrow 0$ of  $\CO$-bounded  $(\g_\sfss, K_{\sfss})$-modules of finite length.

For every Cassleman-Wallach representation $\pi$ of $G_\sfss$, write $\pi^{\mathrm{alg}}$ for the underlying $(\g_\sfss, K_\sfss)$-module of $\pi$. When $\pi$ is $\CO$-bounded, we define the associated cycle $\mathrm{AC}_\CO(\pi):=\mathrm{AC}_\CO(\pi^{\mathrm{alg}})$.


\subsection{Algebraic theta lifts and commutative theta lifts}


We retain the notation of Section \ref{sec:Nil} and Section \ref{sec:Integrals}.
Denote by $\omega^{\mathrm{alg}}_{\mathsf s, \mathsf s'}$ the $\h_{\mathsf s, \mathsf s'}$-submodule of $\omega_{\mathsf s, \mathsf s'}$ generated by  $ \omega_{\mathsf s, \mathsf s'}^{\cX_{\mathsf s, \mathsf s'}}$. This is an $(\g_\mathsf s\times \g_{\mathsf s'}, K_\mathsf s\times K_{\mathsf s'})$-module. For every $ (\g_{\mathsf s'}, K_{\mathsf s'})$-module $\rho'$, define its full theta lift (the algebraic theta lift) to be the $(\g_{\mathsf s}, K_{\mathsf s})$-module
\[
   \check \Theta_{\mathsf s'}^{\mathsf s}(\rho'):=(\omega^{\mathrm{alg}}_{\mathsf s, \mathsf s'}\otimes \rho')_{\g_{\mathsf s'}, K_{\mathsf s'}} \qquad (\textrm{the  coinvariant space}).
\]
It has finite length whenever $\rho'$ has finite length (\cite[Section 4]{Howe89}). 



Recall from \eqref{momentmap} the moment maps
  \[
    \xymatrix@R=0em@C=3.0em{
     \p_\sfss^*=\fpp_\mathsf s &\ar[l]_{M_\mathsf s} \cX_{\mathsf s, \mathsf s'}\ar[r]^{M_{\mathsf s'}}& \p_{\sfss'}^*=\fpp_{\mathsf s'},\\
    \qquad \phi^* \phi  &\ar@{|->}[l] \phi \ar@{|->}[r] & \phi \phi^*.\qquad
    }
  \]
We view $\C[\cX_{\mathsf s, \mathsf s'}]$ as an $\C[\p_\sfss]\otimes \C[\p_{\sfss'}]$-algebra by using the moment maps. The natural action of $K_{\sfss,\C}\times K_{\sfss',\C}$ on $\cX_{\mathsf s, \mathsf s'}$ yields a locally algebraic linear action of  $K_{\sfss,\C}\times K_{\sfss',\C}$ on $\C[\cX_{\mathsf s, \mathsf s'}]$. Thus $\C[\cX_{\mathsf s, \mathsf s'}]$  is naturally a  $(\C[\p_{\sfss}]\otimes \C[\p_{\sfss'}], K_{\sfss, \C}\times K_{\sfss', \C})$-module.
%: it is an $\oS(\p_\sfss)\otimes \oS(\p_{\sfss'})$-algebra via  the moment maps, and it is a locally algebraic representation of $K_{\sfss,\C}\times K_{\sfss',\C}$ via the action of $K_{\sfss,\C}\times K_{\sfss',\C}$ on $\cX_{\mathsf s, \mathsf s'}$.

For every  $(\C[\p_{\sfss'}], K_{\sfss', \C})$-module $\sigma'$, define its commutative theta lift to be
\[
   \check \Theta_{\mathsf s'}^{\mathsf s}(\sigma'):=(\C[\CX_{\sfss, \sfss'}]\otimes_{\C[\p_{\sfss'}]} \sigma'\otimes \zeta_{\sfss, \sfss'})_{K_{\mathsf s', \C}} \qquad (\textrm{the  coinvariant space}),
\]
which is naturally a  $(\C[\p_{\sfss}], K_{\sfss, \C})$-module.

\begin{lem}
Suppose that $\sigma'$ is a finitely generated  $(\C[\p_{\sfss'}], K_{\sfss', \C})$-module. Then the  $(\C[\p_{\sfss}], K_{\sfss, \C})$-module   $\check \Theta_{\mathsf s'}^{\mathsf s}(\sigma')$ is also finitely generated.
\end{lem}
\begin{proof} This follows from the structural analysis of the space of joint harmonics, as in \cite[Section 4]{Howe89}. 
\end{proof}

\begin{lem}\label{lm}
Suppose that $\rho'$ is a  $(\g_{\sfss'}, K_{\sfss'})$-module of finite length. Then there exist a good filtration $\CF'$ on $\rho'$, a good filtration $\CF$ on $\check \Theta_{\mathsf s'}^{\mathsf s}(\rho')$, and a surjective $(\C[\p_{\sfss}], K_{\sfss, \C})$-module homomorphism
\[
  \check \Theta_{\mathsf s'}^{\mathsf s}(\mathrm{Gr}(\rho',\CF')) \rightarrow \mathrm{Gr}(\check \Theta_{\mathsf s'}^{\mathsf s}(\rho'),\CF).
\]
\end{lem}
\begin{proof} This follows from the discussions in \cite[Section 3.2]{LM}.
\end{proof}


Recall from \eqref{momentmap2}  the moment maps
\[
    \xymatrix@R=0em@C=3em{
      \g_\mathsf s &\ar[l]_{\tilde M_\mathsf s} W_{\mathsf s, \mathsf s'}\ar[r]^{\tilde M_{\mathsf s'}}& \g_{\mathsf s'},\\
     \phi^* \phi & \ar@{|->}[l] \phi \ar@{|->}[r] & \phi \phi^*.
    }
  \]


\begin{lem}\label{comobound}
Suppose that $\CO'\in \mathrm{Nil}(\g_{\sfss'})$ and  $M_{\sfss'}^{-1}(\overline{\CO'})\subset M_{\sfss'}^{-1}(\overline{\CO})$, where  $\overline{\CO'}$ denotes the Zariski closure of $\CO'$ in $\g_{\sfss'}$.  Then 
\begin{itemize}
\item
$\check \Theta_{\mathsf s'}^{\mathsf s}(\sigma')$ is $\CO$-bounded for every $\CO'$-bounded $(\C[\p_{\sfss'}], K_{\sfss', \C})$-module $\sigma'$;
\item
 $\check \Theta_{\mathsf s'}^{\mathsf s}(\rho')$ is $\CO$-bounded for every $\CO'$-bounded $(\g_{\mathsf s'}, K_{\sfss'})$-module $\rho'$ of finite length;
 \item $\check \Theta_{\mathsf s'}^{\mathsf s}(\pi')$ is $\CO$-bounded for every $\CO'$-bounded Casselman-Wallach representation $\pi'$ of $G_{\mathsf s'}$.
 \end{itemize}
\end{lem}
\begin{proof} The assumption of the lemma implies  that 
\[
 ( I_{\overline{\CO}\cap \p_\sfss})^i\cdot \C[\CX_{\sfss, \sfss'}] \subset I_{\overline{\CO'}\cap \p_{\sfss'}}\cdot  \C[\CX_{\sfss, \sfss'}]\qquad \textrm{for some } i\in \bN,
\]
where $I_{\overline{\CO'}\cap \p_{\sfss'}}$ denotes  the radical ideal of $\C[\p_\sfss]$ corresponding to the closed subvariety $\overline{\CO'}\cap \p_{\sfss'}$.
This implies the first assertion of the lemma. In view of Lemma \ref{l62}, the second assertion is implied by the first one and Lemma \ref{lm}. The third assertion is implied by the second one since there is a surjective 
$(\g_{\sfss}, K_{\sfss})$-module homomorphisms
\[
\check \Theta_{\mathsf s}^{\mathsf s'}(\pi'^{\mathrm{alg}})\rightarrow \left (\check \Theta_{\mathsf s'}^{\mathsf s_1}(\pi')\right )^{\mathrm{alg}}.
 \]

\end{proof}


\subsection{Regular descent of a nilpotent orbit}\label{regud}

\begin{defn}
The orbit $\CO\in \mathrm{Nil}(\g_\mathsf s)$ is regular for $\DD_{\mathsf s'}^{\mathsf s}$ if
either
\[
\abs{\sfss'}=\abs{\DD_\mathrm{naive}(\CO)},
\]
 or
 \[
 \abs{\sfss'}>\abs{\DD_\mathrm{naive}(\CO)}\qquad\textrm{and}\qquad \mathbf c_1(\CO)=\mathbf c_2(\CO).
\]

\end{defn}



In the rest of this section we assume that $\CO$ is regular for $\DD_{\mathsf s'}^{\mathsf s}$. Then Lemma \ref{imageofmm} implies that  $\CO$ is contained in the image of the moment map $\tilde M_{\mathsf s}$. Put  $\CO':=\DD_{\mathsf s'}^{\mathsf s}(\CO)\in  \mathrm{Nil}(\g_{\mathsf s'})$, and let  $\overline{\CO'}$ denote the Zariski closure of $\CO'$ in $\g_{\sfss'}$.


\begin{lem}\label{liftop000}
As subsets of $\g_\sfss$,
\[
  \tilde M_{\mathsf s}(\tilde M_{\mathsf s'}^{-1}(\overline{\CO'}))=\overline{\CO}.
\]
\end{lem}
\begin{proof}
This is implied by \cite[Theorems 5.2 and 5.6]{DKPC}.
\end{proof}


\begin{lem}\label{comobound2}
Suppose that $\sigma'$ is an $\CO'$-bounded $(\C[\p_{\sfss'}], K_{\sfss', \C})$-module.  Then the  $(\C[\p_{\sfss}], K_{\sfss, \C})$-module   $\check \Theta_{\mathsf s'}^{\mathsf s}(\sigma')$ is $\CO$-bounded.
\end{lem}
\begin{proof}
This follows from Lemmas \ref{liftop000} and \ref{comobound}. 
\end{proof}


Put
\[
\partial :=\overline \CO\setminus \CO\qquad\textrm{and}\qquad  \bar \partial :=\g_\sfss\setminus \partial.
\]
Then $\bar \partial$ is an open subvariety of $\g_\sfss$ and $\CO$ is a closed subvariety of $\bar \partial$.
Put
\[
W_{\sfss, \sfss'}^{\bar \partial}:=\tilde M_{\sfss}^{-1}(\bar \partial )
\qquad\textrm{and}\qquad
  W_{\sfss, \sfss'}^{\CO, \CO'}:=\tilde M_{\sfss}^{-1}(\CO)\cap  \tilde M_{\sfss'}^{-1}(\CO').
\]
Then $W_{\sfss, \sfss'}^{\bar \partial}$ is an open subvariety of $W_{\sfss, \sfss'}$, and the following lemma implies that $W_{\sfss, \sfss'}^{\CO, \CO'}$ is a closed subvariety of $W_{\sfss, \sfss'}^{\bar \partial}$.

\begin{lem}\label{liftop0}
The set $W_{\sfss, \sfss'}^{\CO, \CO'}$ is  a single $G_{\sfss, \C}\times G_{\sfss',\C}$-orbit. Moreover, it is  contained in $W_{\sfss,\sfss'}^\circ$ and equals
\[
   W_{\sfss, \sfss'}^{\bar \partial} \cap \tilde M_{\mathsf s'}^{-1}(\overline{\CO'}).
\]
\end{lem}
\begin{proof}
The first assertion is implied by \cite[Theorem 3.6]{DKPC}. Recall from \eqref{kkpo2} that $W_{\sfss,\sfss'}^\circ\cap \tilde M_{\sfss}^{-1}(\CO)$ is a single  $G_{\sfss, \C}\times G_{\sfss',\C}$-orbit  whose image under the moment map $\tilde M_{\sfss'}$ equals $\CO'$. Thus
\[
W_{\sfss,\sfss'}^\circ\cap W_{\sfss, \sfss'}^{\CO, \CO'}=W_{\sfss,\sfss'}^\circ\cap \tilde M_{\sfss}^{-1}(\CO),
\]
which  is also a   single  $G_{\sfss, \C}\times G_{\sfss',\C}$-orbit. Hence $W_{\sfss, \sfss'}^{\CO, \CO'}\subset W_{\sfss,\sfss'}^\circ$. 

In fact, \cite[Theorem 3.6]{DKPC} also implies that
\[
W_{\sfss, \sfss'}^{\CO, \CO'}=\tilde M_{\sfss}^{-1}(\CO)\cap  \tilde M_{\sfss'}^{-1}(\overline{\CO'}).
\]
Thus the last assertion is a direct consequence of Lemma \ref{liftop000}.

\end{proof}



Write
\[
    \CX_{\sfss, \sfss'}^{\bar \partial}:=\CX_{\sfss, \sfss'} \cap W_{\sfss, \sfss'}^{\bar \partial}\qquad\textrm{and}\qquad \CX_{\sfss, \sfss'}^{\CO, \CO'}:=\CX_{\sfss, \sfss'} \cap W_{\sfss, \sfss'}^{\CO, \CO'}.
   \]
Then  $ \CX_{\sfss, \sfss'}^{\bar \partial}$ is an open subvariety of $ \CX_{\sfss, \sfss'}$ and  $\CX_{\sfss, \sfss'}^{\CO, \CO'}$ is a  closed subvariety of  $\CX_{\sfss, \sfss'}^{\bar \partial}$. We have a decomposition
 \[
   \CX_{\sfss, \sfss'}^{\CO, \CO'}=\bigsqcup_{\sO\textrm{ is a   $K_{\sfss, \C}$-orbit in $ \CO\cap \p_\sfss$ that is contained in the image of $M_\sfss$}}  \CX_{\sfss, \sfss'}^{\sO, \CO'},
 \]
 where
 \[
   \CX_{\sfss, \sfss'}^{\sO, \CO'}:=M_\sfss^{-1}(\sO)\cap \CX_{\sfss, \sfss'}^{\CO, \CO'},
 \]
 which is a Zariski open and closed subset of $\CX_{\sfss, \sfss'}^{\CO, \CO'}$. Lemmas \ref{descko} and \ref{liftop0} imply that $ \CX_{\sfss, \sfss'}^{\sO, \CO'}$ is a single $K_{\sfss, \C}\times K_{\sfss',\C}$-orbit.

\medskip

We defer the proof of the following proposition to Sections \ref{sec:GG} and \ref{secpp}.
\begin{prop}\label{propreduced}
 The scheme theoretic  fibre product
\[
\CX_{\sfss, \sfss'}^{\bar \partial}\times_{\p_{\sfss'}} ({\CO'}\cap \p_{\sfss'})
\]
is reduced, where $\CX_{\sfss, \sfss'}^{\bar \partial}$ is viewed as a $\p_{\sfss'}$-scheme via the moment map $M_{\sfss'}$.
\end{prop}

By Lemma \ref{liftop0} and Proposition \ref{propreduced}, we know that
\be\label{cxee}
 \CX_{\sfss, \sfss'}^{\CO, \CO'}=\CX_{\sfss, \sfss'}^{\bar \partial}\times_{\p_{\sfss'}} ({\CO'}\cap \p_{\sfss'})=\CX_{\sfss, \sfss'}^{\bar \partial}\times_{\p_{\sfss'}} ({\overline{\CO'}}\cap \p_{\sfss'}),
 \ee
 which  is a smooth closed subvariety of $\CX_{\sfss, \sfss'}^{\bar \partial}$.


For every element $\mathbf e\in \CO\cap \p_\sfss$, write $K_\mathbf e$ for the stabilizer of $\mathbf e$ in $K_{\sfss, \C}$. Let $\C_\mathbf e$ be the field $\C$ viewing as a  $\C[\overline \CO\cap \p_\sfss]$-algebra via the evaluation map at $\mathbf e$. Put
\[
  \CX_{\sfss, \sfss'}^{\mathbf e, \CO'}:=M_\sfss^{-1}(\mathbf e)\cap M_{\sfss'}^{-1}(\CO'\cap\p_{\sfss'})=M_\sfss^{-1}(\mathbf e)\cap M_{\sfss'}^{-1}(\overline{\CO'}\cap\p_{\sfss'}),
\]
which is a closed subvariety of $\CX_{\sfss, \sfss'}$. If $\mathbf e$ is not contained in the image of $M_\sfss$, then $\CX_{\sfss, \sfss'}^{\mathbf e, \CO'}=\emptyset$. Otherwise, it is a single
$(K_\mathbf e\times K_{\sfss',\C})$-orbit, and Lemma \ref{descko} implies that it is also a single $K_{\sfss',\C}$-orbit.
\begin{lem}\label{fiber111}
For every element $\mathbf e\in \CO\cap \p_\sfss$,
\[
  \C_\mathbf e \otimes_{\C[\p_{\sfss}]} \C[\CX_{\sfss, \sfss'}]\otimes_{\C[\p_{\sfss'}]} \C[\overline{\CO'}\cap \p_{\sfss'}]=\C[\CX_{\sfss, \sfss'}^{\mathbf e, \CO'}].
\]
\end{lem}
\begin{proof}
As schemes, we have that
\begin{eqnarray*}
  &&\{\mathbf e\}\times_{\p_\sfss}{\CX_{\sfss, \sfss'}}\times_{\p_{\sfss'}} (\overline{\CO'}\cap \p_{\sfss'})\\
  &=& \{\mathbf e\}\times_{\p_\sfss}{\CX^{\bar \partial}_{\sfss, \sfss'}}\times_{\p_{\sfss'}} (\overline{\CO'}\cap \p_{\sfss'})\\
    &=& \{\mathbf e\}\times_{\p_\sfss}   \CX_{\sfss, \sfss'}^{\CO, \CO'}\\
    &=& \{\mathbf e\} \times_{\CO\cap \p_\sfss} ( (\CO\cap \p_\sfss) \times_{\p_\sfss} \CX_{\sfss, \sfss'}^{\CO, \CO'})\\
     &=& \{\mathbf e\} \times_{\CO\cap \p_\sfss} \CX_{\sfss, \sfss'}^{\CO, \CO'}\\
   &=& \CX_{\sfss, \sfss'}^{\mathbf e, \CO'}.
\end{eqnarray*}
The last equality holds because the morphism $M_\sfss: \CX_{\sfss, \sfss'}^{\CO, \CO'}\rightarrow \CO\cap \p_\sfss$  is  $K_{\sfss, \C}\times K_{\sfss',\C}$-equivariant and hence smooth in the sense of algebraic geometry. This proves the lemma.
%Proposition \ref{propreduced} and the equalities  in \eqref{cxee}
\end{proof}



\subsection{Commutative theta lifts and  geometric theta lifts}

The purpose of this subsection is to prove the following proposition.
\begin{prop}\label{propthetag}
Suppose that $\sigma'$ is a finitely generated  $(\C[\overline{\CO'}\cap \p_{\sfss'}], K_{\sfss', \C})$-module. Then
\[
\mathrm{AC}_{\CO}( \check \Theta_{\mathsf s'}^{\mathsf s}(\sigma'))= \check \vartheta_{\CO'}^{\CO}(\mathrm{AC}_{\CO'}(\sigma'))
\]
as elements of $\CK_\sfss(\CO)$.

\end{prop}


By a quasi-coherent module over a scheme $Z$, we mean a quasi-coherent module over the structure sheaf of $Z$. We say that a quasi-coherent module $\CM$ over a scheme $Z$ descends to a closed subscheme $Z_1$ of $Z$ if $\CM$ is isomorphic to the push-forward of a
 quasi-coherent module over $Z_1$ via the closed embedding $Z_1\rightarrow Z$. This is equivalent to saying that the ideal sheaf defining $Z_1$ annihilates $\CM$.

When $Z$ is an affine complex algebraic variety and $\sigma$ is a $\C[Z]$-module,
 we write $\CM_\sigma$ for  the quasi-coherent module over $Z$ corresponding to $\sigma$.

\begin{lem}\label{geotheta1}
Let $\sigma$ be a finitely generated $\CO$-bounded $(\C[\p_{\sfss}],K_{\sfss})$-module.  Assume that  $(\CM_\sigma)|_{\bar \partial\cap \p_\sfss}$ descends to $\CO\cap \p_\sfss$. Then
\[
  \mathrm{AC}_{\CO}(\sigma)=\mathrm{AC}_{\CO}(\C[\overline \CO\cap \p_{\sfss}]\otimes_{\C[\p_{\sfss}]} \sigma).
\]
\end{lem}
\begin{proof}
Let $\mathbf e\in \CO\cap \p_\sfss$. Note that for every ideal $I$ of $\C[\p_{\sfss}]$,
\[
  (I\cdot \sigma)_\mathbf e=(I_\mathbf e)\cdot \sigma_\mathbf e\subset  \sigma_\mathbf e,
\]
where a subscript $\mathbf e$ indicates the localization at $\mathbf e$. The assumption of the lemma implies that
\[
  ((I_{\overline{\CO}\cap \p_\sfss})^i)_\mathbf e \cdot \sigma_\mathbf e=0\qquad \textrm{for all } i\in \bN^+.
\]
Thus
\[
  ((I_{\overline{\CO}\cap \p_\sfss})^i \cdot \sigma)_\mathbf e=0,
\]
which implies the  lemma.
\end{proof}


\begin{lem}\label{geotheta2}
Suppose that  $Z$ is an affine complex algebraic variety with a transitive algebraic action of $K_{\sfss, \C}$. Let $\sigma$ be a finitely generated $(\C[Z], K_{\sfss, \C})$-module.
Let $z\in Z$ and write $K_z$ for the stabilizer of $z$ in $K_{\sfss, \C}$. Then  the natural map
\[
  \sigma^{K_{\sfss, \C}}\rightarrow (\C_z\otimes_{\C[Z]} \sigma)^{K_{z}}
\]
is a linear isomorphism. Here $\C_z$ is the field $\C$ viewing as a $\C[Z]$-algebra via the evaluation map at $z$, and a superscript group indicates the space of  invariant vectors under the group action.
\end{lem}
\begin{proof}
Note that the set
\[
 \bigsqcup_{x\in Z} \C_x\otimes_{\C[Z]} \sigma
\]
is naturally a $K_{\sfss, \C}$-equivariant algebraic vector bundle over $Z$, and  $\sigma$ is identified with the space of algebraic sections of this bundle. Thus
\[
  \sigma\cong \,^{\mathrm{alg}}\Ind_{K_z}^{K_{\sfss, \C}}(\C_x\otimes_{\C[Z]}\sigma)\qquad(\textrm{algebraically induced representation}).
\]
The lemma then follows by the algebraic version of the Frobenius reciprocity.
\end{proof}


\begin{lem}\label{geotheta3}
Suppose that $\sigma'$ is a finitely generated  $(\C[\overline{\CO'}\cap \p_{\sfss'}], K_{\sfss', \C})$-module, and write $\sigma:=\check \Theta_{\mathsf s'}^{\mathsf s}(\sigma')$. Then the  quasi-coherent module $(\CM_\sigma)|_{\bar \partial \cap \p_\sfss}$ descends  to $\CO\cap \p_\sfss$.
\end{lem}
\begin{proof}
Write
\[
 \tilde \sigma:=\C[\CX_{\sfss, \sfss'}]\otimes_{\C[\p_{\sfss'}]} \sigma'\otimes \zeta_{\sfss, \sfss'}=(\C[\CX_{\sfss, \sfss'}]\otimes_{\C[\p_{\sfss'}]} {\C[\overline{\CO'}\cap \p_{\sfss'}]})\otimes_{\C[\overline{\CO'}\cap \p_{\sfss'}]}(\sigma'\otimes \zeta_{\sfss, \sfss'}),
\]
to be viewed as a $\C[\CX_{\sfss, \sfss'}]$-module. Write $\tilde \sigma_0:=\sigma$, viewing as a $\C[\p_\sfss]$-module via the moment map $M_\sfss$.

Proposition \ref{propreduced} and the equalities  in \eqref{cxee} imply that
$(\CM_{ \tilde \sigma})|_{\CX_{\sfss, \sfss'}^{\bar \partial}}$ descends to $ \CX_{\sfss, \sfss'}^{\CO, \CO'}$. This implies that $(\CM_{ \tilde \sigma_0})|_{\bar \partial\cap \p_\sfss}$ descends to $ \CO\cap \p_\sfss$. The lemma then follows since $\sigma$ is a direct summand of $\tilde \sigma_0$.
\end{proof}

We are now ready to prove Proposition \ref{propthetag}.

\begin{proof}[Proof of Proposition \ref{propthetag}]
Let  $\sigma'$ be a finitely generated  $(\C[\overline{\CO'}\cap \p_{\sfss'}], K_{\sfss', \C})$-module, and write  $\sigma:=\check \Theta_{\mathsf s'}^{\mathsf s}(\sigma')$.
 Lemmas \ref{geotheta3} and \ref{geotheta1} imply that
\[
  \mathrm{AC}_{\CO}(\sigma)=\mathrm{AC}_{\CO}(\C[\overline \CO\cap \p_{\sfss}]\otimes_{\C[\p_{\sfss}]} \sigma).
\]

Suppose that $\mathbf e\in \CO\cap \p_\sfss$. Then
\begin{eqnarray*}
  && \C_\mathbf e \otimes_{\C[\overline \CO\cap \p_{\sfss}]} (\C[\overline \CO\cap \p_{\sfss}]\otimes_{\C[\p_{\sfss}]} \sigma)\\
  &=&  \C_\mathbf e \otimes_{\C[\p_{\sfss}]} \sigma\\
  &=&  \C_\mathbf e \otimes_{\C[\p_{\sfss}]} (\C[\CX_{\sfss, \sfss'}]\otimes_{\C[\p_{\sfss'}]} \sigma'\otimes \zeta_{\sfss, \sfss'})_{K_{\mathsf s', \C}} \\
   &=&\left( (\C_\mathbf e \otimes_{\C[\p_{\sfss}]} \C[\CX_{\sfss, \sfss'}]\otimes_{\C[\p_{\sfss'}]} \C[\overline{\CO'}\cap \p_{\sfss'}]) \otimes_{\C[\overline{\CO'}\cap \p_{\sfss'}]}\sigma'\otimes \zeta_{\sfss, \sfss'}\right)_{K_{\mathsf s', \C}} \\
      &=&\left( \C[\CX_{\sfss, \sfss'}^{\mathbf e, \CO'}] \otimes_{\C[\overline{\CO'}\cap \p_{\sfss'}]}\sigma'\otimes \zeta_{\sfss, \sfss'}\right)_{K_{\mathsf s', \C}} \qquad (\textrm{by Lemma \ref{fiber111}}).
\end{eqnarray*}

If $\mathbf e$ is not in the image of $M_\sfss$, then the above module is zero. Now we assume that $\mathbf e$ is in the image of $M_\sfss$ so that $\CX_{\sfss, \sfss'}^{\mathbf e, \CO'}$ is a single $K_{\sfss',\C}$-orbit. Pick an arbitrary element $\phi\in \CX_{\sfss, \sfss'}^{\mathbf e, \CO'}$, and write $\mathbf e':=M_{\sfss'}(\phi)$. Let
\[
  1\rightarrow  (K_{\mathsf s',\C})_\phi \rightarrow (K_{\mathsf s,\C}\times K_{\mathsf s', \C})_\phi\xrightarrow{\textrm{the projection to the first factor}} (K_{\mathsf s,\C})_{\mathbf e}\rightarrow 1
\]
be the exact  sequence as in Lemma \ref{descko}. %Note that $ K_{\mathsf s_1,\C}$ equals the stabilizer of $\phi$ in $K_{\sfss', \C}$.


 Let $\C_\phi$ denote the field $\C$ viewing as a $\C[\CX_{\sfss, \sfss'}^{\mathbf e, \CO'}]$-algebra via the evaluation map at $\phi$. Then we have that
\begin{eqnarray*}
      &&\left( \C[\CX_{\sfss, \sfss'}^{\mathbf e, \CO'}] \otimes_{\C[\overline{\CO'}\cap \p_{\sfss'}]}\sigma'\otimes \zeta_{\sfss, \sfss'}\right)_{K_{\mathsf s', \C}}\\
             &=&\left( \C[\CX_{\sfss, \sfss'}^{\mathbf e, \CO'}] \otimes_{\C[\overline{\CO'}\cap \p_{\sfss'}]}\sigma'\otimes \zeta_{\sfss, \sfss'}\right)^{K_{\mathsf s', \C}}\quad \qquad (\textrm{because $K_{\sfss',\C}$ is reductive})\\
             &=&\left( \C_\phi\otimes_{\C[\CX_{\sfss, \sfss'}^{\mathbf e, \CO'}]} \C[\CX_{\sfss, \sfss'}^{\mathbf e, \CO'}] \otimes_{\C[\overline{\CO'}\cap \p_{\sfss'}]}\sigma'\otimes \zeta_{\sfss, \sfss'}\right)^{ (K_{\mathsf s',\C})_\phi} \qquad (\textrm{by Lemma \ref{geotheta2}})\\
               &=&\left( \C_{\mathbf e'}  \otimes_{\C[\overline{\CO'}\cap \p_{\sfss'}]}\sigma'\otimes \zeta_{\sfss, \sfss'}\right)^{ (K_{\mathsf s',\C})_\phi}\\
                 &=&\left( \C_{\mathbf e'}  \otimes_{\C[\overline{\CO'}\cap \p_{\sfss'}]}\sigma'\otimes \zeta_{\sfss, \sfss'}\right)_{ (K_{\mathsf s',\C})_\phi} \quad \qquad (\textrm{because $ (K_{\mathsf s',\C})_\phi$ is reductive}).
\end{eqnarray*}
Putting all things together, we have an identification
\[
\C_\mathbf e \otimes_{\C[\overline \CO\cap \p_{\sfss}]} (\C[\overline \CO\cap \p_{\sfss}]\otimes_{\C[\p_{\sfss}]} \sigma)=\left( \C_{\mathbf e'}  \otimes_{\C[\overline{\CO'}\cap \p_{\sfss'}]}\sigma'\otimes \zeta_{\sfss, \sfss'}\right)_{ (K_{\mathsf s',\C})_\phi}.
\]
It is routine to check that this identification respects the $(K_{\mathsf s,\C}\times K_{\mathsf s', \C})_\phi$-actions, and $ (K_{\mathsf s',\C})_\phi$ acts trivially on the both sides. Thus the identification respects the $K_{\mathbf e}$-actions. In view of Lemmas \ref{geotheta1} and \ref{geotheta3}, this proves the proposition.
\end{proof}


\subsection{Algebraic theta lifts and  geometric theta lifts}

Proposition \ref{propthetag} has the following consequence.
\begin{prop}\label{propthetag2}
Suppose that $\sigma'$ is a finitely generated $\CO'$-bounded $(\C[\p_{\sfss'}], K_{\sfss', \C})$-module. Then
\[
\mathrm{AC}_{\CO}( \check \Theta_{\mathsf s'}^{\mathsf s}(\sigma'))\preceq  \check \vartheta_{\CO'}^{\CO}(\mathrm{AC}_{\CO'}(\sigma'))
\]
in $\CK_\sfss(\CO)$.

\end{prop}
\begin{proof}
Suppose that  $0\rightarrow \sigma'_1\rightarrow \sigma'_2\rightarrow \sigma'_3\rightarrow 0$ be an exact sequence of  $\CO'$-bounded finitely generated $(\C[\p_{\sfss'}], K_{\sfss', \C})$-modules. Then
\[
 \check \Theta_{\mathsf s'}^{\mathsf s}(\sigma'_1)\rightarrow \check \Theta_{\mathsf s'}^{\mathsf s}(\sigma'_2)\rightarrow \check \Theta_{\mathsf s'}^{\mathsf s}(\sigma'_3)\rightarrow 0
 \]
 is  an exact sequence of  $\CO$-bounded finitely generated $(\C[\p_{\sfss}], K_{\sfss, \C})$-modules.
Thus
 \[
   \mathrm{AC}_{\CO}( \check \Theta_{\mathsf s'}^{\mathsf s}(\sigma'_2))\preceq \mathrm{AC}_{\CO}( \check \Theta_{\mathsf s'}^{\mathsf s}(\sigma'_1))+\mathrm{AC}_{\CO}( \check \Theta_{\mathsf s'}^{\mathsf s}(\sigma'_3)).
 \]
 On the other hand, it is clear that
  \[
  \vartheta_{\CO'}^{\CO}(\mathrm{AC}_{\CO'}(\sigma'_2))= \vartheta_{\CO'}^{\CO}(\mathrm{AC}_{\CO'}(\sigma'_1))+ \vartheta_{\CO'}^{\CO}(\mathrm{AC}_{\CO'}(\sigma'_3)).
  \]
The proposition then easily follows by Proposition \ref{propthetag}.
\end{proof}

Finally, we are ready to prove the main result of this section. Recall that  $\cO$ is assumed to be  regular  for $\DD_{\mathsf s'}^{\mathsf s}$.
\begin{thm}\label{prop:GDS.AC}

  Let $\rho'$ be an $\CO'$-bounded $(\g_{\mathsf s'}, K_{\mathsf s'})$-module of finite length. Then  $\check \Theta_{\mathsf s'}^{\mathsf s}(\rho')$ is $\CO$-bounded, and
    \[
    \mathrm{AC}_{\cO}(\check \Theta_{\mathsf s'}^{\mathsf s}(\rho'))\preceq \check \vartheta_{\cO'}^\cO(\mathrm{AC}_{\cO'}(\rho')).
  \]
\end{thm}
\begin{proof}
Let $\CF'$ and $\CF$ be  good filtrations  on $\rho'$ and  $\check \Theta_{\mathsf s'}^{\mathsf s}(\rho')$ respectively as in Lemma \ref{lm} so that there exists a
 surjective $(\C[\p_{\sfss}], K_{\sfss, \C})$-module homomorphism
\be\label{surpkm}
  \check \Theta_{\mathsf s'}^{\mathsf s}(\mathrm{Gr}(\rho',\CF')) \rightarrow \mathrm{Gr}(\check \Theta_{\mathsf s'}^{\mathsf s}(\rho'),\CF).
\ee
The first assertion then follows from Lemmas \ref{l62} and \ref{comobound2}.

Put $\sigma':=\mathrm{Gr}(\rho',\CF')$.
Then
\begin{eqnarray*}
      && \mathrm{AC}_{\cO}(\check \Theta_{\mathsf s'}^{\mathsf s}(\rho'))\\
      &=&  \mathrm{AC}_{\cO}(\mathrm{Gr}(\check \Theta_{\mathsf s'}^{\mathsf s}(\rho'),\CF))\\
             &\preceq& \mathrm{AC}_{\cO}(\check \Theta_{\mathsf s'}^{\mathsf s}(\sigma')) \quad\  \qquad (\textrm{by \eqref{surpkm}})\\
          &  \preceq & \check \vartheta_{\CO'}^{\CO}(\mathrm{AC}_{\CO'}(\sigma'))\quad \qquad (\textrm{by Proposition \ref{propthetag2}})\\
               &=&\check \vartheta_{\CO'}^{\CO}(\mathrm{AC}_{\CO'}(\rho')).
               \end{eqnarray*}
This proves the theorem.
\end{proof}


\subsection{Geometries of regular descent}
\label{sec:GG}


\def\UU{{\bar \partial}}
\def\dbM{\breve{M}}
\def\dbMM{\breve{MM}}
\def\dbX{\breve{X}}
\def\dbfpp{\breve{\fpp}}
\def\ZdbX{\cZ_{\dbX}}
\def\aV{\acute{V}}
\def\fggs{\fgg_{\sfss}}
\def\fggsp{\fgg_{\sfss'}}
\def\fggspp{\fgg_{\sfss''}}
\def\fggspo{\fgg_{\sfss'_0}}
\def\fggspt{\fgg_{\sfss'_1}}
\def\fggspi{\fgg_{\sfss'_i}}
\def\fkks{\fkk_{\sfss}}
\def\fkksp{\fkk_{\sfss'}}
\def\fkkspo{\fkk_{\sfss'_0}}
\def\fkkspt{\fkk_{\sfss'_1}}
\def\fkkspi{\fkk_{\sfss'_i}}
\def\fpps{\fpp_{\sfss}}
\def\fppsp{\fpp_{\sfss'}}
\def\fppspo{\fpp_{\sfss'_0}}
\def\fppspt{\fpp_{\sfss'_1}}
\def\fppspi{\fpp_{\sfss'_i}}
\def\DDss{\DD_{\sfss'}^{\sfss}}
\def\DDsso{\DD_{\sfss'_0}^{\sfss}}
\def\cOpo{\cOp_{0}}
\def\Mss{M_{\sfss,\sfss'}}
\def\Ms{M_{\sfss}}
\def\Msp{M_{\sfss'}}


\def\Ks{{K}_{\sfss}}
\def\Ksp{{K}_{\sfss'}}
\def\Kspo{{K}_{\sfss'_0}}
\def\Kspt{{K}_{\sfss'_1}}

\def\Gs{{G}_{\sfss}}
\def\Gsp{{G}_{\sfss'}}
\def\Gspo{{G}_{\sfss'_0}}
\def\Gspt{{G}_{\sfss'_1}}
\def\CMs{{\tilde M}_{\sfss}}
\def\CMsp{{\tilde M}_{\sfss'}}
\def\CMss{{\tilde M}_{\sfss,\sfss'}}
\def\Wss{{W_{\sfss,\sfss'}}}
\def\Woss{{W^{\circ}_{\sfss,\sfss'}}}
\def\CXss{\cX_{\sfss,\sfss'}}
\def\X{\bfee}
\def\Xp{{\bfee'}}
\def\Xpo{{\bfee'_0}}
\def\ww{\phi}
\def\wwo{\phi_0}
\def\sOp{\sO'}
\def\fggpsb{{\fgg^{\boxslash}_{\sfss'}}}
\def\fpppsb{{\fpp^{\boxslash}_{\sfss'}}}
\def\fpppso{\fpp_{\sfss'_0}}
\def\fpppst{\fpp_{\sfss'_1}}

\def\Vs{V_\sfss}
\def\Vsp{V_{\sfss'}}
\def\Vspp{V_{\sfss''}}
\def\Vspo{V_{\sfss'_0}}
\def\Vspt{V_{\sfss'_1}}
\def\Vspi{V_{\sfss'_i}}
\def\Wssi{W_{\sfss,\sfss'_i}}
\def\Wsso{W_{\sfss,\sfss'_0}}
\def\Wsst{W_{\sfss,\sfss'_1}}

\def\CXUO{\cX_{\sfss,\sfss'}^{\bar\partial ,\cOp}}
\def\CXOO{\cX_{\sfss,\sfss'}^{\cO ,\cOp}}
%We investigate the geometric properties of regular descent.



The following lemma is a form of  the Jacobian criterion for
regularity (see \cite[Theorem~2.19]{LiuAG}).
\begin{lem}\label{jacobic}
Let $Z$ and $Z'$ be smooth complex algebraic varieties, and let $z'\in Z$. Let $f: Z\rightarrow Z'$ be a morphism of algebraic varieties such that $f^{-1}(z')$ is a smooth subvariety of $Z$. Then the scheme theoretic fibre product $Z\times_{Z'} \{z'\}$ is reduced if and only if the sequence 
\be\label{jc0}
  \mathrm T_z(f^{-1}(z'))\xrightarrow{\textrm{ inclusion }} \mathrm T_z(Z)\xrightarrow{\textrm{the differential of $f$}} \mathrm T_{z'}(Z')
\ee
is exact for all $z\in f^{-1}(z')$. 
\end{lem}

Here and as usual $\mathrm T$ indicates the tangent space. 

Recall that  $\CO\in \mathrm{Nil}(\g_\sfss)$ is regular for $\DD_{\mathsf s'}^{\mathsf s}$. 
Let $\mathbf e'\in \CO'=\DD_{\mathsf s'}^{\mathsf s}(\CO)$. Denote by $G_{\mathbf e'}$ the stabilizer of $\mathbf e'$ in $G_{\sfss',\C}$, and by $\g_{\mathbf e'}$ the Lie algebra of $G_{\mathbf e'}$. Consider the map
\[
 \tilde M':=  (\tilde M_{\sfss'})|_{W_{\sfss, \sfss'}^{\bar \partial}} : W_{\sfss, \sfss'}^{\bar \partial}\rightarrow \g_{\sfss'}. 
\]
By Lemma \ref{liftop0}, its fibre at $\mathbf e'$ equals the set
\[
W_{\sfss, \sfss'}^{\CO, \mathbf e'}:=\tilde M_{\sfss}^{-1}(\CO)\cap  \tilde M_{\sfss'}^{-1}(\mathbf e'),
\]
and this set is a single $G_{\sfss,\C}\times G_{\mathbf e'}$-orbit. Thus the set $W_{\sfss, \sfss'}^{\CO, \mathbf e'}$ is a smooth closed subvariety of $W_{\sfss, \sfss'}^{\bar \partial}$. 

Let $\phi\in W_{\sfss, \sfss'}^{\CO, \mathbf e'}$. Then we have a sequence 
\be\label{jc1}
    \mathrm T_\phi (W_{\sfss, \sfss'}^{\CO, \mathbf e'})\xrightarrow{\textrm{ inclusion}} \mathrm T_\phi(W_{\sfss, \sfss'}^{\bar \partial})\xrightarrow{\textrm{the differential of $\tilde M'$}} \mathrm T_{\mathbf e'}(\g_{\sfss'})
\ee
as in \eqref{jc0}.  
The tangent spaces $\mathrm T_\phi(W_{\sfss, \sfss'}^{\bar \partial})$ and $\mathrm T_{\mathbf e'}(\g_{\sfss'})$ are respectively identified with $W_{\sfss, \sfss'}$ and $\g_{\sfss'}$ as usual. Then the second map in \eqref{jc1} equals the map 
\be\label{differ1}
  W_{\sfss, \sfss'}\rightarrow \g_{\sfss'}, \qquad b\mapsto b\ww^{*}+\ww b^{*} . 
\ee
Note that the  image of the first map in \eqref{jc1} is identified with the image of the following map
\be\label{differ2}
 \fggs\times  \g_{\mathbf e'} \rightarrow W_{\sfss, \sfss'}, \qquad (a,a')\mapsto a'\ww - \ww a.
\ee
Thus the sequence \eqref{jc1} is exact if and only if the following sequence is exact: 
\begin{equation}\label{eq:CGG}
  \fggs\times  \g_{\mathbf e'} \xrightarrow{\eqref{differ2} } W_{\sfss, \sfss'}
  \xrightarrow{\eqref{differ1}} \fggsp.
\end{equation}




\begin{lem}\label{lemexact1}
The sequence \eqref{jc1} is exact when $\abs{\sfss'}=\abs{\nabla_{\mathrm{naive}}(\CO)}$.
\end{lem}
\begin{proof}
The map \eqref{differ1} equals the composition of 
\be\label{differ3}
  W_{\sfss, \sfss'}\xrightarrow{b\mapsto b\phi^*} \g\mathfrak l(V_{\sfss'})\xrightarrow{x\mapsto x-x^*} \g_{\sfss'}.
\ee
Here $ \g\mathfrak l(V_{\sfss'})$ denotes the algebra of  linear endomorphisms of $V_{\sfss'}$, and $x^*\in  \g\mathfrak l(V_{\sfss'})$ is specified by requiring that
\be\label{ajoint4}
  \la x u, v\ra_{\sfss'}=\la u, x^* v\ra_{\sfss'},\qquad \textrm{for all }u,v\in V_{\sfss'}. 
\ee
The second map in \eqref{differ3} is always surjective. 

Note that $\abs{\sfss'}=\abs{\nabla_{\mathrm{naive}}(\CO)}$ if and only if  $\phi: V_{\sfss}\rightarrow V_{\sfss'}$ is surjective. Assume this is the case. Then $\phi^*$ is injective, and consequently the first map in  \eqref{differ3}  is  surjective. Thus the map \eqref{differ1} is also surjective. Since $\phi\in W_{\sfss, \sfss'}^{\CO, \mathbf e'}$ is arbitrary, this implies that the scheme theoretic  fibre product 
$W_{\sfss, \sfss'}^{\bar \partial}\times_{\g_{\sfss'}}\{\mathbf e'\}$ is reduced (see \cite[Proposition~10.4]{HS}). Lemma \ref{jacobic} then implies that the sequence \eqref{jc1} is exact.


\end{proof}

\begin{lem}\label{lemexact2}
The sequence \eqref{jc1} is exact when $\mathbf c_1(\CO)=\mathbf c_2(\CO)$.
\end{lem}
\begin{proof}
Write 
\be\label{v12}
V_{\sfss'}=V_1'\oplus V_2',
\ee
where $V_1':=\phi(V_{\sfss})$, which is a non-degenerate subspace of $V_{\sfss'}$, and $V_2'$ is the orthogonal complement of $V_1'$ in $V_{\sfss'}$. Then
\[
   W_{\sfss, \sfss'}=\left\{\begin{bmatrix}
        b_1\\
        b_2
      \end{bmatrix} \mid b_1\in  \Hom_\C(V_\sfss, V_1'), \ b_2\in  \Hom_\C(V_\sfss, V_2')\right\} \qquad\textrm{and}\qquad \phi=\begin{bmatrix}
        \phi_1\\
        0
      \end{bmatrix},
\]
where $\phi_1\in \Hom_\C(V_\sfss, V_1')$ is a surjective linear map. 

Using the decomposition \eqref{v12}, we also have that 
\[
 \g_{\sfss'}:= \Set{\begin{bmatrix}
        a_1 & -\psi^*\\
        \psi & a_2
      \end{bmatrix}\mid a_1\in \g_1', \, a_2\in \g_2', \,\psi \in \Hom_\C(V_1', V_2')},
\]
where $\g_1'$ and $\g_2'$ are respectively the Lie algebras of the isometry groups of $V_1'$ and $V_2'$. Here and henceforth, the adjoint operation $\psi\mapsto \psi^*$ is defined as in 
\eqref{ajoint4} and \eqref{adjointmap}. 

Note that 
\[
  \mathbf e'=\begin{bmatrix}
        \mathbf e'_1 & 0\\
        0 & 0
      \end{bmatrix}, \qquad \textrm{where  } \, \mathsf e_1':=\phi_1 \phi_1^*\in \g_1',%\qquad (\textrm{$\phi_1^*$ is defined as in \eqref{ajoint4}}),
\]
and 
\[
 \g_{\mathbf e'}:= \Set{\begin{bmatrix}
        a_1 & -\psi^*\\
        \psi & a_2
      \end{bmatrix}\mid a_1\in \g_{\mathbf e_1'}, \, a_2\in \g_2', \,\psi \in \Hom_\C(V_1', V_2')^{\mathbf e_1'}},
\]
where $\g_{\mathbf e_1'}$ denotes the centralizer of $\mathbf e_1'$ in $\g_1'$, and 
\[
   \Hom_\C(V_1', V_2')^{\mathbf e_1'}:=\{\psi\in  \Hom_\C(V_1', V_2')\mid \psi  {\mathbf e_1'}=0\}.
\]


The two maps in \eqref{eq:CGG} are respectively given by
\[
  \left(a, \begin{bmatrix}
        a_1 & -\psi^*\\
        \psi & a_2
      \end{bmatrix}\right)\mapsto   \begin{bmatrix}
        a_1 \phi_1-\phi_1 a\\
        \psi \phi_1
      \end{bmatrix}
\]
and
\[
  \begin{bmatrix}
       b_1\\
       b_2
      \end{bmatrix}\mapsto   \begin{bmatrix}
       b_1 \phi_1^*+\phi_1 b_1^*  & \phi_1 b_2^*\\
        b_2 \phi_1^* & 0
      \end{bmatrix}.
\]
In order to prove the lemma, it suffices to show that the sequences 
 \[
  \fggs\times  \g_{\mathbf e_1'} \xrightarrow{(a,a_1)\mapsto  a_1 \phi_1-\phi_1 a } \Hom_\C(V_\sfss, V_1')
  \xrightarrow{b_1\mapsto  b_1\phi_1^{*}+\phi_1 b_1^{*} } \g_1'
  \]
and 
\be\label{exacthom}
  \Hom_\C(V_1', V_2')^{\mathbf e_1'} \xrightarrow{\psi \mapsto  \psi \phi_1 } \Hom_\C(V_\sfss, V_2')
  \xrightarrow{b_2\mapsto  b_2\phi_1^{*}}  \Hom_\C(V_1', V_2') 
\ee
are both exact. Lemma \ref{lemexact1} implies that the first sequence is exact. 

Note that the Young diagram of the nilpotent orbit in $\g_1'$ containing $\mathbf e_1'$ equals $\nabla_{\mathrm{naive}}(\CO)$. Thus 
\[
  \dim (V_1'/\mathbf e_1'(V_1'))=\mathbf c_1(\nabla_{\mathrm{naive}}(\CO))=\mathbf c_2(\CO). 
\]
On the other hand, 
\begin{eqnarray*}
  \dim (V_{\sfss}/\phi_1^*(V_1'))&=&\dim (V_{\sfss})-\dim (\phi(V_{\sfss}))\\
  &=&\dim(\Ker(\phi))\\
  &=&\dim(\Ker(\phi^*\phi))\\
  &=&\mathbf c_1(\CO). 
\end{eqnarray*}
The first map of \eqref{exacthom} is injective since $\phi_1$ is surjective. Then we have that  
\begin{eqnarray*}
     && \dim ( \Hom_\C(V_1', V_2')^{\mathbf e_1'} ) \\
      &=& \dim V_2' \cdot \dim (V_1'/\mathbf e_1'(V_1'))\\
      & = &\dim V_2' \cdot  \bfcc_{2}(\cO)\\
      & = &\dim V_2' \cdot  \bfcc_{1}(\cO)\\ 
        & = &\dim V_2' \cdot  \dim (V_{\sfss}/\phi_1^*(V_1'))\\ 
        &=& \textrm{dimension of the kernel of the second map of  \eqref{exacthom}}.
    \end{eqnarray*}
Therefore the sequence \eqref{exacthom} is exact since the composition of  \eqref{exacthom} is the zero map. This finishes the proof the lemma. 

\end{proof}

\subsection{A proof of Proposition \ref{propreduced}} \label{secpp}
%We are aimed to prove Proposition \ref{propreduced} in this subsection.
Now we suppose that $\mathbf e'\in \CO'\cap \p_{\sfss'}$. Denote by $K_{\mathbf e'}$ the stabilizer of $\mathbf e'$ in $K_{\sfss',\C}$. Consider the map
\[
 M':=  M_{\sfss'}|_{\CX_{\sfss, \sfss'}^{\bar \partial}} : \CX_{\sfss, \sfss'}^{\bar \partial}\rightarrow \p_{\sfss'}. 
\]
As before, its fibre at $\mathbf e'$ equals the variety
\[
\CX_{\sfss, \sfss'}^{\CO, \mathbf e'}:=M_{\sfss}^{-1}(\CO\cap \p_\sfss)\cap  M_{\sfss'}^{-1}(\mathbf e').
\]
This variety is a finite union of open and closed 
$K_{\sfss,\C}\times K_{\mathbf e'}$-orbits: 
\[
   \CX_{\sfss, \sfss'}^{\CO, \mathbf e'}=\bigsqcup_{\sO\textrm{ is a   $K_{\sfss, \C}$-orbit in $ \CO\cap \p_\sfss$ such that $\sO\subset M_\sfss(\CX_{\sfss, \sfss'})$ and $\mathbf e'\in \nabla^\sfss_{\sfss'}(\sO)$}}  \CX_{\sfss, \sfss'}^{\sO, \mathbf e'},
 \]
 where
 \[
   \CX_{\sfss, \sfss'}^{\sO, \mathbf e'}:=M_\sfss^{-1}(\sO)\cap \CX_{\sfss, \sfss'}^{\CO, \mathbf e'}. 
 \]
Thus  $\CX_{\sfss, \sfss'}^{\CO, \mathbf e'}$ is a smooth closed subvariety of $\CX_{\sfss, \sfss'}^{\bar \partial}$. 

\begin{lem}\label{jacobic22}
Suppose that  $\phi\in \CX_{\sfss, \sfss'}^{\CO, \mathbf e'}$. Then the sequence 
\be\label{jc111}
    \mathrm T_\phi (\CX_{\sfss, \sfss'}^{\CO, \mathbf e'})\xrightarrow{\textrm{ inclusion}} \mathrm T_\phi(\CX_{\sfss, \sfss'}^{\bar \partial})\xrightarrow{\textrm{the differential of $M'$}} \mathrm T_{\mathbf e'}(\p_{\sfss'})
\ee
is exact. 
\end{lem}
\begin{proof}
Recall $\dot \epsilon\in \{\pm 1\}$ form \eqref{epsilond}. We have commutative diagrams
\[
 \begin{CD} 
 \fggs \times  \g_{\mathbf e'} @>  \qquad  \textrm{\eqref{differ2} }\qquad  >>  W_{\sfss, \sfss'}
  @> \qquad  \textrm{\eqref{differ1}} \qquad  >>\fggsp\\
  @V(a,\,a') \mapsto  (L_\sfss\circ a \circ L_{\sfss}^{-1},\,  L_{\sfss'} \circ a' \circ L_{\sfss'}^{-1}) VV  @V b\mapsto - \dot \epsilon \sqrt{-1} L_{\sfss'}\circ b \circ L_\sfss^{-1} VV 
  @V a' \mapsto - L_{\sfss'} \circ a' \circ L_{\sfss'}^{-1} VV \\
\fggs \times  \g_{\mathbf e'} @> \qquad   \textrm{\eqref{differ2} }\qquad  >>  W_{\sfss, \sfss'}
  @>\qquad  \textrm{\eqref{differ1}} \qquad  >> \fggsp.
\end{CD}
\]
Lemmas \ref{lemexact1} and \ref{lemexact2} imply that the horizontal sequences are exact. The lemma then follows by taking the fixed points of the vertical arrows. 

\end{proof}

Lemmas \ref{jacobic} and \ref{jacobic22} implies that the scheme theoretic fibre product $\CX_{\sfss, \sfss'}^{\bar \partial}\times_{\p_{\sfss'}} \{\mathbf e'\}$ is reduced. 
In other words, 
\[
\left(\CX_{\sfss, \sfss'}^{\bar \partial}\times_{\p_{\sfss'}} ({\CO'}\cap \p_{\sfss'})\right)\times_{\CO'\cap \p_{\sfss'}}\{\mathbf e'\}
\]
is reduced. 
Since $\mathbf e'\in \CO\cap \p_{\sfss'}$ is arbitrary, this further implies that  the scheme theoretic  fibre product
\[
\CX_{\sfss, \sfss'}^{\bar \partial}\times_{\p_{\sfss'}} ({\CO'}\cap \p_{\sfss'})
\]
is reduced (see \cite[Proposition~11.3.13]{EGAIV3} and \cite[Th\'eor\`eme~6.9.1]{EGAIV2}). Proposition  \ref{propreduced} is now proved. 


\delete{
We now recall the following lemma in EGA.
\begin{lem}[{\cite[Proposition~11.3.13]{EGAIV3}}] \label{lem:red}
  Suppose $f\colon X\rightarrow Y$ is a morphism between schemes, $x$ is a point
  in $X$ such that $f$ is flat at $x$. Let $y=f(x)$.
  \begin{enuma}
    \item  If $X$ is reduced at the point $x$, then $Y$ is reduced at $y$.
    \item We assume $f$ is of finite presentation at the point $x$.
    If $Y$ is reduced at $x$ and $x$ is reduced at the scheme theoretical fiber
    $X_{y}$, then $X$ is reduced at $x$. \qed
  \end{enuma}
\end{lem}

\begin{proof}[Proof of \Cref{propreduced}]
  Consider the $\Ks\times \Ksp$-equivariant morphism
  \[
    \CXUO:=\CX_{\sfss, \sfss'}^{\bar \partial}\times_{\p_{\sfss'}} ({\CO'}\cap \p_{\sfss'}) \longrightarrow  \cO'\cap \fppsp
  \]
  Since $\cO'\cap \fppsp$
  is a finite union of $\Ksp$-orbits, the above morphism is flat  by generic
  flatness \cite[Th\'eor\`eme~6.9.1]{EGAIV2}.
  By \Cref{lem:GG}, $\CXUO \times_{\fppsp} \Xp$ is regular at every point $\ww$ since
  it is a open subscheme of $\CXss\times_{\fppsp}\Xp $ and $\ww\in \CXOO$.
  Clearly $\cOp\cap \fppsp$ is reduced. We conclude that $\CXUO$ is reduced by \Cref{lem:red}.
\end{proof}


Using the same argument we also have the following lemma which we will not use in the
paper.
\begin{lem}\label{lem:RDS.C}
  Suppose $\cO$ regular for $\DDss$.
  Then
  \[
    \Wss\times_{\fggs\times \fggsp} \big( (\fggs - \partial \cO)\times \cOp\big)
  \]
  is a reduced scheme. \qed
\end{lem}

}

\trivial[h]{
%   Note that $\UU\times \cOp \rightarrow \UU \times \bcOp$ is \'etale, so
%   $W_{\UU,\cOp}\rightarrow W_{\UU,\bcOp}$.
  Generic flatness: Suppose $f: X\rightarrow S $ is a morphism between scheme
  and $\cF$ is a quasi-coherent sheaf of $\cO_{X}$ module, such that (1) $S$ is
  reduced, (2) $f$ is of finite type, and (3) $\cF$ is a finite type
  $\cO_{X}$-module.
Then there is a open dense subscheme $U\subset S$ such that
$X_{U}\rightarrow U $
is flat and of finite presentation, and $\cF|_{X_U}$ is flat over $U$
and of finite presentation over $\cO_{X_{U}}$.
}

\subsection{Geometric theta lifts of admissible orbit data} 

\def\wedgetop{{\bigwedge}^{\mathrm{top}}}

  % Let $S_{\ww} := \Ks\times \Ksp$, $\fss_{\ww} := \Lie(S_{\ww})$,
  % $S_{\wwo} := \Stab_{\Ks\times \Kspo}(\wwo)$ and $\fss_{\wwo} := \Lie(S_{\wwo})$.

\medskip

\DeclareDocumentCommand{\dliftv}{O{\sfss'} O{\sfss}}{
  {{\check \vartheta}_{#1}^{#2}}}
\def\dlift{{\check\vartheta}}
\def\AOD{\mathrm{AOD}}

Recall that  $\CO$ is regular for $\DDss$. Suppose that $\phi\in \CXOO$, $\mathbf e=M_\sfss(\phi)$ and $\mathbf e'=M_{\sfss'}(\phi)$. As before, $K_\mathbf e$ and $K_{\mathbf e'}$ are respectively the  stabilizers of $\mathbf e$ and $\mathbf e'$ in $K_{\sfss, \C}$ and $K_{\sfss', \C}$.  Write $\mathfrak k_{\mathbf e}$ and $\mathfrak k_{\mathbf e'}$ for the Lie algebras of $K_{\mathbf e}$ and $K_{\mathbf e'}$, respectively. As before, $(K_{\sfss, \C}\times K_{\sfss', \C})_{\ww}$ denotes the stabilizer of $\phi$ in $K_{\sfss, \C}\times K_{\sfss', \C}$. 
   Write $\mathfrak s_\phi$ for the Lie algebra of  $(\Ks\times \Ksp)_{\ww}$.
  
\begin{lem}\label{lem:tan}
 As representations of  $\mathfrak s_\phi$, 
  \begin{equation}\label{eq:aod}
    \wedgetop \mathfrak k_{\X} \otimes \wedgetop \CXss  \cong \wedgetop \mathfrak k_{\Xp}.
  \end{equation}
\end{lem}
\begin{proof}
  We retain the notation in the proof of \Cref{lemexact2}. Since $\phi\in \CX_{\sfss, \sfss'}$, both $V_1'$ and $V_2'$ are $L_{\sfss'}$-stable.  Write
  \[
    \g_i'=\mathfrak k_i'\oplus \p_i'\qquad (i=1,2),
  \]
  where $\mathfrak k_i'$ is the centralizer of $L_{\sfss'}$ in $\g_i'$, and $\p_i'$ is its orthogonal complement in $\g_i'$ under the trace form. 

   The sequence \eqref{jc111} leads to a short exact sequence
  \[
    0\longrightarrow (\fkks\oplus \mathfrak k_{\mathbf e'})/ \fss_{\ww}
    \longrightarrow \CXss \longrightarrow \p_{\sfss'}/\p_2'  \longrightarrow 0
  \]
  of representations of $(\Ks\times \Ksp)_{\ww}$. 
By 
Lemma \ref{descko}, we have an exact sequence 
\[
  0\rightarrow \mathfrak k_2'\rightarrow \mathfrak s_\phi\rightarrow \mathfrak k_{\mathbf e}\rightarrow 1
\]
of representations of $(\Ks\times \Ksp)_{\ww}$. Therefore, as  representations of $(\Ks\times \Ksp)_{\ww}$, we have that
\begin{eqnarray*}
      &&\wedgetop \mathfrak k_{\X} \otimes \wedgetop \CXss\\
      & \cong & \wedgetop (\p_{\sfss'}/\p_2')  \otimes \wedgetop (\fkks\oplus \mathfrak k_{\mathbf e'}) \otimes
      \left(\wedgetop \mathfrak k_2'\right)^{-1}\\
       & \cong & \wedgetop (\p_{\sfss'}/\p_2')  \otimes  \wedgetop \fkks \otimes
      \left(\wedgetop \mathfrak k_2'\right)^{-1}\otimes \wedgetop  \mathfrak k_{\mathbf e'}. \\
\end{eqnarray*}
The lemma then follows since $\mathfrak s_\phi$ acts trivially on $ \wedgetop (\p_{\sfss'}/\p_2')  \otimes  \wedgetop \fkks \otimes
      \left(\wedgetop \mathfrak k_2'\right)^{-1}$. 
      
      \end{proof}



\begin{lem}\label{lem:aod}
  Suppose that $\sO$ is a $K_{\sfss,\C}$-orbit in $\cO\cap \fpps$ that is contained in the image of $M_\sfss$,  and
  $\sOp = \DDss(\sO)\subset \cOp\cap \fppsp$.
  Then for every $\cE' \in \AOD_{\sfss'}(\sOp)$, 
   the geometric theta lift $\dlift_{\sO'}^{\sO}(\cE')$ is either zero or an element of  $\AOD_{\sfss}(\sO)$.
\end{lem}
\begin{proof}
  Note that $(\zeta_{\sfss,\sfss'})^{2} \cong (\wedgetop \CXss)^{-1}$ as representations of $\mathfrak k_\sfss\times \mathfrak k_{\sfss'}$, and $\cE'$ is represented by a line bundle. Thus  
   the lemma is a direct consequence of  \Cref{lem:tan}.
\end{proof}

\section{Equating  the associated  cycles and a proof of Theorem \ref{thmpitau}}\label{sec:equac}



As before, $\mathsf s=(\star, p,q)$ and $ \mathsf s'=(\star', p',q')$ are classical signatures such that $\star'$ is the Howe dual of $\star$.


\subsection{Good descents and doubling method}


\begin{defn}
An nilpotent orbit $\CO\in \mathrm{Nil}(\g_\mathsf s)$ is good for $\DD_{\mathsf s'}^{\mathsf s}$ if it is regular for $\DD_{\mathsf s'}^{\mathsf s}$ and satisfies the following  condition:
\[
\begin{cases}
   \mathbf c_1(\CO)>\mathbf c_2(\CO), \qquad  &\textrm{if $\star \in\{B,D\}$}; \\
      \abs{\sfss'}-\abs{\DD_\mathrm{naive}(\CO)}\in\{0,1\},\qquad  &\textrm{if $\star \in\{C, \widetilde C\}$}; \\
\abs{\sfss'}=\abs{\DD_\mathrm{naive}(\CO)}, \qquad &\textrm{if $\star \in \{C^*, D^*\}$}.
  \end{cases}
\]


\end{defn}

In the rest of this section we suppose that $\CO\in \mathrm{Nil}(\g_\sfss)$ is good for $\DD_{\mathsf s'}^{\mathsf s}$. Put $\CO':=\DD_{\mathsf s'}^{\mathsf s}(\CO)$ as before. Write
 \[
  p_0:=q_0:=\begin{cases}
    \abs{\sfss}-\abs{\sfss'}-1, \qquad  &\textrm{if $\star \in\{B,D\}$}; \\
       \abs{\sfss}-\abs{\sfss'}+1, \qquad  &\textrm{if $\star \in\{C, \widetilde C\}$}; \\
 \abs{\sfss}-\abs{\sfss'},\qquad  &\textrm{if $\star \in \{C^*, D^*\}$},
  \end{cases}
\]
which is a non-negative integer.
Similar to \eqref{p123}, we have that 
\be\label{p1234}
-p_0-1=
  \begin{cases}
 \nu_{\sfss'}-\abs{\sfss} +2,\quad& \textrm{if $\star =C^*$};\\
  \nu_{\sfss'}-\abs{\sfss},\quad& \textrm{otherwise}.
   \end{cases}
   \ee
Thus    $\pi'$ is convergent for $\check \Theta_{\mathsf s'}^{\mathsf s}$ whenever $\pi'$ is a Casselman-Wallach representation of $G_{\sfss'}$  that is  $\nu'$-bounded for some
$
  \nu'>
 -p_0-1.
$

We will prove the following theorem in this section. 
\begin{thm}\label{thm:GDS.AC}
  Let $\pi'$ be an $\CO'$-bounded Casselman-Wallach representation of $G_{\sfss'}$  that is  $\nu'$-bounded for some
$
  \nu'>
 -p_0-1.
$
%Let $\rho'$ be the   $(\g_{\mathsf s'}, K_{\mathsf s'})$-module associated to $\pi'$.  
Then $\Thetab_{\mathsf s'}^{\mathsf s}(\pi')$, $\check \Theta_{\mathsf s'}^{\mathsf s}(\pi')$  and $\check \Theta_{\mathsf s'}^{\mathsf s}(\pi'^{\mathrm{alg}})$ are all $\CO$-bounded,  and
    \[
  \mathrm{AC}_{\cO}(\Thetab_{\mathsf s'}^{\mathsf s}(\pi'))=    \mathrm{AC}_{\cO}(\check \Theta_{\mathsf s'}^{\mathsf s}(\pi'))= \mathrm{AC}_{\cO}(\check \Theta_{\mathsf s'}^{\mathsf s}(\pi'^{\mathrm{alg}}))=   \check \vartheta_{\cO'}^\cO(\mathrm{AC}_{\cO'}(\pi'))\in \CK_\sfss(\CO).
  \]
\end{thm}

%We remark that by Lemma \ref{doublelift5}, $\pi'$ is convergent for $\check \Theta_{\mathsf s'}^{\mathsf s}$ so that $\Thetab_{\mathsf s'}^{\mathsf s}(\pi')$ is defined. 



%Theorem \ref{thm:GDS.AC} is obviously true when $\abs{\sfss'}=0$. In the rest of this subsection we further assume that $\abs{\sfss'}>0$. 
  


  Define three classical signatures
\[
  \dsfss:=(\dot \star, \dot p, \dot q):=\begin{cases}
    (\star', \abs{\sfss'}+p_0,  \abs{\sfss'}+q_0), \qquad  &\textrm{if $\star'\neq B$}; \\
        (D, \abs{\sfss'}+p_0,  \abs{\sfss'}+q_0),\qquad  &\textrm{if $\star'=B$},
  \end{cases}
\]
\[
  \sfss'':=(\star', p'+p_0, q'+q_0)\qquad\textrm{and}\qquad \sfss_0:=(\dot \star, p_0, q_0). %:=( \star'', p'', q'')
  \]
Then we have decompositions 
\[
  V_{\dsfss}=V_{\sfss'^-}\oplus V_{\sfss''}= V_{\sfss'^-}\oplus V_{\sfss'}\oplus  V_{\sfss_0}= (V_{\sfss'}^\triangle \oplus X_{\sfss_0})\oplus (V_{\sfss'}^\nabla \oplus Y_{\sfss_0}).
\]
as in \eqref{decomv0}. 

Theorem \ref{thm:GDS.AC} is obviously true when $\abs{\sfss}\leq 1$. Thus  we  assume in the rest of this subsection that $\abs{\sfss}\geq 2$. As in \eqref{dotp}, we have that 
\[
 \dot p=\dot q= \begin{cases}
 \abs{\sfss}-1,\quad   &\textrm{if } \star\in \{B,D\};\\
 \abs{\sfss}+1,  & \textrm{if $\star\in \{C, \widetilde C\}$;}\\
 \abs{\sfss} & \textrm{if $ \star\in\{C^*,D^*\}$.}
\end{cases}
\]
Thus we are in the situation of 
Section \ref{doublep} (with $\abs{\sfss}$ equals the integer $k$ of Section \ref{doublep}).

Let 
\[
  P_{\sfss'',\sfss_0}=R_{\sfss'',\sfss_0}\ltimes N_{\sfss'',\sfss_0}=(G_{\sfss'}\cdot R_{\sfss_0})\ltimes N_{\sfss'',\sfss_0}
\]
be the parabolic subgroup of $G_{\sfss''}$ as in \eqref{parabolic} and \eqref{parabolic2}. Write $\mathfrak r_{\sfss_0}$ and $\mathfrak n_{\sfss'',\sfss_0}$ for the complexified Lie algebras of $R_{\sfss_0}$ and $N_{\sfss'',\sfss_0}$ respectively so that  the complexified Lie algebra of $P_{\sfss'', \sfss_0}$ equals $(\g_{\sfss'}\times \mathfrak r_{\sfss_0})\ltimes \mathfrak n_{\sfss'',\sfss_0}$.   
Define
\[
  \CO'':= \Ind_{P_{\sfss'',\sfss_0}}^{G_{\sfss''}}  \CO'%  \Ind_{\g_{\sfss'}\times \mathfrak r_{\sfss_0}}^{\g_{\sfss''}}(\CO'\times \{0\}).
\]
to be the unique nilpotent orbit in $\mathrm{Nil}(\g_{\sfss''})$ that contains a non-empty Zariski open subset of  $
  \CO'+\mathfrak n_{\sfss'',\sfss_0}$ (see \cite[Theorem 7.1.1]{CM}).


\begin{lem}
The nilpotent orbit $\CO''$ is good for  $\nabla_\mathsf s^{\sfss''}$ and $\nabla_\mathsf s^{\sfss''}(\CO'')=\CO$.  
\end{lem}
\begin{proof}
This follows from the description of induced orbits in \cite[Section 7.3]{CM}. 
\end{proof}


\begin{lem}\label{lem74}
Suppose that $\CE\in \CK_{\sfss}(\CO)$ and $0\preceq \CE\preceq \check \vartheta_{\sfss', \CO'}^{\sfss, \CO}(\CE')$ for some $\CE'\in \CK^+_{\sfss'}(\CO')$. If $\CE\neq 0$, then  $\check \vartheta_{\sfss, \CO}^{\sfss'', \CO''}(\CE)\neq 0$.
\end{lem}
\begin{proof}
  Without loss of generality we may assume that $\cE'$ is irreducible and
  supported on a nilpotent orbit $\sO'\in \Nil_{\sfss'}(\cO')$. 
  We also can assume that $\cE$ is supported on a nilpotent orbit $\sO\in \Nil_\sfss(\cO)$. 
  By the definition of geometric lift, $\sO'=\DDss(\sO)$. Using \Cref{prop:DDss}, 
  one can check that there is a nilpotent orbit $\sO''\in \Nil_{\sfss''}(\cO'')$ such that 
  $\sO = \DD_{\sfss}^{\sfss''}(\sO'')$. 
  When $\cO = \DDn(\cO'')$, the non-vanishing of $\check \vartheta_{\sfss, \CO}^{\sfss'', \CO''}(\cE)$
  follows from the inejctivity of $\dlift_{\sO}^{\sO''}$. 
  When $\abs{\cO} > \abs{\DDn(\cO'')}$, we have $\star \in \set{B,D}$, $\sO'=\DDn(\sO)$
  and so $\cE = \dlift_{\sO'}^{\sO}(\cE')$.  
  Now is straightforward to check that $\vartheta_{\sfss,\sO}^{\sfss,\sO''}(\cE) \neq 0$.
\end{proof}

\subsection{Weak associated cycles}

For every classical signature $\sfss_1$ and every $\CO_1\in \mathrm{Nil}(\g_{\sfss_1})$, write $\CK^{\mathrm{weak}}_{\sfss_1}(\CO_1)$
for the free abelian group generated by the set of $K_{\sfss_1}$-orbits in $\CO_1\cap \p_{\sfss_1}$. Define a homomorphism
\[
  \CK_{\sfss_1}(\CO_1)\rightarrow \CK^{\mathrm{weak}}_{\sfss_1}(\CO_1), \qquad \CE\mapsto \CE^{\mathrm{weak}}
\]
such that if $\CE$ is represented by a $K_{\sfss_1,\C}$-equivariant algebraic vector bundle $\CE_0$ over a $K_{\sfss_1,\C}$-orbit $\sO_1\subset \CO_1\cap \p_{\sfss_1}$, then
\[
   \CE^{\mathrm{weak}}=\textrm{(rank of $\CE_0$)}\cdot \sO_1. 
\]

For all $\sfss_1\in S_{\star, \abs{\sfss}}$,  $\g_{\sfss_1}$ is identified with $\g_{\sfss}$ and we consider $\CO$ also as an element of $\mathrm{Nil}(\g_{\sfss_1})$. 


Note that for every Cassleman-Wallach representation $\pi_1$ of $R_{\sfss'',\sfss_0}=G_{\sfss'}\cdot R_{\sfss_0}$ such that $(\pi_1)|_{G_{\sfss'}}$ is an $\CO'$-bounded Casselman-Wallach representation,   the parabolic induction
$\Ind_{P_{\sfss'',\sfss_0}}^{G_{\sfss''}} \pi_1$ is an $\CO''$-bounded Casselman-Wallach representation  of $G_{\sfss''}$ (see \cite[Corollary~5.0.10]{B.Orbit}).




\begin{prop}\label{prpaseq}
Let   $\pi'$ be an $\CO'$-bounded Casselman-Wallach representation of $G_{\sfss'}$, and let $\chi_1: R_{\sfss_0}\rightarrow \C^\times$ be a character. 
Assume that $\pi'$ is genuine when $\star'=\widetilde C$. 

\noindent
(a) If $ \star\in \{B,D\}$,  then
\[
    \bigoplus_{\sfss_1\in S_{\star, \abs{\sfss}}}  \left(\check \vartheta^{\sfss'', \CO''}_{\sfss_1, \CO}(\check \vartheta^{\sfss_1, \CO}_{\sfss',\CO'}(\mathrm{AC}_{\CO'}(\pi')))\right)^{\mathrm{weak}}=2\cdot \left( \mathrm{AC}_{\CO'}(\Ind_{P_{\sfss'',\sfss_0}}^{G_{\sfss''}} (\pi'\otimes \chi_1))\right)^{\mathrm{weak}}.
        \]

  \smallskip
  

    \noindent
(b) If $\star \in \{C, \widetilde C\}$, then
\begin{eqnarray*}
   && \left(\check \vartheta^{\sfss'', \CO''}_{\sfss, \CO}(\check \vartheta^{\sfss, \CO}_{\sfss',\CO'}(\mathrm{AC}_{\CO'}(\pi')))\right)^{\mathrm{weak}} +  \left(\check \vartheta^{\sfss'', \CO''}_{\sfss, \CO}(\check \vartheta^{\sfss, \CO}_{\sfss',\CO'}(\mathrm{AC}_{\CO'}(\pi'\otimes \det )))\right)^{\mathrm{weak}}\\
   & =&\left( \mathrm{AC}_{\CO''}(\Ind_{P_{\sfss'',\sfss_0}}^{G_{\sfss''}} (\pi'\otimes \chi_1))\right)^{\mathrm{weak}}. 
\end{eqnarray*}
  \smallskip

\noindent
(c) If $\star=D^*$, then
\[
 \left(\check \vartheta^{\sfss'', \CO''}_{\sfss, \CO}(\check \vartheta^{\sfss, \CO}_{\sfss',\CO'}(\mathrm{AC}_{\CO'}(\pi')))\right)^{\mathrm{weak}} = \left( \mathrm{AC}_{\CO''}(\Ind_{P_{\sfss'',\sfss_0}}^{G_{\sfss''}} (\pi'\otimes \chi_1))\right)^{\mathrm{weak}}.
 \]
 
\smallskip

\noindent
(d) If $\star=C^*$, then 
\[
  \bigoplus_{\sfss_1\in S_{\star, \abs{\sfss}} } \left(\check \vartheta^{\sfss'', \CO''}_{\sfss_1, \CO}(\check \vartheta^{\sfss_1, \CO}_{\sfss',\CO'}(\mathrm{AC}_{\CO'}(\pi')))\right)^{\mathrm{weak}}=\left( \mathrm{AC}_{\CO'}(\Ind_{P_{\sfss'',\sfss_0}}^{G_{\sfss''}} (\pi'\otimes \chi_1))\right)^{\mathrm{weak}}.
\]
\end{prop}
\begin{proof}
  The proposition follows by comaparing \Cref{lem:ind} with the direct computations of the left hand side of the equations.
\end{proof}


% As in Theorem \ref{thm:GDS.AC}, 
\begin{lem}\label{lem:GDS.AC2}
  Let $\pi'$ be an $\CO'$-bounded Casselman-Wallach representation of $G_{\sfss'}$  that is  $\nu'$-bounded for some
$
  \nu'>
 -p_0-1.
$
Then 
 \begin{eqnarray*}
  && \left( \bigoplus_{\sfss_1\in S_{\star, \abs{\sfss}} } \mathrm{AC}_{\cO''}( \Thetab^{\mathsf s''}_{\mathsf s_1}(\Thetab_{\mathsf s'}^{\mathsf s_1}(\pi')))\right)^{\mathrm{weak}} 
\\
&=& \left( \bigoplus_{\sfss_1\in S_{\star, \abs{\sfss}}}   \vartheta^{\sfss'', \CO''}_{\sfss_1, \CO}(\vartheta^{\sfss_1, \CO}_{\sfss',\CO'}(\mathrm{AC}_{\CO'}(\pi')))\right)^{\mathrm{weak}}.        
\end{eqnarray*}

\end{lem}
\begin{proof}
There is no loss of generality to assume that $\pi'$ is genuine when $\star'=\widetilde C$. When $\star\neq C^*$, the lemma is a direct consequence of Theorem \ref{doubtt} and Proposition \ref{prpaseq}. 

Now assume that $\star=C^*$.  Let $\CJ$ be as in Theorem \ref{doubtt}. In view of Lemma \ref{comobound}, and using formulas in  \cite[Theorems 5.2 and 5.6]{DKPC}, one checks that $\CJ$ is $\CO_2$-bounded for a certain oribit $\CO_2\in \mathrm{Nil}(\g_{\sfss''})$ that is contained in the boundary $\overline{\CO''}\setminus \CO''$ of  $\CO''$. Thus 
$\CJ$ is $\cO''$-bounded, and $ \mathrm{AC}_{\cO''}(\CJ)=0$. Hence  the  lemma in the case when $\star=C^*$  is also a consequence of Theorem \ref{doubtt} and Proposition \ref{prpaseq}. 

\end{proof}


\subsection{A proof of Theorem \ref{thm:GDS.AC}}
%\begin{proof}[Proof of Theorem \ref{thm:GDS.AC}] 
 We are now ready to prove Theorem \ref{thm:GDS.AC}.
Let $\pi'$ be as in Theorem \ref{thm:GDS.AC}. 
 %First assume that $\star=B$. 
 %The proof in the other cases is similar and is omitted. 
% Without loss of generality we assume that $\pi'$ is genuine.
For every $\sfss_1\in S_{\star, \abs{s}}$ we have 
surjective $(\g_{\sfss_1}, K_{\sfss_1})$-module homomorphisms
\[
\check \Theta_{\mathsf s'}^{\mathsf s_1}(\pi'^{\mathrm{alg}})\rightarrow \left (\check \Theta_{\mathsf s'}^{\mathsf s_1}(\pi')\right )^{\mathrm{alg}}\rightarrow 
 \left (\Thetab_{\mathsf s'}^{\mathsf s_1}(\pi')\right )^{\mathrm{alg}}. 
 \]
 Thus Theorem \ref{prop:GDS.AC} implies that  $\check \Theta_{\mathsf s'}^{\mathsf s_1}(\rho')$, $\check \Theta_{\mathsf s'}^{\mathsf s_1}(\pi')$ and $\Thetab_{\mathsf s'}^{\mathsf s_1}(\pi')$ are all $\CO$-bounded,  and
    \be\label{ineq000}
   \mathrm{AC}_{\cO}( \Thetab_{\mathsf s'}^{\mathsf s_1}(\pi'))\preceq  \mathrm{AC}_{\cO}(\check \Theta_{\mathsf s'}^{\mathsf s_1}(\pi'))\preceq  \mathrm{AC}_{\cO}(\check \Theta_{\mathsf s'}^{\mathsf s_1}(\rho'))\preceq \check \vartheta_{\sfss', \cO'}^{\sfss_1, \cO}(\mathrm{AC}_{\cO'}(\pi')).
  \ee
  Consequently, 
   \be\label{acco}
   \mathrm{AC}_{\cO}( \Thetab_{\mathsf s'}^{\mathsf s_1}(\pi'))\preceq   \check \vartheta_{\sfss', \cO'}^{\sfss_1, \cO}(\mathrm{AC}_{\cO'}(\pi')).
  \ee

By Lemma \ref{doublelift5},  $ \Thetab_{\mathsf s'}^{\mathsf s_1}(\pi')$ is over-convergent for $ \check \Theta_{\mathsf s_1}^{\mathsf s''}$ so that $\Thetab_{\mathsf s_1}^{\mathsf s''}( \Thetab_{\mathsf s'}^{\mathsf s_1}(\pi'))$ is defined.  Similar to \eqref{acco}, we have that
   \be\label{acco2}
   \mathrm{AC}_{\cO''}( \Thetab_{\mathsf s_1}^{\mathsf s''}(\Thetab_{\mathsf s'}^{\mathsf s_1}(\pi')))\preceq   \check \vartheta_{\sfss_1, \cO}^{\sfss'', \cO''}(\mathrm{AC}_{\cO}( \Thetab_{\mathsf s'}^{\mathsf s_1}(\pi'))).
  \ee
Combining \eqref{acco} and \eqref{acco2}, we obtain that
\be\label{acineq}
   \mathrm{AC}_{\cO''}( \Thetab_{\mathsf s_1}^{\mathsf s''}(\Thetab_{\mathsf s'}^{\mathsf s_1}(\pi')))\preceq   \check \vartheta_{\sfss_1, \cO}^{\sfss'', \cO''}(\check \vartheta_{\sfss', \cO'}^{\sfss_1, \cO}(\mathrm{AC}_{\cO'}(\pi'))).
\ee
Then Lemma \ref{lem:GDS.AC2} implies that the equality in \eqref{acineq} holds. In particular, 
\be\label{acineq2}
   \mathrm{AC}_{\cO''}( \Thetab_{\mathsf s}^{\mathsf s''}(\Thetab_{\mathsf s'}^{\mathsf s}(\pi')))=  \check \vartheta_{\sfss, \cO}^{\sfss'', \cO''}(\check \vartheta_{\sfss', \cO'}^{\sfss, \cO}(\mathrm{AC}_{\cO'}(\pi'))).
\ee

In view of \eqref{acco}, write 
\[
  \check \vartheta_{\sfss', \cO'}^{\sfss, \cO}(\mathrm{AC}_{\cO'}(\pi'))=\mathrm{AC}_{\cO}( \Thetab_{\mathsf s'}^{\mathsf s}(\pi'))+\CE,\qquad\textrm{where $\ \CE\in \CK^+_\sfss(\CO)$}.
\]
Then 
\begin{eqnarray}
\nonumber \check \vartheta_{\sfss, \cO}^{\sfss'', \cO''}(\check \vartheta_{\sfss', \cO'}^{\sfss, \cO}(\mathrm{AC}_{\cO'}(\pi')))
\nonumber &=&\mathrm{AC}_{\cO''}( \Thetab_{\mathsf s}^{\mathsf s''}(\Thetab_{\mathsf s'}^{\mathsf s}(\pi')))\\
\nonumber &\preceq &  \check \vartheta_{\sfss, \cO}^{\sfss'', \cO''}(\mathrm{AC}_{\cO}( \Thetab_{\mathsf s'}^{\mathsf s}(\pi')))\\
\label{prece}  &\preceq &  \check \vartheta_{\sfss, \cO}^{\sfss'', \cO''}(\mathrm{AC}_{\cO}( \Thetab_{\mathsf s'}^{\mathsf s}(\pi'))+\CE)\\
\nonumber &=&\check \vartheta_{\sfss, \cO}^{\sfss'', \cO''}(\check \vartheta_{\sfss', \cO'}^{\sfss, \cO}(\mathrm{AC}_{\cO'}(\pi'))).
\end{eqnarray}
In particular,  the equality holds in \eqref{prece}. Therefore  $\check \vartheta_{\sfss, \cO}^{\sfss'', \cO''}(\CE)=0$, and 
Lemma \ref{lem74} imply that $\CE=0$. This proves Theorem \ref{thm:GDS.AC}. 


%Theorem \ref{prop:GDS.AC} and   In view of \eqref{ineq000}, 

 
 \subsection{A proof of Theorem \ref{thmpitau} }\label{sec:comANDgeo}
  As in Section \ref{secd}, $\check \CO$ is a Young diagram that has $\star$-good parity,   and 
$\uptau=(\tau, \wp)\in \PBPe_\star(\check \CO,\mathsf s)$.  We have the dual descent 
\[
  \check \CO':=\check \nabla(\check \CO), 
\]
and the descent 
 \[
  \uptau' := (\tau',\wp'):=\nabla(\uptau):= (\DD(\tau), \ckDD(\wp))\in \mathrm{\PBPe}_{\star'}(\check \CO').
 \]
 Set
 \[
  \mathrm d'_\tau:=
    \dim \tau'.
    \]

Suppose that $\CO\in \mathrm{Nil}(\g_\sfss)$ equals the Barbasch-Vogan dual of $\check \CO$.   Recall that $\pi_\uptau$ is the Casselman-Wallach representation of $G_{\mathsf s}$, defined in the introductory section. 
Also recall the element $\mathrm{AC}(\uptau)\in \CK_{\mathsf s}(\CO)$ from Section \ref{subsecass}.

Recall the character $\chi(\check \CO): \oU(\g_\sfss)^{G_{\sfss, \C}}\rightarrow \C$  defined in the Introduction.
\begin{lem}\label{inf1}
The algebra $\oU(\g_\sfss)^{G_{\sfss, \C}}$ acts on $\pi_\uptau$ through the character $\chi(\check \CO)$.% as defined in the Introduction. 
\end{lem}
\begin{proof}
By induction, this is implied by \cite[Theorem
1.19]{PrzInf}.
\end{proof}



Recall the ideal $I_{\check \CO}\subset \oU(\g_\sfss)$ from the Introduction. 
\begin{lem}\label{inf2}
Let $\pi$ be an $\CO$-bounded irreducible Casselman-Wallach representation of $G_{\sfss}$ such that $\oU(\g_\sfss)^{G_{\sfss, \C}}$ acts on it  through the character $\chi(\check \CO)$. Then $\mathrm{AC}_{\CO}(\pi)\neq 0$, and  $I_{\check \CO}$ annihilated $\pi$. 
%by   $I_{\check \CO}$ if and only if $\pi$ is $\CO$-bounded.  
\end{lem}
\begin{proof}
This is implied by \cite[Korollar 3.6]{BK} and \cite[Theorem 8.4]{Vo89}. 

\end{proof}


We will prove in the rest of this subsection the following theorem, which, together with Lemmas \ref{inf1} and \ref{inf2},  implies  Theorem \ref{thmpitau}. 
\begin{thm}\label{thmpitau222}
The representation $\pi_\uptau$ is irreducible, unitarizable, $\CO$-bounded and $( \mathrm d'_\tau-\nu_{\sfss})$-bounded.     Moreover,
\[
\mathrm{AC}_\CO(\pi_\uptau)=\mathrm{AC}(\uptau)\in \CK_{\mathsf s}(\CO).
\]
\end{thm}

We prove Theorem \ref{thmpitau222} by induction on $\mathbf c_1(\check \CO)$ (which equals the number of nonempty rows of $\check \CO$).  Theorem \ref{thmpitau222} is obvious when $\mathbf c_1(\check \CO)=0$. Now we assume $\mathbf c_1(\check \CO)>0$ and Theorem \ref{thmpitau222} holds for  smaller $\mathbf c_1(\check \CO)$. 

 \begin{lem}\label{lem78}
 The orbit $\CO$ is good for $\nabla^\sfss_{\sfss_{\tau'}}$, and $\nabla^\sfss_{\sfss_{\tau'}}(\CO)$ equals the Barbasch-Vogan dual  of $\check \CO'$. 
 \end{lem}
\begin{proof}
The first assertion is routine to check, and the second one is Lemma \ref{dualdesc}. %, this is  elementary to check by using Lemma \ref{bvdual}.
\end{proof}

In view of Lemma \ref{lem78}, we assume that $\sfss'=\sfss_{\tau'}$. Then   
\[
\CO':= \nabla^\sfss_{\sfss'}(\CO)=\textrm{the Barbasch-Vogan dual  of $\check \CO'$}.
\]



By the induction hypothesis, the representation $\pi_{\uptau'}$ is irreducible, unitarizable, $\CO'$-bounded and $( \mathrm d'_{\tau'}-\nu_{\sfss'})$-bounded,     and
\[
\mathrm{AC}_\CO(\pi_{\uptau'})=\mathrm{AC}(\uptau')\in \CK_{\mathsf s'}(\CO').
\]

It is routine to check case by case  that 
\[
\mathrm d'_{\tau'}-\nu_{\sfss'}>\max \left \{\nu^\circ_{\sfss'}-\abs{\sfss}, -p_0-1\right \}=
  \begin{cases}
\nu^\circ_{\sfss'}-\abs{\sfss} +1,\quad& \textrm{if $\star =C^*$};\\
 \nu^\circ_{\sfss'}-\abs{\sfss},\quad& \textrm{otherwise}.
   \end{cases}
\]
Thus $\pi_{\uptau'}\otimes \chi$ is over-convergent for $\check \Theta_{\sfss'}^{\sfss}$ for every unitary character $\chi$ of $G_\sfss$. 



Recall from \eqref{eq:def-pi} that 
\[
    \pi_{\uptau}:=\left\{
     \begin{array}{ll}
          %\textrm{the trivial representation $\C$}, &\hbox{if $\abs{\check \CO_\tau}\leq 1$}; \medskip\\
         \check \Theta_{\sfss'}^{\sfss}(\pi_{\uptau'})\otimes (1_{p_\tau, q_\tau}^{+,-})^{\varepsilon_{\tau}}, &\hbox{if  $\star=B$ or $D$}; \\
         \check \Theta_{\sfss'}^{\sfss}(\pi_{\uptau'}\otimes \det^{\varepsilon_{\wp}}), &\hbox{if $\star=C$ or $\widetilde C$}; \\
              \check \Theta_{\sfss'}^{\sfss}(\pi_{\uptau'}), &\hbox{if $\star=C^*$ or $D^*$}. \\
            \end{array}
   \right.
\]
Similarly define
\[
   \bar \pi_{\uptau}:=\left\{
     \begin{array}{ll}
          %\textrm{the trivial representation $\C$}, &\hbox{if $\abs{\check \CO_\tau}\leq 1$}; \medskip\\
         \Thetab_{\sfss'}^{\sfss}(\pi_{\uptau'})\otimes (1_{p_\tau, q_\tau}^{+,-})^{\varepsilon_{\tau}}, &\hbox{if  $\star=B$ or $D$}; \\
          \Thetab_{\sfss'}^{\sfss}(\pi_{\uptau'}\otimes \det^{\varepsilon_{\wp}}), &\hbox{if $\star=C$ or $\widetilde C$}; \\
              \check \Thetab_{\sfss'}^{\sfss}(\pi_{\uptau'}), &\hbox{if $\star=C^*$ or $D^*$}. \\
            \end{array}
   \right.
\]
Theorem \ref{positivity000} implies that 
$\bar \pi_{\uptau}$ is unitarizable.  Theorem \ref{thm:GDS.AC} and Proposition \ref{thmac1} imply that $\bar \pi_{\uptau}$ and $\pi_{\uptau}$ are $\CO$-bounded, and 
\[
    \mathrm{AC}_{\cO}( \bar \pi_{\uptau})=\mathrm{AC}_{\cO}(\pi_{\uptau})=\mathrm{AC}(\uptau)\neq 0. 
\]
Since $\bar \pi_{\uptau}$ is a completely reducible quotient of $\pi_{\uptau}$, and $\pi_{\uptau}$ has at most one  irreducible quotient by the fundamental result of Howe (\cite[Theorem 1A]{Howe89}), we conclude that $\bar \pi_{\uptau}$ is irreducible. 

Write 
$\CJ_\uptau$ for the kernel of the natural surjective homomorphism $\pi_{\uptau}\rightarrow \bar \pi_{\uptau}$. By Lemma \ref{inf1},  $\oU(\g_\sfss)^{G_{\sfss, \C}}$ acts on $\CJ_{\uptau}$ through the character $\chi(\check \CO)$. Moreover, $\CJ_\uptau$  is $\CO$-bounded and  $\mathrm{AC}(\CJ_{\uptau})=0$. This implies that $\CJ_{\uptau}=0$ by Lemma \ref{inf2}. 
Thus $\pi_{\uptau}=\bar \pi_{\uptau}$. Finally, Proposition \ref{boundm} implies that  $\bar \pi_{\uptau}$ is $(\mathrm d'_\tau-\nu_{\sfss})$-bounded. This finishes the proof of Theorem 
\ref{thmpitau222}. 



\section{Nilpotent orbits: odds and ends}
%\section{Nilpotent orbits: combinatorics}

\def\upp{{\rotatebox[origin=c]{45}{$+$}}}
\def\umm{{\rotatebox[origin=c]{45}{$-$}}}

\NewDocumentCommand\KC{o}{
  \IfNoValueTF{#1}{{K_{\bC}}}{{K_{#1,\bC}}}
}

\def\YD{{\mathsf{YD}}}
\def\SYD{{\mathsf{SYD}}}
\def\MK{\mathsf{MK}}
\def\MK{\widetilde{\mathsf{CS}}}
\def\MYD{{\mathsf{MYD}}}
\def\AND{\quad\text{and}\quad}
\def\deti{{\det}_{\sqrt{-1}}}

\def\AOD{\mathrm{AOD}}
\def\oAC#1{\AC(#1)}
\def\owAC#1{\wAC(#1)}
\def\pac#1{\ac_{#1}^+}
\def\nac#1{\ac_{#1}^-}
\def\ttail#1{{#1}_{\bftt}}
\def\Forall{\text{for all }}


\def\AC{\mathrm{AC}}
\def\wAC{\mathrm{AC}^{\mathrm{weak}}}

% \NewDocumentCommand{\ac}{k{^_}{{}{}}}{\cL^{#1}_{#2}}
\def\ac{\cL}
\def\lotimes{\otimes}
%\NewDocumentCommand{\lotimes}{D(){(0,0)}}{\otimes ({#1})}

\def\KM{{\mathcal{K_{\mathsf{M}}}}}

\def\sqii{\sqrt{-1}}
\def\St#1{\mathrm{St}_{#1}}
\def\VV#1{{}^{#1}V}
\def\SLT{\varphi}
\def\SLTK{\varphi_{\sfss}}
\def\GC{G_{\sfss,\bC}}
\def\Js{J_\sfss}
\def\Ls{L_\sfss}

In this section, we review the combinatorial parametrization of nilpotent
orbits, and describe several constructions related to nilpotent orbits in combinatorial terms.

\subsection{Signed Young diagrams and the descent map}
Let $(\Vs,\inn{}{} \Js,\Ls)$ be a classical space with signature $\sfss$, as in Section \ref{sec:Nil}.

%\subsubsection*{Young diagram classification of nilpotent orbits over $\bC$}
The set $\Nil(\fggs)$ of nilpotent $\GC$-orbits in $\fggs$ is parameterized by
Young diagrams: the nilpotent obit $\GC\cdot \X$
generated by a nilpotent element $\X$ in $\fggs$ is attached to the Young
diagram $\cO$ such that
\[
\bfcc_{i}(\cO) = \dim (\Ker(\X^{i})/\Ker(\X^{i-1})) \quad \textrm{for all  }\,
 i\in \bN^{+}.
\]
See \cite[Section~5.1]{CM}. By abuse of notation, we identify the Young
diagram $\cO$ with the orbit $\GC\cdot \X$. Let $\YD_\star$ denote the set of all Young diagrams of type $\star$.

\def\Xslt{\mathring{\bfee}}
\def\eslt{\mathring{X}}

Let $\slt$ be the complex Lie algebra consisting of $2\times 2$ complex matrices
of trace zero. Put
\[
  \Xslt := \begin{pmatrix}1/2 & \phantom{-} \sqii/2 \\
    \phantom{-} \sqii/2 & -1/2 \end{pmatrix},
  \quad \text{and} \quad
\eslt := \begin{pmatrix}0 & 1\\ 0 & 0 \end{pmatrix}.
\]
By the Jacobson-Morozov theorem,  there is a
Lie algebra homomorphism
\begin{equation}\label{eq:slt}
\SLT \colon \slt \rightarrow \fggs \subset \gl(\Vs)
\quad \text{such that $\SLT(\Xslt)=\X$. }
 \end{equation}

 Let $\St{i}=\Sym^{i-1}(\bC^{2})$ denote the realization of the irreducible
 $\slt$-module $\SLT_{i}$ on the $(i-1)^{\text{th}}$-symmetric power of the standard
 representation $\bC^{2}$.

As an $\slt$-module via $\SLT$, we have
\begin{equation}\label{eq:Vl.1}
\Vs = \bigoplus_{i\in \bN^{+}} \VV{i} \otimes_\bC \St{i},
\end{equation}
where
\[
\VV{i} := \Hom_{\slt}( \St{i},\Vs)
\]
is the multiplicity space and
$\dim \VV{i} = \bfcc_{i}(\cO) - \bfcc_{i-1}(\cO)$.
We view %the Young diagram
$\cO$ as the function
\[
  \cO\colon \bN^{+} \rightarrow \bN \text{ such that } \cO(i) = \dim \VV{i}.
\]

% Let $\SLT_{i}\colon  \slt\rightarrow \mathfrak{gl}_{n}(\bC)$ be the representation
%  of $\slt$ on $\St{i}$.

%\subsubsection*{Signed Young diagram classification of rational nilpotent orbits}

We equip $\St{i}$ with a ``standard'' classical space structure
 $(\St{i},\inn{}{}_{\St{i}},J_{\St{i}},L_{\St{i}})$. % with classical symbol $(\star_{i},p_{i},q_{i})$.
 Specifically,
 \begin{itemize}
   \item the classical signature of $\St{i}$ is $(B,(i+1)/2,(i-1)/2)$ if $i$ is odd and
         $(C,i/2,i/2)$ if $i$ is even,
   \item $J_{\St{i}}$ is the complex conjugation on $\St{i}=\Sym^{i-1}(\bC^{2})$,
   \item $L_{\St{i}} = (-1)^{\floor{\frac{i-1}{2}}}\, \Sym^{i-1}({\tiny\begin{pmatrix}\phantom{-}0 & 1 \\-1 & 0 \end{pmatrix}})$,
   \item $\inn{}{}_{\St{i}}$ is uniquely determined by $J_{\St{i}}$ and $L_{\St{i}}$ up to
         a positive scalar. (We fix one for each $i$.)
          \end{itemize}

\trivial[h]{\color{red} $L_{\St{i}}$ should be the Lie group element action.}


Suppose we have a rational nilpotent orbit $ \sO\in \Nil_\sfss(\cO)$,
%:=\Nil_{\Ks}(\fpp_{\sfss})\cap \cO $
and $\X\in \sO$. Then the Lie-algebra
homomorphism \eqref{eq:slt} can be made compatible with the classical space
structure on $\Vs$ so that
% Let $ \sO\in \Nil_\sfss(\cO)$.
% By \cite{Se} (also see \cite[Section~6]{Vo89}), the Lie-algebra
% homomorphism $\SLT \colon \slt \rightarrow \fgg \subset \gl(V)$ such that
%$\SLT(\Xslt)\in \sO$ and compatible with the classical space structure on $V$.
%  We define a equivalent relation on the set $\set{B,C,\wtC,D,C^{*},D^{*}}$ with
%  the relation $B\sim D$ and $C\sim \wtC$.
$\slt$-isotypic components $\VV{i}\otimes \St{i}$
are invariant under the $\Js$ and $\Ls$ actions. See \cite{Se} (and also \cite[Section~6]{Vo89}).
When this is done, the multiplicity space $\VV{i}$ will have a unique classical space
structure $(\VV{i}, \inn{}{}_{\VV{i}},J_{i},L_{i})$ with classical signature
$\sfss_{i}=(\star_{i},p_{i},q_{i})$ such that $\star_{i}\neq \wtC$ and
when restricted on $\VV{i}\otimes \St{i}$, the following conditions hold:
\begin{itemize}
  \item $\inn{}{}$ is given by $\inn{}{}_{\VV{i}}\otimes \inn{}{}_{\St{i}}$,
  \item $\Js$ is given by $J_{i}\otimes J_{\St{i}}$,
  \item $\Ls$ is given by $L_{i}\otimes L_{\St{i}}$,
  \item the symbol $\star_{i}$ satisfies the following ``type'' condition
        \begin{equation} \label{eq:ssrel}
          \begin{cases}
            \star_{i} \sim\star_{\sfss} & \text{if $i$ is odd}\\
            \star_{i} \sim \star'_{\sfss} & \text{if $i$ is even}
          \end{cases}
      \end{equation}
\end{itemize}
Here $\star'_{\sfss}$ is the Howe dual of $\star_{\sfss}$, and
 $\sim$ is the equivalent relation on $\set{B,C,\wtC,C^{*},D,D^{*}}$ such that
 $B\sim D$ and $C\sim \wtC$.
Moreover the classical signature $\sfss_{i}$ (for each $i\in \bN^{+}$) is independent of the choice of $\SLT$
and together they uniquely determine the orbit $\sO$.

\smallskip

\def\CCSS{\overline{\mathsf{CS}}}
\def\CCSS{{\mathsf{CS}}}
Let
\[
\CCSS := \set{(\star,p,q)| \star\neq \wtC \text{ and }(\star,p,q) \text{ is a classical signature}}.
\]
We identify $\sO$ with the function
\[
  \sO\colon \bN^{+}\rightarrow \CCSS, \qquad i \mapsto \sfss_{i}.
\]
Let $\sfmm \colon \CCSS\rightarrow \bN$ be the map given by $(\star,p,q)\mapsto
p+q$. The complexification $\GC\cdot \sO$ of a rational nilpotent orbit $\sO$ is then
given by $\cO := \sfmm\circ \sO$.

\smallskip

We now recall the signed Young diagram classification of rational nilpotent orbits \cite[Chapter 9]{CM}.
Define
\[
  \begin{split}
  \SYD_{\star}(\cO) &:=\Set{\sO\colon \bN^{+}\rightarrow \CCSS| \begin{minipage}{9em}
      $\star_{\sO(i)}$ satisfies \eqref{eq:ssrel},\\
      $\sfmm\circ \sO = \cO$
      \end{minipage}
    }\\
    \SYD_{\star} &:= \bigsqcup_{\cO\in \Nil_{\star}} \SYD_{\star}(\cO).
  \end{split}
\]
Then $\SYD_{\star}$ is naturally identified with the set
$\bigsqcup_{\sfss}\Nil_{\Ks}(\fpp_{\sfss})$ where $\sfss$ runs over all
classical signatures of type $\star$.

\def\Gsr{G_{\sfss}}
\def\fggsr{\fgg_{\sfss,\bR}}

An important property of a compatible map $\SLT$ is that the Kostant-Sekiguchi
correspondence is given by
\[
  \Gsr \cdot \SLT(\eslt) \longleftrightarrow  \Ks\cdot \SLT(\Xslt).
\]
% Therefore, we also identify $\SYD_\star$ with the set of real nilpotent orbits
% $\bigsqcup_{\sfss}\Nil_{\Gsr}(\fgg_{\sfss})$ where $\sfss$ running over all
% classical signatures of type $\star$.
Via this correspondence, we may therefore make the following
identification:
\begin{equation}\label{eq:KS}
\Nil_{\Gs}(\fggsr) = \SYD_\sfss = \Nil_{\Ks}(\fpps).
\end{equation}

\smallskip

For any $\sO\in \SYD_{\star}$, we define $\Sign(\sO) = (p,q)$, again called the signature,
 if $\sO$ corresponds to a nilpotent orbit in $\Nil_{\sfss}(\fpps)$ with the classical
signature $\sfss=(\star,p,q)$. Also we write
\[
\SYD_{\sfss}(\cO) := \set{\sO\in \SYD_{\star}(\cO)| \Sign(\sO)=(p,q)}.
\]
which is identified with $\Nil_{\sfss}(\cO)$.

The signature $\Sign(\sO)$ can be computed from the signatures of $\sO(i)$ ($i\in \bN^+$)
by the following formula:
\[
  \begin{split}
    \Sign(\sO) = &
    \sum_{i\in \bN^{+}} \big(i \cdot p_{2i} + i \cdot q_{2i},
    i \cdot p_{2i}+i\cdot q_{2i}\big)\\
    &  +
    \sum_{i\in \bN^{+}} \big(i \cdot p_{2i-1} + (i-1)\cdot q_{2i-1},
    (i-1) \cdot p_{2i-1}+i \cdot q_{2i-1}\big),
  \end{split}
\]
where $\sO(i)=(\star_{i}, p_{i},q_{i})$. The signature $\lsign(\sO)$ of the ``leftmost column'' of the
signed Young diagram $\sO$ (see \Cref{eg:MYD}) is also useful, which is computed by
\[
  \lsign(\sO) =
    \sum_{i\in \bN^{+}} \big(q_{2i},p_{2i} \big)
      + \sum_{i\in \bN^{+}} \big(p_{2i-1},q_{2i-1} \big).
\]

For each $\sO\in \SYD_\star$ define the naive descent $\DDn(\sO)\in
\SYD_{\star'}$ by
\[\big(\DDn(\sO)\big)(i) = \sO(i+1) \quad \text{for } i \in \bN^+.\]

We can now explicitly describe the decent map $\DDss$.
\begin{prop}\label{prop:DDss}
Suppose $\sfss=(\star,p,q)$ and
$\sfss'=(\star,p',q')$. A nilpotent orbit $\sO\in \Nil_{\Ks}(\fpps)$ is in the
image of moment map if and only if
\begin{equation}\label{eq:DDss1}
  (p_0,q_0):= (p',q') - \Sign(\DDn(\sO)) \succeq (0,0).
\end{equation}
When \eqref{eq:DDss1} is satisfied, the descent $\sO' := \DDss(\sO)$
is given by
\begin{itemize}
\item $\sO'(1) = (\star'_1,p_2+p_0, q_2+q_0)$ where $\star'_1\sim \star'$ and
$(\star_{2},p_2,q_2)=\sO(2)$;
\item $\sO'(i) = \sO(i+1)$, for $i\geq 2$. \qed
\end{itemize}
\end{prop}

\subsection{Marked Young diagram classification of equivariant line bundles}
\label{sec:MYD}
\def\KCs{K_{[\sfss],\bC}}
\def\KCsX{(K_{[\sfss],\bC})_{\X}}
\def\KCsoX{(K_{[\sfss],\bC})_{\X}^{\circ}}
%For a classical signature $\sfss$,
Suppose $\sO\in \Nil_{\Ks}(\fpps)$ and $\SLT$ is
compatible with $\sO$. Let  $\X := \SLT(\Xslt)\in \sO$,
\[
  \KCs := \GC^{L_{\sfss}}, \quad \text{and}\quad
  \KCsX:=\Stab_{\KC[[\sfss]]}(\X),
\]
the stabilizer of $\X$ in $\KCs$.
% centralizer of $\Im(\phi)$ in $\KC[\sfss]$.
Using the signed Young diagram classification of $\sO$, we have the following
well-known fact on the isotropy subgroup $\KCsX$.

\begin{lem}\label{lem:KX1}
  The isotropy subgroup have decomposition $\KCsX= R_\X\ltimes U_\X$, where
  $U_\X$ is the unipotent radical of $\KCsX$ and $R_{\X}$ is a reductive group
  canonically identified with
  \[
    \prod_{i\in \bN^{+}} \KC[[\sfss_{\sO(i)}]].
  \]
 Consequently $R_{\X}$ is a product of complex general linear groups
with complex symplectic groups or orthogonal groups. \qed
\end{lem}


% As a consequence, $R_{\X}$ is a products of complex general linear group
% with  complex symplectic groups or orthogonal groups
% \[
%   \prod_{\substack{\sO(i)=(\star_{i},p_{i},q_{i})\\
%       \star_{i}\in \set{B,D}, p_{i}+q_{i}>0
%     }} \rO_{p_{i}}(\bC)\times \rO_{q_{i}}(\bC).
% \]



\medskip

\def\DCO#1{\mathsf{LB}_{#1}(\sO)}
\def\KsoX{{(\Ks)_{\X}^{\circ}}}
\def\KsX{(\Ks)_{\X}}

Recall that $\Ks$ is a double cover of $\KCs$ if $\star =  \wtC$ and $\Ks = \KCs$ otherwise.
Fix a one-dimensional character $\chi$ %of the connected component $\KsoX$
 of $\KsX$. We assume $\chi$ is a genuine character if $\star=\wtC$.

Let $\KsoX$ be the connected component of $\KsX$ and
\[
\DCO{\chi} := \set{\text{$\Ks$-equivariant line bundle $\cE$ on $\sO$
such that $\cE_{\X}\mid_{\KsoX}=\chi\mid_{\KsoX}$}}.
\]
By the structure of $R_{\X}$, a line bundle $\cE\in \DCO{\chi}$ is completely
determined by the restriction of the $\KsX$-module $\cE_{\X}$ on the orthogonal group factors of $R_{\X}$. We introduce marked Young diagrams to
parameterize the restrictions.

Let
\[
  \MK = \CCSS \cup \set{(\star,r,s)|\star\in \set{B,D}, r,s\in \bZ}.
\]
Suppose $\sO(i)=(\star_{i},p_{i}, q_{i})$. We identify $\cE \in \DCO{\chi}$ with the map
\[
  \cE\colon \bN^{+}\rightarrow  \MK
\]
such that
\begin{itemize}
  \item $\cE(i) := \sO(i)$ if $\star_{i} \notin \set{B,D}$;
  \item $\cE(i) := (\star_{i},(-1)^{\epsilon^{+}_{i}}p_{i},(-1)^{\epsilon^{-}_{i}}q_{i})$
        if $\star_{i}\in \set{B,D}$ and  the factor
        $\KC[[\cE(i)]] =\rO_{p_{i}}(\bC)\times \rO_{q_{i}}(\bC)$ acts by
        ${\det}^{\epsilon^{+}_{i}}\otimes {\det}^{\epsilon^{-}_{i}}$
        on
        \[
        \begin{cases}
          \cE, & \text{when } \star = D,\\
          \cE\otimes \deti^{-1} , & \text{when } \star = B.
        \end{cases}
        \]
\end{itemize}

%  $\sF\colon \MK\rightarrow \CCSS$ be the map given by
% $(\star,r,s)\mapsto (\star,\abs{r},\abs{s})$
Let
\begin{equation}\label{eq:F1}
\sF\colon \MK\rightarrow \CCSS\qquad (\star,r,s)\mapsto (\star,\abs{r},\abs{s}),
\end{equation}
and
\[
  \MYD(\sO) := \Set{\cE\colon \bN^{+}\rightarrow \MK|\sF\circ \cE=\sO }.
\]
From the structure of $R_{\X}$, the above recipe will then yield an identification of $\MYD(\sO)$ with the set $\DCO{\chi}$ whenever the later set is non-empty.


%The signature of $\cE$ is defined as the sginature of $\sf\circ\cE$.
%We wreite $\Sign(\cE):= \Sign$

Let
\[
  \MYD_{\sfss}(\cO) := \bigcup_{\sO\in \SYD_{\sfss}(\cO)}\MYD(\sO).% \quad
\]

Recall the set
\[
  \AOD_{\sfss}(\cO)=\bigsqcup_{\sO\textrm{ is a $\Ks$-orbit in
      $\cO\cap \fpps$}} \AOD_{\Ks}(\sO),
\]
where $\AOD_{\sfss}(\sO)$ is the set of isomorphism classes of admissible orbit
 data over $\sO$ (\Cref{defaod}). We may thus identify $\AOD_{\sfss}(\cO)$ with
 a subset of $\MYD_{\sfss}(\cO)$. We also set
\[
  % \MYD_{\sfss} := \bigcup_{\sO\in \Nil_{\Ks}(\fpps)}\MYD(\sO) \AND
  \MYD_\star := \bigcup_{\sfss, \cO\in \YD_\star} \MYD_{\sfss}(\cO)
\]
where $\sfss$ runs over all possible classical signatures of type $\star$.
%  % By abuse of notation, we identify $\SYD_{\star}$ with a subset of
% Set
% \[
%   \MYD_{\star} := \bigcup_{\cO\in \Nil_{\star}}\MYD_{\star}(\cO),
% \]
% In the next section, we will define some operations on the  free abelian groups
% $\bZ[\SYD_{\star}]$ and $\bZ[\MYD_{\star}]$.


The map $\sF$ natrually extends to a homomorphism
\begin{equation}\label{eq:F}
\sF\colon \bZ[\MYD_\star] \rightarrow \bZ[\SYD_\star]
\end{equation}
sending $\cE\in \MYD_\star$ to $\sF\circ \cE \in \SYD_\star$.

\smallskip

Suppose $\cE\in \MYD_{\star}$ such that $\cE(i)=(\sfss_{i},p_{i},q_{j})$.
To ease the notation, we will use
\[
  ( i_{1}^{(p_{i_{1}},q_{i_{1}})}i_{2}^{(p_{i_{2}},q_{i_{2}})}\cdots i_{k}^{(p_{i_{k}},q_{i_{k}})} )_{\star}
\]
to represent $\cE$ where
$i_{1},\cdots, i_{k}$ are all the indices $i$ such that $(p_{i},q_{i})\neq (0,0)$.
A Young diagram in $\YD_{\star}$ is represented similarly.

We adopt the usual convention to depict a signed Young diagram where the signature
$(p_{i}, q_{i})$ equals the total signature for the rightmost columns of
length $i$ rows.
We use ``$\upp$'' and ``$\umm$'' instead of ``$+$'' and ``$-$'' to mark the rows
when $p_{i}$ or $q_{i}$ is negative.

\begin{eg}\label{eg:MYD}
The expression $(1^{(r,s)})_{\star}$ represents the marked Young diagram in $\MYD_{\star}$ such that
 \[
   (1^{(r,s)})_{\star}(i) := \begin{cases}
     (\star,r,s) & \text{if } i=1\\
     (\star_{i},0,0) & \text{if $i>0$}\\
     %(\star',0,0) & \text{if $i>0$ and $i$ is even}
   \end{cases}
 \]
 Here the symbols $\star_{i}$ are uniquely determined by \eqref{eq:ssrel}.\\
The marked Young diagram  $\cE:=(1^{(1,-1)}2^{(2,0)}3^{(0,-1)})_{B}$ is depicted as follows:
\[
  \ytb{\umm\upp\umm,-+,-+,+,\umm}
\]
The singed Young diagram $\sO:=\sF\circ \cE$ is
$(1^{(1,1)}2^{(2,0)}3^{(0,1)})_{B}$ and the complexification of $\sO$
is $(1^{2}\,2^{2}\,3^{1})_{B}$.
\end{eg}

%\subsection{Operations on marked Young diagrams}
\subsection{Theta lifts of admissible orbit data: combinatorial descriptions}
In this section, we will first define some operations on marked Young diagrams. We then use these operations to describe theta lifts of admissible orbit data.

\smallskip

Suppose $\star\in \set{B,D}$. We define a $\bZ/2\bZ\times \bZ/2\bZ$ action on
$\MYD_{\star}$, as follows. Given $\cE\in \MYD_{\star}$ such that $(\star_{i},p_{i},q_{i}):=\cE(i)$, and $(\epsilon^{+},\epsilon^{-})\in \bZ/2\bZ\times \bZ/2\bZ$,
we define
$\cE \lotimes (\epsilon^{+},\epsilon^{-})\in \MYD_{\star}$ by
\[
(\cE \lotimes(\epsilon^{+},\epsilon^{-}))(i) := \begin{cases}
  (\star_{i}, (-1)^{\frac{\epsilon^{+}(i+1)+\epsilon^{-}(i-1)}{2}}p_{i},(-1)^{\frac{\epsilon^{+}(i-1)+\epsilon^{-}(i+1)}{2}}q_{i})
  & \text{if $i$  is odd,}\\
  (\star_{i}, p_{i},q_{i}) & \text{otherwise.}
\end{cases}
\]

For each $\cE\in \AOD_{\Ks}(\cO)$, its twist
$\cE\otimes  (1^{{-,+}}_{p,q})^{\epsilon^{+}}\otimes (
  1^{+,-}_{p,q})^{\epsilon^{-}}$
is also  in $\AOD_{\Ks}(\cO)$.
As elements in $\MYD_{\star}$, we have
\[
  \cE\otimes  (1^{{-,+}}_{p,q})^{\epsilon^{+}}\otimes (1^{+,-}_{p,q})^{\epsilon^{-}} = \cE \otimes (\epsilon^{+},\epsilon^{-}).
\]

\medskip

Suppose $\star\in \set{C,\wtC,D^{*}}$. We define an involution $\maltese$ on
$\MYD_{\star}$:
For $\cE\in \MYD_{\star}$ such that $(\star_{i},p_{i},q_{i}):=\cE(i)$,
we define $\maltese\cE\in \MYD_{\star}$ such that
\[
(\maltese\cE)(i) := \begin{cases}
  (\star_{i}, -p_{i},-q_{i})& \text{if $i\equiv 2 \pmod{4}$ and $\star_{i}\in \set{B,D}$,}\\
  (\star_{i}, p_{i},q_{i}) & \text{otherwise.}
\end{cases}
\]
For each $\cE\in \DCO{\chi}$, its twist $\cE\otimes \deti^{-1}$
is a line bundle  in $\DCO{\chi'}$ where  $\chi'$ is a certain character. % $\chi'$.
As elements in $\MYD_{\star}$, we have
\[
  \cE\otimes \deti^{-1} = \maltese (\cE).
\]

\medskip

We now define the argumentation operation on marked Young diagrams.
Suppose that $(p_{0},q_{0})\in \bZ\times \bZ$ and there is a
$\star'_{0}\sim \star'$ such that
$\sfss_{0} := (\star'_{0},p_{0},q_{0})\in \CCSS$.
For each $\cE\in \MYD_{\star}$,
let $\cE\cdot (p_{0},q_{0})$ be the marked Young diagram in $\MYD_{\star'}$ given by
\[
  \big(\cE\cdot (p_{0},q_{0})\big)(i) :=
  \begin{cases}
    \sfss_{0} & \text{if } i=1,\\
    \cE(i-1) & \text{if } i>1.
  \end{cases}
\]
We will use the same notation to denote the  natural extension of above
operation to $\bZ[\MYD_{\star}]$.
% and we will use the same notation

\medskip

We introduce a partial order $\succeq$ on $\bZ\times \bZ$ by
\[
  (a,b) \succeq (c,d) \Leftrightarrow
  \begin{cases}a\geq c\geq 0\\ b\geq d\geq 0\end{cases}
  \quad  \text{or} \quad
  \begin{cases} a\leq c\leq 0\\ b\leq d\leq 0 \end{cases}.
\]
We also define an involution on $\bZ\times \bZ$ % $\rightarrow \bZ\times \bZ$
by $(a,b)\mapsto (a,b)^{\sharp}:=(b,a)$. For a $\cE\in \MYD_{\star}$ and $(p_{0},q_{0})\in \bZ\times \bZ$, we write
\begin{equation}
  \label{eq:sub}
  \cE\sqsupseteq (p_{0},q_{0}) \Leftrightarrow
  (p_{1},q_{1})\succeq (p_{0},q_{0})
  \text{ where }\cE(1) = (\star_{1},p_{1},q_{1}).
\end{equation}

Suppose $\cE\sqsupseteq (p_{0},q_{0})$, we define the truncation
$\Lambda_{(p_0,q_0)}\cE$ by
\[
  (\Lambda_{(p_{0},q_{0})}(\cE))(i) = \begin{cases}
    (\star_{1},p_{1}-p_{0},q_{1}-q_{0}) & \text{if } i=1,\\
    \cE(i) & \text{if } i>1.
  \end{cases}
\]
We extend the truncation operation $\Lambda_{( p_{0},q_{0} )}$ to
$\bZ[\MYD_{\star}]$
by setting $\Lambda_{p_{0},q_{0}}(\cE) := 0$ when $(p_{0},q_{0})\nsqsubseteq \cE\in \MYD_{\star}$.
Note that $\Lambda_{(0,0)}$ is the identity map.

The augmentation and truncation of a signed Young diagram are defined by the same
formula.

\medskip

% For any $\cE\in \MYD_{\star}$, let $\Sign(\cE) = (p,q)$ if $\sF\circ \cE$
% corresponds to a nilpotent orbit in $\Nil_{\sfss}(\fpp)$ with the classical
% signature $\sfss=(\star,p,q)$. When $\sO(i)=(\sfss_{i},p_{i},q_{i})$, the
% signature is
% \[
%   \begin{split}
%     \Sign(\cE) = &
%     \sum_{i\in \bN^{+}} \big(i \cdot\abs{p_{2i}} + i \cdot \abs{q_{2i}},
%     i \cdot\abs{p_{2i}}+i\cdot\abs{q_{2i}}\big)\\
%     &  +
%     \sum_{i\in \bN^{+}} \big(i \cdot\abs{p_{2i-1}} + (i-1)\cdot \abs{q_{2i-1}},
%     (i-1) \cdot\abs{p_{2i-1}}+i \cdot \abs{q_{2i-1}}\big).
%   \end{split}
% \]
% The signature $\lsign(\cE)$ of the ``most left column'' of the
% marked Young diagram $\cE$ (see \Cref{eg:MYD}) is given by
% \[
%   \lsign(\cE) =
%     \sum_{i\in \bN^{+}} \big(\abs{q_{2i}},\abs{p_{2i}} \big)
%       + \sum_{i\in \bN^{+}} \big(\abs{p_{2i-1}},\abs{q_{2i-1}} \big).
% \]




\def\cEp{\cE'}
\def\dliftso{{\check \vartheta}_{\sfss',\cO'}^{\sfss\,,\,\cO}}

We are now ready to describe theta lifting of admissible orbit data in terms of marked Young diagrams.

Let $\sfss$ and $\sfss'$ be two fixed classical signatures.
Suppose $\cO\in \Nil(\fggs)$ and $\cO'\in \Nil(\fggsp)$ such that
$\cO'= \DDss(\cO)$.
The theta lifting of marked Young diagrams is a homomorphism
\[
  \dliftso \colon \bZ[\MYD_{\sfss'}(\cO')]\rightarrow \bZ[\MYD_{\sfss}(\cO)]
\]
defined as follows. Set
\[
  t := \bfcc_{1}(\cO')-\bfcc_{2}(\cO).
\]

\medskip

Suppose $\sfss'=(\star',n,n)$ and $\sfss =(\star,p,q)$ such that
$\star'\in \set{ C,\wtC, D^{*}}$  and $\star \in \set{D,B, C^{*}}$.
For $\cEp\in \MYD_{\sfss}(\cO')$, we define
\begin{equation}\label{eq:LCD}
  \dliftso(\cEp) :=
  \begin{cases}
    \left(\maltese^{\frac{p-q}{2}}(\Lambda_{(\frac{t}{2},\frac{t}{2})}( \cEp))\right)\cdot (p_{0},q_{0}) &
    \text{if } ( p_{0},q_{0} ) \succeq (0,0) \text{ and } \star=D, \\
    \left(\maltese^{\frac{p-q+1}{2}}(\Lambda_{(\frac{t}{2},\frac{t}{2})}( \cEp))\right)\cdot (p_{0},q_{0}) &
    \text{if } ( p_{0},q_{0} ) \succeq (0,0) \text{ and } \star=B, \\
    \big(\Lambda_{(\frac{t}{2},\frac{t}{2})}( \cEp)\big)\cdot (p_{0},q_{0}) &
    \text{if } ( p_{0},q_{0} ) \succeq (0,0) \text{ and } \star=C^{*}. \\
    0 & \text{otherwise}
  \end{cases}
\end{equation}
Here $(p_{0}, q_{0})  := (p,q) -(n,n)-\lsign(\cEp)^{\sharp} + (\frac{t}{2},\frac{t}{2})$. Note that $t$ is a nonnegative even integer for the mentioned cases.

\trivial[h]{
When $t=0$, we define
\[
  \dliftao(\cEp) :=
  \begin{cases}
    (\maltese^{\frac{p-q}{2}}(\cE))\cdot (D,p_{0},q_{0}) &
    \text{if } ( p_{0},q_{0} ) \succeq (0,0) \\
    0 & \text{otherwise}
  \end{cases}
\]
where $(p_{0}, q_{0})  := (p,q) -(n,n)-\lsign(\cEp)^{\sharp} $.
When $t>0$, we define
\[
  \dliftv(\cEp) :=
  \begin{cases}
    (\maltese^{\frac{p-q}{2}}\Lambda_{(\frac{t}{2},\frac{t}{2})}(\cEp)) \cdot (D,0,0) &
    \text{if } ( p_{0},q_{0} ) \succeq (0,0) \\
    0 & \text{otherwise}
  \end{cases}
\]
where $(p_{0}, q_{0})  := (p,q) -(n,n)-\lsign(\cEp)^{\sharp} + (\frac{t}{2},\frac{t}{2})$.
}

\medskip

Suppose $\sfss' =(\star',p,q)$ and $\sfss=(\star,n,n)$ such that
$\star' \in \set{ D,B,C^{*}}$  and $\star \in \set{C,\wtC,D^{*}}$.
For $\cEp\in \MYD_{\sfss}(\cO')$, we define
\begin{equation}\label{eq:LDC}
  \dliftso(\cEp) :=
  \begin{cases}
    \maltese^{\frac{p-q}{2}}\big(\sum_{j=0}^{t} \Lambda_{(j,t-j)}( \cEp)\cdot (n_{0},n_{0})\big) & \text{if
    } \star = C,\\
    \maltese^{\frac{p-q-1}{2}}\big(\sum_{j=0}^{t} \Lambda_{(j,t-j)}( \cEp)\cdot (n_{0},n_{0})\big) & \text{if
    } \star = \wtC,\\
    \sum_{j=0}^{\frac{t}{2}} \Lambda_{(2j,t-2j)}( \cEp)\cdot (n_{0},n_{0}) & \text{if
    } \star = D^{*}.
  \end{cases}
\end{equation}
Here $2n_{0} = \bfcc_{1}(\cO)-\bfcc_{2}(\cO)$.

\trivial[h]{
\[
  \dliftv(\cEp) :=
  \begin{cases}
    \maltese^{\frac{p-q}{2}}\big(\cEp\cdot (C,n_{0}, n_{0}) \big) &
    \text{if } t=0\\
    \maltese^{\frac{p-q}{2}}\big((\sum_{j=0}^{t} \Lambda_{(j,t-j)} \cE)\cdot (C,0,0)\big)& \text{if } t>0\\
  \end{cases}
\]
where $2n_{0} = \bfcc_{1}(\cO)-\bfcc_{2}(\cO)$.
}
% We remark that
% $\maltese^{\frac{p-q}{2}}((\dagger \cL)\cdot \dagger_{n_{0}, n_{0}})= (\maltese^{\frac{p-q}{2}}(\dagger\cL))\cdot \dagger_{n_{0},n_{0}}$.

\trivial[h]{
  We take a splitting $\rO(p,q)\times \Sp(n,\bR)$ in to the big metaplectic
  group $\Mp$ such that $\rO(p,\bC)\times\rO(q,\bC)$ acts on the Fock model linearly.
  The maximal compact $K_{\Sp(n,\bR)}= \rU(2n)$ acts on the Fock model by the
  character $\zeta=\det^{(p-q)/2}$.
  For the component of the rational nilpotent orbit $\Sp(2n,\bR)$,
  the factor of the component group is $\Sp$ if the corresponding row has odd
  length.
  Otherwise the component group is $\rO(p_{2k})\times\rO(q_{2k})$ where
  $(p_{2k},q_{2k})$ is the signature corresponding to the $2k$-rows.
  $\zeta|_{\rO(p_{2k})}= \det^{(p-q)k/2}_{\rO(p_{2k})}$ which is nontrivial if
  and only if $p-q\equiv 2\pmod{4}$ and $2k\equiv 2\pmod{4}$.
  This gives the formula above.
}


\trivial[h]{
  For odd orthogonal-metaplectic group case, we still take the splitting such
  that $\rO(p,q)$ acts linearly.

  The maximal compact $\wtK_{\Sp(n,\bR)}= \widetilde{\rU(2n)}$ acts on the Fock model by the
  character $\zeta=\det^{(p-q)/2}$.

  Then
  $\det^{(p-q)/2}|_{\rO(p_{2k})\times \rO(q_{2k})}=\det^{\frac{(p-q)k}{2}}_{\rO(p_{2k})}\boxtimes
  \det^{\frac{(p-q)k}{2}}_{\rO(q_{2k})}. $
  Note that $p-q$ is an odd number.
  When $k$ is even the character on $\widetilde{\rO(p_{2k})}/\widetilde{\rO(q_{2k})}$ are
  $\bfone$ or $\det$. When $k$ is odd the character would be $\det^{\half}$ or
  $\det^{\half+1}$.

  We assume the default character on $2k$-rows is $\det^{\frac{k}{2}}$, and
  we use $+/-$ to mark the row. Otherwise, we use $\upp/\umm$ to mark the rows.
}

Suppose $\cO$ is regular with respect to $\DDss$, then admissible orbit data are
lifted to admissible orbit data, in view of \Cref{lem:aod}. The following lemma is now routine to check, which we skip.
\begin{lem}\label{lem:clift}
Suppose $\cO$ is regular with respect to $\DDss$. Then the following diagram is commute
\[
  \begin{tikzcd}
    \bZ[\AOD_{\sfss'}(\cO')]\ar[r,"\dliftso"]\ar[d] & \bZ[\AOD_{\sfss}(\cO)]
     \ar[d] \\
    \bZ[\MYD_{\sfss'}(\cO')] \ar[r,"\dliftso"] & \bZ[\MYD_{\sfss'}(\cO')].\\
  \end{tikzcd}
\]
Here the vertical arrows are given by the parameterization of admissible orbit data of this section, and the top horizontal map is defined in  \Cref{sec:dlift}. \qed
\end{lem}

\subsection{Induced orbits and double geometric lifts}
\label{subsec:induced}
\def\indsss{\Ind_{R_{\sfss'',\sfss_0}}^{G_{\sfss''}}}
\def\indss{\Ind_{R_{\sfss'',\sfss_0,\bC}}^{G_{\sfss'',\bC}}}
\def\WFw{{\mathrm{WF^{weak}}}}
\def\Gor{G_1}
\def\Ror{R_1}
\def\fggor{\fgg_{1,\bR}}
\def\frror{\frr_{1,\bR}}
\def\fnnor{\fnn_{1,\bR}}
\def\sOpr{\sO'_\bR}
\def\sOr{\sO_\bR}
\def\Indrg{\Ind_{R_{1}}^{G_{1}}}
\def\Indrgc{\Ind_{R_{1,\bC}}^{G_{1,\bC}}}

In this section, we describe the weak associated cycles
of the induced representations appearing in \Cref{doubtt}.

We first recall the notion of induction of real nilpotent orbits. Let $G_1$ be
an arbitrary real reductive group and let $P_1$ be a real parabolic subgroup of
$G_1$, namely a closed subgroup of $G_1$ whose Lie algebra is a parabolic
subalgebra of $\fgg_{1,\bR}$. Let $N_1$ be the unipotent radical of $P_1$ and
$R_1:=P_1/N_1$. Let $\fpp_{1,\bR} := \Lie(P_1)$, $\frr_{1,R} := \Lie(R_1)$
\[
r_1:  \fpp_{1, \bR} \longrightarrow  \, \frr_{1, \bR}
\]
be the natural quotient map.

Define a morphism
\[
  \Indrg \colon \bZ[\Nil_{\Ror}(\frror)]\longrightarrow  \bZ[\Nil_{\Gor}(\fggor)]
\]
by
\[
  \Indrg (\sOpr) =  \sum_{\sOr} \frac{\# C_{G_1}(v)}{\#
  C_{P_1}(v)} \sOr
\]
where $\sOr$ runs over all open orbits in $G_1\cdot r_1^{-1}(\sOpr)$,
$v\in \sOr$, $C_{G_2}(v)$ and $C_{P_1}(v)$ are the component groups of the
centralizers of $v$ in $G_1$ and $P_1$ respectively. It is well-known that this definition only depends on the Levi subgroup $R_1$ but independent of the choice
of the parobolic subgroup $P_1$.

Via the identification in \eqref{eq:KS}, we translate  $\Ind_{R_1}^{G_1}$  to a morphism
\[
  \Ind_{R_1}^{G_1}\colon \bZ[\Nil_{K_{R_1,\bC}}(\fpp_{\frr_1})]\rightarrow
\bZ[\Nil_{K_{G_1,\bC}}(\fpp_{\fgg_1})].
\]
Here $K_{R_1,\bC}$ and $K_{G_1,\bC}$
are maximal compact subgroups of $R_1$ and $G_1$ respectively; $\fpp_{\frr_1}$
and $\fpp_{\fgg_1}$ are the non-compact parts of $\frr_1$ and $\fgg_1$ respectively.

We remark that the notion of induced nilpotent orbits was first
introduced by Lusztig and Spaltenstein for complex reductive groups \cite{LS}.
Let $r_{1,\bC}\colon \fpp_1 \rightarrow \frr_1$ be the natural quotient map
where $\fpp_1$ and $\frr_1$ are the complexification of $\fpp_{1,\bR}$ and
$\frr_{1,\bR}$ respectively.
For a complex nilpotent orbit $\cO'\in \Nil_{R_{1,\bC}}(\bfrr_1)$, there is a
unique open orbit in $G_{1,\bC}\cdot r_{1,\bC}^{-1}(\cO')$ which is, by definition, the induced orbit $\Indrgc(\cO')$.

  %When $G_1$ is moreover a  complex reductive group,  write $\fgg_1$ for its Lie algebra and write $\fmm_1$ for the Lie algebra of $M_1$. Then for each complex nilpotent orbit $\cO_{1}\in \Nil_{M_1}(\fmm^*_{1})$, the induced %orbit $\Ind_{P_1}^{G_1} \cO_{1}\in \Nil_{G_1}(\fgg^*_{1})$ is similarly defined, as in \cite[Chapter 7]{CM}.




\begin{lem}[\emph{cf}. Barbasch {\cite[Corollary~5.0.10]{B.Orbit}}]
  \label{thm:Bar}
  Let $\cO'\in \Nil_{R_{1,\bC}}(\bfrr_1)$ and  $\pi'$ be an $\cO'$-bounded Casselman-Wallach
  representation of $R_1$. Then $\Ind^{G_1}_{P_1}\pi'$
  is $\cO:= \Indrgc(\cO')$-bounded. Moreover
\[
\WFw(\Ind^{G_1}_{P_1} \pi') = \Indrg (\WFw(\pi'))
\]
and
\[
\wAC_{\cO}(\Ind^{G_1}_{P_1} \pi') = \Indrg (\wAC_{\cO'}(\pi')).
\]
\end{lem}
\begin{proof}
Barbasch proved the required formula for weak wavefront cycles when $G_1$ is the group of
real points of a connected reductive algebraic group. The proof carries over in
the slightly more general setting here.

According to the fundamental result of Schimd-Vilonen \cite{SV}, the weak wave
front cycle and the weak associated cycle agree under the Kostant-Sekiguchi
correspondence. Therefore, we obtain the required formula for weak associated cycles.
\end{proof}

\begin{remark} We thank Professor Vilonen for confirming that the results of \cite{SV} carry over to nonlinear groups.
\end{remark}

\medskip
\def\Rsppo{R_{\sfss'',\sfss_0}}
\def\Rso{R_{\sfss_0}}

We retain the notation in \Cref{sec:DP} and proceed to describe inductions of
certain nilpotent orbits in terms of signed Young diagram
parameterization. These results are used in the proof of \Cref{prpaseq}.

Let $\cOp\in \Nil(\fggsp)$.
Suppose that  $\sfss_0 = (\star_0, p_0,q_0)$ such that $p_0=q_0\geq
\bfcc_2(\cOp)$. Recall that $\sfss' = (\star',p',q')$, $\sfss''
=(\star',p'+p_0,q'+q_0)$ and the Levi subgroup $\Rsppo = \Rso\times \Gsp$. We
identify $\sOp\in \Nil_{\Gsp}(\fggsp)$ with the nilpotent orbit $\set{0}\times
\sOp$ of $\Rsppo$.


\def\po{a}
\def\qo{b}
\def\lll{p_0}

\begin{lem}\label{lem:ind}
Let $\cOpp:=\indss(\cOp)$, $\sOp\in
\Nil_{\sfss'}(\cOp)$ and $t = p_0-\bfcc_1(\cOp)$.
 Then $\cOpp$ satisfies
    $\bfcc_i(\cOpp)=\bfcc_{i-2}(\cOp)$ for $i\geq 4$.
  \begin{enuma}
    \item Suppose $\star'\in \set{B,D}$.
    Then
    \[
      (\bfcc_1(\cOpp), \bfcc_2(\cOpp),\bfcc_3(\cOpp))
      =\begin{cases}
        (\lll,\lll,\bfcc_1(\cOp))     & \text{if $t\geq 0$ is even}, \\
          (\lll+1,\lll-1,\bfcc_1(\cOp)) & \text{if $t\geq 0$ is odd},  \\
          (\bfcc_1(\cOp),\lll,\lll)     & \text{if $t<0$}.             \\
      \end{cases}
    \]
    We have
    \[
      \indsss(\sOp) =
      \begin{cases}
        \sOp\cdot (\frac{t}{2},\frac{t}{2})\cdot (0,0)
         & \text{if $t \geq 0$ is even}, \\
        2  \left(\sOp\cdot (\frac{t-1}{2},\frac{t-1}{2})\cdot (1,1) \right)
         & \text{if $t\geq 0$ is odd},   \\
        \sum_{\po+\qo=-t}
        (\Lambda_{(\po,\qo)}\sOp)\cdot (0,0)\cdot (\po,\qo)
         & \text{if $t<0$}.              \\
      \end{cases}
    \]
    \item Suppose $\star'\in \set{C,\wtC}$.
    Then
    \[
      (\bfcc_1(\cOpp), \bfcc_2(\cOpp),\bfcc_3(\cOpp))
      =\begin{cases}
        (\lll,\lll,\bfcc_1(\cOp))     & \text{if $t\geq 0$}, \\
        (\bfcc_1(\cOp),\lll,\lll) & \text{if $t < 0$ is even},  \\
        (\bfcc_1(\cOp),\lll+1,\lll-1)     & \text{if $t<0$ is odd}.             \\
      \end{cases}
    \]

    We have
    \[
      \indsss(\sOp) =
      \begin{cases}
        \sum_{t=\po+\qo}\sOp\cdot (\po,\qo)\cdot (0,0)
           & \text{if $t\geq 0$}, \\
        (\Lambda_{(-\frac{t}{2},-\frac{t}{2})}\sOp) \cdot (0,0)
        \cdot (-\frac{t}{2},-\frac{t}{2})
         & \text{if $t < 0$ is even},  \\
        \sum_{\po+\qo = 2}(\Lambda_{(-\frac{t-1}{2},-\frac{t-1}{2})}\sOp)
        \cdot (\po,\qo) \cdot (\frac{1-t}{2},\frac{1-t}{2})
            & \text{if $t<0$ is odd}.             \\
      \end{cases}
    \]
    \item Suppose $\star' \in \set{C^*,D^*}$.
    Note that  $\lll$ and $t$ are even integers in this case.
    Then
    \[
      (\bfcc_1(\cOpp), \bfcc_2(\cOpp),\bfcc_3(\cOpp))
      =\begin{cases}
        (\lll,\lll,\bfcc_1(\cOp))     & \text{if $t\geq 0$}, \\
        (\bfcc_1(\cOp),\lll,\lll)     & \text{if $t<0$}.             \\
      \end{cases}
    \]

    When $\star' = D^*$, we have
    \[
      \indsss(\sOp) =
      \begin{cases}
        \sum_{\po+\qo =\frac{t}{2}}
        \sOp\cdot (2\po,2\qo)\cdot (0,0)
         & \text{if $t \geq 0$}, \\
        \left(\Lambda_{(-\frac{t}{2},-\frac{t}{2})} \sOp \right)\cdot (0,0)
        \cdot (-\frac{t}{2},-\frac{t}{2})
         & \text{if $t < 0$}. \\
      \end{cases}
    \]

    When $\star'=C^*$, we have
    \[
      \indsss(\sOp) =
      \begin{cases}
        \sOp\cdot (\frac{t}{2},\frac{t}{2}) \cdot (0,0)
         & \text{if $t \geq 0$}, \\
         \sum_{\po+\qo = -\frac{t}{2}}
        \left(\Lambda_{(2\po,2\qo)} \sOp \right)\cdot (0,0)
        \cdot (2\po,2\qo)
         & \text{if $t < 0$}. \\
      \end{cases}
    \]
    \qed
  \end{enuma}
\end{lem}


\delete{
Retain the notation in \Cref{lem:ind}.
%Let $n\in \bN$ such that $n\geq \abs{\DDn(\cOpp)}$.
Let
\[
  \sF\colon \bigsqcup_{\sfss,\cO} \cK_{\sfss}(\cO)\longrightarrow
  \bigsqcup_{\sfss,\cO} \bZ[ \Nil_{\sfss}(\cO) ]
  \]
  be the map defined  by $\cE \mapsto \dim (\cE_{\bfee''})\sO''$ where
$\bfee''\in \sO'' \subset \cO''\cap \fpp_{\sfss''}$ and
$\cE\in \cK_{\sfss''}(\cO'')$.


\def\dblift{{\check \vartheta^2}}

\begin{lem}\label{lem:dblift}
  Let  $\cE' \in \cK_{\sfss'}(\cO)$.
%$\sOp \in \Nil_{\sfss'}(\cO)$ and
  \begin{enuma}
 \item Suppose $\star' \in \set{B,D}$ and  $2n = \abs{\DDn(\cOpp)}$.
 Let $\sfss = (\star, n,n)$ and
 \[
   \dblift(\cE') := \dlift_{\sfss}^{\sfss''}(\dlift_{\sfss'}^{\sfss}(\cE'))
   + \dlift_{\sfss}^{\sfss''}(\dlift_{\sfss'}^{\sfss}
   (\cE' \otimes {\det}))\otimes {\det}.
   \]

  Then
  \[
    \sF(\dblift(\cE')) = \begin{cases}
     2 \indsss(\sF(\cE')) & \text{if $t\geq 0$ is even},\\
     \indsss(\sF(\cE')) & \text{if $t\geq -1$ is odd}.\\
     %\indsss(\sO') & \text{if $t <-1$ is odd}.\\
    \end{cases}
  \]
  When $t <-1$, $\sF(\dblift(\cE')) \preceq \indsss(\sF(\cE'))$ and the inequality
  can be strict in general.

 \item Suppose $\star' \in \set{C,\wtC}$.
  %and  $m = \abs{\DDn(\cOpp)}+d$.
  Let $d\in \bN$ and $m = \abs{\DDn(\cOpp)}+d$.
  When $t\geq 0$ or $(-t,d) \in (2\bN)\times \set{0}$, define
 \[
   \dblift(\cE') :=
   \sum_{\substack{\sfss = (\star,p,q)\\ p+q=m}}
   \dlift_{\sfss}^{\sfss''}(\dlift_{\sfss'}^{\sfss}(\cE')).
   \]
   When $t\geq 0$ or $(t,d)=(-2,0)$,\\
  \[
    \sF(\dblift(\cE')) = (d+1) \indsss(\sF(\cE')).
  \]
   When $t< -2$ is even and $d=0$,
    $\sF(\dblift(\cE')) \preceq \indsss(\sF(\cE'))$  and the inequality can be strict
    in general
  \trivial[h]{
    When $t<0$ is odd, the operation $\dlift_{\sfss'}^{\sfss}$ is out
    of our scope.
  }

 \item Suppose $\star' = C^*$ and $2n = \abs{\DDn(\cOpp)}$.
  Let $\sfss = (\star,n,n)$ and
 \[
   \dblift(\cE') :=
      \dlift_{\sfss}^{\sfss''}(\dlift_{\sfss'}^{\sfss}(\cE')).
   \]
  When $t\geq -2$,
  \[
    \sF(\dblift(\cE')) = \indsss(\sF(\cE')).
  \]
  When $t <-2$, $\sF(\dblift(\cE')) \preceq \indsss(\sF(\cE'))$ and the inequality
  is strict in general.

 \item Suppose $\star' = D^*$.
    Let $d\in \bN$ and $2m = \abs{\DDn(\cOpp)}+2d$.
    When $t\geq 0$ or $(-t,d) \in (2\bN)\times \set{0}$, define
 \[
   \dblift(\cE') :=
   \sum_{\substack{\sfss = (\star,2p,2q)\\ 2p+2q=2m}}
   \dlift_{\sfss}^{\sfss''}(\dlift_{\sfss'}^{\sfss}(\cE')).
   \]
  When $t\geq 0$,
  \[
    \sF(\dblift(\cE')) = (d+1) \indsss(\sF(\cE')).
  \]
  When $t< 0$ and $d=0$,  $\sF(\dblift(\cE')) = \indsss(\sF(\cE'))$
  and the inequality can be strict in general. \qed
  \end{enuma}
\end{lem}
}



\section{Properties of painted bipartitions}
\def\ckfgg{{\check \fgg}}
\def\pcT{\cT^+}
\def\ncT{\cT^-}

In this section, we study properties of the descent maps. The results of this section are
crucial for the calculation of associated cycles in \Cref{sec:ACC}.


\subsection{Vanishing proposition}

Because of the following vanishing proposition, we will only need to consider
quasi-distinguished nilpotent orbits $\ckcO$ when $\star\in \set{C^*, D^*}$.

\begin{prop}\label{prop:CD*}
  Suppose that $\star\in \set{C^*, D^*}$. If the set $\PBP_\star(\ckcO)$ is nonempty, then $\check \CO$ is quasi-distinguished.
\end{prop}
\begin{proof}
  Suppose that $\tau=(\imath,\cP)\times(\jmath,\cQ)\times \alpha \in  \mathrm{PBP}_\star(\check \CO)$. If  $\star=C^*$, then  the definition of painted bipartitions implies that
 \[
 \bfcc_i(\imath)\leq \bfcc_i(\jmath) \qquad \textrm{for all } i=1,2,3, \cdots.
 \]
This forces that $\check \CO$ is quasi-distinguished.

 If  $\star=D^*$, then  the definition of painted bipartitions implies that
 \[
 \bfcc_{i+1}(\imath)\leq \bfcc_i(\jmath) \qquad \textrm{for all } i=1,2,3, \cdots.
 \]
This  also forces that   $\check \CO$ is quasi-distinguished.
 \end{proof}

\subsection{Tails of painted bipartitions}
\label{sec:tail}
In this subsection, we assume that $\star\in\{B, D, C^*\}$ and define the notion of ``tail'' of a painted bipartition.
Let $(\imath,\jmath) = (\imath_{\star}(\ckcO),\jmath_{\star}(\ckcO))$.
%Note that $l\geq l'$ if $\star\in \{B,C^*\}$,   and $l\geq l'+1$ if $\star=D$.
Put
\[
  \star_{\mathbf t}:= \begin{cases}
  D, & \ \text{ if $\star\in \{B,D\}$}; \\
  C^*, &\  \text{ if $\star=C^*$}.
\end{cases}
\quad
\text{and}
\quad
k := \begin{cases}
  \frac{\bfrr_{1}(\ckcO)-\bfrr_{2}(\ckcO)}{2} + 1 &
    \text{if $\star\in \{B,D\}$}; \\
\frac{\bfrr_{1}(\ckcO)-\bfrr_{2}(\ckcO)}{2} -1 &  \text{if $\star=C^*$}.
  \end{cases}
\]
We have $k = \bfcc_{1}(\jmath)-\bfcc_{1}(\imath)+1$,
$\bfcc_{1}(\jmath)-\bfcc_{1}(\imath)$,
and $\bfcc_{1}(\imath)-\bfcc_{1}(\jmath)$
when $\star = B,D,C^{*}$, respectively.

From the pair $(\star, \ckcO)$, we define a Young diagram $\ckcO_{\bftt}$ as follows:
\begin{itemize}
    \item If $\star \in \set{B,D}$,
then $\ckcO_{\bftt}$  consists of two rows with lengths $2k-1$ and $1$.
\item
If $\star =C^*$, then $\ckcO_{\bftt}$ consists of one row
with length  $2k+1$.
\end{itemize}
Note that in all three cases
 $\check \CO_{\mathbf t}$ has $\star_{\mathbf t}$-good parity and every element in $\PBP_{\star_\bftt}(\ckcO_\bftt)$ has the form
 \begin{equation}
 \label{tail0}
  \ytb{{x_1} , {x_2} , {\enon\vdots},{\enon{\vdots}},{x_k}  } \times \emptyset \times
  D,\qquad \qquad  \ytb{{x_1} , {x_2} , {\enon\vdots},{\enon{\vdots}},{x_k}  } \times \emptyset \times
  D\qquad\textrm{or}\qquad \emptyset \times  \ytb{{x_1} , {x_2} , {\enon\vdots},{\enon{\vdots}},{x_k}  } \times
 C^*,
\end{equation}
according to $\star=B, D$ or $C^*$, respectively. %Here $k$ can be zero if $\star = C^*$.

% Here $k=l-l'+1, l-l'$ or $l-l'$ respectively.
%\subsubsection{The case when $\star = B$}

\medskip


Let $ \tau=(\imath,\cP)\times(\jmath,\cQ)\times \alpha \in
\mathrm{PBP}_\star(\check \CO) $. The tail $\tau_\bftt$ of $\tau$ will be a painted bipartition in
$\PBP_{\star_\bftt}(\ckcO_\bftt)$, which is defined below case by case.

\subsubsection*{The case when $\star = B$:}
In this case, we define the tail $\tau_\bftt$ to be the first painted bipartition in \eqref{tail0} such that the multiset $\{x_1, x_2, \cdots, x_k\}$ is the
union of the multiset
\[
  \set{\cQ(j,1)| \bfcc_{1}(\imath)+1 \leq j \leq  \bfcc_{1}(\jmath) }
    % \cQ(\bfcc_{1}(\imath)+1,1),\cQ(\bfcc_{1}(\imath)+2,1),\cdots, \cQ(\bfcc_{1}(\jmath),1)}
\]
with the set
\[
  \begin{cases}
 \set{c}, &
 \qquad
  \text{if $\alpha = B^+$, and either $\bfcc_{1}(\imath)=0$ or $\cQ(\bfcc_{1}(\imath),1)\in \set{\bullet,s}$};  \\
 \set{s},&
  \qquad \text{if $\alpha = B^-$, and either $\bfcc_{1}(\imath)=0$ or $\cQ(\bfcc_{1}(\imath),1)\in \set{\bullet,s}$}; \\
%  \qquad\text{when } \alpha_\tau = B^-, \text{ and, } l'=0 \textrm{ or } \cQ_\tau(l',1)\in \set{\bullet,s},  \\
\set{\cQ(\bfcc_{1}(\imath),1)},&
\qquad
\text{otherwise.}
%\text{if $\bfcc_{1}(\imath)>0$ and $\cQ(\bfcc_{1}(\imath),1)\in \{r,d\}$.}
\end{cases}
\]

\subsubsection*{The case when $\star = D$:}
In this case, we define the tail $\tau_\bftt$ to be the second painted
bipartition in \eqref{tail0} such that the multiset $\{x_1, x_2, \cdots, x_k\}$
is the union of the multiset
\[
\set{\cP(j,1)| \bfcc_{1}(\jmath)+2 \leq j \leq \bfcc_{1}(\imath)}
\]
with the set
\[
\begin{cases}
    \set{c},                          &
    \ \text{if $\bfrr_2(\ckcO)=\bfrr_3(\ckcO)$,}                                                                         \\
                                      & \quad \text{$(\cP(\bfcc_{1}(\jmath)+1,1) ,\cP(\bfcc_{1}(\jmath)+1,2)) = (r,c)$ } \\
                                      & \quad \text{ and $\cP(\bfcc_1(\imath),1)\in \set{r,d}$};                                       \\
    \set{\cP(\bfcc_{1}(\jmath)+1,1)}, &
    \    \text{otherwise.}
  \end{cases}
\]


\subsubsection{The case $\star = C^*$:}
If $k=0$, we define the tail $\tau_{\bftt}$ to be
$\emptyset\times \emptyset \times C^{*}$.
If $k> 0$, we define the tail $\tau_\bftt$ to be the third painted bipartition in \eqref{tail0} such that
\[
  (x_1, x_2, \cdots, x_k)= (\cQ(\bfcc_{1}(\imath)+1,1),\cQ(\bfcc_{1}(\imath)+2,1),\cdots, \cQ(\bfcc_{1}(\jmath),1)).
\]


 When $\star \in \set{B,D}$, the symbol in the last box of the tail $\tau_\bftt\in \PBP_{\star_\bftt}(\ckcO_\bftt)$ will be important for us. We write $x_\tau$ for it, namely
\[
x_\tau := \cP_{\tau_\bftt}(k,1).
\]
 The following lemma is easy to check.

\begin{lem}\label{tailtip}
If $\star=B$, then
\[
\hspace{9em} x_\tau=s\Longleftrightarrow
\begin{cases}
  \alpha=B^-;\\
  \bfcc_1(\jmath)=0 \text{ or }\cQ(\bfcc_{1}(\jmath),1) \in\set{\bullet, s},
  \end{cases}
%\quad \textrm{if and only if}\quad \alpha=B^- \ \textrm{ and }\  \cQ(l,1) = s,
\]
and
\[
x_\tau=d \Longleftrightarrow
%\quad \textrm{if and only if}\quad
\cQ(\bfcc_{1}(\jmath),1) =d.
\]
If $\star=D$, then
\[
x_\tau=s\Longleftrightarrow \cP(\bfcc_{1}(\imath),1) = s,
\]
and
\[
x_\tau=d\Longleftrightarrow \cP(\bfcc_{1}(\imath),1) =d.
\]
\qed
\end{lem}


\subsection{Key properties of the descent map}
\label{sec:DDP}
In this subsection, we establish some relevant properties
of the descent of painted bipartitions.
These properties will be used extensively in \Cref{sec:ACC}.
Recall that $\ckcO$ is assumed to be
quasi-distinguished when $\star\in \set{C^*,D^*}$.

\medskip

The key properties of the descent map when $\star\in \set{C,\wtC,D^*}$ are
summarized in the following proposition.

\begin{prop}\label{prop:CC.bij}
Suppose that $\star \in \set{C,\wtC,D^*}$ and
consider the descent map
\begin{equation}\label{eq:DD.CC}
\nabla: \PBP_\star(\ckcO)\longrightarrow  \PBP_{\star'}(\ckcOp).
\end{equation}
\begin{enuma}
\item If
$\star=D^*$ or $\bfrr_1(\ckcO)>\bfrr_2(\ckcO)$, then
the map \eqref{eq:DD.CC}  is bijective.

\item If  $\star\in \{C,\widetilde C\}$ and $\bfrr_1(\ckcO)=\bfrr_2(\ckcO)$,
then the  map \eqref{eq:DD.CC} is injective and its image equals
\[
\Set{\tau'\in \PBP_{\star'}(\ckcOp)| x_{\tau'}\neq s}.
\]
\end{enuma}
\end{prop}

\begin{proof}
  We give the detailed proof of part (b) when $\star=\wtC$.
  The proofs in the other  cases are similar, which we skip.

%We assume that $\star = \wtC$ and $\bfrr_1(\ckcO)=\bfrr_2(\ckcO)$.
By the definition of descent map (see \eqref{eq:def.alphap}), we have a well-defined map
% , the map \eqref{eq:DD.CC} induces a map
\begin{equation}\label{eq:DD.CC1}
\DD: \Set{\tau\in \PBP_{\star}(\ckcO)| \cP_\tau(\bfcc_{1}(\imath),1)\neq c}\rightarrow \Set{\tau'\in \PBP_{\star'}(\ckcOp)|  \alpha_{\tau'}=B^+}.
\end{equation}
Suppose that $\tau'$ is an element  in the codomain of the map \eqref{eq:DD.CC1}. Similar to the proof of Lemma \ref{lemDDn1}, there is a unique element in $\tau:=\nabla^{-1}(\tau')\in \PBP_{\star}(\ckcO)$ such that
for all $i=1,2, \cdots,\bfcc_{1}(\imath)$,
\[
  \cP_{\tau}(i,1)\in \{\bullet, s\},
\]
and
for all $(i,j)\in \BOX(\tau')$,
\[
%\begin{equation}
     \cP_\tau(i,j+1)=\begin{cases}
    \bullet \textrm{ or } s,&\textrm{ if  $\ \cP_{\tau'}(i,j)\in \{\bullet, s\}$;} \smallskip \\
  \cP_{\tau'}(i,j),& \textrm{ if $\ \cP_{\tau'}(i,j)\notin \{\bullet, s\}$},\end{cases}
%   \end{equation}
\]
 and
 \[
%   \begin{equation}
     \cQ_\tau(i,j)=\begin{cases}
    \bullet \textrm{ or } s,&\textrm{ if  $\ \cQ_{\tau'}(i,j)\in \{\bullet, s\}$;} \smallskip \\
  \cQ_{\tau'}(i,j), & \textrm{ if $\ \cQ_{\tau'}(i,j)\notin \{\bullet, s\}$}.  \end{cases}
%   \end{equation}
\]
Note that $\tau$ is in the domain of \eqref{eq:DD.CC1}. It is then routine to check that the map
\[
  \nabla^{-1}: \Set{\tau'\in \PBP_{\star'}(\ckcOp)|  \alpha_{\tau'}=B^+}
  \rightarrow  \Set{\tau\in \PBP_{\star}(\ckcO)| \cP_\tau(\bfcc_{1}(\imath),1)\neq c}
\]
and the map \eqref{eq:DD.CC1} are inverse to each other. Hence the map \eqref{eq:DD.CC1} is bijective.

\smallskip

Similarly, the map
\begin{equation}\label{eq:DD.CC2}
  \nabla: \Set{\tau\in \PBP_{\star}(\ckcO)| \cP_\tau(\bfcc_{1}(\imath),1)= c}
  \rightarrow \Set{\tau'\in \PBP_{\star'}(\ckcOp)|
 \begin{array}{l}
  \alpha_{\tau'}=B^- \text{ and } \\
  \cQ_{\tau'}(\bfcc_{1}(\imath),1)\in\{r,d\}
 \end{array}
  }
\end{equation}
is well-defined, and shown to be bijective by constructing its inverse.
In view of \Cref{tailtip}, this
proves the proposition for the case concerned.
\trivial[h]{
  Suppose $\star = \wtC$ and $\bfrr_{1}(\ckcO) >\bfrr_{2}(\ckcO)$. Then
  $\bfcc_{1}(\imath)>\bfcc_{1}(\jmath)$ and the ``inverse'' of descent is
  constructed in an obvious way such that
  $\cP(\bfcc_{1}(\imath),1) = c\Leftrightarrow \alpha = B^{-}$.


  Suppose $\star = C$ and $\bfrr_{1}(\ckcO) -\bfrr_{2}(\ckcO)$. Then
  $\bfcc_{1}(\imath)=\bfcc_{1}(\jmath)+1$. So
  $\cP_{\tau}(\bfcc_{1}(\imath),1)\neq s$. Hence
  $\cP_{\tau'}(\bfcc_{1}(\imath),1)= \cP_{\tau}(\bfcc_{1}(\imath),1) \neq s$.
  The ``inverse'' of descent is
  constructed in an obvious way.

  Suppose $\star = D^{*}$. Then
  $\bfcc_{1}(\imath)\geq \bfcc_{1}(\jmath)+1$.
  The ``inverse'' of descent is
  constructed in an obvious way.
}
\end{proof}

\medskip


Now we discuss the case when $\star \in \set{B,D,C^*}$, which is more
involved.
%The following equation of signatures will be crucial in our computation of the local system in the next section.
% \begin{prop}\label{prop:D.sign}
%   \begin{enuma}
%     \item
%     Suppose $k\in \bN^{+}$, and either
%     \begin{itemize}
%       \item $\star = B$ and $\ckcO$ consists of one row with length $2k$,
%              or
%       \item $\star = D$ and $\ckcO$ consists of two rows with lengths $2k-1$ and
%             $1$.  %(so $\cO= (1^{2k})_{D}$), or
%     \end{itemize}
% Then $\cO = (1^{\bfrr_{1}(\ckcO)+1})_{\star}$ and the following map is bijective:
% \[
%   \begin{array}{rcl}
%     \PBP_{\star}(\ckcO)& \longrightarrow& \Set{
%     (p,q,\varepsilon)|
%     \begin{minipage}{13em}
%       $p,q \in \bN$ such that $p+q=\abs{\cO}$,\\
%       % $\varepsilon \in \bZ/2\bZ$
%       $\varepsilon=1$ if $pq=0$, and \\
%       $\varepsilon\in \set{0,1}$ otherwise.
%     \end{minipage}
%     },\\
%     \tau &  \mapsto & (p_{\tau},q_{\tau}, \varepsilon_{\tau}).
%   \end{array}
% \]
% \item
% Suppose $k\in \bN$, $\star = C^{*}$ and $\ckcO$ consists of one row with length $2k+1$.
% The following map is bijective:
% \[
%   \PBP_{\star}(\ckcO) \longrightarrow \set{(2p,2q)|p,q \in \bN \text{ such that } p+q=k}, \quad \tau \mapsto (p_{\tau},q_{\tau}).
% \]
% \end{enuma}
% \qed
% \end{prop}

% Recall that $\cO$ denotes the Barbasch-Vogan dual of $\ckcO$. % We also consider it as a Young diagram.
The following lemma deals with the case when the Barbasch-Vogan dual $\cO$ of
$\ckcO$ has only one column. This is the most basic case, which can be verified
easily.

\begin{lem}
\label{lem:D.sign}
  Suppose $k\in \bN^{+}$, $\star\in \set{B,D,C^{*}}$. Let $\ckcO$ be the
    $\star$-good parity orbit such that   $\bfrr_{3}(\ckcO)=0$ and
    \[
      (\bfrr_{1}(\ckcO), \bfrr_{2}(\ckcO))
      =
      \begin{cases}
        (2k-2,0) & \text{if } \star=B,\\
        (2k-1,1) & \text{if } \star=D,\\
        (2k-1,0) & \text{if } \star=C^{*}.\\
      \end{cases}
    \]
    Then $\ckcO$ is the regular nilpotent orbit % in $\check \fgg$
    and $\cO = (1^{\bfrr_{1}(\ckcO)+1})_{\star}$.

When $\star\in \set{B,D}$, the following map is bijective:
\[
  \begin{array}{rcl}
    \PBP_{\star}(\ckcO)& \longrightarrow& \Set{
    (p,q,\varepsilon) \in \bN\times\bN\times \bZ/2\bZ|
    \begin{array}{l}
    p+q=\abs{\cO} \text{ and }\\
    \varepsilon=1 \text{ if } pq=0\\
    \end{array}
    },\\
    \tau &  \mapsto & (p_{\tau},q_{\tau}, \varepsilon_{\tau}).
  \end{array}
\]

When $\star = C^{*}$, the following map is bijective:
\[
  \PBP_{\star}(\ckcO) \longrightarrow \set{(p,q)|p,q \in 2\bN \text{ and }
    p+q=\abs{\cO} }, \quad \tau \mapsto (p_{\tau},q_{\tau}).
\]
\end{lem}


For every painted bipartition $\tau\in \PBP_{\star}(\ckcO)$, write
\[
  \ssign(\tau):=(p_\tau, q_\tau).
\]
The key properties of the descent map when $\star\in
\set{D,B,C^*}$ are summarized in the following lemma.

\begin{lem}
\label{lem:delta}
Suppose that $\star \in \set{D,B,C^*}$ and $\bfrr_2(\ckcO)>0$. Write $\ckcOpp := \ckDD(\ckcO')$ and consider the map
\begin{equation}\label{eq:delta}
  \delta  \colon \PBP_\star(\ckcO)\longrightarrow
    \PBP_\star(\ckcOpp)\times \PBP_{\star_\bftt}(\ckcO_\bftt),
    \qquad \tau \mapsto (\DD^2(\tau),\tau_\bftt).
\end{equation}
\begin{enuma}
\item Suppose that
$\star = C^*$ or $\bfrr_2(\ckcO)>\bfrr_3(\ckcO)$. Then the map \eqref{eq:delta} is bijective, and for every $\tau\in  \PBP_\star(\ckcO) $,
    % We have the following equation of signatures.
\begin{equation}\label{eq:sign.D}
\ssign(\tau)
=(\bfcc_2(\cO),\bfcc_2(\cO))+\ssign(\DD^2(\tau))+\ssign(\tau_\bftt).
\end{equation}

\item Suppose that  $\star \in \set{B,D}$ and $\bfrr_2(\ckcO)=\bfrr_3(\ckcO)>0$.
Then the map \eqref{eq:delta} is an  injection and its  image equals
\begin{equation}\label{eq:delta.I}
    \Set{ (\tau'',\tau_0)  \in \PBP_\star(\ckcOpp)\times \PBP_D(\ckcO_\bftt)  |
    \begin{array}{l}
        \text{either
    $x_{\tau''} = d$, or} \\
    \text{$x_{\tau''}\in \set{r,c}$  and
    $\cP_{\tau_0}^{-1}(\set{s,c})\neq \emptyset$}
    \end{array}
}.
\end{equation}
Moreover,  for every $\tau\in  \PBP_\star(\ckcO) $,
\begin{equation}\label{eq:sign.GD}
\ssign(\tau)
=(\bfcc_2(\cO)-1,\bfcc_2(\cO)-1)+\ssign(\DD^2(\tau))+\ssign(\tau_\bftt).
\end{equation}
\end{enuma}
\end{lem}

\begin{proof}
  We give the detailed proof of part (b) when $\star=B$.
  The proofs in the other  cases are similar, which we skip.
  Let $\tau = (\imath,\cP)\times (\jmath,\cQ)\times \alpha$ and $\tau'' = \DD^{2}(\tau)$.
  % $\tau' = \DD(\tau)  = (\imath',\cP')\times (\jmath',\cQ')\times \alpha$.
  % $\tau'' = \DD(\tau')  = (\imath'',\cP'')\times (\jmath'',\cQ')\times \alpha$.

  According to the descent algorithm (see \Cref{sec:comb}), we have
  \begin{equation}\label{eq:delta.1}
    \begin{split}
    \cP(i,j) &= \begin{cases}
      \bullet & \text{if } i< \bfcc_{1}(\imath), j=1;\\
      \cP_{\tau''}(i,j-1) & \text{if } i< \bfcc_{1}(\imath) \text{ or } j>1;\\
      \end{cases}\\
    \cQ(i,j) &= \begin{cases}
      \bullet & \text{if } i< \bfcc_{1}(\imath), j=1;\\
      \cQ_{\tau''}(i,j-1) & \text{if } i< \bfcc_{1}(\imath) \text{ or }  j>2;\\
      \end{cases}
    \end{split}
  \end{equation}
  We label the boxes in $\cP$ and $\cQ$ which are not considered in \eqref{eq:delta.1} as the following:
  \begin{equation}\label{eq:ydelta}
  \tau: \hspace{1em}
  \ytb{
    {x_{0}},
    \none,\none,\none,\none}
  \times
  \ytb{
    {x_{1}}{x''},
    {x_{2}},{\enon{\vdots}},{\enon{\vdots}},{x_{k}}}
  \times
  \alpha
   \quad
  \mapsto
  \quad
  \tau'': \ytb{{\none[\emptyset]},\none,\none,\none,\none}
  \times \ytb{\none{y''},\none,\none,\none,\none}
  \times \alpha'',
\end{equation}
where $k := \bfcc_{1}(\jmath)-\bfcc_{1}(\imath)+1$, $x_{0}$, $x_{1}$ and $y''$ have
coordinate $(\bfcc_{1}(\imath),1)$ in the corresponding painted Young diagram.
It is easy to check that $x'' = y''$ always holds.
Also note that $\bfcc_{2}(\cO) = 2\bfcc_{1}(\imath)$.
% We have the following two cases, $x_{1}\in \set{r,d}$ or
% $x_{1}\in \set{\bullet,s}$.
To establish \eqref{eq:sign.GD}, we examine case by case according to the symbol $x_{1}$.

When $x_{1} = s$, we have $(x_{0}, \alpha'') = (c,B^{-})$.
So
\begin{equation}\label{eq:sign.GD2}
  \begin{split}
    & \Sign(\tau) - \Sign(\tau'') \\
    &= (2\bfcc_{1}(\imath)-2,2\bfcc_{1}(\imath)-2)
    + (1,1) + (0,2) - (0,1) + \Sign(\alpha) + \Sign(x_{2}\cdots x_{k})\\
    & = (\bfcc_{2}(\cO)-1, \bfcc_{2}(\cO)-1)+ \Sign(\tau_{\bftt}).
  \end{split}
\end{equation}
Here $\Sign$ counts the signature of various symbols in the same way as
\eqref{ptqt}. In the second line of the above formula, the terms count the
signatures of extra bullets, $x_{0}$, $x_{1}$, $\alpha''$, $\alpha$ and
$x_{1}\cdots x_{k}$ respectively.

When $x_{1} = \bullet$, we have $(x_{0}, \alpha'') = (\bullet,B^{+})$. When
$x_{1} \in \set{r,d}$, we have $x'' = y''= d$, $(x_{0}, \alpha'') = (c,\alpha)$.
In these two cases, the signature formula can be checked similarly.

\smallskip

It is clear that the image of $\delta$ is in \eqref{eq:delta.I} and is
straightforward to construct the inverse of $\delta$ from \eqref{eq:delta.I} to
$\PBP_{\star}(\ckcO)$. To illustrate the idea, we give the detail construction
of $\tau:=(\imath,\cP)\times (\jmath,\cQ)\times
\alpha:=\delta^{-1}(\tau'',\tau_{0})$ when $\cP_{\tau_{0}}(k,1)=c$. We set
$\alpha:=B^{-}$. Most of the entries in $\cP$ and $\cQ$ are already determined
by
\eqref{eq:delta.1}. We refer to \eqref{eq:ydelta} for the labeling of the rest
of the entries in $\cP$ and $\cQ$. We set  $x'':= \cQ_{\tau''}(\bfcc_{1}(\imath),1)$,
\[
(x_{0},x_{1}) := \begin{cases}
  (\bullet,\bullet) & \text{if } \alpha''= B^{+},\\
  (c,s) & \text{if } \alpha''= B^{-},
\end{cases}
\quad \text{and}\quad
x_{i} := \cP_{\tau_{0}}(i-1,1) \quad \text{for } i = 2,3,\cdots,k.
\]
This gives $\tau := \delta^{-1}(\tau'',\tau_0)$ and finishes the proof for the case concerned.
\trivial[h]{
In fact, it suffices to check the proposition for the orbit $\ckcO$
consists of three rows with lengths $2k$, $2$, and $2$.
}
\trivial[h]{
  Suppose $\star =  C^{*}$.
  Then $\tau$ is obtained by attaching a column of $\bfcc_{1}(\imath)$
  bullets on the left of $\cP$, and a column of  $\bfcc_{1}(\imath)$
  bullets concatenated with $\tau_{\bftt}$ on the left of $\cQ$. The bijectivity is clear. The
  signature formula follows from $2\bfcc_{1}(\imath)  = \bfcc_{2}(\cO)$.


  Suppose $\star =  D$ and $\bfrr_{2}(\ckcO)>\bfrr_{3}(\ckcO)$.
  Then $\tau$ is obtained by attaching a column of $\bfcc_{1}(\jmath)$
  bullets concatenated with $\tau_{\bftt}$ on the left of $\cP$, and a column of  $\bfcc_{1}(\jmath)$
  bullets on the left of $\cQ$. The bijectivity is clear. The
  signature formula follows from $2\bfcc_{1}(\jmath)  = \bfcc_{2}(\cO)$.

  % Suppose $\star =  D$ and $\bfrr_{2}(\ckcO)=\bfrr_{3}(\ckcO)$.
  % Then $\tau$ is obtained by attaching a column of $\bfcc_{1}(\jmath)$
  % bullets concatenated with $\tau_{\bftt}$ on the left of $\cP$, and a column of  $\bfcc_{1}(\jmath)$
  % bullets on the left of $\cQ$. The bijectivity is clear. The
  % signature formula follows from $2\bfcc_{1}(\jmath)  = \bfcc_{2}(\cO)$.
}
\end{proof}

Combining the injectivity results in \Cref{lem:delta} and
\Cref{prop:CC.bij}, we get the following lemma.
\begin{lem}\label{cor:D.inj1}
Suppose that $\star \in \set{D,B,C^*}$ and $\bfrr_{2}(\ckcO)>0$.
Then the map
\begin{equation}\label{eq:D.BD}
  \delta' \colon \PBP_{\star}(\ckcO)\longrightarrow
   \PBP_{\star'}(\ckcOp)\times \PBP_{\star_\bftt}(\ckcO_\bftt)
   \qquad \tau \mapsto (\DD(\tau), \tau_\bftt)
\end{equation}
% \begin{equation}\label{eq:D.BD}
%   \delta' \colon \PBPes(\ckcO)\longrightarrow
%    \PBPesp(\ckcOp)\times \PBP_{\star_\bftt}(\ckcO_\bftt)
%    \qquad \tau \mapsto (\DD(\tau), \tau_\bftt)
% \end{equation}
is injective. Moreover, \eqref{eq:D.BD} is bijective
unless $\star\in \set{D,B}$ and $\bfrr_2(\ckcO)=\bfrr_3(\ckcO)>0$.
% when $\star = C^*$
\end{lem}

Combining Lemmas \ref{lem:D.sign}, \ref{lem:delta} and \ref{cor:D.inj1}, we obtain the following proposition.
\begin{prop}\label{cor:dpinj}
If $\star \in \set{B, D,C^*}$, then the map
\begin{equation}
  \begin{array}{rcl}
   \PBP_{\star}(\ckcO)&\rightarrow&
   \PBP_{\star'}(\ckcOp)\times \BN\times \bN\times \Z/2\Z, \smallskip\\
   \tau& \mapsto & (\DD(\tau), p_\tau, q_\tau, \varepsilon_\tau)
   \end{array}
\end{equation}
is injective. \qed
\end{prop}





\section{Properties of associated cycles: proofs}
\label{sec:ACC}

\def\dsign{{}^d\mathrm{Sign}}

\def\acm{\cL}
\def\acme{\tilde{\cL}}
\def\dlifttso{{\check \vartheta}_{\sfss_{\tau'},\cO'}^{\sfss_{\tau},\;\cO}}
\def\DDtss{\DD_{\sfss_{\tau'}}^{\sfss_{\tau}}}
\def\taut{\tau_{\bftt}}

In this section, we study properties of the associated cycles $\AC(\uptau)$
defined in \Cref{subsecass}.
Let $\uptau=(\tau,\wp)\in \PBPes(\ckcO)$, $\uptau' := (\tau',\wp'):=\DD(\uptau)$ and  $\ckcOp:=\ckDD(\ckcO)$.

\subsection{Computing $\AC(\uptau)$ via $\ac_{\uptau}$}

\begin{lem}\label{lem:actau}
  The cycle $\AC(\uptau)$ is an element in $\bN[\AOD_{\sfss_\tau}(\cO)]$.
\end{lem}
\begin{proof}
We prove by induction on the number of rows in $\ckcO$. The lemma clearly holds
for the initial case by the definition in \Cref{subsecass}. For the general
case, we have $\AC(\uptaup)\in \bN[\AOD_{\sfss_{\tau'}}(\cO')]$ by the induction
hypothesis and $\cO$ is regular for $\DDtss$ by \Cref{lem78}. Therefore
$\AC(\uptau)\in \bN[\AOD_{\sfss_\tau}(\cO)]$  by \Cref{lem:aod}
\end{proof}


Let $\ac_{\uptau}\in \bN[\MYD_\star]$ be the element corresponding to the
associated cycle $\AC(\uptau)$, as in \Cref{sec:MYD}.
The study of $\AC(\uptau)$ is now translated to
the corresponding combinatorial problem for $\ac_{\uptau}$.
Via \Cref{lem:clift}, $\ac_{\uptau}$ may also be
defined combinatorially as follows:
\begin{itemize}
\item
When $\abs{\ckcO}=0$,
\[
  \ac_{\uptau} :=
  \begin{cases}
   (1^{\Sign(\tau)})_{\star}\lotimes(0,1), & \text{if } \star=B;\\
   (1^{(0,0)})_{\star}, & \text{otherwise}.
  \end{cases}
\]
\item
When $\abs{\ckcO} >0$,
\[
\ac_{\uptau} :=
\begin{cases}
  \dlifttso(\ac_{\uptaup}\lotimes (\varepsilon_{\wp},\varepsilon_{\wp}))
  & \text{if } \star\in \set{C,\wtC};\\
  \dlifttso(\ac_{\uptaup})\lotimes (0,\varepsilon_{\tau})
  & \text{if } \star \in \set{B,D}; \\
  \dlifttso(\ac_{\uptaup})
  & \text{if } \star \in \set{C^*,D^{*}}. \\
\end{cases}
\]
Here $\dlifttso$ is defined by \eqref{eq:LCD} and \eqref{eq:LDC}.
\end{itemize}
For $\star\in \set{B,D}$ and $\cE\in \bZ[\MYD_{\star}]$, we write
\[
\cE^{+} := \Lambda_{(1,0)}(\cE) \AND \cE^{-} := \Lambda_{(0,1)}(\cE).
\]
We adopt the convention that $\cE\cdot \emptyset = 0$ for $\cE\in\bZ[\MYD]$.
We have the following more explicit formulas for $\ac_{\uptau}$.
\begin{lem}\label{lem:dlift}
  Suppose that $\ckcO$ such that $\bfrr_{2}(\ckcO)>0$. Let
  $\uptau = (\tau,\wp)\in \PBPes(\ckcO)$ and $\uptaup = \DD(\uptau)$.
\begin{enuma}
  \item
  Suppose that $\star\in \set{C,\wtC}$.  Then
  \begin{equation}\label{eq:C}
    \ac_{\uptau}=
    \begin{cases}
      \maltese^{y}\big( (\ac_{\uptau'}\lotimes (\varepsilon_{\wp},\varepsilon_{\wp}))
      \cdot (n_{0},n_{0})\big), & \text{if } \bfrr_{1}(\ckcO)>\bfrr_{2}(\ckcO);\\
      %\maltese^{y}\big( (\pac{\uptau'} + \nac{\uptaup})\cdot (0,0) \big),
      \maltese^{y} (\pac{\uptau'} \cdot (0,0)) + \maltese^y(\nac{\uptaup}\cdot (0,0)),
      & \text{if } \bfrr_{1}(\ckcO)=\bfrr_{2}(\ckcO);
    \end{cases}
  \end{equation}
  Here %\begin{itemize}
   % \item
    $n_{0} = \frac{\bfcc_{1}(\cO)-\bfcc_{2}(\cO)}{2}$,
   % \item
    $y = \frac{p_{\tau'}-q_{\tau'}}{2}$ if $\star=C$ and
    $y = \frac{p_{\tau'}-q_{\tau'}-1}{2}$ if $\star=\wtC$.
  %\end{itemize}
  \item
  Suppose that $\star\in \set{B,D}$ and $\bfrr_{2}(\cO)>\bfrr_{3}(\cO)$. Then
  \begin{equation}\label{eq:BD}
    \ac_{\uptau}= \big((\maltese^{y}\ac_{\uptau'})\cdot (p_{\tau_{\bftt}},q_{\tau_{\bftt}})\big)
    \lotimes (0,\varepsilon_{\tau})
  \end{equation}
  Here %\begin{itemize}
    $y = \frac{p_{\tau}-q_{\tau}+1}{2}$ if $\star=B$ and
    $y = \frac{p_{\tau}-q_{\tau}}{2}$ if $\star=D$.
  %where $y$ is an integer determined by $\Sign(\uptau)$.
  In particular,
  \begin{equation}\label{eq:TBD}
    \cE(1) = \ac_{(\tau_{\bftt},\emptyset)}(1)
  \end{equation}
  for each $\cE\in \MYD_\star$ has nonzero multiplicity in $\ac_{\uptau}$.
  \item
  Suppose that $\star\in \set{B,D}$ and $\bfrr_{2}(\cO)=\bfrr_{3}(\cO)$.
  Then
  \begin{equation}\label{eq:BD2}
    \ac_{\uptau} =
    \big((\maltese^t (\pac{\uptaupp}\cdot (0,0)) \cdot \pcT_{\tau}
    +  (\maltese^t(\nac{\uptaupp}\cdot (0,0)) ) \cdot \ncT_{\tau}\big)
    \lotimes(0,\varepsilon_{\tau}).
  \end{equation}
  Here $\uptaupp = \DD^{2}(\uptau)$ and
  \[
  \begin{split}
    t &  = \frac{\abs{\ssign(\tau_{\bftt})}}{2},\\
    \pcT_{\uptau}& = \begin{cases} (p_{\tau_{\bftt}},q_{\tau_{\bftt}}-1) &
       \text{when
      } q_{\tau_{\bftt}}-1\geq 0 , \\
      \emptyset & \text{otherwise},
    \end{cases}\\
    \ncT_{\uptau}& = \begin{cases} (p_{\tau_{\bftt}}-1,q_{\tau_{\bftt}}) &
       \text{when
      } p_{\tau_{\bftt}}-1\geq 0  ,\\
      \emptyset & \text{otherwise}.
    \end{cases}\\
  \end{split}
\]
\item
  Suppose that $\star = D^{*}$.  Then
  \[
    \ac_{\uptau}=\ac_{\uptau'} \cdot (n_{0},n_{0}),
  \]
  where $n_{0} = \frac{\bfcc_{1}(\cO)-\bfcc_{2}(\cO)}{2}$.
\item
  Suppose that $\star = C^{*}$.  Then
  \[
    \ac_{\uptau}=\ac_{\uptau'} \cdot (p_{\tau_{\bftt}},q_{\tau_{\bftt}}).
\]
\end{enuma}
\end{lem}
\begin{proof}
 The lemma follows from \eqref{eq:LCD}, \eqref{eq:LDC}, and
 the signature formulas \eqref{eq:sign.D}, \eqref{eq:sign.GD}
 in \Cref{lem:delta}.
\end{proof}



\trivial[h]{Hint of the proof of (c). It suffice to consider the most left three
  columns of the peduncle part: this part has signature $\lsign(\cT_{\uptau})
  +(1,1) = \ssign(\tau_{\bftt}x_{\uptaupp})=
  \ssign(\tau_{\bftt})+\lsign(\cD_{\uptaupp})$. Therfore,
  \[\ssign(\tau_{\bftt}) = \lsign(\cT_{\uptau})-\lsign(\cD_{\uptaupp})
  + (1,1) = \lsign(\ac_{\uptau})-\lsign(\ac_{\uptaupp})+(1,1).
  \]
}

\subsection{Properties of $\ac_{\uptau}$}
\label{sec:ac}
In this subsection, we summarize the key properties of $\ac_{\uptau}$. The proofs will be given in the next three sections, and
will go by induction on the number of columns in $\cO$.

Recall that $\ckcO$ is quasi-distinguished when $\star = C^*, D^*$. See \Cref{prop:CD*}. All the claims of this subsection for $\star\in \set{C^{*},D^{*}}$ are relatively straightforward to verify, and as such we shall only give the detailed proofs for the non-quaternionic cases in the next three sections.

\smallskip


We say an element $\cA$ in $\bZ[\MYD_\star]$ is multiplicity free if $\sF(\cA)$ is a sum of
distinct elements in $\SYD_\star$, where $\sF$ is defined in \eqref{eq:F}.

The following lemma will imply the non-vanishing and multiplicity freeness of
$\AC(\tau)$, as in \Cref{thmac1}.

\begin{lem}\label{lem:ac0}
  For each $\uptau\in \PBPes(\ckcO)$,
  $\ac_{\uptau}\neq 0$ and $\ac_\uptau$ is multiplicity free.
\end{lem}

\smallskip

The following three lemmas will imply \Cref{thmac2}, \Cref{thmac3}, \Cref{thmac4}, and
\Cref{thmac5}.
% and \Cref{thmac0} part (b) and (c).

Define the map
  \[
      \acm \colon
      \PBPes( \ckcO )\longrightarrow \bZ[\MYD_{\star}(\cO)],
      \quad \uptau\mapsto \ac_{\uptau}.
      %\AC\colon \PBPes( \cOp )\longrightarrow \cK_{\star}(\cO).
    \]

\begin{lem}\label{lem:C*}
Suppose that $\star = C^{*}$. Then the map $\acm$ 
        is injective with image $\MYD_{\star}(\cO)$.
\end{lem}
\begin{lem}\label{lem:C}
  Suppose that $\star \in \set{C,\wtC,D^{*}}$ and $\ckcO\neq \emptyset$. 
  \begin{enuma}
    \item Let $\uptau_{1} = (\tau_{1},\wp_{1})$ and
    $\uptau_{2}=(\tau_{2},\wp_{2})$ be two elements in $\PBPesp(\ckcOp)$ such
    that $\ac_{\uptau_{1}} =  \ac_{\uptau_{2}}$. Then
    \[
      \Sign(\tau'_{1})=\Sign(\tau'_{2}) \quad \text{and} \quad
    \varepsilon_{\wp_{1}}=\varepsilon_{\wp_{2}}, \]
    where $\tau'_{i} :=
    \DD(\tau_{i})$ for $i=1,2$.
    % \item Let $\uptau_{1} = (\tau_{1},\wp_{1})$ and $\uptau_{2}=(\tau_{2},\wp_{2})$ be two distinct elements in
    % $\PBPesp(\ckcOp)$. Then
    % \begin{itemize}
    %   \item either $\oAC{\uptau_{1}}\neq \oAC{\uptau_{2}}$;
    %   \item or $(\tau'_{1},\varepsilon_{\wp_{1}})\neq (\tau'_{2},\varepsilon_{\wp_{2}})$
    %         and $\Sign(\tau'_{1})=\Sign(\tau'_{2})$ where
    %         $\tau'_{i} := \DD(\tau_{i})$ for $i=1,2$.
    % \end{itemize}
    \item When $\ckcO$ is weakly-distinguished, then the map $\acm$ is injective.
    \item When $\ckcO$ is quasi-distinguished, then the image of the map $\acm$ is $\MYD_{\star}(\cO)$.
    %$\AC( \PBPes( \cOp )\longrightarrow K_{\star}(\cO)$ is injective.
  \end{enuma}
\end{lem}


\begin{lem}\label{lem:BD}
Suppose $\star \in \set{B,D}$ and $(\star, \check \CO)\neq (D, \emptyset)$. Define the map
 \[
    \acme \colon
    \PBPes( \ckcO )\times \bZ/2\bZ \longrightarrow \bZ[\MYD_{\star}(\cO)],\quad
    (\uptau,\epsilon) \mapsto \ac_{\uptau}\otimes (\epsilon, \epsilon).
  \]
  \begin{enuma}
  \item Let $\uptau_i=(\tau_i, \wp_i)\in \PBPe_\star(\ckcO)$ and
  $\epsilon_i\in \Z/2\Z$ ($i=1,2$) such that
  \[
    \ac_{\uptau_1}\lotimes (\epsilon_{1},\epsilon_{1})= \ac_{\uptau_2}\lotimes( \epsilon_{2},\epsilon_{2} ).
  \]
  Then
  \[
    \epsilon_1=\epsilon_2,\quad \varepsilon_{\tau_1}=\varepsilon_{\tau_2} \AND \ac_{\uptau_{1}}=\ac_{\uptau_{2}}.
  \]

  \item When $\ckcO$ is weakly-distinguished, then the map $\acme$ is injective.
  \item When $\ckcO$ is quasi-distinguished, then the image of the map $\acme$ is $\MYD_{\star}(\cO)$.
\end{enuma}
\end{lem}



\smallskip

We will also need to use the following technical lemmas.
\begin{lem}\label{lem:BD2}
Suppose that $\star \in \set{B,D}$ and $\ckcO \neq \emptyset$.
\begin{enuma}
  \item If $x_{\uptau}=s$, then $\ac_{\uptau}^{+} =0$ and $\ac_{\uptau}^{-}=0$.
  \item If $x_{\uptau}\in \set{r,c}$, then $\ac_{\uptau}^{+} \neq 0$ and $\ac_{\uptau}^{-}=0$.
  \item If $x_{\uptau}=d$, then $\ac_{\uptau}^{+} \neq 0$ and
  $\ac_{\uptau}^{-}\neq 0$.
\end{enuma}
\end{lem}

\begin{lem}\label{lem:BD3}
Suppose that $\star \in \set{B,D}$, $\ckcO \neq \emptyset$ and $\ckcO$ is weakly-distinguished.
\begin{enuma}
    \item The map
    \begin{equation}\label{eq:up}
      \begin{array}{rcl}
        \Upsilon_{\ckcO} \colon \set{\ac_{\uptau}|\uptau\in \PBPes(\ckcO)
        \text{ s.t. } \pac{\uptau}\neq 0 }
        & \rightarrow &  \bZ[\MYD]\times \bZ[\MYD] \\ %\set{(\pac{\uptau},\nac{\uptau})|\pac{\uptau}\neq 0} \\
        \ac_{\uptau} &\mapsto & (\pac{\uptau},\nac{\uptau})
      \end{array}
    \end{equation}
    is injective.
    % \footnote{In fact, we only need to know whether $\nac{\uptau}$ is
    %   non-zero or not.}
  \item When $\bfrr_{1}(\ckcO)> \bfrr_{3}(\ckcO)$, the maps
    % $\pUpsilon_{1}:=\pr_{1}\circ \Upsilon$
    \[
      \begin{array}{rcl}
        \pUpsilon_{\ckcO} \colon \set{\ac_{\uptau}|\uptau\in \PBPes(\ckcO)
        \text{ s.t. } \pac{\uptau}\neq 0 } &\rightarrow  & \set{\pac{\uptau}}\\
       \ac_{\uptau} & \mapsto & \pac{\uptau},\qquad \AND\\
      % \end{array} \quad \text{and}
      % \begin{array}{rcl}
        \nUpsilon_{\ckcO} \colon \set{\ac_{\uptau}|\uptau\in \PBPes(\ckcO)
        \text{ s.t. } \pac{\uptau}\neq 0} &\rightarrow & \set{\nac{\uptau}}\\
       \ac_{\uptau} &\mapsto & \nac{\uptau}
     \end{array}
   \]
    are injective.
    \item Suppose that $\bfrr_{1}(\ckcO)> \bfrr_{3}(\ckcO)$ and $x_{\tau}=d$.
    Then there is $\cE\in \MYD$ with a non-zero coefficient in $\nac{\uptau}$ such that $\cE\sqsupseteq (1,0)$.
    In particular, there does not exist $\uptau_2\in \PBPes(\ckcO)$ such that
    \begin{equation}\label{eq:pnac}
     \pac{\uptau_2} \cdot (0,0) = \maltese (\nac{\uptau}\cdot (0,0)).
    \end{equation}
  \end{enuma}
\end{lem}


\subsection{Establishing properties of $\ac_{\uptau}$: the initial cases}
\label{sec:init}
Suppose $\bfcc_{1}(\cO)=0$.
Then the group $G_{\sfss_\tau,\bC}$ is the trivial group and everything are
clear.

\smallskip

We now assume that $\bfcc_{1}(\cO)>0$ and $\bfcc_{2}(\cO)=0$. Suppose that
$\star \in \set{C,\wtC}$.
%$\star \in \set{C,\wtC,D^*}$.
There there is $k\in\bN^{+}$ such that $\cO =
(1^{2k})_{\star}$ and $\ckcO$ is the regular nilpotent orbit in $\ckfgg$.
% \[ \ckcO =
%   \begin{cases}
%    ((2k+1)^{1})_{B} & \text{when } \star=C,\\
%    ((2k)^{1})_{C} & \text{when } \star=\wtC,\\
%    (1^{1}\, (2k-1)^{1} )_{D} & \text{when } \star=D^{*}. \\
%   \end{cases}
% \]
Now $\PBPes(\ckcO)$ has only one element, say $\uptau_{0}$, for which
we have $\ac_{\uptau_{0}} = (1^{(k,k)})_{\star}$.

\smallskip

Suppose that $\star \in \set{B,D}$.
%$\star \in \set{B,D,C^{*}}$.
There is $k\in \bN^{+}$ such that
$\cO$ and $\ckcO$ are given as in \Cref{lem:D.sign}.
We also have
%When $\star \in \set{B,D}$,
\[
  \ac_{\uptau} = (1^{(p_{\tau},(-1)^{\varepsilon_{\tau}}q_{\tau})})_{\star},
  \quad  \Forall \uptau\in \PBPes(\ckcO).
\]
Thanks to \Cref{lem:D.sign}, it is easy to see that
\[
  \begin{array}{rcl}
    \PBPes(\ckcO)\times \bZ/2\bZ & \rightarrow & \MYD_{\star}(\cO)\\
    (\uptau,\epsilon) & \mapsto & \ac_{\uptau}\otimes (\epsilon,\epsilon)
  \end{array}
\]
is a bijection.

%\smallskip

%Suppose that $\star = C^{*}$. %for $\uptau\in \PBPes(\ckcO)$,
%Then
%\[
% \ac_{\uptau} = (1^{(p_{\tau},q_{\tau})})_{\star}\quad \Forall \uptau\in \PBPes(\ckcO).
%\]
%and
%\[
%    \PBPes(\ckcO) \rightarrow  \MYD_{\star}(\cO), \quad
%    \uptau \mapsto \ac_{\uptau}
%\]
%is a bijection by \Cref{lem:D.sign}.

\smallskip

In all initial cases being considered, the orbit $\ckcO$ is quasi-distinguished and the rest of claims in all lemmas
of \Cref{sec:ac} are straightforward to verify.


\subsection{Establishing properties of $\ac_{\uptau}$: the general case for $\star\in \set{C,\wtC}$}
Retain the notation of
\Cref{lem:C}. We now assume $\bfcc_{2}(\cO)>0$ and all the lemmas in
\Cref{sec:ac} hold for $\ckcO'$.

Suppose $\sum_{j=1}^{b} m_{j} \cE_{j}\in \bZ[\MYD_{\star}]$ such that $m_{j}\neq
0$ and $\cE_{j}\in \MYD_{\star}$. We define the ``descent signature'' of
$\sum_{j=1}^{b} m_{j} \cE_{j}$ as a subset of $\bN\times \bN$:
\[
  \dsign(\sum_{j=1}^{b} m_{j} \cE_{j}) := \set{\Sign(\sO'_{j})|j=1,2,\cdots, b}.
\]
Here $\sO'_{j} \in \SYD_{\star'}$ is given by $\sO'_{j}(i) := \sF(\cE_{j})(i+1)$
for all $i\in \bN^{+}$. % (see \eqref{eq:F}).

\subsection*{The case when $\bfrr_{1}(\ckcO)>\bfrr_{2}(\ckcO)$}

\subsubsection*{Proof of \Cref{lem:ac0}} Since $\ac_{\uptau'}\neq 0$ and
is multiplicity free, $\ac_{\uptau}\neq 0$ and is multiplicity free by \eqref{eq:C}.
This proves \Cref{lem:ac0}.

\smallskip

\subsubsection*{Proof of \Cref{lem:C}~(a)} Suppose
$\ac_{\uptau_{1}}=\ac_{\uptau_{2}}$. By \eqref{eq:C},
$\set{\Sign(\uptau'_{1})}=\dsign(\ac_{\uptau_{1}}) =\dsign(\ac_{\uptau_{2}}) =
\set{\Sign(\uptau'_{2})}$. In particular,
$\Sign(\uptau'_{1})=\Sign(\uptau'_{2})$ and the integer ``$y$'' in \eqref{eq:C}
is the same for $\uptau_{1}$ and $\uptau_{2}$. By \eqref{eq:C}, we get
$\ac_{\uptau'_{1}}\otimes
(\varepsilon_{\wp_{1}},\varepsilon_{\wp_{1}})=\ac_{\uptau'_{2}}\otimes
(\varepsilon_{\wp_{1}}, \varepsilon_{\wp_{2}})$. So
$\varepsilon_{\wp_{1}}=\varepsilon_{\wp_{2}}$ and
$\ac_{\uptau'_{1}}=\ac_{\uptau'_{2}}$ by \Cref{lem:BD}~(a) for $\cOp$.



\subsubsection*{Proof of \Cref{lem:C}~(b)}
When $\ckcO$ is weakly-distinguished, $\ckcO'$ is also weakly-distinguished and we get $\uptau'_{1}=\uptau'_{2}$ by
\Cref{lem:BD}~(b) for $\cOp$. Hence $\uptau_{1}=\uptau_{2}$ by \Cref{prop:CC.bij}.


\subsubsection*{Proof of \Cref{lem:C}~(c)}
By \eqref{eq:C},  $\dlifttso$ induces a bijection from $\MYD_{\star'}(\cO')$ to
$\MYD_{\star}(\cO)$.
When $\ckcO$ is quasi-distinguished, $\ckcO'$ is quasi-distinguished
and the claim follows from
\Cref{lem:BD}~(c).
%Therefore, $\ac_{\uptau}\in \MYD_{\star}(\cO)$ and $\Set{\ac_{\uptau}|} = \MYD_{\star}(\cO)$.

\subsection*{The case when $\bfrr_{1}(\ckcO)=\bfrr_{2}(\ckcO)>0$}

\subsubsection*{Proof of \Cref{lem:ac0}}
According to \Cref{prop:CC.bij},  $x_{\taup}\neq s$.
Thanks to \Cref{lem:BD2}, $\pac{\uptaup} \neq 0$ and so $\ac_{\uptau}\neq 0$ by \eqref{eq:C}.

We now prove the multiplicity freeness of $\ac_{\uptau}$.
To ease the notation, we write
\[
 \pac{\uptau}:=\maltese^y (\pac{\uptau'}\cdot (0,0)) \AND
 \nac{\uptau}:=\maltese^y (\nac{\uptau'}\cdot (0,0)).
\]
Then
\begin{equation}
\label{eq:dsign1}
 \begin{split}
\dsign(\ac_{\uptau}) & = \dsign(\pac{\uptau})\cup \dsign(\nac{\uptau}), \\
\dsign(\pac{\uptau}) & = \set{\Sign(\uptau') - (1,0)}, \quad \text{ and }\\
\dsign(\nac{\uptau}) & =
\begin{cases}
\set{\Sign(\uptau') - (0,1)} & \text{if }  x_{\taup} =d, \\
\emptyset & \text{otherwise}. \\
\end{cases}
\end{split}
\end{equation}
Since $\dsign(\pac{\uptau})\cap \dsign(\nac{\uptau})=\emptyset$,
$\sF(\pac{\uptau})$ and $\sF(\nac{\uptau})$ have no common terms. Now the
multiplicity freeness of $\ac_{\uptau}$ follows from the multiplicity freeness
of $\ac_{\uptau'}$ and the fact that the operations $\Lambda_{(1,0)}$,
$\Lambda_{(0,1)}$ and ``$\cdot (0,0)$'' preserve
multiplicity freeness.


\smallskip

\subsubsection*{Proof of \Cref{lem:C}~(a)}
Suppose $\ac_{\uptau_{1}}=\ac_{\uptau_{2}}$. By \eqref{eq:dsign1},
we conclude that $\Sign(\uptau'_{1})=\Sign(\uptau'_{2})$. Since we have
$\varepsilon_{\wp_{1}}=\varepsilon_{\wp_{2}}=0$, \Cref{lem:C}~(a) is proven.



\subsubsection*{Proof of \Cref{lem:C}~(b)}
By considering the $\dsign$,  we get $\pac{\uptau_1}=\pac{\uptau_2}$
and $\nac{\uptau_1}=\nac{\uptau_2}$. So $\pac{\uptau'_{1}}=\pac{\uptau'_{2}}$ and
$\nac{\uptau'_{1}}=\nac{\uptau'_{2}}$ by \eqref{eq:C}.
Suppose $\ckcO$ is weakly-distinguished, then $\ckcO'$ is
weakly-distinguished.
Due to the injectivity of $\Upsilon_{\ckcO'}$ in \Cref{lem:BD3}~(a), we get
$\ac_{\uptau'_{1}}=\ac_{\uptau'_{2}}$ and so $\uptau'_{1}=\uptau'_{2}$ by \Cref{lem:BD}~(b).
Now \Cref{prop:CC.bij} implies $\uptau_{1}=\uptau_{2}$, which proves \Cref{lem:C}~(b).

\smallskip

Note that $\ckcO$ is not quasi-distinguished, so \Cref{lem:C}~(c) is vacant.


\subsection{Establishing properties of $\ac_{\uptau}$: the general case for $\star\in \set{B,D}$}\label{generalBD}


To ease the discussion, we adopt the following notation (c.f. \eqref{eq:sub}).
Suppose $\cA\in \bZ[\MYD]$. We write $\cA \sqsupseteq (p_{0},q_{0})$ if there is an marked Young diagram $\cE$ with
a non-zero coefficient in $\cA$ such that $\cE\sqsupseteq (p_{0},q_{0})$.


\smallskip
Retain the notation in \cref{lem:BD,lem:BD2,lem:BD3}.
We now assume $\bfcc_{2}(\cO)>0$ and all the lemmas in \Cref{sec:ac}
hold for $\ckcO':=\ckDD(\ckcO)$ and $\ckcO'' := \ckDD(\ckcO')$.



\subsection*{The case when $\bfrr_{2}(\ckcO)>\bfrr_{3}(\ckcO)$}

\subsubsection*{Proof of \Cref{lem:ac0}}

Since $\ac_{\uptau'}\neq 0$ and is multiplicity free, we conclude that $\ac_{\uptau}\neq 0$ and
is multiplicity free by \eqref{eq:BD}.


\subsubsection*{Proof of \Cref{lem:BD}~(a)}
Suppose $\ac_{\uptau_1}\lotimes (\epsilon_{1},\epsilon_{1})
= \ac_{\uptau_2}\lotimes( \epsilon_{2},\epsilon_{2} )$.
According to \eqref{eq:TBD}, we have
$\cL_{((\tau_{1})_{\bftt},\emptyset)}\lotimes(\epsilon_{1},\epsilon_{1}) =
\cL_{((\tau_{2})_{\bftt},\emptyset)}\lotimes(\epsilon_{2},\epsilon_{2})$.
By the initial cases (see \Cref{sec:init}), we get $\epsilon_{1}=\epsilon_{2}$ and
$\varepsilon_{\tau_{1}}=\varepsilon_{(\tau_{1})_{\bftt}}=\varepsilon_{(\tau_{2})_{\bftt}}=\varepsilon_{\tau_{2}}$.
%This proves \Cref{lem:BD}~(a).
By $\varepsilon_{\tau_1} = \varepsilon_{\tau_2}$ and
$\Sign(\tau_1)=\Sign(\tau_2)$, we deduce that
$\ac_{\uptaup_{1}}=\ac_{\uptaup_{2}}$, and so $\ac_{\uptau_1}= \ac_{\uptau_2}$ by from  \eqref{eq:BD}. 

\subsubsection*{Proof of \Cref{lem:BD}~(b)}

Since $\ckcO$ is
weakly-distinguished, $\ckcO'$ is also weakly-distinguished and so
$\uptaup_{1}=\uptaup_{2}$ by \Cref{lem:C}~(b). Applying \Cref{cor:dpinj}, we
conclude that $\uptau_{1}=\uptau_{2}$.

\subsubsection*{Proof of \Cref{lem:BD}~(c)}

Suppose $\ckcO$ is quasi-distinguished.
Then $\ckcO'$ is quasi-distinguished and $\ac_{\uptau'}\in \MYD_{\star'}(\cO')$.
By \Cref{eq:BD}, $\ac_{\uptau}\in \MYD_{\star}(\cO)$.

%The bijectivity of $\acme\colon $


For $\cE\in \MYD_{\star}$ with $(p,q)=\Sign(\cE)$, we shall construct $(\uptau, \epsilon)\in
\PBPes(\ckcO)\times \bZ/2\bZ$ such that $\ac_{\uptau}\lotimes(\epsilon, \epsilon)=\cE$,
which implies $\acme(\PBPes(\ckcO)) = \MYD_{\star}(\cO)$. By the initial cases
(see \Cref{sec:init}), there is a unique pair $(\tau_0, \epsilon)$ such that
\[
 \cE(1) = \ac_{(\tau_0,\emptyset)}\lotimes (\epsilon,\epsilon).
\]
By \Cref{lem:C}~(c), there is a unique $\uptau' = (\tau',\wp')\in \PBPesp(\ckcO')$
such that
\[
 \big(((\maltese^{y'}\ac_{\uptau'})\cdot (p_{\tau_0},q_{\tau_0}))\lotimes (0,\varepsilon_{\tau_0})\big) (i)
 = \big(\cE\lotimes (\epsilon,\epsilon)\big)(i) \qquad \text{for all} \quad i\geq 2.
\]
Here $y' = \frac{p-q+1}{2}$ if $\star = B$ and $y' = \frac{p-q+1}{2}$ if $\star
= D$. Using \Cref{cor:D.inj1}, we obtain a unique $\tau$ such that
$(\DD(\tau),\taut) = (\tau',\tau_0)$. Let $\wp\in \PP(\ckcO)$ be the unique
element such that $\DD(\wp) = \wp'$. Then $\uptau = (\tau, \wp)$ is the desired element.


\subsubsection*{Proof of \Cref{lem:BD2,lem:BD3}}

Using \eqref{eq:TBD}, all the claims follow from that of $\ckcO_{\bftt}$. See 
\Cref{sec:tail,sec:init}.
% (Note that we always have
% $\bfrr_{1}(\ckcO)>\bfrr_{3}(\ckcO)$. So \Cref{lem:BD2}~(b) is not vacant.)

\trivial[h]{
Note that $\Sign(\tau_{1})=\Sign(\tau_{2})$,
}


% \subsection*{The case when $\bfrr_{2}(\ckcO)>\bfrr_{3}(\ckcO)$}

% \subsubsection*{Proof of \Cref{lem:ac0}}

% Since $\ac_{\uptau'}\neq 0$ and mulitplicity free,  $\ac_{\uptau}\neq 0$ and multiplicity free by \eqref{eq:BD}.


% \subsubsection*{Proof of \Cref{lem:BD}~(a)}
% Suppose $\ac_{\uptau_1}\lotimes (\epsilon_{1},\epsilon_{1})
% = \ac_{\uptau_2}\lotimes( \epsilon_{2},\epsilon_{2} )$.
% According to \eqref{eq:TBD}, we have
% $\cL_{((\tau_{1})_{\bftt},\emptyset)}\lotimes(\epsilon_{1},\epsilon_{1}) =
% \cL_{((\tau_{2})_{\bftt},\emptyset)}\lotimes(\epsilon_{2},\epsilon_{2})$.
% By the  initial case (see \Cref{sec:init}), we get $\epsilon_{1}=\epsilon_{2}$ and
% $\varepsilon_{\tau_{1}}=\varepsilon_{(\tau_{1})_{\bftt}}=\varepsilon_{(\tau_{2})_{\bftt}}=\varepsilon_{\tau_{2}}$.
% %This proves \Cref{lem:BD}~(a).


% \subsubsection*{Proof of \Cref{lem:BD}~(b)}
% When $\ckcO$ is weakly-distinguished in addition, $\ckcO'$ is also
% weakly-distinghuished.
% We get $\ac_{\uptaup_{1}}=\ac_{\uptaup_{2}}$ by \eqref{eq:TBD} and so $\uptaup_{1}=\uptaup_{2}$.
% By \Cref{cor:dpinj}, we conclude that $\uptau_{1}=\uptau_{2}$.
% %This proves \Cref{lem:BD}~(b).


% \smallskip

% Suppose $\ckcO$ is quasi-distinghuished. Then $\ac_{\uptau}\in \MYD_{\star}$
% since $\ac_{\uptau'}\in \MYD_{\star'}$ by the quasi-distinguishness of
% $\ckcO'$.
% This proves \Cref{lem:BD}~(c).


% \smallskip

% Using \eqref{eq:TBD}, all the claims in \Cref{lem:BD2,lem:BD3} follows from that of
% $\ckcO_{\bftt}$, see \Cref{sec:tail}. (Note that we always have
% $\bfrr_{1}(\ckcO)>\bfrr_{3}(\ckcO)$. So \Cref{lem:BD2}~(b) is not vacant.)

% \trivial[h]{
% Note that $\Sign(\tau_{1})=\Sign(\tau_{2})$,
% }


\subsection*{The case when $\bfrr_{2}(\ckcO)=\bfrr_{3}(\ckcO)$}
\def\ppac#1{\cL_{#1,+}}
\def\nnac#1{\cL_{#1,-}}

We write $\uptaupp :=(\tau'',\wp''):= \DD^{2}(\uptau)$.
% Since $x_{\uptaupp}\neq s$, $\AC(\uptaupp)\neq 0$.
To ease the notation, we write
\begin{equation}
 \label{eq:BD4}
  \begin{split}
    \ac_{\uptau} & = \ppac{\uptau} + \nnac{\uptau} \qquad \text{with} \\
  \ppac{\uptau} & := \left(\maltese^t \big(\pac{\uptaupp}\cdot (0,0)\big) \cdot \pcT_{\uptau}\right)\lotimes (0,\varepsilon_\tau) \AND \\
  \nnac{\uptau} & := \left(\maltese^t \big(\nac{\uptaupp}\cdot (0,0)\big) \cdot \ncT_{\uptau}\right)\lotimes (0,\varepsilon_\tau).
  \end{split}
\end{equation}
Recall also the number $k = \frac{\bfrr_1(\ckcO)-\bfrr_2(\ckcO)}{2}+1$ (\Cref{sec:tail}).

\subsubsection*{Proof of \Cref{lem:ac0}. }

When $\Sign(\tau_{\bftt})\succeq (0,1)$, the term $\ppac{\uptau}$ in \eqref{eq:BD4} does 
not vanish since $\pac{\uptaupp}\neq 0$, by \Cref{lem:BD2} for $\ckcO''$ and $\pcT_{\uptau}\neq \emptyset$. Note that $x_{\tau''}\neq s$ by \Cref{lem:delta}.

When $\Sign(\tau_{\bftt})\nsucceq (0,1)$, we must have $\Sign(\tau_{\bftt}) =
(2k,0)$ and so $\cP_{\tau_{\bftt}}^{-1}(\set{s,c})=\emptyset$. Hence
$x_{\uptaupp}=d$ by \Cref{lem:delta}. Now $\nac{\uptaupp} \neq 0$ by
\Cref{lem:BD2} and $\nnac{\uptau}$  in \eqref{eq:BD4} does not vanish.
This proves the non-vanishing of $\ac_{\uptau}$.

Note that
$\sF(\ppac{\uptau})$ and $\sF(\nnac{\uptau})$ have no common terms since
$\pcT_{\uptau}\neq \ncT_{\uptau}$. Therefore the multiplicity freeness of
$\ac_{\uptau}$ follows from the multiplicity freeness of $\ac_{\uptau'}$.
This proves \Cref{lem:ac0}


\subsubsection*{Proof of \Cref{lem:BD2} }

It is clear that $\nac{\uptau}=0$ when $x_{\tau}\in \set{s,r,c}$ because of
the twist ``$\lotimes(0, 1)$'' in \eqref{eq:BD2}.
  %In the following proof $\cE\in \MYD_{\star}$ be a

  %\begin{enumPF}
  \begin{enuma}
    \item Suppose $x_{\uptau}=s$. We have $\Sign(\tau_{\bftt})=(0,2k)$.
    Therefore,  $\ppac{\uptau}\neq 0$, $\nnac{\uptau}=0$ and $\cE(1) =
    (B,0,-(2k-1))$ for every $\cE\in \MYD_{\star}$ with a non-zero coefficient in
    $\ac_{\uptau}$. Hence $\pac{\uptau}=0$.
    \item Suppose $x_{\uptau} \in \set{r,c}$. When
    $\Sign(\tau_{\bftt})\succeq (1,1)$, the term $\ppac{\uptau}$ in \eqref{eq:BD4} is
    non-zero.
    So
    \[
    \pac{\uptau}\succeq \Lambda_{(1,0)} (\ppac{\uptau}) \neq 0.
    \]
    When $\ssign(\tau_{\bftt})\nsucceq (1,1)$,  we have $\Sign(\tau_{\bftt})=(2k,0)$
    and $\cP_{\tau_{\bftt}}^{-1}(\set{s,c})=\emptyset$.
    So $x_{\uptaupp}=d$ by \Cref{lem:delta}. Now $\ncL_{\uptau''}\neq 0$
    and $\ncT_{\uptau}\succeq (1,0)$.
    Hence  $\nnac{\uptau}\neq 0$ and
    \[
    \pac{\uptau}\succeq \Lambda_{(1,0)} (\nnac{\uptau}) \neq 0.
    \]
    In both cases, we conclude that $\pac{\uptau}\neq 0$.
    \item Suppose $x_{\uptau}=d$. When $\ssign(\tau_{\bftt})\succeq (1,2)$,
    the term $\ppac{\uptau}\neq 0$ and so
    $\ac_{\uptau}\sqsupseteq (1,1)$. Therefore $\pac{\uptau}\neq 0$ and
    $\nac{\uptau}\neq 0$.
    When $\ssign(\tau_{\bftt})\nsucceq(1,2)$, we must have  $\Sign(\tau_{\bftt}) = (2k-1,1)\succeq (1,1)$.
    In particular, $\cP_{\tau_{\bftt}}^{-1}(\set{s,c})=\emptyset$ and
    so $x_{\uptaupp}=d$ by \Cref{lem:delta}.
    Now both $\ppac{\uptau}$ and $\nnac{\uptau}$ in \eqref{eq:BD4} are non-zero, which leads to the desired conclusions.
  %\end{enumPF}
  \end{enuma}


\subsubsection*{ Proof of \Cref{lem:BD}~(a) }
\def\opac{\cL'^+}
\def\onac{\cL'^-}
\def\tpac{\widetilde{\cL'}^+}
\def\tnac{\widetilde{\cL'}^-}
The following proof only depends on \Cref{lem:BD2}.
Suppose $\cL' = \ac_{\uptau}\lotimes (\epsilon,\epsilon)$ for $(\uptau,\epsilon)\in \PBPes(\ckcO)\times \bZ/2\bZ$.
Let
  \begin{align*}
    \opac &:= \Lambda_{(1,0)}(\cL'), &
    \onac &:= \Lambda_{(0,1)}(\cL'), \\
    \tpac &:= \Lambda_{(1,0)}(\cL'\lotimes (1,1)),\text{ and } &
    \tnac &:= \Lambda_{(0,1)}(\cL'\lotimes (1,1)).
  \end{align*}
Note that $\bfcc_{1}(\cO)-\bfcc_{2}(\cO)>0$, and so at least one of $\opac$,
$\onac$, $\tpac$ and $\tnac$ is non-vanishing.

Using \Cref{lem:BD2}, we
have the following non-vanishing criteria in terms of $(\varepsilon_{\tau},\epsilon)$:
\[
\begin{split}
  (\varepsilon_{\tau},\epsilon) = (0,0) & \Leftrightarrow
  \begin{cases} \opac \neq 0\\ \onac \neq 0 \end{cases};\\
  (\varepsilon_{\tau},\epsilon) = (0,1) & \Leftrightarrow
  \begin{cases} \tpac \neq 0\\ \tnac \neq 0 \end{cases};\\
  (\varepsilon_{\tau},\epsilon) = (1,0) & \Leftrightarrow
  \begin{cases} \opac \neq 0 \\ \onac  = 0 \end{cases} \text{ or} \quad
  \begin{cases} \tpac = 0 \\ \tnac  \neq 0 \end{cases}; \\
  (\varepsilon_{\tau},\epsilon) = (1,1) & \Leftrightarrow
  \begin{cases} \tpac \neq 0 \\ \tnac  = 0 \end{cases} \text{ or} \quad
  \begin{cases} \opac = 0 \\ \onac  \neq 0 \end{cases}.
\end{split}
\]
This implies \Cref{lem:BD}~(a).


\subsubsection*{Proof of \Cref{lem:BD}~(b)}
\def\tauot{(\tau_1)_{\bftt}}
\def\tautt{(\tau_2)_{\bftt}}
The weakly-distinguishness of $\ckcO$ implies that
\begin{equation}\label{eq:wplus}
  \bfrr_{1}(\ckcO'') = \bfrr_{3}(\ckcO)=\bfrr_{2}(\ckcO)>\bfrr_{5}(\ckcO) = \bfrr_{3}(\ckcO'').
\end{equation}

Suppose that  $\ac_{\uptau_{i}} \sqsupseteq (0,1)$ or $\ac_{\uptau_{i}} \sqsupseteq (0,-1)$.
% The first term in \eqref{eq:BD2} for both $\uptau_{1}$ and $\uptau_{2}$ must be
% non-zero and equal.
We must have $\Sign(\tauot)\succeq (0,1)$, $\Sign(\tautt)\succeq (0,1)$ and so
$\ppac{\uptau_1} = \ppac{\uptau_2}\neq 0$.
This implies $\pac{\uptaupp_{1}}=\pac{\uptaupp_{2}}\neq 0$. From \eqref{eq:wplus},
we see that $\Upsilon^+_{\ckcO''}$ is injective by \Cref{lem:BD3}~(b), and so $\ac_{\uptaupp_{1}}
= \ac_{\uptaupp_{2}}$.

\smallskip

Now consider the case when $\ac_{\uptau_{i}}\nsqsupseteq (0,1)$ and
$\ac_{\uptau_{i}}\nsqsupseteq (0,-1)$. By \Cref{lem:BD2}, $x_{\uptau}\neq d$.
From \eqref{eq:BD2}, we conclude that $\set{\Sign((\tau_{i})_{\bftt})|i=1,2}
\subset\set{ (2k,0),(2k-1,1)}$.

We now discuss case by case according to the set $\set{\Sign((\tau_{i})_{\bftt})|i=1,2}$:
  \begin{enumPF}
    \item Suppose
    $\set{\Sign(({\tau_{i}})_{\bftt})|i=1,2} = \set{(2k-1,1)}$. We have
    $\ppac{\uptau_1} = \ppac{\uptau_2}\neq 0$ and so
    $\pac{\uptaupp_{1}}=\pac{\uptaupp_{2}}\neq 0$.
    %By \eqref{eq:wplus}, $\Upsilon^+_{\ckcO''}$ is injective \Cref{lem:BD3}~(b).
   %So  $\ac_{\uptaupp_{1}} = \ac_{\uptaupp_{2}}$.
    By \eqref{eq:wplus} and \Cref{lem:BD3}~(b), we deduce
    that $\ac_{\uptaupp_{1}}=\ac_{\uptaupp_{2}}$.
    \item
    Suppose $\set{\Sign(({\tau_{i}})_{\bftt})|i=1,2} = \set{(2k,0)}$. We have
   $\ppac{\uptau_1}=\nnac{\uptau_2}\neq 0$ and so
   $\nac{\uptaupp_{1}}=\nac{\uptaupp_{2}}\neq 0$. By \eqref{eq:wplus} and
   \Cref{lem:BD3}~(b), we deduce that $\ac_{\uptaupp_{1}}=\ac_{\uptaupp_{2}}$.
    \item
    Without loss of generality, we may assume that $\Sign(({\tau_{1}})_{\bftt}) =
    (2k-1,1)$ and $\Sign(({\tau_{2}})_{\bftt}) = (2k,0)$. By \eqref{eq:BD2}, we
    get $\ppac{\uptau_1} = \nnac{\uptau_2}$ and so
    \[
      \pac{\uptaupp_{1}}\cdot (0,0)=\maltese(\nac{\uptaupp_{2}}\cdot (0,0)),
    \]
    which contradicts \Cref{lem:BD3}~(c) for $\ckcO''$.
  \end{enumPF}

In all the cases, we have $\ac_{\uptaupp_1} = \ac_{\uptaupp_2}$.
Since $\cO''$ is weakly-distinguished,   $\uptaupp_{1}=\uptaupp_{2}$ by
\Cref{lem:BD}~(b). Note that $\varepsilon_{\tau_1}=\varepsilon_{\tau_2}$ by
\Cref{lem:BD}~(a) and $\Sign(\uptau_{1})=\Sign(\uptau_{2})$. We conclude that
$\uptau_{1}=\uptau_{2}$ by combining \Cref{cor:dpinj} and \Cref{prop:CC.bij}.
This finishes the proof of \Cref{lem:BD}~(b).

\subsubsection*{\Cref{lem:BD}~(c) is vacant in the current case.}


\subsubsection*{Proof of \Cref{lem:BD3}~(b)}

Note that the condition $\bfrr_{1}(\ckcO)>\bfrr_{3}(\ckcO)=\bfrr_{2}(\ckcO)$
implies $k\geq 2$.

\medskip

We first prove the injectivity of $\Upsilon^{+}_{\ckcO}$.

Note that
  $\ac_{\uptau}\sqsupseteq (2,0)$ is equivalent to $\pac{\uptau}\sqsupseteq
  (1,0)$. By \eqref{eq:BD2},  $\ac_{\uptau}\sqsupseteq (2,0)$ implies that $\ac_{\uptau}\rhd (1,0)$, which means 
  that $\cE \sqsupseteq (1,0)$, for any marked Young diagram $\cE$ with a non-zero coefficient in $\ac_{\uptau}$. So we obtain the following  bijection
  \[
    \set{\ac_{\uptau}|\uptau\in \PBPes(\ckcO) \text{ and }\ac_{\uptau}\sqsupseteq (2,0)} \rightarrow
     \set{\pac{\uptau}|\uptau\in \PBPes(\ckcO) \text{ and }\pac{\uptau}\sqsupseteq (1,0)}.
  \]
  % Hence we
  % $\ac_{\uptau} := \pac{\uptau}\cdot 1^{(1,0)}$.

  Now we consider the set
  \[
   \set{\ac_{\uptau}|\uptau\in \PBPes(\ckcO),
  \pac{\uptau}\neq 0 \text{ and } \ac_{\uptau}\nsqsupseteq (2,0)}.
  \]
   Let $\ac_{\uptau_{1}}$
  and $\ac_{\uptau_{2}}$ be two elements in this set such that
  $\pac{\uptau_{1}}=\pac{\uptau_{2}}$.

  By \eqref{eq:BD2}, we have $\Sign(\ttail{(\tau_{i})}) = (1,2k-1)$ and
  \begin{equation}\label{eq:ft1}
  \ppac{\uptau_1} = \ppac{\uptau_2}\neq 0.
  \end{equation}

    Note that $\cE(1) = (B,1, (-1)^{\varepsilon_{\tau_{i}}}(2k-2))$ for each  marked Young diagram $\cE$ having a non-zero
    coefficient in $\ppac{\uptau_1}$. So $\varepsilon_{\tau_{1}}=\varepsilon_{\tau_{2}}$.
   % Compairing the first term of \eqref{eq:BD2} gives $\pac{\uptaupp_{1}}=\pac{\uptaupp_{2}}$.
    Note that $\ckcO''$ is weakly-distinguished and $\bfrr_{1}(\ckcO'')>\bfrr_{3}(\ckcO'')$, and so we deduce that
    $\uptaupp_1 = \uptaupp_2$, $\uptau_1 =\uptau_2$ and $\ac_{\uptau_1} = \ac_{\uptau_2}$ as in the proof of \Cref{lem:BD}~(b).
    % $\ac_{\uptaupp_{1}}=\ac_{\uptaupp_{2}}$ by the injectivity of $\Upsilon_{\ckcO''}^{+}$
    % and \eqref{eq:ft1}.
    % Now we conclude that $\ac_{ \uptau_{1} }=\ac_{ \uptau_{2} }$ by \eqref{eq:BD2} and
    % finished the proof of the injectivity of $\Upsilon_{\ckcO}^{+}$.

    \medskip

    The injectivity of $\Upsilon_{\ckcO}^{-}$ is proved similarly, which we give below. 

  Note that $\ac_{\uptau}\sqsupseteq (0,2)$ is equivalent to
  $\nac{\uptau}\sqsupseteq (0,1)$.
  We get the bijection
  \[
    \set{\ac_{\uptau}|\uptau\in \PBPes(\ckcO),\ac_{\uptau}\sqsupseteq (0,2)} \rightarrow
    \set{\nac{\uptau}|\uptau\in \PBPes(\ckcO),\nac{\uptau}\sqsupseteq (0,1)}.
  \]
  %$\ac_{\uptau} := \pac{\uptau}\cdot 1^{(0,1)}$.

  Now we consider the set
 \[
  \set{\ac_{\uptau}|\uptau\in \PBPes(\ckcO),\nac{\uptau}\neq 0 \text{ and } \ac_{\uptau}\nsqsupseteq (0,2)}.
 \]
  Let $\ac_{\uptau_{1}}$ and $\ac_{\uptau_{2}}$ be two elements in the set such that
  $\nac{\uptau_{1}}=\nac{\uptau_{2}}\neq 0$.
  By $k\geq 2$ and \eqref{eq:BD2}, we get
  \begin{equation}\label{eq:ft2}
  \set{\Sign((\tau_{i})_{\bftt})| i = 1,2} \subseteq \set{ (2k-1,1),(2k-2,2)}.
  \end{equation}
  Note that \eqref{eq:ft2} cannot be equality since it will contradict \Cref{lem:BD3}~(c).
  We consider case by case according to the set $\set{\Sign((\tau_{i})_{\bftt})|i=1,2}$:
  \begin{enumPF}
    \item When
    $\set{\Sign((\tau_{i})_{\bftt})|i=1,2} = \set{(2k-1,1)}$,
    we have $\nac{\uptau''_{1}} = \nac{\uptau''_{2}}$.
    % , and so
    % $\ac_{\uptaupp_{1}}=\ac_{\uptaupp_{2}}$ by the injectivity of $\Upsilon_{\ckcO''}^{-}$.
    \item When
    $\set{\Sign((\tau_{i})_{\bftt})|i=1,2} = \set{(2k-2,2)}$,
     we have
    $\pac{\uptau''_{1}} = \pac{\uptau''_{2}}$. 
    % , and so
    % $\ac_{\uptaupp_{1}}=\ac_{\uptaupp_{2}}$ by the injectivity of $\Upsilon_{\ckcO''}^{+}$.
    % $\Sign({\tau_{1}}_{\bftt}) = (2k-1,1)$ and $\Sign({\tau_{2}}_{\bftt}) = (2k-2,2)$.
    % But this will leads to
    % $\maltese (\nac{\uptau''_{1}}\cdot (0,0)) = \pac{\uptau''_{2}}\cdot (0,0)$
    % which is contradict to
  \end{enumPF}
  By the same argument as in the proof of \Cref{lem:BD}~(b), 
  we conclude that $\uptau_1 = \uptau_2$, $\ac_{ \uptau_{1} }=\ac_{ \uptau_{2} }$ and
  therefore the injectivity of $\Upsilon_{\ckcO}^{-}$.

  \subsubsection*{Proof of \Cref{lem:BD3}~(c)}

  Note that we have $k\geq 2$ and $x_{\tau}=d$ under the stated condition.
  When $\Sign( {\tau}_{\bftt} )\succeq (1,2)$, $\ppac{\uptau}\sqsupseteq (1,1)$.
  %the first term of \eqref{eq:BD2} is non-vanish.
  Otherwise, $\Sign( \tau_{\bftt} ) = (2k-1,1)\succeq (3,1)$.
  Since $\nac{\uptau}\neq 0$, we have $\nnac{\uptau}\sqsupseteq (2,1)$.
  In all the cases,  $\ac_{\uptau}\sqsupseteq (1,1)$ and so
  $\nac{\uptau}\sqsupseteq (1,0)$.

  Now there is a marked Young diagram $\cA$ having a non-zero coefficient in
  $\maltese (\nac{\uptau}\cdot (0,0))$ such that $\cA(2) = (\star_2, p_2, q_2)$
  and $p_2<0$.
  On the other hand, every marked Young diagram $\cB$ having
  a non-zero coefficient in $\pac{\uptau_2}\cdot(0,0)$ satisfies $p'_2\geq 2$
  with $\cB(2) = (\star_2, p'_2,q'_2)$.
  Therefore \eqref{eq:pnac} never holds.

  %$\uptau_2\in \PBP_{\star}(\ckcO)$ such that $\pac{\uptau_2}$

\subsubsection*{Proof of \Cref{lem:BD3}~(a)}

When $\bfrr_{1}(\ckcO)>\bfrr_{2}(\ckcO)=\bfrr_{3}(\ckcO)$, the injectivity
follows from the injectivity of $\Upsilon_{\ckcO}^{+}$.

We are left to consider the case when $\bfrr_{1}(\ckcO)=\bfrr_{2}(\ckcO)$.
In this case, we have $k=1$ and $\bfrr_{1}(\ckcO'')>\bfrr_{3}(\ckcO'')$.
%Let $\uptau_{i} \PBPes(\ckco)$ such that $\pac{\uptau_{i}}\neq 0$ for $i=1,2$.
Suppose $\uptau_1, \uptau_2\in \PBPes(\ckcO)$ such that
$\Upsilon_{\ckcO}(\ac_{\uptau_{1}}) = \Upsilon_{\ckcO}(\ac_{\uptau_{2}})$.

When $\nac{\uptau_{1}}=\nac{\uptau_{2}}  \neq 0$, we have $x_{\tau_{1}}=x_{\tau_{2}}=d$ and
$\pac{\uptau''_{1}} = \pac{\uptau''_{2}}$ by \eqref{eq:BD2}.

When $\nac{\uptau_{1}}=\nac{\uptau_{2}}=0$, by a similar argument as in the proof of \Cref{lem:BD}~(b),
we conclude that there are only two possibilities:
\begin{itemize}
  \item $\pac{\uptau''_{1}} = \pac{\uptau''_{2}}$ and
  $x_{\tau_{1}}=x_{\tau_{2}} = c$;
  \item $\nac{\uptau''_{1}} = \nac{\uptau''_{2}}$ and
  $x_{\tau_{1}}=x_{\tau_{2}} = r$.
\end{itemize}

In all cases,
we conclude that $\uptau_1=\uptau_2$ and $\ac_{\uptau_1} = \ac_{\uptau_2}$, 
as in the proof of \Cref{lem:BD}~(b). 
%as before using \Cref{lem:BD3}~(b), \Cref{lem:BD}~(b) and \Cref{cor:dpinj}.
% we get $\ac_{\uptau''_{1}}=\ac_{\uptau''_{2}}$ by the
% injectivity of $\Upsilon_{\ckcO''}^{+}$ or $\Upsilon_{\ckcO''}^{-}$.
% Hence $\ac_{\uptau_{1}}=\ac_{\uptau_{2}}$ by \eqref{eq:BD2}.



\begin{bibdiv}
  \begin{biblist}
% \bib{AB}{article}{
%   title={Genuine representations of the metaplectic group},
%   author={Adams, Jeffrey},
%   author = {Barbasch, Dan},
%   journal={Compositio Mathematica},
%   volume={113},
%   number={01},
%   pages={23--66},
%   year={1998},
% }

\bib{Ad83}{article}{
  author = {Adams, J.},
  title = {Discrete spectrum of the reductive dual pair $(O(p,q),Sp(2m))$ },
  journal = {Invent. Math.},
  number = {3},
 pages = {449--475},
 volume = {74},
 year = {1983}
}

\bib{Ad07}{article}{
author = {Adams, J.},
 title = {The theta correspondence over R},
 journal = {Harmonic analysis, group representations, automorphic forms and invariant theory,  Lect. Notes Ser. Inst. Math. Sci. Natl. Univ. Singap., 12},
pages = {1--39},
 year = {2007}
  publisher={World Sci. Publ.}
}


\bib{ABV}{book}{
  title={The Langlands classification and irreducible characters for real reductive groups},
  author={Adams, J.},
  author={Barbasch, B.},
  author={Vogan, D. A.},
  series={Progress in Math.},
  volume={104},
  year={1991},
  publisher={Birkhauser}
}

\bib{AC}{article}{
  title={Algorithms for representation theory of
    real reductive groups},
  volume={8},
  DOI={10.1017/S1474748008000352},
  number={2},
  journal={Journal of the Institute of Mathematics of Jussieu},
  publisher={Cambridge University Press},
  author={Adams, Jeffrey}
  author={du Cloux,
    Fokko},
  year={2009},
  pages={209-259}
}

\bib{ArPro}{article}{
  author = {Arthur, J.},
  title = {On some problems suggested by the trace formula},
  journal = {Lie group representations, II (College Park, Md.), Lecture Notes in Math. 1041},
 pages = {1--49},
 year = {1984}
}


\bib{ArUni}{article}{
  author = {Arthur, J.},
  title = {Unipotent automorphic representations: conjectures},
  %booktitle = {Orbites unipotentes et repr\'esentations, II},
  journal = {Orbites unipotentes et repr\'esentations, II, Ast\'erisque},
 pages = {13--71},
 volume = {171-172},
 year = {1989}
}

\bib{AK}{article}{
  author = {Auslander, L.},
  author = {Kostant, B.},
  title = {Polarizations and unitary representations of solvable Lie groups},
  journal = {Invent. Math.},
 pages = {255--354},
 volume = {14},
 year = {1971}
}

\bib{B.Class}{article}{
  author = {Barbasch, D.},
  title = {The unitary dual for complex classical Lie groups},
  journal = {Invent. Math.},
  number = {1},
 volume = {96},
     pages = {103--176},
      year = {1989},
}

\bib{B.Uni}{article}{
  author = {Barbasch, D.},
  title = {Unipotent representations for real reductive groups},
 %booktitle = {Proceedings of ICM, Kyoto 1990},
 journal = {Proceedings of ICM (1990), Kyoto},
   % series = {Proc. Sympos. Pure Math.},
 %   volume = {68},
     pages = {769--777},
 publisher = {Springer-Verlag, The Mathematical Society of Japan},
      year = {2000},
}

\bib{B.W}{article}{
  author={Barbasch, D.},
  author={Vogan, David},
  editor={Trombi, P. C.},
  title={Weyl Group Representations and Nilpotent Orbits},
  bookTitle={Representation Theory of Reductive Groups:
    Proceedings of the University of Utah Conference 1982},
  year={1983},
  publisher={Birkh{\"a}user Boston},
  address={Boston, MA},
  pages={21--33},
  %doi={10.1007/978-1-4684-6730-7_2},
}



\bib{B.Orbit}{article}{
  author = {Barbasch, D.},
  title = {Orbital integrals of nilpotent orbits},
 %booktitle = {The mathematical legacy of {H}arish-{C}handra ({B}altimore,{MD}, 1998)},
    journal = {The mathematical legacy of {H}arish-{C}handra, Proc. Sympos. Pure Math.},
    %series={The mathematical legacy of {H}arish-{C}handra, Proc. Sympos. Pure Math},
    volume = {68},
     pages = {97--110},
 publisher = {Amer. Math. Soc., Providence, RI},
      year = {2000},
}



\bib{B10}{article}{
  author = {Barbasch, D.},
  title = {The unitary spherical spectrum for split classical groups},
  journal = {J. Inst. Math. Jussieu},
% number = {9},
 pages = {265--356},
 volume = {9},
 year = {2010}
}



\bib{B17}{article}{
  author = {Barbasch, D.},
  title = {Unipotent representations and the dual pair correspondence},
  journal = {J. Cogdell et al. (eds.), Representation Theory, Number Theory, and Invariant Theory, In Honor of Roger Howe. Progress in Math.}
  %series ={Progress in Math.},
  volume = {323},
  pages = {47--85},
  year = {2017},
}

\bib{BMSZ1}{article}{
  author = {Barbasch, D.},
  author = {Ma, J.-J.},
  author = {Sun, B.-Y.},
  author = {Zhu, C.-B.},
  title = {On the notion of metaplectic Barbasch-Vogan duality},
  journal = {arXiv:2010.16089},
}

\bib{BMSZ2}{article}{
  author = {Barbasch, D.},
  author = {Ma, J.-J.},
  author = {Sun, B.-Y.},
  author = {Zhu, C.-B.},
  title = {Counting special unipotent representations: orthogonal and symplectic groups},
  journal = {in preparation},
}

\bib{BV83}{article}{
 author = {Barbasch, D.},
 author = {Vogan, D. A.},
 title = {Weyl group representations and nilpotent orbits},
 journal = {in Representation theory of reductive groups (Park City, Utah, 1982), Progress in Math.},
 volume = {40},
 pages = {21--33},
 year = {1983}
}

\bib{BVUni}{article}{
 author = {Barbasch, D.},
 author = {Vogan, D. A.},
 journal = {Annals of Math.},
 number = {1},
 pages = {41--110},
 title = {Unipotent representations of complex semisimple groups},
 volume = {121},
 year = {1985}
}

\bib{Br}{article}{
  author = {Brylinski, R.},
  title = {Dixmier algebras for classical complex nilpotent orbits via Kraft-Procesi models. I},
  journal = {The orbit method in geometry and physics (Marseille, 2000). Progress in Math.}
  volume = {213},
  pages = {49--67},
  year = {2003},
}

\bib{Bor}{article}{
 author = {Borho, W.},
 journal = {S\'eminaire Bourbaki, Exp. No. 489},
 pages = {1--18},
 title = {Recent advances in enveloping algebras of semisimple Lie-algebras},
 year = {1976/77}
}

\bib{BK}{article}{
author={Borho, Walter},
author={Kraft, Hanspeter},
title={\"{U}ber die Gelfand-Kirillov-Dimension},
journal={Math. Ann.},
volume={220},
date={1976},
number={1},
pages={1--24},
issn={0025-5831},
review={\MR{412240}},
doi={10.1007/BF01354525},
}

\bib{Carter}{book}{
   author={Carter, R. W.},
   title={Finite groups of Lie type},
   series={Wiley Classics Library},
   %note={Conjugacy classes and complex characters;
   %Reprint of the 1985 original;
   %A Wiley-Interscience Publication},
   publisher={John Wiley \& Sons, Ltd., Chichester},
   date={1993},
   pages={xii+544},
   isbn={0-471-94109-3},
   %review={\MR{1266626}},
}

\bib{Ca89}{article}{
 author = {Casselman, W.},
 journal = {Canad. J. Math.},
 pages = {385--438},
 title = {Canonical extensions of Harish-Chandra modules to representations of $G$},
 volume = {41},
 year = {1989}
}

\bib{CS21}{article}{
 author = {Chen, Y.},
 author = {Sun, B.},
 journal = {J. Funct. Anal. },
 pages = {108817},
 title = {Schwartz homologies of representations of almost linear Nash groups},
 volume = {280},
 year = {2021},
}

%Chen, Yangyang; Sun, Binyong; Schwartz homologies of representations of almost linear Nash groups. J. Funct. Anal. 280 (2021), no. 7, 108817

\bib{Cl89}{article}{
  author = {du Cloux, F.},
  journal = {J.  Funct.  Anal.},
  number = {2},
  pages = {420--457},
  title = {Repr\'esentations temp\'er\'ees des groupes de Lie nilpotents},
 % url = {http://eudml.org/doc/82297},
  volume = {85},
  year = {1989},
}


\bib{Cl}{article}{
  author = {du Cloux, F.},
  journal = {Ann. Sci. \'Ecole Norm. Sup.},
  number = {3},
  pages = {257--318},
  title = {Sur les repr\'esentations diff\'erentiables des groupes de Lie alg\'ebriques},
  url = {http://eudml.org/doc/82297},
  volume = {24},
  year = {1991},
}

\bib{CM}{book}{
  title = {Nilpotent orbits in semisimple Lie algebra: an introduction},
  author = {Collingwood, D. H.},
  author = {McGovern, W. M.},
  year = {1993}
  publisher = {Van Nostrand Reinhold Co.},
}


\bib{CHH}{article}{
  author = {Cowling, M.},
  author = {Haagerup, U.},
  author = {Howe, R.},
  journal = {J. Reine Angew. Math.},
  %number = {3},
  pages = {97--110},
  title = {Almost L2 matrix coefficients},
  %url = {http://eudml.org/doc/82297},
  volume = {387},
  year = {1988},
}


% \bib{Dieu}{book}{
%    title={La g\'{e}om\'{e}trie des groupes classiques},
%    author={Dieudonn\'{e}, Jean},
%    year={1963},
% 	publisher={Springer},
%  }

\bib{DKPC}{article}{
title = {Nilpotent orbits and complex dual pairs},
journal = {J. Algebra},
volume = {190},
number = {2},
pages = {518 - 539},
year = {1997},
author = {Daszkiewicz, A.},
author = {Kra\'skiewicz, W.},
author = {Przebinda, T.},
}

\bib{DKP2}{article}{
  author = {Daszkiewicz, A.},
  author = {Kra\'skiewicz, W.},
  author = {Przebinda, T.},
  title = {Dual pairs and Kostant-Sekiguchi correspondence. II. Classification
	of nilpotent elements},
  journal = {Central European J. Math.},
  year = {2005},
  volume = {3},
  pages = {430--474},
}


\bib{DM}{article}{
  author = {Dixmier, J.},
  author = {Malliavin, P.},
  title = {Factorisations de fonctions et de vecteurs ind\'efiniment diff\'erentiables},
  journal = {Bull. Sci. Math. (2)},
  year = {1978},
  volume = {102},
  pages = {307--330},
}

%\bibitem[DM]{DM}
%J. Dixmier and P. Malliavin, \textit{Factorisations de fonctions et de vecteurs ind\'efiniment diff\'erentiables}, Bull. Sci. Math. (2), 102 (4),  307-330 (1978).



%\bib{Du77}{article}{
% author = {Duflo, M.},
% journal = {Annals of Math.},
% number = {1},
% pages = {107-120},
% title = {Sur la Classification des Ideaux Primitifs Dans
%   L'algebre Enveloppante d'une Algebre de Lie Semi-Simple},
% volume = {105},
% year = {1977}
%}

\bib{Du82}{article}{
 author = {Duflo, M.},
 journal = {Acta Math.},
  volume = {149},
 number = {3-4},
 pages = {153--213},
 title = {Th\'eorie de Mackey pour les groupes de Lie alg\'ebriques},
 year = {1982}
}



\bib{GZ}{article}{
author={Gomez, R.},
author={Zhu, C.-B.},
title={Local theta lifting of generalized Whittaker models associated to nilpotent orbits},
journal={Geom. Funct. Anal.},
year={2014},
volume={24},
number={3},
pages={796--853},
}

\bib{EGAIV2}{article}{
  title = {\'El\'ements de g\'eom\'etrie alg\'brique IV: \'Etude locale des
    sch\'emas et des morphismes de sch\'emas. II},
  author = {Grothendieck, A.},
  author = {Dieudonn\'e, J.},
  journal  = {Inst. Hautes \'Etudes Sci. Publ. Math.},
  volume = {24},
  year = {1965},
}


\bib{EGAIV3}{article}{
  title = {\'El\'ements de g\'eom\'etrie alg\'brique IV: \'Etude locale des
    sch\'emas et des morphismes de sch\'emas. III},
  author = {Grothendieck, A.},
  author = {Dieudonn\'e, J.},
  journal  = {Inst. Hautes \'Etudes Sci. Publ. Math.},
  volume = {28},
  year = {1966},
}


\bib{HLS}{article}{
    author = {Harris, M.},
    author = {Li, J.-S.},
    author = {Sun, B.},
     title = {Theta correspondences for close unitary groups},
 %booktitle = {Arithmetic Geometry and Automorphic Forms},
    %series = {Adv. Lect. Math. (ALM)},
    journal = {Arithmetic Geometry and Automorphic Forms, Adv. Lect. Math. (ALM)},
    volume = {19},
     pages = {265--307},
 publisher = {Int. Press, Somerville, MA},
      year = {2011},
}

\bib{HS}{book}{
 author = {Hartshorne, R.},
 title = {Algebraic Geometry},
publisher={Graduate Texts in Mathematics, 52. New York-Heidelberg-Berlin: Springer-Verlag},
year={1983},
}

\bib{He}{article}{
author={He, H.},
title={Unipotent representations and quantum induction},
journal={arXiv:math/0210372},
year = {2002},
}

\bib{He2}{article}{
author={He, H.},
title={Unitary representations and theta correspondence for type I classical groups},
journal={J. Funct. Anal.},
year = {2003},
volume = {199},
number = {1},
pages = {92--121},
}


\bib{HL}{article}{
author={Huang, J.-S.},
author={Li, J.-S.},
title={Unipotent representations attached to spherical nilpotent orbits},
journal={Amer. J. Math.},
volume={121},
number = {3},
pages={497--517},
year={1999},
}


\bib{HZ}{article}{
author={Huang, J.-S.},
author={Zhu, C.-B.},
title={On certain small representations of indefinite orthogonal groups},
journal={Represent. Theory},
volume={1},
pages={190--206},
year={1997},
}



\bib{Howe79}{article}{
  title={$\theta$-series and invariant theory},
  author={Howe, R.},
  book = {
    title={Automorphic Forms, Representations and $L$-functions},
    series={Proc. Sympos. Pure Math},
    volume={33},
    year={1979},
  },
  pages={275-285},
}

\bib{HoweRank}{article}{
author={Howe, R.},
title={On a notion of rank for unitary representations of the classical groups},
journal={Harmonic analysis and group representations, Liguori, Naples},
pages={223-331},
year={1982},
}

\bib{Howe89}{article}{
author={Howe, R.},
title={Transcending classical invariant theory},
journal={J. Amer. Math. Soc.},
volume={2},
pages={535--552},
year={1989},
}

\bib{Howe95}{article}{,
  author = {Howe, R.},
  title = {Perspectives on invariant theory: Schur duality, multiplicity-free actions and beyond},
  journal = {Piatetski-Shapiro, I. et al. (eds.), The Schur lectures (1992). Ramat-Gan: Bar-Ilan University, Isr. Math. Conf. Proc. 8,},
  year = {1995},
  pages = {1-182},
}

\bib{JLS}{article}{
author={Jiang, D.},
author={Liu, B.},
author={Savin, G.},
title={Raising nilpotent orbits in wave-front sets},
journal={Represent. Theory},
volume={20},
pages={419--450},
year={2016},
}

\bib{Ki62}{article}{
author={Kirillov, A. A.},
title={Unitary representations of nilpotent Lie groups},
journal={Uspehi Mat. Nauk},
volume={17},
issue ={4},
pages={57--110},
year={1962},
}


\bib{Ko70}{article}{
author={Kostant, B.},
title={Quantization and unitary representations},
journal={Lectures in Modern Analysis and Applications III, Lecture Notes in Math.},
volume={170},
pages={87--208},
year={1970},
}


\bib{KP}{article}{
author={Kraft, H.},
author={Procesi, C.},
title={On the geometry of conjugacy classes in classical groups},
journal={Comment. Math. Helv.},
volume={57},
pages={539--602},
year={1982},
}

\bib{KR}{article}{
author={Kudla, S. S.},
author={Rallis, S.},
title={Degenerate principal series and invariant distributions},
journal={Israel J. Math.},
volume={69},
pages={25--45},
year={1990},
}


\bib{Ku}{article}{
author={Kudla, S. S.},
title={Some extensions of the Siegel-Weil formula},
journal={In: Gan W., Kudla S., Tschinkel Y. (eds) Eisenstein Series and Applications. Progress in Mathematics, vol 258. Birkh\"auser Boston},
%volume={69},
pages={205--237},
year={2008},
}





\bib{LZ1}{article}{
author={Lee, S. T.},
author={Zhu, C.-B.},
title={Degenerate principal series and local theta correspondence II},
journal={Israel J. Math.},
volume={100},
pages={29--59},
year={1997},
}

\bib{LZ2}{article}{
author={Lee, S. T.},
author={Zhu, C.-B.},
title={Degenerate principal series of metaplectic groups and Howe correspondence},
journal = {D. Prasad at al. (eds.), Automorphic Representations and L-Functions, Tata Institute of Fundamental Research, India,},
year = {2013},
pages = {379--408},
}

\bib{Li89}{article}{
author={Li, J.-S.},
title={Singular unitary representations of classical groups},
journal={Invent. Math.},
volume={97},
number = {2},
pages={237--255},
year={1989},
}

\bib{Li90}{article}{
author={Li, J.-S.},
title={Theta lifting for unitary representations with nonzero cohomology},
journal={Duke Math. J.},
volume={61},
number = {3},
pages={913--937},
year={1990},
}

\bib{LiuAG}{book}{
  title={Algebraic Geometry and Arithmetic Curves},
  author = {Liu, Q.},
  year = {2006},
  publisher={Oxford University Press},
}

\bib{LM}{article}{
   author = {Loke, H. Y.},
   author = {Ma, J.},
    title = {Invariants and $K$-spectrums of local theta lifts},
    journal = {Compositio Math.},
    volume = {151},
    issue = {01},
    year = {2015},
    pages ={179--206},
}

\bib{LS}{article}{
   author = {Lusztig, G.},
   author = {Spaltenstein, N.},
    title = {Induced unipotent classes},
    journal = {j. London Math. Soc.},
    volume = {19},
    year = {1979},
    pages ={41--52},
}

\bib{Lu.I}{article}{
   author={Lusztig, G.},
   title={Intersection cohomology complexes on a reductive group},
   journal={Invent. Math.},
   volume={75},
   date={1984},
   number={2},
   pages={205--272},
   issn={0020-9910},
   review={\MR{732546}},
   doi={10.1007/BF01388564},
}


\bib{Ma}{article}{
   author = {Mackey, G. W.},
    title = {Unitary representations of group extentions},
    journal = {Acta Math.},
    volume = {99},
    year = {1958},
    pages ={265--311},
}


\bib{Mc}{article}{
   author = {McGovern, W. M},
    title = {Cells of Harish-Chandra modules for real classical groups},
    journal = {Amer. J.  of Math.},
    volume = {120},
    issue = {01},
    year = {1998},
    pages ={211--228},
}

\bib{Mo96}{article}{
 author={M{\oe}glin, C.},
    title = {Front d'onde des repr\'esentations des groupes classiques $p$-adiques},
    journal = {Amer. J. Math.},
    volume = {118},
    issue = {06},
    year = {1996},
    pages ={1313--1346},
}

\bib{Mo17}{article}{
  author={M{\oe}glin, C.},
  title = {Paquets d'Arthur Sp\'eciaux Unipotents aux Places Archim\'ediennes et Correspondance de Howe},
  journal = {J. Cogdell et al. (eds.), Representation Theory, Number Theory, and Invariant Theory, In Honor of Roger Howe. Progress in Math.}
  %series ={Progress in Math.},
  volume = {323},
  pages = {469--502}
  year = {2017}
}

\bib{MR}{article}{
  author={M{\oe}glin, C.},
    author={Renard, D.},
  title = {Sur les paquets d'Arthur des groupes classiques r\'eels},
  journal = {J. Eur. Math. Soc. },
  %series ={Progress in Math.},
  volume = {22},
    issue = {6},
    year = {2020},
    pages ={1827--1892}
    }


\bib{MVW}{book}{
  volume={1291},
  title={Correspondances de Howe sur un corps $p$-adique},
  author={M{\oe}glin, C.},
  author={Vign\'eras, M.-F.},
  author={Waldspurger, J.-L.},
  series={Lecture Notes in Mathematics},
  publisher={Springer}
  ISBN={978-3-540-18699-1},
  date={1987},
}

\bib{NOTYK}{article}{
   author = {Nishiyama, K.},
   author = {Ochiai, H.},
   author = {Taniguchi, K.},
   author = {Yamashita, H.},
   author = {Kato, S.},
    title = {Nilpotent orbits, associated cycles and Whittaker models for highest weight representations},
    journal = {Ast\'erisque},
    volume = {273},
    year = {2001},
   pages ={1--163},
}

\bib{NOZ}{article}{
  author = {Nishiyama, K.},
  author = {Ochiai, H.},
  author = {Zhu, C.-B.},
  journal = {Trans. Amer. Math. Soc.},
  title = {Theta lifting of nilpotent orbits for symmetric pairs},
  volume = {358},
  year = {2006},
  pages = {2713--2734},
}


\bib{NZ}{article}{
   author = {Nishiyama, K.},
   author = {Zhu, C.-B.},
    title = {Theta lifting of unitary lowest weight modules and their associated cycles},
    journal = {Duke Math. J.},
    volume = {125},
    issue = {03},
    year = {2004},
   pages ={415--465},
}



\bib{Ohta}{article}{
  author = {Ohta, T.},
  %doi = {10.2748/tmj/1178227492},
  journal = {Tohoku Math. J.},
  number = {2},
  pages = {161--211},
  publisher = {Tohoku University, Mathematical Institute},
  title = {The closures of nilpotent orbits in the classical symmetric
    pairs and their singularities},
  volume = {43},
  year = {1991}
}

\bib{Ohta2}{article}{
  author = {Ohta, T.},
  journal = {Hiroshima Math. J.},
  number = {2},
  pages = {347--360},
  title = {Induction of nilpotent orbits for real reductive groups and associated varieties of standard representations},
  volume = {29},
  year = {1999}
}

\bib{Ohta4}{article}{
  title={Nilpotent orbits of $\mathbb{Z}_4$-graded Lie algebra and geometry of
    moment maps associated to the dual pair $(\mathrm{U} (p, q), \mathrm{U} (r, s))$},
  author={Ohta, T.},
  journal={Publ. RIMS},
  volume={41},
  number={3},
  pages={723--756},
  year={2005}
}

\bib{PT}{article}{
  title={Some small unipotent representations of indefinite orthogonal groups and the theta correspondence},
  author={Paul, A.},
  author={Trapa, P.},
  journal={University of Aarhus Publ. Series},
  volume={48},
  pages={103--125},
  year={2007}
}


\bib{PV}{article}{
  title={Invariant Theory},
  author={Popov, V. L.},
  author={Vinberg, E. B.},
  book={
  title={Algebraic Geometry IV: Linear Algebraic Groups, Invariant Theory},
  series={Encyclopedia of Mathematical Sciences},
  volume={55},
  year={1994},
  publisher={Springer},}
}




%\bib{PPz}{article}{
%author={Protsak, V.} ,
%author={Przebinda, T.},
%title={On the occurrence of admissible representations in the real Howe
%    correspondence in stable range},
%journal={Manuscr. Math.},
%volume={126},
%number={2},
%pages={135--141},
%year={2008}
%}


\bib{PrzInf}{article}{
      author={Przebinda, T.},
       title={The duality correspondence of infinitesimal characters},
        date={1996},
     journal={Colloq. Math.},
      volume={70},
       pages={93--102},
}


\bib{Pz}{article}{
author={Przebinda, T.},
title={Characters, dual pairs, and unitary representations},
journal={Duke Math. J. },
volume={69},
number={3},
pages={547--592},
year={1993}
}

\bib{Ra}{article}{
author={Rallis, S.},
title={On the Howe duality conjecture},
journal={Compositio Math.},
volume={51},
pages={333--399},
year={1984}
}

\bib{RT}{article}{
   author={Renard, D.},
   author={Trapa, P.},
   title={Irreducible genuine characters of the metaplectic group:
   Kazhdan-Lusztig algorithm and Vogan duality},
   journal={Represent. Theory},
   volume={4},
   date={2000},
   pages={245--295},
   doi={10.1090/S1088-4165-00-00105-9},
}
	

\bib{Sa}{article}{
author={Sahi, S.},
title={Explicit Hilbert spaces for certain unipotent representations},
journal={Invent. Math.},
volume={110},
number = {2},
pages={409--418},
year={1992}
}

\bib{Se}{article}{
author={Sekiguchi, J.},
title={Remarks on real nilpotent orbits of a symmetric pair},
journal={J. Math. Soc. Japan},
%publisher={The Mathematical Society of Japan},
year={1987},
volume={39},
number={1},
pages={127--138},
}

\bib{SV}{article}{
  author = {Schmid, W.},
  author = {Vilonen, K.},
  journal = {Annals of Math.},
  number = {3},
  pages = {1071--1118},
  %publisher = {Princeton University, Mathematics Department, Princeton, NJ; Mathematical Sciences Publishers, Berkeley},
  title = {Characteristic cycles and wave front cycles of representations of reductive Lie groups},
  volume = {151},
year = {2000},
}

\bib{So}{article}{
author = {Sommers, E.},
title = {Lusztig's canonical quotient and generalized duality},
journal = {J. Algebra},
volume = {243},
number = {2},
pages = {790--812},
year = {2001},
}

\bib{SS}{book}{
  author = {Springer, T. A.},
  author = {Steinberg, R.},
  title = {Seminar on algebraic groups and related finite groups; Conjugate classes},
  series = {Lecture Notes in Math.}
  volume = {131}
publisher={Springer},
year={1970},
}

\bib{SZ1}{article}{
title={A general form of Gelfand-Kazhdan criterion},
author={Sun, B.},
author={Zhu, C.-B.},
journal={Manuscripta Math.},
pages = {185--197},
volume = {136},
year={2011}
}


%\bib{SZ2}{article}{
%  title={Conservation relations for local theta correspondence},
%  author={Sun, B.},
%  author={Zhu, C.-B.},
%  journal={J. Amer. Math. Soc.},
%  pages = {939--983},
%  volume = {28},
%  year={2015}
%}



\bib{Tr}{article}{
  title={Special unipotent representations and the Howe correspondence},
  author={Trapa, P.},
  year = {2004},
  journal={University of Aarhus Publication Series},
  volume = {47},
  pages= {210--230}
}

% \bib{Wa}{article}{
%    author = {Waldspurger, J.-L.},
%     title = {D\'{e}monstration d'une conjecture de dualit\'{e} de Howe dans le cas $p$-adique, $p \neq 2$ in Festschrift in honor of I. I. Piatetski-Shapiro on the occasion of his sixtieth birthday},
%   journal = {Israel Math. Conf. Proc., 2, Weizmann, Jerusalem},
%  year = {1990},
% pages = {267-324},
% }

\bib{V4}{article}{
   author={Vogan, D. A. },
   title={Irreducible characters of semisimple Lie groups. IV.
   Character-multiplicity duality},
   journal={Duke Math. J.},
   volume={49},
   date={1982},
   number={4},
   pages={943--1073},
   issn={0012-7094},
   review={\MR{683010}},
}
\bib{VoBook}{book}{
author = {Vogan, D. A. },
  title={Unitary representations of reductive Lie groups},
  year={1987},
  series = {Ann. of Math. Stud.},
 volume={118},
  publisher={Princeton University Press}
}


\bib{Vo89}{article}{
  author = {Vogan, D. A. },
  title = {Associated varieties and unipotent representations},
 %booktitle ={Harmonic analysis on reductive groups, Proc. Conf., Brunswick/ME (USA) 1989,},
  journal = {Harmonic analysis on reductive groups, Proc. Conf., Brunswick/ME
    (USA) 1989, Prog. Math.},
 volume={101},
  publisher = {Birkh\"{a}user, Boston-Basel-Berlin},
  year = {1991},
pages={315--388},
  editor = {W. Barker and P. Sally},
}

\bib{Vo98}{article}{
  author = {Vogan, D. A. },
  title = {The method of coadjoint orbits for real reductive groups},
 %booktitle ={Representation theory of Lie groups (Park City, UT, 1998)},
 journal = {Representation theory of Lie groups (Park City, UT, 1998). IAS/Park City Math. Ser.},
  volume={8},
  publisher = {Amer. Math. Soc.},
  year = {2000},
pages={179--238},
}

\bib{Vo00}{article}{
  author = {Vogan, D. A. },
  title = {Unitary representations of reductive Lie groups},
 %booktitle ={Mathematics towards the Third Millennium (Rome, 1999)},
 journal ={Mathematics towards the Third Millennium (Rome, 1999). Accademia Nazionale dei Lincei, (2000)},
  %series = {Accademia Nazionale dei Lincei, 2000},
 %volume={9},
pages={147--167},
}


\bib{Wa1}{book}{
  title={Real reductive groups I},
  author={Wallach, N. R.},
  year={1988},
  publisher={Academic Press Inc. }
}

\bib{Wa2}{book}{
  title={Real reductive groups II},
  author={Wallach, N. R.},
  year={1992},
  publisher={Academic Press Inc. }
}


\bib{Weil}{article}{
  title={Sur certain group d'operateurs unitaires},
  author={Weil, A.},
  year = {1964},
  journal={Acta Math.},
  volume = {111},
  pages= {143--211}
}


\bib{Weyl}{book}{
  title={The classical groups: their invariants and representations},
  author={Weyl, H.},
  year={1947},
  publisher={Princeton University Press}
}

\bib{Ya}{article}{
  title={Degenerate principal series representations for quaternionic unitary groups},
  author={Yamana, S.},
  year = {2011},
  journal={Israel J. Math.},
  volume = {185},
  pages= {77--124}
}

\bib{Zh}{article}{
  title={Local theta correspondence and nilpotent invariants},
  author={Zhu, C.-B.},
  year = {2019},
  journal={Proceedings of Symposia in Pure Mathematics},
  volume = {101},
  pages= {427--450}
}

% \bib{EGAIV4}{article}{
%   title = {\'El\'ements de g\'eom\'etrie alg\'brique IV 4: \'Etude locale des
%     sch\'emas et des morphismes de sch\'emas},
%   author = {Grothendieck, Alexandre},
%   author = {Dieudonn\'e, Jean},
%   journal  = {Inst. Hautes \'Etudes Sci. Publ. Math.},
%   volume = {32},
%   year = {1967},
%   pages = {5--361}
% }



\end{biblist}
\end{bibdiv}


\end{document}



%%% Local Variables:
%%% coding: utf-8
%%% mode: latex
%%% TeX-engine: xetex
%%% ispell-local-dictionary: "en_US"
%%% End:



