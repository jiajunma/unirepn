% !TeX program = xelatex
                                      \documentclass[12pt,a4paper]{amsart}
\usepackage[margin=2.5cm,marginpar=2cm]{geometry}

\usepackage[bookmarksopen,bookmarksdepth=2,hidelinks,colorlinks=false]{hyperref}
\usepackage[nameinlink]{cleveref}

% \usepackage[color]{showkeys}
% \makeatletter
%   \SK@def\Cref#1{\SK@\SK@@ref{#1}\SK@Cref{#1}}%
% \makeatother
%% FONTS

\usepackage{amssymb}
%\usepackage{amsmath}
\usepackage{mathrsfs}
\usepackage{mathtools}
%\usepackage{amsrefs}
\usepackage{mathbbol,mathabx}
\usepackage{amsthm}
\usepackage{graphicx}
\usepackage{braket}
%\usepackage[pointedenum]{paralist}
%\usepackage{paralist}


\usepackage{amsrefs}

\usepackage[all,cmtip]{xy}
\usepackage{rotating}
\usepackage{leftidx}
%\usepackage{arydshln}

%\DeclareSymbolFont{bbold}{U}{bbold}{m}{n}
%\DeclareSymbolFontAlphabet{\mathbbold}{bbold}


%\usepackage[dvipdfx,rgb,table]{xcolor}
\usepackage[rgb,table]{xcolor}
%\usepackage{mathrsfs}

\setcounter{tocdepth}{1}
\setcounter{secnumdepth}{2}

%\usepackage[abbrev,shortalphabetic]{amsrefs}


\usepackage[normalem]{ulem}

% circled number
\usepackage{pifont}
\makeatletter
\newcommand*{\circnuma}[1]{%
  \ifnum#1<1 %
    \@ctrerr
  \else
    \ifnum#1>20 %
      \@ctrerr
    \else
      \mbox{\ding{\numexpr 171+(#1)\relax}}%
     \fi
  \fi
}
\makeatother

\usepackage[centertableaux]{ytableau}


% Ytableau tweak
\makeatletter
\pgfkeys{/ytableau/options,
  noframe/.default = false,
  noframe/.is choice,
  noframe/true/.code = {%
    \global\let\vrule@YT=\vrule@none@YT
    \global\let\hrule@YT=\hrule@none@YT
  },
  noframe/false/.code = {%
    \global\let\vrule@YT=\vrule@normal@YT
    \global\let\hrule@YT=\hrule@normal@YT
  },
  noframe/on/.style = {noframe/true},
  noframe/off/.style = {noframe/false},
}

\def\hrule@enon@YT{%
  \hrule width  \dimexpr \boxdim@YT + \fboxrule *2 \relax
  height 0pt
}
\def\vrule@enon@YT{%
  \vrule height \dimexpr  \boxdim@YT + \fboxrule\relax
     width \fboxrule
}

\def\enon{\omit\enon@YT}
\newcommand{\enon@YT}[2][clear]{%
  \def\thisboxcolor@YT{#1}%
  \let\hrule@YT=\hrule@enon@YT
  \let\vrule@YT=\vrule@enon@YT
  \startbox@@YT#2\endbox@YT
  \nullfont
}

\makeatother
%\ytableausetup{noframe=on,smalltableaux}
\ytableausetup{noframe=off,boxsize=1.3em}
\let\ytb=\ytableaushort

\newcommand{\tytb}[1]{{\tiny\ytb{#1}}}


%\usepackage[mathlines,pagewise]{lineno}
%\linenumbers

\usepackage{enumitem}
%% Enumitem
\newlist{enumC}{enumerate}{1} % Conditions in Lemma/Theorem/Prop
\setlist[enumC,1]{label=(\alph*),wide,ref=(\alph*)}
\crefname{enumCi}{condition}{conditions}
\Crefname{enumCi}{Condition}{Conditions}
\newlist{enumT}{enumerate}{3} % "Theorem"=conclusions in Lemma/Theorem/Prop
\setlist[enumT]{label=(\roman*),wide}
\setlist[enumT,1]{label=(\roman*),wide}
\setlist[enumT,2]{label=(\alph*),ref ={(\roman{enumTi}.\alph*)}}
\setlist[enumT,3]{label=(\arabic*), ref ={(\roman{enumTi}.\alph{enumTii}.\alph*)}}
\crefname{enumTi}{}{}
\Crefname{enumTi}{Item}{Items}
\crefname{enumTii}{}{}
\Crefname{enumTii}{Item}{Items}
\crefname{enumTiii}{}{}
\Crefname{enumTiii}{Item}{Items}
\newlist{enumPF}{enumerate}{3}
\setlist[enumPF]{label=(\alph*),wide}
\setlist[enumPF,1]{label=(\roman*),wide}
\setlist[enumPF,2]{label=(\alph*)}
\setlist[enumPF,3]{label=\arabic*).}
\newlist{enumS}{enumerate}{3} % Statement outside Lemma/Theorem/Prop
\setlist[enumS]{label=\roman*)}
\setlist[enumS,1]{label=\roman*)}
\setlist[enumS,2]{label=\alph*)}
\setlist[enumS,3]{label=\arabic*.}
\newlist{enumI}{enumerate}{3} % items
\setlist[enumI,1]{label=\roman*),leftmargin=*}
\setlist[enumI,2]{label=\alph*), leftmargin=*}
\setlist[enumI,3]{label=\arabic*), leftmargin=*}
\newlist{enumIL}{enumerate*}{1} % inline enum
\setlist*[enumIL]{label=\roman*)}
\newlist{enumR}{enumerate}{1} % remarks
\setlist[enumR]{label=\arabic*.,wide,labelwidth=!, labelindent=0pt}
\crefname{enumRi}{remark}{remarks}

\crefname{equation}{}{}
\Crefname{equation}{Equation}{Equations}
\Crefname{lem}{Lemma}{Lemma}
\Crefname{thm}{Theorem}{Theorem}

\newlist{des}{description}{1}
\setlist[des]{font=\sffamily\bfseries}

% editing macros.
\blendcolors{!80!black}
\long\def\okay#1{\ifcsname highlightokay\endcsname
{\color{red} #1}
\else
{#1}
\fi
}
\long\def\editc#1{{\color{red} #1}}
\long\def\mjj#1{{{\color{blue}#1}}}
\long\def\mjjr#1{{\color{red} (#1)}}
\long\def\mjjd#1#2{{\color{blue} #1 \sout{#2}}}
\def\mjjb{\color{blue}}
\def\mjje{\color{black}}
\def\mjjcb{\color{green!50!black}}
\def\mjjce{\color{black}}

\long\def\sun#1{{{\color{cyan}#1}}}
\long\def\sund#1#2{{\color{cyan}#1  \sout{#2}}}
\long\def\mv#1{{{\color{red} {\bf move to a proper place:} #1}}}
\long\def\delete#1{}

%\reversemarginpar
\newcommand{\lokec}[1]{\marginpar{\color{blue}\tiny #1 \mbox{--loke}}}
\newcommand{\mjjc}[1]{\marginpar{\color{green}\tiny #1 \mbox{--ma}}}

\newcommand{\trivial}[2][]{\if\relax\detokenize{#1}\relax
  {%\hfill\break
   % \begin{minipage}{\textwidth}
      \color{orange} \vspace{0em} $[$  #2 $]$
  %\end{minipage}
  %\break
      \color{black}
  }
  \else
\ifx#1h
\ifcsname showtrivial\endcsname
{%\hfill\break
 % \begin{minipage}{\textwidth}
    \color{orange} \vspace{0em}  $[$ #2 $]$
%\end{minipage}
%\break
    \color{black}
}
\fi
\else {\red Wrong argument!} \fi
\fi
}

\newcommand{\byhide}[2][]{\if\relax\detokenize{#1}\relax
{\color{orange} \vspace{0em} Plan to delete:  #2}
\else
\ifx#1h\relax\fi
\fi
}



\newcommand{\Rank}{\mathrm{rk}}
\newcommand{\cqq}{\mathscr{D}}
\newcommand{\rsym}{\mathrm{sym}}
\newcommand{\rskew}{\mathrm{skew}}
\newcommand{\fraksp}{\mathfrak{sp}}
\newcommand{\frakso}{\mathfrak{so}}
\newcommand{\frakm}{\mathfrak{m}}
\newcommand{\frakp}{\mathfrak{p}}
\newcommand{\pr}{\mathrm{pr}}
\newcommand{\rhopst}{\rho'^*}
\newcommand{\Rad}{\mathrm{Rad}}
\newcommand{\Res}{\mathrm{Res}}
\newcommand{\Hol}{\mathrm{Hol}}
\newcommand{\AC}{\mathrm{AC}}
%\newcommand{\AS}{\mathrm{AS}}
\newcommand{\WF}{\mathrm{WF}}
\newcommand{\AV}{\mathrm{AV}}
\newcommand{\AVC}{\mathrm{AV}_\bC}
\newcommand{\VC}{\mathrm{V}_\bC}
\newcommand{\bfv}{\mathbf{v}}
\newcommand{\depth}{\mathrm{depth}}
\newcommand{\wtM}{\widetilde{M}}
\newcommand{\wtMone}{{\widetilde{M}^{(1,1)}}}

\newcommand{\nullpp}{N(\fpp'^*)}
\newcommand{\nullp}{N(\fpp^*)}
%\newcommand{\Aut}{\mathrm{Aut}}

\def\mstar{{\medstar}}


\newcommand{\bfone}{\mathbf{1}}
\newcommand{\piSigma}{\pi_\Sigma}
\newcommand{\piSigmap}{\pi'_\Sigma}


\newcommand{\sfVprime}{\mathsf{V}^\prime}
\newcommand{\sfVdprime}{\mathsf{V}^{\prime \prime}}
\newcommand{\gminusone}{\mathfrak{g}_{-\frac{1}{m}}}

\newcommand{\eva}{\mathrm{eva}}

% \newcommand\iso{\xrightarrow{
%    \,\smash{\raisebox{-0.65ex}{\ensuremath{\scriptstyle\sim}}}\,}}

\def\Ueven{{U_{\rm{even}}}}
\def\Uodd{{U_{\rm{odd}}}}
\def\ttau{\tilde{\tau}}
\def\Wcp{W}
\def\Kur{{K^{\mathrm{u}}}}

\def\Im{\operatorname{Im}}

\providecommand{\bcN}{{\overline{\cN}}}



\makeatletter

\def\gen#1{\left\langle
    #1
      \right\rangle}
\makeatother

\makeatletter
\def\inn#1#2{\left\langle
      \def\ta{#1}\def\tb{#2}
      \ifx\ta\@empty{\;} \else {\ta}\fi ,
      \ifx\tb\@empty{\;} \else {\tb}\fi
      \right\rangle}
\def\binn#1#2{\left\lAngle
      \def\ta{#1}\def\tb{#2}
      \ifx\ta\@empty{\;} \else {\ta}\fi ,
      \ifx\tb\@empty{\;} \else {\tb}\fi
      \right\rAngle}
\makeatother

\makeatletter
\def\binn#1#2{\overline{\inn{#1}{#2}}}
\makeatother


\def\innwi#1#2{\inn{#1}{#2}_{W_i}}
\def\innw#1#2{\inn{#1}{#2}_{\bfW}}
\def\innv#1#2{\inn{#1}{#2}_{\bfV}}
\def\innbfv#1#2{\inn{#1}{#2}_{\bfV}}
\def\innvi#1#2{\inn{#1}{#2}_{V_i}}
\def\innvp#1#2{\inn{#1}{#2}_{\bfV'}}
\def\innp#1#2{\inn{#1}{#2}'}

% choose one of then
\def\simrightarrow{\iso}
\def\surj{\twoheadrightarrow}
%\def\simrightarrow{\xrightarrow{\sim}}

\newcommand\iso{\xrightarrow{
   \,\smash{\raisebox{-0.65ex}{\ensuremath{\scriptstyle\sim}}}\,}}

\newcommand\riso{\xleftarrow{
   \,\smash{\raisebox{-0.65ex}{\ensuremath{\scriptstyle\sim}}}\,}}









\usepackage{xparse}
\def\usecsname#1{\csname #1\endcsname}
\def\useLetter#1{#1}
\def\usedbletter#1{#1#1}

% \def\useCSf#1{\csname f#1\endcsname}

\ExplSyntaxOn

\def\mydefcirc#1#2#3{\expandafter\def\csname
  circ#3{#1}\endcsname{{}^\circ {#2{#1}}}}
\def\mydefvec#1#2#3{\expandafter\def\csname
  vec#3{#1}\endcsname{\vec{#2{#1}}}}
\def\mydefdot#1#2#3{\expandafter\def\csname
  dot#3{#1}\endcsname{\dot{#2{#1}}}}

\def\mydefacute#1#2#3{\expandafter\def\csname a#3{#1}\endcsname{\acute{#2{#1}}}}
\def\mydefbr#1#2#3{\expandafter\def\csname br#3{#1}\endcsname{\breve{#2{#1}}}}
\def\mydefbar#1#2#3{\expandafter\def\csname bar#3{#1}\endcsname{\bar{#2{#1}}}}
\def\mydefhat#1#2#3{\expandafter\def\csname hat#3{#1}\endcsname{\hat{#2{#1}}}}
\def\mydefwh#1#2#3{\expandafter\def\csname wh#3{#1}\endcsname{\widehat{#2{#1}}}}
\def\mydeft#1#2#3{\expandafter\def\csname t#3{#1}\endcsname{\tilde{#2{#1}}}}
\def\mydefu#1#2#3{\expandafter\def\csname u#3{#1}\endcsname{\underline{#2{#1}}}}
\def\mydefr#1#2#3{\expandafter\def\csname r#3{#1}\endcsname{\mathrm{#2{#1}}}}
\def\mydefb#1#2#3{\expandafter\def\csname b#3{#1}\endcsname{\mathbb{#2{#1}}}}
\def\mydefwt#1#2#3{\expandafter\def\csname wt#3{#1}\endcsname{\widetilde{#2{#1}}}}
%\def\mydeff#1#2#3{\expandafter\def\csname f#3{#1}\endcsname{\mathfrak{#2{#1}}}}
\def\mydefbf#1#2#3{\expandafter\def\csname bf#3{#1}\endcsname{\mathbf{#2{#1}}}}
\def\mydefc#1#2#3{\expandafter\def\csname c#3{#1}\endcsname{\mathcal{#2{#1}}}}
\def\mydefsf#1#2#3{\expandafter\def\csname sf#3{#1}\endcsname{\mathsf{#2{#1}}}}
\def\mydefs#1#2#3{\expandafter\def\csname s#3{#1}\endcsname{\mathscr{#2{#1}}}}
\def\mydefcks#1#2#3{\expandafter\def\csname cks#3{#1}\endcsname{{\check{
        \csname s#2{#1}\endcsname}}}}
\def\mydefckc#1#2#3{\expandafter\def\csname ckc#3{#1}\endcsname{{\check{
      \csname c#2{#1}\endcsname}}}}
\def\mydefck#1#2#3{\expandafter\def\csname ck#3{#1}\endcsname{{\check{#2{#1}}}}}

\cs_new:Npn \mydeff #1#2#3 {\cs_new:cpn {f#3{#1}} {\mathfrak{#2{#1}}}}

\cs_new:Npn \doGreek #1
{
  \clist_map_inline:nn {alpha,beta,gamma,Gamma,delta,Delta,epsilon,varepsilon,zeta,eta,theta,vartheta,Theta,iota,kappa,lambda,Lambda,mu,nu,xi,Xi,pi,Pi,rho,sigma,varsigma,Sigma,tau,upsilon,Upsilon,phi,varphi,Phi,chi,psi,Psi,omega,Omega,tG} {#1{##1}{\usecsname}{\useLetter}}
}

\cs_new:Npn \doSymbols #1
{
  \clist_map_inline:nn {otimes,boxtimes} {#1{##1}{\usecsname}{\useLetter}}
}

\cs_new:Npn \doAtZ #1
{
  \clist_map_inline:nn {A,B,C,D,E,F,G,H,I,J,K,L,M,N,O,P,Q,R,S,T,U,V,W,X,Y,Z} {#1{##1}{\useLetter}{\useLetter}}
}

\cs_new:Npn \doatz #1
{
  \clist_map_inline:nn {a,b,c,d,e,f,g,h,i,j,k,l,m,n,o,p,q,r,s,t,u,v,w,x,y,z} {#1{##1}{\useLetter}{\usedbletter}}
}

\cs_new:Npn \doallAtZ
{
\clist_map_inline:nn {mydefsf,mydeft,mydefu,mydefwh,mydefhat,mydefr,mydefwt,mydeff,mydefb,mydefbf,mydefc,mydefs,mydefck,mydefcks,mydefckc,mydefbar,mydefvec,mydefcirc,mydefdot,mydefbr,mydefacute} {\doAtZ{\csname ##1\endcsname}}
}

\cs_new:Npn \doallatz
{
\clist_map_inline:nn {mydefsf,mydeft,mydefu,mydefwh,mydefhat,mydefr,mydefwt,mydeff,mydefb,mydefbf,mydefc,mydefs,mydefck,mydefbar,mydefvec,mydefdot,mydefbr,mydefacute} {\doatz{\csname ##1\endcsname}}
}

\cs_new:Npn \doallGreek
{
\clist_map_inline:nn {mydefck,mydefwt,mydeft,mydefwh,mydefbar,mydefu,mydefvec,mydefcirc,mydefdot,mydefbr,mydefacute} {\doGreek{\csname ##1\endcsname}}
}

\cs_new:Npn \doallSymbols
{
\clist_map_inline:nn {mydefck,mydefwt,mydeft,mydefwh,mydefbar,mydefu,mydefvec,mydefcirc,mydefdot} {\doSymbols{\csname ##1\endcsname}}
}



\cs_new:Npn \doGroups #1
{
  \clist_map_inline:nn {GL,Sp,rO,rU,fgl,fsp,foo,fuu,fkk,fuu,ufkk,uK} {#1{##1}{\usecsname}{\useLetter}}
}

\cs_new:Npn \doallGroups
{
\clist_map_inline:nn {mydeft,mydefu,mydefwh,mydefhat,mydefwt,mydefck,mydefbar} {\doGroups{\csname ##1\endcsname}}
}


\cs_new:Npn \decsyms #1
{
\clist_map_inline:nn {#1} {\expandafter\DeclareMathOperator\csname ##1\endcsname{##1}}
}

\decsyms{Mp,id,SL,Sp,SU,SO,GO,GSO,GU,GSp,PGL,Pic,Lie,Mat,Ker,Hom,Ext,Ind,reg,res,inv,Isom,Det,Tr,Norm,Sym,Span,Stab,Spec,PGSp,PSL,tr,Ad,Br,Ch,Cent,End,Aut,Dvi,Frob,Gal,GL,Gr,DO,ur,vol,ab,Nil,Supp,rank,Sign}

\def\abs#1{\left|{#1}\right|}
\def\norm#1{{\left\|{#1}\right\|}}


% \NewDocumentCommand\inn{m m}{
% \left\langle
% \IfValueTF{#1}{#1}{000}
% ,
% \IfValueTF{#2}{#2}{000}
% \right\rangle
% }
\NewDocumentCommand\cent{o m }{
  \IfValueTF{#1}{
    \mathop{Z}_{#1}{(#2)}}
  {\mathop{Z}{(#2)}}
}


\def\fsl{\mathfrak{sl}}
\def\fsp{\mathfrak{sp}}


%\def\cent#1#2{{\mathrm{Z}_{#1}({#2})}}


\doallAtZ
\doallatz
\doallGreek
\doallGroups
\doallSymbols
\ExplSyntaxOff


% \usepackage{geometry,amsthm,graphics,tabularx,amssymb,shapepar}
% \usepackage{amscd}
% \usepackage{mathrsfs}


\usepackage{diagbox}
% Update the information and uncomment if AMS is not the copyright
% holder.
%\copyrightinfo{2006}{American Mathematical Society}
%\usepackage{nicematrix}
\usepackage{arydshln}

\usepackage{tikz}
\usetikzlibrary{matrix,arrows,positioning,cd,backgrounds}
\usetikzlibrary{decorations.pathmorphing,decorations.pathreplacing}

\usepackage{upgreek}

\usepackage{listings}
\lstset{
    basicstyle=\ttfamily\tiny,
    keywordstyle=\color{black},
    commentstyle=\color{white}, % white comments
    stringstyle=\ttfamily, % typewriter type for strings
    showstringspaces=false,
    breaklines=true,
    emph={Output},emphstyle=\color{blue},
} 

\newcommand{\BA}{{\mathbb{A}}}
%\newcommand{\BB}{{\mathbb {B}}}
\newcommand{\BC}{{\mathbb {C}}}
\newcommand{\BD}{{\mathbb {D}}}
\newcommand{\BE}{{\mathbb {E}}}
\newcommand{\BF}{{\mathbb {F}}}
\newcommand{\BG}{{\mathbb {G}}}
\newcommand{\BH}{{\mathbb {H}}}
\newcommand{\BI}{{\mathbb {I}}}
\newcommand{\BJ}{{\mathbb {J}}}
\newcommand{\BK}{{\mathbb {U}}}
\newcommand{\BL}{{\mathbb {L}}}
\newcommand{\BM}{{\mathbb {M}}}
\newcommand{\BN}{{\mathbb {N}}}
\newcommand{\BO}{{\mathbb {O}}}
\newcommand{\BP}{{\mathbb {P}}}
\newcommand{\BQ}{{\mathbb {Q}}}
\newcommand{\BR}{{\mathbb {R}}}
\newcommand{\BS}{{\mathbb {S}}}
\newcommand{\BT}{{\mathbb {T}}}
\newcommand{\BU}{{\mathbb {U}}}
\newcommand{\BV}{{\mathbb {V}}}
\newcommand{\BW}{{\mathbb {W}}}
\newcommand{\BX}{{\mathbb {X}}}
\newcommand{\BY}{{\mathbb {Y}}}
\newcommand{\BZ}{{\mathbb {Z}}}
\newcommand{\Bk}{{\mathbf {k}}}

\newcommand{\CA}{{\mathcal {A}}}
\newcommand{\CB}{{\mathcal {B}}}
\newcommand{\CC}{{\mathcal {C}}}

\newcommand{\CE}{{\mathcal {E}}}
\newcommand{\CF}{{\mathcal {F}}}
\newcommand{\CG}{{\mathcal {G}}}
\newcommand{\CH}{{\mathcal {H}}}
\newcommand{\CI}{{\mathcal {I}}}
\newcommand{\CJ}{{\mathcal {J}}}
\newcommand{\CK}{{\mathcal {K}}}
\newcommand{\CL}{{\mathcal {L}}}
\newcommand{\CM}{{\mathcal {M}}}
\newcommand{\CN}{{\mathcal {N}}}
\newcommand{\CO}{{\mathcal {O}}}
\newcommand{\CP}{{\mathcal {P}}}
\newcommand{\CQ}{{\mathcal {Q}}}
\newcommand{\CR}{{\mathcal {R}}}
\newcommand{\CS}{{\mathcal {S}}}
\newcommand{\CT}{{\mathcal {T}}}
\newcommand{\CU}{{\mathcal {U}}}
\newcommand{\CV}{{\mathcal {V}}}
\newcommand{\CW}{{\mathcal {W}}}
\newcommand{\CX}{{\mathcal {X}}}
\newcommand{\CY}{{\mathcal {Y}}}
\newcommand{\CZ}{{\mathcal {Z}}}


\newcommand{\RA}{{\mathrm {A}}}
\newcommand{\RB}{{\mathrm {B}}}
\newcommand{\RC}{{\mathrm {C}}}
\newcommand{\RD}{{\mathrm {D}}}
\newcommand{\RE}{{\mathrm {E}}}
\newcommand{\RF}{{\mathrm {F}}}
\newcommand{\RG}{{\mathrm {G}}}
\newcommand{\RH}{{\mathrm {H}}}
\newcommand{\RI}{{\mathrm {I}}}
\newcommand{\RJ}{{\mathrm {J}}}
\newcommand{\RK}{{\mathrm {K}}}
\newcommand{\RL}{{\mathrm {L}}}
\newcommand{\RM}{{\mathrm {M}}}
\newcommand{\RN}{{\mathrm {N}}}
\newcommand{\RO}{{\mathrm {O}}}
\newcommand{\RP}{{\mathrm {P}}}
\newcommand{\RQ}{{\mathrm {Q}}}
%\newcommand{\RR}{{\mathrm {R}}}
\newcommand{\RS}{{\mathrm {S}}}
\newcommand{\RT}{{\mathrm {T}}}
\newcommand{\RU}{{\mathrm {U}}}
\newcommand{\RV}{{\mathrm {V}}}
\newcommand{\RW}{{\mathrm {W}}}
\newcommand{\RX}{{\mathrm {X}}}
\newcommand{\RY}{{\mathrm {Y}}}
\newcommand{\RZ}{{\mathrm {Z}}}

\DeclareMathOperator{\absNorm}{\mathfrak{N}}
\DeclareMathOperator{\Ann}{Ann}
\DeclareMathOperator{\LAnn}{L-Ann}
\DeclareMathOperator{\RAnn}{R-Ann}
\DeclareMathOperator{\ind}{ind}
%\DeclareMathOperator{\Ind}{Ind}



\newcommand{\cod}{{\mathrm{cod}}}
\newcommand{\cont}{{\mathrm{cont}}}
\newcommand{\cl}{{\mathrm{cl}}}
\newcommand{\cusp}{{\mathrm{cusp}}}

\newcommand{\disc}{{\mathrm{disc}}}
\renewcommand{\div}{{\mathrm{div}}}



\newcommand{\Gm}{{\mathbb{G}_m}}



\newcommand{\I}{{\mathrm{I}}}

\newcommand{\Jac}{{\mathrm{Jac}}}
\newcommand{\PM}{{\mathrm{PM}}}


\newcommand{\new}{{\mathrm{new}}}
\newcommand{\NS}{{\mathrm{NS}}}
\newcommand{\N}{{\mathrm{N}}}

\newcommand{\ord}{{\mathrm{ord}}}

%\newcommand{\rank}{{\mathrm{rank}}}

\newcommand{\rk}{{\mathrm{k}}}
\newcommand{\rr}{{\mathrm{r}}}
\newcommand{\rh}{{\mathrm{h}}}

\newcommand{\Sel}{{\mathrm{Sel}}}
\newcommand{\Sim}{{\mathrm{Sim}}}

\newcommand{\wt}{\widetilde}
\newcommand{\wh}{\widehat}
\newcommand{\pp}{\frac{\partial\bar\partial}{\pi i}}
\newcommand{\pair}[1]{\langle {#1} \rangle}
\newcommand{\wpair}[1]{\left\{{#1}\right\}}
\newcommand{\intn}[1]{\left( {#1} \right)}
\newcommand{\sfrac}[2]{\left( \frac {#1}{#2}\right)}
\newcommand{\ds}{\displaystyle}
\newcommand{\ov}{\overline}
\newcommand{\incl}{\hookrightarrow}
\newcommand{\lra}{\longrightarrow}
\newcommand{\imp}{\Longrightarrow}
%\newcommand{\lto}{\longmapsto}
\newcommand{\bs}{\backslash}

\newcommand{\cover}[1]{\widetilde{#1}}

\renewcommand{\vsp}{{\vspace{0.2in}}}

\newcommand{\Norma}{\operatorname{N}}
\newcommand{\Ima}{\operatorname{Im}}
\newcommand{\con}{\textit{C}}
\newcommand{\gr}{\operatorname{gr}}
\newcommand{\ad}{\operatorname{ad}}
\newcommand{\der}{\operatorname{der}}
\newcommand{\dif}{\operatorname{d}\!}
\newcommand{\pro}{\operatorname{pro}}
\newcommand{\Ev}{\operatorname{Ev}}
% \renewcommand{\span}{\operatorname{span}} \span is an innernal command.
%\newcommand{\degree}{\operatorname{deg}}
\newcommand{\Invf}{\operatorname{Invf}}
\newcommand{\Inv}{\operatorname{Inv}}
\newcommand{\slt}{\operatorname{SL}_2(\mathbb{R})}
%\newcommand{\temp}{\operatorname{temp}}
%\newcommand{\otop}{\operatorname{top}}
\renewcommand{\small}{\operatorname{small}}
\newcommand{\HC}{\operatorname{HC}}
\newcommand{\lef}{\operatorname{left}}
\newcommand{\righ}{\operatorname{right}}
\newcommand{\Diff}{\operatorname{DO}}
\newcommand{\diag}{\operatorname{diag}}
\newcommand{\sh}{\varsigma}
\newcommand{\sch}{\operatorname{sch}}
%\newcommand{\oleft}{\operatorname{left}}
%\newcommand{\oright}{\operatorname{right}}
\newcommand{\open}{\operatorname{open}}
\newcommand{\sgn}{\operatorname{sgn}}
\newcommand{\triv}{\operatorname{triv}}
\newcommand{\Sh}{\operatorname{Sh}}
\newcommand{\oN}{\operatorname{N}}

\newcommand{\oc}{\operatorname{c}}
\newcommand{\od}{\operatorname{d}}
\newcommand{\os}{\operatorname{s}}
\newcommand{\ol}{\operatorname{l}}
\newcommand{\oL}{\operatorname{L}}
\newcommand{\oJ}{\operatorname{J}}
\newcommand{\oH}{\operatorname{H}}
\newcommand{\oO}{\operatorname{O}}
\newcommand{\oS}{\operatorname{S}}
\newcommand{\oR}{\operatorname{R}}
\newcommand{\oT}{\operatorname{T}}
%\newcommand{\rU}{\operatorname{U}}
\newcommand{\oZ}{\operatorname{Z}}
\newcommand{\oD}{\textit{D}}
\newcommand{\oW}{\textit{W}}
\newcommand{\oE}{\operatorname{E}}
\newcommand{\oP}{\operatorname{P}}
\newcommand{\PD}{\operatorname{PD}}
\newcommand{\oU}{\operatorname{U}}

\newcommand{\g}{\mathfrak g}
\newcommand{\gC}{{\mathfrak g}_{\C}}
\renewcommand{\k}{\mathfrak k}
\newcommand{\h}{\mathfrak h}
\newcommand{\p}{\mathfrak p}
%\newcommand{\q}{\mathfrak q}
\renewcommand{\a}{\mathfrak a}
\renewcommand{\b}{\mathfrak b}
\renewcommand{\c}{\mathfrak c}
\newcommand{\n}{\mathfrak n}
\renewcommand{\u}{\mathfrak u}
\renewcommand{\v}{\mathfrak v}
\newcommand{\e}{\mathfrak e}
\newcommand{\f}{\mathfrak f}
\renewcommand{\l}{\mathfrak l}
\renewcommand{\t}{\mathfrak t}
\newcommand{\s}{\mathfrak s}
\renewcommand{\r}{\mathfrak r}
\renewcommand{\o}{\mathfrak o}
\newcommand{\m}{\mathfrak m}
\newcommand{\z}{\mathfrak z}
%\renewcommand{\sl}{\mathfrak s \mathfrak l}
\newcommand{\gl}{\mathfrak g \mathfrak l}


\newcommand{\re}{\mathrm e}

\renewcommand{\rk}{\mathrm k}

\newcommand{\Z}{\mathbb{Z}}
\DeclareDocumentCommand{\C}{}{\mathbb{C}}
\newcommand{\R}{\mathbb R}
\newcommand{\Q}{\mathbb Q}
\renewcommand{\H}{\mathbb{H}}
%\newcommand{\N}{\mathbb{N}}
\newcommand{\K}{\mathbb{K}}
%\renewcommand{\S}{\mathbf S}
\newcommand{\M}{\mathbf{M}}
\newcommand{\A}{\mathbb{A}}
\newcommand{\B}{\mathbf{B}}
%\renewcommand{\G}{\mathbf{G}}
\newcommand{\V}{\mathbf{V}}
\newcommand{\W}{\mathbf{W}}
\newcommand{\F}{\mathbf{F}}
\newcommand{\E}{\mathbf{E}}
%\newcommand{\J}{\mathbf{J}}
\renewcommand{\H}{\mathbf{H}}
\newcommand{\X}{\mathbf{X}}
\newcommand{\Y}{\mathbf{Y}}
%\newcommand{\RR}{\mathcal R}
\newcommand{\FF}{\mathcal F}
%\newcommand{\BB}{\mathcal B}
\newcommand{\HH}{\mathcal H}
%\newcommand{\UU}{\mathcal U}
%\newcommand{\MM}{\mathcal M}
%\newcommand{\CC}{\mathcal C}
%\newcommand{\DD}{\mathcal D}
\def\eDD{\mathrm{d}^{e}}
\def\DD{\nabla}
\def\DDc{\boldsymbol{\nabla}}
\def\gDD{\nabla^{\mathrm{gen}}}
\def\gDDc{\boldsymbol{\nabla}^{\mathrm{gen}}}
%\newcommand{\OO}{\mathcal O}
%\newcommand{\ZZ}{\mathcal Z}
\newcommand{\ve}{{\vee}}
\newcommand{\aut}{\mathcal A}
\newcommand{\ii}{\mathbf{i}}
\newcommand{\jj}{\mathbf{j}}
\newcommand{\kk}{\mathbf{k}}

\newcommand{\la}{\langle}
\newcommand{\ra}{\rangle}
\newcommand{\bp}{\bigskip}
\newcommand{\be}{\begin {equation}}
\newcommand{\ee}{\end {equation}}

\newcommand{\LRleq}{\stackrel{LR}{\leq}}

\numberwithin{equation}{section}


\def\flushl#1{\ifmmode\makebox[0pt][l]{${#1}$}\else\makebox[0pt][l]{#1}\fi}
\def\flushr#1{\ifmmode\makebox[0pt][r]{${#1}$}\else\makebox[0pt][r]{#1}\fi}
\def\flushmr#1{\makebox[0pt][r]{${#1}$}}


%\theoremstyle{Theorem}
% \newtheorem*{thmM}{Main Theorem}
% \crefformat{thmM}{main theorem}
% \Crefformat{thmM}{Main Theorem}
\newtheorem*{thm*}{Theorem}
\newtheorem{thm}{Theorem}[section]
\newtheorem{thml}[thm]{Theorem}
\newtheorem{lem}[thm]{Lemma}
\newtheorem{obs}[thm]{Observation}
\newtheorem{lemt}[thm]{Lemma}
\newtheorem*{lem*}{Lemma}
\newtheorem{whyp}[thm]{Working Hypothesis}
\newtheorem{prop}[thm]{Proposition}
\newtheorem{prpt}[thm]{Proposition}
\newtheorem{prpl}[thm]{Proposition}
\newtheorem{cor}[thm]{Corollary}
%\newtheorem*{prop*}{Proposition}
\newtheorem{claim}{Claim}
\newtheorem*{claim*}{Claim}
%\theoremstyle{definition}
\newtheorem{defn}[thm]{Definition}
\newtheorem{dfnl}[thm]{Definition}
\newtheorem*{IndH}{Induction Hypothesis}

\newtheorem*{eg*}{Example}
\newtheorem{eg}[thm]{Example}

\theoremstyle{remark}
\newtheorem*{remark}{Remark}
\newtheorem*{remarks}{Remarks}


\def\cpc{\sigma}
\def\ccJ{\epsilon\dotepsilon}
\def\ccL{c_L}

\def\wtbfK{\widetilde{\bfK}}
%\def\abfV{\acute{\bfV}}
\def\AbfV{\acute{\bfV}}
%\def\afgg{\acute{\fgg}}
%\def\abfG{\acute{\bfG}}
\def\abfV{\bfV'}
\def\afgg{\fgg'}
\def\abfG{\bfG'}

\def\half{{\tfrac{1}{2}}}
\def\ihalf{{\tfrac{\mathbf i}{2}}}
\def\slt{\fsl_2(\bC)}
\def\sltr{\fsl_2(\bR)}

% \def\Jslt{{J_{\fslt}}}
% \def\Lslt{{L_{\fslt}}}
\def\slee{{
\begin{pmatrix}
 0 & 1\\
 0 & 0
\end{pmatrix}
}}
\def\slff{{
\begin{pmatrix}
 0 & 0\\
 1 & 0
\end{pmatrix}
}}\def\slhh{{
\begin{pmatrix}
 1 & 0\\
 0 & -1
\end{pmatrix}
}}
\def\sleei{{
\begin{pmatrix}
 0 & i\\
 0 & 0
\end{pmatrix}
}}
\def\slxx{{\begin{pmatrix}
-\ihalf & \half\\
\phantom{-}\half & \ihalf
\end{pmatrix}}}
% \def\slxx{{\begin{pmatrix}
% -\sqrt{-1}/2 & 1/2\\
% 1/2 & \sqrt{-1}/2
% \end{pmatrix}}}
\def\slyy{{\begin{pmatrix}
\ihalf & \half\\
\half & -\ihalf
\end{pmatrix}}}
\def\slxxi{{\begin{pmatrix}
+\half & -\ihalf\\
-\ihalf & -\half
\end{pmatrix}}}
\def\slH{{\begin{pmatrix}
   0   & -\mathbf i\\
\mathbf i & 0
\end{pmatrix}}
}

\ExplSyntaxOn
\clist_map_inline:nn {J,L,C,X,Y,H,c,e,f,h,}{
  \expandafter\def\csname #1slt\endcsname{{\mathring{#1}}}}
\ExplSyntaxOff


\def\Mop{\fT}

\def\fggJ{\fgg_J}
\def\fggJp{\fgg'_{J'}}

\def\NilGC{\Nil_{\bfG}(\fgg)}
\def\NilGCp{\Nil_{\bfG'}(\fgg')}
\def\Nilgp{\Nil_{\fgg'_{J'}}}
\def\Nilg{\Nil_{\fgg_{J}}}
%\def\NilP'{\Nil_{\fpp'}}
\def\nNil{\Nil^{\mathrm n}}
\def\eNil{\Nil^{\mathrm e}}


\NewDocumentCommand{\NilP}{t'}{
\IfBooleanTF{#1}{\Nil_{\fpp'}}{\Nil_\fpp}
}

\def\KS{\mathsf{KS}}
\def\MM{\bfM}
\def\MMP{M}

\NewDocumentCommand{\KTW}{o g}{
  \IfValueTF{#2}{
    \left.\varsigma_{\IfValueT{#1}{#1}}\right|_{#2}}{
    \varsigma_{\IfValueT{#1}{#1}}}
}
\def\IST{\rho}
\def\tIST{\trho}

\NewDocumentCommand{\CHI}{o g}{
  \IfValueTF{#1}{
    {\chi}_{\left[#1\right]}}{
    \IfValueTF{#2}{
      {\chi}_{\left(#2\right)}}{
      {\chi}}
  }
}
\NewDocumentCommand{\PR}{g}{
  \IfValueTF{#1}{
    \mathop{\pr}_{\left(#1\right)}}{
    \mathop{\pr}}
}
\NewDocumentCommand{\XX}{g}{
  \IfValueTF{#1}{
    {\cX}_{\left(#1\right)}}{
    {\cX}}
}
\NewDocumentCommand{\PP}{g}{
  \IfValueTF{#1}{
    {\fpp}_{\left(#1\right)}}{
    {\fpp}}
}
\NewDocumentCommand{\LL}{g}{
  \IfValueTF{#1}{
    {\bfL}_{\left(#1\right)}}{
    {\bfL}}
}
\NewDocumentCommand{\ZZ}{g}{
  \IfValueTF{#1}{
    {\cZ}_{\left(#1\right)}}{
    {\cZ}}
}

\NewDocumentCommand{\WW}{g}{
  \IfValueTF{#1}{
    {\bfW}_{\left(#1\right)}}{
    {\bfW}}
}




\def\gpi{\wp}
\NewDocumentCommand\KK{g}{
\IfValueTF{#1}{K_{(#1)}}{K}}
% \NewDocumentCommand\OO{g}{
% \IfValueTF{#1}{\cO_{(#1)}}{K}}
\NewDocumentCommand\XXo{d()}{
\IfValueTF{#1}{\cX^\circ_{(#1)}}{\cX^\circ}}
\def\bfWo{\bfW^\circ}
\def\bfWoo{\bfW^{\circ \circ}}
\def\bfWg{\bfW^{\mathrm{gen}}}
\def\Xg{\cX^{\mathrm{gen}}}
\def\Xo{\cX^\circ}
\def\Xoo{\cX^{\circ \circ}}
\def\fppo{\fpp^\circ}
\def\fggo{\fgg^\circ}
\NewDocumentCommand\ZZo{g}{
\IfValueTF{#1}{\cZ^\circ_{(#1)}}{\cZ^\circ}}

% \ExplSyntaxOn
% \NewDocumentCommand{\bcO}{t' E{^_}{{}{}}}{
%   \overline{\cO\sb{\use_ii:nn#2}\IfBooleanTF{#1}{^{'\use_i:nn#2}}{^{\use_i:nn#2}}
%   }
% }
% \ExplSyntaxOff

\NewDocumentCommand{\bcO}{t'}{
  \overline{\cO\IfBooleanT{#1}{'}}}

\NewDocumentCommand{\oliftc}{g}{
\IfValueTF{#1}{\boldsymbol{\vartheta} (#1)}{\boldsymbol{\vartheta}}
}
\NewDocumentCommand{\oliftr}{g}{
\IfValueTF{#1}{\vartheta_\bR(#1)}{\vartheta_\bR}
}
\NewDocumentCommand{\olift}{g}{
\IfValueTF{#1}{\vartheta(#1)}{\vartheta}
}
% \NewDocumentCommand{\dliftv}{g}{
% \IfValueTF{#1}{\ckvartheta(#1)}{\ckvartheta}
% }
\def\dliftv{\vartheta}
\NewDocumentCommand{\tlift}{g}{
\IfValueTF{#1}{\wtvartheta(#1)}{\wtvartheta}
}

\def\slift{\cL}

\def\BB{\bB}


\def\thetaO#1{\vartheta\left(#1\right)}

\def\bbThetav{\check{\mathbbold{\Theta}}}
\def\Thetav{\check{\Theta}}
\def\thetav{\check{\theta}}

\DeclareDocumentCommand{\NN}{g}{
\IfValueTF{#1}{\fN(#1)}{\fN}
}
\DeclareDocumentCommand{\RR}{m m}{
\fR({#1},{#2})
}

%\DeclareMathOperator*{\sign}{Sign}

\NewDocumentCommand{\lsign}{m}{
{}^l\mathrm{Sign}(#1)
}



\NewDocumentCommand\lnn{t+ t- g}{
  \IfBooleanTF{#1}{{}^l n^+\IfValueT{#3}{(#3)}}{
    \IfBooleanTF{#2}{{}^l n^-\IfValueT{#3}{(#3)}}{}
  }
}


% % Fancy bcO, support feature \bcO'^a_b = \overline{\cO'^a_b}
% \makeatletter
% \def\bcO{\def\O@@{\cO}\@ifnextchar'\@Op\@Onp}
% \def\@Opnext{\@ifnextchar^\@Opsp\@Opnsp}
% \def\@Op{\afterassignment\@Opnext\let\scratch=}
% \def\@Opnsp{\def\O@@{\cO'}\@Otsb}
% \def\@Onp{\@ifnextchar^\@Onpsp\@Otsb}
% \def\@Opsp^#1{\def\O@@{\cO'^{#1}}\@Otsb}
% \def\@Onpsp^#1{\def\O@@{\cO^{#1}}\@Otsb}
% \def\@Otsb{\@ifnextchar_\@Osb{\@Ofinalnsb}}
% \def\@Osb_#1{\overline{\O@@_{#1}}}
% \def\@Ofinalnsb{\overline{\O@@}}

% Fancy \command: \command`#1 will translate to {}^{#1}\bfV, i.e. superscript on the
% lift conner.

% \def\defpcmd#1{
%   \def\nn@tmp{#1}
%   \def\nn@np@tmp{@np@#1}
%   \expandafter\let\csname\nn@np@tmp\expandafter\endcsname \csname\nn@tmp\endcsname
%   \expandafter\def\csname @pp@#1\endcsname`##1{{}^{##1}{\csname @np@#1\endcsname}}
%   \expandafter\def\csname #1\endcsname{\,\@ifnextchar`{\csname
%       @pp@#1\endcsname}{\csname @np@#1\endcsname}}
% }

% \def\defppcmd#1{
% \expandafter\NewDocumentCommand{\csname #1\endcsname}{##1 }{}
% }



% \defpcmd{bfV}
% \def\KK{\bfK}\defpcmd{KK}
% \defpcmd{bfG}
% \def\A{\!A}\defpcmd{A}
% \def\K{\!K}\defpcmd{K}
% \def\G{G}\defpcmd{G}
% \def\J{\!J}\defpcmd{J}
% \def\L{\!L}\defpcmd{L}
% \def\eps{\epsilon}\defpcmd{eps}
% \def\pp{p}\defpcmd{pp}
% \defpcmd{wtK}
% \makeatother

\def\fggR{\fgg_\bR}
\def\rmtop{{\mathrm{top}}}
\def\dimo{\dim^\circ}
\def\GKdim{\text{GK-}\dim}

\NewDocumentCommand\LW{g}{
\IfValueTF{#1}{L_{W_{#1}}}{L_{W}}}
%\def\LW#1{L_{W_{#1}}}
\def\JW#1{J_{W_{#1}}}

\def\floor#1{{\lfloor #1 \rfloor}}

\def\KSP{K}
\def\UU{\rU}
\def\UUC{\rU_\bC}
\def\tUUC{\widetilde{\rU}_\bC}
\def\OmegabfW{\Omega_{\bfW}}


\def\BB{\bB}


\def\thetaO#1{\vartheta\left(#1\right)}

\def\Thetav{\check{\Theta}}
\def\thetav{\check{\theta}}

\def\Thetab{\bar{\Theta}}

\def\cKaod{\cK^{\mathrm{aod}}}

\DeclareMathOperator{\sspan}{span}


\def\sp{{\mathrm{sp}}}

\def\bfLz{\bfL_0}
\def\sOpe{\sO^\perp}
\def\sOpeR{\sO^\perp_\bR}
\def\sOR{\sO_\bR}

\def\ZX{\cZ_{X}}
\def\gdliftv{\vartheta}
\def\gdlift{\vartheta^{\mathrm{gen}}}
\def\bcOp{\overline{\cO'}}
\def\bsO{\overline{\sO}}
\def\bsOp{\overline{\sO'}}
\def\bfVpe{\bfV^\perp}
\def\bfEz{\bfE_0}
\def\bfVn{\bfV^-}
\def\bfEzp{\bfE'_0}

\def\totimes{\widehat{\otimes}}
\def\dotbfV{\dot{\bfV}}

\def\aod{\mathrm{aod}}
\def\unip{\mathrm{unip}}
\def\IC{\mathfrak{I}}

\def\PI#1{\Pi_{\cI_{#1}}}
\def\Piunip{\Pi^{\mathrm{unip}}}
\def\cf{\emph{cf.} }
\def\Groth{\mathrm{Groth}}
\def\Irr{\mathrm{Irr}}
\def\Irrsp{\mathrm{Irr}^{\text{sp}}}

\def\edrc{\mathrm{DRC}^{\mathrm e}}
\def\drc{\mathrm{DRC}}
\def\LS{\mathrm{LS}}
\def\Unip{\mathrm{Unip}}


% Ytableau tweak
\makeatletter
\pgfkeys{/ytableau/options,
  noframe/.default = false,
  noframe/.is choice,
  noframe/true/.code = {%
    \global\let\vrule@YT=\vrule@none@YT
    \global\let\hrule@YT=\hrule@none@YT
  },
  noframe/false/.code = {%
    \global\let\vrule@YT=\vrule@normal@YT
    \global\let\hrule@YT=\hrule@normal@YT
  },
  noframe/on/.style = {noframe/true},
  noframe/off/.style = {noframe/false},
}
\makeatother 


\def\wAV{\AV^{\mathrm{weak}}}
\def\ckG{\check{G}}
\def\ckGc{\check{G}_{\bC}}
\def\dBV{d_{\mathrm{BV}}}
\def\CP{\mathsf{CP}}
\def\YD{\mathsf{YD}}
\def\SYD{\mathsf{SYD}}
\def\DD{\nabla}

\def\lamck{\lambda_\ckcO}
\def\lamckb{\lambda_{\ckcO_b}}
\def\lamckg{\lambda_{\ckcO_g}}
\def\Wint#1{W_{[#1]}}
\def\CLam{\Coh_{\Lambda}}
\def\Cint#1{\Coh_{[#1]}}
\def\PP{\mathrm{PP}}
\def\BOX#1{\mathrm{Box}(#1)}
\DeclareDocumentCommand{\bigtimes}{}{\mathop{\scalebox{1.7}{$\times$}}}
\providecommand\mapsfrom{\scalebox{-1}[1]{$\mapsto$}}

\def\Gc{G_\bC}
\def\Gcad{G_\bC^{\text{ad}}}
\def\hha{{}^a\fhh}
\def\aX{{}^aX}
\def\aQ{{}^aQ}
\def\aP{{}^aP}
\def\aR{{}^aR}
\def\aRp{{}^aR^+}
\def\asRp{{}^a \Delta^+}
\def\Gfin{\cG(\Gc)}
\def\PiGfin{\Pi_{\mathrm{fin}}( \Gc )}
\def\PiGlfin{\Pi_{\Lambda_0}( \Gc )}
\def\adGfin{\cG_{\mathrm{ad}}(\Gc)}
\def\Ggk{\cG(\fgg,K)}
\def\WT#1{\Delta(#1)}
\def\WG{W(\Gc)}
\def\ch{\mathrm{ch}\,}
\def\Wlam{W_{[\lambda]}}
\def\aLam{a_{\Lambda}}
\def\WLam{W_{\Lambda}}
\def\WLamck{W_{[\lambda_{\ckcO}]}}
\def\Wlamck{W_{\lamck}}
\def\Rlam{R_{[\lambda]}}
\def\RLam{R_\Lambda}
\def\RLamp{R_\Lambda^+}
\def\Rplam{R^+(\lambda)}
\def\Glfin{\cG_{\Lambda}(\Gc)}
\def\CL{{\sC}^{\scriptscriptstyle L}}
\def\CR{{\sC}^{\scriptscriptstyle R}}
\def\CLR{{\sC}^{\scriptscriptstyle LR}}
\def\LC{{}^{\scriptscriptstyle L}\sC}
\def\RC{{}^{\scriptscriptstyle R}\sC}
\def\LRC{{}^{\scriptscriptstyle LR}\sC}
\def\ckLC{{}^{\scriptscriptstyle L}\check{\sC}}
\def\LV{{}^{\scriptscriptstyle L}\sV}
\def\ckLV{{}^{\scriptscriptstyle L}\check\sV}

\def\bVL{{\overline{\sV}}^{\scriptscriptstyle L}}
\def\bVR{{\overline{\sV}}^{\scriptscriptstyle R}}
\def\bVLR{{\overline{\sV}}^{\scriptscriptstyle LR}}
\def\VL{{\sV}^{\scriptscriptstyle L}}
\def\VR{{\sV}^{\scriptscriptstyle R}}
\def\VLR{{\sV}^{\scriptscriptstyle LR}}

\def\Con{\sfC}
\def\bCon{\overline{\sfC}}
\def\Re{\mathrm{Re}}
\def\Im{\mathrm{Im}}
\def\AND{\quad \text{and} \quad}
\def\Coh{\mathrm{Coh}}
\def\Cohlm{\Coh_{\Lambda}(\cM)}
\def\ev#1{{\mathrm{ev}_{#1}}}

\def\ppp{\times}
\def\mmm{\slash}


\def\cuprow{{\stackrel{r}{\sqcup}}}
\def\cupcol{{\stackrel{c}{\sqcup}}}

\def\Spr{\mathrm{Springer}}
\def\Prim{\mathrm{Prim}}


\def\CQ{\overline{\sfA}}% Lusztig's canonical quotient
\def\CPP{\mathfrak{P}}
\def\CPPs{\mathfrak{P}_{\star}}


\def\ceil#1{\lceil #1 \rceil}
\def\symb#1#2{{\left(\substack{{#1}\\{#2}}\right)}}
\def\cboxs#1{\mbox{\scalebox{0.25}{\ytb{\ ,\vdots,\vdots,\ }}}_{#1}}

\def\hsgn{\widetilde{\mathrm{sgn}}}

\def\PBPe{\mathrm{PBP}^{\mathrm{ext}}}
\def\PBPes{\mathrm{PBP}^{\mathrm{ext}}_{\star}}
\def\bev#1{\overline{\mathrm{ev}}_{#1}}

\def\Prim{\mathrm{Prim}}
% \def\leqL{\stackrel{L}{\leq}}
% \def\leqR{\stackrel{R}{\leq}}
% \def\leqLR{\stackrel{LR}{\leq}}

% \def\leqL{{\leq_L}}
% \def\leqR{{\leq_R}}
% \def\leqLR{{\leq_{LR}}}


\def\lneqL{\mathrel{\mathop{<}\limits_{\scriptscriptstyle L}}}
\def\lneqR{\mathrel{\mathop{<}\limits_{\scriptscriptstyle R}}}
\def\lneqLR{\mathrel{\mathop{<}\limits_{\scriptscriptstyle LR}}}

\def\leqL{\mathrel{\mathop{\leq}\limits_{\scriptscriptstyle L}}}
\def\leqR{\mathrel{\mathop{\leq}\limits_{\scriptscriptstyle R}}}
\def\leqLR{\mathrel{\mathop{\leq}\limits_{\scriptscriptstyle LR}}}


\def\approxL{\mathrel{\mathop{\approx}\limits_{\scriptscriptstyle L}}}
\def\approxR{\mathrel{\mathop{\approx}\limits_{\scriptscriptstyle R}}}
\def\approxLR{\mathrel{\mathop{\approx}\limits_{\scriptscriptstyle LR}}}


\def\dphi{\rdd \phi}
\def\CPH{C(H)}
\def\whCPH{\widehat{C(H)}}

\def\Greg{G_{\text{reg}}}
\def\Hnreg{H^-_{\text{reg}}}
\def\Hireg{H_{i,\text{reg}}}
\def\Hnireg{H^-_{i,\text{reg}}}


\def\tsgn{\widetilde{\sgn}}
\def\PBP{\mathsf{PBP}}

\def\ckstar{{\check \star}}
\def\ckfgg{{\check \fgg}}

\def\imathp{\imath_{\sP}}
\def\jmathp{\jmath_{\sP}}

\def\tdBV{\tdd_{\text{BV}}}

\def\sP{\wp}

\begin{document}


\title[]{Counting special unipotent representations of real classical groups}

\author [D. Barbasch] {Dan M. Barbasch}
\address{the Department of Mathematics\\
  310 Malott Hall, Cornell University, Ithaca, New York 14853 }
\email{dmb14@cornell.edu}

\author [J.-J. Ma] {Jia-jun Ma}
\address{School of Mathematical Sciences\\
  Shanghai Jiao Tong University\\
  800 Dongchuan Road, Shanghai, 200240, China} \email{hoxide@sjtu.edu.cn}

\author [B. Sun] {Binyong Sun}
% MCM, HCMS, HLM, CEMS, UCAS,
\address{Academy of Mathematics and Systems Science\\
  Chinese Academy of Sciences\\
  Beijing, 100190, China} \email{sun@math.ac.cn}

\author [C.-B. Zhu] {Chen-Bo Zhu}
\address{Department of Mathematics\\
  National University of Singapore\\
  10 Lower Kent Ridge Road, Singapore 119076} \email{matzhucb@nus.edu.sg}




\subjclass[2000]{22E45, 22E46} \keywords{orbit method, unitary dual, unipotent
  representation, classical group, theta lifting, moment map}

% \thanks{Supported by NSFC Grant 11222101}
\maketitle


\tableofcontents
\section{Introduction}

% and $\mathfrak h$ be the abstract Cartan subalgebra of $\mathfrak g$.
%

Barbasch-Vogan \cite{BVUni,BV.W} developed a formula counting the number of
special unipotent representations.

In this paper, we will calculate the number of special unipotent representations
for real classical groups explicitly.



\subsection{Counting theorem for the real classical groups of type BCD}
In this section, we assume $\star \in \set{B,\wtC,C,C^{*}, D, D^{*}}$.

Let
\[
  \Gc =
  \begin{cases}
    \SO(2n+1,\bC) & \text{if } \star \in \set{B},\\
    \Sp(2n,\bC) & \text{if } \star \in \set{\wtC, C, C^{*}},\\
    \SO(2n,\bC) & \text{if } \star \in \set{D,D^{*}}.\\
  \end{cases}
\]
and the dual group of $\Gc$ is defined to be
\[
  \ckGc =
  \begin{cases}
    \Sp(2n,\bC) & \text{if } \star \in \set{B,\wtC},\\
    \SO(2n+1,\bC) & \text{if } \star \in \set{C, C^{*}},\\
    \SO(2n,\bC) & \text{if } \star \in \set{D,D^{*}}.\\
  \end{cases}
\]

Suppose $\ckcO\in \Nil(\ckG_{\bC})$.
Let $\ckcO = \ckcO_{b}\cuprow \ckcO_{g}$ be
the decomposition of $\ckcO$ into good and bad parity parts.

Let $\IC_{\star}(n)$ be the set of real groups given by
\[
  \IC_{\star}(n)
  :=
  \begin{cases}
    \Set{\SO(2p+1,2q)| p,q\in \bN \text{ and } p+q = n}
    & \text{if } \star = B\\
    \set{\Sp(2n,\bR)}
    & \text{if } \star = C\\
    \set{\Mp(2n,\bR)}
    & \text{if } \star = \wtC\\
    \Set{\Sp(p,q)| p,q\in \bN \text{ and } p+q = n}
    & \text{if } \star = C^{*}\\
    \Set{\SO(p,q)| p,q\in \bN \text{ and } p+q = 2n}
    & \text{if } \star = D\\
    \Set{\rO^{*}(2n)}
    & \text{if } \star = D^{*}\\
  \end{cases}
\]
and $\IC_{\star} = \bigcup_{n\in \bN} \IC_{\star}(n)$.

Let $ \unip_{\star}(\ckcO)$ be the set of special unipotent representations
of groups in $\IC_{\star}$ attached to $\ckcO$.


\begin{thm}
  Suppose $\ckcO = \ckcO_{g}$. Then
  \[
    \abs{\unip_{\star}(\ckcO)} \leq \abs{\PBPes(\ckcO)}.
  \]
  % \[
  %   \abs{\unip_{\star}(\ckcO)} = \begin{cases}
  %     2 \abs{\PBPes(\ckcO)} & \text{if } \star \in \set{B,D}\\
  %     \abs{\PBPes(\ckcO)} & \text{if } \star \in \set{\wtC,C,C^{*}, D^{*}}
  %     \end{cases}
  % \]
\end{thm}


\begin{thm}
  Let $\ckcO'_{b}$ be the partition such that $\ckcO'_{b}\cuprow \ckcO'_{b}
  = \ckcO_{b}$    % $ such that $ = \left(\dBV(\ckcO_{b})\right)^{t}$
  and
  \[\star' =
  \begin{cases}
     A^{\bR} & \star \in \set{B,\wtC,C,D}\\
     A^{\bH} & \star \in \set{C^{*},D^{*}}\\
    \end{cases}
  \]

  Then we have the following bijection
  \[
    \begin{array}{ccc}
    \Unip_{\star'}(\ckcO'_{b}) \times \Unip_{\star}(\ckcO_{g})
      & \longrightarrow & \Unip_{\star}(\ckcO)\\
      (\pi',\pi_{0})& \mapsto & \pi'\rtimes \pi_{0}.
    \end{array}
  \]
\end{thm}

% \begin{thm}
%   Suppose $\ckcO = \ckcO_{g}$. Then
%   \[
%     \abs{\unip_{\star}(\ckcO)}\leq
%     \abs{\PBPes(\ckcO)}.
%     %& \text{if } \star \in \set{\wtC,C,C^{*}, D^{*}}
%     % \begin{cases}
%     %   2 \abs{\PBPes(\ckcO)} & \text{if } \star \in \set{B,D}\\
%     %   \abs{\PBPes(\ckcO)} & \text{if } \star \in \set{\wtC,C,C^{*}, D^{*}}
%     %   \end{cases}
%   \]
% \end{thm}


\section{Counting formula}

Let $G$ be a real reductive group in the Harish-Chandra class.
Here $G$ can be a non-linear group.

Fixing a Cartan subalgebra $\fhh$ in $\fgg$,
we identify the set of infinitesimal character with $\fhh^{*}/W$
where $W$ is the Weyl group acting on $\fhh$.



Let $\cI_{\mu}$ be the maximal primitive ideal with infinitesimal character
$W\cdot \mu$. Then there is a unique double cell $\cD_{\mu}$
of $\Irr(W_{[\mu]})$ attached to
$\cI_{\mu}$ having associated variety $\cO_{\mu}$.

Let
\[
  \PI{\mu}(G):= \Set{\pi \in \Irr(G)| \cI_{\mu}
    \subseteq \Ann_{\cU(\fgg)}(\pi)}.
\]

Let $Q\subset \fhh^{*}$ be the root lattice and write $[\mu]:= \mu+Q$ for the coset
$\fhh^{*}/Q$ containing $\mu$.
Denote
$\Coh_{[\mu]}(G)$ the space of coherent families based on $[\mu]$ which has a $W_{[\mu]}$-action.

For each $\Gc$-invariant closed subset $\sfS$ in the nilpotent
cone of $\fgg$, let
\[
  \Coh_{[\mu], \sfS}(G):= \Set{
    \Theta \in \Coh_{[\mu]}|\AVC(\Theta(\mu')) \subset \sfS \text{ for }\mu'\in [\mu]}.
\]

The following theorem gives an upper bound of $\abs{\PI{mu}(G)}$.

\begin{thm}[Barbasch-Vogan]\label{thm:count}
  Let $\Pi_{\cO,\mu}(G)$ denote the set of irreducible admissible $G$-module with
  complex associated variety $\overline{\cO}$.
  % Let $W_{\mu}$ be the stabilizer of $\mu$ and $W_{[\mu]}$ be the stabilizer of
  % the lattice
  Then
  \[
    \begin{split}
      \abs{\PI{\mu}(G)} &= \sum_{\sigma\in \cD_{\mu}} [\sigma: \Coh_{[\mu]:\bcO_{\mu}}(G)] \cdot
      [1_{W_{\mu}}, \sigma|_{W_{\mu}}]\\
      & \leq \sum_{\sigma\in \cD_{\mu}} [\sigma: \Coh_{[\mu]}(G)] \cdot [1_{W_{\mu}}: \sigma|_{W_{\mu}}].
    \end{split}
  \]
  % Here $[\sgima : \ ]$ denote the multiplicity of $\sigma$
  % and  $W_{\mu}$ is the stabilizer of $\mu$.
\end{thm}



 \begin{lem}[{\cite{BVUni}*{(5.26), Proposition~5.28}}]\label{lem:lcell.BV}
  Let $\ckcO$ be a nilpotent orbit in $\ckGc$ and $\lamck$ be the infinitesimal
  character attached to $\ckcO$.
  Then the set
  \[
    \LC(\ckcO) := \Set{ \sigma \in \cD_{\lamck}| [1_{W_{\mu}}:\sigma]\neq 0}
  \]
  is a left cell in $\cD_{\lamck}$ given by
  \[
    (J_{\Wlamck}^{\WLamck} \sgn )\otimes \sgn.
  \]
  Moreover, the multiplicity $[1_{W_{\mu}}:\sigma]$ is one
  when $\sigma\in \LC(\ckcO)$.
\end{lem}

\begin{cor}
  Under the notation of \Cref{lem:lcell.BV}, we have
  \[
    \abs{\PI{\mu}(G)} \leq \sum_{\sigma\in \LC(\ckcO)} [\sigma: \Coh_{[\mu]}(G)]
  \]
\end{cor}

\subsection{Coherent family}

Let $Q$ be the $\fhh$ root lattice of $\fgg$. Let $\cG$ be the Grothendieck
group of finite dimensional $\fgg$-modules occur in $S(\fgg)$. Note that $\cG$
is also an algebra under tensor product.
% Via highest weight theory, there is a unique irreducib Via highest weight
% theory, $Q$ is identified with the set of irreducible $\fgg$-modules occur in
% $S(\fgg)$.
For each finite dimensional $\fhh$-module $F$, let $\WT{F}$ denote the multi-set
of $\fhh$-weights in $F$.


Via the highest weight theory, every $W$-orbit $W\cdot \mu$ in $\aQ$ corresponds
with the irreducible finite dimensional representation $F_{\mu}$ with extremal
weight $\mu$.


% Now the Grothendieck group $\Gfin$ of finite dimensional representation of
% $\Gc$ is identified with $\bZ[\aP/W]$. In fact $\Gfin$ is a $\bZ$-algebra
% under the tensor product and equipped with the involution $F\mapsto F^*$.

% Fix a $W$-invariant sub-lattice $\Lambda_0\subset \aX$ containing $\aQ$.

% Let $\Pi$ $\Glfin$ be the $\star$-invariant subalgebra of $\Gfin$ generated by
% irreducible representations corresponds to $\Lambda_0/W$.


For any $\lambda\in \hha^{*}$, we define the lattice
\[
  \Lambda := [\lambda ] := \lambda + Q \subset \fhh^{*}
\]
and define
\begin{equation}
  \label{eq:wlam}
  \begin{split}
    R_{[\lambda]} &:= \Set{\alpha\in R| \inn{\lambda}{\ckalpha}\in \bZ},\\
    W_{[\lambda]} &:=
    \set{w\in W | w\cdot \lambda  - \lambda \in Q}\\
    R_{\lambda} &:= \Set{\alpha\in \aR| \inn{\lambda}{\ckalpha}=0}, \AND\\
    W_{\lambda} &:= \braket{s_\alpha|\alpha\in R_{\lambda}} = \braket{w\in W|w\cdot \lambda = \lambda} \subseteq W.
  \end{split}
\end{equation}
It is known that $R_{[\lambda]}$ is a root system and
\[
  \begin{split}
    W_{[\lambda]} &= \Stab_{W}([\lambda]) = \braket{s_\alpha|\alpha\in R_{[\lambda]}} \subseteq W\quad
    \text{and}\\
    W_{ \lambda } &=   \braket{s_\alpha|\alpha \in R_{\lambda}} \subseteq W_{[\lambda]}.\\
  \end{split}
\]
In fact $W_{\lambda}$ is a parabolic subgroup of $W_{[\lambda]}$ by
Chevalley's theorem \cite{Vg}*{Lemma~6.3.28}.

When $\lambda$ is regular, let
\[
  R^{+}_{[\lambda]} := \Set{\alpha\in R_{[\lambda]}| \inn{\ckalpha}{\lambda}>0}
\]
be the fixed positive root system.

In the following we define the notion of coherent family based on the lattice
$[\lambda]$ in a quite general setting.

\begin{defn}
  Suppose that $\cM$ is be an abelian group with $\cG$-action:
  \[
    \cG\times \cM \ni(F,m)\mapsto F\otimes m.
  \]
  In addition, a subgroup $\cM_{\mu}$ of $\cM$ is fixed for each
  $\mu \in [\lambda]$ such that $\cM_{\mu} = \cM_{w\cdot \mu}$ for any
  $\mu\in [\lambda]$ and $w\in W_{[\lambda]}$.
  % for each $W_{[\lambda]}$-orbit
  % $\barmu := W_{\Lambda} \cdot \mu\in \Lambda/W_{\Lambda}$.

  A function $f\colon \Lambda \rightarrow \cM$ is called a coherent family based
  on $\Lambda$ if it satisfies $f(\mu)\in \cM_\mu$ and
  \[
    F\otimes f(\mu) = \sum_{\nu \in \WT{F}} f(\mu+\nu) \qquad \forall \mu\in \Lambda, F\in \cG.
  \]
  Let $\Coh_{\Lambda}(\cM)$ be the abelian group of all coherent families based
  on $\Lambda$ and taking value in $\cM$. We can define $W_{[\lambda]}$ action
  on $\Coh_[\lambda](\cM)$ by
  \[
    w\cdot f(\mu) = f(w^{-1}\cdot \mu) \qquad \forall \mu\in \Lambda, w\in \WLam.
  \]
\end{defn}



\def\Grt{\cG}

In this paper, we will consider the following cases.

\begin{eg}
  Suppose $\cM$ is a field of characteristic zero and
  \[
    F\otimes m := \dim(F)\cdot m \quad \text{for all } F\in \cG \text{ and
    } m\in \cM.
  \]
  We let $\cM_{\mu} = \cM$ for every $\mu\in \Lambda$. Then the set of
  $W$-harmonic polynomials on $\fhh$ is naturally identified with
  $\Coh_{[\lambda]}(\cM)$ via the restriction on $[\lambda]$ by Vogan
  \cite{VGK}*{Lemma~4.3}. \trivial{ Note that the polynomials are $W$-harmonic
    not necessary $W_{[\lambda]}$-harmonic. ($W_{[\lambda]}$-invariant
    differential operators are more than $W$-invariant differential operators.)
  }
\end{eg}


\begin{eg}\label{eg:hw}
  Fix a Borel subalgebra $\fbb = \fhh\oplus \fnn \subset \fgg$, let
  $\Grt(\fgg,\fhh,\fnn)$ be the Grothendieck group of the category $\cO$ with
  coefficients in $\bC$, i.e. the category of finitely generated
  $\cU(\fgg)$-modules with semisimple $\fhh$-action and locally finite
  $\fnn$-action. For $\lambda\in \fhh^{*}$, let $\Grt_{W\cdot \lambda}(\fgg,\fhh,\fnn)$
  be the subgroup spanned by $\fgg$-modules with infinitesimal character
  $W\cdot \lambda$. Here $\cG$ acts on $\Grt(\fgg,\fhh,\fnn)$ via the tensor
  product of $\fgg$-modules.

  To ease the notation, we write
  \[
    \Coh_{[\lambda]}(\fgg,\fhh,\fnn) := \Coh_{[\lambda]}(\Grt(\fgg,\fhh,\fnn)).
  \]

  % Verma modules gives a basis of We now review the well understood structure
  % of $\Coh_{[\lambda]}$. is well understood.
  For $\lambda\in \fhh^{*}$, let $\rho := \sum_{\alpha\in \WT{\fnn}} \alpha$,
  \[
    M(\lambda) := \cU(\fgg)\otimes_{\cU(\fbb)} \bC_{\lambda-\rho}
  \]
  be the Verma module with highest weight $\lambda-\rho$ and $L(\lambda) $ be
  the unique irreducible quotient of $M(\lambda)$.

  Each $w\in W$ defines a coherent family %such that
  \[
    M_w(\mu) := M(w\cdot \mu) \quad \forall \mu \in [\lambda].
    % M_w(\mu) := M(w \cdot \mu) \quad \forall \mu \in [\lambda].
  \]
  % where $w_{0}$ is the longest element in $W$.

  The map
  \[
    \begin{array}{ccc}
      \bC[W] & \longrightarrow & \Coh_{[\lambda]}(\fgg,\fhh,\fnn)  \\
      w& \mapsto &M_{w}
    \end{array}
  \]
  is $W_{[\lambda]}$-equivariant isomorphism where $W_{[\lambda]}$ acts on $\bC[W]$ by right
  translation.

  One of the crucial property is that each irreducible module can be fitted into
  a coherent family. %in other words.
  More precisely, the evaluation map descents to yields an isomorphism
  $\bev{\mu}$ in the following diagram.
  \begin{equation}\label{eq:bev.catO}
    \begin{tikzcd}
      \Coh_{[\lambda]}(\fgg,\fhh,\fnn)\ar[r,"\Theta\mapsto \Theta(\mu)"] \ar[d]&
      \Grt_{W\cdot \mu}(\fgg,\fhh,\fnn)\\
      \left(\Coh_{[\lambda]}(\fgg,\fhh,\fnn)\right)_{W_{\mu}} \ar[ru,hook,two heads,"\bev{\mu}"']
    \end{tikzcd}
  \end{equation}

  \trivial[]{ The subjectivity is because of Verma modules form a basis of the
    category $\cO$. The LHS of $\bev{\mu}$ has dimension $\abs{W/W_{\mu}}$, the
    RHS has dimension $W\cdot \mu$. Now the isomorphism follows by dimension
    counting. }
  % the The space $\Coh_\Lambda(\cG(\fgg,\fhh,\fnn))$ and
  % $\Coh_{\Lambda(\cG)}$defined similarly.

  % Note that the lattice $\Lambda$ is stable under the $\Wlam$ action.
\end{eg}

\begin{eg}\label{eg:Coh.HC}
  Suppose $G$ is a reductive Lie group in the Harish-Chandra class. Let
  $\Grt(G)$ be the Grothendieck group of finite length admissible $G$-modules
  and $\Grt_{\mu}(G)$ be the subgroup of $\Grt(\fgg,K)$ generated by the set of
  irreducible $G$-modules with infinitesimal character $\mu$.

  By \cite{Vg}*{0.4.6}, we can naturally identify $Q$ with the set of
  $H^{s}$-weights consisting the characters occurs in $S(\fgg)$ where $H^{s}$ is
  a maximally split Cartan in $G$. Therefore, the set of irreducible
  $G$-submodules occur in $S(\fgg)$ is also naturally identified with $Q/W=\cG$.
  We let $\cG$ acts on $\Grt(G)$ by the tensor product of $G$-modules.

  \trivial[]{ Note that by the assumption that $G$ is in the Harish-Chandra
    class, each irreducible $\fgg$-submodule $F$ embeds in $S(\fgg)$ is
    automatically globalized to a $G$-module. The point is that the
    globalization is independent of the embedding of $F$ in $S(\fgg)$! }

  We write
  \[
    \Coh_{[\lambda]}(G) := \Coh_{[\lambda]}(\Grt(G))
  \]
  for the space of coherent family of $G$-modules.
  % Then $\Coh_\Lambda(\cG(\fgg,K))$ is the group of coherent families of
  % Harish-Chandra modules. The space $\Coh_\Lambda(\cG(\fgg,K))$ is equipped
  % with a $\WLam$-action by
  % \[
  %   w\cdot f(\mu) = f(w^{-1} \mu) \qquad \forall \mu\in \Lambda, w\in \WLam, f \in \Coh_{\Lambda}(\Grt(\fgg,K)).
  % \]
\end{eg}

\begin{eg}
  Fix a $\Gc$-invariant closed subset $\sfS$ in the nilpotent cone of $\fgg$.
  Let $\Grt_{\sfS}(G)$ be the Grothendieck group of finite length admissible
  $G$-modules whose complex associated varieties are contained in $\sfS$. Define
  \[
    \Grt_{\mu,\sfS}(G) := \Grt_{\mu}(G)\cap \Grt_{\sfS}(G).
  \]
  and write
  \[
    \Coh_{[\lambda],\sfS}(G):= \Coh_{[\lambda]}(\Grt_{\sfS}(\fgg,K)).
  \]

  Note that
  \[
    \AVC(\pi\otimes F) = \AVC(\pi).
  \]
  for each finite length $G$-module $\pi$ and finite dimensional $G$-module $F$.
  Therefore the $W_{[\lambda]}$-module
  \[
    \Coh_{[\lambda],\sfS}(G) = (\ev{\mu})^{-1}(\Grt_{\mu,\sfS}(G))
  \]
  for any regular $\mu\in [\lambda]$.

  Similarly, we define the space $\Coh_{[\lambda],\sfS}(\fgg,\fhh,\fnn)$ of
  coherent families in category $\cO$ whose associated variety are contained in
  $\sfS$. In particular,
  \[
    \Coh_{[\lambda],\sfS}(\fgg,\fhh,\fnn) = (\ev{\mu})^{-1}(\Grt_{\mu,\sfS}(\fgg,\fhh,\fnn)).
  \]
  % now is also a $\WLam$-submodule of $\Coh_\Lambda(\cG(\fgg,K))$.
\end{eg}


% \begin{eg}
%   For each infinitesimal character $\chi$ and a close $G$-invariant set
%   $\cZ\in \cN_{\fgg}$. Let $\Grt_{\,\cZ}(\fgg,K)$ be the Grothendieck group of
%   $(\fgg,K)$-module with infinitesimal character $\chi$ and complex associated
%   variety contained $\cZ$. Similarly, let $\Grt_{\chi,\cZ}()$
% \end{eg}

A remarkable fact that the diagram \Cref{eq:bev.catO} still holds for
$\Coh_{[\lambda],\sfS}(G)$. This is one of the first step towards the counting
of the set
\[
  \Irr_{\mu,\sfS}(G):=\Set{\pi \in \Irr_{\mu}(G)| \text{$\AVC(\pi)\subset \sfS$} }.
\]
where
\[
  \Irr_{\mu}(G):=\Set{\pi \in \Irr(G)| \pi \text{ has infinitesimal character
    }\mu}.
\]


\begin{lem}\label{lem:coh.count}
  For each $\mu$ and closed $\Gc$-invariant subset $\sfS$ in the nilpotent cone
  of $\fgg$, we have an isomorphism
  \[
    \bev{\mu}\colon \left(\Coh_{[\mu],\sfS}(G)\right)_{W_{\mu}} \longrightarrow \Grt_{\mu,\sfS}(G).
  \]
  In particular,
  \[
    \abs{\Irr_{\mu,\sfS}(G)} = [1_{W_{\mu}}:\Coh_{[\mu],\sfS}(G)]
  \]
  % \[
  %   \dim {\barmu} = \dim (\cohm)_{W_\mu} = [\cohm, 1_{W_\mu}].
  % \]
\end{lem}
\begin{proof}
  This is a consequence of the formal properties of the translation functor,
  especially the theory of $\tau$-invariant.

  \def\Parm{\mathrm{Parm}} \def\cof{\Theta}


  {\bf The properties of the coherent family and translation principal:} (We
  refers to \cite{Vg}*{Section~7} for the proofs which also work in our
  (possibly non-linear) setting.)

  \begin{enumerate}[label=(\alph*)]
    \item \label{it:t1} The evaluation map
          \[
          \ev{\mu}\colon \Coh_{[\mu]}(G)\rightarrow \Grt_{\mu}(G)
          \]
          is surjective for any $\mu \in [\mu]$, see \cite{Vg}*{Theorem~7.2.7}.
  \end{enumerate}
  From now on we fix a regular element $\lambda \in [\mu]$ such that $\mu$ is
  dominant with respect to $R^{+}_{[\lambda]}$
  \begin{enumerate}[resume*]
    \item \label{it:t2} The evaluation map $\ev{\lambda}$ at $\lambda$ is an
          isomorphism \cite{Vg}*{Proposition~7.2.27}.
  \end{enumerate}
  For each $\pi\in \Grt_{\lambda}(G)$, let
  \[
    \Theta_{\pi}:= (\ev{\lambda})^{-1}(\pi)
  \]
  be the unique coherent family such that $\Theta_{\pi}(\lambda) = \pi$,
  \begin{enumerate}[resume*]
    \item \label{it:t3} If $\pi\in \Irr_{\lambda}(G)$, then $\Theta_{\pi}(\mu)$
          is either zero or an irreducible $G$-module
          \cite{Vg}*{Proposition~7.3.10, Corollary~7.3.23}.
    \item \label{it:t4} For $\pi\in \Irr_{\lambda}(G)$,
          \[
          \AV(\Theta_{\pi}(\mu)) = \AV(\pi)
          \]
          whenever $\mu$ is dominant and $\Theta_{\pi}(\mu)$ non-zero.
          \trivial[]{ This because
          $\pi = \psi_{\mu}^{\lambda}(\Theta_{\pi}(\mu))$ and
          $\Theta_{\pi}(\mu) = \psi_{\lambda}^{\mu}(\pi)$. Here is the
          translation functor from $\lambda$ to $\mu$ see
          \cite{Vg}*{Definition~4.5.7}. Translation dose not increase the
          associated variety. }
    \item \label{it:t5} If $\pi$ and $\pi'$ are in $\Irr_{\lambda}(G)$ such that
          $\Theta_{\pi}(\mu) = \Theta_{\pi'}(\mu)$ is non-zero, then $\pi=\pi'$.
  \end{enumerate}
  For $\pi\in \Irr_{\lambda}(G)$, define the $\tau$-invariant of $\pi$ to be
  \begin{equation}\label{eq:taupi}
    \tau(\pi) := \Set{\alpha\in R^{+}_{[\lambda]}|
      \begin{array}{l}
        \text{$\alpha$ is simple and }\\
        s_{\alpha}\cdot \Theta_{\pi}(\lambda) = -\Theta_{\pi}(\lambda)
      \end{array}
    }
  \end{equation}
  \begin{enumerate}[resume*]
    \item
          \label{it:t6}
          $\Theta_{\pi}(\mu) =0$ if and only if
          $\tau(\gamma)\cap R_\mu \neq \emptyset$
          \cite{Vg}*{Corollary~7.3.23~(c)}.
  \end{enumerate}

  Now we start to prove the lemma. By the translation principle,
  \[
    \begin{split}
      & \Set{\Theta_{\pi}(\mu)| \pi\in \Irr_{\lambda, \sfS}(G)
        \text{ s.t. } \Theta_{\pi}(\mu)\neq 0} \\
      = & \Set{\Theta_{\pi}(\mu)| \pi\in \Irr_{\lambda, \sfS}(G) \text{ s.t.
        } \tau(\pi)\cap R_{\mu}= \emptyset}.
    \end{split}
  \]
  forms a basis of $\Grt_{\mu,\sfS}(G)$. \trivial{ The set consists of distinct
    (so linearly independent) irreducible $G$-modules by \ref{it:t3} and
    \ref{it:t5}. They are spanning set by \ref{it:t1}. For the support
    condition, see \ref{it:t4}. The $\tau$-invariant condition is by
    \ref{it:t6}.
    % by \ref{it:t6} and
  } Hence
  \[
    \begin{split}
      \ker \ev{\mu} = & \Span \Set{\Theta_{\pi}| \pi\in \Irr_{\lambda, \sfS}(G) \text{
          s.t. }
        \tau(\pi)\cap R_{\mu}\neq \emptyset}\\
      \subseteq & \Span \Set{\half(\Theta_\pi - s_{\alpha}\cdot \Theta_{\pi}) | \pi\in \Irr_{\lambda, \sfS}(G) \text{
          and } \alpha \in
        \tau(\pi)\cap R_{\mu}}\\
      & \ \ \ \  \text{(by the definition of $\tau(\pi)$ in \eqref{eq:taupi}.)} \\
      \subseteq &\Span\Set{\Theta- w\cdot \Theta |\Theta\in \Coh_{[\mu],\sfS}(G)} \\
      \subseteq & \ker \ev{\mu}. \\
      &\ \ \ \ \text{(by
        $w\cdot \Theta(\mu) = \Theta(w^{-1}\cdot \mu)=\Theta_\pi(\mu)$)}
    \end{split}
  \]
  Since
  $\left(\Coh_{[\mu],\sfS}(G)\right)_{W_{\mu}} =\Coh_{[\mu],\sfS}(G)\slash \Span\set{\Theta- w\cdot \Theta |\Theta\in \Coh_{[\mu],\sfS}(G)} $,
  the lemma follows.
\end{proof}

\section{Primitive ideals and Weyl group representations}

\subsection{Associated varieties of a primitive ideals and double cells in
  $W_{[\lambda]}$}
In this section, we review the notion of double cells and its relation with the
associated varieties of primitive ideals, see \cite{BV2,J.av}.
We retain the notation in \Cref{eg:hw}.

Let $\Prim_{\lambda}(\fgg)$ be the set of primitive ideals in $\cU(\fgg)$ with
infinitesimal character $\lambda$. Let $\lambda \in \fhh^{*}$, each primitive ideal is
the annihilator of a highest weight module by Duflo \cite{Du77}.  In other
words, the following map is
surjective
\[
  \begin{array}{ccl}
    W_{[\lambda]} &\longrightarrow &  \Prim_{\lambda}(\fgg)\\
    w & \mapsto & I(w\cdot \lambda) := \Ann L(w\cdot \lambda).
  \end{array}
\]



%We now recall some results about the blocks in category $\cO$.

By the translation principal, we
concentrate the discussion in the regular infinitesimal character case. From now on, we follows the convention in \cite{BV2}.
Let $\lambda$ be a regular element in $\fhh^{*}$ such that
$R^{+}_{[\lambda]}\subset - \WT{\fnn}$.
\trivial{
  Here $\lambda$ is regular anti-dominant ($\inn{\lambda}{\ckalpha}\notin \bN$ for
  each $\alpha\in \WT{\fnn}$) with respect to the root system defining highest
  weight modules, but it is dominant with respect to $R^{+}_{[\lambda]}$.
}
For each $w\in W_{[\lambda]}$, define
\[
a(w) := \abs{\WT{\fnn}} - \GKdim(L(w\lambda)).
\]
\trivial{
  Suppose $\lambda$ is integral, then
  Under this definition, $a_{w_{0}} = \abs{\WT{\fnn}}$ and
  $a_{e} =0$.
}
For each $w$ one can attach a polynomial $\wtpp_{w}$ such that
$\wtpp_{w}(\mu) = \rank(\cU(\fgg)/\Ann(L(w\mu)))$ when $\mu\in [\lambda]$ is
dominant (i.e. $-\inn{\mu}{\ckalpha}\notin \bN^{+}$ for all
$\alpha\in R^{+}_{[\lambda]}$).
$\wtpp_{w}$ is called the Goldie-rank
polynomial attached to the primitive ideal $\Ann(L(w\lambda))$.
Fix a dominant regular element $\delta$ in $\fhh$ (i.e.
$\inn{\delta}{\alpha}>0$ for each $\alpha\in \WT{\fnn}$).
Let
\[
r_{w} = \sum_{y\in W_{[\lambda]}} a_{y,w} (y^{-1}\delta)^{a(w)} \in S(\fhh)
\]
where $a_{y,w}$ is determined by the equation
\[
  L(w\lambda) = \sum_{y\in W_{[\lambda]}} a_{y,w} M(w\lambda)
\]
in $\Grt(\fgg,\fhh,\fnn)$.
Then $r_{w}$ is a positive multiple of $\wtpp_{w}$ \cite{J2}*{Section~1.4}.

% Let $w_{0}$ (resp. $w_{[\lambda]}$)be the longest element in $W$ (resp.
% $\Wlam$).
A partial order $\leqL$ can be define on $W_{[\lambda]}$ by the following condition
\cite{BV2}*{Proposition~2.9}
\begin{equation}\label{eq:leqL}
  \begin{split}
    w_{1} \leqL w_{2} & \Leftrightarrow
    I(w_{1}\lambda)\subseteq I(w_{2}\lambda)\\
    & \Leftrightarrow
    %[L(w_{2}^{-1}\lambda), L(w_{1}^{-1}\lambda)\otimes S(\fgg)] \neq 0.
    L(w_{2}^{-1}\lambda) \text{ is a subquotient of } L(w_{1}^{-1}\lambda)\otimes S(\fgg).
  \end{split}
\end{equation}

We say $w_{1} \approxL w_{2}$ if and only if $w_{1}\leqL w_{2}\leqL w_{1}$.
For $w\in W_{[\lambda]}$, we call
\[
  \CL_{w} := \Set{ w' \in W_{[\lambda]}| w\approxL w'}
\]
the left cell in $W_{[\lambda]}$ containing $w$.

In summary, we have a bijection
\[
  \begin{array}{ccc}
    W_{[\lambda]}/ \approxL &\longrightarrow & \Prim_{\lambda}(\fgg)\\
    \CL_{w} & \mapsto & \Ann L(w\lambda).
  \end{array}
\]
% here $w$ is a arbitrary element in $\LC$.
% Moreover, for each left cell $\LC_{w}$



The partial order $\leqR$ is defined by
\[
  w_{1}\leqR w_{2} \Leftrightarrow w_{1}^{-1} \leqL w_{2}^{-1}.
\]
The partial order $\leqLR$ is defined to be the minimal partial order
containing $\leqL$ and $\leqR$.
The relation $\approxR$, $\approxLR$, right cell $\CR_{
  w}$ and double cells
$\CLR_{w}$ are defined similarly.

Since the Kazhdan-Lusztig conjecture has been proven (for the integral
infinitesimal character case by \cite{BB,BK} and reduced to the integral
infinitesimal character case by \cite{Soergel} (see \cite{H}*{Section~13.13})),
the definition of the order $\leqL$ is the same as the partial order defined by
Kazhdan-Lusztig  \cite{KL} which only depends on the Coxeter group structure of
$W_{[\lambda]}$, see \cite{BV2}*{Corollary~2.3}.
\trivial[]{
  Note that $x\lneqL y$ implies $a(x)<a(y)$
  and $x\approxLR y$ implies $a(x)=a(y)$.
}

Note that the left cells are exactly the fibers of the map $w\mapsto \wtpp_{w}$.
Take a double cell $\CLR_{w}$ in $W_{[\lambda]}$ and a set of representatives
$\set{w_{1}, w_{2}, \cdots, w_{k}}$ of the left cells in $\CLR_{w}$.
% and decompose it into disjoint union of left-cells
% \[
%   \LRC_{w} = \bigsqcup_{i=1}^{k} \LC_{w_{i}}.
% \]
% Due to Joseph\cite{J2} and Barbasch-Vogan\cite{BV1,BV2}, we have the following
% statements:
%
Due to Barbasch-Vogan\cite{BV1,BV2} and Joseph\cite{J1,J2,J3,J.av},  the following
statements holds:
\begin{itemize}
  \item the set of Goldie rank polynomials
  $\set{\wtpp_{w_{i}}|i = 1,2,\cdots,k}$ form a basis of a special
  representation $\sigma_{w}$ of $W_{[\lambda]}$ realized in
  $S^{a(w)}(\fhh)$;
  \item the multiplicity of $\sigma_{w}$ in $S^{a(w)}(\fhh)$ is one,
  \item $a(w)$ is the minimal degree $m$ such that $\sigma_{w}$ occurs in
  $S^{m}(\fhh)$ which is the fake degree and the generic degree of the special
        representation $\sigma_{w}$. \trivial[]{ When
    $W_{[\lambda]} = W$, this is the definition of the fake degree.
    Otherwise, $\fhh = \fhh_{0}\oplus \fhh^{W_{\lambda}}$ where $\fhh_{0}$
    is the span of coroots of $W_{[\lambda]}$. Then $S(\fhh_{0})$ is embeds
    in $S(\fhh)$. }
  \item the map
  $W_{[\lambda]}/\approxLR \; \ni \CLR_{w}\mapsto \sigma_{w}\in \Irr(W_{[\lambda]})$
  yields a bijection between the set of double cells and the set
  $\Irrsp(W_{[\lambda]})$ of special representations of $W_{[\lambda]}$.
  \item Under the $W$ action, the $W_{[\lambda]}$-module
  $\sigma_{w}\subset S^{a(w)}(\fhh)$ generates an irreducible $W$-module
  $\wtsigma_{w}:=j_{W_{[\lambda]}}^{W}\sigma_{w}$. The $W$-module
  $\wtsigma_{w}$ corresponds to the a nilpotent orbit $\cO_{\wtsigma_{w}}$
  with trivial local system. Now the complex associated variety
  \[
    \Gc^{ad}\AV(L(w\lambda)) = \AV(\Ann(L(w\lambda))) =\overline{\cO_{\wtsigma_{w}}},
  \]
  where $\Gc^{ad}$ is the adjoint group of $\fgg$, see
  \cite{J.av}*{Section~2.10}.
\end{itemize}
\trivial[]{
  Note that the $j$-induction is not injective in general.
  For example,
  $j_{S_{a}\times S_{b}}^{S_{a+b}} \tau_{a}\otimes \tau_{b} = \tau_{a}\cupcol \tau_{b}$
  where $\tau_{a}$, $\tau_{b}$ are partitions.
}

% \subsubsection*{Clan of primitive ideals}
% Under the above notation,
% we say two
Following Joseph, we say two primitive ideals $I(w\lambda)$ and  $I(w'\lambda)$
with $w, w'\in W_{[\lambda]}$are in the same
\emph{clan} if and only if $w\approxLR w'$ or equivalently $\sigma_{w}=\sigma_{w'}$.

Obvious, the associated varieties of two primitive ideals in the
same clan has the same associated variety. However the reverse dose not holds in
the non-integral infinitesimal character case in general.

\subsubsection*{Coherent continuations}
Now we recall
the relationships between cells and the coherent continuation
representations.

For each $w\in W$, we define a coherent family $L_{w}$ by the condition
  $L_{w}(\lambda) = L(w\lambda)$.

For each $\mu\in \fhh^{*}$, let
$\cO_{[\mu]}(\fgg,\fhh,\fnn)$ be the subcategory of the category $\cO(\fgg,\fhh,\fnn)$
consists of modules whose $\fhh$-weights are contained in $[\mu]$,
\[
\Grt_{W\cdot\lambda,[\mu]}(\fgg,\fhh,\fnn) = \Grt_{[\mu]}(\fgg,\fhh,\fnn)\cap \Grt_{W\cdot \lambda}(\fgg,\fhh,\fnn)
\]
which is spanned by a block in the category $\cO$ if
$W\cdot \lambda \cap [\mu+\rho]\neq \emptyset$.

Consider the following subgroup of coherent continuation
\[
 \Coh_{[\lambda]}(\fgg,\fhh,\fnn; [\mu]) =
 \Set{\Theta\in \Coh_{[\lambda]}(\fgg,\fhh,\fnn)| \Theta(\lambda) \in
   \Grt_{W\cdot \lambda, [\mu]}(\fgg,\fhh,\fnn)
 }.
\]

We identify $\bC[W_{[\lambda]}]$ with
$\Coh_{[\lambda]}(\fgg,\fhh,\fnn,[\lambda-\rho])$ via $w \mapsto M_{w}$.

For each $w\in W_{[\lambda]}$,
let
\[
  \begin{split}
\bVR_{w}& :=\Span\Set{L_{w'}|w\leqR w'} \\
\VR_{w}&= \bVR _{w}\left/ \sum_{w\lneqL w'} \bVR_{w'} \right.
\end{split}
\]
By \eqref{eq:leqL},
$\bVR_{w}$ and $\VR_{w}$ are $W_{[\lambda]}$-modules under right
translation/coherent continuation action.

We define left $W_{[\lambda]}$-module $\bVL_{w}$ and $\VL_{w}$ using $\leqL$ and
$W_{[\lambda]}\times W_{[\lambda]}$-module $\bVLR_{w}$ and $\VLR_{w}$ using
$\leqLR$ similarly.

Suppose $\sigma_{1}, \sigma_{2}\in \Irr(W_{[\lambda]})$. We define
\[
  \sigma_{1}\leqLR \sigma_{2} \Leftrightarrow
  \exists w\in W_{[\lambda]} \text{ such that }
  \begin{cases}
  \sigma_{1} \otimes \sigma_{1} \text{ occurs in } \VLR_{w}
  \text{ and } \\
  \sigma_{2} \otimes \sigma_{2} \text{ occurs in } \bVLR_{w}.
\end{cases}
\]
Now $\sigma_{1}\approxLR \sigma_{2}$ if and only if there exists a
$w\in W_{[\lambda]}$ such that $\sigma_{1} \otimes \sigma_{1}$ and
$\sigma_{2} \otimes \sigma_{2}$ both occur in $\VLR_{w}$.
Now $\leqLR$ is a well
defined partial order and  $\approxLR$ is an equivalent
relation on $\Irr(W_{[\lambda]})$ respectively.
We write $\LRC_{\sigma}\subseteq \Irr(\Wlam)$ for the double cell containing
$\sigma$.
\trivial[]{
  A priori $\sigma_{1}\approxLR \sigma_{2}\Leftrightarrow
  \sigma_{1}\leqLR \sigma_{2}\leqLR \sigma_{1}$.

  But note that $\bigoplus_{w\in \Wlam/\approxLR} \VLR_{w} \cong \bC[\Wlam]$
  and $\sigma\otimes \sigma$ has multiplicity one in $\bC[\Wlam]$
  which implies the claim.
}

A left (resp. right cell) in $\Irr(\Wlam)$ is the multiset of the irreducible
constituents in $\VL_{w}$  (resp. left cell) for some $w\in \Wlam$.

The equivalence of Barbasch-Vogan's definition  and
  Lusztig's definition of cells in $\Irr(\Wlam)$ is a consequence of
  Kazadan-Lusztig conjecture, see \cite{BV2}*{remarks after Corollary~2.16}.

The structure of double and left cells are explicitly described in
\cite{Lu}*{Section~4}.
In particular, $\sigma_{w}$ is the unique special representation occurs
in the double cell
\[
  \LRC_{w}:= \Set{\sigma| \sigma\otimes \sigma \text{ occurs in } \VLR_{w}}
  \subseteq \Irr(\Wlam).
\]
For this reason, we also write
\[
  \LRC_{\sigma}:=\LRC_{w}
\]
where $\sigma=\sigma_{w}$ is the unique special representation in $\LRC_{w}$.
The generic degree ``a''-function is constant on the double cells and order
preserving: for each $\sigma'\in
\LRC_{\sigma}$, the generic degree $a(\sigma')=a(\sigma)$;
$a(\sigma')<a(\sigma'')$ if $\sigma'\lneqLR \sigma''$.

In summary, we have bijections
\[
  \Wlam/\approxLR \longleftrightarrow\Irrsp(\Wlam)\longleftrightarrow \Irr(\Wlam)/\approxLR.
\]
We write $\VLR_{\sigma}$ to be the unique double cell representation
containing $\sigma$ and $\bVLR_{\sigma}$ to be the unique upper cone
representation which is isomorphic to
$\bigoplus_{\sigma\leqLR \sigma'}\sigma'\otimes \sigma'$.


\subsection{Compare blocks}
Now we compare different blocks. Without of loss of generality, we assume
$\lambda$ is in the anti-dominant cone of $\WT{\fnn}$, i.e
$\inn{\lambda}{\ckalpha}<0$ for all $\alpha\in \WT{\fnn}$.

Let $k=\abs{W/\Wlam}$ and
\[
%\Set{r_1,\cdots, r_{k}}
% \Set{r|\text{the length of $r$ with respect to simple roots in $-\WT{\fnn}$ is minimal in $r\Wlam$} }
\Set{r_1,\cdots, r_{k}} :=
\Set{r|l(r) \text{ is minimal among elements in }r\Wlam }
\]
% \[
% \set{r_{i}\Wlam|i =1, 2, \cdots, k}
% \] be the list of right cosets of $W/\Wlam$.
be the set of distinguished representatives of the right cosets of $\Wlam$ where
$l(r)$ denote the length function with respect to the simple roots in $-\WT{\fnn}$. In
other words, $r_{i}$ is the unique element in the coset $r_{i}\Wlam$
such that $r_{i}\lambda$ is anti-dominant, i.e.   $R^{+}_{[r_{i}\lambda]}\subseteq -\WT{\fnn}$.
% We choose $r_{i}$ to be the unique element in the coset $r_{i}\Wlam$
% such that $r_{i}\lambda$ is anti-dominant, i.e.   $R^{+}_{[r_{i}\lambda]}\subseteq -\WT{\fnn}$.

%Let $S_{[\lambda]}$ be the set of simple roots in $R^{+}_{[\lambda]}$.
%Apply Soergel's theorem \cite{H}*{13.13}, we have
% Then
% \begin{itemize}
%   \item The map $L(w\lambda) \mapsto
%   L(r_{i} w\lambda)$ with $w$ running over $w\in W_{[\lambda]}$ induces an equivalence of
%   category from $\cO_{W\cdot \lambda, [\lambda]}(\fgg,\fhh,\fnn)\cap$ to
%   $\cO_{W\cdot\lambda, [r_{i}\lambda]}$ by Soegel's theorem \cite{H}*{13.13}.
%   \item $w\mapsto r_{i} w r_{i}^{-1}$ induces isomorphism
%   $W_{[\lambda]}\rightarrow W_{[r_{i}\lambda]}$ and preserves the cell
%   structures.
%   \item The following $\Wlam$-module isomorphism
%   \[
%     \begin{tikzcd}
%       & \bC[W_{}]& \\
%     \end{tikzcd}
%   \]
% \end{itemize}

Now  the map $L(w\lambda) \mapsto
  L(r_{i} w\lambda)$ with $w$ running over $w\in W_{[\lambda]}$ induces an equivalence of
  category from $\cO_{W\cdot \lambda, [\lambda]}(\fgg,\fhh,\fnn)\cap$ to
  $\cO_{W\cdot\lambda, [r_{i}\lambda]}$ by Soegel's theorem \cite{H}*{13.13}.
In particular $w\mapsto r_{i} w r_{i}^{-1}$ induces isomorphism
  $W_{[\lambda]}\rightarrow W_{[r_{i}\lambda]}$ and preserves the cell
  structures.
  In other words, the following $\Wlam$-module isomorphism
  \[
    \begin{tikzcd}
      & \bC[W_{}]\ar[dl, "w\mapsto M_{w}"'] \ar[dr,"w\mapsto M_{r_{i}w}"]& \\
      \Coh_{[\lambda]}(\fgg,\fhh,\fnn;[\lambda])\ar[rr]
      & &
      \Coh_{[\lambda]}(\fgg,\fhh,\fnn;[r_{i}\lambda])
    \end{tikzcd}
  \]
  maps cell representations to cell representations.

\trivial{
  Let $C = \set{x\in \fhh| \inn{x}{\alpha} >0  \ \forall \alpha\in \WT{\fnn})}$
    and
$D_{\lambda} = \set{x\in \fhh| -\inn{x}{\beta} >0 \ \forall \beta\in R^+_{[\lambda]}}$.
$C$ and $D_{\lambda}$ are fundamental domains of $\fhh$ under $W$ and
$W_{[\lambda]}$-actions.
Clearly, $D_{w\lambda} = w D_{\lambda}$.
The condition that $R^{+}_{[\mu]}\subset - \WT{\fnn}$ is equivalent to $D_{\mu}\supset C$.

Now it is clear $D_{\lambda}$ is the union of $r_{i}^{-1} C$
when $r_{i}$ running over the preferred  coset representatives
of $W/W_{[\lambda]}$.
}

Recall the definition of clan of the primitive ideals.
By the comparing the definition of Goldie rank polynomials, we see that
$I(r_{i}w)$ and $I(r_{j}w')$ in the same clan if and only if
$w\approxLR w'$.
In other word, the clan is only depends on the $W_{[\lambda]}$-type of the
double cell containing $L(w\lambda)$ where $w\in W$.

\medskip

\def\Dsp{\cD^{\text{sp}}}
\def\Csp{\sfC^{\text{sp}}}
The above discussion yields the following.
\begin{lem}\label{lem:C.S}
  Fix a $\Gcad$-invariant subset $\sfS$ in the nilpotent cone of $\fgg$.

  Let
  \begin{equation}\label{eq:C.S}
    \begin{split}
  \Csp_{\sfS} &:= \Set{\sigma \in \Irrsp(\Wlam)| \Spr(j_{\Wlam}^{W}\sigma)\subseteq \sfS}, \\
  \sfC_{\sfS} &:=
  \Set{\sigma'\in \Irr(\Wlam)| \exists \sigma \in \Csp_{\sfS}
    \text{ such that} \sigma \leqLR \sigma'}
  % \bigcup_{\sigma\in \Dsp_{\sfS}}
  \end{split}
\end{equation}
% Let $\VLR_{\sigma}$ be the unique
  Then, as an $W_{[\lambda]}$-module
  \[
    \begin{split}
      \Coh_{[\lambda],\sfS}(\fgg,\fhh,\fnn) &= \bigoplus_{i=1}^{k}
      \Coh_{[\lambda], \sfS}(\fgg,\fhh,\fnn;[r_{i}\lambda-\rho])\\
      & \cong \bigoplus_{i=1}^{k} \sum_{\sigma\in \Csp_{\sfS}} \VLR_{\sigma}\\
      & \cong \bigoplus_{i=1}^{k} \bigoplus_{\sigma\in \sfC_{\sfS}}
      (\dim \sigma) \sigma
    \end{split}
\]
\end{lem}

As a baby case of the counting theorem for special unipotent representation
of real reductive groups, we have the following counting theorem in the category
$\cO$.

We fix an regular element $\lambda\in [\mu]$ such that $\mu$ is dominant with
respect to $R^{+}_{[\lambda]}$. Let $S_{[\lambda]}$ be the set of simple roots
in $R^{+}_{[\lambda]}$. Let $\sfS_{\mu}$ be the subset of simple roots in
$R^{+}_{[\lambda]}$ orthogonal to $\mu$. Observe that $W_{\mu}$ is always a
parabolic subgroup attached to $\sfS_{\mu}$ in $W_{[\lambda]} = W_{[\mu]}$. Let
$\sfD_{\sfS_{\mu}}$ be the set of distinguished right coset representatives of
$W_{[\lambda]}/W_{\mu}$.
\trivial[]{
  $r\in \sfD_{\sfS_{\mu}}$ is the element with minimal lenght in $rW_{\mu}$.
  Recall that $\tau(w) = \set{\alpha\in S_{[\lambda]}|w\alpha\notin R^{+}_{[\lambda]} }$
  Note that
  $\tau(w)\cap R_{\mu}\neq \emptyset$ is equivalent to require that
  $w \sfS_{\mu}\subseteq R^{+}_{[\lambda]}$, i.e. $w$ is a minimal length
  element. See for example, Carter, Simple groups of Lie type, Theorem~2.5.8.
}

\begin{thm}
  Let $\cO$ be an nilpotent orbit in $\fgg$ and $\mu\in \fhh$.
  Let $\Pi_{W\cdot \mu, \cO}$ be the set of irreducible highest
  weight modules $\pi$ such that  $\AVC(\pi) = \bcO$.
  Let
  \begin{equation}\label{eq:DC.O.mu}
    \begin{split}
      \Dsp_{\cO,\mu} &:= \Set{\sigma\in \Irrsp(W_{[\mu]})|\Spr(j_{W_{[\mu]}}^{W}\sigma) = \cO}\quad \text{
        and }\\
      \cD_{\cO,\mu} &= \bigcup_{\sigma\in \Dsp_{\cO,\mu}} \LRC_{\sigma}.
    \end{split}
  \end{equation}
  % $ be the set of special representations such that
  Then
  \[
    \abs{\Pi_{W\cdot\mu,\cO}} = \abs{W/W_{[\mu]}}\cdot \sum_{\sigma\in \cD_{\cO,\mu}}
    \left(\dim \sigma \cdot [1_{W_{\mu}}:\sigma]\right).
  \]


  Let
  \[
  \CLR_{\sigma,\mu} = \CLR_{\sigma,\mu}\cap \sfD_{\sfS_{\mu}}
  \]
  % \[
  % \CLR_{\sigma,\mu} = \Set{w\in \CLR_{\sigma}|\tau(w)\cap R_{\mu}= \emptyset}
  % \]
  and $\CLR_{\cO,\mu} = \bigcup_{\sigma\in \Dsp_{\cO}} \CLR_{\sigma,\mu}$.
  Then
  \[
  \Pi_{W\cdot\mu,\cO} = \Set{L(r_{i}w)|w\in \CLR_{\cO,\mu} \text{ and } i=1,2,\cdots,k}
  .
  \]
  Here $\VLR_{\sigma}$ is understood as a submodule of $\bC[\Wlam]$.
\end{thm}


Fix $\mu\in \fhh$ and let $\cI_{\mu}$ be the maximal primitive ideal with
infinitesimal character $\mu$.
Let $\Pi_{W\cdot \mu}$ be the set of irreducible highest
weight modules $\pi$ such that  $\Ann(\pi) = \cI_{\mu}$.

Let $a(\sigma)$ be the generic degree of a Weyl group representation $\sigma$.

In view of \Cref{thm:count}, we need the following lemma by Barbasch-Vogan. %in \cite{BVUni}.
\begin{lem}[{\cite{BVUni}*{(5.26), Proposition~5.28}}]
  \label{lem:LC.mu}
  Let
  \[
    a_{\mu} = \max\set{a(\sigma)| \sigma \in \Irr(W_{[\mu]}) \text{ and }
    [1_{W_{\mu}}: \sigma]\neq 0}.
  \]
  Let
  \[
    \LC_{\mu} :=
    \set{\sigma \in \Irr(W_{[\mu]}) | a(\sigma) = a_{\mu}
      \text{ and } [1_{W_{\mu}}: \sigma]\neq   0
    }.
  \]
  Then
 %  \begin{itemize}
 %   \item $\Wlamck$ is a Levi subgroup of $\WLamck$, and
 % \item
  $\LC_{\mu}$ is a left cell of $W_{[\mu]}$ given by
  \begin{equation}\label{eq:LC.mu}
  \LC_{\mu}=(J_{W_{\mu}}^{W_{[\mu]}} \sgn )\otimes \sgn
  \end{equation}
  which contains a unique special representation
  \[
    \sigma_{\mu}=(j_{W_{\mu}}^{W_{[\mu]}} \sgn )\otimes \sgn.
  \]
  Moreover, $\LC_{\mu}$ is multiplicity free, which is equivalent to
  \[
  [1_{W_{\mu}}:\sigma]=1 \quad \text{for each } \sigma\in \LC_{\mu}.
  \]

  Let
  \begin{equation}\label{eq:O.mu}
    \cO_{\mu} = \Spr(j_{W_{[\mu]}}^{W}\sigma_{\mu}).
  \end{equation}
Then
  \[
    \LC_{\mu} = \Set{\sigma\in \sfC_{\bcO_{\mu}}| [1_{W_{\mu}}:\sigma]\neq 0}.
  \]
  \qed
  % \e
\end{lem}

\trivial{
  This is essentially contained in \cite{BVUni}.

  We adapt the notation in \cite{BVUni}: two special representations
  $\sigma \LRleq \sigma'$ if and only if $\cO_{\sigma}\supseteq \cO_{\sigma'}$
  where $\cO_{\sigma}:=\Spr(\sigma)$. The generic degree of $\sigma$ is denoted by $a(\sigma)$.
  Note that the ordering of double cells/special representation is the same as the closure relation on special nilpotent orbits, see \cite{BVUni}*{Prop
    3.23}.

  Note that induction maps left cone representation to a left cone
  representation \cite{BVUni}*{Prop~4.14~(a)}. Therefore
  $\Ind_{W_{\mu}}^{W_{[\mu]}}\sgn$ is a left cone representation.
  $J_{W_{\mu}}^{W_{[\mu]}}\sgn$ is a left cell (since $J$-induction preserves
  left cell \cite{BVUni}*{Prop~4.14~(b)}), it consists of the constituents in the induced
  representation with the minimal generic degree (by the definition of
  $J$-induction), it is also the set of constituents in
  $\Ind_{W_{\mu}}^{W_{[\mu]}}\sgn$ sit in the same $\approxLR$ equivalence class (a unique double cell $\cD$).


  Recall that tensoring with sign (or rather twisting $w_{0}$) is an order
  reversing bijection of left cells in $\WLamck$ and induces a $LR$-order
  reversing bijection on $\Irr(\Wlamck)$, see \cite{BV2}*{Prop.~2.25}. Therefore
  $\Ind_{\Wlamck}^{\WLamck} 1 = \left(\Ind_{\Wlamck}^{\WLamck} \sgn\right)\otimes \sgn$
  has a set of constituents which is maximal under the $LR$-order, in particular
  the generic degree takes maximal value on these representations.

  Hence we get the conclusion.
}

For each  $\mu\in \fhh^{*}$,
let $\cI_{\mu}$ be the maximal primitive ideal having infinitesimal character
$\mu$.
Let
  $\Pi_{W\cdot \mu}$ be the set of all irreducible highest weight modules whose
  annihilator ideal are $\cI_{\mu}$.
  Then
  \[
    \Pi_{W\cdot \mu} = \Pi_{W\cdot \mu, \cO_{\mu}}
  \]
  where $\cO_{\mu}$ is given by \eqref{eq:O.mu}.


Combine the above theorem with \Cref{lem:lcell.BV}, we have the following
counting theorem.
\begin{thm}
  %Suppose $\mu = \lambda_{\ckcO}$.
  % Let
  % \[
  % \LC_{\mu} = \left(J_{W_{\mu}}^{W_{[\mu]}} \sgn\right)\otimes \sgn
  % \]
  % be the left cell attached to $\mu$ and
  % \[
  %   \sigma_{\mu} = \left(j_{W_{\mu}}^{W_{[\mu]}} \sgn\right)\otimes \sgn.
  % \]
  % Then $\sigma_{\mu}$ is the unique special representation containing in
  % $\LC_{\mu}$
  % and
  Retain the notation in \Cref{lem:LC.mu}.   \[
    \abs{\Pi_{W\cdot \mu}} = \abs{W/W_{[\mu]}}\cdot
    \dim \LC_{\mu}.
  \]
  Moreover,
  \[
  \Pi_{W\cdot\mu} = \Set{L(r_{i}w)|w\in \CLR_{\sigma_{\mu}}\cap \sfD_{\sfS_{\mu}} \text{ and } i=1,2,\cdots,k}
  \]
  \qed
\end{thm}

% A double cell in $\Irr(W_{[\lambda]})$ is an equivalent class under
% the relation $\approxLR$


% \[
%   \begin{array}{ccc}
%     \bC[W_{[\lambda]}]& \longrightarrow  &\Coh_{[\lambda]}(\fgg,\fhh,\fnn,[\lambda]).\\
%     w & \mapsto M_{w}
%   \end{array}
% \]


% we define the category $\cO'_{S}$ to be the category of $\fgg$-modules
% such that $M\in \cO'_{S}$ if and only if
% \begin{itemize}
%   \item the $\fbb$-action on $M$ is locally finite;
%   \item $M$ is finitely generated $\cU(\fgg)$-module,
%   \item $M = \sum_{\mu \in S} M_{\mu}$
%         where $M_{\mu}$ is the $\mu$-isotypic component of $M$.
% \end{itemize}

% Therefore, the last condition implies that
% \[
%   \begin{split}
%     \sV^{R}(w) &:=
%     \sspan\set{L(w'\lambda) | w'\leqR w}\\
%     & =\sspan\set{L(w' \lambda) | w^{-1}\leqL w'^{-1}}
%   \end{split}
% \]



% Let $\sfN_{d}$ be the union of all nilpotent orbits in $\fgg$ of dimension
% less than equal to $d$.
%



\subsection{A variation}
In this section, let
$(\fgg, H,\fnn)$ be a triple that
\begin{itemize}
  \item $\fgg$ is a complex reductive Lie algebra,
  \item $H$ is a Lie group such that
        $\fhh_{0}:= \Lie(H)$ is a real form of
       a Cartan subalgebra  $\fhh$ of $\fgg$,
  \item $\fnn$ is a maximal nilpotent subalgebra
        stable under the $\fhh$-action.
\end{itemize}
Let $\cO'(\fgg, H,\fnn)$ be the category of $(\fgg, H)$-module such that
$M\in \cO'(\fgg,H,\fnn)$
if and only if
\begin{itemize}
  \item $M$ is finitely generated as $\cU(\fgg)$-module,
  \item $\fnn$ acts on $M$ locally nilpotently, and
  \item $M$ decomposes in to a direct sum of finite dimension $H$-modules.
\end{itemize}
Let $\Grt(\fgg,H,\fnn)$ be the Grothendieck group of $\cO(\fgg,H,\fnn)$.

We write $H_{0}$ for the connected component of $H$ which is abelian.
Since $H_{0} = \exp(\fhh_{0})$ is central in $H$, for each $\phi\in \Irr(H)$
$\phi|_{H_{0}}$ is a multiple of character.
Hence taking the derivative yields a
well defined map
\[
\rdd \colon \Irr(H)\longrightarrow \fhh^{*}
\]
sending $\phi$ to $\dphi$.


Since $\rdd$ restricted on the lattice
\[
\tQ:=\Set{\phi\in \Irr(H)|\phi \text{
    occurs in } S(\fgg)}
\]
 is a bijection onto the root lattice $Q$ \cite{Vg}*{0.4.6}, we identify the root lattice $Q$ with the $\tQ$ in $\Irr(H)$.


Now assume $\phi\in \Irr(H)$ and let
\[
  [\phi] := \Set{\phi+\alpha| \alpha\in \tQ}
\]
and
$\cO'(\fgg,H,\fnn;[\phi])$ be the subcategory of $\cO'(\fgg,H,\fnn)$
consists of modules whose $H$ irreducible components are contained in $[\phi]$.
Define $\Coh_{[\lambda]}(\fgg,H,\fnn;[\phi])$ to be the space of coherent
families taking value in $\cO'(\fgg,H,\fnn;[\phi])$ and
$\Coh_{[\lambda]}(\fgg,H,\fnn;[\phi])$ to be its subspace whose complex
associated variety is contained in $\sfS$ for a $\Ad(\fgg)$-invariant closed subset $\sfS$ in the nilpotent cone of
$\fgg$.


We have the following lemma.
\begin{lem}
  Let $\phi\in \Irr(H)$ and fix a $\lambda\in \fhh^{*}$ such that
  $[\rdd \phi + \rho ]\cap W\cdot \lambda\neq \emptyset$.
  Then the forgetful functor
  \[
    \cF\colon \cO'(\fgg,H,\fnn)\longrightarrow \cO'(\fgg,\fhh,\fnn)
  \]
  induces a $W_{[\lambda]}$-module isomorphism
  \[
    \cF\colon \Coh_{[\lambda],\sfS}(\fgg,H,\fnn;[\phi])\longrightarrow
    \Coh_{[\lambda],\sfS}(\fgg,\fhh,\fnn;[\rdd\phi])
  \]
\end{lem}
\begin{proof}
  When $\sfS$ is the whole nilpotent cone, the isomorphism is
  given by identifying both sides with $\bC[W_{[\lambda]}]$ via Verma modules
  such that
  \[
  \wtM_{1}(\rdd \phi+\rho):=\cU(\fgg)\otimes_{\cU(\fhh\oplus\fnn),H}\phi
  \mapsto (\dim\phi)\cdot M_{1}(\rdd\phi+\rho):= \cU(\fgg)\otimes_{\cU(\fhh\oplus\fnn)}\rdd\phi.
  \]
  For $\phi'\in \Irr(H)$, $M(\phi'+\rho)$ has a unique irreducible quotient
  $L(\phi')$ by the same argument of the same argument for the highest weight module.   Now we have, $L(\phi'+\rho) = \dim(\phi')\cdot L(\rdd\phi'+\rho)$.
  \trivial{
    These claims should be also much more clear from the D-module point of view.
    The middle extension functor only see the $\cD_{\lambda}$-module structure
    and keeps the $T$-module structure automatically. Here $T$ is the maximal
    compact subgroup of $H$.
  }
  Since the associated variety only depends on the $\cU(\fgg)$-module structure,
  the rest part of the lemma follows.   {\color{red} Check!!}
\end{proof}

\trivial{
  Note that $H_{0}$ is abelian and $H = H_{0}$.
  By the assumption of Harish-Chandra class,
  $H = H_{0}\times H/H_{0}$. Here $H/H_{0}$ is a finite group maybe non-abelian.
}

The above lemma have the following immediate consequence.
\begin{cor}
  Retain the notation in \Cref{lem:C.S}. Then, for $\sigma\in \Irr(W_{[\lambda]})$
  \[
[\sigma:\Coh_{[\lambda],\sfS}(\fgg,H,\fnn)] \neq 0 \Leftrightarrow
  \sigma\in \sfC_{\sfS}.
  \]   \qed
\end{cor}



\section{Harish-Chandra cells}
In this section, let $G$ be a real reductive group in the Harish-Chandra class.
Here $G$ could be a nonlinear group.
We retain the notation in \Cref{eg:Coh.HC}.
We recall the argument before \cite{Mc}*{Theorem~1}.

We fix a Cartan involution $\theta$ on $G$ and let $K = G^{\theta}$
be the maximal compact subgroup of $G$.

% \begin{thm}[Barbasch-Vogan]\label{thm:count}
%   For a complex nilpotent orbit $\cO$,
%   define
%   \[
%     S_{\cO} = \left\{\sigma \in \widehat{W_{[\mu]}}|
%       \Spr(j_{W_{[\mu]}}^{W} \sigma_{s}) = \cO
%     \right\}
%   \]
%   where $\sigma_{s}$ is the special representation in the double cell containing
%   $\sigma$.

%   Let $\Pi_{\cO,\mu}(G)$ denote the set of irreducible admissible $G$-module with
%   complex associated variety $\overline{\cO}$.
%   % Let $W_{\mu}$ be the stabilizer of $\mu$ and $W_{[\mu]}$ be the stabilizer of
%   % the lattice
%   Then
%   \[
%     \# \Pi_{\cO,\mu}(G) =
%     \sum_{\sigma\in S_{\cO}} [\sigma: \mathrm{Coh}_{[\mu]}(G)] \cdot
%     [1_{W_{\mu}}, \sigma|_{W_{\mu}}].
%   \]
%   % Here $[\sgima : \ ]$ denote the multiplicity of $\sigma$
%   % and  $W_{\mu}$ is the stabilizer of $\mu$.
% \end{thm}

\subsection{An embedding of coherent families of Harish-Chandra modules into
  that of category $\cO$}
In this section, we recall a result in \cite{Cas}. Let $\fbb = \fhh\oplus \fnn$
be a Borel subalgebra in $\fgg$ with the nilradical $\fnn$ and $\fhh$ a Cartan
subalgebra in $\fbb$. For a subalgebra $\fuu$ of $\fnn$ and $q\in \bN$, Casian
defined the localization functors $\gamma_{\fuu}^{q}$ on the category of
$\fuu$-module. By \cite{Cas}*{Proposition~4.8}, $\gamma_{\fuu}^{q}$ can be
defined as the right derived functor of the functor $\gamma_{\fuu}^{0}$ which
sends a $\fuu$-module $M$ to
\[
  \gamma_{\fuu}^{0}(M):= \Set{v\in M| \fuu^{k} v = 0 \text{ for some positive
      integer $k$}}.
\]
In particular, the $\fuu$-action on $\gamma_{\fuu}^{q}$ is locally nilpotent.

Suppose $M$ is a $\fgg$-module, we have (see \cite{Cas}*{Proposition~4.14})
\begin{equation}\label{eq:anngamma}
  \Ann M \subseteq \Ann (\gamma_{\fuu}^{q}(M)).
\end{equation}
Moreover, $\gamma_{\fuu}^{q}$ commutes with tensoring finite diemsional
representations of $\fgg$, i.e. for a finite dimensional $\fgg$-module $F$ there
is a natural isomorphism (c.f. \cite{Cas}*{Proposition~4.11})
\begin{equation}\label{eq:Fgamma}
  F\otimes \gamma_{\fuu}^{q}(M)\cong
  \gamma_{\fuu}^{q}(F\otimes M).
\end{equation}
If the $\fgg$-module have finite dimensional $\fnn$-cohomology, then
$\gamma_{\fnn}^{q}(M)$ is in the category $\cO'$. See
\cite{Cas}*{Proposition~4.9}.



Suppose $M$ is a $(\fgg,K)$-module. From the definition of $\gamma_{\fuu}^{q}$,
we can see that $\gamma_{\fuu}^{q}(M)$ is naturally a $(\fgg, K_{L})$-module
where $K_{L}$ denote the normalizer of $\fuu$ in $K$. Let $\fll$ be the
normailzer of $\fuu$ in $\fgg$. Then there is the a spectrum sequence of
$(\fll,K_{L})$-module convergent to $H^{p+q}(\fuu,M)$ (see
\cite{Cas}*{Proposition~4.4}):
\begin{equation}\label{eq:ugamma}
  H^{q}(\fuu,\gamma_{\fuu}^{p}(M)) \Rightarrow H^{p+q}(\fuu,M).
\end{equation}

For our application, we always assume $M$ is a $(\fgg,K)$-module, and take
$\fuu=\fnn$.

Let $H$ be a $\theta$-stable Cartan subgroup of $G$, $T=H^{\theta}$ be the
maximal compact subgroup of $H$ and $\fhh:=\Lie(H)_{\bC}$ is the corresponding
Cartan subalgebra in $\fgg$. We can view a finite dimensional $H$-module as a
$(\fhh,T)$-module and vice versa.

The localization functor $\gamma_{\fnn}^{q}$ is compatible with coherent
continuation.
\begin{lem}\label{lem:coh.gamma}
  Assume that each irreducible $(\fgg,K)$-module have finite dimensional
  $\fnn$-cohomology. For each $q\in \bN$, we have the following map between
  $\Wlam$-modules:
  \[
    \begin{array}{ccc}
      \gamma_{\fnn}^{q}\colon \Coh_{[\lambda]}(\fgg,K)
      &\longrightarrow
      & \Coh_{[\lambda]}(\fgg,H,\fnn) \\
      \Theta & \mapsto & \gamma_{\fnn}^{q}\circ \Theta.
    \end{array}
  \]
\end{lem}
\begin{proof}
  This is a consequence of \eqref{eq:anngamma} and \eqref{eq:Fgamma}. \trivial[]{
    \[
      \begin{split}
        F\otimes \gamma_{\fnn}^q\Theta(\mu) & =
        \gamma_{\fnn}^{q}(F\otimes \Theta(\mu))\\
        &= \gamma_{\fnn}^{q}(\sum_{\beta\in \WT{F}} \Theta(\mu+\beta))\\
        &= \sum_{\beta\in \WT{F}}\gamma_{\fnn}^{q}( \Theta(\mu+\beta))\\
      \end{split}
    \]
  }
\end{proof}

We fix a positive system of real roots $\Delta^{+}_{\bR}$ and a Borel subalgebra
$\fbb=\fhh\oplus \fnn$ such that $\Delta^{+}_{\bR}\subset \WT{\fnn}$. We let
$e^{\alpha}\in \Irr(H)$ be the $H$-character on the $\alpha$-root space.

For a finite length $(\fgg,K)$-module $M$, we view its global character
$\Theta_{G}(M)$ as a analytic function defined on the set $\Greg$ of regular
semisimple elements on $G$.


The following theorem is crucial.
\begin{thm}[\cite{Cas}*{Theorem~3.1}]\label{thm:gamma.HC}
  % Let $\bfnn$ be the maximal nilptent Lie subalgebra of $\fgg$ with spanned by
  % roots in $\Delta^{+}$.
  Let $M$ be a finite length $(\fgg,K)$-module. The following statements hold.
  \begin{enumT}
    \item The Lie algebra cohomology $H^{q}(\fnn,M)$ is finite dimensional for
    each $q\in \bN$. In particular, $\gamma_{\fnn}^{q}(M)$ is in the category
    $\cO'(\fgg,H,\fnn)$ for eah $q\in \bN$.
    \item
    % $\Ann M \subseteq \Ann (\gamma_{\fnn}^{q}M)$.
    % \item
    Let
    \[
      \Hnreg:= \Set{h\in H|
        \begin{array}{l}
          \text{$h$ is regular semisimple}\\
          \abs{e^{\alpha}(h)}<1 \text{ for each real root }
          \alpha\in \Delta^{+}_{\bR}
        \end{array}
      }
    \]
    Then
    \begin{equation}\label{eq:char}
      \begin{split}
        \Theta_{G}(M)|_{\Hnreg} &= \frac{\sum_{q\in \bN} (-1)^{q}\Theta_{H} \left(H^{q}(\fnn,M)\right)}
        {\prod_{\alpha\in \WT{\fnn}}(1- e^{\alpha})}\\
        &= \frac{\sum_{p,q\in \bN} (-1)^{p+q}\Theta_{H} \left(H^{q}(\fnn,\gamma_{\fnn}^{p}(M))\right)}
        {\prod_{\alpha\in \WT{\fnn}}(1- e^{\alpha})}\\
      \end{split}
    \end{equation}
  \end{enumT}
  \qed
\end{thm}
\begin{proof}
  The theorem is a recollection of Casian's results in loc. cit. The last
  equality in \eqref{eq:char} follows from \eqref{eq:ugamma}. The last
  expression in \eqref{eq:char} is also a finite sum, since only finite may
  terms of $\gamma_{\fnn}^{p}(M)$ are non-zero and they are in the category
  $\cO'$.
\end{proof}


Retain the notation in \Cref{thm:gamma.HC}, we write
\[
  \gamma_{\fnn}:= \sum_{q\in \bN} (-1)^{q} \gamma^{q}_{\fnn}\colon \Grt(\fgg,K) \longrightarrow \Grt(\fgg,H,\fnn)
\]

% \[
%   \gamma_{\fnn}:= \sum_{q\in \bN} (-1)^{q} \gamma^{q}_{\fnn}\colon \Coh_{[\lambda]}(\fgg,K) \longrightarrow \Coh_{[\lambda]}(\fgg,H,\fnn)
% \]

\begin{cor}[c.f. {\cite{Mc}}]\label{cor:HC.embed}
  % [\cite{Cas}*{Theorem~3.1}]
  Let $H_{1}, H_{2}, \cdots, H_{s}$ form a set of representatives of the
  conjugacy class of $\theta$-stable Cartan subgroup of $G$. Fix maximal
  nilpotent Lie subalgebra $\fnn_{i}$ for each $H_{i}$ as in
  \Cref{thm:gamma.HC}. Then
  \[
    \begin{array}{cccc}
      \gamma:=\oplus_{i} \gamma_{\fnn_{i}}: &\Grt(\fgg,K)
      &\longrightarrow & \bigoplus_{i=1}^{s} \Grt(\fgg,H_{i},\fnn_{i})\\
    \end{array}
  \]
  is an embedding of abelian groups.
\end{cor}
\begin{proof}
  Let $\Hireg$ be the set of regular semisimple elements in $H_{i}$. By the
  results of Harish-Chandra, taking the character of the elements induces an
  embedding of $\Grt(\fgg,K)$ into the space of real analytic functions on
  $\bigsqcup_{i} \Hireg$. Since $\Hnireg$ is open in $\Hireg$ and meets all the
  connected components of $\Hireg$, any real analytic function on
  $\bigsqcup_{i} \Hireg$ is determined by its restriction on
  $\bigsqcup_{i} \Hnireg$. Now \eqref{eq:char} implies that the global character
  of any $M\in \Grt(\fgg,K)$ can be computed from the image $\gamma(M)$.
  % $\Theta M$ of an element $M\in \Grt_{\chi}(\fgg,K)$ is completely determined
  % by the formal character $\mathrm{ch}(\gamma(M))$.
\end{proof}

\begin{cor}\label{cor:coh.HC}
  Retain the notation in \Cref{cor:HC.embed}.
  Let $\sfS$ be a $\Gcad$-invariant closed subset in the nilpotent cone of
  $\fgg$ and $\lambda\in \fhh^{*}$.
  Then $\gamma$ induces an embedding of $\Wlam$-module.
  \[
    \begin{array}{cccc}
      \gamma:=\oplus_{i} \gamma_{\fnn_{i}}: &\Coh_{[\lambda],\sfS}(\fgg,K)
      &\longrightarrow & \bigoplus_{i=1}^{s} \Coh_{[\lambda],\sfS}(\fgg,H_{i},\fnn_{i})\\
      &\Theta& \mapsto &\gamma\circ \Theta
    \end{array}
  \]
  In particular, $[\sigma:\Coh_{[\lambda],\sfS}(\fgg,K)]\neq 0$ only if
  $\sigma\in \sfC_{\sfC}$.
\end{cor}
\begin{proof}
  The maps is well-defined by \eqref{eq:anngamma}. It is an embedding by
  \Cref{cor:HC.embed}.
\end{proof}

\medskip

Now we get the following theorem on the upper bound of small representations.
\begin{thm}
  Let $\mu\in \fhh^{*}$. Let $\Pi_{W\cdot \mu}(G)$ be the set of irreducible
  $(\fgg,K)$-modules annihilated by the maximal primitive ideal $\cI_{\mu}$ of
  infinitesimal character $\mu$. Let $\LC_{\mu}$ be the left cell of
  $\Irr(W_{[\mu]})$ defined in \eqref{eq:LC.mu} and $\cO_{\mu}$ be the nilpotent
  orbit defined by \eqref{eq:O.mu}. Then
  \[
    \begin{split}
      \abs{\Pi_{W\cdot \mu}(G)} &= \sum_{\sigma\in \LC_{\mu}}
      [\sigma:\Coh_{[\mu],\bcO_{\mu}}(G)]\\
      &\leq \sum_{\sigma\in \LC_{\mu}}
      [\sigma:\Coh_{[\mu]}(G)]\\
    \end{split}
  \]
\end{thm}
\begin{proof}
  The equality follows from \Cref{lem:coh.count}, \Cref{cor:coh.HC} and
  \Cref{lem:LC.mu}. The inequality is also clear since
  $\Coh_{[\mu],\bcO_{\mu}}(G)$ is a submodule of $\Coh_{[\mu]}(G)$.
\end{proof}

% Then the localization functor induces a homomorphism
% \[
%   \begin{array}{cccc}
%       \gamma_{\fnn}: &\Grt_{\chi,\cZ}(\fgg,K) &\longrightarrow & \Grt_{\chi,\cZ}(\cO_{\whH})\\
%       & M &\mapsto & \sum_{q}  (-1)^{q} \gamma^{q}_{\fnn} M
%     \end{array}
%   \]
%   Fix an infinitesimal character $\chi$, and a close $G$-invariant set
%   $\cZ\in \cN_{\fgg}$.


% \def\VHC{\sV^{\mathrm{HC}}}
% \begin{cor}
%   Fix an irreducible $(\fgg,K)$-module $\pi$ with infinitesimal character
%   $\chi_{\lambda}$. Let $\VHC(\pi)$ be the Harish-Chandra cell representation
%   containing $\pi$ and $\cD$ be the double cell in $\widehat{W_{[\lambda]}}$
%   containing the special representation $\sigma(\pi)$attached to $\Ann(\pi)$.
%   Then $[\sigma, \VHC(\pi)]\neq 0$ only if $\sigma \in \cD$. Moreover,
%   $\sigma(\pi)$ always occures in $\VHC(\pi)$
% \end{cor}
% \begin{proof}
%   The occurrence of $\sigma(\pi)$ is a result of King.

%   Note that we have an embedding
%   \[
%     \gamma \colon \Grt_{\chi}(\fgg,K)\longrightarrow \bigoplus_{i}\Grt(\fgg,\fbb,\lambda).
%   \]
%   where the left hand sides is identified with a finite copies of
%   $\bC[W_{[\lambda]}]$.

%   Since $\Ann(\pi)\subseteq \Ann (\gamma_{\fnn}^{q}(\pi))$, we conclude that
%   $[\sigma, \VHC(\pi)]\neq 0$ implies that $\sigma(\pi)\leqLR \sigma$.

%   By the Vogan duality, $\cD\otimes \sgn$ is also a Harish-Chandra cell. So we
%   have $\sigma(\pi)\otimes \sgn \leqLR \sigma\otimes \sgn$.

%   Therefore, $\sigma(\pi)\approxLR$
% \end{proof}
\begin{remark}
  Let $\lambda$ be a regular element in $[\mu]$. Assume that for each
  Harish-Chandra cell representation $\sV^{HC}$ at the infinitesimal character
  $\lambda$, the $W_{[\mu]}$-representations occur in $\sV^{HC}$ are all
  contained in a unique double cell. Then one can prove that the inequality in
  the above theorem is sharp (c.f. \cite{BV.W}). However, we do not know such
  kind of claim holds in a general.
\end{remark}


\subsection{Coherent continuation representation of Harish-Chandra modules}

Recall the definition of regular characters \cite{Vg}*{Definition~6.6.1}.

A regular character of $G$ is a tuple a tuple $\gamma := (H,\Gamma,\bargamma)$
such that
\begin{itemize}
  \item $H$ is a $\theta$-stable Cartan subgroup of $G$,
  \item $\Gamma$ is a continuous character of $H$,
  \item if $\alpha$ is imaginary, then $\inn{\bargamma}{\ckalpha}$ is real and
        non-zero,
  \item the differential of $\Gamma$ is
        \[
        \bargamma+ \rho_{i} - 2\rho_{ic}
        \]
        where %$\rho_{i}$ is the half sum of imaginary roots $\$ such that $\inn{\bargamma}{\ckalpha}$
        \[
        \rho_{i} =\half\sum_{\substack{\alpha \text{
        imaginary}\\ \inn{\bargamma}{\ckalpha}>0}} \alpha \quad\text{and}\quad \rho_{ic} =\half\sum_{\substack{\alpha \text{
        compact imaginary}\\ \inn{\bargamma}{\ckalpha}>0}} \alpha.
        \]
\end{itemize}
We say $\gamma$ is non-singular if $\bargamma$ is regular in $\fhh^*$. The group $K$ acts on the set of regular characters of $G$ we let $[\gamma]$ to denote the $K$-conjugacy class of regular character containing $\gamma$.

Let $\cR(G)$ denote the set of regular characters of $G$ and $\cR_{\lambda}(G)$ be the subset of $\cR(G)$ consists of regular characters with infinitesimal character $\lambda$. Let $\cP(G):=\cR(G)/K$ and $\cP_{\lambda}(G):=\cR_{\lambda}(G)/K$ be the corresponding sets of $K$-conjugacy classes.
% of regular characters and $\cP_\lambda(G)$ be the subset with infinitesimal character $\lambda$. By abuse of notation, we identify $\gamma$ with its conjugacy class.

For each regular character $\gamma$, we attach a standard representation
$\pi_\gamma$ with infinitesimal character $\bargamma$ and let $\barpi_\gamma$ be
the maximal completely reducible submodule of $\pi_\gamma$
\cite{Vg}*{6.5.2,6.5.11,6.6.3}. Then $\barpi_\gamma$ and the image of
$\pi_\gamma$ in the Grothendieck group of $(\fgg,K)$-module only depends on the
$K$-conjugation class of $\gamma$. This justifies the consideration of $\cP(G)$.


Now we assume that $\lambda\in \fhh^*$ is a regular element. Then
$\barpi_\gamma$ is the unique irreducible submodule in $\pi_\gamma$.
% Let $\Pi_{\lambda}(G)$ be the set of irreducible $(\fgg,K)$-modules with
% infinitesimal character $\lambda$, then
The following map is bijective (Langlands classification)
\[
  \begin{array}{ccc}
    \cP_{\lambda}(G) & \longrightarrow & \Irr_{\lambda}(G)\\
    {[\gamma]} & \mapsto & \barpi_{\gamma}
  \end{array}
\]
and $\Set{\pi_\gamma|\gamma\in \cP_{\lambda}(G)}$ forms a basis of
$\Grt_{\lambda}(G)$. As a consequence, we have an isomorphism of vector spaces
\[
  \begin{array}{ccc}
    \bC[\cP_{\lambda}(G)] & \longrightarrow & \Coh_{[\lambda]}(G)\\
    {[\gamma]} & \mapsto & \Theta_{\gamma}
  \end{array}
\]
where $\Theta_{\gamma}$ is the unique coherent family such that
$\Theta_{\gamma}(\lambda) = \pi_{\gamma}$.
In the following, we identify the two sides implicitly.

\medskip

\def\Wa{W^{a}}
\def\WiR{W_{i\bR}}
\def\fhhiR{\fhh_{i\bR}}
\def\WR{W_{\bR}}
\def\WC{W_{\bC}}
\def\lama{\lambda^{a}}
\def\Wlama{W^a_{[\lambda]}}
For any
regular element $\lambda\in \fhh^{*}$, let $\lambda^{a}$ be the dominant element
in the abstract Cartan $\fhh^{a}$ corresponding to $\lambda$ and
\[
  i_{\lambda}\colon W(\fgg,\fhh)\longrightarrow \Wa
\]
be the identification of $W(\fgg,\fhh)$ with the abstract Weyl group $\Wa$.

From now on we fix a regular element $\lambda \in \fhh^{*}$.

Let $\Wlama:= i_{\lambda}(W_{[\lambda]})$ be the abstract integral Weyl group.
We identify $W_{[\lambda]}$ with $\Wlama$ implicitly.

\newcommand{\cross}{\times} \newcommand{\crossa}{\times^a} In the following, we
assume $\gamma=(H,\Gamma,\bargamma)\in \cR_{\lambda}(G)$. On one hand, a element
$w$ in the real Weyl group $W(G,H)$ acts on $\gamma$ by conjugation and gives a
new regular character denoted by $w\cdot \gamma$. \trivial{ Note that
  $W(G,H) = W(K_{\bC},H_{\bC}):= N_{K_{\bC}}(H_{\bC})/K_{\bC}\cap H_{\bC}$ is a
  subgroup of $W(\fgg,\fhh)$. }

On the other hand, $W_{[\bargamma]}$ acts on $\gamma$ by cross action
$(w,\gamma)\mapsto w\times gamma$ \cite{Vg}*{8.3.1}. On define the abstract Weyl
group $\Wlama$ acts on $\cR_{\lambda}(G)$ by
\[
  w \crossa \gamma := \dot{w}^{-1}\times \gamma \text{ with
  } \dot{w}\in W_{[\lambda]}, w = i_{\bargamma}(\dot{w}),
\]
see \cite{V4}*{Definition~4.2}.

The cross action of $\Wlama$ descents to an action on $\cP_{\lambda}(G)$.
{\color{red} reference?} Let $\Wa_{[\gamma]}\subset \Wlama$ be the stabilizer of
$[\gamma]$ under cross action. Then
\[
  \Wa_{[\gamma]}:= i_{\bargamma}(W_{[\gamma]})
  \quad \text{ with }\quad
  W_{[\gamma]}:=\Set{w\in W(G,H)| w\times \gamma = w\cdot \gamma}.
\]
The $\theta$-action on $\fhh$ induces an action on $W(\fgg,\fhh)$. Then we have
the following tower of groups
\[
  W_{[\gamma]} < W(G,H) < W(\fgg,\fhh)^{\theta}
\]
and
\[
  W(\fgg,\fhh)^{\theta}=(\WC)^{\theta}\ltimes (\WiR\times \WR)
\]
where $\WC$, $\WiR$ and $\WR$ are Weyl groups of complex, compact and real
roots respectively (see \cite{V4}*{Proposition~3.12} and
\cite{AC}*{(12.1)-(12.5)}).


Define a quadratic character on $W(\fgg,\fhh)^{\theta}$ by
\[
  \begin{array}{rccc}
    \sgn_{\fhh}\colon  W(\fgg,\fhh)^{\theta} =
    & (\WC)^{\theta}\ltimes (\WiR\times \WR)
    & \longrightarrow & \Set{\pm 1}\\
    & (w_{\bC},w_{i\bR},w_{\bR}) & \mapsto & \sgn_{\WiR}(w_{i\bR})
  \end{array}
\]
where $\sgn_{\WiR}$ denote the sign character of the imaginary Weyl group. Let
$\fhhiR^{*}$ be the span of imaginary roots, then
\[
  \sgn_{\fhh}(w) = \det(w|_{\fhhiR^{*}}).
\]
Let $\sgn_{[\gamma]}:=\sgn_{\fhh}|_{W_{[\gamma]}}$ be the restriction of
$\sgn_{\fhh}$ on the cross stabilizer $W_{[\gamma]}$.

The following result is a unpublished result of Barbasch-Vogan based on results
in \cite{Vg}*{Chapter 8}, a same argument works equally well with non-linear groups
in the Harish-Chandra class.

\begin{thm}[{c.f. \cite{BV.W}*{Proposition~2.4}}]
  Suppose $\Pi_{\lambda}(G) = \bigsqcup_{i=1}^{k} \Wlama\crossa [\gamma_{i}]$
  where $\gamma_{i}=(H_{i},\Gamma_{i},\bargamma_{i})$ are representatives of the
  $\Wlama$-orbits of $\Pi_{\lambda}(G)$. Then
  \[
    \Coh_{[\lambda]}(G) \cong \bigoplus_{i=1}^{k}
    \Ind_{W_{[\gamma_{i}]}}^{\Wlama} \sgn_{[\gamma_{i}]}.
  \]
\end{thm}
\begin{proof}[Sketch of the proof]
  To avoid the confusion, we use $t(w)$ to denote the coherent continuation
  action.
  The action of simple roots in $R^{+}_{\lambda}$ on the basis $\Theta_{\gamma}$
  is given calculated in \cite{Vg}*{Chapter 8} and summarized in
  \cite{V4}*{Definition~14.4}. The calculation reduced to the representation
  theory of $\SL(2,\bR)$.
  For non-linear groups see \cite{RT3}*{Definition~9.4} and note that
  the formula is exactly the same as that of linear groups for integral simple
  roots.
  \trivial[]{
    Let $\cT_{\lambda_{1}}^{\lambda_{2}}$ be the translation functor from
    infinitesimal character $\lambda_{1}$ to $\lambda_{2}$.
    Fix a
    By abstract non-sense, one have \cite{Vg}*{Prop~7.2.22}
    \[
      \Theta_{\pi}(\lambda) + s_{\alpha}\cdot \Theta_{\pi}(\lambda)=
      \cT_{\lambda_{0}}^{\lambda}\cT_{\lambda}^{\lambda_{0}}(\pi)=:
      \phi_{\alpha} \psi_{\alpha} (\pi)
    \]
    where $\lambda_{0}$ is an element such that $\inn{\lambda_{0}}{\alpha}=0$
    and $\inn{\lambda_{0}}{\beta}>0$ for all
    $\alpha\neq \beta \in R^{+}_{[\lambda]}$.
    By lifting to the covering group, we can assume that
    $\lambda-\lambda_{0}$ is a weight of a finite dimensional representation
    of $G$.

    Now the computation reduces to compute the RHS
    $=\phi_{\alpha} \psi_{\alpha} (\pi)$ of the above equality. (Sometimes, we
    need \cite{Vg}*{Prop~8.3.18} for the explicit computation of cross action.)
    This is computed case by case (let $t(w)$ denote the coherent continuation
    action):
    \begin{itemize}
      \item For compact imaginary roots, by Hecht-Schmid's ``A proof of
      Blattner's conjecture'', where RHS $=0$.
      \[
        t(s_{\alpha}) \gamma = - \gamma = - s_{\alpha}\cross \gamma.
      \]
      \item For non-compact imaginary roots, reduce to $\SL(2,\bR)$
      \cite{Vg}*{8.4.5,8.4.6}.
      \[
        t(s_{\alpha}) \gamma = - s_{\alpha}\cross \gamma + R
      \]
      where $R = c^{\alpha}(\gamma)$ or
      $\gamma^{\alpha}_{+}+\gamma^{\alpha}_{-}$ is a combination of
      regular characters on the Cartan subgroup $H^{\alpha}$ (which has
      higher $\bR$-rank, in fact
      $\rank_{\bR}H^{\alpha} = \rank_{\bR} H +1$).
      \item For real roots we can use \cite{Vg}*{8.3.19} which is a consquence
      of the (cohomological induction) construction of the standard
      module. The result says if $w$ acts trivial on $\ftt=\fhh^{\theta}$,
      then $t(w^{-1})\gamma = w\cross \gamma$. When $\alpha$ is a real
      root, then clearly $s_{\alpha}$ acts on $\ftt$ trivially (since
      $-\alpha(x) = \theta(\alpha)(x) = \alpha(\theta(x)) = \alpha(x) = 0
      \ \forall x\in \ftt$)
      and we have
      \[
        t(s_{\alpha}) \gamma = s_{\alpha}\cross \gamma.
      \]
      \item For complex root, use \cite{Vg}*{8.2.7} (whose proof relies on a
      long exact sequence in \cite{Vg}*{7.4.3(a)} which is also formal) we get:
      $\gamma + s_{\alpha}\cross \gamma = \phi_{\alpha} \psi_{\alpha} (\gamma)$
      (here $\gamma-n\alpha = s_{\alpha}\cross \gamma$ by the definition of
      cross action).
      In particular, we have
      \[
        t(s_{\alpha}) \gamma = s_{\alpha}\cross \gamma.
      \]
    \end{itemize}
 }

 Let
 \[
   \cP_{\lambda,r}(G):= \Set{[(H,\Gamma,\bargamma)]|
     \text{real rank of $H$ is $r$}}
 \]
 and $\cP_{\lambda,\geq r} = \bigsqcup_{l\geq r}\cP_{\lambda,l}$. Define
 \[
   \Coh_{[\lambda],\geq r}(G) :=
   \Span\Set{\Theta_{[\gamma]}|[\gamma]\in \cP_{\lambda,\geq r}}
 \]
 From the explicit formula of the coherent continuation actions on the standard
 modules, we have that, for $[\gamma]\in \cP_{\lambda,\geq r}(G)$,
 \[
   t(s_{\alpha})\; [\gamma]
   \equiv
   \begin{cases}
     - s_{\alpha}\crossa [\gamma] & \text{if $\alpha$ is imaginary,}\\
     \phantom{-} s_{\alpha}\crossa [\gamma] & \text{otherwise.}
   \end{cases}
 \]
 Now it is elementary to deduce that, as $\Wlama$-module,
 \[
   \frac{\Coh_{[\lambda],\geq r}(G)}{\Coh_{[\lambda],\geq r+1}(G)}
   \cong \bigoplus_{\Wlama \crossa [\gamma]}
   \Ind_{W_{[\gamma]}}^{\Wlama} \sgn_{[\gamma]}
 \]
 where the summation runs over the cross action orbits
 in $\cP_{\lambda,r}(G)$.
 Since $\Wlama$ is a finite group, we get the theorem by the complete
 reduciblity of $\Coh_{[\lambda]}(G)$.
\end{proof}

We remark that, when $\lambda$ is integral, the set $\cP_{\lambda}(G)$ can be enumerated using \cite{AC} (the
algorithm is implemented in atlas) for linear groups.
Under atlas' parameters, the cross action is also easy to calculate.
For the metaplectic group, the problem was solved by Renard-Trapa \cite{RT1,RT2}.



\section{Counting in type A}
The results in this section are well known to the experts.


Let $\YD$ be the set of Young diagrams viewed as a finite multiset of positive
integers. The set of nilpotent orbits in $\GL_n(\bC)$ is identified with Young
diagram of $n$ boxes.


Let $\ckGc = \GL_n(\bC)$. 
Fix
an orbit $\ckcO\in \Nil(\ckG)$, let $\ckcO_e$ (resp. $\ckcO_o$) be the partition
consists of all even (resp. odd) rows in $\ckcO$.

Let $S_n$ denote the Weyl group of $\GL_n(\bC)$.
Let $W_n := S_n \ltimes \set{\pm 1}^n$ denote the Weyl group of type $B_n$ or $C_n$.  
Let $\sgn$ denote the sign representation of the Weyl group. 
The group $W_n$ is naturally embedded in $S_{2n}$.
For $W_n$, let $\epsilon$ denote the unique non-trivial character which is trivial on $S_n$. 
Note that $\epsilon$ is also the restriction of the $\sgn$ of $S_{2n}$ on $W_n$. 


\subsection{Special unipotent representations of $G = \GL_n(\bC)$}
By \cite{BVUni}, the set of unipotent representations
of $G = \GL_n(\bC)$ one-one corresponds to nilpotent orbits in $\Nil(\ckGc)$. 
Suppose $\ckcO$ has rows 
\[
\bfrr_1(\ckcO)\geq \bfrr_2(\ckcO)\geq \cdots\geq 
\bfrr_k(\ckcO) >0. 
\]
Then $\cO := \dBV(\ckcO)$ has columns $\bfcc_i(\cO) = \bfrr_i(\ckcO)$ for all 
$i\in \bN^+$. 
The map $\ckcO \mapsto \cO$ is a bijection.


We set $\CP =  \YD$ be the set of Young diagrams. 
For $\uptau \in \CP$ which has $k$ columns,
let $1_{c}$ be the trivial representations of $\GL_c(\bC)$. 
\[
 \pi_\uptau = 1_{\bfcc_1(\uptau)}\times  1_{\bfcc_2(\uptau)}\times \cdots 
 \times 1_{\bfcc_k(\uptau)}.
\]

The Vogan duality gives a duality between Harish-Chandra cells. 
In this case, Harish-Chandra cells is the double cell  
of Lusztig.  
Now we have a duality 
\[
 \pi_\uptau \leftrightarrow \pi_{\uptau^t}. 
\]

Let $\uptau' := \DD(\uptau)$ be the partition obtained by deleting the first column 
of $\uptau$. 
Let $\theta_{a,b}$ (resp. $\Theta_{a,b}$) be the  theta lift (resp. big theta lift) from $\GL_a(\bC)$ to  
$\GL_b(\bC)$. 
Then we have 
\[
  \pi_{\uptau} = \theta_{{\abs{\uptau'}},{\abs{\uptau}}} (\pi_{\uptau}). 
\]

\subsection{Counting unipotent representations of $\GL_n(\bR)$}
Now let $\ckcO\in \Nil(\ckGc)$. 
Recall the decomposition $\ckcO  = \ckcO_e \cup \ckcO_o$.
Let $n_e = \abs{\ckcO_e}, n_o = \abs{\ckcO_o}$ and $\lambda_\ckcO = \half \ckhh$. 

Then 
\[
  W_{\lamck} \cong S_{\abs{\ckcO_e}}\times S_{\abs{\ckcO_o}}.
\]
\[
 W_{\lambda_\ckcO} = \prod_j S_{\bfcc_j(\ckcO_e)}\times \prod_j S_{\bfcc_j(\ckcO_o)} 
\]
By the formula of $a$-function, one can easily see that 
The cell in $W(\lamck)$ consists of the unique representation $J_{W_{\lamck}}^{\Wint{\lamck}} (1)$.
Now the $W$-cell $(J_{W_{\lamck}}^W \sgn)\otimes \sgn$ consists a single
representation
\[
\tau_{\ckcO} = \ckcO_{e}^{t}\boxtimes \ckcO_{o}^{t}.
\]
The representation $j_{W_{\lamck}}^{S_{n}} \tau_{\ckcO}$
corresponds to the orbit $\cO= \ckcO^t $ under the Springer
correspondence.
\trivial{
WLOG, we assume $\ckcO =  \ckcO_o$.

Let $\sigma\in \widehat{S_n}$. We identify $\sigma$ with a Young diagram. 
Let $c_i = \bfcc_i(\sigma)$.
Then $\sigma = J^{S_n}_{W'} \epsilon_{W'}$ where $W' = \prod S_{c_i}$
(see Carter's book). 
This implies Lusztig's a-function takes value
\[
a(\sigma) = \sum_i c_i(c_i-1) /2
\]
Compairing the above with the dimension formula of nilpotent Orbits
\cite{CM}*{Collary~6.1.4}, we get (for the formula, see Bai ZQ-Xie Xun's paper on 
GK dimension of $SU(p,q)$)
\[
\half \dim(\sigma) = \dim(L(\lambda)) = n(n-1)/2 - a(\sigma).
\]
Here $\dim(\sigma)$ is the dimension of nilpotent orbit attached to the Young
diagram of $\sigma$ (it is the Springer correspondence, regular orbit maps to
trivial representation, note that $a(\triv)=0$), $L(\lambda)$ is any highest
weight module in the cell of $\sigma$. 


Return to our question, let $S' = \prod_i S_{\bfcc_i(\ckcO)}$. We want to find
the component $\sigma_0$ in $\Ind_{S'}^{S_n} 1$ whose $a(\sigma_0)$ is maximal,
i.e. the Young diagram of $\sigma_0$ is minimal. 

 By the branaching rule, $\sigma \subset \Ind_{S'}^{S_n} 1$ is given by adding
 rows of lenght $\bfcc_i(\ckcO)$ repeatly (Each time add at most one box in each
 column). 
 Now it is clear that $\sigma_0 = \ckcO^t$ is desired. 

 This agrees with the Barbasch-Vogan duality $\dBV$ given by 
 \[
  \ckcO \xrightarrow{Springer}\ckcO \xrightarrow{\otimes \sgn} \ckcO^t 
  \xrightarrow{Springer} \ckcO^t.
 \]
}

The $\Wint{\lamck}$-module $\Cint{\lamck}$ is given by the following formula:
\[
  \begin{split}
  \Cint{\lamck} &\cong \cC_{n_e}\otimes \cC_{n_o} \quad \text{with} \\ 
 \cC_n &:= \bigoplus_{\substack{s,a,b\\2s+a+b=n}} 
 \Ind_{W_s\times S_a\times S_b}^{S_{n}} \epsilon \otimes 1\otimes 1. % \text{ is a $S_n$-module.} 
  \end{split}
\] 

According to Vogan duality,  we can obtain the above formula by tensoring $\sgn$
on the forumla of the unitary groups in \cite{BV.W}*{Section~4}.

By branching rules of the symmetric groups,  $\Unip_{\ckcO}(G)$ can be parameterized by painted partition. 

\begin{equation}\label{eq:PP.AR}
\PP_{A^{\bR}}(\ckcO) = \Set{\uptau:=(\tau, \cP)|
  \begin{array}{l}
    \text{$\tau = \ckcO^{t}$}\\
    \text{$\Im(\cP)\subset \set{\bullet,c,d}$}\\
    \text{$\#\set{i|\cP(i,j)=\bullet}$ is even}
    % \text{``$\bullet$'' occures with even}\\
    % \text{mulitplicity in each column}
  \end{array}
}.
\end{equation}
For $\uptau:=(\tau,\cP)\in \PP_{A^{\bR}}(\ckcO)$, we write $\cP_{\uptau}:= \cP$.

\trivial{
The typical diagram of all columns with even length $2c$ are
\[
\ytb{\bullet\cdots\bullet\bullet\cdots\bullet,\vdots\vdots\vdots\vdots\vdots\vdots,
\bullet\cdots\bullet c\cdots c,
\bullet\cdots\bullet d\cdots d
}  
\]

The typical diagram of all columns with odd length $2c+1$ are
\[
\ytb{\bullet\cdots\bullet\bullet\cdots\bullet,\vdots\vdots\vdots\vdots\vdots\vdots,
\bullet\cdots\bullet \bullet\cdots\bullet ,
c\cdots c d\cdots d
}  
\]
}

Let $\sgn_n\colon \GL_n(\bR)\rightarrow \set{\pm 1}$ be the sign of determinant. 
Let $1_n$ be the trivial representation of $\GL_n(\bR)$. 
For $\uptau\in \PP{\ckcO}$, we attache the representation 
\begin{equation}\label{eq:u.GLR}
\pi_\uptau := 
\bigtimes_{j} \underbrace{1_j \times \cdots \times 1_j}_{c_j\text{-terms}}\times
\underbrace{\sgn_j \times \cdots \times {\sgn_j} }_{d_j\text{-terms}}.
\end{equation}
Here 
\begin{itemize}
  \item 
$j$ running over all column lengths in $\ckcO^t$, 
\item $d_j$ is the number of
columns of length $j$ ending with the symbol ``d'',
\item  $c_j$ is the number of
columns of length $j$ ending with the symbol ``$\bullet$'' or ``$c$'', and 
\item  ``$\times$'' denote the parabolic induction.  
\end{itemize}

\subsection{Special Unipotent representations of $G=\GL_{m}(\bH)$}

Suppose that $\cO$ is the complexificiation of a rational nilpotent $\GL_{m}(\bH)$-orbit. 
Then $\cO$ has only even length columns. 
Therefore, $\Unip_\ckcO(G) \neq\emptyset$ only if $\ckcO = \ckcO_e$. 

In this case the coherent continuation representation is given by  
\[
  \Cint{\lamck}(G) = \Ind_{W_m}^{S_2m}\epsilon 
\]
and $\Unip_\ckcO(G)$ is a singleton. %We use partition $\tau:= \ckcO^t$ to parameter special unipotent representations of $\GL_{m}(\bH)$. 
For each partition $\tau$ only having even columns, we define 
\[
  \pi_{\tau} := \bigtimes_i 1_{\bfcc_i(\tau)/2}. 
\] 

\subsection{Counting special unipotent repesentations of $\rU(p,q)$}
We call the parity of $\abs{\ckcO}$ the ``good pairity''.  The other pairity is called the ``bad parity''. 
We write $\ckcO = \ckcO_g\cup \ckcO_b$ where $\ckcO_g$ and $\ckcO_b$ consist of good parity length rows
and bad parity rows respectively.

Let $(n_g,n_b) = (\abs{\ckcO_g},\abs{\ckcO_b})$.
Now as the $S_{n_g}\times S_{n_b}$ 
\[
\bigoplus_{\substack{p,q\in \bN\\p+q=n}} \Cint{\lamck}(\rU(p,q)) = \cC_{g}\otimes \cC_{b}
\]
where 
\[
  \begin{split}
 \cC_{g} &= \bigoplus_{\substack{s,a,b\in \bN\\2s+a+b=n_g}} \Ind_{W_{s}\times S_a\times S_b}^{S_{n_g}}
 1\otimes \sgn\otimes \sgn \\
 \cC_{b} &= \begin{cases}
  \Ind_{W_{\frac{n_b}{2}}}^{S_{n_b}} 1 & \text{if $n_b$ is even}\\
  0 & \text{otherwise}. 
 \end{cases}
  \end{split}
\]

By the above formula, we have
\begin{lem}
  \begin{enumT}
    \item
The set $\Unip_{\ckcO_b}(\rU(p,q))\neq \emptyset$ if and only if $p=q$ and 
each row lenght in $\ckcO$ has even multiplicity.
\item
Suppose $\Unip_{\ckcO_b}(\rU(p,p))\neq \emptyset$, let $\ckcO'$ be the Young diagram 
such that $\bfrr_i(\ckcO') = \bfrr_{2i}(\ckcO_b)$ and $\pi'$ be the unique special 
uinpotent representation in $\Unip_{\ckcO'}(\GL_{p}(\bC))$. 
Then the unique element in $\Unip_{\ckcO_b}(\rU(p,p))$  is given by 
\[
  \pi := \Ind_{P}^{\rU(p,p)} \pi'
\]
where $P$ is a parabolic subgroup in $\rU(p,p)$ with Levi factor equals
to $\GL_p(\bC)$.
\item 
In general, when $\Unip_{\ckcO_b}(\rU(p,p))\neq \emptyset$, we have a natural bijection 
\[  
  \begin{array}{rcl}
  \Unip_{\ckcO_g}(\rU(n_1,n_2)) &\longrightarrow& \Unip_{\ckcO}(\rU(n_1+p,n_2+p))\\
  \pi_0 & \mapsto & \Ind_P^{\rU(n_1+p,n_2+p)} \pi'\otimes \pi_0
  \end{array}
\]
where $P$ is a parabolic subgroup with Levi factor $\GL_p(\bC)\times \rU(n_1,n_2)$. 
  \end{enumT}
\end{lem}

The above lemma ensure us to reduce the problem to the case when $\ckcO = \ckcO_g$. 
Now assume $\ckcO = \ckcO_g$ and so $\Cint{\ckcO}$ corresponds to the blocks of 
the infinitesimal character of the trivial representation.   

By \cite{BV.W}*{Theorem~4.2},  Harish-Chandra cells in $\Cint{\ckcO}$ are in one-one
correspondence to real nilpotent orbits in $\cO:=\dBV(\ckcO)=\ckcO^t$. 

\trivial{
From the branching rule, the cell is parametered by painted partition 
\[
\PP{}(\rU):=\set{\uptau\in \PP{}| \begin{array}{l}\Im (\uptau) \subseteq  \set{\bullet, s,r}\\
  \text{``$\bullet$'' occures even times in each row}
\end{array} 
  }.  
\]

The bijection $\PP{}(\rU)\rightarrow \SYD, \uptau\mapsto \sO$ is given by the following recipe:
The shape of $\sO$ is the same as that of $\uptau$. 
$\sO$ is the unique (upto row switching) signed Young diagram such that
\[
  \sO(i,\bfrr_i(\uptau)) := \begin{cases}
    +,  & \text{when }\uptau(i,\bfrr_i(\uptau))=r;\\
    -,  & \text{otherwise, i.e. }\uptau(i,\bfrr_i(\uptau))\in \set{\bullet,s}.
  \end{cases}
\] 

\begin{eg}
  \[
 \ytb{\bullet\bullet\bullet\bullet r,\bullet\bullet , sr,s,r}   
 \quad
 \mapsto\quad
 \ytb{+-+-+,+-, -+,-,+}   
  \]
\end{eg}
}

Now the following lemma is clear. 
\begin{lem}
When $\ckcO=\ckcO_g$, the associated varity of every special unipotent representations in $\Unip_\ckcO(\rU)$  
is irreducible. Moreover, the following map  is a bijection. 
\[  
  \begin{array}{rcl}
  \Unip_{\ckcO_g}(\rU(n_1,n_2)) &\longrightarrow& \set{\text{rational forms of $\ckcO^t$}}\\
  \pi_0 & \mapsto & \wAV(\pi_0).
  \end{array}
\]
\qed
\end{lem}
\begin{remark}
  Note that the parabolic induction of an rational nilpotent orbit can be reducible. 
  Therefore, when $\ckcO_b\neq \emptyset$, the special unipotent representations can have
  reducible associated variety. Meanwhile, it is easy to see that the map
  $\Unip_{\ckcO}(\rU) \ni \pi \mapsto\wAV(\pi)$ is still injective. 
\end{remark}

We will show that every elements in $\Unip_{\ckcO_g}$ can be constructed by iterated theta lifting.  
For each $\uptau$, let $\sO$ be the corresponding real nilpotent orbit. Let
$\Sign(\sO)$ be the signature of $\sO$, $\DD(\sO)$ be the signed Young diagram
obtained by deleteing the first column of $\sO$. 
Suppose $\sO$ has $k$-columns. Inductively we have a sequence of unitary groups
$\rU(p_i,q_i)$ with $(p_i,q_i) = \Sign(\DD^i(\sO))$ for $i=0, \cdots, k$. Then 
\begin{equation}\label{eq:u.U}
  \pi_\tau = \theta^{\rU(p_0,q_0)}_{\rU(p_1,q_1)} \theta^{\rU(p_1,q_1)}_{\rU(p_2,q_2)}\cdots   
\theta^{\rU(p_{k-1},q_{k-1})}_{\rU(p_k,q_k)}(1)
\end{equation}
where $1$ is the trivial representation of $\rU(p_k,q_k)$. 


Suppose $\ckcO = \ckcO_g$. Form the duality between cells of $\rU(p,q)$ and
$\GL(n,\bR)$. We have an ad-hoc (bijective) duality between unipotent
representations: 
\[
  \begin{array}{rcl}
 \dBV\colon \Unip_{\ckcO}(\rU)& \rightarrow &\Unip_{\ckcO^t}(\GL(\bR)) \\
 \pi_\uptau &\mapsto& \pi_{\dBV(\uptau)} \\ 
  \end{array}
\]

Here $\ckcO^t = \dBV(\ckcO)$ and $\dBV(\uptau)$ is the pained bipartition
obtained by transposeing $\uptau$ and replace $s$ and $r$ by $c$ and $d$
respectively. See \eqref{eq:u.U} and \eqref{eq:u.GLR} for the definition of
special unipotent representations on the two sides.  


\section{Counting in type BCD}

In this section, we consider the case when $\ckstar \in \set{B,C,D}$, i.e
$\star \in \set{B,\wtC, C,D,C^{*}, D^{*}}$.

We identify $\fhh^{*}$ with $\bZ^{n}$ where $n = \rank(\Gc)$
and let $\rho$ be the half sum of all positive roots.

Recall that
\[
  \text{good pairity} =
\begin{cases}
 %\text{odd} & \text{when } \ckstar\in \set{B,D}\\
 %\text{even} & \text{when } \ckstar = C\\
 \text{odd} & \text{when } \star \in \set{C,C^{*},D,D^{*}}\\
 \text{even} & \text{when } \star \in \set{B,\wtC}\\
\end{cases}
\]

\def\Wb{W_{b}}
\def\Wg{W_{g}}

  Suppose $\ckcO\in \Nil(\ckcG)$ with decomposition
  $\ckcO = \ckcO_{b}\cuprow \ckcO_{g}$.
  Then $\ckcG_{\lamck} = \ckcG_{b}\times \ckcG_{g}$.
  Let $n_{b}$ and $n_{g}$ be the rank of $\ckcG_{b}$ and $\ckcG_{g}$
  respectively. We have
  \[
    (n_{b}, n_{g}) =
    \begin{cases}
      (\half \abs{\ckcO_{b}}, \half(\abs{\ckcO_{g}}-1)) & \text{when
      } \star \in \set{C,C^{*}}\\
      (\half \abs{\ckcO_{b}}, \half\abs{\ckcO_{g}}) & \text{when
      } \star \in \set{B,\wtC,D,D^{*}}\\
    \end{cases}
  \]
  and integral Weyl group $\WLamck$is a product of two factors
  \[
    W_{[\lamck]} =\Wb\times \Wg
  \]
  where
  \[
    \begin{split}
    \Wb & := \begin{cases}
      \sfW_{n_{b}}  & \text{when } \star \in \set{B, \wtC} \\
      \sfW'_{n_{b}} & \text{when } \star \in \set{C,C^{*},D,D^{*}}
      \end{cases}\\
    \Wg & := \begin{cases}
      \sfW_{n_{g}}  & \text{when } \star \in \set{B,C, C^{*} } \\
      \sfW'_{n_{g}} & \text{when } \star \in \set{\wtC,D,D^{*}}
      \end{cases}
    \end{split}
  \]

  When $\Wb$ or $\Wg$ is a Weyl group of type $D_{n}$, we always have the
  preferred embedding of $\sfS_{n}$ into $\sfW'_{n}$ given by the root system of
  $\ckfgg$. The label $I$ on the irreducible character of $\sfW_{n}$ is refer to
  this particular embedding.

  More precisely, we identify bipartition with $n$ parts with $\Irr(\sfW_{n})$.
  To ease the notations, we let $(\tau_{L},\tau_{R})_{I}$ denote
  the unique irreducible character of $\sfW'_{n}$ given by
  \begin{itemize}
    \item the restriction of
    the irreducible character of $\sfW_{n}$ attached to $(\tau_{L},\tau_{R})$ if
    $\tau_{L}\neq \tau_{R}$, and
    \item
    the character
    $\Ind_{\sfS_{\frac{n}{2}}}^{\sfW_{n}} \tau_{L}$ if $\tau_{L}=\tau_{R}$.
  \end{itemize}
  We remark that we always have
  \[
    (\tau_{L},\tau_{R})_{I}=(\tau_{R},\tau_{L})_{I}
  \]
  as $\sfW'_{n}$-character.


  \subsection{The left cell}
  In this subsection, we described the Lusztig left cell attached to
  $\lambda_{\ckcO}$ in each cases, where $\star \in \set{B,C,\wtC,C^{*},D,D^{*}}$.


  To state the results, we made some definitions first.
  Define the irreducible  $W_{b}$-representation attached to $\ckcO_{b}$ by the following formula
  \[
    \tau_{b} := \begin{cases}
      \Big(\big(\half(\bfrr_{2}(\ckcO_{b})+1), \half(\bfrr_{4}(\ckcO_{b})+1), \cdots, \half(\bfrr_{2c}(\ckcO_{b})+1)\big),\\
       \hspace{1em}\big(\half(\bfrr_{2}(\ckcO_{b})-1), \half(\bfrr_{4}(\ckcO_{b})-1), \cdots, \half(\bfrr_{2c}(\ckcO_{b})-1) \big)\Big)
      & \text{if } \star \in \set{B,\wtC},\\
      \Big( \big(\half\bfrr_{2}(\ckcO_{b}), \half\bfrr_{4}(\ckcO_{b}),\cdots, \half\bfrr_{2c}(\ckcO_{b})\big), \\
      \hspace{1em} \big(\half\bfrr_{2}(\ckcO_{b}), \half\bfrr_{4}(\ckcO_{b}),\cdots, \half\bfrr_{2c}(\ckcO_{b}) \big)\Big)_{I}
      & \text{if } \star \in \set{C,C^{*}, D,D^{*}},\\
    \end{cases}
  \]
  with $2c = \bfcc_{1}(\ckcO_{b})$.

  Set
  \[
    \CPPs(\ckcO_{g}) =
    \begin{cases}
      \set{(2i-1,2i)| \bfrr_{2i-1}(\ckcO_{g})-
        \bfrr_{2i}(\ckcO_{g})\geq 2, %\text{and}
        i\in \bN^{+}} & \text{if $\star\in \Set{C,\wtC,C^{*}}$}\\
    \set{(2i,2i+1)| \bfrr_{2i}(\ckcO_{g})- \bfrr_{2i+1}(\ckcO_{g})\geq 2, %\text{and }
      i\in \bN^{+}} & \text{if $\star\in \Set{B,D,D^{*}}$}.
    \end{cases}
  \]
  Let
  \[
    \wtA(\ckcO) := \bF_{2}[\CPP(\ckcO_{g})]
  \] be the power set of $\CPPs(\ckcO_{g})$.

  For each $\sP\in \wtA(\ckcO)$ we define an element $\tau_{\sP}$ in $\Irr(\Wg)$.
  Here
  \[
    \tau_{\sP} :=
    \begin{cases}
      (\imathp,\jmathp) & \text{when } \star \in \set{B,C, C^{*} } \\
      (\imathp,\jmathp)_{I} & \text{when } \star \in \set{\wtC,D,D^{*}}
    \end{cases}
  \]
  and $(\imathp, \jmathp)$ are given by the following formulas:
  \begin{itemize}
    \item Suppose $\star\in \set{C,C^{*}}$ and let
    $l=\min\set{i| \bfrr_{2i}(\ckcO_{g})=0}$.
    Then
    \[
      (\bfcc_{l}(\imathp), \bfcc_{l}(\jmathp)) :=
      (0,\half(\bfrr_{2l+1}(\ckcO_{g})-1))
    \]
    and, for all $1\leq i< l$
    \[
      (\bfcc_{i}(\imathp), \bfcc_{i}(\jmathp)):=
      \begin{cases}
        (\half (\bfrr_{2i}(\ckcO_{g})+1),
        \half (\bfrr_{2i-1}(\ckcO_{g})-1))
        & \text{if } (2i-1,2i)\notin \sP,\\
        (\half (\bfrr_{2i-1}(\ckcO_{g})+1),\half (\bfrr_{2i}(\ckcO_{g})-1)) & \text{otherwise.}
      \end{cases}
    \]
    \item Suppose $\star\in \set{D,D^{*}}$ and let
    $l=\min\set{i| \bfrr_{2i+1}(\ckcO_{g})=0}$.
    Then
    \[
      \begin{split}
        \bfcc_{1}(\imathp) &:=
        \half(\bfrr_{1}(\ckcO_{g})+1)\\
        (\bfcc_{l+1}(\imathp), \bfcc_{l}(\jmathp)) &:= (0,\half(\bfrr_{2l}(\ckcO_{g})-1))
      \end{split}
    \]
    and, for all $1\leq i<l$
    \[
      (\bfcc_{i+1}(\imathp), \bfcc_{i}(\jmathp)):=
      \begin{cases}
        \left(\half (\bfrr_{2i+1}(\ckcO_{g})+1),
        \half (\bfrr_{2i}(\ckcO_{g})-1)\right)
        & \text{if } (2i,2i+1)\notin \sP,\\
        (\half (\bfrr_{2i}(\ckcO_{g})+1),\half (\bfrr_{2i+1}(\ckcO_{g})-1)) & \text{otherwise.}
      \end{cases}
    \]
    % \[
    %   (\bfcc_{i}(\imathp), \bfcc_{i}(\jmathp)):=
    %   \begin{cases}
    %     (\half (\bfrr_{2i-1}(\ckcO_{g})+1),\half (\bfrr_{2i}(\ckcO_{g})-1)) &\text{if } (2i-1,2i)\in \sP, \\
    %     (\half (\bfrr_{2i}(\ckcO_{g})+1), \half (\bfrr_{2i-1}(\ckcO_{g})-1))
    %     & \text{if } (2i-1,2i)\notin \sP\\
    %     & \text{ and }\bfrr_{2i}(\ckcO_{g})\neq 0,
    %     \\
    %     (0,0)
    %     & \text{if } \bfrr_{2i-1}(\ckcO_{g})=0,\\
    %     (0, \half (\bfrr_{2i-1}(\ckcO_{g})-1)) & \text{otherwise}
    %   \end{cases}
    % \]

    % \[
    %   (\bfcc_{l+1}(\imathp), \bfcc_{l+1}(\jmathp)) := (0,\half(\bfrr_{2l+1}(\ckcO_{g})-1))
    % \]
    % and for all $1\leq i\leq l$
    \item Suppose $\star=B$.
    Then
    \[
      \bfcc_{1}(\jmathp)  := \half\bfrr_{1}(\ckcO_{g})
    \]
    and for all $i\geq 1$
    \[
      (\bfcc_{i}(\imathp), \bfcc_{i+1}(\jmathp)):=
      \begin{cases}
        (\half \bfrr_{2i}(\ckcO_{g}), \half \bfrr_{2i+1}(\ckcO_{g}))
        & \text{if } (2i,2i+1)\notin \sP,\\
        (\half \bfrr_{2i+1}(\ckcO_{g}),\half \bfrr_{2i}(\ckcO_{g})) & \text{otherwise.}
      \end{cases}
    \]
    \item Suppose $\star = \wtC$.
    Then for all  $i\geq 1$
    \[
      (\bfcc_{i}(\imathp), \bfcc_{i}(\jmathp)):=
      \begin{cases}
        (\half \bfrr_{2i-1}(\ckcO_{g}), \half \bfrr_{2i}(\ckcO_{g}))
        & \text{if } (2i-1,2i)\notin \sP,\\
        (\half \bfrr_{2i}(\ckcO_{g}),\half \bfrr_{2i-1}(\ckcO_{g})) & \text{otherwise.}
      \end{cases}
    \]
  \end{itemize}

  For $\sP\subset\CPPs(\ckcO_{g})$, let $\sP^{c}$ be the complement of $\sP$ in
  $\CPPs(\ckcO_{g})$ and we have $\tau_{\sP} = \tau_{\sP^{c}}$ if
  $\star \in \Set{\wtC,D,D^{*}}$.

  We define
  \[
    \barA(\ckcO)=
    \begin{cases}
      \wtA(\ckcO) & \text{when } \star \in \set{B,C,C^{*}},\\
      \wtA(\ckcO)/\wp\sim\wp^{c} & \text{when } \star \in \set{\wtC,D,D^{*}}.\\
    \end{cases}
  \]
  Here $\wtA(\ckcO)/\sP\sim\sP^{c}$ denote the quotient of $\wtA(\ckcO)$ by
  identifying $\sP$ with its complement $\sP^{c}$.

  When $\star\neq \wtC$, $\barA(\ckcO)$ is nothing but the Lusztig canonical
  quotient attached to $\ckcO$.
  \trivial{
    This can be seem from the following lemma, c.f.
    \cite{BVUni}*{Proposition~5.28}.
  }

  Note that by definition, we have
  $\wtA(\ckcO)=\wtA(\ckcO_{g})$ and
  $\barA(\ckcO)=\barA(\ckcO_{g})$.

  Recall that
  \[
    \LV_{\ckcO}:= \left(J_{\Wlamck}^{\WLamck} \sgn\right)
    \otimes \sgn.
  \]
  and $\LC_{\ckcO}$ is the multiset of irreducible components

  \begin{lem}[c.f. Barbasch-Vogan{\cite{BVUni}*{Proposition~5.28}}]
    \label{lem:Lcell}
    In all the cases, $\LC_{\ckcO}$ is multiplicity free and
    we have the following bijections
    \[
      \begin{array}{lccccccc}
        \barA(\ckcO)&=&\barA(\ckcO_{g}) & \longrightarrow & \LC(\ckcO_{g})
        & \longrightarrow & \LC(\ckcO)\\
       &  &\sP & \mapsto & \tau_{\sP} &
         \mapsto & \tau_{b}\otimes \tau_{\sP}.
      \end{array}
    \]
    Moreover,
    \[
      \tau_{\ckcO}=\tau_{b}\otimes \tau_{\emptyset}
    \] is the unique special
    representation in $\LC_{\ckcO}$ and
    \begin{equation}\label{eq:dBV.W}
      \Spr(j_{\WLamck}^{W}(\tau_{\ckcO})) = \dBV(\ckcO).
    \end{equation}
    Here $\dBV$ is the metaplectic dual if $\star=\wtC$ and
    is the Barbasch-Vogan dual otherwise.
  \end{lem}
  \begin{proof}
    For $\ckcO_{g}$, the lemma is given by \cite{BVUni}*{Proposition~5.28}.
    For all the cases, the lemma follow from an induction on number of columns using
    Lusztig's formula of $J$-induction in \cite{Lu}*{\S 4.4-4.6}.
    The equality \eqref{eq:dBV.W} is due to Barbasch-Vogan for linear groups
    \cite{BVUni}*{Proposition~A2}.

   \trivial[]{
  {\bf Suppose $\star=C$.}

      In this case, bad parity is even and each row length occur with even
  multiplicity. Suppose
  $\ckcO_{b} = (C_{1}, C_{1}, C_{2},C_{2}, \cdots, C_{k'},C_{k'})$ with
  $c_{1}=2k$ and $k' = \bfrr_{1}(\ckcO_{b})$.
  \[
    W_{\lamckb} = S_{C_{1}}\times S_{C_{2}}\times \cdots S_{C_{k'}}.
  \]
  The symbol of trivial representation of trivial group of type D is
  \[
    \binom{0,1, \cdots, k-1}{0,1, \cdots, k-1}.
  \]
  Now it is easy to see that (use the similar computation as below)
  \[
    J_{W_{\lamckb}}^{W_{b}}\sgn = ((\half C_{1}, \half C_{2},\cdots, \half C_{k'}),(\half C_{1}, \half C_{2},\cdots, \half C_{k'})).
  \]


  For the good parity part.
  Let $r'_{i}  = \floor{\half(\bfrr_{i}(\ckcO_{g})-\bfrr_{i+1}(\ckcO_{g}))}$.
  Suppose $\ckcO_{g}$ has $2l+1$ columns (superscripts denote the multiplicity)
  \[
    \ckcO_{g} = ((2l+1)^{2r'_{2l+1}+1}, 2l^{2r'_{2l}}, (2l-1)^{2r'_{2l-1}},
    \cdots, 2^{2r'_{2}}, 1^{2r'_{1}}  )
  \]
  and
  % $\ckcO_{g} = (2c_{1}+1, C_{2}, C_{2},C_{3},C_{3},\cdots, C_{k'},C_{k'})$ with
  % $2c_{1}+1=2l+1$ and $2k'+1 = \bfrr_{1}(\ckcO_{g})$.
  \[
    W_{\lamckg} = W_{l}\times
    \underbrace{S_{2l+1}\times \cdots \times S_{2l+1}}_{2r'_{2l+1}\text{-terms}}
    \times \prod_{i<2l+1}
    \underbrace{S_{i}\times \cdots\times S_{i}}_{r'_{i}\text{-terms}}
  \]

  The symbol of sign representation of $W_{l}$ is
  \[
    \binom{0,1,2, \cdots, l}{1,2, \cdots, l}.
  \]
  The induction begins with the longest columns to the shorter columns

  Induce to include all $2l+1$-length columns yields
  \[
    \binom{r'_{2l+1}+0,r'_{2l+1}+1,r'_{2l+1}+2, \cdots, r'_{2l+1}+l}{
r'_{2l+1}+1,r'_{2l+1}+2, \cdots, r'_{2l+1}+l}.
  \]
  Now move the the shorter columns, we see that when even columns
  $(2i)^{2r'_{2i}}$ occurs, it adds $(i)^{r'_{2i}}$ columns on the both sides of
  the bipartition; when odd columns $(2i+1)^{r'_{2i+1}}$ occur,  the bifurcation
  happens: one can
  \begin{itemize}
    \item attach columns $(i+1)^{r'_{2i+1}}$ on the left and
    columns $(i)^{r'_{2i+1}}$ on the right, which corresponds to
    $(2i+1,2i+2)\neq \sP$, or
    \item
    attach columns $(i)^{r'_{2i+1}}$ on the left and
    columns $(i+1)^{r'_{2i+1}}$ on the right, which corresponds to
    $(2i+1,2i+2)\in \sP$,
  \end{itemize}

  Therefore,
  \[
    \begin{array}{ccc}
      J_{W_{\lamckg}}^{W_{g}} \sgn
      &\leftrightarrow&  \bF_{2}(\CPP(\ckcO_{g}))\\
     (\cktau_{L},\cktau_{R}) =:\cktau_{\sP}&\leftrightarrow & \sP
    \end{array}
  \]
  where
  \[
  \bfrr_{l+1}(\cktau_{L})  =
      r'_{2l+1} =
      \half (\bfrr_{2l+1}(\ckcO_{g})-1)
  \]
  and, if $(2i-1,2i)\notin \sP$,
  \[
    \begin{split}
      \bfrr_{i}(\cktau_{L}) & = \sum_{l\geq 2i-1} r'_{l}
      = \half(\bfrr_{2i-1}(\ckcO)-1)\\
      \bfrr_{i}(\cktau_{R}) & = 1 + \sum_{l\geq 2i} r'_{l}
      = \half(\bfrr_{2i}(\ckcO)+1)
    \end{split}
  \]
  if $(2i-1,2i)\in \sP$,
  \[
    \begin{split}
      \bfrr_{i}(\cktau_{L}) & = \sum_{l\geq 2i} r'_{l}
      = \half(\bfrr_{2i}(\ckcO)-1)\\
      \bfrr_{i}(\cktau_{R}) & = 1 + \sum_{l\geq 2i-1} r'_{l}
      = \half(\bfrr_{2i-1}(\ckcO)+1)
    \end{split}
  \]

  % \[
  %   \begin{split}
  %     \bfrr_{l+1}(\cktau_{L}) & =
  %     r'_{2l+1} =
  %     \half (\bfrr_{2l+1}(\ckcO_{g})-1)\\
  %     (\bfrr_{i}(\cktau_{L}), \bfrr_{i}(\cktau_{R})) & =
  %     \begin{cases}
  %       (\half(\bfrr_{2i-1}(\ckcO_{g})-1), \half(\bfrr_{2i}(\ckcO_{g})+1)) & (2i-1,2i)\notin \sP\\
  %       (\half(\bfrr_{2i}(\ckcO_{g})-1), \half(\bfrr_{2i-1}(\ckcO_{g})+1)) & (2i-1,2i)\in \sP
  %     \end{cases}
  %   \end{split}
  % \]

  Since $\tau_{\sP} = \cktau_{\sP}\otimes \sgn$, we get the claim.

  We adopt the convention that
  \[
    \sfS_{\cO} := \prod_{i\in \bN^{+}}\sfS_{\bfcc_{i}(\cO)}
  \]
  so that $j_{\sfS_{\cO}}^{\sfS_{\abs{\cO}}}\sgn = \cO$ for each partition $\cO$.

  Now consider the orbit under the Springer correspondence.

  Let
  $\ckcO'_{b}: = [\bfrr_{2}(\ckcO_{b}), \bfrr_{4}(\ckcO_{b}),\cdots, \bfrr_{2k}(\ckcO_{b})]$,
  $\cO'_{b}:=(\ckcO'_{b})^{t}$ and $\cO_{b}:=\cO'_{b}\cupcol \cO'_{b}$.
  Clearly, $\ckcO_{b} = \ckcO'_{b}\cuprow \ckcO'_{b}$.
  Note that $\tau_{b} = j_{S_{\cO'_{b}}}^{W'_{b}} \sgn$ (by the formula of fake
  degree see Lusztig or Carter's book). So, by induction by stage of
  $j$-induction, we have
  \[
    \wttau_{\cO}:= j_{W'_{b}\times W_{g}}^{W_{n}} (\tau_{b}\otimes \tau_{\emptyset}) = j_{S_{\cO'_{b}}\times W_{g}}^{W_{n}} \sgn\otimes \tau_{\sP}.
  \]
  By Barbasch-Vogan, $\cO_{g}:=\Spr(\tau_{\emptyset}) = d_{BV}(\ckcO_{g})$,
  which is well know how to calculate. (In fact, one can deduce the result by
  our computation. )

  Since the Springer correspondence commutes with parabolic induction, we get
  $\Spr(\wttau) = \Ind_{\GL_{\cO'_{b}}\times \Sp(2g)}^{\Sp(2n)} 0\times \cO_{g} = \cO_{b}\cupcol \cO_{g}$.


  \medskip

  {\bf Suppose $\star=D$.}

  The bad parity part is the same as that of the case when $\star = C$.

  Now consider the good parity part.
  \[
    \ckcO_{g} = ((2l)^{2r'_{2l}+1}, (2l-1)^{2r'_{2l-1}}, (2l-2)^{2r'_{2l-2}},
    \cdots, 2^{2r'_{2}}, 1^{2r'_{1}}  )
  \]
  and
  \[
    W_{\lamckg} = W'_{l}\times
    \underbrace{S_{2l}\times \cdots \times S_{2l}}_{2r'_{2l}\text{-terms}}
    \times \prod_{i<2l}
    \underbrace{S_{i}\times \cdots\times S_{i}}_{r'_{i}\text{-terms}}
  \]

  The symbol of sign representation of $W'_{l}$ is
  \[
    \binom{0,1, \cdots, l-1}{1,2, \cdots, l\phantom{-1}}.
  \]
  (Here we always made the choice of the top and bottom row to compatible with the
  type $C$ case. )

  Induce to include all $2l$-length columns yields
  \[
    \binom{r'_{2l}+0,r'_{2l}+1, \cdots, r'_{2l}+l-1}{
      r'_{2l}+1,r'_{2l}+2, \cdots, r'_{2l}+l\phantom{-1}}.
  \]
  Now move the the shorter columns. When odd columns
  $(2i+1)^{2r'_{2i+1}}$ occurs, it adds $(i)^{r'_{2i+1}}$ columns on the left
  and $(i+1)^{r'_{2i+1}}$ on the right.
  When even columns $(2i)^{r'_{2i}}$ occur,  the bifurcation
  happens: one can
  \begin{itemize}
    \item attach columns $(i)^{r'_{2i}}$ on the left and
    columns $(i)^{r'_{2i}}$ on the right, which corresponds to
    $(2i,2i+1)\neq \sP$, or
    \item
    attach columns $(i-1)^{r'_{2i}}$ on the left and
    columns $(i+1)^{r'_{2i}}$ on the right, which corresponds to
    $(2i,2i+
    1)\in \sP$,
  \end{itemize}

  Therefore,
  \[
    \begin{array}{ccc}
      \bF_{2}(\CPP(\ckcO_{g}))&\longrightarrow
      & J_{W_{\lamckg}}^{W_{g}} \sgn \\
      \sP&\mapsto&    (\cktau_{L},\cktau_{R}) =:\cktau_{\sP}
    \end{array}
  \]
  where
  \[
  \bfrr_{l}(\cktau_{L})  =
      r'_{2l} =
      \half (\bfrr_{2l}(\ckcO_{g})-1)
  \]
  \[
  \bfrr_{1}(\cktau_{R})  =
      1+ \sum_{i} r'_{i} =
      \half (\bfrr_{1}(\ckcO_{g})+1)
  \]
  and, if $(2i,2i+1)\notin \sP$,
  \[
    \begin{split}
      \bfrr_{i}(\cktau_{L}) & = \sum_{l\geq 2i} r'_{l}
      = \half(\bfrr_{2i}(\ckcO)-1)\\
      \bfrr_{i+1}(\cktau_{R}) & = 1 + \sum_{l\geq 2i+1} r'_{l}
      = \half(\bfrr_{2i+1}(\ckcO)+1)
    \end{split}
  \]
  if $(2i,2i+1)\in \sP$,
  \[
    \begin{split}
      \bfrr_{i}(\cktau_{L}) & = \sum_{l\geq 2i+1} r'_{l}
      = \half(\bfrr_{2i+1}(\ckcO)-1)\\
      \bfrr_{i}(\cktau_{R}) & = 1 + \sum_{l\geq 2i} r'_{l}
      = \half(\bfrr_{2i}(\ckcO)+1)
    \end{split}
  \]

  Also note that $\cktau_{\sP}=\cktau_{\sP^{c}}$.
  The rest parts are the same as that of type $C$.

  {\bf Suppose $\star=B$. }

  In this case, bad parity is odd and every odd row occurs with with even times.

  We can write $r'_{i} := \floor{\half(\bfrr_{i}(\ckcO_{b})-\bfrr_{i-1}(\ckcO_{b}))}$
  \[
    \ckcO_{b} % = [2r_{1}+1, 2r_{1}+1, \cdots, 2r_{k}+1,2r_{k}+1]
    %= (2c_{0},2c_{1},2c_{1}, \cdots, 2c_{l}, 2c_{l}).
    = ((2l)^{2r'_{2l}+1}, (2l-1)^{2r'_{2l-1}},\cdots, 1^{2r'_{1}})
  \]
 % with $k = c_{0}$ and $l = r_{1}$.
  Then
  \[
    W_{\lamckb} = W_{l} \times
    \underbrace{S_{2l}\times \cdots \times S_{2l}}_{2r'_{2l}\text{-terms}}
    \times \prod_{i<2l}
    \underbrace{S_{i}\times \cdots\times S_{i}}_{r'_{i}\text{-terms}}
  \]
  (Note that in the product, $r'_{i}=0$ if $i$ is odd.)
  The computation of
  $\cksigma_{b} = J_{W_{\lamckb}}^{W_{b}} \sgn$ is similar to that of the good
  parity for type $C$ with no bifurcating, one deduce that $J$-induction and
  $j$-induction gives the same result.
  \[
    \begin{split}
    \cksigma_{b} &=
       \binom{0, 1+r_{l}, 2+r_{l-1}\cdots, l+r_{1}}{1+r_{l},2+r_{l-1}, \cdots, l+r_{1}}\\
       & = ([r_{1},r_{2},\cdots, r_{l}],[r_{1}+1,r_{2}+1,\cdots,r_{l}+1])\\
     \end{split}
   \]
  with $r_{i} = \half\bfrr_{2i-1}(\ckcO_{b}) =  \half\bfrr_{2i}(\ckcO_{b})$.
  Now
  \[
    \sigma_{b} =
    ((r_{1}+1,r_{2}+1,\cdots,r_{l}+1),
    (r_{1},r_{2},\cdots, r_{l}))
    = j_{S_{\cO'_{b}}}^{W_{b}}\sgn
  \]
  where $\cO'_{b}=(\bfrr_{2}(\ckcO_{b}),\bfrr_{4}(\ckcO_{b}),\cdots,
 \bfrr_{2l}(\ckcO_{b}))$.
 Under the Springer correspondence of type $B$, it corresponds to
 $\Ind_{\GL_{b}}^{\SO(2b+1)}\cO'_{b} = \cO'_{b}\cuprow \cO'_{b}\cuprow (1)$.

  % \[
  %   \begin{split}
  %     \cksigma_{b} &:= \sigma_{b}\otimes \sgn = j_{W_{\lamckb}}^{W_{b}} \sgn \\
  %     & = %\dagger_{2c_{l}}\cdots \dagger_{2c_{1}}
  %     \sigma_{b}\otimes \sgn = j_{W_{\lamckb}}^{W_{b}} \sgn\otimes
  %     \binom{0, 1, \cdots, c_{0}}{1, \cdots, c_{0}}\\
  %     & =
  %     \binom{0, 1+r_{k}, 2+r_{k-1}\cdots, c_{0}+r_{1}}{1+r_{k},2+r_{k-1}, \cdots, c_{0}+r_{1}}\\
  %     & = ([r_{1},r_{2},\cdots, r_{k}],[r_{1}+1,r_{2}+1,\cdots,r_{k}+1])\\
  %     &= ((c_{1},c_{2},\cdots, c_{k}),(c_{0},c_{1}, \cdots, c_{l}))\\
  %   \end{split}
  % \]


  % We take the convention that
  % $\dagger \cO = [r_{i}+1]$.
  % By abuse of notation, let $\dagger_{n} \sigma$  denote the
  % $j_{S_{n} \times W_{\abs{\sigma}}}^{W_{n+\abs{\sigma}}} \sgn\otimes \sigma$.
  % We can write
  % \[
  %   \ckcO_{b} = [2r_{1}+1, 2r_{1}+1, \cdots, 2r_{k}+1,2r_{k}+1]
  %   = (2c_{0},2c_{1},2c_{1}, \cdots, 2c_{l}, 2c_{l})
  % \]
  % with $k = c_{0}$ and $l = r_{1}$.

% \[
% \begin{split}
%   W_{\lamckb} &= W_{c_{0}} \times S_{2c_{1}} \times S_{2c_{2}}\times \cdots \times S_{2c_{l}}\\
%   \cksigma_{b} &:= \sigma_{b}\otimes \sgn = j_{W_{\lamckb}}^{W_{b}} \sgn \\
%   & = \dagger_{2c_{l}}\cdots \dagger_{2c_{1}}
%   \binom{0, 1, \cdots, c_{0}}{1, \cdots, c_{0}}\\
%   & =
%   \binom{0, 1+r_{k}, 2+r_{k-1}\cdots, c_{0}+r_{1}}{1+r_{k},2+r_{k-1}, \cdots, c_{0}+r_{1}}\\
%   & = ([r_{1},r_{2},\cdots, r_{k}],[r_{1}+1,r_{2}+1,\cdots,r_{k}+1])\\
%   &= ((c_{1},c_{2},\cdots, c_{k}),(c_{0},c_{1}, \cdots, c_{l}))\\
% \end{split}
% \]

% Therefore
% \[
%   \begin{split}
%     \sigma_{b} &= \cksigma_{b}\otimes \sgn = ((r_{1}+1,r_{2}+1,\cdots,r_{k}+1),(r_{1},r_{2},\cdots, r_{k})) \\
%     & = j_{S_{2r_{1}+1}\times \cdots S_{2r_{k}+1}}^{W_{b}} \sgn\\
%     & = j_{S_{b}}^{W_{b}} (2r_{1}+1, 2r_{2}+1, \cdots, 2r_{k}+1)
%   \end{split}
% \]
% which corresponds to the orbit
% \[
%   \cO_{b} = (2r_{1}+1, 2r_{1}+1,2r_{2}+1, 2r_{2}+1,  \cdots,2r_{k}+1, 2r_{k}+1 ) = \ckcO_{b}^{t}.
% \]
% (Note that $\cO'_{b} = (2r_{1}+1,2r_{2}+1, \cdots, 2r_{k}+1)$ which corresponds
% to $j_{W_{L_{b}}}^{S_{b}}\sgn$ and $\ind_{L}^{G} \cO'_{b} = \cO_{b}$.
% )
% This implies the unique special representation is
% \[
%   \sigma_{b} = (j_{W_{\lamckb}}^{W_{b}}\sgn), \quad \text{where } W_{L,b} = \prod_{i=1}^{k} S_{2r_{i}+1}.
% \]
% The $J$-induction is calculated by \cite{Lu}*{(4.5.4)}.
% It is easy to see that in our case $J_{W_{\lamckb}}^{W_{b}} \sgn$ consists of
% the single special representation by induction.


 Now we consider the good parity parts, where each row of $\ckcO_{g}$ has even
 length.

 Assume $r'_{i} := \half(\bfrr_{i}(\ckcO_{g})-\bfrr_{i-1}(\ckcO_{g}))$
 and so
  \[
    \ckcO_{g} % = [2r_{1}+1, 2r_{1}+1, \cdots, 2r_{k}+1,2r_{k}+1]
    %= (2c_{0},2c_{1},2c_{1}, \cdots, 2c_{l}, 2c_{l}).
    = ((2l+1)^{2r'_{2l+1}}, (2l)^{2r'_{2l}},\cdots, 1^{2r'_{1}})
  \]
% Consider
% \[
% \cO_{g} = [2r_{1},2r_{2}, \cdots, 2r_{2k-1},2r_{2k}]
% = (C_{1},C_{1}, C_{2},C_{2},\cdots, C_{l}, C_{l}).
% \]
with $l =\min\set{i|\bfrr_{2i+2}(\ckcO_{g}) = 0}$.

Then
  \[
    W_{\lamckg} =
    \times \prod_{i\leq 2l+1}
    \underbrace{S_{i}\times \cdots\times S_{i}}_{r'_{i}\text{-terms}}
  \]

Note that the trivial representation of the trivial group has symbol
\[
\binom{0,1, 2, \cdots, l\phantom{-1}}{0,1, \cdots, l-1}.
\]


  Induce to include all $2l+1$-length columns yields
  \[
    \binom{r'_{2l+1}+0,r'_{2l+1}+1,r'_{2l+1}+2,\cdots, r'_{2l+1}+l\phantom{-1}}{
      r'_{2l+1}+0,r'_{2l+1}+1, \cdots, r'_{2l+1}+l-1}.
  \]
  Now move the the shorter columns. When odd columns
  $(2i+1)^{2r'_{2i+1}}$ occurs, it adds $(i+1)^{r'_{2i+1}}$ columns on the left
  and $(i)^{r'_{2i+1}}$ on the right.
  When even columns $(2i)^{r'_{2i}}$ occur,  the bifurcation
  happens: one can
  \begin{itemize}
    \item attach columns $(i)^{r'_{2i}}$ on the left and
    columns $(i)^{r'_{2i}}$ on the right, which corresponds to
    $(2i,2i+1)\neq \sP$, or
    \item
    attach columns $(i-1)^{r'_{2i}}$ on the left and
    columns $(i+1)^{r'_{2i}}$ on the right, which corresponds to
    $(2i,2i+1)\in \sP$.
  \end{itemize}


  Therefore,
  \[
    \begin{array}{ccc}
      \bF_{2}(\CPP(\ckcO_{g}))&\longrightarrow
      & J_{W_{\lamckg}}^{W_{g}} \sgn \\
      \sP&\mapsto&    (\cktau_{L},\cktau_{R}) =:\cktau_{\sP}
    \end{array}
  \]
  where
  % \[
  % \bfrr_{2l+1}(\cktau_{L})  =
  %     r'_{2l+1} =
  %     \half \bfrr_{2l+1}(\ckcO_{g})
  % \]
  \[
  \bfrr_{1}(\cktau_{L})  =
      \sum_{i} r'_{i} =
      \half \bfrr_{1}(\ckcO_{g})
  \]
  and, if $(2i,2i+1)\notin \sP$,
  \[
    \begin{split}
      \bfrr_{i+1}(\cktau_{L}) & = \sum_{l\geq 2i+1} r'_{l}
      = \half\bfrr_{2i+1}(\ckcO)\\
      \bfrr_{i}(\cktau_{R}) & = \sum_{l\geq 2i} r'_{l}
      = \half\bfrr_{2i}(\ckcO)
    \end{split}
  \]
  if $(2i,2i+1)\in \sP$,
  \[
    \begin{split}
      \bfrr_{i+1}(\cktau_{L}) & = \sum_{l\geq 2i} r'_{l}
      = \half\bfrr_{2i}(\ckcO)\\
      \bfrr_{i}(\cktau_{R}) & = \sum_{l\geq 2i+1} r'_{l}
      = \half\bfrr_{2i+1}(\ckcO)
    \end{split}
  \]

  Some remarks on the BV-dual.
  The calculation of $\cO_{g}$ from $\tau_{\emptyset}$ can be reduced
  to the case of quasi-distinguished orbits
  (other case are deduced from this by parabolic induction, corresponds to
  attach two even columns for the balanced pairs).
  Compare Sommer's description of Springer correspondence with ours,
  we deduce that
  \[
    \cO_{g} = (\bfrr_{1}(\ckcO_{1})+1,\bfrr_{2}(\ckcO_{2})-1,\bfrr_{3}(\ckcO_{3})+1,
    \cdots, \bfrr_{2l}(\ckcO_{2l})-1,\bfrr_{2l+1}(\ckcO_{2l+1})+1)
  \]
  The rest parts are similar to that of type $D$.
  }


  We sketch the proof for the case when $\star = \wtC$.


  For a partition $\cO$, we set
  \[
    \sfS_{\cO} := \prod_{i\in \bN^{+}}\sfS_{\bfcc_{i}(\cO)}
  \]
  so that $j_{\sfS_{\cO}}^{\sfS_{\abs{\cO}}}\sgn = \cO$.


  We first consider the good parity (even) part.

 Now we consider the good parity parts, where each row of $\ckcO_{g}$ has even
 length.

 We set $r'_{i} := \half(\bfrr_{i}(\ckcO_{g})-\bfrr_{i-1}(\ckcO_{g}))$,
 $l =\min\set{i|\bfrr_{2i+1}(\ckcO_{g})=0}$,
 and write
  \[
    \ckcO_{g} % = [2r_{1}+1, 2r_{1}+1, \cdots, 2r_{k}+1,2r_{k}+1]
    %= (2c_{0},2c_{1},2c_{1}, \cdots, 2c_{l}, 2c_{l}).
    = ((2l)^{2r'_{2l}}, (2l-1)^{2r'_{2l-1}},\cdots, 1^{2r'_{1}})
  \]
  where $i^{r'}$ denotes $r'$-copies of length $i$ columns.
% Consider
% \[
% \cO_{g} = [2r_{1},2r_{2}, \cdots, 2r_{2k-1},2r_{2k}]
% = (C_{1},C_{1}, C_{2},C_{2},\cdots, C_{l}, C_{l}).
% \]
 The Weyl group of good parity is $\sfW'_{n_{g}}$ with
 $n_{g} = \half\abs{\ckcO_{g}}$.
 For $1\leq k\leq l$, let
  \[
    % S_{r,s}  =
    %  \prod_{i=r}^{s}
    %  \underbrace{\sfS_{i}\times \cdots\times
    %  \sfS_{i}}_{r'_{i}\text{-terms}}
    \vec{S}_{i}  =
     \underbrace{\sfS_{i}\times \cdots\times \sfS_{i}}_{r'_{i}\text{-terms}}
     \AND
     n_{k}  = \sum_{i=k}^{l} i\cdot r'_{i}.
     % \AND
     % n_{r,s}  = \sum_{i=r}^{s} i\cdot r'_{i}.
\]

Then $W_{\lamckg}=\prod_{i=1}^{l} \vec{S}_{i}$ and
\[
    \begin{split}
      \ckLV_{\ckcO_{g}}& :=J_{W_{\lamckg}}^{\Wg}\sgn\\
      & =  J_{\vec{S}_{1}\times \sfW'_{n_{2}}}^{W_{n_{1}}}
      \Big(\sgn \otimes
      J_{\vec{S}_{2}\times \sfW'_{n_{3}}}^{\sfW'_{n_{2}}}\Big(\sgn
      \otimes \cdots\big(J_{\vec{S}_{l}} \sgn\big)\cdots \Big)\Big) \\
    \end{split}
  \]
  Applying \cite{Lu}*{(4.6.2)} inductively, we see that the operation
  $J_{\vec{S}_{i}\times \sfW'_{n_{i+1}}}^{\sfW'_{n_{i}}}(\sgn \otimes \underline{\ \ \ })$
  doubles (resp. keeps) the number of irreducible components if $i$ is odd
  (resp. even).

  Suppose $\CPPs(\ckcO_{g}) = \emptyset$, then $\ckLV_{\ckcO_{g}}$
  is irreducible and marked by label $I$.

  Now assume $\CPPs(\ckcO_{g})\neq \emptyset$.
  Then the two parts of the bipartition of an irreducible
  component are different.
  Let $i_{0}:= \min\Set{i| (2i-1,2i)\in \CPPs(\ckcO_{g})}$.
  % $(2i_{0}-1,2i_{0})$ be the element in such that $i_{0}$ is minimal.
  Then we have a bijection
  \[
    \set{\sP\in \bF_{2}[\CPPs(\ckcO_{g})]|(2i_{0}-1,2i_{0})\notin \sP}  \longrightarrow  \ckLV_{\ckcO_{g}}
  \]
  to record the bifurcation when attaching odd length columns.
  Here we send
  $\sP$ to $\cktau_{\sP}:= (\cktau_{L},\cktau_{R})$ such that
  \[
    (\bfrr_{i}(\cktau_{L}),\bfrr_{i}(\cktau_{R}))
    := \begin{cases}
      (\half\bfrr_{2i}(\ckcO_{g}),\half\bfrr_{2i-1}(\ckcO_{g}))
      & \text{if } (2i-1,2i)\notin \sP\\
      (\half\bfrr_{2i-1}(\ckcO_{g}),\half\bfrr_{2i}(\ckcO_{g}))
      & \text{if } (2i-1,2i)\in \sP\\
    \end{cases}
  \]
  We obtain the structure of $\LC_{\ckcO_{g}}$ by tensoring the sign
  representation.

  \trivial[]{
  Note that the trivial representation of the trivial group is represented by the symbol
  \[
    \binom{0,1, 2, \cdots, l-1}{0,1,2, \cdots, l-1}_{I}.
  \]
  % Induce to include all $2l$-length columns yields
  % \[
  %   \binom{r'_{2l+1}+0,r'_{2l+1}+1,r'_{2l+1}+2,\cdots, r'_{2l+1}+l\phantom{-1}}{
  %     r'_{2l+1}+0,r'_{2l+1}+1, \cdots, r'_{2l+1}+l-1}.
  % \]

  Now move the the shorter columns. When even columns
  $(2i)^{2r'_{2i}}$ occurs, it adds $(i)^{r'_{2i}}$ columns on the left
  and $(i)^{r'_{2i}}$ on the right.
  When odd columns $(2i-1)^{r'_{2i-1}}$ occur,  the bifurcation
  happens: one can
  \begin{itemize}
    \item attach columns $(i-1)^{r'_{2i-1}}$ on the left and
    columns $(i)^{r'_{2i-1}}$ on the right, which corresponds to
    $(2i-1,2i)\neq \sP$, or
    \item
    attach columns $(i)^{r'_{2i-1}}$ on the left and
    columns $(i-1)^{r'_{2i-1}}$ on the right, which corresponds to
    $(2i-1,2i)\in \sP$.
  \end{itemize}
  Note that when we first encounter the longest odd column, we make the choice
  that the size of left part is larger than that of the right part.
  Now
  If $(2i-1,2i)\notin \sP$,
  \[
    \begin{split}
      \bfrr_{i}(\cktau_{L}) & = \sum_{l\geq 2i} r'_{l}
      = \half\bfrr_{2i}(\ckcO_{g})\\
      \bfrr_{i}(\cktau_{R}) & = \sum_{l\geq 2i-1} r'_{l}
      = \half\bfrr_{2i-1}(\ckcO_{g})
    \end{split}
  \]
  if $(2i-1,2i)\in \sP$,
  \[
    \begin{split}
      \bfrr_{i}(\cktau_{L}) & = \sum_{l\geq 2i-1} r'_{l}
      = \half\bfrr_{2i-1}(\ckcO_{g})\\
      \bfrr_{i}(\cktau_{R}) & = \sum_{l\geq 2i} r'_{l}
      = \half\bfrr_{2i}(\ckcO_{g})
    \end{split}
  \]
  }


  Now we consider the bad parity (odd) part.
  Suppose $\ckcO_{b}$ is nonempty such that
  \[
    \ckcO_{b} = (2c_{0},2c_{1}, 2c_{1}, 2c_{2},2c_{2}, \cdots, 2c_{k}, 2c_{k})
  \]
  where $2k+1=\bfrr_{1}(\ckcO_{b})$ and $2c_{i} = \bfcc_{2i+1}(\ckcO_{b})$.
  Now
  \[
    W_{\lamckb} = \sfW_{c_{0}} \times \sfS_{2c_{1}} \times \sfS_{2c_{2}}\times \cdots \times \sfS_{2c_{k}}
  \]
  and %$J_{W_{\lamckb}}^{W_{b}}\sgn$ is irreducible by
  \[
    \begin{split}
      \cktau_{b} &:= J_{W_{\lamckb}}^{W_{b}} \sgn
      = ((c_{1},c_{2},\cdots, c_{k}),(c_{0},c_{1}, \cdots, c_{l}))\\
      & = \big([\half(\bfrr_{2}(\ckcO)-1),\half(\bfrr_{4}(\ckcO)-1),\cdots, \half(\bfrr_{2c_{0}}(\ckcO)-1)],\\
      & \hspace{2em}
      [\half(\bfrr_{2}(\ckcO)+1),\half(\bfrr_{4}(\ckcO)+1),\cdots, \half(\bfrr_{2c_{0}}(\ckcO)+1)]\big)
    \end{split}
  \]
  is irreducible by \cite{Lu}*{(4.5.4)}.
  Tensoring with sign yields the formula of $\tau_{b}$. Moreover,
  by the fake degree formula (see \cite{Carter}*{p~376}), we have
\[
    \tau_{b} = \cktau_{b}\otimes \sgn
      = j_{\sfS_{\cO'_{b}}}^{\sfW_{n_{b}}} \sgn.
\]
where
$ \cO'_{b} := (\ckcO'_{b})^{t}:=(\bfrr_{2}(\ckcO),\bfrr_{4}(\ckcO),\cdots , \bfrr_{2c_{0}}(\ckcO))$.

\medskip
\def\ckfll{\check{\fll}}

Now we sketch the proof of \eqref{eq:dBV.W}.

Recall the definition of the metaplectic Barbasch-Vogan dual in \cite{BMSZ1}.
The duality map commutes with parabolic induction:
Suppose $\ckfll\subset \check \ckfgg$ is a parabolic subalgebra of $\ckfgg$ and
$\fll$ is the corresponding parabolic subalgebra in $\fgg$, then
\begin{equation}\label{eq:inddBV}
 \dBV(\ckcO) =  \Ind_{\fll}^{\fgg}(\dBV(\ckcO_{\ckfll}))
\end{equation}
for each nilpotent orbit $\ckcO$ in $\ckfgg$ such that
$\ckcO_{\ckfll}:=\ckcO\cap \ckfll\neq \emptyset$. This is clear by reducing to
the type $B$ case, see \cite{BMSZ1}*{Proposition~3.8}. By removing pairs of rows
with the same lengths in $\ckcO$, we reduced to check the equality in the case
when $\ckcO_{b}=\emptyset$ and $\bfrr_{2i-1}(\ckcO_{g})>\bfrr_{2i}(\ckcO_{g})$
for all $i$ such that $i\leq \bfcc_{1}(\ckcO_{g})$. In this case, both sides of
\eqref{eq:dBV.W} can be easily computed directly, which equals to
\[
  (\bfrr_{1}(\ckcO)-1, \bfrr_{2}(\ckcO)+1, \cdots,\bfrr_{2c-1}(\ckcO)-1,\bfrr_{2c}(\ckcO)+1)
\]
with $c = \min\set{i|\bfrr_{2i+1}(\ckcO)=0}$.

% Now we compare the metaplectic dual defined in \cite{BMSZ1} with the
% Weyl group representations.
\trivial[]{
  Compare Sommer's description of Springer correspondence
  we deduce that the RHS is
  \[
    \cO_{g} = (\bfrr_{1}(\ckcO_{1})-1,\bfrr_{2}(\ckcO_{2})+1,\bfrr_{3}(\ckcO_{3})+1,
    \cdots, \bfrr_{2l-1}(\ckcO_{2l-1})-1,\bfrr_{2l}(\ckcO_{2l})+1)
  \]
  The LHS is calculated by
  $((((\ckcO^{t})_{D})^{+})^{-})_{C}$.
  We write $R_{i}=\bfrr_{i}(\ckcO)=2r_{i}$.
  Now under our assumption, $R_{2i-1}>R_{2i}$, we have
  \[
    \begin{split}
      ((((\ckcO^{t})_{D})^{+})^{-})_{C} &=
      ((((R_{1},R_{2}, \cdots, R_{2l-1},R_{2l})_{D})^{+})^{-})_{C}\\
      &=((R_{1}-1,R_{2}, \cdots, R_{2l-1},R_{2l},1))_{C}\\
      &=(R_{1}-1,R_{2}+1, \cdots, R_{2l-1}-1,R_{2l}+1)\\
    \end{split}
  \]
  So the proof is done.

}

  \end{proof}

\begin{remark}
  When $\star=\wtC$, one can see that
  $\dBV(\ckcO_{b}\cuprow \ckcO_{g}) = \ckcO_{b}^{t} \cupcol \dBV(\ckcO_{g})$
  using \eqref{eq:inddBV}.
  \trivial[]{
  This could be checked using the formula of Springer correspondence
  directly, see Sommer's.  }
\end{remark}

\subsection{Coherent continuation representations}

We identify $\fhh^{*}$ with $\bC^{n}$.
Let $Q$ be the root lattice in $\fhh^{*}$
which is
\[
Q = \begin{cases}
  \bZ^{n} & \text{if  $\star = B$}\\
  %\set{(a_{1},a_{2},\cdots, a_{n})\in \bZ^{n}|\sum_{i=1}^{n}a_{i} \in 2\bZ}
  \Set{(a_{i})\in \bZ^{n}|\sum_{i=1}^{n}a_{i} \text{ is even}}
    & \text{if  $\star \in \set{C,\wtC,C^{*},D,D^{*}}$}\\
\end{cases}
\]
For $n_{b}, n_{g}\in \bN$ such that $n_{b}+n_{g}=n$, we consider the lattice
\[
  \Lambda_{n_{b},n_{g}} =
  (\underbrace{\half, \cdots, \half}_{n_{b}\text{-terms}}, \underbrace{0, \cdots, 0}_{n_{g}\text{-terms}}) + Q \subset \fhh^{*}.
\]

\begin{bibdiv}
  \begin{biblist}
% \bib{AB}{article}{
%   title={Genuine representations of the metaplectic group},
%   author={Adams, Jeffrey},
%   author = {Barbasch, Dan},
%   journal={Compositio Mathematica},
%   volume={113},
%   number={01},
%   pages={23--66},
%   year={1998},
% }

\bib{Ad83}{article}{
  author = {Adams, J.},
  title = {Discrete spectrum of the reductive dual pair $(O(p,q),Sp(2m))$ },
  journal = {Invent. Math.},
  number = {3},
 pages = {449--475},
 volume = {74},
 year = {1983}
}

%\bib{Ad07}{article}{
%  author = {Adams, J.},
%  title = {The theta correspondence over R},
%  journal = {Harmonic analysis, group representations, automorphic forms and invariant theory,  Lect. Notes Ser. Inst. Math. Sci. Natl. Univ. Singap., 12},
% pages = {1--39},
% year = {2007}
% publisher={World Sci. Publ.}
%}


\bib{ABV}{book}{
  title={The Langlands classification and irreducible characters for real reductive groups},
  author={Adams, J.},
  author={Barbasch, D.},
  author={Vogan, D. A.},
  series={Progress in Math.},
  volume={104},
  year={1991},
  publisher={Birkhauser}
}

\bib{AC}{article}{
  title={Algorithms for representation theory of
    real reductive groups},
  volume={8},
  DOI={10.1017/S1474748008000352},
  number={2},
  journal={Journal of the Institute of Mathematics of Jussieu},
  publisher={Cambridge University Press},
  author={Adams, Jeffrey},
  author={du Cloux, Fokko},
  year={2009},
  pages={209-259}
}

\bib{ArPro}{article}{
  author = {Arthur, J.},
  title = {On some problems suggested by the trace formula},
  journal = {Lie group representations, II (College Park, Md.), Lecture Notes in Math. 1041},
 pages = {1--49},
 year = {1984}
}


\bib{ArUni}{article}{
  author = {Arthur, J.},
  title = {Unipotent automorphic representations: conjectures},
  %booktitle = {Orbites unipotentes et repr\'esentations, II},
  journal = {Orbites unipotentes et repr\'esentations, II, Ast\'erisque},
 pages = {13--71},
 volume = {171-172},
 year = {1989}
}

\bib{AK}{article}{
  author = {Auslander, L.},
  author = {Kostant, B.},
  title = {Polarizations and unitary representations of solvable Lie groups},
  journal = {Invent. Math.},
 pages = {255--354},
 volume = {14},
 year = {1971}
}


\bib{B.Uni}{article}{
  author = {Barbasch, D.},
  title = {Unipotent representations for real reductive groups},
 %booktitle = {Proceedings of ICM, Kyoto 1990},
 journal = {Proceedings of ICM (1990), Kyoto},
   % series = {Proc. Sympos. Pure Math.},
 %   volume = {68},
     pages = {769--777},
 publisher = {Springer-Verlag, The Mathematical Society of Japan},
      year = {2000},
}


\bib{BMSZ1}{article}{
      title={On the notion of metaplectic Barbasch-Vogan duality},
      year={2020},
      author={Barbasch, Dan M.},
      author = {Ma, Jia-jun},
      author = {Sun, Binyong},
      author = {Zhu, Chen-Bo},
      eprint={2010.16089},
      archivePrefix={arXiv},
      primaryClass={math.RT}
}

\bib{BMSZ2}{article}{
      title={Special unipotent representations: orthogonal and symplectic groups},
      author={Barbasch, Dan M.},
      author = {Ma, Jia-jun},
      author = {Sun, Binyong},
      author = {Zhu, Chen-Bo},
      year={2021},
      eprint={1712.05552},
      archivePrefix={arXiv},
      primaryClass={math.RT}
}

\bib{BV1}{article}{
   author={Barbasch, Dan},
   author={Vogan, David},
   title={Primitive ideals and orbital integrals in complex classical
   groups},
   journal={Math. Ann.},
   volume={259},
   date={1982},
   number={2},
   pages={153--199},
   issn={0025-5831},
   review={\MR{656661}},
   doi={10.1007/BF01457308},
}

\bib{BV2}{article}{
   author={Barbasch, Dan},
   author={Vogan, David},
   title={Primitive ideals and orbital integrals in complex exceptional
   groups},
   journal={J. Algebra},
   volume={80},
   date={1983},
   number={2},
   pages={350--382},
   issn={0021-8693},
   review={\MR{691809}},
   doi={10.1016/0021-8693(83)90006-6},
}

\bib{BV.W}{article}{
  author={Barbasch, Dan},
  author={Vogan, David},
  editor={Trombi, P. C.},
  title={Weyl Group Representations and Nilpotent Orbits},
  bookTitle={Representation Theory of Reductive Groups:
    Proceedings of the University of Utah Conference 1982},
  year={1983},
  publisher={Birkh{\"a}user Boston},
  address={Boston, MA},
  pages={21--33},
  %doi={10.1007/978-1-4684-6730-7_2},
}



\bib{B.Orbit}{article}{
  author = {Barbasch, D.},
  title = {Orbital integrals of nilpotent orbits},
 %booktitle = {The mathematical legacy of {H}arish-{C}handra ({B}altimore,{MD}, 1998)},
    journal = {The mathematical legacy of {H}arish-{C}handra, Proc. Sympos. Pure Math.},
    %series={The mathematical legacy of {H}arish-{C}handra, Proc. Sympos. Pure Math},
    volume = {68},
     pages = {97--110},
 publisher = {Amer. Math. Soc., Providence, RI},
      year = {2000},
}



\bib{B10}{article}{
  author = {Barbasch, D.},
  title = {The unitary spherical spectrum for split classical groups},
  journal = {J. Inst. Math. Jussieu},
% number = {9},
 pages = {265--356},
 volume = {9},
 year = {2010}
}



\bib{B17}{article}{
  author = {Barbasch, D.},
  title = {Unipotent representations and the dual pair correspondence},
  journal = {J. Cogdell et al. (eds.), Representation Theory, Number Theory, and Invariant Theory, In Honor of Roger Howe. Progress in Math.},
  %series ={Progress in Math.},
  volume = {323},
  pages = {47--85},
  year = {2017},
}

\bib{BVUni}{article}{
 author = {Barbasch, D.},
 author = {Vogan, D. A.},
 journal = {Annals of Math.},
 number = {1},
 pages = {41--110},
 title = {Unipotent representations of complex semisimple groups},
 volume = {121},
 year = {1985}
}

\bib{BB}{article}{
   author={Beilinson, Alexandre},
   author={Bernstein, Joseph},
   title={Localisation de $\mathfrak g$-modules},
   language={French, with English summary},
   journal={C. R. Acad. Sci. Paris S\'{e}r. I Math.},
   volume={292},
   date={1981},
   number={1},
   pages={15--18},
   issn={0249-6291},
   review={\MR{610137}},
}

\bib{Br}{article}{
  author = {Brylinski, R.},
  title = {Dixmier algebras for classical complex nilpotent orbits via Kraft-Procesi models. I},
  journal = {The orbit method in geometry and physics (Marseille, 2000). Progress in Math.}
  volume = {213},
  pages = {49--67},
  year = {2003},
}

\bib{BK}{article}{
   author={Brylinski, J.-L.},
   author={Kashiwara, M.},
   title={Kazhdan-Lusztig conjecture and holonomic systems},
   journal={Invent. Math.},
   volume={64},
   date={1981},
   number={3},
   pages={387--410},
   issn={0020-9910},
   review={\MR{632980}},
   doi={10.1007/BF01389272},
}

\bib{Carter}{book}{
   author={Carter, Roger W.},
   title={Finite groups of Lie type},
   series={Wiley Classics Library},
   %note={Conjugacy classes and complex characters;
   %Reprint of the 1985 original;
   %A Wiley-Interscience Publication},
   publisher={John Wiley \& Sons, Ltd., Chichester},
   date={1993},
   pages={xii+544},
   isbn={0-471-94109-3},
   %review={\MR{1266626}},
}

\bib{Cas}{article}{
   author={Casian, Luis G.},
   title={Primitive ideals and representations},
   journal={J. Algebra},
   volume={101},
   date={1986},
   number={2},
   pages={497--515},
   issn={0021-8693},
   review={\MR{847174}},
   doi={10.1016/0021-8693(86)90208-5},
}

\bib{Ca89}{article}{
 author = {Casselman, W.},
 journal = {Canad. J. Math.},
 pages = {385--438},
 title = {Canonical extensions of Harish-Chandra modules to representations of $G$},
 volume = {41},
 year = {1989}
}



\bib{Cl}{article}{
  author = {Du Cloux, F.},
  journal = {Ann. Sci. \'Ecole Norm. Sup.},
  number = {3},
  pages = {257--318},
  title = {Sur les repr\'esentations diff\'erentiables des groupes de Lie alg\'ebriques},
  url = {http://eudml.org/doc/82297},
  volume = {24},
  year = {1991},
}

\bib{CM}{book}{
  title = {Nilpotent orbits in semisimple Lie algebra: an introduction},
  author = {Collingwood, D. H.},
  author = {McGovern, W. M.},
  year = {1993},
  publisher = {Van Nostrand Reinhold Co.},
}


% \bib{Dieu}{book}{
%    title={La g\'{e}om\'{e}trie des groupes classiques},
%    author={Dieudonn\'{e}, Jean},
%    year={1963},
% 	publisher={Springer},
%  }

\bib{DKPC}{article}{
title = {Nilpotent orbits and complex dual pairs},
journal = {J. Algebra},
volume = {190},
number = {2},
pages = {518 - 539},
year = {1997},
author = {Daszkiewicz, A.},
author = {Kra\'skiewicz, W.},
author = {Przebinda, T.},
}

\bib{DKP2}{article}{
  author = {Daszkiewicz, A.},
  author = {Kra\'skiewicz, W.},
  author = {Przebinda, T.},
  title = {Dual pairs and Kostant-Sekiguchi correspondence. II. Classification
	of nilpotent elements},
  journal = {Central European J. Math.},
  year = {2005},
  volume = {3},
  pages = {430--474},
}


\bib{DM}{article}{
  author = {Dixmier, J.},
  author = {Malliavin, P.},
  title = {Factorisations de fonctions et de vecteurs ind\'efiniment diff\'erentiables},
  journal = {Bull. Sci. Math. (2)},
  year = {1978},
  volume = {102},
  pages = {307--330},
}

%\bibitem[DM]{DM}
%J. Dixmier and P. Malliavin, \textit{Factorisations de fonctions et de vecteurs ind\'efiniment diff\'erentiables}, Bull. Sci. Math. (2), 102 (4),  307-330 (1978).



\bib{Du77}{article}{
  author = {Duflo, M.},
  journal = {Annals of Math.},
  number = {1},
  pages = {107-120},
  title = {Sur la Classification des Ideaux Primitifs Dans
    L'algebre Enveloppante d'une Algebre de Lie Semi-Simple},
  volume = {105},
  year = {1977}
}

\bib{Du82}{article}{
 author = {Duflo, M.},
 journal = {Acta Math.},
  volume = {149},
 number = {3-4},
 pages = {153--213},
 title = {Th\'eorie de Mackey pour les groupes de Lie alg\'ebriques},
 year = {1982}
}



\bib{GZ}{article}{
author={Gomez, R.},
author={Zhu, C.-B.},
title={Local theta lifting of generalized Whittaker models associated to nilpotent orbits},
journal={Geom. Funct. Anal.},
year={2014},
volume={24},
number={3},
pages={796--853},
}

\bib{EGAIV2}{article}{
  title = {\'El\'ements de g\'eom\'etrie alg\'brique IV: \'Etude locale des
    sch\'emas et des morphismes de sch\'emas. II},
  author = {Grothendieck, A.},
  author = {Dieudonn\'e, J.},
  journal  = {Inst. Hautes \'Etudes Sci. Publ. Math.},
  volume = {24},
  year = {1965},
}


\bib{EGAIV3}{article}{
  title = {\'El\'ements de g\'eom\'etrie alg\'brique IV: \'Etude locale des
    sch\'emas et des morphismes de sch\'emas. III},
  author = {Grothendieck, A.},
  author = {Dieudonn\'e, J.},
  journal  = {Inst. Hautes \'Etudes Sci. Publ. Math.},
  volume = {28},
  year = {1966},
}


\bib{HLS}{article}{
    author = {Harris, M.},
    author = {Li, J.-S.},
    author = {Sun, B.},
     title = {Theta correspondences for close unitary groups},
 %booktitle = {Arithmetic Geometry and Automorphic Forms},
    %series = {Adv. Lect. Math. (ALM)},
    journal = {Arithmetic Geometry and Automorphic Forms, Adv. Lect. Math. (ALM)},
    volume = {19},
     pages = {265--307},
 publisher = {Int. Press, Somerville, MA},
      year = {2011},
}

\bib{HS}{book}{
 author = {Hartshorne, R.},
 title = {Algebraic Geometry},
publisher={Graduate Texts in Mathematics, 52. New York-Heidelberg-Berlin: Springer-Verlag},
year={1983},
}

\bib{He}{article}{
author={He, H.},
title={Unipotent representations and quantum induction},
journal={arXiv:math/0210372},
year = {2002},
}

\bib{HL}{article}{
author={Huang, J.-S.},
author={Li, J.-S.},
title={Unipotent representations attached to spherical nilpotent orbits},
journal={Amer. J. Math.},
volume={121},
number = {3},
pages={497--517},
year={1999},
}


\bib{HZ}{article}{
author={Huang, J.-S.},
author={Zhu, C.-B.},
title={On certain small representations of indefinite orthogonal groups},
journal={Represent. Theory},
volume={1},
pages={190--206},
year={1997},
}



\bib{Howe79}{article}{
  title={$\theta$-series and invariant theory},
  author={Howe, R.},
  book = {
    title={Automorphic Forms, Representations and $L$-functions},
    series={Proc. Sympos. Pure Math},
    volume={33},
    year={1979},
  },
  pages={275-285},
}

\bib{HoweRank}{article}{
author={Howe, R.},
title={On a notion of rank for unitary representations of the classical groups},
journal={Harmonic analysis and group representations, Liguori, Naples},
pages={223-331},
year={1982},
}

\bib{Howe89}{article}{
author={Howe, R.},
title={Transcending classical invariant theory},
journal={J. Amer. Math. Soc.},
volume={2},
pages={535--552},
year={1989},
}

\bib{Howe95}{article}{,
  author = {Howe, R.},
  title = {Perspectives on invariant theory: Schur duality, multiplicity-free actions and beyond},
  journal = {Piatetski-Shapiro, I. et al. (eds.), The Schur lectures (1992). Ramat-Gan: Bar-Ilan University, Isr. Math. Conf. Proc. 8,},
  year = {1995},
  pages = {1-182},
}

\bib{H}{book}{
   author={Humphreys, James E.},
   title={Representations of semisimple Lie algebras in the BGG category
   $\scr{O}$},
   series={Graduate Studies in Mathematics},
   volume={94},
   publisher={American Mathematical Society, Providence, RI},
   date={2008},
   pages={xvi+289},
   isbn={978-0-8218-4678-0},
   review={\MR{2428237}},
   doi={10.1090/gsm/094},
}

\bib{J1}{article}{
   author={Joseph, A.},
   title={Goldie rank in the enveloping algebra of a semisimple Lie algebra.
   I},
   journal={J. Algebra},
   volume={65},
   date={1980},
   number={2},
   pages={269--283},
   issn={0021-8693},
   review={\MR{585721}},
   doi={10.1016/0021-8693(80)90217-3},
}

\bib{J2}{article}{
   author={Joseph, A.},
   title={Goldie rank in the enveloping algebra of a semisimple Lie algebra.
   II},
   journal={J. Algebra},
   volume={65},
   date={1980},
   number={2},
   pages={284--306},
   issn={0021-8693},
   review={\MR{585721}},
   doi={10.1016/0021-8693(80)90217-3},
}

\bib{J3}{article}{
   author={Joseph, A.},
   title={Goldie rank in the enveloping algebra of a semisimple Lie algebra.
   III},
   journal={J. Algebra},
   volume={73},
   date={1981},
   number={2},
   pages={295--326},
   issn={0021-8693},
   review={\MR{640039}},
   doi={10.1016/0021-8693(81)90324-0},
}
	


\bib{J.av}{article}{
   author={Joseph, Anthony},
   title={On the associated variety of a primitive ideal},
   journal={J. Algebra},
   volume={93},
   date={1985},
   number={2},
   pages={509--523},
   issn={0021-8693},
   review={\MR{786766}},
   doi={10.1016/0021-8693(85)90172-3},
}

\bib{J.ann}{article}{
   author={Joseph, Anthony},
   title={Annihilators and associated varieties of unitary highest weight
   modules},
   journal={Ann. Sci. \'{E}cole Norm. Sup. (4)},
   volume={25},
   date={1992},
   number={1},
   pages={1--45},
   issn={0012-9593},
   review={\MR{1152612}},
}

\bib{J.hw}{article}{
   author={Joseph, Anthony},
   title={On the variety of a highest weight module},
   journal={J. Algebra},
   volume={88},
   date={1984},
   number={1},
   pages={238--278},
   issn={0021-8693},
   review={\MR{741942}},
   doi={10.1016/0021-8693(84)90100-5},
}
	

\bib{JLS}{article}{
author={Jiang, D.},
author={Liu, B.},
author={Savin, G.},
title={Raising nilpotent orbits in wave-front sets},
journal={Represent. Theory},
volume={20},
pages={419--450},
year={2016},
}

\bib{Ki62}{article}{
author={Kirillov, A. A.},
title={Unitary representations of nilpotent Lie groups},
journal={Uspehi Mat. Nauk},
volume={17},
issue ={4},
pages={57--110},
year={1962},
}


\bib{Ko70}{article}{
author={Kostant, B.},
title={Quantization and unitary representations},
journal={Lectures in Modern Analysis and Applications III, Lecture Notes in Math.},
volume={170},
pages={87--208},
year={1970},
}


\bib{KP}{article}{
author={Kraft, H.},
author={Procesi, C.},
title={On the geometry of conjugacy classes in classical groups},
journal={Comment. Math. Helv.},
volume={57},
pages={539--602},
year={1982},
}

\bib{KR}{article}{
author={Kudla, S. S.},
author={Rallis, S.},
title={Degenerate principal series and invariant distributions},
journal={Israel J. Math.},
volume={69},
pages={25--45},
year={1990},
}


\bib{Ku}{article}{
author={Kudla, S. S.},
title={Some extensions of the Siegel-Weil formula},
journal={In: Gan W., Kudla S., Tschinkel Y. (eds) Eisenstein Series and Applications. Progress in Mathematics, vol 258. Birkh\"auser Boston},
%volume={69},
pages={205--237},
year={2008},
}





\bib{LZ1}{article}{
author={Lee, S. T.},
author={Zhu, C.-B.},
title={Degenerate principal series and local theta correspondence II},
journal={Israel J. Math.},
volume={100},
pages={29--59},
year={1997},
}

\bib{LZ2}{article}{
author={Lee, S. T.},
author={Zhu, C.-B.},
title={Degenerate principal series of metaplectic groups and Howe correspondence},
journal = {D. Prasad at al. (eds.), Automorphic Representations and L-Functions, Tata Institute of Fundamental Research, India,},
year = {2013},
pages = {379--408},
}

\bib{Li89}{article}{
author={Li, J.-S.},
title={Singular unitary representations of classical groups},
journal={Invent. Math.},
volume={97},
number = {2},
pages={237--255},
year={1989},
}

\bib{LiuAG}{book}{
  title={Algebraic Geometry and Arithmetic Curves},
  author = {Liu, Q.},
  year = {2006},
  publisher={Oxford University Press},
}

\bib{LM}{article}{
   author = {Loke, H. Y.},
   author = {Ma, J.},
    title = {Invariants and $K$-spectrums of local theta lifts},
    journal = {Compositio Math.},
    volume = {151},
    issue = {01},
    year = {2015},
    pages ={179--206},
}

\bib{DL}{article}{
   author={Deligne, P.},
   author={Lusztig, G.},
   title={Representations of reductive groups over finite fields},
   journal={Ann. of Math. (2)},
   volume={103},
   date={1976},
   number={1},
   pages={103--161},
   issn={0003-486X},
   review={\MR{393266}},
   doi={10.2307/1971021},
}

\bib{KL}{article}{
   author={Kazhdan, David},
   author={Lusztig, George},
   title={Representations of Coxeter groups and Hecke algebras},
   journal={Invent. Math.},
   volume={53},
   date={1979},
   number={2},
   pages={165--184},
   issn={0020-9910},
   review={\MR{560412}},
   doi={10.1007/BF01390031},
}

\bib{Lu}{book}{
   author={Lusztig, George},
   title={Characters of reductive groups over a finite field},
   series={Annals of Mathematics Studies},
   volume={107},
   publisher={Princeton University Press, Princeton, NJ},
   date={1984},
   pages={xxi+384},
   isbn={0-691-08350-9},
   isbn={0-691-08351-7},
   review={\MR{742472}},
   doi={10.1515/9781400881772},
}


\bib{LS}{article}{
   author = {Lusztig, G.},
   author = {Spaltenstein, N.},
    title = {Induced unipotent classes},
    journal = {j. London Math. Soc.},
    volume = {19},
    year = {1979},
    pages ={41--52},
}

\bib{Lu.I}{article}{
   author={Lusztig, G.},
   title={Intersection cohomology complexes on a reductive group},
   journal={Invent. Math.},
   volume={75},
   date={1984},
   number={2},
   pages={205--272},
   issn={0020-9910},
   review={\MR{732546}},
   doi={10.1007/BF01388564},
}
	

\bib{Ma}{article}{
   author = {Mackey, G. W.},
    title = {Unitary representations of group extentions},
    journal = {Acta Math.},
    volume = {99},
    year = {1958},
    pages ={265--311},
}


\bib{Mc}{article}{
   author = {McGovern, W. M},
    title = {Cells of Harish-Chandra modules for real classical groups},
    journal = {Amer. J.  of Math.},
    volume = {120},
    issue = {01},
    year = {1998},
    pages ={211--228},
}

\bib{Mo96}{article}{
 author={M{\oe}glin, C.},
    title = {Front d'onde des repr\'esentations des groupes classiques $p$-adiques},
    journal = {Amer. J. Math.},
    volume = {118},
    issue = {06},
    year = {1996},
    pages ={1313--1346},
}

\bib{Mo17}{article}{
  author={M{\oe}glin, C.},
  title = {Paquets d'Arthur Sp\'eciaux Unipotents aux Places Archim\'ediennes et Correspondance de Howe},
  journal = {J. Cogdell et al. (eds.), Representation Theory, Number Theory, and Invariant Theory, In Honor of Roger Howe. Progress in Math.}
  %series ={Progress in Math.},
  volume = {323},
  pages = {469--502}
  year = {2017}
}

\bib{MR19}{article}{
   author={M\oe glin, Colette},
   author={Renard, David},
   title={Sur les paquets d'Arthur des groupes unitaires et quelques
   cons\'{e}quences pour les groupes classiques},
   language={French, with English and French summaries},
   journal={Pacific J. Math.},
   volume={299},
   date={2019},
   number={1},
   pages={53--88},
   issn={0030-8730},
   review={\MR{3947270}},
   doi={10.2140/pjm.2019.299.53},
}


\bib{MVW}{book}{
  volume={1291},
  title={Correspondances de Howe sur un corps $p$-adique},
  author={M{\oe}glin, C.},
  author={Vign\'eras, M.-F.},
  author={Waldspurger, J.-L.},
  series={Lecture Notes in Mathematics},
  publisher={Springer}
  ISBN={978-3-540-18699-1},
  date={1987},
}

\bib{NOTYK}{article}{
   author = {Nishiyama, K.},
   author = {Ochiai, H.},
   author = {Taniguchi, K.},
   author = {Yamashita, H.},
   author = {Kato, S.},
    title = {Nilpotent orbits, associated cycles and Whittaker models for highest weight representations},
    journal = {Ast\'erisque},
    volume = {273},
    year = {2001},
   pages ={1--163},
}

\bib{NOZ}{article}{
  author = {Nishiyama, K.},
  author = {Ochiai, H.},
  author = {Zhu, C.-B.},
  journal = {Trans. Amer. Math. Soc.},
  title = {Theta lifting of nilpotent orbits for symmetric pairs},
  volume = {358},
  year = {2006},
  pages = {2713--2734},
}


\bib{NZ}{article}{
   author = {Nishiyama, K.},
   author = {Zhu, C.-B.},
    title = {Theta lifting of unitary lowest weight modules and their associated cycles},
    journal = {Duke Math. J.},
    volume = {125},
    number= {03},
    year = {2004},
   pages ={415--465},
}



\bib{Ohta}{article}{
  author = {Ohta, T.},
  %doi = {10.2748/tmj/1178227492},
  journal = {Tohoku Math. J.},
  number = {2},
  pages = {161--211},
  publisher = {Tohoku University, Mathematical Institute},
  title = {The closures of nilpotent orbits in the classical symmetric
    pairs and their singularities},
  volume = {43},
  year = {1991}
}

\bib{Ohta2}{article}{
  author = {Ohta, T.},
  journal = {Hiroshima Math. J.},
  number = {2},
  pages = {347--360},
  title = {Induction of nilpotent orbits for real reductive groups and associated varieties of standard representations},
  volume = {29},
  year = {1999}
}

\bib{Ohta4}{article}{
  title={Nilpotent orbits of $\mathbb{Z}_4$-graded Lie algebra and geometry of
    moment maps associated to the dual pair $(\mathrm{U} (p, q), \mathrm{U} (r, s))$},
  author={Ohta, T.},
  journal={Publ. RIMS},
  volume={41},
  number={3},
  pages={723--756},
  year={2005}
}

\bib{PT}{article}{
  title={Some small unipotent representations of indefinite orthogonal groups and the theta correspondence},
  author={Paul, A.},
  author={Trapa, P.},
  journal={University of Aarhus Publ. Series},
  volume={48},
  pages={103--125},
  year={2007}
}


\bib{PV}{article}{
  title={Invariant Theory},
  author={Popov, V. L.},
  author={Vinberg, E. B.},
  book={
  title={Algebraic Geometry IV: Linear Algebraic Groups, Invariant Theory},
  series={Encyclopedia of Mathematical Sciences},
  volume={55},
  year={1994},
  publisher={Springer},}
}




%\bib{PPz}{article}{
%author={Protsak, V.} ,
%author={Przebinda, T.},
%title={On the occurrence of admissible representations in the real Howe
%    correspondence in stable range},
%journal={Manuscr. Math.},
%volume={126},
%number={2},
%pages={135--141},
%year={2008}
%}


\bib{PrzInf}{article}{
      author={Przebinda, T.},
       title={The duality correspondence of infinitesimal characters},
        date={1996},
     journal={Colloq. Math.},
      volume={70},
       pages={93--102},
}


\bib{Pz}{article}{
author={Przebinda, T.},
title={Characters, dual pairs, and unitary representations},
journal={Duke Math. J. },
volume={69},
number={3},
pages={547--592},
year={1993}
}

\bib{Ra}{article}{
author={Rallis, S.},
title={On the Howe duality conjecture},
journal={Compositio Math.},
volume={51},
pages={333--399},
year={1984}
}

\bib{RT1}{article}{
   author={Renard, David A.},
   author={Trapa, Peter E.},
   title={Irreducible genuine characters of the metaplectic group:
   Kazhdan-Lusztig algorithm and Vogan duality},
   journal={Represent. Theory},
   volume={4},
   date={2000},
   pages={245--295},
   review={\MR{1795754}},
   doi={10.1090/S1088-4165-00-00105-9},
}

\bib{RT2}{article}{
   author={Renard, David A.},
   author={Trapa, Peter E.},
   title={Irreducible characters of the metaplectic group. II.
   Functoriality},
   journal={J. Reine Angew. Math.},
   volume={557},
   date={2003},
   pages={121--158},
   issn={0075-4102},
   review={\MR{1978405}},
   doi={10.1515/crll.2003.028},
}

\bib{RT3}{article}{
   author={Renard, David A.},
   author={Trapa, Peter E.},
   title={Kazhdan-Lusztig algorithms for nonlinear groups and applications
   to Kazhdan-Patterson lifting},
   journal={Amer. J. Math.},
   volume={127},
   date={2005},
   number={5},
   pages={911--971},
   issn={0002-9327},
   review={\MR{2170136}},
}


\bib{Sa}{article}{
author={Sahi, S.},
title={Explicit Hilbert spaces for certain unipotent representations},
journal={Invent. Math.},
volume={110},
number = {2},
pages={409--418},
year={1992}
}

\bib{Se}{article}{
author={Sekiguchi, J.},
title={Remarks on real nilpotent orbits of a symmetric pair},
journal={J. Math. Soc. Japan},
%publisher={The Mathematical Society of Japan},
year={1987},
volume={39},
number={1},
pages={127--138},
}

\bib{SV}{article}{
  author = {Schmid, W.},
  author = {Vilonen, K.},
  journal = {Annals of Math.},
  number = {3},
  pages = {1071--1118},
  %publisher = {Princeton University, Mathematics Department, Princeton, NJ; Mathematical Sciences Publishers, Berkeley},
  title = {Characteristic cycles and wave front cycles of representations of reductive Lie groups},
  volume = {151},
year = {2000},
}


\bib{Soergel}{article}{
   author={Soergel, Wolfgang},
   title={Kategorie $\scr O$, perverse Garben und Moduln \"{u}ber den
   Koinvarianten zur Weylgruppe},
   language={German, with English summary},
   journal={J. Amer. Math. Soc.},
   volume={3},
   date={1990},
   number={2},
   pages={421--445},
   issn={0894-0347},
   review={\MR{1029692}},
   doi={10.2307/1990960},
}


\bib{So}{article}{
author = {Sommers, E.},
title = {Lusztig's canonical quotient and generalized duality},
journal = {J. Algebra},
volume = {243},
number = {2},
pages = {790--812},
year = {2001},
}

\bib{SS}{book}{
  author = {Springer, T. A.},
  author = {Steinberg, R.},
  title = {Seminar on algebraic groups and related finite groups; Conjugate classes},
  series = {Lecture Notes in Math.},
  volume = {131},
publisher={Springer},
year={1970},
}

\bib{SZ1}{article}{
title={A general form of Gelfand-Kazhdan criterion},
author={Sun, B.},
author={Zhu, C.-B.},
journal={Manuscripta Math.},
pages = {185--197},
volume = {136},
year={2011}
}


%\bib{SZ2}{article}{
%  title={Conservation relations for local theta correspondence},
%  author={Sun, B.},
%  author={Zhu, C.-B.},
%  journal={J. Amer. Math. Soc.},
%  pages = {939--983},
%  volume = {28},
%  year={2015}
%}



\bib{Tr}{article}{
  title={Special unipotent representations and the Howe correspondence},
  author={Trapa, P.},
  year = {2004},
  journal={University of Aarhus Publication Series},
  volume = {47},
  pages= {210--230}
}

% \bib{Wa}{article}{
%    author = {Waldspurger, J.-L.},
%     title = {D\'{e}monstration d'une conjecture de dualit\'{e} de Howe dans le cas $p$-adique, $p \neq 2$ in Festschrift in honor of I. I. Piatetski-Shapiro on the occasion of his sixtieth birthday},
%   journal = {Israel Math. Conf. Proc., 2, Weizmann, Jerusalem},
%  year = {1990},
% pages = {267-324},
% }

\bib{VGK}{article}{
   author={Vogan, David A., Jr.},
   title={Gel\cprime fand-Kirillov dimension for Harish-Chandra modules},
   journal={Invent. Math.},
   volume={48},
   date={1978},
   number={1},
   pages={75--98},
   issn={0020-9910},
   review={\MR{506503}},
   doi={10.1007/BF01390063},
}

\bib{Vg}{book}{
   author={Vogan, David A.},
   title={Representations of real reductive Lie groups},
   series={Progress in Mathematics},
   volume={15},
   publisher={Birkh\"{a}user, Boston, Mass.},
   date={1981},
   pages={xvii+754},
   isbn={3-7643-3037-6},
   review={\MR{632407}},
}

\bib{V4}{article}{
   author={Vogan, D. A. },
   title={Irreducible characters of semisimple Lie groups. IV.
   Character-multiplicity duality},
   journal={Duke Math. J.},
   volume={49},
   date={1982},
   number={4},
   pages={943--1073},
   issn={0012-7094},
   review={\MR{683010}},
}

\bib{VoBook}{book}{
author = {Vogan, D. A. },
  title={Unitary representations of reductive Lie groups},
  year={1987},
  series = {Ann. of Math. Stud.},
 volume={118},
  publisher={Princeton University Press}
}


\bib{Vo89}{article}{
  author = {Vogan, D. A. },
  title = {Associated varieties and unipotent representations},
 %booktitle ={Harmonic analysis on reductive groups, Proc. Conf., Brunswick/ME (USA) 1989,},
  journal = {Harmonic analysis on reductive groups, Proc. Conf., Brunswick/ME
    (USA) 1989, Prog. Math.},
 volume={101},
  publisher = {Birkh\"{a}user, Boston-Basel-Berlin},
  year = {1991},
pages={315--388},
  editor = {W. Barker and P. Sally},
}

\bib{Vo98}{article}{
  author = {Vogan, D. A. },
  title = {The method of coadjoint orbits for real reductive groups},
 %booktitle ={Representation theory of Lie groups (Park City, UT, 1998)},
 journal = {Representation theory of Lie groups (Park City, UT, 1998). IAS/Park City Math. Ser.},
  volume={8},
  publisher = {Amer. Math. Soc.},
  year = {2000},
pages={179--238},
}

\bib{Vo00}{article}{
  author = {Vogan, D. A. },
  title = {Unitary representations of reductive Lie groups},
 %booktitle ={Mathematics towards the Third Millennium (Rome, 1999)},
 journal ={Mathematics towards the Third Millennium (Rome, 1999). Accademia Nazionale dei Lincei, (2000)},
  %series = {Accademia Nazionale dei Lincei, 2000},
 %volume={9},
pages={147--167},
}


\bib{Wa1}{book}{
  title={Real reductive groups I},
  author={Wallach, N. R.},
  year={1988},
  publisher={Academic Press Inc. }
}

\bib{Wa2}{book}{
  title={Real reductive groups II},
  author={Wallach, N. R.},
  year={1992},
  publisher={Academic Press Inc. }
}


\bib{Weyl}{book}{
  title={The classical groups: their invariants and representations},
  author={Weyl, H.},
  year={1947},
  publisher={Princeton University Press}
}

\bib{Ya}{article}{
  title={Degenerate principal series representations for quaternionic unitary groups},
  author={Yamana, S.},
  year = {2011},
  journal={Israel J. Math.},
  volume = {185},
  pages= {77--124}
}



% \bib{EGAIV4}{article}{
%   title = {\'El\'ements de g\'eom\'etrie alg\'brique IV 4: \'Etude locale des
%     sch\'emas et des morphismes de sch\'emas},
%   author = {Grothendieck, Alexandre},
%   author = {Dieudonn\'e, Jean},
%   journal  = {Inst. Hautes \'Etudes Sci. Publ. Math.},
%   volume = {32},
%   year = {1967},
%   pages = {5--361}
% }



\end{biblist}
\end{bibdiv}


\end{document}


%%% Local Variables:
%%% coding: utf-8
%%% mode: latex
%%% TeX-engine: tex
%%% ispell-local-dictionary: "en_US"
%%% End:
