% !TeX program = xelatex
\documentclass[12pt,a4paper]{amsart}
\usepackage[margin=2.5cm,marginpar=2cm]{geometry}

\usepackage[bookmarksopen,bookmarksdepth=2,hidelinks,colorlinks=false]{hyperref}
\usepackage[nameinlink]{cleveref}

% \usepackage[color]{showkeys}
% \makeatletter
%   \SK@def\Cref#1{\SK@\SK@@ref{#1}\SK@Cref{#1}}%
% \makeatother
%% FONTS

\usepackage{amssymb}
%\usepackage{amsmath}
\usepackage{mathrsfs}
\usepackage{mathtools}
%\usepackage{amsrefs}
%\usepackage{mathbbol,mathabx}
\usepackage{amsthm}
\usepackage{graphicx}
\usepackage{braket}
%\usepackage[pointedenum]{paralist}
%\usepackage{paralist}


\usepackage{amsrefs}

\usepackage[all,cmtip]{xy}
\usepackage{rotating}
\usepackage{leftidx}
%\usepackage{arydshln}

%\DeclareSymbolFont{bbold}{U}{bbold}{m}{n}
%\DeclareSymbolFontAlphabet{\mathbbold}{bbold}


%\usepackage[dvipdfx,rgb,table]{xcolor}
\usepackage[rgb,table]{xcolor}
%\usepackage{mathrsfs}

\setcounter{tocdepth}{1}
\setcounter{secnumdepth}{2}

%\usepackage[abbrev,shortalphabetic]{amsrefs}


\usepackage[normalem]{ulem}

% circled number
\usepackage{pifont}
\makeatletter
\newcommand*{\circnuma}[1]{%
  \ifnum#1<1 %
    \@ctrerr
  \else
    \ifnum#1>20 %
      \@ctrerr
    \else
      \mbox{\ding{\numexpr 171+(#1)\relax}}%
     \fi
  \fi
}
\makeatother

\usepackage[centertableaux]{ytableau}


% Ytableau tweak
\makeatletter
\pgfkeys{/ytableau/options,
  noframe/.default = false,
  noframe/.is choice,
  noframe/true/.code = {%
    \global\let\vrule@YT=\vrule@none@YT
    \global\let\hrule@YT=\hrule@none@YT
  },
  noframe/false/.code = {%
    \global\let\vrule@YT=\vrule@normal@YT
    \global\let\hrule@YT=\hrule@normal@YT
  },
  noframe/on/.style = {noframe/true},
  noframe/off/.style = {noframe/false},
}

\def\hrule@enon@YT{%
  \hrule width  \dimexpr \boxdim@YT + \fboxrule *2 \relax
  height 0pt
}
\def\vrule@enon@YT{%
  \vrule height \dimexpr  \boxdim@YT + \fboxrule\relax
     width \fboxrule
}

\def\enon{\omit\enon@YT}
\newcommand{\enon@YT}[2][clear]{%
  \def\thisboxcolor@YT{#1}%
  \let\hrule@YT=\hrule@enon@YT
  \let\vrule@YT=\vrule@enon@YT
  \startbox@@YT#2\endbox@YT
  \nullfont
}

\makeatother
%\ytableausetup{noframe=on,smalltableaux}
\ytableausetup{noframe=off,boxsize=1.3em}
\let\ytb=\ytableaushort

\newcommand{\tytb}[1]{{\tiny\ytb{#1}}}


%\usepackage[mathlines,pagewise]{lineno}
%\linenumbers

\usepackage{enumitem}
%% Enumitem
\newlist{enumC}{enumerate}{1} % Conditions in Lemma/Theorem/Prop
\setlist[enumC,1]{label=(\alph*),wide,ref=(\alph*)}
\crefname{enumCi}{condition}{conditions}
\Crefname{enumCi}{Condition}{Conditions}
\newlist{enumT}{enumerate}{3} % "Theorem"=conclusions in Lemma/Theorem/Prop
\setlist[enumT]{label=(\roman*),wide}
\setlist[enumT,1]{label=(\roman*),wide}
\setlist[enumT,2]{label=(\alph*),ref ={(\roman{enumTi}.\alph*)}}
\setlist[enumT,3]{label=(\arabic*), ref ={(\roman{enumTi}.\alph{enumTii}.\alph*)}}
\crefname{enumTi}{}{}
\Crefname{enumTi}{Item}{Items}
\crefname{enumTii}{}{}
\Crefname{enumTii}{Item}{Items}
\crefname{enumTiii}{}{}
\Crefname{enumTiii}{Item}{Items}
\newlist{enumPF}{enumerate}{3}
\setlist[enumPF]{label=(\alph*),wide}
\setlist[enumPF,1]{label=(\roman*),wide}
\setlist[enumPF,2]{label=(\alph*)}
\setlist[enumPF,3]{label=\arabic*).}
\newlist{enumS}{enumerate}{3} % Statement outside Lemma/Theorem/Prop
\setlist[enumS]{label=\roman*)}
\setlist[enumS,1]{label=\roman*)}
\setlist[enumS,2]{label=\alph*)}
\setlist[enumS,3]{label=\arabic*.}
\newlist{enumI}{enumerate}{3} % items
\setlist[enumI,1]{label=\roman*),leftmargin=*}
\setlist[enumI,2]{label=\alph*), leftmargin=*}
\setlist[enumI,3]{label=\arabic*), leftmargin=*}
\newlist{enumIL}{enumerate*}{1} % inline enum
\setlist*[enumIL]{label=\roman*)}
\newlist{enumR}{enumerate}{1} % remarks
\setlist[enumR]{label=\arabic*.,wide,labelwidth=!, labelindent=0pt}
\crefname{enumRi}{remark}{remarks}

\crefname{equation}{}{}
\Crefname{equation}{Equation}{Equations}
\Crefname{lem}{Lemma}{Lemma}
\Crefname{thm}{Theorem}{Theorem}

\newlist{des}{description}{1}
\setlist[des]{font=\sffamily\bfseries}

% editing macros.
\blendcolors{!80!black}
\long\def\okay#1{\ifcsname highlightokay\endcsname
{\color{red} #1}
\else
{#1}
\fi
}
\long\def\editc#1{{\color{red} #1}}
\long\def\mjj#1{{{\color{blue}#1}}}
\long\def\mjjr#1{{\color{red} (#1)}}
\long\def\mjjd#1#2{{\color{blue} #1 \sout{#2}}}
\def\mjjb{\color{blue}}
\def\mjje{\color{black}}
\def\mjjcb{\color{green!50!black}}
\def\mjjce{\color{black}}

\long\def\sun#1{{{\color{cyan}#1}}}
\long\def\sund#1#2{{\color{cyan}#1  \sout{#2}}}
\long\def\mv#1{{{\color{red} {\bf move to a proper place:} #1}}}
\long\def\delete#1{}

%\reversemarginpar
\newcommand{\lokec}[1]{\marginpar{\color{blue}\tiny #1 \mbox{--loke}}}
\newcommand{\mjjc}[1]{\marginpar{\color{green}\tiny #1 \mbox{--ma}}}

\newcommand{\trivial}[2][]{\if\relax\detokenize{#1}\relax
  {%\hfill\break
   % \begin{minipage}{\textwidth}
      \color{orange} \vspace{0em} $[$  #2 $]$
  %\end{minipage}
  %\break
      \color{black}
  }
  \else
\ifx#1h
\ifcsname showtrivial\endcsname
{%\hfill\break
 % \begin{minipage}{\textwidth}
    \color{orange} \vspace{0em}  $[$ #2 $]$
%\end{minipage}
%\break
    \color{black}
}
\fi
\else {\red Wrong argument!} \fi
\fi
}

\newcommand{\byhide}[2][]{\if\relax\detokenize{#1}\relax
{\color{orange} \vspace{0em} Plan to delete:  #2}
\else
\ifx#1h\relax\fi
\fi
}



\newcommand{\Rank}{\mathrm{rk}}
\newcommand{\cqq}{\mathscr{D}}
\newcommand{\rsym}{\mathrm{sym}}
\newcommand{\rskew}{\mathrm{skew}}
\newcommand{\fraksp}{\mathfrak{sp}}
\newcommand{\frakso}{\mathfrak{so}}
\newcommand{\frakm}{\mathfrak{m}}
\newcommand{\frakp}{\mathfrak{p}}
\newcommand{\pr}{\mathrm{pr}}
\newcommand{\rhopst}{\rho'^*}
\newcommand{\Rad}{\mathrm{Rad}}
\newcommand{\Res}{\mathrm{Res}}
\newcommand{\Hol}{\mathrm{Hol}}
\newcommand{\AC}{\mathrm{AC}}
%\newcommand{\AS}{\mathrm{AS}}
\newcommand{\WF}{\mathrm{WF}}
\newcommand{\AV}{\mathrm{AV}}
\newcommand{\AVC}{\mathrm{AV}_\bC}
\newcommand{\VC}{\mathrm{V}_\bC}
\newcommand{\bfv}{\mathbf{v}}
\newcommand{\depth}{\mathrm{depth}}
\newcommand{\wtM}{\widetilde{M}}
\newcommand{\wtMone}{{\widetilde{M}^{(1,1)}}}

\newcommand{\nullpp}{N(\fpp'^*)}
\newcommand{\nullp}{N(\fpp^*)}
%\newcommand{\Aut}{\mathrm{Aut}}

\def\mstar{{\medstar}}


\newcommand{\bfone}{\mathbf{1}}
\newcommand{\piSigma}{\pi_\Sigma}
\newcommand{\piSigmap}{\pi'_\Sigma}


\newcommand{\sfVprime}{\mathsf{V}^\prime}
\newcommand{\sfVdprime}{\mathsf{V}^{\prime \prime}}
\newcommand{\gminusone}{\mathfrak{g}_{-\frac{1}{m}}}

\newcommand{\eva}{\mathrm{eva}}

% \newcommand\iso{\xrightarrow{
%    \,\smash{\raisebox{-0.65ex}{\ensuremath{\scriptstyle\sim}}}\,}}

\def\Ueven{{U_{\rm{even}}}}
\def\Uodd{{U_{\rm{odd}}}}
\def\ttau{\tilde{\tau}}
\def\Wcp{W}
\def\Kur{{K^{\mathrm{u}}}}

\def\Im{\operatorname{Im}}

\providecommand{\bcN}{{\overline{\cN}}}



\makeatletter

\def\gen#1{\left\langle
    #1
      \right\rangle}
\makeatother

\makeatletter
\def\inn#1#2{\left\langle
      \def\ta{#1}\def\tb{#2}
      \ifx\ta\@empty{\;} \else {\ta}\fi ,
      \ifx\tb\@empty{\;} \else {\tb}\fi
      \right\rangle}
\def\binn#1#2{\left\lAngle
      \def\ta{#1}\def\tb{#2}
      \ifx\ta\@empty{\;} \else {\ta}\fi ,
      \ifx\tb\@empty{\;} \else {\tb}\fi
      \right\rAngle}
\makeatother

\makeatletter
\def\binn#1#2{\overline{\inn{#1}{#2}}}
\makeatother


\def\innwi#1#2{\inn{#1}{#2}_{W_i}}
\def\innw#1#2{\inn{#1}{#2}_{\bfW}}
\def\innv#1#2{\inn{#1}{#2}_{\bfV}}
\def\innbfv#1#2{\inn{#1}{#2}_{\bfV}}
\def\innvi#1#2{\inn{#1}{#2}_{V_i}}
\def\innvp#1#2{\inn{#1}{#2}_{\bfV'}}
\def\innp#1#2{\inn{#1}{#2}'}

% choose one of then
\def\simrightarrow{\iso}
\def\surj{\twoheadrightarrow}
%\def\simrightarrow{\xrightarrow{\sim}}

\newcommand\iso{\xrightarrow{
   \,\smash{\raisebox{-0.65ex}{\ensuremath{\scriptstyle\sim}}}\,}}

\newcommand\riso{\xleftarrow{
   \,\smash{\raisebox{-0.65ex}{\ensuremath{\scriptstyle\sim}}}\,}}









\usepackage{xparse}
\def\usecsname#1{\csname #1\endcsname}
\def\useLetter#1{#1}
\def\usedbletter#1{#1#1}

% \def\useCSf#1{\csname f#1\endcsname}

\ExplSyntaxOn

\def\mydefcirc#1#2#3{\expandafter\def\csname
  circ#3{#1}\endcsname{{}^\circ {#2{#1}}}}
\def\mydefvec#1#2#3{\expandafter\def\csname
  vec#3{#1}\endcsname{\vec{#2{#1}}}}
\def\mydefdot#1#2#3{\expandafter\def\csname
  dot#3{#1}\endcsname{\dot{#2{#1}}}}

\def\mydefacute#1#2#3{\expandafter\def\csname a#3{#1}\endcsname{\acute{#2{#1}}}}
\def\mydefbr#1#2#3{\expandafter\def\csname br#3{#1}\endcsname{\breve{#2{#1}}}}
\def\mydefbar#1#2#3{\expandafter\def\csname bar#3{#1}\endcsname{\bar{#2{#1}}}}
\def\mydefhat#1#2#3{\expandafter\def\csname hat#3{#1}\endcsname{\hat{#2{#1}}}}
\def\mydefwh#1#2#3{\expandafter\def\csname wh#3{#1}\endcsname{\widehat{#2{#1}}}}
\def\mydeft#1#2#3{\expandafter\def\csname t#3{#1}\endcsname{\tilde{#2{#1}}}}
\def\mydefu#1#2#3{\expandafter\def\csname u#3{#1}\endcsname{\underline{#2{#1}}}}
\def\mydefr#1#2#3{\expandafter\def\csname r#3{#1}\endcsname{\mathrm{#2{#1}}}}
\def\mydefb#1#2#3{\expandafter\def\csname b#3{#1}\endcsname{\mathbb{#2{#1}}}}
\def\mydefwt#1#2#3{\expandafter\def\csname wt#3{#1}\endcsname{\widetilde{#2{#1}}}}
%\def\mydeff#1#2#3{\expandafter\def\csname f#3{#1}\endcsname{\mathfrak{#2{#1}}}}
\def\mydefbf#1#2#3{\expandafter\def\csname bf#3{#1}\endcsname{\mathbf{#2{#1}}}}
\def\mydefc#1#2#3{\expandafter\def\csname c#3{#1}\endcsname{\mathcal{#2{#1}}}}
\def\mydefsf#1#2#3{\expandafter\def\csname sf#3{#1}\endcsname{\mathsf{#2{#1}}}}
\def\mydefs#1#2#3{\expandafter\def\csname s#3{#1}\endcsname{\mathscr{#2{#1}}}}
\def\mydefcks#1#2#3{\expandafter\def\csname cks#3{#1}\endcsname{{\check{
        \csname s#2{#1}\endcsname}}}}
\def\mydefckc#1#2#3{\expandafter\def\csname ckc#3{#1}\endcsname{{\check{
      \csname c#2{#1}\endcsname}}}}
\def\mydefck#1#2#3{\expandafter\def\csname ck#3{#1}\endcsname{{\check{#2{#1}}}}}

\cs_new:Npn \mydeff #1#2#3 {\cs_new:cpn {f#3{#1}} {\mathfrak{#2{#1}}}}

\cs_new:Npn \doGreek #1
{
  \clist_map_inline:nn {alpha,beta,gamma,Gamma,delta,Delta,epsilon,varepsilon,zeta,eta,theta,vartheta,Theta,iota,kappa,lambda,Lambda,mu,nu,xi,Xi,pi,Pi,rho,sigma,varsigma,Sigma,tau,upsilon,Upsilon,phi,varphi,Phi,chi,psi,Psi,omega,Omega,tG} {#1{##1}{\usecsname}{\useLetter}}
}

\cs_new:Npn \doSymbols #1
{
  \clist_map_inline:nn {otimes,boxtimes} {#1{##1}{\usecsname}{\useLetter}}
}

\cs_new:Npn \doAtZ #1
{
  \clist_map_inline:nn {A,B,C,D,E,F,G,H,I,J,K,L,M,N,O,P,Q,R,S,T,U,V,W,X,Y,Z} {#1{##1}{\useLetter}{\useLetter}}
}

\cs_new:Npn \doatz #1
{
  \clist_map_inline:nn {a,b,c,d,e,f,g,h,i,j,k,l,m,n,o,p,q,r,s,t,u,v,w,x,y,z} {#1{##1}{\useLetter}{\usedbletter}}
}

\cs_new:Npn \doallAtZ
{
\clist_map_inline:nn {mydefsf,mydeft,mydefu,mydefwh,mydefhat,mydefr,mydefwt,mydeff,mydefb,mydefbf,mydefc,mydefs,mydefck,mydefcks,mydefckc,mydefbar,mydefvec,mydefcirc,mydefdot,mydefbr,mydefacute} {\doAtZ{\csname ##1\endcsname}}
}

\cs_new:Npn \doallatz
{
\clist_map_inline:nn {mydefsf,mydeft,mydefu,mydefwh,mydefhat,mydefr,mydefwt,mydeff,mydefb,mydefbf,mydefc,mydefs,mydefck,mydefbar,mydefvec,mydefdot,mydefbr,mydefacute} {\doatz{\csname ##1\endcsname}}
}

\cs_new:Npn \doallGreek
{
\clist_map_inline:nn {mydefck,mydefwt,mydeft,mydefwh,mydefbar,mydefu,mydefvec,mydefcirc,mydefdot,mydefbr,mydefacute} {\doGreek{\csname ##1\endcsname}}
}

\cs_new:Npn \doallSymbols
{
\clist_map_inline:nn {mydefck,mydefwt,mydeft,mydefwh,mydefbar,mydefu,mydefvec,mydefcirc,mydefdot} {\doSymbols{\csname ##1\endcsname}}
}



\cs_new:Npn \doGroups #1
{
  \clist_map_inline:nn {GL,Sp,rO,rU,fgl,fsp,foo,fuu,fkk,fuu,ufkk,uK} {#1{##1}{\usecsname}{\useLetter}}
}

\cs_new:Npn \doallGroups
{
\clist_map_inline:nn {mydeft,mydefu,mydefwh,mydefhat,mydefwt,mydefck,mydefbar} {\doGroups{\csname ##1\endcsname}}
}


\cs_new:Npn \decsyms #1
{
\clist_map_inline:nn {#1} {\expandafter\DeclareMathOperator\csname ##1\endcsname{##1}}
}

\decsyms{Mp,id,SL,Sp,SU,SO,GO,GSO,GU,GSp,PGL,Pic,Lie,Mat,Ker,Hom,Ext,Ind,reg,res,inv,Isom,Det,Tr,Norm,Sym,Span,Stab,Spec,PGSp,PSL,tr,Ad,Br,Ch,Cent,End,Aut,Dvi,Frob,Gal,GL,Gr,DO,ur,vol,ab,Nil,Supp,rank,Sign}

\def\abs#1{\left|{#1}\right|}
\def\norm#1{{\left\|{#1}\right\|}}


% \NewDocumentCommand\inn{m m}{
% \left\langle
% \IfValueTF{#1}{#1}{000}
% ,
% \IfValueTF{#2}{#2}{000}
% \right\rangle
% }
\NewDocumentCommand\cent{o m }{
  \IfValueTF{#1}{
    \mathop{Z}_{#1}{(#2)}}
  {\mathop{Z}{(#2)}}
}


\def\fsl{\mathfrak{sl}}
\def\fsp{\mathfrak{sp}}


%\def\cent#1#2{{\mathrm{Z}_{#1}({#2})}}


\doallAtZ
\doallatz
\doallGreek
\doallGroups
\doallSymbols
\ExplSyntaxOff


% \usepackage{geometry,amsthm,graphics,tabularx,amssymb,shapepar}
% \usepackage{amscd}
% \usepackage{mathrsfs}


\usepackage{diagbox}
% Update the information and uncomment if AMS is not the copyright
% holder.
%\copyrightinfo{2006}{American Mathematical Society}
%\usepackage{nicematrix}
\usepackage{arydshln}

\usepackage{tikz}
\usetikzlibrary{matrix,arrows,positioning,cd,backgrounds}
\usetikzlibrary{decorations.pathmorphing,decorations.pathreplacing}

\usepackage{upgreek}

\usepackage{listings}
\lstset{
    basicstyle=\ttfamily\tiny,
    keywordstyle=\color{black},
    commentstyle=\color{white}, % white comments
    stringstyle=\ttfamily, % typewriter type for strings
    showstringspaces=false,
    breaklines=true,
    emph={Output},emphstyle=\color{blue},
} 

\newcommand{\BA}{{\mathbb{A}}}
%\newcommand{\BB}{{\mathbb {B}}}
\newcommand{\BC}{{\mathbb {C}}}
\newcommand{\BD}{{\mathbb {D}}}
\newcommand{\BE}{{\mathbb {E}}}
\newcommand{\BF}{{\mathbb {F}}}
\newcommand{\BG}{{\mathbb {G}}}
\newcommand{\BH}{{\mathbb {H}}}
\newcommand{\BI}{{\mathbb {I}}}
\newcommand{\BJ}{{\mathbb {J}}}
\newcommand{\BK}{{\mathbb {U}}}
\newcommand{\BL}{{\mathbb {L}}}
\newcommand{\BM}{{\mathbb {M}}}
\newcommand{\BN}{{\mathbb {N}}}
\newcommand{\BO}{{\mathbb {O}}}
\newcommand{\BP}{{\mathbb {P}}}
\newcommand{\BQ}{{\mathbb {Q}}}
\newcommand{\BR}{{\mathbb {R}}}
\newcommand{\BS}{{\mathbb {S}}}
\newcommand{\BT}{{\mathbb {T}}}
\newcommand{\BU}{{\mathbb {U}}}
\newcommand{\BV}{{\mathbb {V}}}
\newcommand{\BW}{{\mathbb {W}}}
\newcommand{\BX}{{\mathbb {X}}}
\newcommand{\BY}{{\mathbb {Y}}}
\newcommand{\BZ}{{\mathbb {Z}}}
\newcommand{\Bk}{{\mathbf {k}}}

\newcommand{\CA}{{\mathcal {A}}}
\newcommand{\CB}{{\mathcal {B}}}
\newcommand{\CC}{{\mathcal {C}}}

\newcommand{\CE}{{\mathcal {E}}}
\newcommand{\CF}{{\mathcal {F}}}
\newcommand{\CG}{{\mathcal {G}}}
\newcommand{\CH}{{\mathcal {H}}}
\newcommand{\CI}{{\mathcal {I}}}
\newcommand{\CJ}{{\mathcal {J}}}
\newcommand{\CK}{{\mathcal {K}}}
\newcommand{\CL}{{\mathcal {L}}}
\newcommand{\CM}{{\mathcal {M}}}
\newcommand{\CN}{{\mathcal {N}}}
\newcommand{\CO}{{\mathcal {O}}}
\newcommand{\CP}{{\mathcal {P}}}
\newcommand{\CQ}{{\mathcal {Q}}}
\newcommand{\CR}{{\mathcal {R}}}
\newcommand{\CS}{{\mathcal {S}}}
\newcommand{\CT}{{\mathcal {T}}}
\newcommand{\CU}{{\mathcal {U}}}
\newcommand{\CV}{{\mathcal {V}}}
\newcommand{\CW}{{\mathcal {W}}}
\newcommand{\CX}{{\mathcal {X}}}
\newcommand{\CY}{{\mathcal {Y}}}
\newcommand{\CZ}{{\mathcal {Z}}}


\newcommand{\RA}{{\mathrm {A}}}
\newcommand{\RB}{{\mathrm {B}}}
\newcommand{\RC}{{\mathrm {C}}}
\newcommand{\RD}{{\mathrm {D}}}
\newcommand{\RE}{{\mathrm {E}}}
\newcommand{\RF}{{\mathrm {F}}}
\newcommand{\RG}{{\mathrm {G}}}
\newcommand{\RH}{{\mathrm {H}}}
\newcommand{\RI}{{\mathrm {I}}}
\newcommand{\RJ}{{\mathrm {J}}}
\newcommand{\RK}{{\mathrm {K}}}
\newcommand{\RL}{{\mathrm {L}}}
\newcommand{\RM}{{\mathrm {M}}}
\newcommand{\RN}{{\mathrm {N}}}
\newcommand{\RO}{{\mathrm {O}}}
\newcommand{\RP}{{\mathrm {P}}}
\newcommand{\RQ}{{\mathrm {Q}}}
%\newcommand{\RR}{{\mathrm {R}}}
\newcommand{\RS}{{\mathrm {S}}}
\newcommand{\RT}{{\mathrm {T}}}
\newcommand{\RU}{{\mathrm {U}}}
\newcommand{\RV}{{\mathrm {V}}}
\newcommand{\RW}{{\mathrm {W}}}
\newcommand{\RX}{{\mathrm {X}}}
\newcommand{\RY}{{\mathrm {Y}}}
\newcommand{\RZ}{{\mathrm {Z}}}

\DeclareMathOperator{\absNorm}{\mathfrak{N}}
\DeclareMathOperator{\Ann}{Ann}
\DeclareMathOperator{\LAnn}{L-Ann}
\DeclareMathOperator{\RAnn}{R-Ann}
\DeclareMathOperator{\ind}{ind}
%\DeclareMathOperator{\Ind}{Ind}



\newcommand{\cod}{{\mathrm{cod}}}
\newcommand{\cont}{{\mathrm{cont}}}
\newcommand{\cl}{{\mathrm{cl}}}
\newcommand{\cusp}{{\mathrm{cusp}}}

\newcommand{\disc}{{\mathrm{disc}}}
\renewcommand{\div}{{\mathrm{div}}}



\newcommand{\Gm}{{\mathbb{G}_m}}



\newcommand{\I}{{\mathrm{I}}}

\newcommand{\Jac}{{\mathrm{Jac}}}
\newcommand{\PM}{{\mathrm{PM}}}


\newcommand{\new}{{\mathrm{new}}}
\newcommand{\NS}{{\mathrm{NS}}}
\newcommand{\N}{{\mathrm{N}}}

\newcommand{\ord}{{\mathrm{ord}}}

%\newcommand{\rank}{{\mathrm{rank}}}

\newcommand{\rk}{{\mathrm{k}}}
\newcommand{\rr}{{\mathrm{r}}}
\newcommand{\rh}{{\mathrm{h}}}

\newcommand{\Sel}{{\mathrm{Sel}}}
\newcommand{\Sim}{{\mathrm{Sim}}}

\newcommand{\wt}{\widetilde}
\newcommand{\wh}{\widehat}
\newcommand{\pp}{\frac{\partial\bar\partial}{\pi i}}
\newcommand{\pair}[1]{\langle {#1} \rangle}
\newcommand{\wpair}[1]{\left\{{#1}\right\}}
\newcommand{\intn}[1]{\left( {#1} \right)}
\newcommand{\sfrac}[2]{\left( \frac {#1}{#2}\right)}
\newcommand{\ds}{\displaystyle}
\newcommand{\ov}{\overline}
\newcommand{\incl}{\hookrightarrow}
\newcommand{\lra}{\longrightarrow}
\newcommand{\imp}{\Longrightarrow}
%\newcommand{\lto}{\longmapsto}
\newcommand{\bs}{\backslash}

\newcommand{\cover}[1]{\widetilde{#1}}

\renewcommand{\vsp}{{\vspace{0.2in}}}

\newcommand{\Norma}{\operatorname{N}}
\newcommand{\Ima}{\operatorname{Im}}
\newcommand{\con}{\textit{C}}
\newcommand{\gr}{\operatorname{gr}}
\newcommand{\ad}{\operatorname{ad}}
\newcommand{\der}{\operatorname{der}}
\newcommand{\dif}{\operatorname{d}\!}
\newcommand{\pro}{\operatorname{pro}}
\newcommand{\Ev}{\operatorname{Ev}}
% \renewcommand{\span}{\operatorname{span}} \span is an innernal command.
%\newcommand{\degree}{\operatorname{deg}}
\newcommand{\Invf}{\operatorname{Invf}}
\newcommand{\Inv}{\operatorname{Inv}}
\newcommand{\slt}{\operatorname{SL}_2(\mathbb{R})}
%\newcommand{\temp}{\operatorname{temp}}
%\newcommand{\otop}{\operatorname{top}}
\renewcommand{\small}{\operatorname{small}}
\newcommand{\HC}{\operatorname{HC}}
\newcommand{\lef}{\operatorname{left}}
\newcommand{\righ}{\operatorname{right}}
\newcommand{\Diff}{\operatorname{DO}}
\newcommand{\diag}{\operatorname{diag}}
\newcommand{\sh}{\varsigma}
\newcommand{\sch}{\operatorname{sch}}
%\newcommand{\oleft}{\operatorname{left}}
%\newcommand{\oright}{\operatorname{right}}
\newcommand{\open}{\operatorname{open}}
\newcommand{\sgn}{\operatorname{sgn}}
\newcommand{\triv}{\operatorname{triv}}
\newcommand{\Sh}{\operatorname{Sh}}
\newcommand{\oN}{\operatorname{N}}

\newcommand{\oc}{\operatorname{c}}
\newcommand{\od}{\operatorname{d}}
\newcommand{\os}{\operatorname{s}}
\newcommand{\ol}{\operatorname{l}}
\newcommand{\oL}{\operatorname{L}}
\newcommand{\oJ}{\operatorname{J}}
\newcommand{\oH}{\operatorname{H}}
\newcommand{\oO}{\operatorname{O}}
\newcommand{\oS}{\operatorname{S}}
\newcommand{\oR}{\operatorname{R}}
\newcommand{\oT}{\operatorname{T}}
%\newcommand{\rU}{\operatorname{U}}
\newcommand{\oZ}{\operatorname{Z}}
\newcommand{\oD}{\textit{D}}
\newcommand{\oW}{\textit{W}}
\newcommand{\oE}{\operatorname{E}}
\newcommand{\oP}{\operatorname{P}}
\newcommand{\PD}{\operatorname{PD}}
\newcommand{\oU}{\operatorname{U}}

\newcommand{\g}{\mathfrak g}
\newcommand{\gC}{{\mathfrak g}_{\C}}
\renewcommand{\k}{\mathfrak k}
\newcommand{\h}{\mathfrak h}
\newcommand{\p}{\mathfrak p}
%\newcommand{\q}{\mathfrak q}
\renewcommand{\a}{\mathfrak a}
\renewcommand{\b}{\mathfrak b}
\renewcommand{\c}{\mathfrak c}
\newcommand{\n}{\mathfrak n}
\renewcommand{\u}{\mathfrak u}
\renewcommand{\v}{\mathfrak v}
\newcommand{\e}{\mathfrak e}
\newcommand{\f}{\mathfrak f}
\renewcommand{\l}{\mathfrak l}
\renewcommand{\t}{\mathfrak t}
\newcommand{\s}{\mathfrak s}
\renewcommand{\r}{\mathfrak r}
\renewcommand{\o}{\mathfrak o}
\newcommand{\m}{\mathfrak m}
\newcommand{\z}{\mathfrak z}
%\renewcommand{\sl}{\mathfrak s \mathfrak l}
\newcommand{\gl}{\mathfrak g \mathfrak l}


\newcommand{\re}{\mathrm e}

\renewcommand{\rk}{\mathrm k}

\newcommand{\Z}{\mathbb{Z}}
\DeclareDocumentCommand{\C}{}{\mathbb{C}}
\newcommand{\R}{\mathbb R}
\newcommand{\Q}{\mathbb Q}
\renewcommand{\H}{\mathbb{H}}
%\newcommand{\N}{\mathbb{N}}
\newcommand{\K}{\mathbb{K}}
%\renewcommand{\S}{\mathbf S}
\newcommand{\M}{\mathbf{M}}
\newcommand{\A}{\mathbb{A}}
\newcommand{\B}{\mathbf{B}}
%\renewcommand{\G}{\mathbf{G}}
\newcommand{\V}{\mathbf{V}}
\newcommand{\W}{\mathbf{W}}
\newcommand{\F}{\mathbf{F}}
\newcommand{\E}{\mathbf{E}}
%\newcommand{\J}{\mathbf{J}}
\renewcommand{\H}{\mathbf{H}}
\newcommand{\X}{\mathbf{X}}
\newcommand{\Y}{\mathbf{Y}}
%\newcommand{\RR}{\mathcal R}
\newcommand{\FF}{\mathcal F}
%\newcommand{\BB}{\mathcal B}
\newcommand{\HH}{\mathcal H}
%\newcommand{\UU}{\mathcal U}
%\newcommand{\MM}{\mathcal M}
%\newcommand{\CC}{\mathcal C}
%\newcommand{\DD}{\mathcal D}
\def\eDD{\mathrm{d}^{e}}
\def\DD{\nabla}
\def\DDc{\boldsymbol{\nabla}}
\def\gDD{\nabla^{\mathrm{gen}}}
\def\gDDc{\boldsymbol{\nabla}^{\mathrm{gen}}}
%\newcommand{\OO}{\mathcal O}
%\newcommand{\ZZ}{\mathcal Z}
\newcommand{\ve}{{\vee}}
\newcommand{\aut}{\mathcal A}
\newcommand{\ii}{\mathbf{i}}
\newcommand{\jj}{\mathbf{j}}
\newcommand{\kk}{\mathbf{k}}

\newcommand{\la}{\langle}
\newcommand{\ra}{\rangle}
\newcommand{\bp}{\bigskip}
\newcommand{\be}{\begin {equation}}
\newcommand{\ee}{\end {equation}}

\newcommand{\LRleq}{\stackrel{LR}{\leq}}

\numberwithin{equation}{section}


\def\flushl#1{\ifmmode\makebox[0pt][l]{${#1}$}\else\makebox[0pt][l]{#1}\fi}
\def\flushr#1{\ifmmode\makebox[0pt][r]{${#1}$}\else\makebox[0pt][r]{#1}\fi}
\def\flushmr#1{\makebox[0pt][r]{${#1}$}}


%\theoremstyle{Theorem}
% \newtheorem*{thmM}{Main Theorem}
% \crefformat{thmM}{main theorem}
% \Crefformat{thmM}{Main Theorem}
\newtheorem*{thm*}{Theorem}
\newtheorem{thm}{Theorem}[section]
\newtheorem{thml}[thm]{Theorem}
\newtheorem{lem}[thm]{Lemma}
\newtheorem{obs}[thm]{Observation}
\newtheorem{lemt}[thm]{Lemma}
\newtheorem*{lem*}{Lemma}
\newtheorem{whyp}[thm]{Working Hypothesis}
\newtheorem{prop}[thm]{Proposition}
\newtheorem{prpt}[thm]{Proposition}
\newtheorem{prpl}[thm]{Proposition}
\newtheorem{cor}[thm]{Corollary}
%\newtheorem*{prop*}{Proposition}
\newtheorem{claim}{Claim}
\newtheorem*{claim*}{Claim}
%\theoremstyle{definition}
\newtheorem{defn}[thm]{Definition}
\newtheorem{dfnl}[thm]{Definition}
\newtheorem*{IndH}{Induction Hypothesis}

\newtheorem*{eg*}{Example}
\newtheorem{eg}[thm]{Example}

\theoremstyle{remark}
\newtheorem*{remark}{Remark}
\newtheorem*{remarks}{Remarks}


\def\cpc{\sigma}
\def\ccJ{\epsilon\dotepsilon}
\def\ccL{c_L}

\def\wtbfK{\widetilde{\bfK}}
%\def\abfV{\acute{\bfV}}
\def\AbfV{\acute{\bfV}}
%\def\afgg{\acute{\fgg}}
%\def\abfG{\acute{\bfG}}
\def\abfV{\bfV'}
\def\afgg{\fgg'}
\def\abfG{\bfG'}

\def\half{{\tfrac{1}{2}}}
\def\ihalf{{\tfrac{\mathbf i}{2}}}
\def\slt{\fsl_2(\bC)}
\def\sltr{\fsl_2(\bR)}

% \def\Jslt{{J_{\fslt}}}
% \def\Lslt{{L_{\fslt}}}
\def\slee{{
\begin{pmatrix}
 0 & 1\\
 0 & 0
\end{pmatrix}
}}
\def\slff{{
\begin{pmatrix}
 0 & 0\\
 1 & 0
\end{pmatrix}
}}\def\slhh{{
\begin{pmatrix}
 1 & 0\\
 0 & -1
\end{pmatrix}
}}
\def\sleei{{
\begin{pmatrix}
 0 & i\\
 0 & 0
\end{pmatrix}
}}
\def\slxx{{\begin{pmatrix}
-\ihalf & \half\\
\phantom{-}\half & \ihalf
\end{pmatrix}}}
% \def\slxx{{\begin{pmatrix}
% -\sqrt{-1}/2 & 1/2\\
% 1/2 & \sqrt{-1}/2
% \end{pmatrix}}}
\def\slyy{{\begin{pmatrix}
\ihalf & \half\\
\half & -\ihalf
\end{pmatrix}}}
\def\slxxi{{\begin{pmatrix}
+\half & -\ihalf\\
-\ihalf & -\half
\end{pmatrix}}}
\def\slH{{\begin{pmatrix}
   0   & -\mathbf i\\
\mathbf i & 0
\end{pmatrix}}
}

\ExplSyntaxOn
\clist_map_inline:nn {J,L,C,X,Y,H,c,e,f,h,}{
  \expandafter\def\csname #1slt\endcsname{{\mathring{#1}}}}
\ExplSyntaxOff


\def\Mop{\fT}

\def\fggJ{\fgg_J}
\def\fggJp{\fgg'_{J'}}

\def\NilGC{\Nil_{\bfG}(\fgg)}
\def\NilGCp{\Nil_{\bfG'}(\fgg')}
\def\Nilgp{\Nil_{\fgg'_{J'}}}
\def\Nilg{\Nil_{\fgg_{J}}}
%\def\NilP'{\Nil_{\fpp'}}
\def\nNil{\Nil^{\mathrm n}}
\def\eNil{\Nil^{\mathrm e}}


\NewDocumentCommand{\NilP}{t'}{
\IfBooleanTF{#1}{\Nil_{\fpp'}}{\Nil_\fpp}
}

\def\KS{\mathsf{KS}}
\def\MM{\bfM}
\def\MMP{M}

\NewDocumentCommand{\KTW}{o g}{
  \IfValueTF{#2}{
    \left.\varsigma_{\IfValueT{#1}{#1}}\right|_{#2}}{
    \varsigma_{\IfValueT{#1}{#1}}}
}
\def\IST{\rho}
\def\tIST{\trho}

\NewDocumentCommand{\CHI}{o g}{
  \IfValueTF{#1}{
    {\chi}_{\left[#1\right]}}{
    \IfValueTF{#2}{
      {\chi}_{\left(#2\right)}}{
      {\chi}}
  }
}
\NewDocumentCommand{\PR}{g}{
  \IfValueTF{#1}{
    \mathop{\pr}_{\left(#1\right)}}{
    \mathop{\pr}}
}
\NewDocumentCommand{\XX}{g}{
  \IfValueTF{#1}{
    {\cX}_{\left(#1\right)}}{
    {\cX}}
}
\NewDocumentCommand{\PP}{g}{
  \IfValueTF{#1}{
    {\fpp}_{\left(#1\right)}}{
    {\fpp}}
}
\NewDocumentCommand{\LL}{g}{
  \IfValueTF{#1}{
    {\bfL}_{\left(#1\right)}}{
    {\bfL}}
}
\NewDocumentCommand{\ZZ}{g}{
  \IfValueTF{#1}{
    {\cZ}_{\left(#1\right)}}{
    {\cZ}}
}

\NewDocumentCommand{\WW}{g}{
  \IfValueTF{#1}{
    {\bfW}_{\left(#1\right)}}{
    {\bfW}}
}




\def\gpi{\wp}
\NewDocumentCommand\KK{g}{
\IfValueTF{#1}{K_{(#1)}}{K}}
% \NewDocumentCommand\OO{g}{
% \IfValueTF{#1}{\cO_{(#1)}}{K}}
\NewDocumentCommand\XXo{d()}{
\IfValueTF{#1}{\cX^\circ_{(#1)}}{\cX^\circ}}
\def\bfWo{\bfW^\circ}
\def\bfWoo{\bfW^{\circ \circ}}
\def\bfWg{\bfW^{\mathrm{gen}}}
\def\Xg{\cX^{\mathrm{gen}}}
\def\Xo{\cX^\circ}
\def\Xoo{\cX^{\circ \circ}}
\def\fppo{\fpp^\circ}
\def\fggo{\fgg^\circ}
\NewDocumentCommand\ZZo{g}{
\IfValueTF{#1}{\cZ^\circ_{(#1)}}{\cZ^\circ}}

% \ExplSyntaxOn
% \NewDocumentCommand{\bcO}{t' E{^_}{{}{}}}{
%   \overline{\cO\sb{\use_ii:nn#2}\IfBooleanTF{#1}{^{'\use_i:nn#2}}{^{\use_i:nn#2}}
%   }
% }
% \ExplSyntaxOff

\NewDocumentCommand{\bcO}{t'}{
  \overline{\cO\IfBooleanT{#1}{'}}}

\NewDocumentCommand{\oliftc}{g}{
\IfValueTF{#1}{\boldsymbol{\vartheta} (#1)}{\boldsymbol{\vartheta}}
}
\NewDocumentCommand{\oliftr}{g}{
\IfValueTF{#1}{\vartheta_\bR(#1)}{\vartheta_\bR}
}
\NewDocumentCommand{\olift}{g}{
\IfValueTF{#1}{\vartheta(#1)}{\vartheta}
}
% \NewDocumentCommand{\dliftv}{g}{
% \IfValueTF{#1}{\ckvartheta(#1)}{\ckvartheta}
% }
\def\dliftv{\vartheta}
\NewDocumentCommand{\tlift}{g}{
\IfValueTF{#1}{\wtvartheta(#1)}{\wtvartheta}
}

\def\slift{\cL}

\def\BB{\bB}


\def\thetaO#1{\vartheta\left(#1\right)}

\def\bbThetav{\check{\mathbbold{\Theta}}}
\def\Thetav{\check{\Theta}}
\def\thetav{\check{\theta}}

\DeclareDocumentCommand{\NN}{g}{
\IfValueTF{#1}{\fN(#1)}{\fN}
}
\DeclareDocumentCommand{\RR}{m m}{
\fR({#1},{#2})
}

%\DeclareMathOperator*{\sign}{Sign}

\NewDocumentCommand{\lsign}{m}{
{}^l\mathrm{Sign}(#1)
}



\NewDocumentCommand\lnn{t+ t- g}{
  \IfBooleanTF{#1}{{}^l n^+\IfValueT{#3}{(#3)}}{
    \IfBooleanTF{#2}{{}^l n^-\IfValueT{#3}{(#3)}}{}
  }
}


% % Fancy bcO, support feature \bcO'^a_b = \overline{\cO'^a_b}
% \makeatletter
% \def\bcO{\def\O@@{\cO}\@ifnextchar'\@Op\@Onp}
% \def\@Opnext{\@ifnextchar^\@Opsp\@Opnsp}
% \def\@Op{\afterassignment\@Opnext\let\scratch=}
% \def\@Opnsp{\def\O@@{\cO'}\@Otsb}
% \def\@Onp{\@ifnextchar^\@Onpsp\@Otsb}
% \def\@Opsp^#1{\def\O@@{\cO'^{#1}}\@Otsb}
% \def\@Onpsp^#1{\def\O@@{\cO^{#1}}\@Otsb}
% \def\@Otsb{\@ifnextchar_\@Osb{\@Ofinalnsb}}
% \def\@Osb_#1{\overline{\O@@_{#1}}}
% \def\@Ofinalnsb{\overline{\O@@}}

% Fancy \command: \command`#1 will translate to {}^{#1}\bfV, i.e. superscript on the
% lift conner.

% \def\defpcmd#1{
%   \def\nn@tmp{#1}
%   \def\nn@np@tmp{@np@#1}
%   \expandafter\let\csname\nn@np@tmp\expandafter\endcsname \csname\nn@tmp\endcsname
%   \expandafter\def\csname @pp@#1\endcsname`##1{{}^{##1}{\csname @np@#1\endcsname}}
%   \expandafter\def\csname #1\endcsname{\,\@ifnextchar`{\csname
%       @pp@#1\endcsname}{\csname @np@#1\endcsname}}
% }

% \def\defppcmd#1{
% \expandafter\NewDocumentCommand{\csname #1\endcsname}{##1 }{}
% }



% \defpcmd{bfV}
% \def\KK{\bfK}\defpcmd{KK}
% \defpcmd{bfG}
% \def\A{\!A}\defpcmd{A}
% \def\K{\!K}\defpcmd{K}
% \def\G{G}\defpcmd{G}
% \def\J{\!J}\defpcmd{J}
% \def\L{\!L}\defpcmd{L}
% \def\eps{\epsilon}\defpcmd{eps}
% \def\pp{p}\defpcmd{pp}
% \defpcmd{wtK}
% \makeatother

\def\fggR{\fgg_\bR}
\def\rmtop{{\mathrm{top}}}
\def\dimo{\dim^\circ}
\def\GKdim{\text{GK-}\dim}

\NewDocumentCommand\LW{g}{
\IfValueTF{#1}{L_{W_{#1}}}{L_{W}}}
%\def\LW#1{L_{W_{#1}}}
\def\JW#1{J_{W_{#1}}}

\def\floor#1{{\lfloor #1 \rfloor}}

\def\KSP{K}
\def\UU{\rU}
\def\UUC{\rU_\bC}
\def\tUUC{\widetilde{\rU}_\bC}
\def\OmegabfW{\Omega_{\bfW}}


\def\BB{\bB}


\def\thetaO#1{\vartheta\left(#1\right)}

\def\Thetav{\check{\Theta}}
\def\thetav{\check{\theta}}

\def\Thetab{\bar{\Theta}}

\def\cKaod{\cK^{\mathrm{aod}}}

\DeclareMathOperator{\sspan}{span}


\def\sp{{\mathrm{sp}}}

\def\bfLz{\bfL_0}
\def\sOpe{\sO^\perp}
\def\sOpeR{\sO^\perp_\bR}
\def\sOR{\sO_\bR}

\def\ZX{\cZ_{X}}
\def\gdliftv{\vartheta}
\def\gdlift{\vartheta^{\mathrm{gen}}}
\def\bcOp{\overline{\cO'}}
\def\bsO{\overline{\sO}}
\def\bsOp{\overline{\sO'}}
\def\bfVpe{\bfV^\perp}
\def\bfEz{\bfE_0}
\def\bfVn{\bfV^-}
\def\bfEzp{\bfE'_0}

\def\totimes{\widehat{\otimes}}
\def\dotbfV{\dot{\bfV}}

\def\aod{\mathrm{aod}}
\def\unip{\mathrm{unip}}
\def\IC{\mathfrak{I}}

\def\PI#1{\Pi_{\cI_{#1}}}
\def\Piunip{\Pi^{\mathrm{unip}}}
\def\cf{\emph{cf.} }
\def\Groth{\mathrm{Groth}}
\def\Irr{\mathrm{Irr}}
\def\Irrsp{\mathrm{Irr}^{\text{sp}}}

\def\edrc{\mathrm{DRC}^{\mathrm e}}
\def\drc{\mathrm{DRC}}
\def\LS{\mathrm{LS}}
\def\Unip{\mathrm{Unip}}


% Ytableau tweak
\makeatletter
\pgfkeys{/ytableau/options,
  noframe/.default = false,
  noframe/.is choice,
  noframe/true/.code = {%
    \global\let\vrule@YT=\vrule@none@YT
    \global\let\hrule@YT=\hrule@none@YT
  },
  noframe/false/.code = {%
    \global\let\vrule@YT=\vrule@normal@YT
    \global\let\hrule@YT=\hrule@normal@YT
  },
  noframe/on/.style = {noframe/true},
  noframe/off/.style = {noframe/false},
}
\makeatother 


\def\wAV{\AV^{\mathrm{weak}}}
\def\ckG{\check{G}}
\def\ckGc{\check{G}_{\bC}}
\def\dBV{d_{\mathrm{BV}}}
\def\CP{\mathsf{CP}}
\def\YD{\mathsf{YD}}
\def\SYD{\mathsf{SYD}}
\def\DD{\nabla}

\def\lamck{\lambda_\ckcO}
\def\lamckb{\lambda_{\ckcO_b}}
\def\lamckg{\lambda_{\ckcO_g}}
\def\Wint#1{W_{[#1]}}
\def\CLam{\Coh_{\Lambda}}
\def\Cint#1{\Coh_{[#1]}}
\def\PP{\mathrm{PP}}
\def\BOX#1{\mathrm{Box}(#1)}
\DeclareDocumentCommand{\bigtimes}{}{\mathop{\scalebox{1.7}{$\times$}}}
\providecommand\mapsfrom{\scalebox{-1}[1]{$\mapsto$}}

\def\Gc{G_\bC}
\def\Gcad{G_\bC^{\text{ad}}}
\def\hha{{}^a\fhh}
\def\aX{{}^aX}
\def\aQ{{}^aQ}
\def\aP{{}^aP}
\def\aR{{}^aR}
\def\aRp{{}^aR^+}
\def\asRp{{}^a \Delta^+}
\def\Gfin{\cG(\Gc)}
\def\PiGfin{\Pi_{\mathrm{fin}}( \Gc )}
\def\PiGlfin{\Pi_{\Lambda_0}( \Gc )}
\def\adGfin{\cG_{\mathrm{ad}}(\Gc)}
\def\Ggk{\cG(\fgg,K)}
\def\WT#1{\Delta(#1)}
\def\WG{W(\Gc)}
\def\ch{\mathrm{ch}\,}
\def\Wlam{W_{[\lambda]}}
\def\aLam{a_{\Lambda}}
\def\WLam{W_{\Lambda}}
\def\WLamck{W_{[\lambda_{\ckcO}]}}
\def\Wlamck{W_{\lamck}}
\def\Rlam{R_{[\lambda]}}
\def\RLam{R_\Lambda}
\def\RLamp{R_\Lambda^+}
\def\Rplam{R^+(\lambda)}
\def\Glfin{\cG_{\Lambda}(\Gc)}
\def\CL{{\sC}^{\scriptscriptstyle L}}
\def\CR{{\sC}^{\scriptscriptstyle R}}
\def\CLR{{\sC}^{\scriptscriptstyle LR}}
\def\LC{{}^{\scriptscriptstyle L}\sC}
\def\RC{{}^{\scriptscriptstyle R}\sC}
\def\LRC{{}^{\scriptscriptstyle LR}\sC}
\def\ckLC{{}^{\scriptscriptstyle L}\check{\sC}}


\def\bVL{{\overline{\sV}}^{\scriptscriptstyle L}}
\def\bVR{{\overline{\sV}}^{\scriptscriptstyle R}}
\def\bVLR{{\overline{\sV}}^{\scriptscriptstyle LR}}
\def\VL{{\sV}^{\scriptscriptstyle L}}
\def\VR{{\sV}^{\scriptscriptstyle R}}
\def\VLR{{\sV}^{\scriptscriptstyle LR}}

\def\Con{\sfC}
\def\bCon{\overline{\sfC}}
\def\Re{\mathrm{Re}}
\def\Im{\mathrm{Im}}
\def\AND{\quad \text{and} \quad}
\def\Coh{\mathrm{Coh}}
\def\Cohlm{\Coh_{\Lambda}(\cM)}
\def\ev#1{{\mathrm{ev}_{#1}}}

\def\ppp{\times}
\def\mmm{\slash}


\def\cuprow{{\stackrel{r}{\sqcup}}}
\def\cupcol{{\stackrel{c}{\sqcup}}}

\def\Spr{\mathrm{Springer}}
\def\Prim{\mathrm{Prim}}


\def\CQ{\overline{\sfA}}% Lusztig's canonical quotient
\def\CPP{\mathfrak{P}}
\def\CPPs{\mathfrak{P}_{\star}}


\def\ceil#1{\lceil #1 \rceil}
\def\symb#1#2{{\left(\substack{{#1}\\{#2}}\right)}}
\def\cboxs#1{\mbox{\scalebox{0.25}{\ytb{\ ,\vdots,\vdots,\ }}}_{#1}}

\def\hsgn{\widetilde{\mathrm{sgn}}}

\def\PBPe{\mathrm{PBP}^{\mathrm{ext}}}
\def\PBPes{\mathrm{PBP}^{\mathrm{ext}}_{\star}}
\def\bev#1{\overline{\mathrm{ev}}_{#1}}

\def\Prim{\mathrm{Prim}}
% \def\leqL{\stackrel{L}{\leq}}
% \def\leqR{\stackrel{R}{\leq}}
% \def\leqLR{\stackrel{LR}{\leq}}

% \def\leqL{{\leq_L}}
% \def\leqR{{\leq_R}}
% \def\leqLR{{\leq_{LR}}}


\def\lneqL{\mathrel{\mathop{<}\limits_{\scriptscriptstyle L}}}
\def\lneqR{\mathrel{\mathop{<}\limits_{\scriptscriptstyle R}}}
\def\lneqLR{\mathrel{\mathop{<}\limits_{\scriptscriptstyle LR}}}

\def\leqL{\mathrel{\mathop{\leq}\limits_{\scriptscriptstyle L}}}
\def\leqR{\mathrel{\mathop{\leq}\limits_{\scriptscriptstyle R}}}
\def\leqLR{\mathrel{\mathop{\leq}\limits_{\scriptscriptstyle LR}}}


\def\approxL{\mathrel{\mathop{\approx}\limits_{\scriptscriptstyle L}}}
\def\approxR{\mathrel{\mathop{\approx}\limits_{\scriptscriptstyle R}}}
\def\approxLR{\mathrel{\mathop{\approx}\limits_{\scriptscriptstyle LR}}}


\def\dphi{\rdd \phi}
\def\CPH{C(H)}
\def\whCPH{\widehat{C(H)}}

\providecommand{\nsubset}{\not\subset}

\begin{document}


\title[]{Counting special unipotent representations of real classical groups}

\author [D. Barbasch] {Dan M. Barbasch}
\address{the Department of Mathematics\\
  310 Malott Hall, Cornell University, Ithaca, New York 14853 }
\email{dmb14@cornell.edu}

\author [J.-J. Ma] {Jia-jun Ma}
\address{School of Mathematical Sciences\\
  Shanghai Jiao Tong University\\
  800 Dongchuan Road, Shanghai, 200240, China} \email{hoxide@sjtu.edu.cn}

\author [B. Sun] {Binyong Sun}
% MCM, HCMS, HLM, CEMS, UCAS,
\address{Academy of Mathematics and Systems Science\\
  Chinese Academy of Sciences\\
  Beijing, 100190, China} \email{sun@math.ac.cn}

\author [C.-B. Zhu] {Chen-Bo Zhu}
\address{Department of Mathematics\\
  National University of Singapore\\
  10 Lower Kent Ridge Road, Singapore 119076} \email{matzhucb@nus.edu.sg}




\subjclass[2000]{22E45, 22E46} \keywords{special unipotent
  representation, coherent continuation representation, classical group}

% \thanks{Supported by NSFC Grant 11222101}
\maketitle


\tableofcontents

\section{Introduction and the main results}

Let $G$ be a real reductive group in the Harish-Chandra class, which may be linear or non-linear. Write $\fgg$ for the complexified Lie algebra of $G$ and let $\h$ denote its universal Cartan subalgebra. Let $\lambda \in \h^*$ (a superscript $*$ indicates the dual space). By Harish-Chandra isomorphism, it determines  an algebraic character $\chi_\lambda: \CZ(\g)\rightarrow \C$. Here $\CZ(\g)$ denotes the center of the universal enveloping algebra $\mathcal U(\g)$. Denote by $\Irr(G)$ the set of isomorphism classes of irreducible Casselman-Wallach representations of $G$, and by $\Irr_\lambda(G)$ its subset consisting of the representations with infinitesimal character $\chi_\lambda$. The latter set has finite cardinality. 


Let $\mathrm{Nil}(\g^*)$ denote the set of nilpotent elements in $\g^*$. It has only finitely many orbits under the coadjoint action of the inner automorphism group $\mathrm{Inn}(\g)$ of $\g$. Let $\sfS$ be an  $\mathrm{Inn}(\g)$-stable Zariski closed subset of $\mathrm{Nil}(\g^*)$. Put
\[
  \Irr_{\lambda,\sfS}(G):=\Set{\pi \in \Irr_{\lambda}(G)| \text{$\AVC(\pi)\subset \sfS$} }.
\]
Here $\AVC(\pi)$ denotes the complex associated variety of $\pi$, namely the associated variety of the annihilate ideal of $\pi$. It is an  $\mathrm{Inn}(\g)$-stable Zariski closed subset of $\mathrm{Nil}(\g^*)$. 
An interesting  problem of representation theory is to understand the cardinality of the finite set $\Irr_{\lambda,\sfS}(G)$. 
The coherent continuation representation is a powerful tool for this problem.

\subsection{The coherent continuation representation}


Consider the category of Cassleman-Wallach representations of $G$ that have generalized infinitesimal character $\lambda$ and whose complex associated variety is contained in $\sfS$. 
Write  $\CK_{\lambda,\sfS}(G)$ for the  Grothendieck group with $\C$-coefficients  of this category. Then
\[
  \sharp (\Irr_{\lambda,\sfS}(G))=\dim \CK_{\lambda,\sfS}(G)\qquad(\sharp\textrm{ indicates the cardinality of a set}).
\]
We also have that
\[
  \CK_\sfS(G)=\bigoplus_{\mu\in W\backslash  \h^*} \CK_{\mu,\sfS}(G)\qquad (W\textrm{ denotes the Weyl group}),
\]
where $\CK_{\sfS}(G)$ is  the Grothendieck group, with $\C$-coefficients,  of the category of Cassleman-Wallach representations of $G$ whose complex associated variety is contained in $\sfS$. 


Write $\mathcal R(\g)$ for the Grothendieck group, with $\C$-coefficients,  of the category of finite-dimensional algebraic representations of $\mathrm{Inn}(\g)$. It is a commutative $\C$-algebra by using the tensor products. By pulling back through the adjoint representation $G\rightarrow \mathrm{Inn}(\g)$, every representation of $\mathrm{Inn}(\g)$ is viewed as a representation of $G$. Then by using the tensor products, $\CK_\sfS(G)$ is naturally a $\mathcal R(\g)$-module.






Write $\Delta\subset Q\subset \h^*$ for the root system and the root lattice of $\fgg$, respectively. Let $\Lambda\subset \h^*$ be a $Q$-coset, and write $W_\Lambda$ for its stabilizer group in $W$. Then $W_\Lambda$ equals the Weyl group of the root system
\[
  \{\alpha \in \Delta\mid \langle \mu, \alpha^\vee\rangle \in \Z\textrm{ for all }\mu\in \Lambda\}\qquad (\alpha^\vee \textrm{ denotes the corresponding coroot}). 
\]


\begin{defn}\label{defcoh}
  Let $\CK$  be a $\mathcal R(\g)$-module equipped with a family $\{\CK_\mu\}_{\mu\in  \h^*}$ of subspaces such that $\CK_{w\cdot \mu}=\CK_\mu$ for all $w\in W_\Lambda$ and $\mu\in \h^*$. 
  A $\CK$-valued coherent family on $\Lambda$ is a map
  \[
    \Theta: \Lambda\rightarrow \CK%, \qquad \mu\mapsto \Theta_\mu
  \]
  satisfying the following two conditions:
  \begin{itemize}
  \item for all $\mu\in \Lambda$,
    $\Theta(\mu)\in \CK_\mu$;
  \item for all finite-dimensional algebraic representations $F$ of $\mathrm{Inn}(\g)$, and all $\mu\in \Lambda$, 
      \[
      F\cdot (\Theta(\mu)) = \sum_{\nu} \Theta(\mu+\nu),
  \]
  where $\nu$ runs over all weights of $F$, counted with multiplicities, and $F$ is obviously identified with an element of $\mathcal R(\g)$.
  \end{itemize}
  \end{defn}


In the notation of Definition \ref{defcoh},   let $\Coh_{\Lambda}(\CK)$ denote the vector space  of all $\mathcal K$-valued coherent families 
  on $\Lambda$. It is a representation of  $W_{\Lambda}$ under the action
  \[
    (w\cdot \Theta)(\mu) = \Theta(w^{-1}\cdot \mu), \qquad \textrm{for all }\  w\in W_\Lambda, \ \mu\in \Lambda.
  \]
For simplicity in notation, put
\[
\Coh_{\Lambda,\sfS}(G):=\Coh_{\Lambda}(\CK_\sfS(G)).
\]


\subsection{Counting the irreducible representations with bounded complex associated varieties}

Suppose that $\lambda\in \Lambda$ and denote by $W_\lambda$ the stabilizer of $\lambda$ in $W$. Then $W_\lambda\subset W_\Lambda$. Write $1_{W_\lambda}$  for the trivial representation of $W_{\lambda}$.

The following Theorem is due to Vogan. We will provide a proof for the lack of reference. 
\begin{thm}\label{count1}
The equality 
  \[
    \sharp(\Irr_{\lambda,\sfS}(G)) = [1_{W_{\lambda}}:\Coh_{\Lambda,\sfS}(G)]
  \]
  holds. 
  % \[
  %   \dim {\barmu} = \dim (\cohm)_{W_\mu} = [\cohm, 1_{W_\mu}].
  % \]
\end{thm}
Here and henceforth, $[\ : \ ]$ indicates the multiplicity of the first (irreducible) representation in the second one. 
Theorem \ref{count1} implies that
 \begin{equation}\label{countlg}
    \sharp(\Irr_{\lambda,\sfS}(G)) = \sum_{\sigma\in \Irr(W_\Lambda)} [1_{W_{\lambda}}: \sigma]\cdot [\sigma: \Coh_{\Lambda,\sfS}(G)].
  \end{equation}
Thus it suffices to understand the multiplicity $ [\sigma: \Coh_{\Lambda,\sfS}(G)]$ for every $\sigma\in \Irr(W_\Lambda)$.  Let $\sigma\in  \Irr(W_\Lambda)$.  Define the nilpotent orbit
\[
\CO_\sigma:=\mathrm{Springer}(j_{W_\Lambda}^W \sigma_0)\subset \mathrm{Nil}(\g^*) \quad (``\mathrm{Springer}"\textrm{ indicates the Springer correspondence}),
\]
where  $\sigma_0$ denote the special irreducible representation of $W_\Lambda$ that lies in the same double cell as  $\sigma$, and $j_{W_\Lambda}^W \sigma_0$ denotes the $j$-induction which is an irreducible representation of $W$. 
%In what follows, we shall attach to $\sigma$ an $\mathrm{Inn}(\g)$-orbit $\CO_\sigma\subset\mathrm{Nil}(\g^*)$. Let $\sigma_0$ denote the special irreducible representation of $W_\Lambda$ that lies in the same double cell as  $\sigma$. Then among all the irreducible representations of $W$ occurring in the induced representation $\Ind_{W_\Lambda}^W \sigma_0$, there is a unique one with minimal fake degree. Moreover, this irreducible representation occurs with multiplicity one in $\Ind_{W_\Lambda}^W \sigma_0$. Write $j_{W_\Lambda}^W \sigma_0\subset \Ind_{W_\Lambda}^W \sigma_0 $ for this irreducible representation. Finally we define

%Here and henceforth, for every $\sigma'\in \Irr(W)$, $\mathrm{Springer}(\sigma')$ denotes the
%Under Springer correspondence, $j_{W_\Lambda}^W \sigma_0$ corresponds to the trivial $\mathrm{Inn}(\g)$-equivariant local system on an $\mathrm{Inn}(\g)$-orbit in $\mathrm{Nil}(\g^*)$. Then $\CO_\sigma$ is defined to be this orbit. 


\begin{prop}\label{count2}
Suppose that $\sigma\in  \Irr(W_\Lambda)$ and $\CO_\sigma\nsubset \sfS$. Then 
  \[
    [\sigma: \Coh_{\Lambda,\sfS}(G)]=0.
  \]
 
  % \[
  %   \dim {\barmu} = \dim (\cohm)_{W_\mu} = [\cohm, 1_{W_\mu}].
  % \]
\end{prop}

Combining Theorems \ref{count1} and \ref{count2}, we conclude that
  \begin{equation}\label{leq2}
  \sharp(\Irr_{\lambda,\sfS}(G)) = \sum_{\sigma\in \Irr(W_\Lambda), \CO_\sigma\subset \sfS} [1_{W_{\lambda}}: \sigma]\cdot [\sigma: \Coh_{\Lambda,\sfS}(G)].
    \end{equation}

Write
\[
  \Coh_{\Lambda}(G):= \Coh_{\Lambda,\mathrm{Nil}(\g^*)}(G),
  \]
  which contains $ \Coh_{\Lambda,\sfS}(G)$ as a subrepresentation. It is obvious that
  \begin{equation}\label{leq1}
    [\sigma: \Coh_{\Lambda,\sfS}(G)]\leq [\sigma: \Coh_{\Lambda}(G)].
  \end{equation}
  We expect that 
   \begin{equation}\label{leq1}
  \textrm{if $\ \CO_\sigma\subset \sfS\ $, then  }\  [\sigma: \Coh_{\Lambda,\sfS}(G)]= [\sigma: \Coh_{\Lambda}(G)].
  \end{equation}
  
  
 % the equality always holds. Combining Theorem \ref{count1}, Theorem \ref{count2} and \eqref{leq1}, we conclude that
  %\begin{equation}\label{leq2}
  %\sharp(\Irr_{\lambda,\sfS}(G)) \leq \sum_{\sigma\in \Irr(W_\Lambda), \CO_\sigma\subset \sfS} [1_{W_{\lambda}}: \sigma]\cdot [\sigma: \Coh_{\Lambda}(G)].
    %\end{equation}





\subsection{Counting the irreducible representations annihilated by a maximal primitive ideal}
Write $I_\lambda$ for the maximal ideal of $\mathcal U(\g)$ with infinitesimal character $\lambda$. Its associated variety equals the Zariski closure $\overline{\CO_\lambda}$ of an $\mathrm{Inn}(\g)$-orbit  $\CO_\lambda\subset\mathrm{Nil}(\g^*) $.  Note that an irreducible Casselman-Wallach representation of $G$ lies in $\Irr_{\lambda,\overline{\CO_\lambda}}(G)$ if and only if it is annihilated by $I_\lambda$. 


Let $\LC_{\lambda}\subset \Irr(W_\Lambda)$ be the subset consisting all the irreducible representations that occur in 
  \[
    (J_{W_{\lambda}}^{W_{\Lambda}} \sgn )\otimes \sgn,
  \]
where $J_{W_{\lambda}}^{W_{\Lambda}} $ indicates the $J$-induction, and $\sgn$ denotes the sign character. 



 \begin{prop}[{\cite{BVUni}*{(5.26), Proposition~5.28}}]\label{lem:lcell.BV0}
 The set
 \[
    \LC_{\lambda} = \Set{\sigma\in  \Irr(W_\Lambda)\mid \CO_\sigma\subset \overline{\CO_\lambda}, \    [1_{W_{\lambda}}:\sigma]\neq 0}.
   \]
  Moreover, the multiplicity $[1_{W_{\lambda}}:\sigma]$ is one
  when $\sigma\in \LC_\lambda$.
\end{prop}

Combining \eqref{countlg}, Proposition \ref{count2}
and Proposition \ref{lem:lcell.BV0}, we obtain the following inequality (and the equality is expected).

\begin{cor}
 % Under the notation of \Cref{lem:lcell.BV}, we have 
 The inequality 
  \begin{equation}\label{boundc}
    \sharp(\Irr_{\lambda,\overline{\CO_\lambda}}(G)) \leq \sum_{\sigma\in \LC_\lambda} [\sigma: \Coh_{\Lambda}(G)]
  \end{equation}
  holds. 
\end{cor}



\subsection{Special unipotent representations of classical groups}

We are particularly interested in counting special unipotent representations of  real classical groups.  

Let $\star$ be one of the  14 symbols
\[
\textrm{ $A$, $A^\C$, $A^\bH$, $A^*$, $B$, $D$,   $B^\C$, $D^\C$, $C$, $C^\C$, $D^*$, $C^*$, $\widetilde C$, $\widetilde C^\C$. }
\]
Suppose that $G$ is a classical Lie group of type $\star$, namely $G$ respectively equals 
one of the following Lie groups:
\[
\begin{array}{c}
   \GL_n(\R), \ \GL_n(\C), \  \GL_n(\bH),\  \oU(p,q),\smallskip\\
    \SO(p,q)\ (p+q\, \textrm{ is odd}),  \  \SO(p,q)\  (p+q\, \textrm{ is even}),\smallskip\\  
     \SO_n(\C) \ (n\, \textrm{ is odd}),  \ 
     \SO_n(\C) \ (n\, \textrm{ is even}),\smallskip \\
     \Sp_{2n}(\R), \ \Sp_{2n}(\C), \  \oO^*(2n), \  \Sp(p,q),\   \widetilde \Sp_{2n}(\R), \ \Sp_{2n}(\C) \qquad (n, p, q\geq 0).
     \end{array}
\]
Here $\widetilde \Sp_{2n}(\R)$ denotes the metaplectic double cover of the symplectic group $\Sp_{2n}(\R)$ that does not split unless $n=0$. 
As usual, the universal Cartan subalgebra $\h$ of 
 the complexified Lie algebra $\g$ of $G$ is respectively identified with
\[
\begin{array}{c}
  \C^n, \ \C^n\times \C^n, \ \C^{2n},  \ \C^{p+q},\smallskip \\
\C^{\frac{p+q-1}{2}},\ \C^{\frac{p+q}{2}}, \smallskip \\
\C^{\frac{p+q-1}{2}}\times \C^{\frac{p+q-1}{2}},\  \C^{\frac{p+q}{2}}\times \C^{\frac{p+q}{2}}, \smallskip \\
\C^n,\ \C^n\times \C^n, \ \C^n, \, \C^{p+q},   \ \C^n,\ \C^n\times \C^n.
     \end{array}
\]
 

We define the Langlands dual $\check G$ of $G$ to be the respective complex group
\[
 \begin{array}{c}
   \GL_n(\C), \ \GL_n(\C), \  \GL_{2n}(\C),\  \GL_{p+q}(\C),\smallskip\\
    \Sp_{p+q-1}(\C)\ (p+q\, \textrm{ is odd}),  \  \oO_{p+q}(\C)\  (p+q\, \textrm{ is even}),\smallskip\\  
     \Sp_{n-1}(\C) \ (n\, \textrm{ is odd}),  \ 
     \oO_n(\C) \ (n\, \textrm{ is even}),\smallskip \\
     \oO_{2n+1}(\C), \ \oO_{2n+1}(\C), \  \oO_{2n}(\C), \  \oO_{2p+2q+1}(\C),\    \Sp_{2n}(\C) \  \textrm{or } \  \Sp_{2n}(\C).
     \end{array}
\]
Write $n_{\check G}$ for the dimension of the standard representation of $\check G$, which respectively equals
\[
 n, n, 2n, p+q, p+q-1, p+q, n-1, n, 2n+1, 2n+1, 2n, 2p+2q+1, 2n, \textrm{ or }\ 2n. 
\]

For a Young diagram $\imath$, write
\[
 \mathbf r_1(\imath)\geq \mathbf r_2(\imath)\geq \mathbf r_3(\imath)\geq \cdots
\]
for its row lengths, and similarly,
write
\[
 \mathbf c_1(\imath)\geq \mathbf c_2(\imath)\geq \mathbf c_3(\imath)\geq \cdots
\]
for its column lengths.
Denote by $\abs{\imath}:=\sum_{i=1}^\infty \mathbf r_i(\imath)$ the total size of $\imath$. 
Let $\check \CO$ be a Young diagram with  the following property:
\begin{itemize}
\item $\abs{\check \CO}=n_{\check G}$; 
\item if $\star\in\{D,  D^\C,  D^*,C, C^\C, C^* \} $, then all even rows occur in $\check \CO$ with even multiplicity;
\item if $\star\in \{B, \widetilde C, B^\C, \widetilde C^\C \}$, then all odd rows occur in $\check \CO$ with even multiplicity.
\end{itemize}
As usual, $\check \CO$ is identified with a nilpotent orbit in the Lie algebra $\check \g$ of $\check G$.

%Define 
%\[
 % r_{\check \CO}:= r_{\star, \check \CO}:=  \begin{cases}
 % \abs{\check \CO}, & \text{if } \star \in \set{A, A^\bH, A^*};\\
  % 2\cdot \abs{\check \CO}, & \text{if } \star =A^\C;\\
%\left  \lfloor \frac{\abs{\check \CO}}{2}\right \rfloor, & \text{if } \star \in \set{B, C, D, D^*, C^*,\widetilde C};\\
 % 2\cdot \left \lfloor \frac{\abs{\check \CO}}{2}\right \rfloor, & \text{if } \star \in \set{B^\C, C^\C, D^\C, \widetilde C^\C};\\
  %\end{cases}
% \]
We define an element  $\lambda_{\check \CO}:=\lambda_{\star, \check \CO} \in \h^*$  as in what follows. 
For every  $a\in \bN^+$ (the set of positive integers), write
\[
  \rho(a):=\left\{ \begin{array}{ll}
                  (\frac{a-1}{2}, \frac{a-3}{2}, \cdots, 2,1), \quad &\textrm{if $a$ is odd;}\smallskip\\
                    (\frac{a-1}{2}, \frac{a-3}{2}, \cdots, \frac{3}{2}, \frac{1}{2}), \quad &\textrm{if $a$ is even,}\\
                    \end{array}
                 \right. 
\]
and 
\[
   \tilde \rho(a):=
                  (\frac{a-1}{2}, \frac{a-3}{2}, \cdots, \frac{3-a}{2},\frac{1-a}{2}).                             
\]
By convention, $\rho(1)$ is the empty sequence. Write $a_1\geq  a_2\geq \cdots\geq a_s>0$ ($s\geq 0$)  for the nonzero row lengths of  $\check \CO$. Then we define
\[
  \lambda_{\check \CO}:= \begin{cases}
 (\tilde \rho( a_1), \tilde \rho(a_2),  \cdots, \tilde \rho(a_s)), & \text{if } \star \in \set{A, A^\bH, A^*};\\
  (\tilde \rho( a_1), \tilde \rho(a_2),  \cdots, \tilde \rho(a_s); \tilde \rho( a_1), \tilde \rho(a_2),  \cdots, \tilde \rho(a_s)), & \text{if } \star =A^\C;\\
  (\rho( a_1), \rho(a_2),  \cdots, \rho(a_s), 0^{l_1} ) , & \text{if } \star \in \set{B, C, D, D^*, C^*,\widetilde C};\\
(\rho( a_1), \rho(a_2),  \cdots, \rho(a_s), 0^{l_1} ;  \rho( a_1), \rho(a_2),  \cdots, \rho(a_s), 0^{l_1} ) , & \text{if } \star \in \set{B^\C, C^\C, D^\C, \widetilde C^\C}.\\
   \end{cases}
\]
Here 
\[
l_1:= \left\lfloor\frac{\textrm{the number of odd rows of the Young diagram of $\check \CO$}}{2}\right\rfloor, 
\]
and $0^{l_1}$ denotes the sequence of $0$'s of length $l_1$. 

By using  Harish-Chandra isomorphism, we view $\lambda_{\check \CO}$ as a character $\lambda_{\check \CO}: \mathcal Z(\g)\rightarrow \C$. Write $I_{\check \CO}:=I_{\star, \check \CO}$ for the maximal ideal of $\mathcal U(\g)$ with infinitesimal character $\lambda_{\check \CO}$. If $\star=D$ or $D^*$, then $\g=\o_{m}(\C)$ with $m=p+q$  or $2n$ respectively. In this case, put $I'_{\check \CO}:=\Ad_g(I_{\check \CO})$, where $g\in \oO_m(\C)\setminus \SO_m(\C)$ and $\Ad$ indicates the adjoint action. If $\star=D^\C$, then $\g=\o_{n}(\C)\times \o_n(\C)$, and we put $I'_{\check \CO}:=\Ad_{(g,g)}(I_{\check \CO})$, where $g\in \oO_n(\C)\setminus \SO_n(\C)$. In all these  cases, $I_{\check \CO}=I'_{\check \CO}$ unless $\check \CO$ has no odd rows. 

Finally, we define the set of the special unipotent representations of $G$ attached to $\check \CO$ by
\begin{eqnarray*}
&&\Unip_{\check \CO}(G):= \Unip_{\star, \check \CO}(G) \\
&:=& \begin{cases}
   \{\pi\in \Irr(G)\mid \pi \textrm{ is annihilated by $ I_{\check \CO}$ or $I'_{\check \CO}$}\}, & \text{if } \star \in \set{D, D^\C, D^*};\\
  \{\pi\in \Irr(G)\mid \pi \textrm{ is genuine  and annihilated by } I_{\check \CO}\}, & \text{if } \star =\widetilde C;\\
   \{\pi\in \Irr(G)\mid \pi \textrm{ is annihilated by } I_{\check \CO}\}, & otherwise.\\
      \end{cases}
\end{eqnarray*}
Here ``genuine" means that the representation $\pi$ of $\widetilde \Sp_{2n}(\R)$ does not descend to $\Sp_{2n}(\R)$.
Our ultimate goal is to parametrize the set $\Unip_{\check \CO}(G)$ and to construct  all the representations in this set (\cite{BMSZ2}). 




\subsection{The cases of general linear groups and unitary groups}

For any Young diagram $\imath$, we introduce the set $\mathrm{Box}(\imath)$ of boxes of $\imath$ as the following subset
of $\bN^+\times \bN^+$: 
\begin{equation}\label{eq:BOX}
\mathrm{Box}(\imath):=\Set{(i,j)\in\bN^+\times \bN^+| j\leq \bfrr_i(\imath)}.
\end{equation}
%We will also call a subset of $\bN^+\times \bN^+$  of the form \eqref{eq:BOX} a Young diagram.

%We say that a Young diagram $\imath'$ is contained
%in $\imath$ (and write $\imath'\subset \imath$) if
%\[
%  \mathbf r_i(\imath')\leq \mathbf r_i(\imath)\qquad \textrm{for all } i=1,2, 3, \cdots.
%\]
%When  this is the case, $\mathrm{Box}(\imath')$ is viewed as a subset of $\mathrm{Box}(\imath)$ concentrating on the upper-left corner.
%We say that a subset of $\mathrm{Box}(\imath)$ is a Young subdiagram if it equals $\mathrm{Box}(\imath')$ for a Young diagram $\imath'\subset \imath$.
%  In this case, we call $\imath'$ the Young diagram corresponding to this Young subdiagram.

\renewcommand{\CP}{\mathcal{P}}
We also introduce five symbols $\bullet$, $s$, $r$, $c$ and $d$, and make the following definitions.
\begin{defn}
A painting on a Young diagram $\imath$ is a map
\[
  \mathcal P: \mathrm{Box}(\imath) \rightarrow \{\bullet, s, r, c, d \}
\]
with the following properties:
\begin{itemize}
\item
 $\mathcal P^{-1}(S)$ is the set of boxes of a Young diagram when $S=\{\bullet\}, \{\bullet, s \}, \{\bullet, s, r\}$ or $\{\bullet, s, r, c \} $;
 \item
 when $S=\{s\}$ or $ \{r\}$, every row of $\imath$ has at most one  box in $\CP^{-1}(S)$;
   \item
 when $S=\{c\}$ or $ \{d \}$, every column of $\imath$ has at most one  box in $\CP^{-1}(S)$.
 \end{itemize}
\end{defn}



\begin{defn}\label{defpbp0}
Suppose that $\star\in \{A, A^\bH, A^*\}$.  A painting $\CP$ on a Young diagram $\imath$ has type $\star$ if 
 \begin{itemize}
   \item 
 the image of $\CP$ is contained in
 \[
 \left\{
     \begin{array}{ll}
         \{\bullet, c, d\}, &\hbox{if $\star=A$}; \smallskip\\
            \{\bullet\}, &\hbox{if $\star=A^\bH$}; \smallskip\\
          \{\bullet, s, r\}, &\hbox{if $\star=A^*$},            \end{array}
   \right.
 \]
 \item
 if $\star=A$ or $A^\bH$, then $\CP^{-1}(\bullet)$ has even number of boxes in every column of $\imath$, 
  \item
 if $\star=A^*$, then $\CP^{-1}(\bullet)$ has even number of boxes in every row of $\imath$.
 \end{itemize}
 Write $\mathrm{PP}_\star(\imath)$ for the set of paintings on $\imath^t$ (the transpose of $\imath$) that has type $\star$.
 \end{defn}


 
 The special unipotent representations of the general linear groups are well-understood. In particular, we have the following result on the counting. 
\begin{thm}
The equality 
\[
\sharp(\Unip_{\check \CO}(G))= \left\{
     \begin{array}{ll}
        \sharp(\mathrm{PP}_\star(\check \CO)), &\hbox{if $\star\in\{A, A^\bH\}$}; \smallskip\\
           1, &\hbox{if $\star=A^\C$}  \end{array}
   \right.
\]
holds. 

\end{thm}
\begin{remark}
If $\star=A$, then  
\[
\sharp(\mathrm{PP}_\star(\check \CO))=\prod_{i\in \bN^+} (1+\textrm{the number of rows of length $i$ in $\check \CO$})
\]
If 
 $\star=A^\bH$, then  
\[
\sharp(\mathrm{PP}_\star(\check \CO))= \left\{
     \begin{array}{ll}
        1, &\hbox{if all row lengths of $\check \CO$ are even}; \smallskip\\
           0, &\hbox{otherwise}.  \end{array}
   \right.
\]

\end{remark}

Suppose that $\imath$ is a Young diagram and $\CP$ is a painting on  $\imath$ that has type $A^*$. Let $$\mathrm{AC}_\CP: \mathrm{Box}(\imath)\rightarrow \{+, -\}$$ to be the map such that 
  \begin{itemize}
    \item the symbols $+$ and $-$ occur alternatively in each row of $\imath$;
    \item for all $1\leq i\leq \mathbf c_1(\imath) $, 
    \[
       \mathrm{AC}_\CP (i,\bfrr_i(\imath)) := \begin{cases}
    +,  & \text{if  }\CP(i,\bfrr_i(\imath))=r;\\
    -,  & \text{if } \CP(i,\bfrr_i(\imath))\in \set{\bullet,s}.
  \end{cases}
    \]
  \end{itemize}
Define the signature of $\CP$ to be the pair
\[
  (p_\CP, q_\cP):=(\sharp( \mathrm{AC}_\CP^{-1}(+)),\, \sharp( \mathrm{AC}_\CP^{-1}(-))). 
\]

\begin{eg}
 Suppose that \[
\check \CO=\ytb{\ \ \ \ \ , \ \ \ , \ , \ , \   }\quad \textrm{and}\quad  \CP=\ytb{\bullet\bullet\bullet\bullet r,\bullet\bullet , sr,s,r}\in \mathrm{PP}_{A^*}(\check \CO) . 
 \]
 Then
 \[
  \mathrm{AC}_\CP=
 \ytb{+-+-+,+-, -+,-,+}\quad \textrm{and}  \quad  (p_\CP, q_\cP)=(6,5). 
  \]
\end{eg}

Given two Young diagrams $\imath$ and $\jmath$, write $\imath\sqcup \jmath$ for the Young diagram whose multiset of nonzero row lengths equals the union of those of $\imath$ and $\jmath$. 

For unitary groups, we have the following result.
\begin{thm}
Assume that $\star=A^*$ so that $G=\oU(p,q)$. If there is a decomposition 
 \[
   \check \CO=\check \CO_{\mathrm g} \sqcup \check \CO_{\mathrm b}\sqcup \check \CO_{\mathrm b}
 \]
 such that all nonzero row lengths of $\check \CO_{\mathrm g}$ have the same parity as $p+q$, and all nonzero row lengths of $\check \CO_{\mathrm b}$ have different parity as $p+q$, then 
 \[
\sharp(\Unip_{\check \CO}(G))=
        \sharp\{\CP\in \mathrm{PP}_\star(\check \CO_{\mathrm g})\mid (p_\CP+\abs{\CO_{\mathrm b}}, q_\CP+\abs{\CO_{\mathrm b}})=(p,q)\} 
        \]
If there is no such decomposition, then $\sharp(\Unip_{\check \CO}(G))=0$. 

\end{thm}

\subsection{Orthogonal and symplectic groups: reduction to good parity}

Now we assume that  
\[
\star\in \{ B, D,   B^\C, D^\C, C, C^\C, D^*, C^*, \widetilde C, \widetilde C^\C\}. 
\]
Then there is a unique decomposition  
 \[
   \check \CO=\check \CO_{\mathrm g} \sqcup \check \CO_{\mathrm b}\sqcup \check \CO_{\mathrm b}
 \]
 such that $\check \CO_{\mathrm g}$ has $\star$-good parity in the sense that all its nonzero row lengths are
 \[
 \left\{
     \begin{array}{ll}
       \textrm{even}, &\hbox{if $\star\in \{B, B^\C, \widetilde C, \widetilde C^\C\}$}; \smallskip\\
         \textrm{odd}, &\hbox{if $\star\in \{C, D, C^\C, D^\C, D^*, C^*\}$},
          \end{array}
   \right.
   \]
   and $\check \CO_{\mathrm b}$ has $\star$-bad parity in the sense that 
   all its nonzero row lengths are
 \[
 \left\{
     \begin{array}{ll}
       \textrm{odd}, &\hbox{if  $\star\in \{B, B^\C, \widetilde C, \widetilde C^\C\}$}; \smallskip\\
         \textrm{even}, &\hbox{if  $\star\in \{C, D, C^\C, D^\C, D^*, C^*\}$}.
          \end{array}
   \right.
   \]
For simplicity, put
\[
  l:=\abs{\check \CO_{\mathrm b}}, 
\]
and respectively put
\[
 \begin{array}{rl}
    G_l:=  & \SO(p-l,q-l)\ \  (\textrm{when $p,q\geq l$}),   \ \     \SO_{n-2l}(\C),  \  \   \Sp_{2n-2l}(\R), \  \ \Sp_{2n-2l}(\C), \smallskip \\
  & \oO^*(2n-2l), \ \  \Sp(p-\frac{l}{2},q-\frac{l}{2}) \ \  (\textrm{when $p,q\geq 2l$}),  \ \   \widetilde \Sp_{2n-2l}(\R) \ \ \textrm{or }  \ \  \Sp_{2n-2l}(\C),
     \end{array}
\]
when \[
 \begin{array}{rl}
    G=  & \SO(p,q)   \ \     \SO_{n}(\C),  \  \   \Sp_{2n}(\R), \  \ \Sp_{2n}(\C), \smallskip \\
  & \oO^*(2n), \ \  \Sp(p,q),  \ \   \widetilde \Sp_{2n}(\R) \ \ \textrm{or }  \ \  \Sp_{2n}(\C).
     \end{array}
\]

 
\begin{thm}\label{reduction}
(a) Assume that $\star\in \{B,D\}$ so that $G=\SO(p,q)$. Then
 \[
\sharp(\Unip_{\check \CO}(G))=\left\{
     \begin{array}{ll}
      \sharp(\Unip_{\check \CO_{\mathrm g}}(G_l))\times \sharp(\Unip_{\check \CO_{\mathrm b}}(\GL_l(\R)) ), &\hbox{if $p,q\geq l$}; \smallskip\\
           0, &\hbox{otherwise.}  \end{array}
   \right.
        \]

(b) Assume that $\star=C^*$ so that $G=\Sp(p,q)$. Then
 \[
\sharp(\Unip_{\check \CO}(G))=\left\{
     \begin{array}{ll}
      \sharp(\Unip_{\check \CO_{\mathrm g}}(G_l )), &\hbox{if $p,q\geq \frac{l}{2}$}; \smallskip\\
           0, &\hbox{otherwise.}  \end{array}
   \right.
        \]


(c) Assume that $\star\in \{C,\widetilde C\}$ so that $G=\Sp_{2n}(\R)$ or $\widetilde \Sp_{2n}(\R)$. Then
 \[
\sharp(\Unip_{\check \CO}(G))=
      \sharp(\Unip_{\check \CO_{\mathrm g}}(G_l))\times \sharp(\Unip_{\check \CO_{\mathrm b}}(\GL_l(\R)) ).       \]
        
        
(d)  Assume that $\star\in \{B^\C, D^\C, C^\C,\widetilde C^\C, D^*\}$. Then
 \[
\sharp(\Unip_{\check \CO}(G))=
      \sharp(\Unip_{\check \CO_{\mathrm g}}(G_l)).      
       \]
        
        



\end{thm}


\subsection{Orthogonal and symplectic groups: the case of good parity}
Now we further assume that  $\check \CO$ has $\star$-good parity, namely $\check \CO=\check \CO_{\mathrm g}$. By Theorem \ref{reduction}, the counting problem in general is reduced to this  case.



\delete{
\begin{defn}
 A $\star$-pair is a pair  $(i,i+1)$ of consecutive positive integers such that
\[
   \left\{
     \begin{array}{ll}
      i\textrm{ is odd}, \quad &\textrm{if $\star\in\{C, \widetilde{C}, C^*, C^\C, \widetilde C^\C\}$};  \\
      i \textrm{ is even}, \quad &\textrm{if $\star\in\{B, D, D^*, B^\C, D^\C\}$}. \\
       \end{array}
   \right.
\]
A $\star$-pair   $(i,i+1)$ is said to be primitive in $\check \CO$ if    $\mathbf r_i(\check \CO)-\mathbf r_{i+1}(\check \CO)$ is positive and even. 
Denote $\mathrm{PP}_\star(\check \CO)$ the  set of all $\star$-pairs that are primitive in $\check \CO$.
\end{defn}
}



\begin{defn}
 A $\star$-pair is a pair  $(i,i+1)$ of consecutive positive integers such that
\[
   \left\{
     \begin{array}{ll}
      i\textrm{ is odd}, \quad &\textrm{if $\star\in\{C, \widetilde{C}, C^*, C^\C, \widetilde C^\C\}$};  \\
      i \textrm{ is even}, \quad &\textrm{if $\star\in\{B, D, D^*, B^\C, D^\C\}$}. \\
       \end{array}
   \right.
\]
A $\star$-pair   $(i,i+1)$ is said to be
\begin{itemize}
\item
vacant in $\check \CO$, if $\mathbf r_i(\check \CO)=\mathbf r_{i+1}(\check \CO)=0$;
\item
balanced in $\check \CO$,  if  $\mathbf r_i(\check \CO)=\mathbf r_{i+1}(\check \CO)>0$;
\item
tailed in $\check \CO$,  if  $\mathbf r_i(\check \CO)-\mathbf r_{i+1}(\check \CO)$ is positive and odd;
\item
primitive in $\check \CO$, if    $\mathbf r_i(\check \CO)-\mathbf r_{i+1}(\check \CO)$ is positive and even.
\end{itemize}
Denote $\mathrm{PP}_\star(\check \CO)$ the  set of all $\star$-pairs that are primitive in $\check \CO$.
\end{defn}



\begin{thm}\label{complex}
Assume that $\star\in \{B^\C,D^\C, C^\C, \widetilde C^\C\}$. Then
 \[
\sharp(\Unip_{\check \CO}(G))=2^{\sharp(\mathrm{PP}_\star(\check \CO))} .        \]
\end{thm}



We attach to $\check \CO$ a pair of Young diagrams
\[
(\imath_{\check \CO}, \jmath_{\check \CO}):=(\imath_\star(\check \CO), \jmath_\star(\check \CO)),
\]
 as follows.

\medskip

\noindent
{\bf The case when $\star=\{B, B^\C\}$.} In this case,
 \[
   \mathbf c_{1}(\jmath_{\check \CO})=\frac{\mathbf r_1(\check \CO)}{2},
\]
and for all $i\geq 1$,
\[
\left (\mathbf c_{i}(\imath_{\check \CO}), \mathbf c_{i+1}(\jmath_{\check \CO})\right )=
            \left (\frac{\mathbf r_{2i}(\check \CO)}{2},  \frac{\mathbf r_{2i+1}(\check \CO)}{2}\right ).
\]

\medskip

\noindent
{\bf The case when $\star\in \{\widetilde C, \widetilde C^\C\}$.}  In this case, for all $i\geq 1$,
\[
(\mathbf c_{i}(\imath_{\check \CO}), \mathbf c_{i}(\jmath_{\check \CO}))=
           \left (\frac{\mathbf r_{2i-1}(\check \CO)}{2},  \frac{\mathbf r_{2i}(\check \CO)}{2}\right).
\]

\medskip

\noindent
{\bf The case when $\star\in \{C,C^*, C^\C\}$.} In this case, for all $i\geq 1$,
\[
(\mathbf c_{i}(\jmath_{\check \CO}), \mathbf c_{i}(\imath_{\check \CO}))=
   \left\{
     \begin{array}{ll}
        (0,  0), &\hbox{if $(2i-1, 2i)$ is vacant  in $\check \CO$};\smallskip\\
        (\frac{\mathbf r_{2i-1}(\check \CO)-1}{2},  0), & \hbox{if $(2i-1, 2i)$ is tailed in $\check \CO$};\smallskip\\
                  (\frac{\mathbf r_{2i-1}(\check \CO)-1}{2},  \frac{\mathbf r_{2i}(\check \CO)+1}{2}), &\hbox{otherwise}.\\
            \end{array}
   \right.
\]
\medskip

\noindent
{\bf The case when $\star\in \{D,D^*, D^\C\}$.} In this case,
 \[
   \mathbf c_{1}(\imath_{\check \CO})= \left\{
     \begin{array}{ll}
      0,  &\hbox{if $\mathbf r_1(\check \CO)=0$}; \smallskip\\
       \frac{\mathbf r_1(\check \CO)+1}{2},   &\hbox{if $\mathbf r_1(\check \CO)>0$},\\
            \end{array}
   \right.
 \]
and for all $i\geq 1$,
\[
(\mathbf c_{i}(\jmath_{\check \CO}), \mathbf c_{i+1}(\imath_{\check \CO}))=
   \left\{
     \begin{array}{ll}
        (0,  0), &\hbox{if $(2i, 2i+1)$ is vacant in $\check \CO$};\smallskip\\
      \left  (\frac{\mathbf r_{2i}(\check \CO)-1}{2},  0\right ), & \hbox{if $(2i, 2i+1)$ is tailed in $\check \CO$};\smallskip\\
                \left  (\frac{\mathbf r_{2i}(\check \CO)-1}{2},  \frac{\mathbf r_{2i+1}(\check \CO)+1}{2}\right ), &\hbox{otherwise}.\\
            \end{array}
   \right.
\]




\begin{eg} Suppose that $\star=C$, and $\check \CO$ is the following Young diagram which has $\star$-good parity.  
\begin{equation*}\label{eq:sp-nsp.C}
  \tytb{\ \ \ \ \  , \ \ \  , \ \ \ , \ \ \  , \ \ \ , \  ,\  }
   \end{equation*}
   Then 
\[
  \mathrm{PP}_\star(\check \CO)=\{(1,2), (5,6)\}
\]
and
\[
  (\imath_{\check \CO}, \jmath_{\check \CO})= \tytb{\ \ \ ,\ \  } \times \tytb{\ \ \ , \  }. 
\]
 
   
\end{eg}



\delete{
\begin{eg} Suppose that $\star=C$, and $\check \CO$ is the following Young diagram which has $\star$-good parity.  
\begin{equation*}\label{eq:sp-nsp.C}
  \tytb{\ \ \ \ \  , \ \ \  , \ \ \ , \ \ \  , \ \ \ , \  ,\  }
   \end{equation*}
   Then 
\[
  \mathrm{PP}_\star(\check \CO)=\{(1,2), (5,6)\}.
\]
and $(\imath_\star(\check \CO, \wp), \jmath_\star(\check \CO, \wp))$ %\in \mathrm{BP}_\star(\check \CO)$ 
has the  following form. 
 
\begin{equation*}\label{eq:sp-nsp.C}
\begin{array}{rclcrcl}
  \wp=\emptyset & : & \tytb{\ \ \ ,\ \  } \times \tytb{\ \ \ , \  }  & \qquad \quad &  \wp=\{(1,2)\}& : & \tytb{\ \ \  , \ \ , \   } \times \tytb{\ \ \  } \medskip \medskip \medskip \\
    \wp=\{(5,6)\} & : & \tytb{\ \ \ ,\ \ \ } \times \tytb{\ \ , \   }  & \qquad \quad &  \wp=\{(1,2), (5,6)\}  & : & \tytb{\ \ \  , \ \ \ ,  \ } \times \tytb{\ \   } \\
  \end{array}
  \end{equation*}
   
\end{eg}
}

Here and henceforth, when no confusion is possible, we write $\alpha\times \beta$ for a pair $(\alpha, \beta)$.  We will also write $\alpha\times \beta\times \gamma$ for a triple $(\alpha, \beta, \gamma)$.


We introduce two more symbols $B^+$ and $B^-$, and make the following definition.
 \begin{defn}\label{defpbp0}
 A painted bipartition is a triple $\tau=(\imath, \CP)\times (\jmath, \cQ)\times \alpha$, where $(\imath, \CP)$ and $ (\jmath, \mathcal Q)$ are painted Young diagrams, and $\alpha\in \{B^+,B^-, C,D,\widetilde {C}, C^*, D^*\}$, subject to the following conditions:
 \begin{itemize}
  \delete{\item
 $(\imath, \jmath)\in \mathrm{BP}_\alpha$ if $\alpha\notin\{B^+,B^-\}$, and  $(\imath, \jmath)\in \mathrm{BP}_{B}$ if $\alpha\in\{B^+,B^-\}$;}

 \item
 $\CP^{-1}(\bullet)=\mathcal Q^{-1}(\bullet)$;
 \item
 the image of $\CP$ is contained in
 \[
 \left\{
     \begin{array}{ll}
         \{\bullet, c\}, &\hbox{if $\alpha=B^+$ or $B^-$}; \smallskip\\
            \{\bullet,  r, c,d\}, &\hbox{if $\alpha=C$}; \smallskip\\
          \{\bullet, s, r, c,d\}, &\hbox{if $\alpha=D$}; \smallskip\\
            \{\bullet, s, c\}, &\hbox{if $\alpha=\widetilde{ C}$}; \smallskip \\
        \{\bullet\}, &\hbox{if $\alpha=C^*$}; \smallskip \\
          \{\bullet, s\}, &\hbox{if $\alpha=D^*$},\\
            \end{array}
   \right.
 \]
 \item
 the image of $\mathcal Q$ is contained in
 \[
 \left\{
     \begin{array}{ll}
         \{\bullet, s, r, d\}, &\hbox{if $\alpha=B^+$ or $B^-$}; \smallskip\\
           \{\bullet, s\}, &\hbox{if $\alpha=C$}; \smallskip\\
           \{\bullet\}, &\hbox{if $\alpha=D$}; \smallskip\\
             \{\bullet, r, d\}, &\hbox{if $\alpha=\widetilde{ C}$}; \smallskip\\
        \{\bullet, s,r\}, &\hbox{if $\alpha=C^*$}; \smallskip \\
          \{\bullet, r\}, &\hbox{if $\alpha=D^*$}.
            \end{array}
   \right.
 \]

 \end{itemize}
 \end{defn}

  %\begin{remark}
 %The set of painted bipartition counts the multiplicities of an irreducible representation of $W_{r_{\fgg}}$ occurs in the coherent continuation representation at the infinitesimal character of the trivial representation.
%For the relationship between painted bipartitions and the coherent continuation representations of Harish-Chandra modules, see \cite{Mc}.
%\end{remark}

For any painted bipartition $\tau$ as in Definition \ref{defpbp0}, we write
\[
  \imath_\tau:=\imath,\ \cP_\tau:=\cP,\  \jmath_\tau:=\jmath,\  \cQ_\tau:=\cQ,\ \alpha_\tau:=\alpha,
\]
and
\[
  \star_\tau:= \left\{
     \begin{array}{ll}
         B, &\hbox{if $\alpha=B^+$ or $B^-$}; \smallskip\\
            \alpha, & \hbox{otherwise}.           \end{array}
   \right.
  \]
%Its leading column is then defined to be the first column of $(\jmath, \CQ)$ when $\star_\tau\in \{B, C,C^*\}$,
%and the first column of  $(\imath, \CP)$ when $\star_\tau\in \{\widetilde C, D, D^*\}$.

We further define a pair $(p_{\tau}, q_{\tau})$  of natural numbers given by the following recipe.
 \begin{itemize}
  \item
  If $\star_\tau\in \{B, D, C^*\}$, $(p_\tau, q_\tau)$ is given by counting  the various symbols appearing in $(\imath, \CP)$, $(\jmath, \cQ)$ and $\{\alpha\}$ :
  \begin{equation}\label{ptqt}
  \left\{
     \begin{array}{l}
    p_\tau :=( \# \bullet)+ 2 (\# r) +(\# c )+ (\# d) + (\# B^+);\smallskip\\
    q_\tau :=( \# \bullet)+ 2 (\# s) + (\# c) + (\# d) + (\# B^-).\\
    \end{array}
    \right.
\end{equation}
Here
\[
\#\bullet:=\#(\cP^{-1}(\bullet))+\#(\cQ^{-1}(\bullet))\qquad (\textrm{$\#$ indicates the cardinality of a finite set}),
\]
and the other terms are similarly defined.
\item
If $\star_\tau\in \{C, \widetilde C, D^*\}$,  $p_\tau:=q_\tau:=\abs{\tau}$.
\end{itemize}
\smallskip

We also define a classical group
  \begin{equation}\label{def:Gt}
 G_\tau:= \left\{
     \begin{array}{ll}
         \oO(p_\tau, q_\tau), &\hbox{if $\star_\tau=B$ or $D$}; \smallskip\\
            \Sp_{2\abs{\tau}}(\R), &\hbox{if $\star_\tau=C$}; \smallskip\\
           \widetilde{\Sp}_{2\abs{\tau}}(\R), &\hbox{if $\star_\tau=\widetilde{ C}$}; \smallskip \\
        \Sp(\frac{p_\tau}{2}, \frac{q_\tau}{2}), &\hbox{if $\star_\tau=C^*$}; \smallskip \\
          \oO^*(2\abs{\tau}), &\hbox{if $\star_\tau=D^*$}.\\
            \end{array}
   \right.
\end{equation}

Define
\[
\mathrm{PBP}_\star(\check \CO):=\{ 
\tau\textrm{ is a painted bipartition}  \mid    \star_\tau = \star, 
 \text{ and }    (\imath_\tau,\jmath_\tau) = (\imath_{\check \CO}, \jmath_{\check \CO})\}.
\]


\delete{
\[
\begin{array}{rl}
\mathrm{PBP}_\star(\check \CO):=\{ &
\tau\textrm{ is a painted bipartition}  \mid    \star_\tau = \star, 
 \text{ and } \\  & (\imath_\tau,\jmath_\tau) = (\imath_{\check \CO}, \jmath_{\check \CO})\}.
\end{array}
\]
}




\begin{eg} Suppose that $\star=B$ and
\[
 \check \CO =\tytb{\ \ \ \ \ \ , \ \ \ \ \ \ , \ \ , \ \ , \ \  }
\]
Then
\[
\tau:= \tytb{\bullet \bullet ,\bullet , c } \times \tytb{\bullet \bullet  d ,\bullet , d }\times B^+\in \mathrm{PBP}_{\star}(\check \CO),
\]
and
\[
G_\tau=\SO(10,9).
 \]
\end{eg}





\begin{thm}\label{countup}
Assume that $\star\in  \{B, C,D,\widetilde {C}, C^*, D^*\}$. Then
 \[
\sharp(\Unip_{\check \CO}(G))\leq 2^{\sharp(\mathrm{PP}_\star(\check \CO))} \cdot  \sharp \{\tau\in \mathrm{PBP}_\star(\check \CO)\mid  G_\tau=G\}.     
   \]
\end{thm}

In \cite{BMSZ2}, we will construct sufficiently many representations in $\Unip_{\check \CO}(G)$, and show that the equality holds in \Cref{countup}.


\section{Counting theorem for the real classical groups of type BCD}







In this section, we assume $\star \in \set{B,\wtC,C,C^{*}, D, D^{*}}$.

Let
\[
  \Gc =
  \begin{cases}
    \SO(2n+1,\bC) & \text{if } \star \in \set{B},\\
    \Sp(2n,\bC) & \text{if } \star \in \set{\wtC, C, C^{*}},\\
    \SO(2n,\bC) & \text{if } \star \in \set{D,D^{*}}.\\
  \end{cases}
\]
and the dual group of $\Gc$ is defined to be
\[
  \ckGc =
  \begin{cases}
    \Sp(2n,\bC) & \text{if } \star \in \set{B,\wtC},\\
    \SO(2n+1,\bC) & \text{if } \star \in \set{C, C^{*}},\\
    \SO(2n,\bC) & \text{if } \star \in \set{D,D^{*}}.\\
  \end{cases}
\]

Suppose $\ckcO\in \Nil(\ckG_{\bC})$.
Let $\ckcO = \ckcO_{b}\cuprow \ckcO_{g}$ be
the decomposition of $\ckcO$ into good and bad parity parts.

Let $\IC_{\star}(n)$ be the set of real groups given by
\[
  \IC_{\star}(n)
  :=
  \begin{cases}
    \Set{\SO(2p+1,2q)| p,q\in \bN \text{ and } p+q = n}
    & \text{if } \star = B\\
    \set{\Sp(2n,\bR)}
    & \text{if } \star = C\\
    \set{\Mp(2n,\bR)}
    & \text{if } \star = \wtC\\
    \Set{\Sp(p,q)| p,q\in \bN \text{ and } p+q = n}
    & \text{if } \star = C^{*}\\
    \Set{\SO(p,q)| p,q\in \bN \text{ and } p+q = 2n}
    & \text{if } \star = D\\
    \Set{\rO^{*}(2n)}
    & \text{if } \star = D^{*}\\
  \end{cases}
\]
and $\IC_{\star} = \bigcup_{n\in \bN} \IC_{\star}(n)$.

Let $ \unip_{\star}(\ckcO)$ be the set of special unipotent representations
of groups in $\IC_{\star}$ attached to $\ckcO$.


\begin{thm}
  Suppose $\ckcO = \ckcO_{g}$. Then
  \[
    \abs{\unip_{\star}(\ckcO)} \leq \abs{\PBPes(\ckcO)}.
  \]
  % \[
  %   \abs{\unip_{\star}(\ckcO)} = \begin{cases}
  %     2 \abs{\PBPes(\ckcO)} & \text{if } \star \in \set{B,D}\\
  %     \abs{\PBPes(\ckcO)} & \text{if } \star \in \set{\wtC,C,C^{*}, D^{*}}
  %     \end{cases}
  % \]
\end{thm}


\begin{thm}
  Let $\ckcO'_{b}$ be the partition such that $\ckcO'_{b}\cuprow \ckcO'_{b}
  = \ckcO_{b}$    % $ such that $ = \left(\dBV(\ckcO_{b})\right)^{t}$
  and
  \[\star' =
  \begin{cases}
     A^{\bR} & \star \in \set{B,\wtC,C,D}\\
     A^{\bH} & \star \in \set{C^{*},D^{*}}\\
    \end{cases}
  \]

  Then we have the following bijection
  \[
    \begin{array}{ccc}
    \Unip_{\star'}(\ckcO'_{b}) \times \Unip_{\star}(\ckcO_{g})
      & \longrightarrow & \Unip_{\star}(\ckcO)\\
      (\pi',\pi_{0})& \mapsto & \pi'\rtimes \pi_{0}.
    \end{array}
  \]
\end{thm}

% \begin{thm}
%   Suppose $\ckcO = \ckcO_{g}$. Then
%   \[
%     \abs{\unip_{\star}(\ckcO)}\leq
%     \abs{\PBPes(\ckcO)}.
%     %& \text{if } \star \in \set{\wtC,C,C^{*}, D^{*}}
%     % \begin{cases}
%     %   2 \abs{\PBPes(\ckcO)} & \text{if } \star \in \set{B,D}\\
%     %   \abs{\PBPes(\ckcO)} & \text{if } \star \in \set{\wtC,C,C^{*}, D^{*}}
%     %   \end{cases}
%   \]
% \end{thm}


\section{Counting formula}


 \begin{lem}[{\cite{BVUni}*{(5.26), Proposition~5.28}}]\label{lem:lcell.BV}
  Let $\ckcO$ be a nilpotent orbit in $\ckGc$ and $\lamck$ be the infinitesimal
  character attached to $\ckcO$.
  Then the set
  \[
    \LC(\ckcO) := \Set{ \sigma \in \cD_{\lamck}| [1_{W_{\mu}}:\sigma]\neq 0}
  \]
  is a left cell in $\cD_{\lamck}$ given by
  \[
    (J_{\Wlamck}^{\WLamck} \sgn )\otimes \sgn.
  \]
  Moreover, the multiplicity $[1_{W_{\mu}}:\sigma]$ is one
  when $\sigma\in \LC(\ckcO)$.
\end{lem}

\begin{cor}
  Under the notation of \Cref{lem:lcell.BV}, we have
  \[
    \abs{\PI{\mu}(G)} \leq \sum_{\sigma\in \LC(\ckcO)} [\sigma: \Coh_{[\mu]}(G)]
  \]
\end{cor}




\def\Grt{\cG}

In this paper, we will consider the following cases.

\begin{eg}
  Suppose $\cM$ is a field of characteristic zero and
  \[
    F\otimes m := \dim(F)\cdot m \quad \text{for all } F\in \cG \text{ and
    } m\in \cM.
  \]
  We let $\cM_{\mu} = \cM$ for every $\mu\in \Lambda$. Then the set of
  $W$-harmonic polynomials on $\fhh$ is naturally identified with
  $\Coh_{[\lambda]}(\cM)$ via the restriction on $[\lambda]$ by Vogan
  \cite{VGK}*{Lemma~4.3}. \trivial{ Note that the polynomials are $W$-harmonic
    not necessary $W_{[\lambda]}$-harmonic. ($W_{[\lambda]}$-invariant
    differential operators are more than $W$-invariant differential operators.)
  }
\end{eg}


\begin{eg}\label{eg:hw}
  Fix a Borel subalgebra $\fbb = \fhh\oplus \fnn \subset \fgg$, let
  $\Grt(\fgg,\fhh,\fnn)$ be the Grothendieck group of the category $\cO$ with
  coefficients in $\bC$, i.e. the category of finitely generated
  $\cU(\fgg)$-modules with semisimple $\fhh$-action and locally finite
  $\fnn$-action. For $\lambda\in \fhh^{*}$, let $\Grt_{W\cdot \lambda}(\fgg,\fhh,\fnn)$
  be the subgroup spanned by $\fgg$-modules with infinitesimal character
  $W\cdot \lambda$. Here $\cG$ acts on $\Grt(\fgg,\fhh,\fnn)$ via the tensor
  product of $\fgg$-modules.

  To ease the notation, we write
  \[
    \Coh_{[\lambda]}(\fgg,\fhh,\fnn) := \Coh_{[\lambda]}(\Grt(\fgg,\fhh,\fnn)).
  \]

  % Verma modules gives a basis of We now review the well understood structure
  % of $\Coh_{[\lambda]}$. is well understood.
  For $\lambda\in \fhh^{*}$, let $\rho := \sum_{\alpha\in \WT{\fnn}} \alpha$,
  \[
    M(\lambda) := \cU(\fgg)\otimes_{\cU(\fbb)} \bC_{\lambda-\rho}
  \]
  be the Verma module with highest weight $\lambda-\rho$ and $L(\lambda) $ be
  the unique irreducible quotient of $M(\lambda)$.

  Each $w\in W$ defines a coherent family %such that
  \[
    M_w(\mu) := M(w\cdot \mu) \quad \forall \mu \in [\lambda].
    % M_w(\mu) := M(w \cdot \mu) \quad \forall \mu \in [\lambda].
  \]
  % where $w_{0}$ is the longest element in $W$.

  The map
  \[
    \begin{array}{ccc}
      \bC[W] & \longrightarrow & \Coh_{[\lambda]}(\fgg,\fhh,\fnn)  \\
      w& \mapsto &M_{w}
    \end{array}
  \]
  is $W_{[\lambda]}$-equivariant isomorphism where $W_{[\lambda]}$ acts on $\bC[W]$ by right
  translation.

  One of the crucial property is that each irreducible module can be fitted into
  a coherent family. %in other words.
  More precisely, the evaluation map descents to yields an isomorphism
  $\bev{\mu}$ in the following diagram.
  \begin{equation}\label{eq:bev.catO}
    \begin{tikzcd}
      \Coh_{[\lambda]}(\fgg,\fhh,\fnn)\ar[r,"\Theta\mapsto \Theta(\mu)"] \ar[d]&
      \Grt_{W\cdot \mu}(\fgg,\fhh,\fnn)\\
      \left(\Coh_{[\lambda]}(\fgg,\fhh,\fnn)\right)_{W_{\mu}} \ar[ru,hook,two heads,"\bev{\mu}"']
    \end{tikzcd}
  \end{equation}

  \trivial[]{ The subjectivity is because of Verma modules form a basis of the
    category $\cO$. The LHS of $\bev{\mu}$ has dimension $\abs{W/W_{\mu}}$, the
    RHS has dimension $W\cdot \mu$. Now the isomorphism follows by dimension
    counting. }
  % the The space $\Coh_\Lambda(\cG(\fgg,\fhh,\fnn))$ and
  % $\Coh_{\Lambda(\cG)}$defined similarly.

  % Note that the lattice $\Lambda$ is stable under the $\Wlam$ action.
\end{eg}

\begin{eg}\label{eg:Coh.HC}
  Suppose $G$ is a reductive Lie group in the Harish-Chandra class. Let
  $\Grt(G)$ be the Grothendieck group of finite length admissible $G$-modules
  and $\Grt_{\mu}(G)$ be the subgroup of $\Grt(\fgg,K)$ generated by the set of
  irreducible $G$-modules with infinitesimal character $\mu$.

  By \cite{Vg}*{0.4.6}, we can naturally identify $Q$ with the set of
  $H^{s}$-weights consisting the characters occurs in $S(\fgg)$ where $H^{s}$ is
  a maximally split Cartan in $G$. Therefore, the set of irreducible
  $G$-submodules occur in $S(\fgg)$ is also naturally identified with $Q/W=\cG$.
  We let $\cG$ acts on $\Grt(G)$ by the tensor product of $G$-modules.

  \trivial[]{ Note that by the assumption that $G$ is in the Harish-Chandra
    class, each irreducible $\fgg$-submodule $F$ embeds in $S(\fgg)$ is
    automatically globalized to a $G$-module. The point is that the
    globalization is independent of the embedding of $F$ in $S(\fgg)$! }

  We write
  \[
    \Coh_{[\lambda]}(G) := \Coh_{[\lambda]}(\Grt(G))
  \]
  for the space of coherent family of $G$-modules.
  % Then $\Coh_\Lambda(\cG(\fgg,K))$ is the group of coherent families of
  % Harish-Chandra modules. The space $\Coh_\Lambda(\cG(\fgg,K))$ is equipped
  % with a $\WLam$-action by
  % \[
  %   w\cdot f(\mu) = f(w^{-1} \mu) \qquad \forall \mu\in \Lambda, w\in \WLam, f \in \Coh_{\Lambda}(\Grt(\fgg,K)).
  % \]
\end{eg}

\begin{eg}
  Fix a $\Gc$-invariant closed subset $\sfS$ in the nilpotent cone of $\fgg$.
  Let $\Grt_{\sfS}(G)$ be the Grothendieck group of finite length admissible
  $G$-modules whose complex associated varieties are contained in $\sfS$. Define
  \[
    \Grt_{\mu,\sfS}(G) := \Grt_{\mu}(G)\cap \Grt_{\sfS}(G).
  \]
  and write
  \[
    \Coh_{[\lambda],\sfS}(G):= \Coh_{[\lambda]}(\Grt_{\sfS}(\fgg,K)).
  \]

  Note that
  \[
    \AVC(\pi\otimes F) = \AVC(\pi).
  \]
  for each finite length $G$-module $\pi$ and finite dimensional $G$-module $F$.
  Therefore the $W_{[\lambda]}$-module
  \[
    \Coh_{[\lambda],\sfS}(G) = (\ev{\mu})^{-1}(\Grt_{\mu,\sfS}(G))
  \]
  for any regular $\mu\in [\lambda]$.

  Similarly, we define the space $\Coh_{[\lambda],\sfS}(\fgg,\fhh,\fnn)$ of
  coherent families in category $\cO$ whose associated variety are contained in
  $\sfS$. In particular,
  \[
    \Coh_{[\lambda],\sfS}(\fgg,\fhh,\fnn) = (\ev{\mu})^{-1}(\Grt_{\mu,\sfS}(\fgg,\fhh,\fnn)).
  \]
  % now is also a $\WLam$-submodule of $\Coh_\Lambda(\cG(\fgg,K))$.
\end{eg}


% \begin{eg}
%   For each infinitesimal character $\chi$ and a close $G$-invariant set
%   $\cZ\in \cN_{\fgg}$. Let $\Grt_{\,\cZ}(\fgg,K)$ be the Grothendieck group of
%   $(\fgg,K)$-module with infinitesimal character $\chi$ and complex associated
%   variety contained $\cZ$. Similarly, let $\Grt_{\chi,\cZ}()$
% \end{eg}

A remarkable fact that the diagram \Cref{eq:bev.catO} still holds for
$\Coh_{[\lambda],\sfS}(G)$. This is one of the first step towards the counting
of the set
\[
  \Irr_{\mu,\sfS}(G):=\Set{\pi \in \Irr_{\mu}(G)| \text{$\AVC(\pi)\subset \sfS$} }.
\]
where
\[
  \Irr_{\mu}(G):=\Set{\pi \in \Irr(G)| \pi \text{ has infinitesimal character
    }\mu}.
\]


\begin{lem}
  For each $\mu$ and closed $\Gc$-invariant subset $\sfS$ in the nilpotent cone
  of $\fgg$, we have an isomorphism
  \[
    \bev{\mu}\colon \left(\Coh_{[\mu],\sfS}(G)\right)_{W_{\mu}} \longrightarrow \Grt_{\mu,\sfS}(G).
  \]
  In particular,
  \[
    \abs{\Irr_{\mu,\sfS}(G)} = [1_{W_{\mu}}:\Coh_{[\mu],\sfS}(G)]
  \]
  % \[
  %   \dim {\barmu} = \dim (\cohm)_{W_\mu} = [\cohm, 1_{W_\mu}].
  % \]
\end{lem}
\begin{proof}
  This is a consequence of the formal properties of the translation functor,
  especially the theory of $\tau$-invariant.

  \def\Parm{\mathrm{Parm}} \def\cof{\Theta}


  {\bf The properties of the coherent family and translation principal:} (We
  refers to \cite{Vg}*{Section~7} for the proofs which also work in our
  (possibly non-linear) setting.)

  \begin{enumerate}[label=(\alph*)]
    \item \label{it:t1} The evaluation map
          \[
          \ev{\mu}\colon \Coh_{[\mu]}(G)\rightarrow \Grt_{\mu}(G)
          \]
          is surjective for any $\mu \in [\mu]$, see \cite{Vg}*{Theorem~7.2.7}.
  \end{enumerate}
  From now on we fix a regular element $\lambda \in [\mu]$ such that $\mu$ is
  dominant with respect to $R^{+}_{[\lambda]}$
  \begin{enumerate}[resume*]
    \item \label{it:t2} The evaluation map $\ev{\lambda}$ at $\lambda$ is an
          isomorphism \cite{Vg}*{Proposition~7.2.27}.
  \end{enumerate}
  For each $\pi\in \Grt_{\lambda}(G)$, let
  \[
    \Theta_{\pi}:= (\ev{\lambda})^{-1}(\pi)
  \]
  be the unique coherent family such that $\Theta_{\pi}(\lambda) = \pi$,
  \begin{enumerate}[resume*]
    \item \label{it:t3} If $\pi\in \Irr_{\lambda}(G)$, then $\Theta_{\pi}(\mu)$
          is either zero or an irreducible $G$-module
          \cite{Vg}*{Proposition~7.3.10, Corollary~7.3.23}.
    \item \label{it:t4} For $\pi\in \Irr_{\lambda}(G)$,
          \[
          \AV(\Theta_{\pi}(\mu)) = \AV(\pi)
          \]
          whenever $\mu$ is dominant and $\Theta_{\pi}(\mu)$ non-zero.
          \trivial[]{ This because
          $\pi = \psi_{\mu}^{\lambda}(\Theta_{\pi}(\mu))$ and
          $\Theta_{\pi}(\mu) = \psi_{\lambda}^{\mu}(\pi)$. Here is the
          translation functor from $\lambda$ to $\mu$ see
          \cite{Vg}*{Definition~4.5.7}. Translation dose not increase the
          associated variety. }
    \item \label{it:t5} If $\pi$ and $\pi'$ are in $\Irr_{\lambda}(G)$ such that
          $\Theta_{\pi}(\mu) = \Theta_{\pi'}(\mu)$ is non-zero, then $\pi=\pi'$.
  \end{enumerate}
  For $\pi\in \Irr_{\lambda}(G)$, define the $\tau$-invariant of $\pi$ to be
  \begin{equation}\label{eq:taupi}
    \tau(\pi) := \Set{\alpha\in R^{+}_{[\lambda]}|
      \begin{array}{l}
        \text{$\alpha$ is simple and }\\
        s_{\alpha}\cdot \Theta_{\pi}(\lambda) = -\Theta_{\pi}(\lambda)
      \end{array}
    }
  \end{equation}
  \begin{enumerate}[resume*]
    \item
          \label{it:t6}
          $\Theta_{\pi}(\mu) =0$ if and only if
          $\tau(\gamma)\cap R_\mu \neq \emptyset$
          \cite{Vg}*{Corollary~7.3.23~(c)}.
  \end{enumerate}

  Now we start to prove the lemma. By the translation principle,
  \[
    \begin{split}
      & \Set{\Theta_{\pi}(\mu)| \pi\in \Irr_{\lambda, \sfS}(G)
        \text{ s.t. } \Theta_{\pi}(\mu)\neq 0} \\
      = & \Set{\Theta_{\pi}(\mu)| \pi\in \Irr_{\lambda, \sfS}(G) \text{ s.t.
        } \tau(\pi)\cap R_{\mu}= \emptyset}.
    \end{split}
  \]
  forms a basis of $\Grt_{\mu,\sfS}(G)$. \trivial{ The set consists of distinct
    (so linearly independent) irreducible $G$-modules by \ref{it:t3} and
    \ref{it:t5}. They are spanning set by \ref{it:t1}. For the support
    condition, see \ref{it:t4}. The $\tau$-invariant condition is by
    \ref{it:t6}.
    % by \ref{it:t6} and
  } Hence
  \[
    \begin{split}
      \ker \ev{\mu} = & \Span \Set{\Theta_{\pi}| \pi\in \Irr_{\lambda, \sfS}(G) \text{
          s.t. }
        \tau(\pi)\cap R_{\mu}\neq \emptyset}\\
      \subseteq & \Span \Set{\half(\Theta_\pi - s_{\alpha}\cdot \Theta_{\pi}) | \pi\in \Irr_{\lambda, \sfS}(G) \text{
          and } \alpha \in
        \tau(\pi)\cap R_{\mu}}\\
      & \ \ \ \  \text{(by the definition of $\tau(\pi)$ in \eqref{eq:taupi}.)} \\
      \subseteq &\Span\Set{\Theta- w\cdot \Theta |\Theta\in \Coh_{[\mu],\sfS}(G)} \\
      \subseteq & \ker \ev{\mu}. \\
      &\ \ \ \ \text{(by
        $w\cdot \Theta(\mu) = \Theta(w^{-1}\cdot \mu)=\Theta_\pi(\mu)$)}
    \end{split}
  \]
  Since
  $\left(\Coh_{[\mu],\sfS}(G)\right)_{W_{\mu}} =\Coh_{[\mu],\sfS}(G)\slash \Span\set{\Theta- w\cdot \Theta |\Theta\in \Coh_{[\mu],\sfS}(G)} $,
  the lemma follows.
\end{proof}

\section{Primitive ideals and Weyl group representations}

\subsection{Associated varieties of a primitive ideals and double cells in
  $W_{[\lambda]}$}
In this section, we review the notion of double cells and its relation with the
associated varieties of primitive ideals, see \cite{BV2,J.av}.
We retain the notation in \Cref{eg:hw}.

Let $\Prim_{\lambda}(\fgg)$ be the set of primitive ideals in $\cU(\fgg)$ with
infinitesimal character $\lambda$. Let $\lambda \in \fhh^{*}$, each primitive ideal is
the annihilator of a highest weight module by Duflo \cite{Du77}.  In other
words, the following map is
surjective
\[
  \begin{array}{ccl}
    W_{[\lambda]} &\longrightarrow &  \Prim_{\lambda}(\fgg)\\
    w & \mapsto & I(w\cdot \lambda) := \Ann L(w\cdot \lambda).
  \end{array}
\]



%We now recall some results about the blocks in category $\cO$.

By the translation principal, we
concentrate the discussion in the regular infinitesimal character case. From now on, we follows the convention in \cite{BV2}.
Let $\lambda$ be a regular element in $\fhh^{*}$ such that
$R^{+}_{[\lambda]}\subset - \WT{\fnn}$.
\trivial{
  Here $\lambda$ is regular anti-dominant ($\inn{\lambda}{\ckalpha}\notin \bN$ for
  each $\alpha\in \WT{\fnn}$) with respect to the root system defining highest
  weight modules, but it is dominant with respect to $R^{+}_{[\lambda]}$.
}
For each $w\in W_{[\lambda]}$, define
\[
a(w) := \abs{\WT{\fnn}} - \GKdim(L(w\lambda)).
\]
\trivial{
  Suppose $\lambda$ is integral, then
  Under this definition, $a_{w_{0}} = \abs{\WT{\fnn}}$ and
  $a_{e} =0$.
}
For each $w$ one can attach a polynomial $\wtpp_{w}$ such that
$\wtpp_{w}(\mu) = \rank(\cU(\fgg)/\Ann(L(w\mu)))$ when $\mu\in [\lambda]$ is
dominant (i.e. $-\inn{\mu}{\ckalpha}\notin \bN^{+}$ for all
$\alpha\in R^{+}_{[\lambda]}$).
$\wtpp_{w}$ is called the Goldie-rank
polynomial attached to the primitive ideal $\Ann(L(w\lambda))$.
Fix a dominant regular element $\delta$ in $\fhh$ (i.e.
$\inn{\delta}{\alpha}>0$ for each $\alpha\in \WT{\fnn}$).
Let
\[
r_{w} = \sum_{y\in W_{[\lambda]}} a_{y,w} (y^{-1}\delta)^{a(w)} \in S(\fhh)
\]
where $a_{y,w}$ is determined by the equation
\[
  L(w\lambda) = \sum_{y\in W_{[\lambda]}} a_{y,w} M(w\lambda)
\]
in $\Grt(\fgg,\fhh,\fnn)$.
Then $r_{w}$ is a positive multiple of $\wtpp_{w}$ \cite{J2}*{Section~1.4}.

% Let $w_{0}$ (resp. $w_{[\lambda]}$)be the longest element in $W$ (resp.
% $\Wlam$).
A partial order $\leqL$ can be define on $W_{[\lambda]}$ by the following condition
\cite{BV2}*{Proposition~2.9}
\begin{equation}\label{eq:leqL}
  \begin{split}
    w_{1} \leqL w_{2} & \Leftrightarrow
    I(w_{1}\lambda)\subseteq I(w_{2}\lambda)\\
    & \Leftrightarrow
    %[L(w_{2}^{-1}\lambda), L(w_{1}^{-1}\lambda)\otimes S(\fgg)] \neq 0.
    L(w_{2}^{-1}\lambda) \text{ is a subquotient of } L(w_{1}^{-1}\lambda)\otimes S(\fgg).
  \end{split}
\end{equation}

We say $w_{1} \approxL w_{2}$ if and only if $w_{1}\leqL w_{2}\leqL w_{1}$.
For $w\in W_{[\lambda]}$, we call
\[
  \CL_{w} := \Set{ w' \in W_{[\lambda]}| w\approxL w'}
\]
the left cell in $W_{[\lambda]}$ containing $w$.

In summary, we have a bijection
\[
  \begin{array}{ccc}
    W_{[\lambda]}/ \approxL &\longrightarrow & \Prim_{\lambda}(\fgg)\\
    \CL_{w} & \mapsto & \Ann L(w\lambda).
  \end{array}
\]
% here $w$ is a arbitrary element in $\LC$.
% Moreover, for each left cell $\LC_{w}$



The partial order $\leqR$ is defined by
\[
  w_{1}\leqR w_{2} \Leftrightarrow w_{1}^{-1} \leqL w_{2}^{-1}.
\]
The partial order $\leqLR$ is defined to be the minimal partial order
containing $\leqL$ and $\leqR$.
The relation $\approxR$, $\approxLR$, right cell $\CR_{
  w}$ and double cells
$\CLR_{w}$ are defined similarly.

Since the Kazhdan-Lusztig conjecture has been proven (for the integral
infinitesimal character case by \cite{BB,BK} and reduced to the integral
infinitesimal character case by \cite{Soergel} (see \cite{H}*{Section~13.13})),
the definition of the order $\leqL$ is the same as the partial order defined by
Kazhdan-Lusztig  \cite{KL} which only depends on the Coxeter group structure of
$W_{[\lambda]}$, see \cite{BV2}*{Corollary~2.3}.
\trivial[]{
  Note that $x\lneqL y$ implies $a(x)<a(y)$
  and $x\approxLR y$ implies $a(x)=a(y)$.
}

Note that the left cells are exactly the fibers of the map $w\mapsto \wtpp_{w}$.
Take a double cell $\CLR_{w}$ in $W_{[\lambda]}$ and a set of representatives
$\set{w_{1}, w_{2}, \cdots, w_{k}}$ of the left cells in $\CLR_{w}$.
% and decompose it into disjoint union of left-cells
% \[
%   \LRC_{w} = \bigsqcup_{i=1}^{k} \LC_{w_{i}}.
% \]
% Due to Joseph\cite{J2} and Barbasch-Vogan\cite{BV1,BV2}, we have the following
% statements:
%
Due to Barbasch-Vogan\cite{BV1,BV2} and Joseph\cite{J1,J2,J3,J.av},  the following
statements holds:
\begin{itemize}
  \item the set of Goldie rank polynomials
  $\set{\wtpp_{w_{i}}|i = 1,2,\cdots,k}$ form a basis of a special
  representation $\sigma_{w}$ of $W_{[\lambda]}$ realized in
  $S^{a(w)}(\fhh)$;
  \item the multiplicity of $\sigma_{w}$ in $S^{a(w)}(\fhh)$ is one,
  \item $a(w)$ is the minimal degree $m$ such that $\sigma_{w}$ occurs in
  $S^{m}(\fhh)$ which is the fake degree and the generic degree of the special
        representation $\sigma_{w}$. \trivial[]{ When
    $W_{[\lambda]} = W$, this is the definition of the fake degree.
    Otherwise, $\fhh = \fhh_{0}\oplus \fhh^{W_{\lambda}}$ where $\fhh_{0}$
    is the span of coroots of $W_{[\lambda]}$. Then $S(\fhh_{0})$ is embeds
    in $S(\fhh)$. }
  \item the map
  $W_{[\lambda]}/\approxLR \; \ni \CLR_{w}\mapsto \sigma_{w}\in \Irr(W_{[\lambda]})$
  yields a bijection between the set of double cells and the set
  $\Irrsp(W_{[\lambda]})$ of special representations of $W_{[\lambda]}$.
  \item Under the $W$ action, the $W_{[\lambda]}$-module
  $\sigma_{w}\subset S^{a(w)}(\fhh)$ generates an irreducible $W$-module
  $\wtsigma_{w}:=j_{W_{[\lambda]}}^{W}\sigma_{w}$. The $W$-module
  $\wtsigma_{w}$ corresponds to the a nilpotent orbit $\cO_{\wtsigma_{w}}$
  with trivial local system. Now the complex associated variety
  \[
    \Gc^{ad}\AV(L(w\lambda)) = \AV(\Ann(L(w\lambda))) =\overline{\cO_{\wtsigma_{w}}},
  \]
  where $\Gc^{ad}$ is the adjoint group of $\fgg$, see
  \cite{J.av}*{Section~2.10}.
\end{itemize}
\trivial[]{
  Note that the $j$-induction is not injective in general.
  For example,
  $j_{S_{a}\times S_{b}}^{S_{a+b}} \tau_{a}\otimes \tau_{b} = \tau_{a}\cupcol \tau_{b}$
  where $\tau_{a}$, $\tau_{b}$ are partitions.
}

% \subsubsection*{Clan of primitive ideals}
% Under the above notation,
% we say two
Following Joseph, we say two primitive ideals $I(w\lambda)$ and  $I(w'\lambda)$
with $w, w'\in W_{[\lambda]}$are in the same
\emph{clan} if and only if $w\approxLR w'$ or equivalently $\sigma_{w}=\sigma_{w'}$.

Obvious, the associated varieties of two primitive ideals in the
same clan has the same associated variety. However the reverse dose not holds in
the non-integral infinitesimal character case in general.

\subsubsection*{Coherent continuations}
Now we recall
the relationships between cells and the coherent continuation
representations.

For each $w\in W$, we define a coherent family $L_{w}$ by the condition
  $L_{w}(\lambda) = L(w\lambda)$.

For each $\mu\in \fhh^{*}$, let
$\cO_{[\mu]}(\fgg,\fhh,\fnn)$ be the subcategory of the category $\cO(\fgg,\fhh,\fnn)$
consists of modules whose $\fhh$-weights are contained in $[\mu]$,
\[
\Grt_{W\cdot\lambda,[\mu]}(\fgg,\fhh,\fnn) = \Grt_{[\mu]}(\fgg,\fhh,\fnn)\cap \Grt_{W\cdot \lambda}(\fgg,\fhh,\fnn)
\]
which is spanned by a block in the category $\cO$ if
$W\cdot \lambda \cap [\mu+\rho]\neq \emptyset$.

Consider the following subgroup of coherent continuation
\[
 \Coh_{[\lambda]}(\fgg,\fhh,\fnn; [\mu]) =
 \Set{\Theta\in \Coh_{[\lambda]}(\fgg,\fhh,\fnn)| \Theta(\lambda) \in
   \Grt_{W\cdot \lambda, [\mu]}(\fgg,\fhh,\fnn)
 }.
\]

We identify $\bC[W_{[\lambda]}]$ with
$\Coh_{[\lambda]}(\fgg,\fhh,\fnn,[\lambda-\rho])$ via $w \mapsto M_{w}$.

For each $w\in W_{[\lambda]}$,
let
\[
  \begin{split}
\bVR_{w}& :=\Span\Set{L_{w'}|w\leqR w'} \\
\VR_{w}&= \bVR _{w}\left/ \sum_{w\lneqL w'} \bVR_{w'} \right.
\end{split}
\]
By \eqref{eq:leqL},
$\bVR_{w}$ and $\VR_{w}$ are $W_{[\lambda]}$-modules under right
translation/coherent continuation action.

We define left $W_{[\lambda]}$-module $\bVL_{w}$ and $\VL_{w}$ using $\leqL$ and
$W_{[\lambda]}\times W_{[\lambda]}$-module $\bVLR_{w}$ and $\VLR_{w}$ using
$\leqLR$ similarly.

Suppose $\sigma_{1}, \sigma_{2}\in \Irr(W_{[\lambda]})$. We define
\[
  \sigma_{1}\leqLR \sigma_{2} \Leftrightarrow
  \exists w\in W_{[\lambda]} \text{ such that }
  \begin{cases}
  \sigma_{1} \otimes \sigma_{1} \text{ occurs in } \VLR_{w}
  \text{ and } \\
  \sigma_{2} \otimes \sigma_{2} \text{ occurs in } \bVLR_{w}.
\end{cases}
\]
Now $\sigma_{1}\approxLR \sigma_{2}$ if and only if there exists a
$w\in W_{[\lambda]}$ such that $\sigma_{1} \otimes \sigma_{1}$ and
$\sigma_{2} \otimes \sigma_{2}$ both occur in $\VLR_{w}$.
Now $\leqLR$ is a well
defined partial order and  $\approxLR$ is an equivalent
relation on $\Irr(W_{[\lambda]})$ respectively.
We write $\LRC_{\sigma}\subseteq \Irr(\Wlam)$ for the double cell containing
$\sigma$.
\trivial[]{
  A priori $\sigma_{1}\approxLR \sigma_{2}\Leftrightarrow
  \sigma_{1}\leqLR \sigma_{2}\leqLR \sigma_{1}$.

  But note that $\bigoplus_{w\in \Wlam/\approxLR} \VLR_{w} \cong \bC[\Wlam]$
  and $\sigma\otimes \sigma$ has multiplicity one in $\bC[\Wlam]$
  which implies the claim.
}

A left (resp. right cell) in $\Irr(\Wlam)$ is the multiset of the irreducible
constituents in $\VL_{w}$  (resp. left cell) for some $w\in \Wlam$.

The equivalence of Barbasch-Vogan's definition  and
  Lusztig's definition of cells in $\Irr(\Wlam)$ is a consequence of
  Kazadan-Lusztig conjecture, see \cite{BV2}*{remarks after Corollary~2.16}.

The structure of double and left cells are explicitly described in
\cite{Lu}*{Section~4}.
In particular, $\sigma_{w}$ is the unique special representation occurs
in the double cell
\[
  \LRC_{w}:= \Set{\sigma| \sigma\otimes \sigma \text{ occurs in } \VLR_{w}}
  \subseteq \Irr(\Wlam).
\]
For this reason, we also write
\[
  \LRC_{\sigma}:=\LRC_{w}
\]
where $\sigma=\sigma_{w}$ is the unique special representation in $\LRC_{w}$.
The generic degree ``a''-function is constant on the double cells and order
preserving: for each $\sigma'\in
\LRC_{\sigma}$, the generic degree $a(\sigma')=a(\sigma)$;
$a(\sigma')<a(\sigma'')$ if $\sigma'\lneqLR \sigma''$.

In summary, we have bijections
\[
  \Wlam/\approxLR \longleftrightarrow\Irrsp(\Wlam)\longleftrightarrow \Irr(\Wlam)/\approxLR.
\]
We write $\VLR_{\sigma}$ to be the unique double cell representation
containing $\sigma$ and $\bVLR_{\sigma}$ to be the unique upper cone
representation which is isomorphic to
$\bigoplus_{\sigma\leqLR \sigma'}\sigma'\otimes \sigma'$.


\subsection{Compare blocks}
Now we compare different blocks. Without of loss of generality, we assume
$\lambda$ is in the anti-dominant cone of $\WT{\fnn}$, i.e
$\inn{\lambda}{\ckalpha}<0$ for all $\alpha\in \WT{\fnn}$.

Let $k=\abs{W/\Wlam}$ and
\[
%\Set{r_1,\cdots, r_{k}}
% \Set{r|\text{the length of $r$ with respect to simple roots in $-\WT{\fnn}$ is minimal in $r\Wlam$} }
\Set{r_1,\cdots, r_{k}} :=
\Set{r|l(r) \text{ is minimal among elements in }r\Wlam }
\]
% \[
% \set{r_{i}\Wlam|i =1, 2, \cdots, k}
% \] be the list of right cosets of $W/\Wlam$.
be the set of distinguished representatives of the right cosets of $\Wlam$ where
$l(r)$ denote the length function with respect to the simple roots in $-\WT{\fnn}$. In
other words, $r_{i}$ is the unique element in the coset $r_{i}\Wlam$
such that $r_{i}\lambda$ is anti-dominant, i.e.   $R^{+}_{[r_{i}\lambda]}\subseteq -\WT{\fnn}$.
% We choose $r_{i}$ to be the unique element in the coset $r_{i}\Wlam$
% such that $r_{i}\lambda$ is anti-dominant, i.e.   $R^{+}_{[r_{i}\lambda]}\subseteq -\WT{\fnn}$.

%Let $S_{[\lambda]}$ be the set of simple roots in $R^{+}_{[\lambda]}$.
%Apply Soergel's theorem \cite{H}*{13.13}, we have
% Then
% \begin{itemize}
%   \item The map $L(w\lambda) \mapsto
%   L(r_{i} w\lambda)$ with $w$ running over $w\in W_{[\lambda]}$ induces an equivalence of
%   category from $\cO_{W\cdot \lambda, [\lambda]}(\fgg,\fhh,\fnn)\cap$ to
%   $\cO_{W\cdot\lambda, [r_{i}\lambda]}$ by Soegel's theorem \cite{H}*{13.13}.
%   \item $w\mapsto r_{i} w r_{i}^{-1}$ induces isomorphism
%   $W_{[\lambda]}\rightarrow W_{[r_{i}\lambda]}$ and preserves the cell
%   structures.
%   \item The following $\Wlam$-module isomorphism
%   \[
%     \begin{tikzcd}
%       & \bC[W_{}]& \\
%     \end{tikzcd}
%   \]
% \end{itemize}

Now  the map $L(w\lambda) \mapsto
  L(r_{i} w\lambda)$ with $w$ running over $w\in W_{[\lambda]}$ induces an equivalence of
  category from $\cO_{W\cdot \lambda, [\lambda]}(\fgg,\fhh,\fnn)\cap$ to
  $\cO_{W\cdot\lambda, [r_{i}\lambda]}$ by Soegel's theorem \cite{H}*{13.13}.
In particular $w\mapsto r_{i} w r_{i}^{-1}$ induces isomorphism
  $W_{[\lambda]}\rightarrow W_{[r_{i}\lambda]}$ and preserves the cell
  structures.
  In other words, the following $\Wlam$-module isomorphism
  \[
    \begin{tikzcd}
      & \bC[W_{}]\ar[dl, "w\mapsto M_{w}"'] \ar[dr,"w\mapsto M_{r_{i}w}"]& \\
      \Coh_{[\lambda]}(\fgg,\fhh,\fnn;[\lambda])\ar[rr]
      & &
      \Coh_{[\lambda]}(\fgg,\fhh,\fnn;[r_{i}\lambda])
    \end{tikzcd}
  \]
  maps cell representations to cell representations.

\trivial{
  Let $C = \set{x\in \fhh| \inn{x}{\alpha} >0  \ \forall \alpha\in \WT{\fnn})}$
    and
$D_{\lambda} = \set{x\in \fhh| -\inn{x}{\beta} >0 \ \forall \beta\in R^+_{[\lambda]}}$.
$C$ and $D_{\lambda}$ are fundamental domains of $\fhh$ under $W$ and
$W_{[\lambda]}$-actions.
Clearly, $D_{w\lambda} = w D_{\lambda}$.
The condition that $R^{+}_{[\mu]}\subset - \WT{\fnn}$ is equivalent to $D_{\mu}\supset C$.

Now it is clear $D_{\lambda}$ is the union of $r_{i}^{-1} C$
when $r_{i}$ running over the preferred  coset representatives
of $W/W_{[\lambda]}$.
}

Recall the definition of clan of the primitive ideals.
By the comparing the definition of Goldie rank polynomials, we see that
$I(r_{i}w)$ and $I(r_{j}w')$ in the same clan if and only if
$w\approxLR w'$.
In other word, the clan is only depends on the $W_{[\lambda]}$-type of the
double cell containing $L(w\lambda)$ where $w\in W$.

\medskip

\def\Dsp{\cD^{\text{sp}}}
\def\Csp{\sfC^{\text{sp}}}
The above discussion yields the following.
\begin{lem}\label{lem:C.S}
  Fix a $\Gcad$-invariant subset $\sfS$ in the nilpotent cone of $\fgg$.

  Let
  \begin{equation}\label{eq:C.S}
    \begin{split}
  \Csp_{\sfS} &:= \Set{\sigma \in \Irrsp(\Wlam)| \Spr(j_{\Wlam}^{W}\sigma)\subseteq \sfS}, \\
  \sfC_{\sfS} &:=
  \Set{\sigma'\in \Irr(\Wlam)| \exists \sigma \in \Csp_{\sfS}
    \text{ such that} \sigma \leqLR \sigma'}
  % \bigcup_{\sigma\in \Dsp_{\sfS}}
  \end{split}
\end{equation}
% Let $\VLR_{\sigma}$ be the unique
  Then, as an $W_{[\lambda]}$-module
  \[
    \begin{split}
      \Coh_{[\lambda],\sfS}(\fgg,\fhh,\fnn) &= \bigoplus_{i=1}^{k}
      \Coh_{[\lambda], \sfS}(\fgg,\fhh,\fnn;[r_{i}\lambda-\rho])\\
      & \cong \bigoplus_{i=1}^{k} \sum_{\sigma\in \Csp_{\sfS}} \VLR_{\sigma}\\
      & \cong \bigoplus_{i=1}^{k} \bigoplus_{\sigma\in \sfC_{\sfS}}
      (\dim \sigma) \sigma
    \end{split}
\]
\end{lem}

As a baby case of the counting theorem for special unipotent representation
of real reductive groups, we have the following counting theorem in the category
$\cO$.

We fix an regular element $\lambda\in [\mu]$ such that $\mu$ is dominant with
respect to $R^{+}_{[\lambda]}$. Let $S_{[\lambda]}$ be the set of simple roots
in $R^{+}_{[\lambda]}$. Let $\sfS_{\mu}$ be the subset of simple roots in
$R^{+}_{[\lambda]}$ orthogonal to $\mu$. Observe that $W_{\mu}$ is always a
parabolic subgroup attached to $\sfS_{\mu}$ in $W_{[\lambda]} = W_{[\mu]}$. Let
$\sfD_{\sfS_{\mu}}$ be the set of distinguished right coset representatives of
$W_{[\lambda]}/W_{\mu}$.
\trivial[]{
  $r\in \sfD_{\sfS_{\mu}}$ is the element with minimal lenght in $rW_{\mu}$.
  Recall that $\tau(w) = \set{\alpha\in S_{[\lambda]}|w\alpha\notin R^{+}_{[\lambda]} }$
  Note that
  $\tau(w)\cap R_{\mu}\neq \emptyset$ is equivalent to require that
  $w \sfS_{\mu}\subseteq R^{+}_{[\lambda]}$, i.e. $w$ is a minimal length
  element. See for example, Carter, Simple groups of Lie type, Theorem~2.5.8.
}

\begin{thm}
  Let $\cO$ be an nilpotent orbit in $\fgg$ and $\mu\in \fhh$.
  Let $\Pi_{W\cdot \mu, \cO}$ be the set of irreducible highest
  weight modules $\pi$ such that  $\AVC(\pi) = \bcO$.
  Let
  \begin{equation}\label{eq:DC.O.mu}
    \begin{split}
      \Dsp_{\cO,\mu} &:= \Set{\sigma\in \Irrsp(W_{[\mu]})|\Spr(j_{W_{[\mu]}}^{W}\sigma) = \cO}\quad \text{
        and }\\
      \cD_{\cO,\mu} &= \bigcup_{\sigma\in \Dsp_{\cO,\mu}} \LRC_{\sigma}.
    \end{split}
  \end{equation}
  % $ be the set of special representations such that
  Then
  \[
    \abs{\Pi_{W\cdot\mu,\cO}} = \abs{W/W_{[\mu]}}\cdot \sum_{\sigma\in \cD_{\cO,\mu}}
    \left(\dim \sigma \cdot [1_{W_{\mu}}:\sigma]\right).
  \]


  Let
  \[
  \CLR_{\sigma,\mu} = \CLR_{\sigma,\mu}\cap \sfD_{\sfS_{\mu}}
  \]
  % \[
  % \CLR_{\sigma,\mu} = \Set{w\in \CLR_{\sigma}|\tau(w)\cap R_{\mu}= \emptyset}
  % \]
  and $\CLR_{\cO,\mu} = \bigcup_{\sigma\in \Dsp_{\cO}} \CLR_{\sigma,\mu}$.
  Then
  \[
  \Pi_{W\cdot\mu,\cO} = \Set{L(r_{i}w)|w\in \CLR_{\cO,\mu} \text{ and } i=1,2,\cdots,k}
  .
  \]
  Here $\VLR_{\sigma}$ is understood as a submodule of $\bC[\Wlam]$.
\end{thm}


Fix $\mu\in \fhh$ and let $\cI_{\mu}$ be the maximal primitive ideal with
infinitesimal character $\mu$.
Let $\Pi_{W\cdot \mu}$ be the set of irreducible highest
weight modules $\pi$ such that  $\Ann(\pi) = \cI_{\mu}$.

Let $a(\sigma)$ be the generic degree of a Weyl group representation $\sigma$.

In view of \Cref{thm:count}, we need the following lemma by Barbasch-Vogan. %in \cite{BVUni}.
\begin{lem}[{\cite{BVUni}*{(5.26), Proposition~5.28}}]
  \label{lem:LC.mu}
  Let
  \[
    a_{\mu} = \max\set{a(\sigma)| \sigma \in \Irr(W_{[\mu]}) \text{ and }
    [1_{W_{\mu}}: \sigma]\neq 0}.
  \]
  Let
  \[
    \LC_{\mu} =
    \set{\sigma \in \Irr(W_{[\mu]}) | a(\sigma) = a_{\mu}
      \text{ and } [1_{W_{\mu}}: \sigma]\neq   0
    }.
  \]
  Then
 %  \begin{itemize}
 %   \item $\Wlamck$ is a Levi subgroup of $\WLamck$, and
 % \item
  $\LC_{\mu}$ is a left cell of $W_{[\mu]}$ given by
  \[
    (J_{W_{\mu}}^{W_{[\mu]}} \sgn )\otimes \sgn
  \]
  which contains a unique special representation
  \[
    \sigma_{\mu}=(j_{W_{\mu}}^{W_{[\mu]}} \sgn )\otimes \sgn.
  \]
  Moreover, $\LC_{\mu}$ is multiplicity free, which is equivalent to
  \[
  [1_{W_{\mu}}:\sigma]=1 \quad \text{for each } \sigma\in \LC_{\mu}.
  \]

  Let
  \begin{equation}\label{eq:O.mu}
    \cO_{\mu} = \Spr(j_{W_{[\mu]}}^{W}\sigma_{\mu}).
  \end{equation}
Then
  \[
    \LC_{\mu} = \Set{\sigma\in \sfC_{\bcO_{\mu}}| [1_{W_{\mu}}:\sigma]\neq 0}.
  \]
  \qed
  % \e
\end{lem}

\trivial{
  This is essentially contained in \cite{BVUni}.

  We adapt the notation in \cite{BVUni}: two special representations
  $\sigma \LRleq \sigma'$ if and only if $\cO_{\sigma}\supseteq \cO_{\sigma'}$
  where $\cO_{\sigma}:=\Spr(\sigma)$. The generic degree of $\sigma$ is denoted by $a(\sigma)$.
  Note that the ordering of double cells/special representation is the same as the closure relation on special nilpotent orbits, see \cite{BVUni}*{Prop
    3.23}.

  Note that induction maps left cone representation to a left cone
  representation \cite{BVUni}*{Prop~4.14~(a)}. Therefore
  $\Ind_{W_{\mu}}^{W_{[\mu]}}\sgn$ is a left cone representation.
  $J_{W_{\mu}}^{W_{[\mu]}}\sgn$ is a left cell (since $J$-induction preserves
  left cell \cite{BVUni}*{Prop~4.14~(b)}), it consists of the constituents in the induced
  representation with the minimal generic degree (by the definition of
  $J$-induction), it is also the set of constituents in
  $\Ind_{W_{\mu}}^{W_{[\mu]}}\sgn$ sit in the same $\approxLR$ equivalence class (a unique double cell $\cD$).


  Recall that tensoring with sign (or rather twisting $w_{0}$) is an order
  reversing bijection of left cells in $\WLamck$ and induces a $LR$-order
  reversing bijection on $\Irr(\Wlamck)$, see \cite{BV2}*{Prop.~2.25}. Therefore
  $\Ind_{\Wlamck}^{\WLamck} 1 = \left(\Ind_{\Wlamck}^{\WLamck} \sgn\right)\otimes \sgn$
  has a set of constituents which is maximal under the $LR$-order, in particular
  the generic degree takes maximal value on these representations.

  Hence we get the conclusion.
}

For each  $\mu\in \fhh^{*}$,
let $\cI_{\mu}$ be the maximal primitive ideal having infinitesimal character
$\mu$.
Let
  $\Pi_{W\cdot \mu}$ be the set of all irreducible highest weight modules whose
  annihilator ideal are $\cI_{\mu}$.
  Then
  \[
    \Pi_{W\cdot \mu} = \Pi_{W\cdot \mu, \cO_{\mu}}
  \]
  where $\cO_{\mu}$ is given by \eqref{eq:O.mu}.


Combine the above theorem with \Cref{lem:lcell.BV}, we have the following
counting theorem.
\begin{thm}
  %Suppose $\mu = \lambda_{\ckcO}$.
  % Let
  % \[
  % \LC_{\mu} = \left(J_{W_{\mu}}^{W_{[\mu]}} \sgn\right)\otimes \sgn
  % \]
  % be the left cell attached to $\mu$ and
  % \[
  %   \sigma_{\mu} = \left(j_{W_{\mu}}^{W_{[\mu]}} \sgn\right)\otimes \sgn.
  % \]
  % Then $\sigma_{\mu}$ is the unique special representation containing in
  % $\LC_{\mu}$
  % and
  Retain the notation in \Cref{lem:LC.mu}.   \[
    \abs{\Pi_{W\cdot \mu}} = \abs{W/W_{[\mu]}}\cdot
    \dim \LC_{\mu}.
  \]
  Moreover,
  \[
  \Pi_{W\cdot\mu} = \Set{L(r_{i}w)|w\in \CLR_{\sigma_{\mu}}\cap \sfD_{\sfS_{\mu}} \text{ and } i=1,2,\cdots,k}
  \]
  \qed
\end{thm}

% A double cell in $\Irr(W_{[\lambda]})$ is an equivalent class under
% the relation $\approxLR$


% \[
%   \begin{array}{ccc}
%     \bC[W_{[\lambda]}]& \longrightarrow  &\Coh_{[\lambda]}(\fgg,\fhh,\fnn,[\lambda]).\\
%     w & \mapsto M_{w}
%   \end{array}
% \]


% we define the category $\cO'_{S}$ to be the category of $\fgg$-modules
% such that $M\in \cO'_{S}$ if and only if
% \begin{itemize}
%   \item the $\fbb$-action on $M$ is locally finite;
%   \item $M$ is finitely generated $\cU(\fgg)$-module,
%   \item $M = \sum_{\mu \in S} M_{\mu}$
%         where $M_{\mu}$ is the $\mu$-isotypic component of $M$.
% \end{itemize}

% Therefore, the last condition implies that
% \[
%   \begin{split}
%     \sV^{R}(w) &:=
%     \sspan\set{L(w'\lambda) | w'\leqR w}\\
%     & =\sspan\set{L(w' \lambda) | w^{-1}\leqL w'^{-1}}
%   \end{split}
% \]



% Let $\sfN_{d}$ be the union of all nilpotent orbits in $\fgg$ of dimension
% less than equal to $d$.
%



\subsection{A variation}
In this section, let
$(\fgg, H,\fnn)$ be a triple that
\begin{itemize}
  \item $\fgg$ is a complex reductive Lie algebra,
  \item $H$ is a Lie group such that
        $\fhh_{0}:= \Lie(H)$ is a real form of
       a Cartan subalgebra  $\fhh$ of $\fgg$,
  \item $\fnn$ is a maximal nilpotent subalgebra
        stable under the $\fhh$-action.
\end{itemize}
Let $\cO'(\fgg, H,\fnn)$ be the category of $(\fgg, H)$-module such that
$M\in \cO'(\fgg,H,\fnn)$
if and only if
\begin{itemize}
  \item $M$ is finitely generated as $\cU(\fgg)$-module,
  \item $\fnn$ acts on $M$ locally nilpotently, and
  \item $M$ decomposes in to a direct sum of finite dimension $H$-modules.
\end{itemize}
Let $\Grt(\fgg,H,\fnn)$ be the Grothendieck group of $\cO(\fgg,H,\fnn)$.

We write $H_{0}$ for the connected component of $H$ which is abelian.
Since $H_{0} = \exp(\fhh_{0})$ is central in $H$, for each $\phi\in \Irr(H)$
$\phi|_{H_{0}}$ is a multiple of character.
Hence taking the derivative yields a
well defined map
\[
\rdd \colon \Irr(H)\longrightarrow \fhh^{*}
\]
sending $\phi$ to $\dphi$.


Since $\rdd$ restricted on the lattice
\[
\tQ:=\Set{\phi\in \Irr(H)|\phi \text{
    occurs in } S(\fgg)}
\]
 is a bijection onto the root lattice $Q$ \cite{Vg}*{0.4.6}, we identify the root lattice $Q$ with the $\tQ$ in $\Irr(H)$.


Now assume $\phi\in \Irr(H)$ and let
\[
  [\phi] := \Set{\phi+\alpha| \alpha\in \tQ}
\]
and
$\cO'(\fgg,H,\fnn;[\phi])$ be the subcategory of $\cO'(\fgg,H,\fnn)$
consists of modules whose $H$ irreducible components are contained in $[\phi]$.
Define $\Coh_{[\lambda]}(\fgg,H,\fnn;[\phi])$ to be the space of coherent
families taking value in $\cO'(\fgg,H,\fnn;[\phi])$ and
$\Coh_{[\lambda]}(\fgg,H,\fnn;[\phi])$ to be its subspace whose complex
associated variety is contained in $\sfS$ for a $\Ad(\fgg)$-invariant closed subset $\sfS$ in the nilpotent cone of
$\fgg$.


We have the following lemma.
\begin{lem}
  Let $\phi\in \Irr(H)$ and fix a $\lambda\in \fhh^{*}$ such that
  $[\rdd \phi + \rho ]\cap W\cdot \lambda\neq \emptyset$.
  Then the forgetful functor
  \[
    \cF\colon \cO'(\fgg,H,\fnn)\longrightarrow \cO'(\fgg,\fhh,\fnn)
  \]
  induces a $W_{[\lambda]}$-module isomorphism
  \[
    \cF\colon \Coh_{[\lambda],\sfS}(\fgg,H,\fnn;[\phi])\longrightarrow
    \Coh_{[\lambda],\sfS}(\fgg,\fhh,\fnn;[\rdd\phi])
  \]
\end{lem}
\begin{proof}
  When $\sfS$ is the whole nilpotent cone, the isomorphism is
  given by identifying both sides with $\bC[W_{[\lambda]}]$ via Verma modules
  such that
  \[
  \wtM_{1}(\rdd \phi+\rho):=\cU(\fgg)\otimes_{\cU(\fhh\oplus\fnn),H}\phi
  \mapsto (\dim\phi)\cdot M_{1}(\rdd\phi+\rho):= \cU(\fgg)\otimes_{\cU(\fhh\oplus\fnn)}\rdd\phi.
  \]
  For $\phi'\in \Irr(H)$, $M(\phi'+\rho)$ has a unique irreducible quotient
  $L(\phi')$ by the same argument of the same argument for the highest weight module.   Now we have, $L(\phi'+\rho) = \dim(\phi')\cdot L(\rdd\phi'+\rho)$.
  \trivial{
    These claims should be also much more clear from the D-module point of view.
    The middle extension functor only see the $\cD_{\lambda}$-module structure
    and keeps the $T$-module structure automatically. Here $T$ is the maximal
    compact subgroup of $H$.
  }
  Since the associated variety only depends on the $\cU(\fgg)$-module structure,
  the rest part of the lemma follows.   {\color{red} Check!!}
\end{proof}

\trivial{
  Note that $H_{0}$ is abelian and $H = H_{0}$.
  By the assumption of Harish-Chandra class,
  $H = H_{0}\times H/H_{0}$. Here $H/H_{0}$ is a finite group maybe non-abelian.
}

The above lemma have the following immediate consequence.
\begin{cor}
  Retain the notation in \Cref{lem:C.S}. Then, for $\sigma\in \Irr(W_{[\lambda]})$
  \[
[\sigma:\Coh_{[\lambda],\sfS}(\fgg,H,\fnn)] \neq 0 \Leftrightarrow
  \sigma\in \sfC_{\sfS}.
  \]   \qed
\end{cor}



\section{Harish-Chandra cells}
In this section, let $G$ be a real reductive group in the Harish-Chandra class.
Here $G$ could be a nonlinear group.
We retain the notation in \Cref{eq:Coh.HS}.
We recall the argument before \cite{Mc}*{Theorem~1}.


% \begin{thm}[Barbasch-Vogan]\label{thm:count}
%   For a complex nilpotent orbit $\cO$,
%   define
%   \[
%     S_{\cO} = \left\{\sigma \in \widehat{W_{[\mu]}}|
%       \Spr(j_{W_{[\mu]}}^{W} \sigma_{s}) = \cO
%     \right\}
%   \]
%   where $\sigma_{s}$ is the special representation in the double cell containing
%   $\sigma$.

%   Let $\Pi_{\cO,\mu}(G)$ denote the set of irreducible admissible $G$-module with
%   complex associated variety $\overline{\cO}$.
%   % Let $W_{\mu}$ be the stabilizer of $\mu$ and $W_{[\mu]}$ be the stabilizer of
%   % the lattice
%   Then
%   \[
%     \# \Pi_{\cO,\mu}(G) =
%     \sum_{\sigma\in S_{\cO}} [\sigma: \mathrm{Coh}_{[\mu]}(G)] \cdot
%     [1_{W_{\mu}}, \sigma|_{W_{\mu}}].
%   \]
%   % Here $[\sgima : \ ]$ denote the multiplicity of $\sigma$
%   % and  $W_{\mu}$ is the stabilizer of $\mu$.
% \end{thm}

\subsection{An embedding of coherent families of Harish-Chandra modules into
  that of category $\cO$}
In this section, we recall a result in \cite{Cas}.
Let $\fbb = \fhh\oplus \fnn$ be a Borel subalgebra in $\fgg$ with the nilradical
$\fnn$ and $\fhh$ a Cartan subalgebra in $\fbb$.
For a subalgebra $\fuu$ of $\fnn$ and $q\in \bN$,
Casian defined the localization functors $\gamma_{\fuu}^{q}$ on the category of $\fuu$-module.
By \cite{Cas}*{Proposition~4.8}, $\gamma_{\fuu}^{q}$ can be defined as the right
derived functor of the functor $\gamma_{\fuu}^{0}$ which sends a $\fuu$-module $M$
to
\[
\gamma_{\fuu}^{0}(M):= \Set{v\in M| u^{k} = 0 \text{ for some positive integer
    $k$}}.
\]

% Let $M$ be a $(\fgg,K)$-module.


\def\Grt{\cG}
\def\DeltaRp{\Delta^{+}_{\bR}}
\begin{thm}[\cite{Cas}*{Proposition~2.10, Theorem~3.1}]\label{thm:L1}
  Let $H$ be a $\theta$-stable Cartan subgroup of $G$.
  Fix a positive system of real roots $\DeltaRp$ and a positive system
  $\Delta^{+}$ of roots such that $\DeltaRp\subseteq \Delta^{+}$.
  Let $\bfnn$ be the maximal nilptent Lie subalgebra of $\fgg$ with spanned
  by roots in $\Delta^{+}$.
 Let $M$  be a Harish-Chandra $(\fgg,K)$-module, then
  \begin{enumT}
    \item the Lie algebra cohomology $H^{q}(\fnn,M)$ is finite dimensional;
    \item
    the localization $\gamma_{\fnn}^{q}M$ is in the category $\cO'_{\whH}$;
    (see \cite{Cas}*{Section~1})
    \item
    $\Ann M \subseteq \Ann (\gamma_{\fnn}^{q}M)$.
    \item
  Then the localization functor induces a homomorphism
  \[
    \begin{array}{cccc}
      \gamma_{\fnn}: &\Grt_{\chi,\cZ}(\fgg,K) &\longrightarrow & \Grt_{\chi,\cZ}(\cO_{\whH})\\
      & M &\mapsto & \sum_{q}  (-1)^{q} \gamma^{q}_{\fnn} M
    \end{array}
  \]
  \end{enumT}
  % Fix an infinitesimal character $\chi$, and a close $G$-invariant set
  % $\cZ\in \cN_{\fgg}$.
  \qed
\end{thm}

\begin{thm}[\cite{Cas}*{Theorem~3.1}]
  Let $H_{1}, H_{2}, \cdots, H_{s}$ form a set of representatives of the
  conjugacy class of $\theta$-stable Cartan subgroup of $G$. Fix maximal
  nilpotent Lie subalgebra $\fnn_{i}$ for each $H_{i}$ as in \Cref{thm:L1}.
  Then
  \[
    \begin{array}{cccc}
      \gamma:=\oplus_{i} \gamma_{\fnn_{i}}: &\Grt_{\chi}(\fgg,K)
      &\longrightarrow & \bigoplus_{i} \Grt_{\chi}(\cO_{\whH_{i}})\\
    \end{array}
  \]
  is an embedding of $\WLam$-module.
\end{thm}
\begin{proof}
  We retain the notation in \cite{Cas}.
  Let $H^{rs}_{i}$ be the set of regular semisimple elements in $H_{i}$. By
  Harish-Chandra, taking the character of the elements induces an embedding of
  $\Grt_{\chi}(\fgg,K)$ into the space of analytic functions on
  $\bigsqcup_{i} H^{rs}_{i}$. Now \cite{Cas}*{Theorem~3.1} implies that the
  global character $\Theta M$ of an element $M\in \Grt_{\chi}(\fgg,K)$ is
  completely determined by the formal character $\mathrm{ch}(\gamma(M))$.
\end{proof}

\def\VHC{\sV^{\mathrm{HC}}}
\begin{cor}
  Fix an irreducible $(\fgg,K)$-module $\pi$ with infinitesimal character
  $\chi_{\lambda}$. Let $\VHC(\pi)$ be the Harish-Chandra cell representation
  containing $\pi$ and $\cD$ be the double cell in $\widehat{W_{[\lambda]}}$
  containing the special representation $\sigma(\pi)$attached to $\Ann(\pi)$.
  Then $[\sigma, \VHC(\pi)]\neq 0$ only if $\sigma \in \cD$.
  Moreover, $\sigma(\pi)$ always occures in $\VHC(\pi)$
\end{cor}
\begin{proof}
  The occurrence of $\sigma(\pi)$ is a result of King.

  Note that we have an embedding
  \[
    \gamma \colon \Grt_{\chi}(\fgg,K)\longrightarrow \bigoplus_{i}\Grt(\fgg,\fbb,\lambda).
  \]
  where the left hand sides is identified with a finite copies of
  $\bC[W_{[\lambda]}]$.

  Since $\Ann(\pi)\subseteq \Ann (\gamma_{\fnn}^{q}(\pi))$,
  we conclude that $[\sigma, \VHC(\pi)]\neq 0$ implies that
  $\sigma(\pi)\leqLR \sigma$.

  By the Vogan duality, $\cD\otimes \sgn$ is also a Harish-Chandra cell.
  So we have $\sigma(\pi)\otimes \sgn \leqLR \sigma\otimes \sgn$.

  Therefore, $\sigma(\pi)\approxLR$
\end{proof}




\subsection{Counting special unipotent representations of reductive groups}

Now let $\ckcO$ be a nilpotent orbit in $\ckcG$.
It determine an infinitesimal character $\lamck$.

Let $\WLamck$ be the integral Weyl group and $\Wlamck$ be the stabilizer of
$\lamck$.

Special unipotent representations are defined to be the set of
Harish-Chandra modules with minimal GK-dimension.

\section{Counting in type A}
The results in this section are well known to the experts.


Let $\YD$ be the set of Young diagrams viewed as a finite multiset of positive integers. 
The set of nilpotent orbits in $\GL_n(\bC)$ is identified with Young diagram of $n$ boxes.


Let $\ckGc = \GL_n(\bC)$. 
Fix
an orbit $\ckcO\in \Nil(ckG)$, let $\ckcO_e$ (resp. $\ckcO_o$) be the partition
consists of all even (resp. odd) rows in $\ckcO$.

Let $S_n$ denote the Weyl group of $\GL_n(\bC)$.
Let $W_n := S_n \ltimes \set{\pm 1}^n$ denote the Weyl group of type $B_n$ or $C_n$.  
Let $\sgn$ denote the sign representation of the Weyl group. 
The group $W_n$ is naturally embedded in $S_{2n}$.
For $W_n$, let $\epsilon$ denote the unique non-trivial character which is trivial on $S_n$. 
Note that $\epsilon$ is also the restriction of the $\sgn$ of $S_{2n}$ on $W_n$. 


\subsection{Special unipotent representations of $G = \GL_n(\bC)$}
By \cite{BVUni}, the set of unipotent representations
of $G = \GL_n(\bC)$ one-one corresponds to nilpotent orbits in $\Nil(\ckGc)$. 
Suppose $\ckcO$ has rows 
\[
\bfrr_1(\ckcO)\geq \bfrr_2(\ckcO)\geq \cdots\geq 
\bfrr_k(\ckcO) >0. 
\]
Then $\cO := \dBV(\ckcO)$ has columns $\bfcc_i(\cO) = \bfrr_i(\ckcO)$ for all 
$i\in \bN^+$. 
The map $\ckcO \mapsto \cO$ is a bijection.


We set $\CP =  \YD$ be the set of Young diagrams. 
For $\uptau \in \CP$ which has $k$ columns,
let $1_{c}$ be the trivial representations of $\GL_c(\bC)$. 
\[
 \pi_\uptau = 1_{\bfcc_1(\uptau)}\times  1_{\bfcc_2(\uptau)}\times \cdots 
 \times 1_{\bfcc_k(\uptau)}.
\]

The Vogan duality gives a duality between Harish-Chandra cells. 
In this case, Harish-Chandra cells is the double cell  
of Lusztig.  
Now we have a duality 
\[
 \pi_\uptau \leftrightarrow \pi_{\uptau^t}. 
\]

Let $\uptau' := \DD(\uptau)$ be the partition obtained by deleting the first column 
of $\uptau$. 
Let $\theta_{a,b}$ (resp. $\Theta_{a,b}$) be the  theta lift (resp. big theta lift) from $\GL_a(\bC)$ to  
$\GL_b(\bC)$. 
Then we have 
\[
  \pi_{\uptau} = \theta_{{\abs{\uptau'}},{\abs{\uptau}}} (\pi_{\uptau}). 
\]

\subsection{Counting unipotent representations of $\GL_n(\bR)$}
Now let $\ckcO\in \Nil(\ckGc)$. 
Recall the decomposition $\ckcO  = \ckcO_e \cup \ckcO_o$.
Let $n_e = \abs{\ckcO_e}, n_o = \abs{\ckcO_o}$ and $\lambda_\ckcO = \half \ckhh$. 

Then 
\[
  W_{\lamck} \cong S_{\abs{\ckcO_e}}\times S_{\abs{\ckcO_o}}.
\]
\[
 W_{\lambda_\ckcO} = \prod_j S_{\bfcc_j(\ckcO_e)}\times \prod_j S_{\bfcc_j(\ckcO_o)} 
\]
By the formula of $a$-function, one can easily see that 
The cell in $W(\lamck)$ consists of the unique representation $J_{W_{\lamck}}^{\Wint{\lamck}} (1)$.
Now the $W$-cell $(J_{W_{\lamck}}^W \sgn)\otimes \sgn$ consists a single
representation
\[
\tau_{\ckcO} = \ckcO_{e}^{t}\boxtimes \ckcO_{o}^{t}.
\]
The representation $j_{W_{\lamck}}^{S_{n}} \tau_{\ckcO}$
corresponds to the orbit $\cO= \ckcO^t $ under the Springer
correspondence.
\trivial{
WLOG, we assume $\ckcO =  \ckcO_o$.

Let $\sigma\in \widehat{S_n}$. We identify $\sigma$ with a Young diagram. 
Let $c_i = \bfcc_i(\sigma)$.
Then $\sigma = J^{S_n}_{W'} \epsilon_{W'}$ where $W' = \prod S_{c_i}$
(see Carter's book). 
This implies Lusztig's a-function takes value
\[
a(\sigma) = \sum_i c_i(c_i-1) /2
\]
Compairing the above with the dimension formula of nilpotent Orbits
\cite{CM}*{Collary~6.1.4}, we get (for the formula, see Bai ZQ-Xie Xun's paper on 
GK dimension of $SU(p,q)$)
\[
\half \dim(\sigma) = \dim(L(\lambda)) = n(n-1)/2 - a(\sigma).
\]
Here $\dim(\sigma)$ is the dimension of nilpotent orbit attached to the Young
diagram of $\sigma$ (it is the Springer correspondence, regular orbit maps to
trivial representation, note that $a(\triv)=0$), $L(\lambda)$ is any highest
weight module in the cell of $\sigma$. 


Return to our question, let $S' = \prod_i S_{\bfcc_i(\ckcO)}$. We want to find
the component $\sigma_0$ in $\Ind_{S'}^{S_n} 1$ whose $a(\sigma_0)$ is maximal,
i.e. the Young diagram of $\sigma_0$ is minimal. 

 By the branaching rule, $\sigma \subset \Ind_{S'}^{S_n} 1$ is given by adding
 rows of lenght $\bfcc_i(\ckcO)$ repeatly (Each time add at most one box in each
 column). 
 Now it is clear that $\sigma_0 = \ckcO^t$ is desired. 

 This agrees with the Barbasch-Vogan duality $\dBV$ given by 
 \[
  \ckcO \xrightarrow{Springer}\ckcO \xrightarrow{\otimes \sgn} \ckcO^t 
  \xrightarrow{Springer} \ckcO^t.
 \]
}

The $\Wint{\lamck}$-module $\Cint{\lamck}$ is given by the following formula:
\[
  \begin{split}
  \Cint{\lamck} &\cong \cC_{n_e}\otimes \cC_{n_o} \quad \text{with} \\ 
 \cC_n &:= \bigoplus_{\substack{s,a,b\\2s+a+b=n}} 
 \Ind_{W_s\times S_a\times S_b}^{S_{n}} \epsilon \otimes 1\otimes 1. % \text{ is a $S_n$-module.} 
  \end{split}
\] 

According to Vogan duality,  we can obtain the above formula by tensoring $\sgn$
on the forumla of the unitary groups in \cite{BV.W}*{Section~4}.

By branching rules of the symmetric groups,  $\Unip_{\ckcO}(G)$ can be parameterized by painted partition. 

\begin{equation}\label{eq:PP.AR}
\PP_{A^{\bR}}(\ckcO) = \Set{\uptau:=(\tau, \cP)|
  \begin{array}{l}
    \text{$\tau = \ckcO^{t}$}\\
    \text{$\Im(\cP)\subset \set{\bullet,c,d}$}\\
    \text{$\#\set{i|\cP(i,j)=\bullet}$ is even}
    % \text{``$\bullet$'' occures with even}\\
    % \text{mulitplicity in each column}
  \end{array}
}.
\end{equation}
For $\uptau:=(\tau,\cP)\in \PP_{A^{\bR}}(\ckcO)$, we write $\cP_{\uptau}:= \cP$.

\trivial{
The typical diagram of all columns with even length $2c$ are
\[
\ytb{\bullet\cdots\bullet\bullet\cdots\bullet,\vdots\vdots\vdots\vdots\vdots\vdots,
\bullet\cdots\bullet c\cdots c,
\bullet\cdots\bullet d\cdots d
}  
\]

The typical diagram of all columns with odd length $2c+1$ are
\[
\ytb{\bullet\cdots\bullet\bullet\cdots\bullet,\vdots\vdots\vdots\vdots\vdots\vdots,
\bullet\cdots\bullet \bullet\cdots\bullet ,
c\cdots c d\cdots d
}  
\]
}

Let $\sgn_n\colon \GL_n(\bR)\rightarrow \set{\pm 1}$ be the sign of determinant. 
Let $1_n$ be the trivial representation of $\GL_n(\bR)$. 
For $\uptau\in \PP{\ckcO}$, we attache the representation 
\begin{equation}\label{eq:u.GLR}
\pi_\uptau := 
\bigtimes_{j} \underbrace{1_j \times \cdots \times 1_j}_{c_j\text{-terms}}\times
\underbrace{\sgn_j \times \cdots \times {\sgn_j} }_{d_j\text{-terms}}.
\end{equation}
Here 
\begin{itemize}
  \item 
$j$ running over all column lengths in $\ckcO^t$, 
\item $d_j$ is the number of
columns of length $j$ ending with the symbol ``d'',
\item  $c_j$ is the number of
columns of length $j$ ending with the symbol ``$\bullet$'' or ``$c$'', and 
\item  ``$\times$'' denote the parabolic induction.  
\end{itemize}

\subsection{Special Unipotent representations of $G=\GL_{m}(\bH)$}

Suppose that $\cO$ is the complexificiation of a rational nilpotent $\GL_{m}(\bH)$-orbit. 
Then $\cO$ has only even length columns. 
Therefore, $\Unip_\ckcO(G) \neq\emptyset$ only if $\ckcO = \ckcO_e$. 

In this case the coherent continuation representation is given by  
\[
  \Cint{\lamck}(G) = \Ind_{W_m}^{S_2m}\epsilon 
\]
and $\Unip_\ckcO(G)$ is a singleton. %We use partition $\tau:= \ckcO^t$ to parameter special unipotent representations of $\GL_{m}(\bH)$. 
For each partition $\tau$ only having even columns, we define 
\[
  \pi_{\tau} := \bigtimes_i 1_{\bfcc_i(\tau)/2}. 
\] 

\subsection{Counting special unipotent repesentations of $\rU(p,q)$}
We call the parity of $\abs{\ckcO}$ the ``good pairity''.  The other pairity is called the ``bad parity''. 
We write $\ckcO = \ckcO_g\cup \ckcO_b$ where $\ckcO_g$ and $\ckcO_b$ consist of good parity length rows
and bad parity rows respectively.

Let $(n_g,n_b) = (\abs{\ckcO_g},\abs{\ckcO_b})$.
Now as the $S_{n_g}\times S_{n_b}$ 
\[
\bigoplus_{\substack{p,q\in \bN\\p+q=n}} \Cint{\lamck}(\rU(p,q)) = \cC_{g}\otimes \cC_{b}
\]
where 
\[
  \begin{split}
 \cC_{g} &= \bigoplus_{\substack{s,a,b\in \bN\\2s+a+b=n_g}} \Ind_{W_{s}\times S_a\times S_b}^{S_{n_g}}
 1\otimes \sgn\otimes \sgn \\
 \cC_{b} &= \begin{cases}
  \Ind_{W_{\frac{n_b}{2}}}^{S_{n_b}} 1 & \text{if $n_b$ is even}\\
  0 & \text{otherwise}. 
 \end{cases}
  \end{split}
\]

By the above formula, we have
\begin{lem}
  \begin{enumT}
    \item
The set $\Unip_{\ckcO_b}(\rU(p,q))\neq \emptyset$ if and only if $p=q$ and 
each row lenght in $\ckcO$ has even multiplicity.
\item
Suppose $\Unip_{\ckcO_b}(\rU(p,p))\neq \emptyset$, let $\ckcO'$ be the Young diagram 
such that $\bfrr_i(\ckcO') = \bfrr_{2i}(\ckcO_b)$ and $\pi'$ be the unique special 
uinpotent representation in $\Unip_{\ckcO'}(\GL_{p}(\bC))$. 
Then the unique element in $\Unip_{\ckcO_b}(\rU(p,p))$  is given by 
\[
  \pi := \Ind_{P}^{\rU(p,p)} \pi'
\]
where $P$ is a parabolic subgroup in $\rU(p,p)$ with Levi factor equals
to $\GL_p(\bC)$.
\item 
In general, when $\Unip_{\ckcO_b}(\rU(p,p))\neq \emptyset$, we have a natural bijection 
\[  
  \begin{array}{rcl}
  \Unip_{\ckcO_g}(\rU(n_1,n_2)) &\longrightarrow& \Unip_{\ckcO}(\rU(n_1+p,n_2+p))\\
  \pi_0 & \mapsto & \Ind_P^{\rU(n_1+p,n_2+p)} \pi'\otimes \pi_0
  \end{array}
\]
where $P$ is a parabolic subgroup with Levi factor $\GL_p(\bC)\times \rU(n_1,n_2)$. 
  \end{enumT}
\end{lem}

The above lemma ensure us to reduce the problem to the case when $\ckcO = \ckcO_g$. 
Now assume $\ckcO = \ckcO_g$ and so $\Cint{\ckcO}$ corresponds to the blocks of 
the infinitesimal character of the trivial representation.   

By \cite{BV.W}*{Theorem~4.2},  Harish-Chandra cells in $\Cint{\ckcO}$ are in one-one
correspondence to real nilpotent orbits in $\cO:=\dBV(\ckcO)=\ckcO^t$. 

\trivial{
From the branching rule, the cell is parametered by painted partition 
\[
\PP{}(\rU):=\set{\uptau\in \PP{}| \begin{array}{l}\Im (\uptau) \subseteq  \set{\bullet, s,r}\\
  \text{``$\bullet$'' occures even times in each row}
\end{array} 
  }.  
\]

The bijection $\PP{}(\rU)\rightarrow \SYD, \uptau\mapsto \sO$ is given by the following recipe:
The shape of $\sO$ is the same as that of $\uptau$. 
$\sO$ is the unique (upto row switching) signed Young diagram such that
\[
  \sO(i,\bfrr_i(\uptau)) := \begin{cases}
    +,  & \text{when }\uptau(i,\bfrr_i(\uptau))=r;\\
    -,  & \text{otherwise, i.e. }\uptau(i,\bfrr_i(\uptau))\in \set{\bullet,s}.
  \end{cases}
\] 

\begin{eg}
  \[
 \ytb{\bullet\bullet\bullet\bullet r,\bullet\bullet , sr,s,r}   
 \quad
 \mapsto\quad
 \ytb{+-+-+,+-, -+,-,+}   
  \]
\end{eg}
}

Now the following lemma is clear. 
\begin{lem}
When $\ckcO=\ckcO_g$, the associated varity of every special unipotent representations in $\Unip_\ckcO(\rU)$  
is irreducible. Moreover, the following map  is a bijection. 
\[  
  \begin{array}{rcl}
  \Unip_{\ckcO_g}(\rU(n_1,n_2)) &\longrightarrow& \set{\text{rational forms of $\ckcO^t$}}\\
  \pi_0 & \mapsto & \wAV(\pi_0).
  \end{array}
\]
\qed
\end{lem}
\begin{remark}
  Note that the parabolic induction of an rational nilpotent orbit can be reducible. 
  Therefore, when $\ckcO_b\neq \emptyset$, the special unipotent representations can have
  reducible associated variety. Meanwhile, it is easy to see that the map
  $\Unip_{\ckcO}(\rU) \ni \pi \mapsto\wAV(\pi)$ is still injective. 
\end{remark}

We will show that every elements in $\Unip_{\ckcO_g}$ can be constructed by iterated theta lifting.  
For each $\uptau$, let $\sO$ be the corresponding real nilpotent orbit. Let
$\Sign(\sO)$ be the signature of $\sO$, $\DD(\sO)$ be the signed Young diagram
obtained by deleteing the first column of $\sO$. 
Suppose $\sO$ has $k$-columns. Inductively we have a sequence of unitary groups
$\rU(p_i,q_i)$ with $(p_i,q_i) = \Sign(\DD^i(\sO))$ for $i=0, \cdots, k$. Then 
\begin{equation}\label{eq:u.U}
  \pi_\tau = \theta^{\rU(p_0,q_0)}_{\rU(p_1,q_1)} \theta^{\rU(p_1,q_1)}_{\rU(p_2,q_2)}\cdots   
\theta^{\rU(p_{k-1},q_{k-1})}_{\rU(p_k,q_k)}(1)
\end{equation}
where $1$ is the trivial representation of $\rU(p_k,q_k)$. 


Suppose $\ckcO = \ckcO_g$. Form the duality between cells of $\rU(p,q)$ and
$\GL(n,\bR)$. We have an ad-hoc (bijective) duality between unipotent
representations: 
\[
  \begin{array}{rcl}
 \dBV\colon \Unip_{\ckcO}(\rU)& \rightarrow &\Unip_{\ckcO^t}(\GL(\bR)) \\
 \pi_\uptau &\mapsto& \pi_{\dBV(\uptau)} \\ 
  \end{array}
\]

Here $\ckcO^t = \dBV(\ckcO)$ and $\dBV(\uptau)$ is the pained bipartition
obtained by transposeing $\uptau$ and replace $s$ and $r$ by $c$ and $d$
respectively. See \eqref{eq:u.U} and \eqref{eq:u.GLR} for the definition of
special unipotent representations on the two sides.  

\section{Counting in type BCD}
\def\tsgn{\widetilde{\sgn}}
\def\PBP{\mathsf{PBP}}

\def\ckstar{{\check \star}}

In this section, we consider the case when $\ckstar \in \set{B,C,D}$, i.e
$\star \in \set{B,\wtC, C,D,C^{*}, D^{*}}$.

We identify $\fhh^{*}$ with $\bZ^{n}$ where $n = \rank(\Gc)$
and let $\rho$ be the half sum of all positive roots.

Recall that
\[
  \text{good pairity} =
\begin{cases}
 %\text{odd} & \text{when } \ckstar\in \set{B,D}\\
 %\text{even} & \text{when } \ckstar = C\\
 \text{odd} & \text{when } \star \in \set{C,C^{*},D,D^{*}}\\
 \text{even} & \text{when } \star \in \set{B,\wtC}\\
\end{cases}
\]

\def\Wb{W_{b}}
\def\Wg{W_{g}}

  Suppose $\ckcO\in \Nil(\ckcG)$ with decomposition
  $\ckcO = \ckcO_{b}\cuprow \ckcO_{g}$.
  Then $\ckcG_{\lamck} = \ckcG_{b}\times \ckcG_{g}$.
  Let $n_{b}$ and $n_{g}$ be the rank of $\ckcG_{b}$ and $\ckcG_{g}$
  respectively. We have
  \[
    (n_{b}, n_{g}) =
    \begin{cases}
      (\half \abs{\ckcO_{b}}, \half(\abs{\ckcO_{g}}-1)) & \text{when
      } \star \in \set{C,C^{*}}\\
      (\half \abs{\ckcO_{b}}, \half\abs{\ckcO_{g}}) & \text{when
      } \star \in \set{B,\wtC,D,D^{*}}\\
    \end{cases}
  \]
  and integral Weyl group is a product of two factors
  \[
    W_{[\lamck]} =\Wb\times \Wg
  \]
  where
  \[
    \begin{split}
    \Wb & := \begin{cases}
      \sfW_{n_{b}}  & \text{when } \star \in \set{B, \wtC} \\
      \sfW'_{n_{b}} & \text{when } \star \in \set{C,C^{*},D,D^{*}}
      \end{cases}\\
    \Wg & := \begin{cases}
      \sfW_{n_{g}}  & \text{when } \star \in \set{B,C, C^{*} } \\
      \sfW'_{n_{g}} & \text{when } \star \in \set{\wtC,D,D^{*}}
      \end{cases}
    \end{split}
  \]


  \subsection{The left cell}

\begin{lem}\label{lem:Lcell}
  % Suppose $\star \in \set{C, C^{*}}$, $\ckcO_{b}$ has $2k$ rows and $\ckcO_{g}$
  % has $2l+1$-rows. Here each row in $\ckcO_{b}$ has even length and each row in
  % $\ckcO_{g}$ has odd length, and $\WLamck = W'_{b}\times W_{g}$ where
  % $b= \frac{\abs{\ckcO_{b}}}{2}$ and $g = \frac{\abs{\ckcO_{g}}-1}{2}$. Let
  %
  In all the cases there is a irreducible $\Wb$-representation $\tau_{b}$
  such that
  \[
    \begin{array}{ccc}
      \LC(\ckcO_{g}) & \longrightarrow & \LC(\ckcO)\\
      \tau_g & \mapsto & \tau_{b}\otimes \tau_{g}
    \end{array}
  \]
  is a bijection.
  Here
  \[
    \tau_{b} = \begin{cases}
      \Big( \big(\frac{\bfrr_{2}(\ckcO_{b})+1}{2}, \frac{\bfrr_{4}(\ckcO_{b})+1}{2}, \cdots, \frac{\bfrr_{2k}(\ckcO_{b})+1}{2}\big),\\
       \hspace{1em} \big(\frac{\bfrr_{2}(\ckcO_{b})-1}{2}, \frac{\bfrr_{4}(\ckcO_{b})-1}{2}, \cdots, \frac{\bfrr_{2k}(\ckcO_{b})-1}{2}\big)\Big)
      & \text{if } \star \in \set{B,\wtC},\\
      \Big( \big(\half\bfrr_{2}(\ckcO_{b}), \half\bfrr_{4}(\ckcO_{b}),\cdots, \half\bfrr_{2k}(\ckcO_{b})\big), \\
      \hspace{1em} \big(\half\bfrr_{2}(\ckcO_{b}), \half\bfrr_{4}(\ckcO_{b}),\cdots, \half\bfrr_{2k}(\ckcO_{b}) \big)\Big)_{I}
      & \text{if } \star \in \set{C,C^{*}, D,D^{*}}.\\
    \end{cases}
  \]


  Set
  \[
    \CPPs(\ckcO_{g}) =
    \begin{cases}
    \set{(2i-1,2i)| \bfrr_{2i-1}(\ckcO_{g})> \bfrr_{2i}(\ckcO_{g})>0, \text{
        and } i\in \bN^{+}}
    \end{cases}
  \]
  and $\CQ(\ckcO)= \bF_{2}[\CPP(\ckcO_{g})]$.


\end{lem}

\subsection{The left cell}



\subsection{Coherent continuation representations}

Let $Q$ be the root lattice in $\fhh^{*}$
which is
\[
Q = \begin{cases}
  \bZ^{n} & \text{if  $\star = B$}\\
  %\set{(a_{1},a_{2},\cdots, a_{n})\in \bZ^{n}|\sum_{i=1}^{n}a_{i} \in 2\bZ}
  \Set{(a_{i})\in \bZ^{n}|\sum_{i=1}^{n}a_{i} \text{ is even}}
    & \text{if  $\star \in \set{C,\wtC,C^{*},D,D^{*}}$}\\
\end{cases}
\]
We consider the lattice
\[
  \Lambda_{n_{b},n_{g}} =
  (\underbrace{\half, \cdots, \half}_{n_{b}\text{-terms}}, \underbrace{0, \cdots, 0}_{n_{g}\text{-terms}}) + Q.
  =
\]

\section{Type C}

  The dual group of $\Gc = \Sp(2n,\bC)$ is $\ckGc = \SO(2n+1,\bC)$. The even is
  the bad parity, and odd is the good parity.


  Fix $n_{b},n_{g}$ such that $n_{b}+ n_{g} = n$. Let
  \begin{equation} \label{eq:Lam.C}
    \Lambda_{n_{b},n_{g}} = (\underbrace{\half, \cdots, \half}_{n_{b}\text{-terms}}, \underbrace{0, \cdots, 0}_{n_{g}\text{-terms}}) + Q.
  \end{equation}

  Suppose $\ckcO\in \Nil(\SO(2n+1,\bC))$ with decomposition
  $\ckcO = \ckcO_{b}\cuprow \ckcO_{g}$.

  Let $\ckcO'_b$ be the Young diagram such that
  $\bfrr_i(\ckcO'_b) = \bfrr_{2i}(\ckcO_b)$ and $\cO'_{b}$ be the transpose of
  $\ckcO'_{b}$.


  \[
    \tau_{b} = \Big( \left(\half\bfrr_{2}(\ckcO_{b}), \half\bfrr_{4}(\ckcO_{b}),\cdots, \half\bfrr_{2k}(\ckcO_{b})\right), \left(\half\bfrr_{2}(\ckcO_{b}), \half\bfrr_{4}(\ckcO_{b}),\cdots, \half\bfrr_{2k}(\ckcO_{b}) \right)\Big)_{I} \in \Irr(W'_{b}).
  \]
  Set
  \[
    \CPP(\ckcO_{g}) = \set{(2i-1,2i)| \bfrr_{2i-1}(\ckcO_{g})> \bfrr_{2i}(\ckcO_{g})>0, \text{
        and } i\in \bN^{+}}
  \]
  and $\CQ(\ckcO)= \bF_{2}[\CPP(\ckcO_{g})]$.

  For $\sP \in \CQ(\ckcO)$, let
  \[
    \tau_{\sP} := (\imath,\jmath) \in \Irr(W_{g})
  \]
  such that
  \[
    (\bfcc_{l+1}(\imath), \bfcc_{l+1}(\jmath)) := (0,\half(\bfrr_{2l+1}(\ckcO_{g})-1))
  \]
  and for all $1\leq i\leq l$
  \[
    (\bfcc_{i}(\imath), \bfcc_{i}(\jmath)):=
    \begin{cases}
      (\half (\bfrr_{2i}(\ckcO_{g})+1), \half (\bfrr_{2i-1}(\ckcO_{g})-1))
      & \text{if } (2i-1,2i)\notin \sP,\\
      (\half (\bfrr_{2i-1}(\ckcO_{g})+1),\half (\bfrr_{2i}(\ckcO_{g})-1)) & \text{otherwise.}
    \end{cases}
  \]

  Then we have the following bijection
  \[
    \begin{array}{ccc}
      \CQ(\ckcO) &\longrightarrow & \LC(\ckcO)\\
      \sP & \mapsto & \tau_{b}\otimes \tau_{\sP}.
    \end{array}
  \]
  such that $\tau_{b}\otimes \tau_{\emptyset}$ is the special representation in
  $\LC(\ckcO)$.

  Moreover
  $\Spr(j_{W'_{b}\times W_{g}}^{W_{n}} \tau_{b}\otimes \tau_{\emptyset}) = \cO_{b}\cupcol \cO_{g}$
  where $\cO_{g} = \dBV(\ckcO_{g})$ and $\cO_{b} = \ckcO_{b}^{t}$.


\begin{lem}\label{lem:cell.C}
  Suppose $\star \in \set{C, C^{*}}$, $\ckcO_{b}$ has $2k$ rows and $\ckcO_{g}$
  has $2l+1$-rows. Here each row in $\ckcO_{b}$ has even length and each row in
  $\ckcO_{g}$ has odd length, and $\WLamck = W'_{b}\times W_{g}$ where
  $b= \frac{\abs{\ckcO_{b}}}{2}$ and $g = \frac{\abs{\ckcO_{g}}-1}{2}$. Let
  \[
    \tau_{b} = \Big( \left(\half\bfrr_{2}(\ckcO_{b}), \half\bfrr_{4}(\ckcO_{b}),\cdots, \half\bfrr_{2k}(\ckcO_{b})\right), \left(\half\bfrr_{2}(\ckcO_{b}), \half\bfrr_{4}(\ckcO_{b}),\cdots, \half\bfrr_{2k}(\ckcO_{b}) \right)\Big)_{I} \in \Irr(W'_{b}).
  \]
  Set
  \[
    \CPP(\ckcO_{g}) = \set{(2i-1,2i)| \bfrr_{2i-1}(\ckcO_{g})> \bfrr_{2i}(\ckcO_{g})>0, \text{
        and } i\in \bN^{+}}
  \]
  and $\CQ(\ckcO)= \bF_{2}[\CPP(\ckcO_{g})]$.

  For $\sP \in \CQ(\ckcO)$, let
  \[
    \tau_{\sP} := (\imath,\jmath) \in \Irr(W_{g})
  \]
  such that
  \[
    (\bfcc_{l+1}(\imath), \bfcc_{l+1}(\jmath)) := (0,\half(\bfrr_{2l+1}(\ckcO_{g})-1))
  \]
  and for all $1\leq i\leq l$
  \[
    (\bfcc_{i}(\imath), \bfcc_{i}(\jmath)):=
    \begin{cases}
      (\half (\bfrr_{2i}(\ckcO_{g})+1), \half (\bfrr_{2i-1}(\ckcO_{g})-1))
      & \text{if } (2i-1,2i)\notin \sP,\\
      (\half (\bfrr_{2i-1}(\ckcO_{g})+1),\half (\bfrr_{2i}(\ckcO_{g})-1)) & \text{otherwise.}
    \end{cases}
  \]

  Then we have the following bijection
  \[
    \begin{array}{ccc}
      \CQ(\ckcO) &\longrightarrow & \LC(\ckcO)\\
      \sP & \mapsto & \tau_{b}\otimes \tau_{\sP}.
    \end{array}
  \]
  such that $\tau_{b}\otimes \tau_{\emptyset}$ is the special representation in
  $\LC(\ckcO)$.

  Moreover
  $\Spr(j_{W'_{b}\times W_{g}}^{W_{n}} \tau_{b}\otimes \tau_{\emptyset}) = \cO_{b}\cupcol \cO_{g}$
  where $\cO_{g} = \dBV(\ckcO_{g})$ and $\cO_{b} = \ckcO_{b}^{t}$.
\end{lem}

% In that follows, we write
% \[
% \tau_{\ckcO_{g}} := \tau_{\emptyset}
% \]
% in \Cref{lem:cell.C}.

\trivial{ In this case, bad parity is even and each row length occur with even
  multiplicity. Suppose
  $\ckcO_{b} = (C_{1}, C_{1}, C_{2},C_{2}, \cdots, C_{k'},C_{k'})$ with
  $c_{1}=2k$ and $k' = \bfrr_{1}(\ckcO_{b})$.
  \[
    W_{\lamckb} = S_{C_{1}}\times S_{C_{2}}\times \cdots S_{C_{k'}}.
  \]
  The symbol of trivial representation of trivial group of type D is
  \[
    \binom{0,1, \cdots, k-1}{0,1, \cdots, k-1}.
  \]
  Now it is easy to see that
  \[
    J_{W_{\lamckb}}^{W_{b}}\sgn = ((\half C_{1}, \half C_{2},\cdots, \half C_{k'}),(\half C_{1}, \half C_{2},\cdots, \half C_{k'})).
  \]

  For the good parity part. Suppose
  $\ckcO_{g} = (2c_{1}+1, C_{2}, C_{2},C_{3},C_{3},\cdots, C_{k'},C_{k'})$ with
  $2c_{1}+1=2l+1$ and $2k'+1 = \bfrr_{1}(\ckcO_{g})$.
  \[
    W_{\lamckg} = W_{c_{1}}\times S_{C_{2}}\times \cdots \times S_{C_{k'}}.
  \]

  The symbol of sign representation of $W_{c_{1}}$ is
  \[
    \binom{0,1,2, \cdots, c_{1}}{1,2, \cdots, c_{1}}.
  \]

  By induction on number of columns, we see that when even column of length $2c$
  occurs, it adds length $c$ columns on the both sides of the bipartition; when
  odd column $C_{i}=2c_{i}+1$ with $i>1$ and multiplicity $2r'$ occur, the
  bifurcation happens: one can attach $r'$ columns of length $c_{i}+1$ on the
  right and $r'$ columns of length $c_{i}$ on the left (special representations)
  or attach $r'$ columns of length $c_{i}+1$ on the left and $r'$ columns of
  length $c_{i}$ on the right.

  Therefore,
  \[
    \begin{array}{ccc}
      J_{W_{\lamckg}}^{W_{g}} \sgn
      &\leftrightarrow&  \bF_{2}(\CPP(\ckcO_{g}))\\
      \cktau_{\sP}&\leftrightarrow & \sP
    \end{array}
  \]
  where
  \[
    \begin{split}
      \bfrr_{l+1}(\cktau_{L}) & =\half (\bfrr_{2l+1}(\ckcO_{g})-1)\\
      (\bfrr_{i}(\cktau_{L}), \bfrr_{i}(\cktau_{R})) & =
      \begin{cases}
        (\half(\bfrr_{2i-1}(\ckcO_{g})-1), \half(\bfrr_{2i}(\ckcO_{g})+1)) & (2i-1,2i)\notin \sP\\
        (\half(\bfrr_{2i}(\ckcO_{g})-1), \half(\bfrr_{2i-1}(\ckcO_{g})+1)) & (2i-1,2i)\in \sP
      \end{cases}
    \end{split}
  \]

  Now tensor with sign yields the result.

  We adopt the convention that
  \[
    S_{\cO} := \prod_{i\in \bN^{+}}S_{\bfrr_{i}(\cO)}
  \]
  so that $j_{S_{\cO}}\sgn = \cO$ for each partition $\cO$.

  Consider the orbit under the Springer correspondence. Let
  $\cO'_{b} = (r_{2}(\ckcO_{b}), r_{4}(\ckcO_{b}),\cdots, r_{2k}(\ckcO_{b}))$.
  Note that $\tau_{b} = j_{S_{\cO'_{b}}}^{W'_{b}} \sgn$. So
  \[
    \wttau:= j_{W'_{b}\times W_{g}}^{W_{n}} \tau_{b}\otimes \tau_{\emptyset}) = j_{S_{\cO'_{b}}\times W_{g}}^{W_{n}} \sgn\otimes \tau_{\sP}.
  \]
  Since the Springer correspondence commutes with parabolic induction, we get
  $\Spr(\wttau) = \Ind_{\GL_{\cO'_{b}}\times \Sp(2g)}^{\Sp(2n)} 0\times \cO_{g} = \cO_{b}\cupcol \cO_{g}$
}


\subsection{Counting special unipotent representation of $G=\Sp(p,q)$}.

The dual group of $\Gc = \Sp(2n,\bC)$ is $\ckGc = \SO(2n+1,\bC)$.

We set $(2m+1,2m') = (\abs{\ckcO_g},\abs{\ckcO_b})$.

We view $W_{2t}$ as the reflection group acts on $\bC^{t}$ as usual. Let
$H_{t}\cong W_t\ltimes \set{\pm 1}^t$ be the subgroup in $W_{2t}$ such that
\begin{itemize}
  \item the first factor $W_{t}$ sits in $S_{2t}$ commuting with the involution
        $(12)(34)\cdots ((2t-1)(2t))$.
  \item The element $(1,\cdots,1, \underbrace{-1}_{i\text{-th
        term}}, 1, \cdots, 1)\in \set{\pm 1}^{t}$ acts on $\bC^{2t}$ by
        \[
          % (x_{1},\cdots, x_{2i-2}, x_{2i-1}, x_{2i},x_{2i+1},\cdots, x_{2t} )
        (x_{1},x_{2},\cdots, x_{2t} ) \mapsto (x_{1},\cdots, x_{2i-2}, -x_{2i},-x_{2i-1},x_{2i+1},\cdots, x_{2t}).
        \]
\end{itemize}
Note that $H_{t}$ is also a subgroup of $W'_{2t}$.
Define the quadratic character
\[
  \begin{array}{rccc}
    \hsgn := q\otimes \sgn\colon & H_{t}=  W_{t}\ltimes \set{\pm 1}^{t}& \longrightarrow & \set{\pm 1}\\
    & (g,(a_{1},a_{2},\cdots, a_{t})) & \mapsto & a_{1}a_{2}\cdots a_{t}.
  \end{array}
\]
The most important formula is
\begin{equation}\label{eq:CC.C}
  \Ind_{H_{t}}^{W_{2t}} \hsgn = \sum_{\sigma\in \Irr(S_{t})} (\sigma,\sigma).
\end{equation}
Note that we have chosen the embedding $W_{t}\subset S_{2t}$ in $W_{2t}$. We have
\[
  \Ind_{H_{t}}^{W'_{2t}} \hsgn = \sum_{\sigma\in \Irr(S_{t})} (\sigma,\sigma)_{I}.
\]

\trivial{ In McGovern's paper, the coherent continuation representation is
  described as:
  \[
    \sum_{t,s,a,b}\Ind_{W_t\times (W_s\ltimes W(A_1)^s)\times W_a\times W_b}^{W_{t+2s+a+b}} \sgn\otimes (\triv \otimes \sgn)\otimes \triv\otimes \triv
  \]
  Now \eqref{eq:CC.C} was obtained by the following branching formula:
  \cite[p220 (6)]{Mc}
  \[
    I_n:= \Ind_{(W_s\ltimes W(A_1)^s)}^{W_{2s}}\triv\otimes \sgn = \sum \lambda\times \lambda
  \]
  where $\lambda$ running over all Young diagrams of size $s$. As McGovern
  claimed the proof of the above formula is similar to Barbasch's proof of
  \cite[Lemma~4.1]{B.W}:
  \[
    \Ind_{W_n}^{S_{2n}} \triv = \sum \sigma \quad \text{where $\sigma$ has even
      rows only}.
  \]

  Sketch of the proof (use branching rule and dimension counting): Note that
  $\dim I_n = \frac{(2p)! 2^{2p}}{p! 2^{2p}} = (2p)!/p! =\sum_\lambda \dim \lambda\times \lambda$
  (For the last equality:
  $\dim \lambda\times \lambda = (2p)! (\dim \lambda)^2/(p!)^2$ where
  $\dim \lambda$ is the dimension of $S_n$ representation determined by
  $\lambda$; But $\sum (\dim \lambda)^2 = p!$). On the other hand,
  $H :=W_s\ltimes W(A_1)^s\cap W_s\times W_s = \Delta W_s \subset W_{2s}$.
  $\triv \otimes \sgn|_H = \sgn$ of $\Delta W_s$ Therefore,
  $\lambda\times \lambda$ appears in $I_n$ by Mackey formula. Now by dimension
  counting, we get the formula. }

% By Vogan duality, the dual group for class $[\lamck]$ is
% \[
%   \SO(2m+1,\bC)\times \rO(2m',\bC).
% \]


\begin{lem}
  Recall \eqref{eq:Lam.C} for the definition of lattice $\Lambda_{n_{b},n_{g}}$.
  We have the following formula on the coherent continuation
  representations based on $\Lambda_{n_{b},n_{g}}$:
  \[
    \bigoplus_{p+q=n} \Coh_{\Lambda_{n_{b}, n_{g}}}(\Sp(p,q)) \cong \cC_{b}\times \cC_{g}.
  \]
  with
  \[
    \begin{split}
      \cC_g %& = \bigoplus_{p+q=m} \Cint{\rho}(\Sp(p,q)) \\
      &=\bigoplus_{\substack{t,s,r\in\bN\\2t+s+r=n_{g}}} \Ind_{H_{t} \times W_s\times W_t}^{W_{n_{g}}}
       \hsgn \otimes \sgn \otimes \sgn \\
      & =\bigoplus_{\substack{t,s,r\in\bN\\2t+s+r=n_{g}}} \Ind_{W_{2t}\times W_s\times W_r}^{W_{n_{g}}}
      (\sigma,\sigma)\otimes \sgn \otimes \sgn \\
      \cC_b & =
      \begin{cases}
        \Ind_{H_{t}}^{W'_{n_{b}}} \sgn &
        \text{if $n_{b}=2t'$ is even} \\
        0 & \text{otherwise}\\
      \end{cases}\\
    \end{split}
  \]
\end{lem}

Let
\[
  \PBP_{\star}(\ckcO_{b}) = \Set{(\imath,\jmath, \cP,\cQ) | %
    \begin{array}{l}
      (\imath,\jmath) = \tau_{b}\\
      \Im \cP \subset \set{\bullet}, \Im \cQ \subset\set{\bullet
      }\\
    \end{array}
  }
\]

\begin{lem}\label{lem:bp.C*}
  The set $\PBP_{\star}(\ckcO_{b})$ is a singleton.
  Let $\ckcO_{1}$ be the partition with only one box.
  Then  there only one special unipotent representation
  attached to $\ckcO_{b}\cuprow \ckcO_{1}$ which is given by
  \[
    \pi_{\star,\ckcO_{b}}:=
    \Ind_{\GL_{b}(\bH)}^{\Sp(b,b)}\pi_{\ckcO'_{b}}
  \]
  where $\pi_{\ckcO'_{b}}$ is the unique special unipotent representation
  attached
  to $\ckcO'_{b}$.
  % the parabolic induction from
  % the unique special unipotent representation of $\GL_{b}(\bH)$ to
  % $\Sp(b,b)$.
\end{lem}
\begin{proof}
  The claim of the size of $\PBP_{\star}$ is clear, i.e. there is only one
  special unipotent representation attached to $\ckcO'_{b}$. A prior,
  $\Ind_{\GL_{b}(\bH)}^{\Sp(b,b)}\pi_{\ckcO'_{b}}$ maybe a finite copies of the
  unique special unipotent representation. Meanwhile its associated variety is
  multiplicity free. hence $\pi_{\star,\ckcO_{b}}$ must be irreducible.
\end{proof}

For each $\sP\subset \CPP(\ckcO_{g})$, let
\[
  \PBP_{\star,\sP}(\ckcO_g)= \Set{(\imath,\jmath,\cP,\cQ)| %
    \begin{array}{l}
      (\imath,\jmath) = \tau_{\sP}\\
      \Im \cP \subset \set{\bullet}, \Im \cQ \subset\set{\bullet, s,r}\\
    \end{array}
  }
\]
where $\tau_{\sP}$ is defined in \Cref{lem:cell.C}.
By abuse of notation, we write
\[
\PBP_{\star}(\ckcO_g) :=\PBP_{\star,\emptyset}(\ckcO_g)
\]


\begin{prop}
  Suppose $\ckcO\in \Nil(\SO(2n+1,\bC))$ with decomposition
  $\ckcO = \ckcO_{b}\cuprow \ckcO_{g}$.
  \begin{enumT}
    \item The size of $\Unip_{\star}(\ckcO)$ is counted by
    $\PBP_{\star}(\ckcO_{g})$.
    \item The set $\PBP_{\star}(\ckcO_{g})$ is non-empty only if
    \[
      \bfrr_{2i-1}(\ckcO)> \bfrr_{2i}(\ckcO)
    \]
    for each $i\in \bN^{^{+}}$ with $\bfrr_{2i}(\ckcO)>0$.
    \item We have
    \[
      \abs{\PBP_{\star}(\ckcO_{g})} = \abs{\Nil_{C^{*}}(\cO_{g})}
    \]
    with $\cO_{g} := \dBV(\ckcO_{g})$.
    \item When $\Unip_{C^{*}}(\ckcO)\neq \emptyset$, there is a bijection
    \[
      \begin{array}{ccccc}
        \Nil_{C^{*}}(\cO_{g})& \longrightarrow & \Unip_{C^{*}}(\ckcO_{g})
        &\longrightarrow &\Unip_{C^{*}}(\ckcO) \\
        \sO_{g} & \mapsto & \pi_{\sO_{g}} & \mapsto
                         & \pi'\rtimes \pi_{\sO_{g}}.
      \end{array}
    \]
    Here $\pi'$ is the unique special unipotent representation of $\GL_p(\bH)$
    with associated variety $\cO_{b}$, $\pi_{\sO_{g}}$ is the unique special
    unipotent representation of $\Sp(p,q)$ with associated variety $\sO_{g}$
    such that $\Sign(\sO_{g}) = (2p,2q)$, and $\pi'\rtimes \pi_{\sO_{g}}$ denote
    the parabolic induction.
  \end{enumT}
\end{prop}
\begin{proof}
  \begin{enumPF}
    \item
    Suppose $(2i-1,2i)\in \sP\subset \CPP(\ckcO_{g})$.
    Let $(\imath,\jmath) :=\tau_{\sP}$. Then
    $\bfcc_{i}(\imath)>\bfcc_{i}(\jmath)$ which implies that
    $\PBP_{\star,\sP}(\ckcO_{g}) = \emptyset$.
    \item
    The first claim is similar to part (i), see \cite{BMSZ2}*{Proposition~10.1}.
    \item This is \cite{Mc}*{Theorem 6}, also see \cite{BMSZ2}.
    \item
    It follows from \Cref{lem:bp.C*} and the Vogan duality.
    See appendix.
    % One probably can prove the claim similar to that in
    % \cite{MR19}*{Proposition~5.5}.
    % We use Vogan duality, see Appendix.
  \end{enumPF}
\end{proof}

% In $\SO(2m',\bC)$, the double cell corresponds to $\ckcO_b$ consists of a single
% representation
% \[
%   \cktau_b = (\cksigma_b,\cksigma_b) \text{ such that
%   } \bfrr_i(\cksigma_b) = \bfrr_{2i}(\ckcO_b)/2
% \]
% Let $\ckcO'_b$ be the Young diagram such that
% $\bfrr_i(\ckcO'_b) = \bfrr_{2i}(\ckcO_b)$ and $\pi'_b$ be the unique special
% unipotent representation attached to $\GL_p(\bH)$
% \begin{lem}
%   The set $\Unip_{\ckcO_b}(\Sp(p,q))\neq \emptyset$ only if
%   $p=q = \abs{\cksigma_b}$ and in this case it consists a single element
%   \[
%     \pi_b := \Ind_P^{\Sp(p,p)} \pi'_{\cksigma_b} %^{\GL_p(\bH)}
%   \]
%   where $P$ has the Levi subgroup $\GL_p(\bH)$.
% \end{lem}


% The special reprsentation corresponds to $\ckcO_b$ is
% \[
%   aa
% \]

% \begin{lem}
%   The set of $\Unip_{\ckcO}(\Sp(p,q))$ is paramterized by the set such that
%   \[
%     ss
%   \]
% \end{lem}


\trivial{Let $W = W(\Gc)$ where $\Gc$ is natrualy embeded in $\GL(n,\bC)$. Let
  $s_{\varepsilon_i i,\varepsilon_j j}$ be the permutation matrix of index
  $i,j$, where the $(i,j)$-th entry is $\varepsilon_i$, the $(j,i)$-th entry is
  $\varepsilon_j$ and the other place is the identity matrix. Let
  $w_{i,\pm j}^{\epsilon}$ be the element such that
  \begin{itemize}
    \item it is in $\Gc$ and entries in $\set{0}\cup \mu_4$
    \item it lifts the element $e_i \leftrightarrow \pm e_j$ in the Weyl group.
    \item $(w_{i,\pm j}^{\epsilon})^2 = \epsilon 1\in \set{\pm 1}$.
  \end{itemize}
  Let \[h_{\pm i}^+ = \diag(1,\cdots, 1, \pm 1, 1, \cdots, 1)\] where $\pm 1$ is
  the $i$-th place. Let
  \[h_{\pm i}^- = \diag(1,\cdots, 1, \pm \sqrt{-1}, 1, \cdots, 1)\] where
  $\pm \sqrt{-1}$ is the $i$-th place. Let $e_{\pm i} = s_{\pm i,\pm (n-i+1)}$.

  Let $\ckww_{i,\pm j}^\pm$ be the lift of $e_i \leftrightarrow \pm e_j$ such
  that
  \begin{itemize}
    \item it is in $\Gc$ and entries in $\set{0}\cup \mu_4$
    \item it lifts the element $e_i \leftrightarrow \pm e_j$ in the Weyl group.
    \item $(w_{i,\pm j}^{\epsilon})^2|_{[i,j]} = \epsilon 1\in \set{\pm 1}$ here
          ``$|_{[i,j]}$'' means restricts on the $e_i,e_j,-e_i,-e_j$-weights
          space.
  \end{itemize}

  Let
  \[
    \begin{split}
      x_{b,s,r} &= w_{1,2}^+\cdots w_{2b-1,2b}^+\, h_{2b+1}^+\cdots h_{2b+s}^+ \\
      y_{b,s,r}^+ &= \ckww_{1,-2}^+\cdots \ckww_{2b-1,-2b}^+\,
      e_{2b+1}\cdots e_{2b+s+r} \\
      y_{b,s,r}^- &= \ckww_{1,-2}^-\cdots \ckww_{2b-1,-2b}^-
    \end{split}
  \]

  We compute the parameter space
  \[
    \begin{split}
      \cZ_g &=  \bigcup_{2b+s+r=m} W_m\cdot (x_{b,s,r}^+, y_{b,s,r}^+)\\
      \cZ_b &=  \bigcup_{2b=m} W_m\cdot (x_{b,0,0}^+, y_{b,0,0}^-)\\
    \end{split}
  \]

}


\subsection{Counting special unipotent representation of $G = \Sp(2n,\bR)$}

In this section, we consider the case where $\star = C$.

Recall \eqref{eq:Lam.C} for the definition of the lattice
$\Lambda_{n_{b},n_{g}}$.

\begin{lem}
  We have the following formula on the coherent continuation representations
  based on $\Lambda_{n_{b},n_{g}}$:
  \[
    \Coh_{\Lambda_{n_{b},n_{g}}}(\Sp(2n,\bR)) = \cC_g\otimes \cC_b
  \]
  with
  \[
    \begin{split}
      \cC_g &=
      \bigoplus_{2t+a+c+d=n_{g}}\Ind_{H_{t} \times S_{a} \times W_c\times W_d}^{W_{n_{g}}} 1 \otimes \sgn \otimes 1 \otimes 1\\
      \cC_b &= \Res_{W_{n_{b}}}^{W'_{n_{b}}}\left( \bigoplus_{\substack{2t+a=n_{b}}} \Ind_{H_{t} \times S_a}^{W_{n_{b}}} \hsgn\otimes 1\right) \\
    \end{split}
  \]
  \qed
\end{lem}

\trivial{ Consider $\cC_{b}$. The real Cartan must be
  ${\bC^{\times}}^{t} \times {\bR^{\times}}^{t}$ In real Weyl group is
  $H_{t}\times W_{a}$ generated by The $H_{t}$ action is known. $W_{a}$ is
  generated by the reflections of $e_{i}\pm e_{j}$ $n-2t<i\neq j\leq n$. We
  consider the regular character
  $\gamma = *\otimes \underbrace{\abs{}^{\half}\otimes \cdots \otimes \abs{}^{\half}}_{a\text{-terms}}$.
  The cross action of $s_{e_{n-1}+e_{n}}$ is given by
  \[
    \begin{split}
      s_{e_{n-1}+e_{n}}\times \gamma & = *\otimes \underbrace{\abs{}^{\half}\otimes \cdots \otimes \abs{}^{\half} \otimes \sgn \abs{}^{-\half}
        \otimes \sgn \abs{}^{-\half}}_{a\text{-terms}} \\
      &\neq
      s_{e_{n-1}+e_{n}}\cdot \gamma \\
      &= *\otimes \underbrace{\abs{}^{\half}\otimes \cdots \otimes \abs{}^{\half} \otimes \abs{}^{-\half} \otimes \abs{}^{-\half}}_{a\text{-terms}}.
    \end{split}
  \]
  Now we see that the cross stabilizer should be
  $H_{t}\times S_{a}$.
}

Now $[\tau_{b}:\cC_{b}]$ is counted by the size of the following set
\[
  \PBP_{\star}(\ckcO_b) := \Set{(\imath,\jmath,\cP,\cQ)| %
    \begin{array}{l}
      (\imath,\jmath) = \tau_{b}\\
      \Im \cP \subset \set{\bullet,c}, \Im \cQ \subset\set{\bullet,d}\\
    \end{array}
  }
\]
and for each $\sP\subset \CPP(\ckcO_{g})$, the multiplicity
$[\tau_{\sP}: \cC_{g}]$ is counted by the size of
\[
  \PBP_{\star}(\ckcO_g,\sP):= \Set{(\imath,\jmath,\cP,\cQ)| %
    \begin{array}{l}
      (\imath,\jmath) = \tau_{\sP}\\
      \Im \cP \subset \set{\bullet,r,c,d}, \Im \cQ \subset\set{\bullet, s}\\
    \end{array}
  }
\]
and for each $\sP$

Recall the definition of $\PP_{A^{\bR}}(\ckcO'_{b})$ in  \eqref{eq:PP.AR}.

\begin{lem}
  For each $\uptau' \in \PP_{A^{\bR}}(\ckcO'_{b})$, there is a unique painted
  bipartition $\uptau\in \PBP_{\star}(\ckcO_{b})$ such that
  \[
    \cP_{\uptau'}(i,j) = d \quad % \text{if and only if}
    \Leftrightarrow
    \quad
    \cQ_{\uptau}(i,j) = d \quad \forall (i,j)\in \BOX{\imath}\\
  \]
  The map gives a bijection
  \[
    \begin{array}{ccccccc}
      \Unip_{\ckcO'_{b}}(\GL_{b}(\bR))&\longleftarrow
      &\PP_{A^{\bR}}(\ckcO'_{b}) & \longleftrightarrow
      & \PBP_{\star}(\ckcO_{b}) & \longrightarrow
      & \Unip_{\ckcO_{b}\cuprow \ckcO_{1}}(\Sp(2b,\bR))\\
      \pi_{\uptau'} & \mapsfrom & \uptau' & \mapsto
      & \uptau & \mapsto & \pi_{\uptau}.
    \end{array}
  \]
  Here $\pi_{\uptau'}$ is defined in \eqref{eq:u.GLR} and
  \[
    \pi_{\uptau} := \Ind_{\GL_{b}(\bR)}^{\Sp_{2b}(\bR)} \pi_{\uptau'}.
  \]
\end{lem}
\begin{proof}
  The claims about painted bipartitions are clear.
  The irreduciblity of $\pi_{\uptau}$ follows from the multiplicity one
  of its wavefront cycle.

  The representations $\pi_{\uptau_{1}}\neq \pi_{\uptau_{2}}$ if
  $\uptau_{1}\neq \uptau_{2}$ since they have different cuspidal data by the
  construction.
\end{proof}

% \[
%   \begin{split}
%     \cC_g &= \bigoplus_{p+q=2m+1}\Cint{\rho}(\SO(p,q))  \\
%     &=
%     \bigoplus_{2b+s+r+c+d=m}\Ind_{W_b\ltimes \set{-1}^b\times W_s\times W_r\times W_c\times W_d}^{W_m} 1 \otimes \det \otimes \det \otimes 1 \otimes 1\\
%     \cC_b &= \Cint{\rho}(\SO^*(2m)) = \bigoplus_{\substack{2s+b=m}}\Ind_{W_s\ltimes \set{-1}^s\times S_b}^{W_m} 1 \otimes\sgn\otimes \sgn
%   \end{split}
% \]

We define
\[
 \PBPes(\ckcO_{g}) := \PBP_{\star}(\ckcO_{g})\times \CQ(\ckcO_{g}).
\]

Now we consider the good parity part. We defer the proof of the following lemma
in the \Cref{app:comb}

\begin{lem}
  For each $\sP\in \subset \CPP(\ckcO_{g})$,
  we have
  \[
    \abs{\PBP_{\star}(\ckcO_{g},\sP)} =
    \abs{\PBP_{\star}(\ckcO_{g},\emptyset)}.
  \]
  In particular, we have
  \[
    \Unip_{\star}(\ckcO_{g}) = \abs{\PBPes(\ckcO_{g})}.
  \]
\end{lem}

\section{Type $\wtC$}


The left cell.
\begin{lem}
  Suppose $\star = \wtC$, $\ckcO_{b}$ has $2k$ rows. Here each row in $\ckcO_{b}$ has
  odd length and each row in $\ckcO_{g}$ has even length, and
  $\WLamck = W_{b}\times W'_{g}$ where $b=\half \abs{\ckcO_{b}}$ and
  $g = \half\abs{\ckcO_{g}}$.
  Let
  \[
    \begin{split}
      \tau_{b} =  & \left( (\frac{\bfrr_{2}(\ckcO_{b})+1}{2}, \frac{\bfrr_{4}(\ckcO_{b})+1}{2}, \cdots, \frac{\bfrr_{2k}(\ckcO_{b})+1}{2}),\right.\\
        &\ \ \left. (\frac{\bfrr_{2}(\ckcO_{b})-1}{2}, \frac{\bfrr_{4}(\ckcO_{b})-1}{2}, \cdots, \frac{\bfrr_{2k}(\ckcO_{b})-1}{2})\right) \in \Irr(W_{b}).
    \end{split}
  \]
  Set
  \[
    \CPP(\ckcO_{g}) = \set{(2i-1,2i)| \bfrr_{2i-1}(\ckcO_{g})> \bfrr_{2i}(\ckcO_{g}), \text{
        and } i\in \bN^{+}}
  \]
  and $\CQ(\ckcO)= \bF_{2}[\CPP(\ckcO_{g})]$.

  For $\sP \in \CQ(\ckcO)$, let
  \[
    \tau_{\sP} := (\imath,\jmath) \in \Irr(W'_{g})
  \]
  such that for all $i\geq 1$
  \[
  (\bfcc_{i}(\imath), \bfcc_{i}(\jmath)):=
  \begin{cases}
    (\half \bfrr_{2i-1}(\ckcO_{g}), \half \bfrr_{2i}(\ckcO_{g}))
    & \text{if } (2i-1,2i)\notin \sP,\\
    (\half \bfrr_{2i}(\ckcO_{g}),\half \bfrr_{2i-1}(\ckcO_{g})) & \text{otherwise.}
  \end{cases}
  \]

  Then we have the following bijection
  \[
    \begin{array}{ccc}
      \CQ(\ckcO) &\longrightarrow & \LC(\ckcO)\\
      \sP & \mapsto & \tau_{b}\otimes \tau_{\sP},
    \end{array}
  \]
  such that $\tau_{b}\otimes \tau_{\sP}$ is the special representation in
  $\LC(\ckcO)$.

  We remark that if $\CPP(\ckcO_{g})=\emptyset$ the the
  representation $\tau_{\emptyset}$ has label $I$.
\end{lem}

\trivial{
  The bad parity part is the same as the case when $\star= B$.


  For the good parity part, note that the trivial representation of
  the trivial group has symbol
  \[
    \binom{0,1,\cdots, r}{0,1,\cdots, r}.
  \]
  Here we assume $\ckcO_{g}$ has at most $2r$ rows.

  Now the bifurcation happens for the odd length column.
  }


  The coherent continuation representation.


  Fix $n_{b},n_{g}$ such that $n_{b}+ n_{g} = n$. Let
  \begin{equation} \label{eq:Lam.C}
    \Lambda_{n_{b},n_{g}} = (\underbrace{0, \cdots, 0}_{n_{b}\text{-terms}}, \underbrace{\half, \cdots, \half}_{n_{g}\text{-terms}}) + Q.
  \end{equation}

  We use Renard-Trapa's result.

\begin{lem}
  We have the following formula on the coherent continuation
  representations based on $\Lambda_{n_{b},n_{g}}$:
  \[
    \bigoplus_{p+q=n} \Coh_{\Lambda_{n_b, n_g}}(\Mp(2n,\bR)) \cong \cC_{b}\otimes \cC_{g}.
  \]
  with
  \[
    \begin{split}
      \cC_g %& = \bigoplus_{p+q=m} \Cint{\rho}(\Sp(p,q)) \\
      &  = \Res_{W_{n_{g}}}^{W'_{n_{g}}}
      \left( \bigoplus_{\substack{t,a,a'\in \bN\\2t+a+a'=n_{g}}}
        \Ind_{H_{t}\times S_{a}\times S_{a'}}^{W_{n_{b}}}
        \hsgn\otimes 1 \otimes\sgn \right)\\
      % & =\bigoplus_{\substack{t,s,r\in\bN\\2t+s+r=n_{g}}} \Ind_{W_{2t}\times W_s\times W_r}^{W_{n_{g}}}
      % (\sigma,\sigma)\otimes \sgn \otimes \sgn \\
      \cC_b & =
      \bigoplus_{\substack{t,c,d\in \bN\\2t+c+d=n_{b}}}
      \Ind_{H_{t}\times W_{c}\times W_{d}}^{W_{n_{b}}} \hsgn\otimes 1\otimes 1
    \end{split}
  \]
\end{lem}
\begin{proof}
  This is contained in \cite{RT1,RT2}.
  By \cite{RT2}*{Theorem~5.2} and \cite{RT2}*{Corollary~4.5~(2)},
  $ \Coh_{\Lambda_{n_{b},n_{g}}}(\Mp(2n,\bR)) $
  is dual to $\cC_{g}\otimes \ckcC_{b}$ where
  $\cC_{g}$ is the coherent continuation representation based on the
  lattice $\Lambda_{0,n_{g}}$
  and
  \[
    \begin{split}
      \ckcC_{2n_{b},\bR} &\cong \bigoplus_{\substack{p,q\in \bN\\p+q=n_{b}}}
      \Coh_{\Lambda_{n_{b},0}}(\Sp(p,q))\\
      & = \bigoplus_{\substack{t,c,d\in \bN\\2t+c+d=n_{b}}} \Ind_{H_{t}\times W_{c}\times W_{d}}^{W_{n_{b}}} \hsgn\otimes \sgn \otimes\sgn
    \end{split}
  \]

  The main result in \cite{RT1} implies that
  \[
    \cC_{g} = \Res_{W_{n_{g}}}^{W'_{n_{g}}} \left(
  \bigoplus_{\substack{t,a,a'\in \bN\\2t+a+a'=n_{g}}} \Ind_{H_{t}\times S_{a}\times S_{a'}}^{W_{n_{b}}} \hsgn\otimes 1 \otimes\sgn
  \right)
  \]
  is self-dual.

  Tensor with the sign representation yields the lemma.
\end{proof}

% \subsection{}
% In this section, we count special unipotent representations of real orthogonal
% groups (type $B$,$D$), symplectic groups (type $C$) and metaplectic groups (type
% $\wtC$).

% Recall that
% \[
%   \text{good pairity} =
% \begin{cases}
%  \text{odd} & \text{in type $C$ and $D$}\\
%  \text{even} & \text{in type $C$ and $D$}\\
% \end{cases}
% \]
% We decompose $\ckcO  = \ckcO_g\cup \ckcO_b$ as before.

% Let $\tsgn$ be the character of $W_n$ inflated from the sign character of $S_n$ via the
% natural map $W_n \rightarrow S_n$.
% Recall the following formula
% \[
%   \Ind_{W_b\ltimes \set{\pm 1}^b}^{W_2b} 1 \otimes \sgn = \sum_{\sigma} (\sigma,\sigma)
% \]
% where $\sigma$ running over all Young diagrams of size $b$.


%\subsection{Type C}


\section{Type $B$}

  The dual group of $\Gc = \SO(2n+1,\bC)$ is $\ckGc = \Sp(2n,\bC)$. The odd is
  the bad parity, and even is the good parity.


  Fix $n_{b},n_{g}$ such that $n_{b}+ n_{g} = n$. Let
  \begin{equation} \label{eq:Lam.C}
    \Lambda_{n_{b},n_{g}} = (\underbrace{\half, \cdots, \half}_{n_{b}\text{-terms}}, \underbrace{0, \cdots, 0}_{n_{g}\text{-terms}}) + Q.
  \end{equation}

  Suppose $\ckcO\in \Nil(\SO(2n+1,\bC))$ with decomposition
  $\ckcO = \ckcO_{b}\cuprow \ckcO_{g}$.

  Let $\ckcO'_b$ be the Young diagram such that
  $\bfrr_i(\ckcO'_b) = \bfrr_{2i}(\ckcO_b)$ and $\cO'_{b}$ be the transpose of
  $\ckcO'_{b}$.

  \subsection{The left cell}

\begin{lem}
  Suppose $\star = B$, $\ckcO_{b}$ has $2k$ rows. Here each row in $\ckcO_{b}$ has
  odd length and each row in $\ckcO_{g}$ has even length, and
  $\WLamck = W_{b}\times W_{g}$ where $b=\half \abs{\ckcO_{b}}$ and
  $g = \half\abs{\ckcO_{g}}$.
  Let
  \[
    \begin{split}
      \tau_{b} =  & \left( (\frac{\bfrr_{2}(\ckcO_{b})+1}{2}, \frac{\bfrr_{4}(\ckcO_{b})+1}{2}, \cdots, \frac{\bfrr_{2k}(\ckcO_{b})+1}{2}),\right.\\
        &\ \ \left. (\frac{\bfrr_{2}(\ckcO_{b})-1}{2}, \frac{\bfrr_{4}(\ckcO_{b})-1}{2}, \cdots, \frac{\bfrr_{2k}(\ckcO_{b})-1}{2})\right) \in \Irr(W_{b}).
    \end{split}
  \]
  Set
  \[
    \CPP(\ckcO_{g}) = \set{(2i,2i+1)| \bfrr_{2i+1}(\ckcO_{g})> \bfrr_{2i}(\ckcO_{g}), \text{
        and } i\in \bN^{+}}
  \]
  and $\CQ(\ckcO)= \bF_{2}[\CPP(\ckcO_{g})]$.

  For $\sP \in \CQ(\ckcO)$, let
  \[
    \tau_{\sP} := (\imath,\jmath) \in \Irr(W_{b})
  \]
  such that
  \[
    \bfcc_{1}(\jmath)  := \half\bfrr_{1}(\ckcO_{g})
  \]
  and for all $i\geq 1$
  \[
  (\bfcc_{i}(\imath), \bfcc_{i+1}(\jmath)):=
  \begin{cases}
    (\half \bfrr_{2i}(\ckcO_{g}), \half \bfrr_{2i+1}(\ckcO_{g}))
    & \text{if } (2i,2i+1)\notin \sP,\\
    (\half \bfrr_{2i+1}(\ckcO_{g}),\half \bfrr_{2i}(\ckcO_{g})) & \text{otherwise.}
  \end{cases}
  \]

  Then we have the following bijection
  \[
    \begin{array}{ccc}
      \CQ(\ckcO) &\longrightarrow & \LC(\ckcO)\\
      \sP & \mapsto & \tau_{b}\otimes \tau_{\sP},
    \end{array}
  \]
  such that $\tau_{b}\otimes \tau_{\sP}$ is the special representation in
  $\LC(\ckcO)$.
\end{lem}

\trivial{
  In this case, bad parity is odd and every odd row occurs with with even times.
  We take the convention that
  % $2\cO = [2r_{i}]$ if $\cO = [r_{i}]$.
  % We also write $[r_{i}]\cup [r_{j}] = [r_{i},r_{j}]$.
  $\dagger \cO = [r_{i}+1]$.
  By abuse of notation, let $\dagger_{n} \sigma$  denote the
  $j_{S_{n} \times W_{\abs{\sigma}}}^{W_{n+\abs{\sigma}}} \sgn\otimes \sigma$.
  We can write
  \[
    \ckcO_{b} = [2r_{1}+1, 2r_{1}+1, \cdots, 2r_{k}+1,2r_{k}+1]
    = (2c_{0},2c_{1},2c_{1}, \cdots, 2c_{l}, 2c_{l})
  \]
  with $k = c_{0}$ and $l = r_{1}$.

\[
\begin{split}
  W_{\lamckb} &= W_{c_{0}} \times S_{2c_{1}} \times S_{2c_{2}}\times \cdots \times S_{2c_{l}}\\
  \cksigma_{b} &:= \sigma_{b}\otimes \sgn = j_{W_{\lamckb}}^{W_{b}} \sgn \\
  & = \dagger_{2c_{l}}\cdots \dagger_{2c_{1}}
  \binom{0, 1, \cdots, c_{0}}{1, \cdots, c_{0}}\\
  & =
  \binom{0, 1+r_{k}, 2+r_{k-1}\cdots, c_{0}+r_{1}}{1+r_{k},2+r_{k-1}, \cdots, c_{0}+r_{1}}\\
  & = ([r_{1},r_{2},\cdots, r_{k}],[r_{1}+1,r_{2}+1,\cdots,r_{k}+1])\\
  &= ((c_{1},c_{2},\cdots, c_{k}),(c_{0},c_{1}, \cdots, c_{l}))\\
\end{split}
\]

Therefore
\[
  \begin{split}
    \sigma_{b} &= \cksigma_{b}\otimes \sgn = ((r_{1}+1,r_{2}+1,\cdots,r_{k}+1),(r_{1},r_{2},\cdots, r_{k})) \\
    & = j_{S_{2r_{1}+1}\times \cdots S_{2r_{k}+1}}^{W_{b}} \sgn\\
    & = j_{S_{b}}^{W_{b}} (2r_{1}+1, 2r_{2}+1, \cdots, 2r_{k}+1)
  \end{split}
\]
which corresponds to the orbit
\[
  \cO_{b} = (2r_{1}+1, 2r_{1}+1,2r_{2}+1, 2r_{2}+1,  \cdots,2r_{k}+1, 2r_{k}+1 ) = \ckcO_{b}^{t}.
\]
(Note that $\cO'_{b} = (2r_{1}+1,2r_{2}+1, \cdots, 2r_{k}+1)$ which corresponds
to $j_{W_{L_{b}}}^{S_{b}}\sgn$ and $\ind_{L}^{G} \cO'_{b} = \cO_{b}$.
)
% This implies the unique special representation is
% \[
%   \sigma_{b} = (j_{W_{\lamckb}}^{W_{b}}\sgn), \quad \text{where } W_{L,b} = \prod_{i=1}^{k} S_{2r_{i}+1}.
% \]
The $J$-induction is calculated by \cite{Lu}*{(4.5.4)}.
It is easy to see that in our case $J_{W_{\lamckb}}^{W_{b}} \sgn$ consists of
the single special representation by induction.


Now we consider the good parity parts.


%First assume that there is even number of rows.
Consider
\[
\cO_{g} = [2r_{1},2r_{2}, \cdots, 2r_{2k-1},2r_{2k}]
= (C_{1},C_{1}, C_{2},C_{2},\cdots, C_{l}, C_{l}).
\]
with $l = r_{1}$ and $k = \ceil{C_{1}/2}$.
Write  $\ckLC_{\ckcO} = J_{W_{\lamck}}^{W_{[\lamck]}}\sgn$.



Note that the trivial representation of the trivial group has symbol
\[
\binom{0,1, 2, \cdots, k\phantom{-1}}{0,1, \cdots, k-1}.
\]
Now it easy to deduce that
\[
\begin{split}
\cksigma_{[\underbrace{2r,2r, \cdots, 2r}_{2k+1}]}
=& ([\underbrace{r,r, \cdots, r}_{k+1}], [\underbrace{r,r, \cdots, r}_{k}]),
% \cksigma_{[\underbrace{2r,2r, \cdots, 2r}_{2k+1}]}
% =& ([\underbrace{r,r, \cdots, r}_{k+1}], [\underbrace{r,r, \cdots, r}_{k}])\quad
\text{and}\\
\ckLC_{[\underbrace{2r',2r', \cdots, 2r',2r}_{2k+1\text{ terms}}]}
=&
\begin{cases}
  ([\underbrace{r',r', \cdots, r',r'}_{k+1}], [\underbrace{r',r', \cdots, r',r'}_{k+1}]) &
  \text{if } r'=r\geq 0 \\
  ([\underbrace{r',r', \cdots, r',r}_{k+1}], [\underbrace{r',r', \cdots, r',r'}_{k}]) &\\
 +([\underbrace{r',r', \cdots, r',r'}_{k+1}], [\underbrace{r',r', \cdots, r',r}_{k}]) &
  \text{if } r'>r \geq 0\\
 \text{(the first term is special)}& \\
\end{cases}
\end{split}
\]
%where the first term is the special representation.

Now
\[
  \begin{split}
    \cksigma_{\ckcO_{g}} & =
    J_{S_{C_{1}}\times \cdots \times S_{C_{l}}}^{W_{a}} \sgn\\
    % =& ((\ceil{C_{1}/2},\ceil{C_{2}/2}, \ceil{C_{1}/2}),
    % (\floor{C_{1}/2},\floor{C_{2}/2}, \floor{C_{1}/2}))\\
    =& \ckLC_{[\underbrace{2r_{2k},2r_{2k}, \cdots,2r_{2k},2r_{2k+1}}_{2k+1\text{terms}}]}\\
    & \cuprow \LC_{[2(r_{1}-r_{2k}), 2(r_{2}-r_{2k}), \cdots, 2(r_{2k-1}-r_{2k})]}.
  \end{split}
\]
By induction on the number of columns, we conclude that $\ckLC_{\ckcO_{g}}$ is in one-one corresponds to
the subsets of
\[
\CPP(\ckcO_{g}) = \set{(2i,2i+1)| \bfrr_{2i+1}(\ckcO_{g})> \bfrr_{2i}(\ckcO_{g}), \text{ and
  } i\in \bN^{+}}.
\]
For $\sP\in \CQ(\ckcO_{g})$, let $\sigma_{\sP} = (\imath,\jmath)$
such that
\[
\begin{split}
  \bfrr_{1}(\imath) &:= \half \bfrr_{1}(\ckcO_{g})\\
  (\bfrr_{l+1}(\imath), \bfrr_{l}(\jmath))&:=
  \begin{cases}
    (\half \bfrr_{2l+1}(\ckcO_{g}), \half \bfrr_{2l}(\ckcO_{g}))
    & \text{if } (2l,2l+1)\notin \sP\\
    (\half \bfrr_{2l}(\ckcO_{g}), \half \bfrr_{2l+1}(\ckcO_{g}))
    & \text{otherwise}
  \end{cases}
\end{split}
\]


Note that according to Lusztig and BV, the subsets of $\CPP(\ckcO)$ is in
one-one correspondence to the canonical quotient $\CQ(\ckcO) = \bF_{2}[\CPP(\ckcO)]$.
}

\subsection{Counting special unipotent representation of $G=\SO(2p+1,2q)$ with
$p+q=n$}.

The dual group of $\Gc = \SO(2n,\bC)$ is $\ckGc = \SO(2n+1,\bC)$.

Let \[
  \Lambda_{n_{1}, n_{2}} = (\underbrace{0,\cdots,0}_{n_{1}\text{-terms}}, \underbrace{\half,\cdots,\half}_{n_{2}\text{-terms}}, )
\]

We set $(2n_{g},2n_{b}) = (\abs{\ckcO_g},\abs{\ckcO_b})$.

\begin{lem}
  We have the following formula on the coherent continuation
  representations based on $\Lambda_{n_{1},n_{2}}$:
  \[
    \bigoplus_{p+q=n} \Coh_{\Lambda_{n_1, n_2}}(\SO(2p+1,2q)) \cong \cC_{b}\otimes \cC_{g}.
  \]
  with
  \[
    \begin{split}
      \cC_g %& = \bigoplus_{p+q=m} \Cint{\rho}(\Sp(p,q)) \\
      &=\bigoplus_{\substack{t,s,r\in\bN\\2t+s+r=n_{g}}} \Ind_{H_{t} \times S_{a}\times W_s\times W_r}^{W_{n_{g}}}
      \hsgn \otimes 1 \otimes \sgn \otimes \sgn \\
      % & =\bigoplus_{\substack{t,s,r\in\bN\\2t+s+r=n_{g}}} \Ind_{W_{2t}\times
      % W_s\times W_r}^{W_{n_{g}}}
      % (\sigma,\sigma)\otimes \sgn \otimes \sgn \\
      \cC_b & =
      \bigoplus_{\substack{t,c,d\in \bN\\2t+c+d=n_{b}}}
      \Ind_{H_{t}\times W_{c}\times W_{d}}^{W_{n_{b}}} \hsgn\otimes 1\otimes 1
    \end{split}
  \]
\end{lem}


From now on, we set $ (2n_{1},2n_{2}) := (2g,2b) :=(\abs{\ckcO_g},\abs{\ckcO_b})$.

Clearly $[\tau_{b}:\cC_{b}]$ is counted by the size of the following set
\[
  \PBP_{\star}(\ckcO_b) := \Set{(\imath,\jmath,\cP,\cQ)| %
    \begin{array}{l}
      (\imath,\jmath) = \tau_{b}\\
      \Im \cP \subset \set{\bullet,c,d}, \Im \cQ \subset\set{\bullet}\\
    \end{array}
  }
\]
and for each $\sP\subset \CPP(\ckcO_{g})$, the multiplicity
$[\tau_{\sP}: \cC_{g}]$ is counted by the size of
\[
  \PBP_{\star}(\ckcO_g,\sP):= \Set{(\imath,\jmath,\cP,\cQ)| %
    \begin{array}{l}
      (\imath,\jmath) = \tau_{\sP}\\
      \Im \cP \subset \set{\bullet,c}, \Im \cQ \subset\set{\bullet, s,r,d}\\
    \end{array}
  }
\]
and for each $\sP$.

Recall the definition of $\PP_{A^{\bR}}(\ckcO'_{b})$ in  \eqref{eq:PP.AR}.

\begin{lem}
  For each $\uptau' \in \PP_{A^{\bR}}(\ckcO'_{b})$, there is a unique painted
  bipartition $\uptau\in \PBP_{\star}(\ckcO_{b})$ such that
  \[
    \cP_{\uptau'}(i,j) = d \quad % \text{if and only if}
    \Leftrightarrow
    \quad
    \cP_{\uptau}(i,j) = d \quad \forall (i,j)\in \BOX{\imath}\\
  \]
  The map gives a bijection
  \[
    \begin{array}{ccccccc}
      \Unip_{\ckcO'_{b}}(\GL_{b}(\bR))&\longleftarrow
      &\PP_{A^{\bR}}(\ckcO'_{b}) & \longleftrightarrow
      & \PBP_{\star}(\ckcO_{b}) & \longrightarrow
      & \Unip_{\ckcO_{b}}(\SO(2b+1,2b))\\
      \pi_{\uptau'} & \mapsfrom & \uptau' & \mapsto
      & \uptau & \mapsto & \pi_{\uptau}.
    \end{array}
  \]
  Here $\pi_{\uptau'}$ is defined in \eqref{eq:u.GLR} and
  \[
    \pi_{\uptau} := \Ind_{\GL_{b}(\bR)\times \SO(1,0)}^{\SO(2b+1,2b)} \pi_{\uptau'}.
  \]
\end{lem}
\begin{proof}
  The claims about painted bipartitions are clear.
  The irreduciblity of $\pi_{\uptau}$ follows from the multiplicity one
  of its wavefront cycle.

  The representations $\pi_{\uptau_{1}}\neq \pi_{\uptau_{2}}$ if
  $\uptau_{1}\neq \uptau_{2}$ since they have different cuspidal data by the
  construction.
\end{proof}

Now we consider the good parity part. We defer the proof of the following lemma
in the \Cref{app:comb}

\begin{lem}
  For each $\sP\in \subset \CPP(\ckcO_{g})$,
  we have
  \[
    \abs{\PBP_{\star}(\ckcO_{g},\sP)} =
    \abs{\PBP_{\star}(\ckcO_{g},\emptyset)}.
  \]
  In particular, we have
  \[
    \Unip_{\star}(\ckcO_{g}) = \abs{\PBPes(\ckcO_{g})}.
  \]
  Here $\Unip_{\star}(\ckcO_{g})$ denote the set of all unipotent
  representations attached to the inner class $\SO(2p+1,2q)$.
\end{lem}


\section{Type D}


Let $\ckcO = \ckcO_{b}\cuprow \ckcO_{g}$.

\subsection{The left cell $\LC_{\ckcO}$}
%We compute the left cell $\LC_{\ckcO}$ case by case.

\begin{lem}
  Suppose $\star \in \set{D, D^{*}}$, $\ckcO_{b}$ has $2k$ rows and $\ckcO_{g}$
  has $2l$-rows. Here each row in $\ckcO_{b}$ has even length and each row in
  $\ckcO_{g}$ has odd length, and $\WLamck = W'_{b}\times W'_{g}$ where
  $b= \frac{\abs{\ckcO_{b}}}{2}$ and $g = \frac{\abs{\ckcO_{g}}}{2}$. Let
  \[
    \tau_{b} = \left( (\half\bfrr_{2}(\ckcO_{b}), \half\bfrr_{4}(\ckcO_{b}),\cdots, \half\bfrr_{2k}(\ckcO_{b}), (\half\bfrr_{2}(\ckcO_{b}), \half\bfrr_{4}(\ckcO_{b}),\cdots, \half\bfrr_{2k}(\ckcO_{b}) \right)_{I}\in \Irr(W'_{b}).
  \]
  Set
  \[
    \CPP(\ckcO_{g}) = \set{(2i,2i+1)| \bfrr_{2i}(\ckcO_{g})> \bfrr_{2i+1}(\ckcO_{g})>0, \text{
        and } i\in \bN^{+}}
  \]
  and $\CQ(\ckcO)= \bF_{2}[\CPP(\ckcO_{g})]$.

  For $\sP \in \CQ(\ckcO)$, let
  \[
    \tau_{\sP} := (\imath,\jmath) \in \Irr(W'_{g})
  \]
  such that
  \[
    \begin{split}
      \bfcc_{1}(\imath)  &:= \half(\bfrr_{1}(\ckcO_{g})+1)\\
      (\bfcc_{l+1}(\imath), \bfcc_{l}(\jmath))  &:= (0,\half(\bfrr_{2l}(\ckcO_{g})-1))
    \end{split}
  \]
  and for all $1\leq i< l$
  \[
  (\bfcc_{i+1}(\imath), \bfcc_{i}(\jmath)):=
  \begin{cases}
    (\half (\bfrr_{2i+1}(\ckcO_{g})+1),
    \half (\bfrr_{2i}(\ckcO_{g})-1))
    & \text{if } (2i,2i+1)\notin \sP,\\
    (\half (\bfrr_{2i}(\ckcO_{g})+1),\half (\bfrr_{2i+1}(\ckcO_{g})-1)) & \text{otherwise.}
  \end{cases}
  \]

  Then we have the following bijection
  \[
    \begin{array}{ccc}
      \CQ(\ckcO) &\longrightarrow & \LC(\ckcO)\\
      \sP & \mapsto & \tau_{b}\otimes \tau_{\sP}.
    \end{array}
  \]
  such that $\tau_{b}\otimes \tau_{\sP}$ is the special representation in
  $\LC(\ckcO)$.
\end{lem}
\trivial{
  The bad parity part is the same as that of the case when $\star = C$.

  For the good parity part.
  Suppose
  $\ckcO_{g} = (2c_{1}, C_{2}, C_{2},C_{3},C_{3},\cdots, C_{k'},C_{k'},C_{k'+1})$ with
  $2c_{1}=2l$ and $2k'+2 = \bfrr_{1}(\ckcO_{g})$.

  We use the two facts:
  \[
    W_{\lamckg} = W_{c_{1}}\times S_{C_{2}}\times \cdots
    \times S_{C_{k'}}.
  \]

  The symbol of sign representation of $W'_{c_{1}}$ is
  \[
    \binom{0,1, \cdots, c_{1}-1}{1,2, \cdots, c_{1}\phantom{-1}}.
  \]
  The bifurcation happens for even length columns.
  % Eg: 0,1,2,
  %     1,2,3,

}


Let \[
  \Lambda_{n_{1}, n_{2}} = (\underbrace{\half,\cdots,\half}_{n_{1}\text{-terms}}, \underbrace{0,\cdots,0}_{n_{2}\text{-terms}}, )
\]

We set $(2n_{g},2n_{b}) = (\abs{\ckcO_g},\abs{\ckcO_b})$.

\begin{lem}
  We have the following formula on the coherent continuation
  representations based on $\Lambda_{n_{1},n_{2}}$:
  \[
    \bigoplus_{p+q=2n} \Coh_{\Lambda_{n_1, n_2}}(\SO(p,q)) \cong \cC_{b}\otimes \cC_{g}.
  \]
  with
  \[
    \begin{split}
      \cC_g %& = \bigoplus_{p+q=m} \Cint{\rho}(\Sp(p,q)) \\
      &=\bigoplus_{\substack{t,c,d,s,r\in\bN\\2t+c+d+s+r=n_{g}}} \Ind_{H_{t} \times \times W_s\times W_r\times W_{c}\times W_{d} }^{W_{n_{g}}}
      \hsgn \otimes \sgn \otimes \sgn \otimes 1\otimes 1\\
      % & =\bigoplus_{\substack{t,s,r\in\bN\\2t+s+r=n_{g}}} \Ind_{W_{2t}\times
      % W_s\times W_r}^{W_{n_{g}}}
      % (\sigma,\sigma)\otimes \sgn \otimes \sgn \\
      \cC_b & =
      \bigoplus_{\substack{t,a\in \bN\\2t+a=n_{1}}}
      \Ind_{H_{t}\times S_{a}}^{W_{n_{1}}} \hsgn\otimes 1
    \end{split}
  \]
\end{lem}


Clearly $[\tau_{b}:\cC_{b}]$ is counted by the size of the following set
\[
  \PBP_{\star}(\ckcO_b) := \Set{(\imath,\jmath,\cP,\cQ)| %
    \begin{array}{l}
      (\imath,\jmath) = \tau_{b}\\
      \Im \cP \subset \set{\bullet,c,d}, \Im \cQ \subset\set{\bullet}\\
    \end{array}
  }
\]
and for each $\sP\subset \CPP(\ckcO_{g})$, the multiplicity
$[\tau_{\sP}: \cC_{g}]$ is counted by the size of
\[
  \PBP_{\star}(\ckcO_g,\sP):= \Set{(\imath,\jmath,\cP,\cQ)| %
    \begin{array}{l}
      (\imath,\jmath) = \tau_{\sP}\\
      \Im \cP \subset \set{\bullet,s,r,c,d}, \Im \cQ \subset\set{\bullet}\\
    \end{array}
  }
\]
and for each $\sP$.


\begin{lem}
  For each $\uptau' \in \PP_{A^{\bR}}(\ckcO'_{b})$, there is a unique painted
  bipartition $\uptau\in \PBP_{\star}(\ckcO_{b})$ such that
  \[
    \cP_{\uptau'}(i,j) = d \quad % \text{if and only if}
    \Leftrightarrow
    \quad
    \cP_{\uptau}(i,j) = d \quad \forall (i,j)\in \BOX{\imath}\\
  \]
  The map gives a bijection
  \[
    \begin{array}{ccccccc}
      \Unip_{\ckcO'_{b}}(\GL_{b}(\bR))&\longleftarrow
      &\PP_{A^{\bR}}(\ckcO'_{b}) & \longleftrightarrow
      & \PBP_{\star}(\ckcO_{b}) & \longrightarrow
      & \Unip_{\ckcO_{b}}(\SO(2b+1,2b))\\
      \pi_{\uptau'} & \mapsfrom & \uptau' & \mapsto
      & \uptau & \mapsto & \pi_{\uptau}.
    \end{array}
  \]
  Here $\pi_{\uptau'}$ is defined in \eqref{eq:u.GLR} and
  \[
    \pi_{\uptau} := \Ind_{\GL_{b}(\bR)}^{\SO(2b,2b)} \pi_{\uptau'}.
  \]
\end{lem}
\begin{proof}
  The claims about painted bipartitions are clear.
  The irreduciblity of $\pi_{\uptau}$ follows from the multiplicity one
  of its wavefront cycle.

  The representations $\pi_{\uptau_{1}}\neq \pi_{\uptau_{2}}$ if
  $\uptau_{1}\neq \uptau_{2}$ since they have different cuspidal data by the
  construction.
\end{proof}


\appendix
\section{Combinatorics of Weyl group representations in the classical types}

\subsection{The $j$-induction}
If $\mu$ and $\nu$ are two partitions representing two symmetric groups
representations.
Then
\[
  j_{S_{\abs{\mu}}\times S_{\abs{\nu}}}^{S_{\abs{\mu}+\abs{\nu}}}
  \mu\boxtimes \nu
  = \mu\cup \nu,
\]
where $\mu\cup \nu$ is the partition such that
\[
  \set{\bfcc_{i}(\mu\cup\nu) | i\in \bN^{+}} =
  \set{\bfcc_{i}(\mu) | i\in \bN^{+}}
  \cup
  \set{\bfcc_{i}(\nu) | i\in \bN^{+}}
\]
as multisets.

\trivial{
Use the inductive by stage of $j$-induction
\[
  \begin{split}
  &j_{S_{\abs{\mu}}\times S_{\abs{\nu}}}^{S_{\abs{\mu}+\abs{\nu}}}
  \mu\boxtimes \nu\\
  &=
j_{S_{\abs{\mu}}\times S_{\abs{\nu}}}^{S_{\abs{\mu}+\abs{\nu}}}
j^{S_{\abs{\mu}}\times S_{\abs{\nu}}}_{\prod_{i}S_{\bfcc_{i}(\mu)}\times
  \prod_{j} S_{\bfcc_{j}(\nu)}}\sgn\\
 &=
j^{S_{\abs{\mu}+\abs{\nu}}}_{\prod_{i}S_{\bfcc_{i}(\mu\cup \nu)}}\sgn\\
  &= \mu\cup \nu.
  \end{split}
\]
}



We have
\[
  j_{S_{n}}^{W_{n}} \sgn = \begin{cases}
    (\cboxs{k},\cboxs{k}) & \text{if $n=2k$ is even,}\\
    (\cboxs{k+1},\cboxs{k}) & \text{if $n=2k+1$ is odd.}\\
    \end{cases}
\]
\trivial{
  The symbol of trivial of trivial group is
  \[
    \symb{0, 1, \cdots, k}{0,\cdots, k-1} .
  \]
  Apply the formula \cite{Lu}*{4.5.4},
  When $n$ is even, the symbol of the induce is
  \[
    \symb{0,2, \cdots, k+1}{1,\cdots, k}.
  \]
  corresponds to $(\cboxs{k}, \cboxs{k})$.

  When $n$ is even, the symbol of the induce is
  \[
    \symb{1,2, \cdots, k+1}{1,\cdots, k}.
  \]
  corresponds to $(\cboxs{k+1}, \cboxs{k})$.
}

If $\tau = (\tau_{L}, \tau_{R})$ and $\sigma = (\sigma_{L}, \sigma_{R})$
be two bipartition. Then
\[
  j_{W_{\abs{\tau}}\times W_{\abs{\sigma}}}^{W_{\abs{\tau}+\abs{\sigma}}}
  \tau \boxtimes \sigma = (\tau_{L}\cup \sigma_{L}, \tau_{R}\cup \sigma_{R})
\]


\delete{
\subsection{Parameterize of Unipotent representations}
We fix an abstract complex Cartan subgroup $\bfH_a$ and $\fhh_a$ in $\bfG$ and a
set of simple roots $\Pi_a$.  Let $\cP(\bfG)$ be the set of all Langlands
parameters of $G$-modules with character $\rho$ (i.e. the infinitesimal
character of the trivial representation). For $\gamma\in \cP(\bfG)$, let
$\cL(\gamma)$, $\cS(\gamma)$ and $\Phi_\gamma$ be the corresponding Langlands
quotient, standard module and coherent family such that
$\Phi_\gamma(\rho) = \cL(\gamma)$. Let $\cM(\bfG)$ be the span of $\cL(\gamma)$.
Let $\set{\bB}$ be the set of all blocks. Then $\cP(\bfG) = \bigsqcup_\cB \cB$.
The Weyl group $W = W(G)$ acts on $\cM(\bfG)$ by coherent continuation.  Let
$\cM_{\cB}$ be the submodule of $\cM(\bfG)$ spanned by $\gamma\in\cB$, then
\[
  \cM(\bfG) = \bigoplus_\cB \cM_{\cB}
\]
Let $\tau(\gamma)\subset \Pi_a$ be the $\tau$-invariant of $\gamma$.

Let $\ckcO$ be even orbit. $\lambda= \half \ckhh$.  Define
\[
  S(\lambda) = \set{\alpha\in \Pi_a| \inn{\alpha}{\lambda}=0}.
\]
Let $\cP_{\lambda}(\bfG)$ be the set of all Langlands parameters with
infinitesimal character $\lambda$. Let $T_{\lambda,\rho}$ be the translation
functor.  Let
\[
  \cB(S) = \set{\gamma\in \cB|S\cap \tau(\gamma)=\emptyset}
\]
and
\[
  \cP(\bfG,S) = \bigsqcup_{\cB} \cB(S)
\]


Then
\[
  \begin{tikzcd}[row sep=0em]
    \cP(\bfG,S) \ar[r] & \cP_{\lambda}(\bfG)\\
    \gamma \ar[r, maps to]& T_{\lambda, \rho}(\gamma)
  \end{tikzcd}
\]

Let $\cO$ be a complex nilpotent orbit in $\fgg$.  Let
\[
  \cB(S,\cO) = \set{\gamma\in \cB(S)|\AVC(\cL(\gamma))\subset \bcO}
\]

Let
\[
  \begin{aligned}
    m_S(\sigma) &= [\sigma: \Ind_{W(S)}^{W}\bfone]\\
    m_\cB(\sigma)& = [\sigma: \cM_\cB]
  \end{aligned}
\]


Barbasch \cite{B10}*{Theorem~9.1} established the following theorem.
\begin{thm}
  \[
    |\cB(S,\cO)| = \sum_{\sigma} m_\cB(\sigma)m_S(\sigma)
  \]
  Here $\sigma\times \sigma$ running over the $W\times W$ appears in the double
  cell $\cC(\cO)$.
\end{thm}
\begin{proof}
  We need to take the graded module of $\cM(\bfG)$ with respect to the
  $\LRleq$. By abuse of notation, we identify the basis $\cP(\bfG)$ with its
  image in the graded module.  Note that $S\cap \tau(\lambda)=\emptyset$ if and
  only if $W(S)$ acts on $\gamma$ trivially by \cite[Lemma~14.7]{V4}.  On the
  other hand, by \cite[Theorem~14.10, and page 58]{V4},
  $\AVC(\cL(\gamma))\subset \bcO$ only if $\gamma$ generate a $W$-module in the
  double cell of $\cO$.
\end{proof}

Now assume $S=S(\lambda)$. By \cite[Cor~5.30 b) and c)]{BVUni},
$[\sigma: \Ind_{W(S)}^{W}\bfone]=[\bfone|_{W(S)}:\sigma]\leq 1$.

}



\section{Remarks on the Counting theorem of unipotent representations}



\subsection{Coherent family}
For each finite dimensional $\fgg$-module or $\Gc$-module $F$, let
$F^*$ be its contragredient representation and let
$\WT{F}\subseteq \aX$ denote the multi-set of weights in $F$.

Let $\PiGlfin$ be the set of irreducible finite dimensional representations of $\Gc$
with extreme weight in $\Lambda_0$ and $\Glfin$ be the subgroup generated by $\PiGlfin$.
Let
\[
\aP  := \set{\mu \in \aX| \text{$\mu$ is a $\hha$-weight of an $F\in \PiGfin$}}.
\]
Via the highest weight theory,
every $W$-orbit $W\cdot \mu$ in $\aP$ corresponds with the irreducible finite dimensional representation
$F\in \PiGfin$ with extremal weight $\mu$.

Now the Grothendieck group $\Gfin$ of finite dimensional representation of $\Gc$
is identified with $\bZ[\aP/W]$. In fact $\Gfin$ is a $\bZ$-algebra under the
tensor product and equipped with the involution $F\mapsto F^*$.

Fix a $W$-invariant sub-lattice $\Lambda_0\subset \aX$ containing $\aQ$.

%  Let $\Pi$
% $\Glfin$ be the $\star$-invariant subalgebra of $\Gfin$ generated by irreducible
% representations corresponds to $\Lambda_0/W$.


For any $\lambda\in \hha^{*}$, we define
\begin{equation}
  \label{eq:wlam}
  \begin{split}
  [\lambda ]  &:= \lambda  +  \aQ,\\
  R_{[\lambda]} &:= \Set{\alpha\in \aR| \inn{\lambda}{\ckalpha}\in \bZ},\\
  W_{[\lambda]} &:= \braket{s_\alpha|\alpha\in \Rlam} \subseteq W,\\
  R_{\lambda} &:= \Set{\alpha\in \aR| \inn{\lambda}{\ckalpha}=0}, \AND\\
  W_{\lambda} &:= \braket{s_\alpha|\alpha\in R_{\lambda}} = \braket{w\in W|w\cdot \lambda = \lambda} \subseteq W.
  \end{split}
\end{equation}

For any  lattice  $\Lambda = \lambda + \Lambda_0 \in \fhh^*/\Lambda_{0}$ with $\lambda \in \bCon$,
we define
\[
  W_{\Lambda} := \set{w\in W | w\cdot \Lambda  = \Lambda}.
\]
Clearly, we have
\[
  W_{[\lambda]} < W_{\Lambda}, \quad \forall \ \lambda \in \Lambda.
\]



\begin{defn}
Suppose $\cM$ is an abelian group with $\Glfin$-action
\[
  \Glfin\times \cM \ni(F,m)\mapsto F\otimes m.
\]
In addition,  we fix a subgroup $\cM_{\barmu}$ of $\cM$ for each
 for each $W_{\Lambda}$-orbit $\barmu = W_{\Lambda} \cdot \mu\in \Lambda/W_{\Lambda}$.

A function $f\colon \Lambda \rightarrow \cM$ is called
  a coherent family based on $\Lambda$ if it satisfies
  $f(\mu)\in \cM_\mu$ and
  \[
  F\otimes f(\mu)  = \sum_{\nu \in \WT{F}} f(\mu+\nu) \qquad \forall \mu\in \Lambda, F\in \PiGlfin.
  \]
Let $\Cohlm$ be the abelian group of all coherent families based on $\Lambda$ and value in $\cM$.
\end{defn}

\def\Grt{\cG}

In this paper, we will consider the following cases.

\begin{eg}
Suppose $\cM = \bQ$ and $F\otimes m = \dim(F)\cdot m$ for $F\in \PiGlfin$ and $m\in \cM$.
We let $\cM_{\barmu} = \cM$ for every $\mu\in \Lambda$.
When $\Lambda = \Lambda_0$, the set of $W$-harmonic polynomials on $\hha^*$ is naturally
identified with $\Cohlm$ via restriction (Vogan's result)

\end{eg}

\begin{eg}
Let $\Grt(\fgg,K)$ be the Grothendieck group of finite length $(\fgg,K)$-modules
and $\Grt_{\chi}(\fgg,K)$ be the subgroup of $\Grt(\fgg,K)$ generated by the
set of irreducible $(\fgg,K)$-modules with infinitesimal character $\chi$.

Then $\Coh_\Lambda(\cG(\fgg,K))$ is the group of coherent families of Harish-Chandra modules.
The space $\Coh_\Lambda(\cG(\fgg,K))$ is equipped with a $\WLam$-action by
  \[
    w\cdot f(\mu) =  f(w^{-1} \mu) \qquad \forall \mu\in \Lambda, w\in \WLam,
    f \in \Coh_{\Lambda}(\Grt(\fgg,K)).
  \]
\end{eg}

\begin{eg}
  Fix a $\Gc$-invariant closed subset $\cZ$ in the nilpotent cone of $\fgg$. Let
  $\Grt_{\cZ}(\fgg,K)$ be the Grothendieck group of $(\fgg,K)$-modules
  whose complex associated varieties are contained in $\cZ$.
  We define
  \[
    \Grt_{\chi,\cZ}(\fgg,K) := \Grt_{\chi}(\fgg,K)\cap \Grt_{\cZ}(\fgg,k).
  \]

  Now $\Coh_{\Lambda}(\Grt_{\cZ}(\fgg,K))$ is also a $\WLam$-submodule of
  $\Coh_\Lambda(\cG(\fgg,K))$.
\end{eg}

\begin{eg}
  Fixing a Borel subalgebra $\fbb = \fhh\oplus \fnn \subset \fgg$, let
  $\cG(\fgg,\fhh,\fnn)$ be the Grothendieck group of the category $\cO$.
  The space $\cG_{\cZ}(\fgg,\fhh,\fnn)$ is defined similarly.
  The
  space $\Coh_\Lambda(\cG(\fgg,\fhh,\fnn))$ and $\Coh_{\Lambda(\cG)}$defined
  similarly.

%Note that the lattice $\Lambda$ is stable under the $\Wlam$ action.
We can define $W_{\Lambda}$ action on $\Coh_\Lambda(\cM)$ by
\[
   w\cdot f(\mu) =  f(w^{-1} \mu) \qquad \forall \mu\in \Lambda, w\in \WLam.
\]
\end{eg}

\begin{eg}
For each infinitesimal character $\chi$ and
a close $G$-invariant set $\cZ\in \cN_{\fgg}$.
Let $\Grt_{\chi,\cZ}(\fgg,K)$ be the Grothendieck group of $(\fgg,K)$-module
with infinitesimal character $\chi$ and complex associated variety contained
$\cZ$.
Similarly, let $\Grt_{\chi,\cZ}()$
\end{eg}


\def\Parm{\mathrm{Parm}}
\def\cof{\Theta}
\subsection*{Translation principal assumption}
Recall that we have fixed a set of simple roots of $W_\Lambda$.

We make the following assumption for $\Coh_\Lambda(\cM)$.
\begin{itemize}
\item There is a basis $\set{\cof_\gamma|\gamma\in \Parm}$ of
$\Coh_\Lambda(\cM)$ where $\Parm$ is a parameter set;
\item For every $\mu\in \Lambda$, the evaluation at $\mu$ is surjective;
  \[
    \ev{\mu}\colon \Coh_\Lambda(\cM)\rightarrow \cM_{\barmu} \qquad f\mapsto f(\mu)
  \]
  is surjective;
\item a subset $\tau(\gamma)$ of the simple roots of $W_\Lambda$ is attached
to each $\gamma\in \Parm$ such that $s\cdot \cof_\gamma = - \cof_\gamma$;
\item $\cof_\gamma(\mu) =0$ if and only if $\tau(\gamma)\cap R_\mu \neq \emptyset$.
\item $\set{\cof_\gamma(\mu)| \tau(\gamma)\cap R_\mu = \emptyset}$ form a basis of
$\cM_{\bargamma}$.
\end{itemize}

The translation principle assumption implies
\begin{equation}\label{eq:kevmu}
  \Ker\ev{\mu} = \sspan\set{{\cof}_{\gamma}|\tau(\gamma)\cap R_\mu \neq \emptyset } %= \Ker \ev{\mu}.
\end{equation}


\def\cohm{\Coh_\Lambda(\cM)}
Now we have the following counting lemma.
\begin{lem}
 For each $\mu$, we have
 \[
    \dim \cM_{\barmu}  = \dim (\cohm)_{W_\mu} = [\cohm, 1_{W_\mu}].
 \]
\end{lem}
\begin{proof}
  Clearly
  \[
  \sspan\set{\cof_\gamma - w\cof_\gamma | \gamma\in \Parm, w\in W_\Lambda} \subseteq \ker \ev{\mu}
  \]
  since $w\cdot \cof_\gamma(\mu) = \cof_\gamma(w^{-1}\cdot \mu)=\cof_\gamma(\mu)$ for $w\in W_\mu$.
  Combine this with \eqref{eq:kevmu}, we conclude that
  \[
   \sspan\set{\cof_\gamma - w\cof_\gamma | \gamma\in \Parm, w\in W_\Lambda} = \ker \ev{\mu}.
  \]
  Therefore, $\ev{\mu}$ induces an isomorphism $(\cohm)_{W_\mu}\rightarrow \cM_{\barmu}$.
  Now the dimension equality follows.
\end{proof}

\begin{eg}
  For the case of category $\cO$. We can take $\Parm  = W$.
  Let $\cof_w(\mu) = L(w\mu)$ for each $w\in W_\Lambda$ with
  $\tau(w) = \set{s_\alpha |\alpha\in \Delta^+, w\alpha \not\in R^+ }$.
\end{eg}

\begin{eg}
 In the Harish-Chandra module case,
 $\Parm$ consists of certain irreducible $K$-equivariant local systems on a $K$-orbit of the flag variety of $G$.
 $\tau(\gamma)$ is the $\tau$-invariants of the parameter $\gamma$.
\end{eg}
  % Clearly the evaluation map factor through
  % \[
  % (\cohm)_{W_\mu}  = \cohm / \braket{\cof_\gamma - w\cof_\gamma | \gamma\in \Parm, w\in W_\Lambda}.
  % \]
  %



\subsection{Primitive ideals and left cells}
In this section, we recall some results about the primitive ideals and cells developed by Barbasch Vogan, Joseph and Lusztig etc.

\subsection{Highest weight module}
Let $\fgg$ be a reductive Lie algebra, $\fbb = \fhh \oplus \fnn$ is a fixed Borel.
%Fix $\lambda \in \fhh^*$ dominant i.e. $\inn{\lambda}{\ckalpha} \geq 0 $
Let
\[
  M(\lambda)  := \cU(\fgg)\otimes_{\cU(\fbb)} \bC_{\lambda-\rho}
\]
and $L(\lambda) $ be the unique irreducible quotient of $M(\lambda)$.

In the rest of this section, we fix a lattice $\Lambda  \in   \fhh^*/ X^*$.
Let $\RLam$ and $\RLamp$ be the set of integral root system and the set of positive integral roots.
Write $\WLam$ for the integral Weyl group.


For $\mu\in \Lambda$, % is called if
\[
  \begin{array}{ccccc}
  \mu \succeq 0&\Leftrightarrow &\mu \text{ is dominant} &\Leftrightarrow&  \inn{\lambda}{\ckalpha}\geq 0 \quad \forall \alpha \in \RLamp \\
  \mu \not\sim 0&\Leftrightarrow& \mu \text{ is regular} &\Leftrightarrow&   \inn{\lambda}{\ckalpha}\neq 0 \quad \forall \alpha \in \RLamp
  \end{array}
\] %for all $\alpha \in R^+_{\Lambda}$and is called
regular if $\inn{\lambda}{\ckalpha}\neq 0$.

For each $w\in W$, it give a coherent family such that
\[
M_w(\mu) = M(w\mu) \quad \forall \mu \in \Lambda %\text{ dominant}
\]
Let $L_w$ be the unique coherent family such that $L_w(\mu) = L(w\mu)$ for any
regular dominant $\mu$ in $\Lambda$.
\trivial{Note that the Grothendieck group
  $\cK(\cO)$ of category $\cO$ is naturally embedded in the space of formal
  character. $M_w$ is a coherent family: the formal character $\ch M_w (\mu)$ of
  $M_w(\mu)$ is
  \[
    \ch M_{w}(\mu)=\frac{e^{w\mu-\rho}}{\prod_{\alpha\in R^+} (1-e^{-\alpha})}
    =\frac{e^{w\mu}}{\prod_{\alpha\in R^+} (e^{\alpha/2}-e^{-\alpha/2})}
  \]
  It is clear
  that $\ch M_w$ satisfies the condition for coherent continuation.

  From now on, we fix a regular dominant weight $\lambda \in \fhh^{*}$.
  Then $w[\lambda] = [w\cdot \lambda]$ for any $w\in W$.

 Now $W_{w[\lambda]} = w\, W_{[\lambda]}\, w^{-1}$.

 As $W_{[\lambda]}$-module, we have the following decomposition
 \[
   \begin{split}
     \Coh_{[\lambda]} &= \bigoplus_{r\in W/W_{[\lambda]}}
     \Coh_{r}\quad \text{with}\\
     \Coh_{r} &=\Coh_{r\lambda} := \set{\cof\in \Coh_{[\lambda]}| \cof(\lambda)\in \cO'_{[r\cdot \lambda]}}
   \end{split}
 \]

 Here $\cO'_{S}$ is the set of highest weight module whose $\fhh$-weights are in
 $S\subset \fhh^{*}$.

 % Here
 % \[
 %   \begin{split}
 %     \Coh_{r}
 %    % &:= \sspan_{\bZ}\set{M_{w}| w \in r\, W_{[\lambda]}}\\
 %     &= \set{\cof\in \Coh_{[\lambda]}| \cof(\lambda)\in \cO'_{[r\cdot \lambda]}}
 %   \end{split}
 % \]

 Note that the following map
 \[
   \begin{array}{ccccc}
     \bC[W] & \longrightarrow & \Coh_{[\lambda]} &\longrightarrow & \bC[S]\\
     w & \mapsto & (\mu\mapsto M_{w}(\mu)) & \mapsto & w\cdot \lambda
   \end{array}
 \]
 is $W_{[\lambda]}$-equivariant
 where $W_{[\lambda]}$ acts by right translation on $\bC[W]$ and $S = W\cdot \lambda$.
 The action of $W_{[\lambda]}$ on $S$ is by transport of structure and so
 $(a, w\,\lambda)\mapsto wa^{-1}\,  \lambda$.

 Now $\bC[r W_{[\lambda]}] \subset \bC[W]$ and
 $\bC[r W_{[\lambda]}\cdot \lambda]$
 are identified with $\Coh_{r}$.


}



% Fix $\lambda \in \fhh^{*}$ and regular dominant.

% As $W_{\lambda}$ module, $\Coh_{[\lambda]}$ can be


For each $w\in \WLam$, the function
\[
  \wtpp_w(\mu) := \rank (\cU(\fgg)/\Ann (L(w \mu))) \qquad \forall \mu \in \Lambda \text{ dominant}.
\]
extends to a Harmonic polynomial on $\fhh^*$.
In particular, $\wtpp_w\in P(\fhh^*) = S(\fhh)$.
\trivial[]{
  Let $\Lambda^+$ be the set of dominant weights in $\Lambda$.
Assume $\lambda \succ 0$, i.e. dominant regular. Then
$\R^0_\lambda = \empty$, $w^{w\lambda} = w^{-1}$. The set
\[
 \hat F_{w\lambda} = \set{\mu \in \Lambda |w^{-1}\mu \succeq 0 }
=  w \Lambda^+.
\]
Joseph's $p_w(\mu) = \rank \cU(\fgg)/(\Ann L(w\mu))$ is defined on $\hat F_{w\lambda} = w\Lambda^+$.
Now his $\wtpp_w(\mu) = w^{-1} p_w$ is defined on $\Lambda^+$ and given by
$\wtpp_w(\mu) = \rank \cU(\fgg)/(\Ann L(w\mu))$.
}
\def\PIP#1{\cP^{#1}}

There is a unique function $\aLam \colon \WLam \times \WLam\rightarrow \bQ$ such that
\[
L_w  = \sum_{w'\in \WLam} \aLam (w,w') M_{w'}
\]
\trivial[h]{
In fact, we have, for any $\mu \succ 0$,
\[
L(w\mu) = \sum_{} \aLam(w,w') M(w'\mu)  %\qquad \forall \mu \succ 0.
\]
The coefficient is independent of $\mu$. } Suppose $V$ is a module in $\cO$ whose
Gelfand-Kirillov dimension $\leq d$ Let $c(V)$ be the Bernstein degree. Let
$\Cint{\Lambda}^d$ be the submodule of $\Cint{\Lambda}$ generated by $L_w$ such
that the $\dim L(w\mu)\leq d$. Then the map
\[
\PIP{d}: \CLam^d\rightarrow S(\fhh) \qquad \Theta \mapsto (\mu \mapsto c(\Theta(\mu)) )
\]
is $\WLam$-equivariant. Let $c_w := c\circ (L_w(\mu))\in S(\fhh)$.

The following result of Joseph implies that the set of primitive ideals can be
parameterized by Goldie rank polynomials.
\begin{thm}[Joseph] %, Barbasch-Vogan]
  \label{thm:GR}
  {\color{red} correct formulation?}
  For $\mu$ dominant, the primitive ideal $\Ann L(w\mu) = \Ann L(w'\mu)$ if and only if $p_w = p_w'$. (\cite[ref?]{J1})
%   Let $r = \dim \fnn$.
%   The following holds:
%  \begin{enumT}
%   \item \label{it:j.1}
%   For $\mu$ dominant, the primitive ideal $\Ann L(w\mu) = \Ann L(w'\mu)$ if and only if $p_w = p_w'$. (\cite[ref?]{J12})
%  \end{enumT}
\end{thm}

\begin{thm}
 Let $r = \dim \fnn$.  Suppose $\dim L(w\nu) = d$.
 \begin{enumT}
  \item Let $x\in \fhh$ such that $\alpha(x) = 1$ for every simple roots $\alpha$.
  Then there are non-zero rational numbers $a,a'$ such that  \cite{J12}*{Theorem~5.1 and 5.7}
  \[
   a \, \wtpp_w(\mu)  = a' \, c_w(\mu) = \sum_{w'\in \WLam} \aLam (w,w') \inn{\mu}{w'^{-1} x}^{r-d}.
  \]
  \item The submodule $\bC \WLam \wtpp_w$ in $S(\fhh)$ is
  a special  irreducible $\WLam$-representation.
  Moreover all special representations of $\WLam$ occurs as a certain
  $\sigma(w)$.
  See \cite{BV1}*{Theorem~D}, \cite{BV2}*{Theorem~1.1}.
  \item Let $\sigma(w)$ be the submodule of $S(\fhh)$ generated by
  $\wtpp_w$. Then
  $\sigma(w)$ is irreducible and it
  occurs in $S^{r-d}(\fhh)$ with multiplicity $1$, see \cite{J.hw}*{5.3},
  % occurs in
  % $\wtpp_{w^{-1}}$ also generate $\sigma(w)$ and the module
 \end{enumT}
\end{thm}
By \Cref{thm:GR}, the module $\sigma(w)$ only depends on the primitive ideal $\cJ :=\Ann L(w\mu)$. So we can also write
$\sigma(\cJ):= \sigma(w)$.

\trivial[h]{
A representation $\sigma \in \widehat{W}$ is called univalent, if $\sigma$ occur in
$S^{n(\sigma)}(\fhh)$ with multiplicity one. Here $n(\sigma)$ is the minimal degree such that $\sigma$ occur in $S(\fhh)$.
}

From the above theorem, we see that the map $\PIP{d}$ factor through
$\CLam^{d}/\CLam^{d-1}$ and its image is contained in the space $\cH^{d}$ of degree $d$-Harmonic
polynomials.

Note that the associated variety of a primitive ideal $\cJ$ is the closure of
a nilpotent orbit in $\fgg$.
%$\Ann L_w (\mu)$ for $\mu\succ 0$
Now the associated variety of $\cJ$  is computed by the following result of Joseph (and include Hotta ?).
\begin{thm}
  %Suppose $J = \Ann L(w\mu)$
  Let $\cJ$ be a primitive ideal in $\cU(\fgg)$.
  Then $\sigma(\cJ)$ is in the image of Springer correspondence.
  Moreover, the associated variety of $\cJ$ is the orbit $\cO$
  corresponding to $\sigma(\cJ)$.
See \cite{J.av}*{Proposition~2.10}.
\end{thm}


\begin{lem}
  Let $\mu\in \Lambda$, $W_{\mu} = \set{w\in W_{\Lambda}| w\cdot \mu = \mu}$ and
  \[
  a_{\mu} = \max\set{a(\sigma)|  \widehat{\WLam} \ni\sigma \text{ occurs in } \Ind_{W_{\mu}}^{\WLam} 1}.
  \]
  Suppose $W_{\mu}$ is a parabolic subgroup of $\WLam$.
  % Then there is a unique special representation $\sigma_{\mu}$ such that
  % $a(\sigma_{\mu}) = a_{\mu}$.
  Then the set of all irreducible representations $\sigma$ occurs in
  $\Ind_{W_{\mu}}^{\WLam} 1$ such that $a(\sigma) = a_{\mu}$ forms a left cell
  given by
  \[
  \left(J_{W_{\mu}}^{\WLam} \sgn\right) \otimes \sgn.
  \]
\end{lem}


Now the maximal primitive ideal having infinitesimal character $\mu$
has the associated variety $\cO$.

\def\Wb{W_{b}}
\def\Wg{W_{g}}
\def\WcOb{W_{\ckcO_b}}
\def\WcOg{W_{\ckcO_g}}

Recall that $\WLam = \Wb\times \Wg$.
Let
\[
\begin{split}
  \sigma_{b} &= (j_{\WcOb}^{\Wb}\sgn)\otimes \sgn\\
  \sigma_{g} &= (j_{\WcOg}^{\Wg}\sgn)\otimes \sgn\\
\end{split}
\]
Then $\sigma_{b}\otimes \sigma_{g}$ is a special representation of $\WLam$.
Moreover $\cO$ corresponds to the representation
\[
 \sigma := j_{\Wb\times \Wg}^{W} \sigma_{b}\otimes \sigma_{g}
\]

% Then $W_{[\lambda_{\ckcO}]} = W'_n$ and the left cell is given by
% \[
% (J_{W_\ckcO}^{W'_n} \sgn) \otimes \sgn \quad \text{ with } \quad W_\ckcO = \prod_{i\in \bN^+} S_{\bfcc_{2i}(\ckcO)}.
% \]
% Here $W_\ckcO$ is an subgroup of $S_n$ and the embedding of $S_n$ in $W'_n$ is fixed.
% When $n$ is even,we the symbol of $J_{S_n}^{W'_n} \sgn$ is degenerate and we label it by ``$I$''.

\trivial{
  We recall some facts about the $j$-induction and $J$-induction.
  If $W'$ is a parabolic subgroup in $W$ then $j_{W'}^{W}$ maps special
  representation to special representation.
  It maps irreducible representation to irreducible ones.
  The $j$ induction has induction by stage \cite{Carter}*{11.2.4}.

  For classical group, the representations satisfies property B.
  When $W'$ is parabolic, the orbit $\cO$ corresponds to $j_{W'}^{W}$ is the orbit
  $\Ind_{L}^{G} \cO$.

  The $J$-induction maps left cell to left cell, which is only defined for
  parabolic subgroups.


  The $j$-induction for classical group is in \cite{Carter}*{Section 11.4}.
  Basically, $W_{n}$ has $D_{c_{i}}$, $B/C_{c_{i}}$ factors,
  \[
    j_{\prod_{i}D_{c_{i}}\times \prod_{j} B/C_{c_{j}}} \sgn  =
      ((c_{i}), (c_{j})).
  \]
  Here $(c_{i})$ denote the Young diagram with columns $c_{i}$ etc.

  Moreover, we have the following formula for $j$-induction.
  \[
    \ind_{A_{2m}}^{W_{2m}} = ((m),(m)) \qquad
    \ind_{A_{2m+1}}^{W_{2m+1}} = ((m),(m+1)).
  \]
}


\subsection{Harish-Chandra cells and primitive ideals}


For simplicity we assume $G$ is connected in this section.

We recall McGovern and Casian's works on the Harish-Chandra cells.

Fix a $\theta$ stable Cartan subgroup $H=TA$ of $G$ such that $T$ is the maximal
compact subgroup of $H$ and $A$ is the split part of $H$.
% Let $\cO'_{\whH}$ be the category of $(\fgg, T)$-module such that the $\fnn$-action
% is locally nilpotent.
%
Let $\whH$ denote the set of irreducible representations of $\whH$ (also viewed as an
$(\fhh,T)$-module).
Let $\CPH$ be component group of $H$, which is also the component group of $T$.
Let $\rdd \colon \whH\longrightarrow \fhh^{*}$ be the map takes $\phi\in \whH$
to its derivative.
Then $\whH$ is a $\whCPH$-torsor over $\fhh^{*}$.
\trivial{
  This means for each $\lambda\in \fhh^{*}$, its fiber
  $\rdd^{-1}(\lambda)$ is either empty or with a set with free transitive
  $\whCPH$-action (the tensor product action).
}

\trivial{
  The classification of Primitive ideals.
  \[
    \begin{array}{cccc}
      \Wlam &\longrightarrow &  \Prim_{\lambda}&\\
      w & \mapsto & I(w\lambda)& := \Ann L(w\lambda).
    \end{array}
  \]

  We now recall some results about the blocks in category $\cO$.

  Assume $\lambda\in \sfC$ where
  $\sfC = \set{\nu\in \fhh^{*}|\inn{\nu}{\ckalpha}>0}$ is the positive cone.
  Let $\lambda' = w_{0}\lambda$ which is negative.

  Let $w_{0}$ (resp. $w_{[\lambda]}$)be the longest element in $W$
  (resp. $\Wlam$).

  Define
  (We adapt the convention that $e\leqL w_{[\lambda]}$.)
  \[
    \begin{split}
      w_{1} \leqL w_{2}
      & \Leftrightarrow
      I(w_{1}\lambda')\subseteq I(w_{2}\lambda')\\
      & \Leftrightarrow
      I(w_{1}w_{0}\lambda)\subseteq I(w_{2}w_{0}\lambda)\\
      & \Leftrightarrow
      [L(w_{2}^{-1}\lambda'), L(w_{1}^{-1}\lambda')\otimes S(\fgg)] \neq 0\\
      % \quad \text{for some f.d. repn $F$}.\\
      % & \Leftrightarrow
      % [L(w_{2}^{-1}\lambda'), L(w_{1}^{-1}\lambda')\otimes F] \neq 0
      % \quad \text{for some f.d. repn $F$}.
    \end{split}
  \]

  Note that
  \[
    w_{1}\leqL w_{2} \Leftrightarrow w_{2}^{-1}\leqR w_{1}^{-1}.
  \]
  Therefore, the last condition implies that
  \[
    \begin{split}
      \sV^{R}(w) &:=
      \sspan\set{L(w'\lambda) | w'\leqR w}\\
     &  =\sspan\set{L(w' \lambda) | w^{-1}\leqL w'^{-1}}
    \end{split}
  \]
}




We also fix a positive system $\Delta^{+}(\fgg,\fhh)$ and let $\fnn$ be the
span of positive roots.
Let
\[
Q:= \bZ[\Delta(\fgg,\fhh)]\subset \fhh^{*}
\]
be the root lattice of $\fgg$, which is exactly the set of $\fhh$-weights occur in $S(\fgg)$.


For each $\lambda\in \fhh^{*}$, let
$\braket{\lambda} := W_{[\lambda]}\cdot \lambda$.
A set $S\subset \fhh^{*}$ is called a \emph{type} if
\[
  S = \bigcup_{\lambda\in S} \braket{\lambda}.
\]
Clearly, $\braket{\lambda}$ is the smallest type containing $\lambda$.


Suppose $S$ is a type,
we define the category $\cO'_{S}$ to be the category of $\fgg$-modules
such that $M\in \cO'_{S}$ if and only if
\begin{itemize}
  \item the $\fbb$-action on $M$ is locally finite;
  \item $M$ is finitely generated $\cU(\fgg)$-module,
  \item $M = \sum_{\mu \in S} M_{\mu}$
        where $M_{\mu}$ is the $\mu$-isotypic component of $M$.
\end{itemize}
Clearly, $\cO'_{S}$ is a union of blocks in $\cO'$.


For each set $\tS\subset \whH$, let
$[\tS] := \set{\phi\otimes \alpha| \phi \in \tS, \alpha\in Q}$ be the
$Q$-translations of $\tS$.
We say $\tS$ is $Q$-saturated if $\tS = \rdd^{{-1}}(\rdd(\tS))\cap [\tS]$.
We say $\tS$ is a type if $\rdd(S)$ is a type.
Suppose $\tS$ is $Q$-saturated and

For $\phi\in \whH$, we write
\[
\braket{\phi}:= [\phi]\cap \phi^{-1}(\braket{\rdd \phi})
\]
which is the smallest $W$-invariant $Q$-saturated subset in $\whH$ containing
$\phi$.
Let $M(\phi) := \rU(\fgg)\otimes_{\fbb,T} \phi$ where $\phi$ is inflated to a
$(\fbb,T)$-module.

Suppose $\tS$ is $Q$-saturated type,
we define the category $\cO'_{S}$ to be the category of $(\fgg,T)$-module
such that $M\in \cO'_{S}$ if and only if
\begin{itemize}
  \item the $\fbb$-action is locally finite;
  \item $M$ is finitely generated $\cU(\fgg)$-module,
  \item $M = \sum_{\mu \in \tS} M_{\mu}$
        where $M_{\mu}$ is the $\mu$-isotypic component of $M$.
\end{itemize}
In particular, $\cO'_{\whH}$ is the category of finitely generated $(\fgg,T)$-module
such that $\fbb$-acts locally finite.

Let $\cF\colon \cO'_{\whH}\rightarrow \cO'_{\fhh}$ be the forgetful functor of
the $T$-module structure.
The following lemma is clear.
\begin{lem}\label{lem:eqcat}
  The forgetful functor $\cF$ yields an equivalence of
  category
  \[
    \cO'_{\braket{\phi}}\rightarrow \cO'_{\braket{\rdd\phi}}
  \]
  which induces an isomorphism of $W_{[\dphi]}$-module
  \[
    \Coh_{\phi}\rightarrow \Coh_{\rdd\phi}.
  \]
\end{lem}
Suppose $\gamma\in \whCPH$.
By \Cref{lem:eqcat}, we have an equivalent of category
\[
  \cO'_{\braket{\phi}}\cong \cO'_{\braket{\rdd\phi}}\cong \cO'_{\braket{\phi\otimes \gamma}}
\]
which sends $M(\phi)$ to $M(\phi\otimes \gamma)$.
This isomorphism induces a $W_{[\dphi]}$-module isomorphism
\[
    \Coh_{\phi}\xrightarrow{\otimes \gamma} \Coh_{\phi\otimes \gamma}.
\]

\def\tCoh{\widetilde{\Coh}}
Let $\tCoh_{[\lambda]}:= \cF^{-1}(\Coh_{[\lambda]})$.

In summary, we conclude that
%\[
$
\tCoh_{[\lambda]}\xrightarrow{\cF} \Coh_{[\lambda]}
$
%\]
is a $\whCPH$-torsor over $\Coh_{[\lambda]}$.

Fix a set $\set{r_{1}, \cdots, r_{k}}$ of representatives of
$W/W_{[\lambda]}$ and fix
an element $\phi_{i}\in \rdd^{-1}(r_{i}\lambda)$ for each $r_{i}$.
Let $\Phi := \set{\phi_{i}|i=1, 2, \cdots, k}$.

Now we have an isomorphism of $W_{[\lambda]}$-module
\[
  \cF_{\Phi}\colon \bigoplus_{i}\Coh_{\phi_{i}} \longrightarrow %\xrightarrow{\ \ \cF\ \ }
  \bigoplus_{i} \Coh_{r_{i}\lambda} = \Coh_{[\lambda]}.
\]
Now
\[
  \Coh_{[\lambda]}\otimes \bZ[\whCPH]
  \xrightarrow{\ \ \cong \ \ } \tCoh_{[\lambda]}.
\]
given by $\Theta \otimes \alpha \mapsto
\cF_{\Phi}^{-1}(\Theta)\otimes \alpha$ is a
$W_{[\lambda]}\times \whCPH$-module isomorphism.



\medskip
\def\Grt{\cG}
\def\DeltaRp{\Delta^{+}_{\bR}}
Let $K = G^{\theta}$.


Recall Casian's result:
\begin{thm}[\cite{Cas}*{Proposition~2.10, Theorem~3.1}]\label{thm:L1}
  Let $H$ be a $\theta$-stable Cartan subgroup of $G$.
  Fix a positive system of real roots $\DeltaRp$ and a positive system
  $\Delta^{+}$ of roots such that $\DeltaRp\subseteq \Delta^{+}$.
  Let $\bfnn$ be the maximal nilptent Lie subalgebra of $\fgg$ with spanned
  by roots in $\Delta^{+}$.
 Let $M$  be a Harish-Chandra $(\fgg,K)$-module, then
  \begin{enumT}
    \item the Lie algebra cohomology $H^{q}(\fnn,M)$ is finite dimensional;
    \item
    the localization $\gamma_{\fnn}^{q}M$ is in the category $\cO'_{\whH}$;
    (see \cite{Cas}*{Section~1})
    \item
    $\Ann M \subseteq \Ann (\gamma_{\fnn}^{q}M)$.
    \item
  Then the localization functor induces a homomorphism
  \[
    \begin{array}{cccc}
      \gamma_{\fnn}: &\Grt_{\chi,\cZ}(\fgg,K) &\longrightarrow & \Grt_{\chi,\cZ}(\cO_{\whH})\\
      & M &\mapsto & \sum_{q}  (-1)^{q} \gamma^{q}_{\fnn} M
    \end{array}
  \]
  \end{enumT}
  % Fix an infinitesimal character $\chi$, and a close $G$-invariant set
  % $\cZ\in \cN_{\fgg}$.
  \qed
\end{thm}

\begin{thm}[\cite{Cas}*{Theorem~3.1}]
  Let $H_{1}, H_{2}, \cdots, H_{s}$ form a set of representatives of the
  conjugacy class of $\theta$-stable Cartan subgroup of $G$. Fix maximal
  nilpotent Lie subalgebra $\fnn_{i}$ for each $H_{i}$ as in \Cref{thm:L1}.
  Then
  \[
    \begin{array}{cccc}
      \gamma:=\oplus_{i} \gamma_{\fnn_{i}}: &\Grt_{\chi}(\fgg,K)
      &\longrightarrow & \bigoplus_{i} \Grt_{\chi}(\cO_{\whH_{i}})\\
    \end{array}
  \]
  is an embedding of $\WLam$-module.
\end{thm}
\begin{proof}
  We retain the notation in \cite{Cas}.
  Let $H^{rs}_{i}$ be the set of regular semisimple elements in $H_{i}$. By
  Harish-Chandra, taking the character of the elements induces an embedding of
  $\Grt_{\chi}(\fgg,K)$ into the space of analytic functions on
  $\bigsqcup_{i} H^{rs}_{i}$. Now \cite{Cas}*{Theorem~3.1} implies that the
  global character $\Theta M$ of an element $M\in \Grt_{\chi}(\fgg,K)$ is
  completely determined by the formal character $\mathrm{ch}(\gamma(M))$.
\end{proof}

\def\VHC{\sV^{\mathrm{HC}}}
\begin{cor}
  Fix an irreducible $(\fgg,K)$-module $\pi$ with infinitesimal character
  $\chi_{\lambda}$. Let $\VHC(\pi)$ be the Harish-Chandra cell representation
  containing $\pi$ and $\cD$ be the double cell in $\widehat{W_{[\lambda]}}$
  containing the special representation $\sigma(\pi)$attached to $\Ann(\pi)$.
  Then $[\sigma, \VHC(\pi)]\neq 0$ only if $\sigma \in \cD$.
  Moreover, $\sigma(\pi)$ always occures in $\VHC(\pi)$
\end{cor}
\begin{proof}
  The occurrence of $\sigma(\pi)$ is a result of King.

  Note that we have an embedding
  \[
    \gamma \colon \Grt_{\chi}(\fgg,K)\longrightarrow \bigoplus_{i}\Grt(\fgg,\fbb,\lambda).
  \]
  where the left hand sides is identified with a finite copies of
  $\bC[W_{[\lambda]}]$.

  Since $\Ann(\pi)\subseteq \Ann (\gamma_{\fnn}^{q}(\pi))$,
  we conclude that $[\sigma, \VHC(\pi)]\neq 0$ implies that
  $\sigma(\pi)\leqLR \sigma$.

  By the Vogan duality, $\cD\otimes \sgn$ is also a Harish-Chandra cell.
  So we have $\sigma(\pi)\otimes \sgn \leqLR \sigma\otimes \sgn$.

  Therefore, $\sigma(\pi)\approxLR$
\end{proof}


\section{Metaplectic Barbasch-Vogan duality and Weyl group representations}

In this section, we exam primitive ideals for $\fgg=\fsp(2n,\bC)$ at half
integral infinitesimal character and explain the metaplectic Barbasch-Vogan dual
in terms of the cells of $\sfW'_{n}$.

Let $\ckcO$ be a metaplectic ``good'' nilpotent orbit in $\fsp(2n,\bC)$, i.e.
$\bfrr_i(\ckcO)$ is even for each $i\in \bN^+$.

%We write $ \bfcc_{2i} (\ckcO)  =: 2 c_i + \epsilon_i =: C_i$ such that $\epsilon_i \in \set{0,1}$.
Then $W_{[\lambda_{\ckcO}]} = W'_n$ and the left cell is given by
\[
(J_{W_\ckcO}^{W'_n} \sgn) \otimes \sgn \quad \text{ with } \quad W_\ckcO = \prod_{i\in \bN^+} S_{\bfcc_{2i}(\ckcO)}.
\]
Here $W_\ckcO$ is an subgroup of $S_n$ and the embedding of $S_n$ in $W'_n$ is fixed.
When $n$ is even,we the symbol of $J_{S_n}^{W'_n} \sgn$ is degenerate and we label it by ``$I$''.

\def\PPtC{\PP{\tC}}
 Let
 \[
  \PPtC(\ckcO) = \set{(2i-1,2i+2)|i\in \bN^+, \bfrr_{2i-1}(\ckcO)\neq \bfrr_{2i}(\ckcO)}.
 \]
  For each $\wp\subset \PPtC(\ckcO)$, let $\tau_\wp := (\imath_\wp,\jmath_\wp)$ be the bipartition
  given by
  \[
    (\bfrr_i(\imath_\wp),\bfrr_i(\jmath_\wp)) =\begin{cases}
      (\frac{\bfrr_{2i-1}(\ckcO)}{2}, \frac{\bfrr_{2i}(\ckcO)}{2}),   & \text{if } (2i-1,2i)\notin \wp, \\
      (\frac{\bfrr_{2i}(\ckcO)}{2}, \frac{\bfrr_{2i-1}(\ckcO)}{2}), & \text{otherwise.}
    \end{cases}
  \]
  If $\tau_\wp$ is degenerate, it represent the element in $\widehat{W'_n}$
  with label $I$.
  Let $\wp^c$ be the complement of $\wp$ in $\PPtC(\ckcO)$.
  Then $\tau_\wp$ and $\tau_{\wp^c}$ represent the same irreducible $W'_n$-module.

  \def\LCC{{}^L\sC'}
Using the induction formula in \cite{L}*{(4.6.6) (4.6.7)}, we have the following lemma.
\begin{lem}
The representation
\[
  \LCC(\ckcO) :=J_{W_\ckcO}^{W'_n} \sgn
\]
is multiplicity free.
The map
\[
  \bZ[\PPtC(\ckcO)]/\sim \longrightarrow \set{\tau\in \widehat{W'_n}| \tau\subset \LCC(\ckcO)}
  \qquad \wp \mapsto \tau_\wp
\]
is a bijection where $\sim$ is the equivalent relation identifying $\wp$ with $\wp^c$.
% irreducible components in $\LCC(\ckcO)$ are in one-one correspond
% to the quotient of  by the equivalent relation $\wp\sim \wp^c$
Moreover, the special representation in $\LCC(\ckcO)$ is $\tau_\emptyset$.
\qed
\end{lem}

In the following, we write
\[
\tau_{\ckcO}:=\tau_\emptyset
\]
for the unique special
representation in $\LCC(\ckcO)$.

\trivial[h]{
 Use the formula, if $\ckcO$ has only two rows $[2r_1,r_2]$.
 Then $W_{\ckcO} = \underbrace{S_2\times \cdots \times S_2}_{r_2} \times
 \underbrace{ S_1 \times \cdots \times S_1}_{r_1-r_2}$ and the claim is clear.

 The corresponding symbol is
 \[
  \binom{r_1}{r_2}
 \]

 We prove by induction on the number of rows.
 Suppose $\ckcO  = [2k+2r_1, 2r+2r_2, \ckcO'] $ where $2k$ is the number of columns in $\ckcO'$.
 If $r_1=r_2 =0$, then
 the symbol of $\ckcO$ are given by
 \[
  \binom{a_1,a_2, \cdots, a_k, 2k+r_1}{b_1, b_2, \cdots, b_k, 2k+r_2 }
  \text{ or }
  \binom{a_1,a_2, \cdots, a_k, 2k+r_2}{b_1, b_2, \cdots, b_k, 2k+r_1 }.
 \]
where
\[
  \binom{a_1,a_2, \cdots, a_k}{b_1, b_2, \cdots, b_k}
\]
is the symbol attached to $\ckcO'$.

This gives the claim.
}


Let $\tau_\ckcO  = (\imath,\jmath)$ such that $\bfrr_i(\imath)\leq \bfrr_i(\jmath)$ for all $i\in \bN^+$.
Then the unique special representation
in $\Ind_{W_\ckcO}^{W'_n} 1$ where the $a$ function take maximal value is given
by the bipartition
\[
  \tau_{\cO}:=(\jmath^t,\imath^t).
\]
Let $\sigma'(\cO)$ denote the $\tau_{\cO}$-isotypic component of $W'_n$-harmonic polynomials in $S(\fhh)$.
Comparing the  fake degree formulas of type $C$ and $D$ (see \cite{Carter}*{Proposition 11.4.3, 11.4.4}), we conclude that
the $W_n$-representation
\begin{equation}\label{eq:sigma.tC}
\sigma(\cO):= \bC[W_n]\cdot \sigma'(\cO)
\end{equation}
is irreducible whose type is also given by the
bipartition $\tau_{\cO}$. The type of $\sigma(\cO)$ is $j_{W'}^{W} \sigma'(\cO)$.

\begin{lem}\label{lem:MD1}
  Suppose $\sigma'$ is a $W'_n$-special representation.% and $\sigma = j_{W'_n}^{W_n}$.
  Let $\ckcO'$ be the partition of type $D$ corresponds to the $W'_n$-special representation
  $\sigma'\otimes \sgn_D$. Then $\sigma:=j_{W'_n}^{W_n} \sigma'$ corresponds to the partition $\ckcO'^t$ under the Springer correspondence of
  type C.
\end{lem}
\begin{proof}
  %By the algorithm computing the Springer correspondence, it is clear that
  First note that $\ckcO'$ is a specail nilpotent orbit of type $D_n$,
  therefore $\ckcO'^t$ is a partition of type $C_n$ (see \cite{CM}*{Proposition~6.3.7}).

  By the explicitly formula of the Lusztig-Spaltenstein duality, we known that the D-collapsing
  $(\ckcO'^t)_D$ of $\ckcO'^t$ corresponds to the special representation
  $\sigma'$ of type $D$.

  The Springer correspondence algorithm for classical groups can be naturally extends to all partitions.
  Sommers showed that two partitions are mapped to the same Weyl group representations
  if and only if they has the same $D$-collapsing \cite{So}*{Lemma~9}.
  Note that the algorithms computing the Springer correspondence for type C and D are essentially the same
  (see \cite{Carter}*{Section~13.3} or \cite{So}*{Section~7}).
  By the injectivity of the Springer correspondence, we conclude that the type $C$ partition $\ckcO'^t$
  must corresponds to $\sigma$.
  \trivial[h]{
    Two partition $\lambda\sim_X \mu$ if they have the same $X$-collapsing.
  Sommers showed that two partition $\lambda$ and $\mu$ maps to the same module $E_\lambda = E_\mu$
  if they have the same X-collapse. But the Springer correspondence is injective,
  so we have the map $\sP/\sim_X \xrightarrow{1-1} \sP_X \hookrightarrow \widehat{W}$
  which gives the claim.
  }
\end{proof}



%As a corrollary, we can obtain the following simple formule for
The following lemma is a direct consequence of \Cref{lem:MD1}.
\begin{lem}
  Suppose $\ckcO$ has good parity of type $\tC$. Under the Springer
  correspondence of type $C$, the representation $\sigma(\ckcO)$ (see
  \eqref{eq:sigma.tC}) corresponds to the nilpotent orbit $\cO$ defined by
  \[
    (\bfcc_{2i-1}(\cO),\bfcc_{2i}(\cO)) =\begin{cases}
      (\bfrr_{2i-1}(\ckcO),\bfrr_{2i}(\ckcO)), & \text{if } \bfrr_{2i-1}(\ckcO),\bfrr_{2i}(\ckcO)\\
      (\bfrr_{2i-1}(\ckcO)-1,\bfrr_{2i}(\ckcO)+1), & \text{otherwise} \\
    \end{cases}
  \]
  for all $i\in \bN^+$.
\end{lem}
\begin{proof}
  Apply the algorithm of Springer correspondence of type D to the representation $\sigma'(\ckcO)$ gives the above formula.
\end{proof}
\trivial{
A technical point, the representation of $W'_n$ is given by symbol $\binom{\xi}{\mu}$ or
$\binom{\mu}{\xi}$. However, using the Springer correspondence formula, only one of the arrangement
can give a valid type D partition.

It is a interesting fact that a type D orbit $\cO$ is special if and only if $\cO^t$ is of type C.



}



\def\tdBV{\tdd_{\text{BV}}}
\begin{lem}
Suppose $\ckcO = \ckcO_{b}\cup \ckcO_{g}$.
Then $\cO = \cO_{b} \cup \cO_{g}$ where $\cO_{b} = \ckcO_{b}^{t}$ and
$\cO_{g}=\tdBV(\ckcO_{g})$.
\end{lem}

% Let $\ckcO'_{g} = (\ckcO_{g})_{D}$. Then $\sigma_{D}(\ckcO_{g}) = \sigma_{D}(\ckcO'_{g})$.
% Let $\sigma_{\ckcO_{g}} = \sigma_{\ckcO'_{g}}$
% Let $\ckcO' = \ckcO_{b}\cup \ckcO'_{g}$.

\trivial[]{
  We take the convention that $2\cO = [2r_{i}]$ if $\cO = [r_{i}]$.
  We also write $[r_{i}]\cup [r_{j}] = [r_{i},r_{j}]$.
  $\dagger \cO = [r_{i}+1]$.

We suppose
\[
\ckcO_{b} = [2r_{1}+1, 2r_{1}+1, \cdots, 2r_{k}+1,2r_{k}+1]
= (2c_{0},2c_{1},2c_{1}, \cdots, 2c_{l}, 2c_{l})
\]
where $l = r_{1}$.

Now
\[
\begin{split}
  W_{\ckcO_{b}} &= W_{c_{0}} \times S_{2c_{1}} \times S_{2c_{2}}\times \cdots \times S_{2c_{l}}\\
  \cksigma_{b} &:= j_{W_{\ckcO_{b}}}^{W_{b}} \sgn = ((c_{1},c_{2},\cdots, c_{k}),(c_{0},c_{1}, \cdots, c_{l}))\\
  & = ([r_{1},r_{2},\cdots, r_{k}],[r_{1}+1,r_{2}+1,\cdots,r_{k}+1])
\end{split}
\]
Therefore
\[
\sigma_{b} = \cksigma_{b}\otimes \sgn = ((r_{1}+1,r_{2}+1,\cdots,r_{k}+1),(r_{1},r_{2},\cdots, r_{k}))
\]
which corresponds to the orbit
\[
  \cO_{b} = (2r_{1}+1, 2r_{1}+1,2r_{2}+1, 2r_{2}+1,  \cdots,2r_{k}+1, 2r_{k}+1 ) = \ckcO_{b}^{t}.
\]
This implies
\[
  \sigma_{b} = j_{W_{L_{b}}}^{W_{b}}\sgn, \quad \text{where } W_{L,b} = \prod_{i=1}^{k} S_{2r_{i}+1}.
\]
(Note that $\cO'_{b} = (2r_{1}+1,2r_{2}+1, \cdots, 2r_{k}+1)$ which corresponds
to $j_{W_{L_{b}}}^{S_{b}}\sgn$ and $\ind_{L}^{G} \cO'_{b} = \cO_{b}$.
)

Now we deduce that
\[
  \begin{split}
    \sigma &:= j_{W_{b}\times W_{g}}^{W_{n}} \sigma_{b}\otimes \sigma_{g}\\
    & = j_{W_{L_{b}}\times W_{g}}^{W_{n}} \sgn \otimes \sigma_{g}
  \end{split}
\]
where $W_{L_{b}}\times W_{g}$ is a parabolic subgroup of $W_{n}$
corresponds to the Levi factor $L$ of type
\[
A_{2r_{1}+1}\times A_{2r_{2}+1}\times \cdots \times A_{2r_{k}+1} \times W_{g}.
\]
Therefore
\[
\cO = \ind_{L}^{G} \triv\times \cO_{g} = \cO_{b}\cup \cO_{g}.
\]


% Note that
% \[
%   j_{S_{b}}^{W_{b}} 1
%   = (j_{S_{b}}^{W_{b}}\sgn)\otimes \sgn
%   = (J_{W_{\ckcO_{b}}}^{W_{b}} \sgn)\otimes
%   \sgn
% \]

% It suffice to compute
% \[
%   (J_{W_{\ckcO_{b}}\times W_{\ckcO_{g}}}^{\WLam} \sgn)\otimes \sgn
%   = (J_{W_{b}\times W_{g}}^{W} \sigma_{b} \otimes \sigma_{g}) \otimes \sgn
% \]
%
\def\ckfll{\check\fll}
\def\ckfgg{\check\fgg}

We claim that the map $\tdd\colon \ckcO \mapsto \cO$ defined here
coincide with our Metaplectic BV duality paper.

Note that $\tdBV$ is compatible with parabolic induction.
Suppose $\ckfll$ is a Levi subgroup of $\ckfgg$ and
$\ckcO_{\ckfll}:=\ckcO\cap \ckfll\neq \emptyset$.
Then
\[
\tdBV(\ckcO) = \ind_{\fll}^{\fgg} \tdBV(\ckcO_{\ckfll}).
\]
(Since $\tdBV$ commute with the descent map, the claim follows from the
prosperity of $\dBV$)

The map $\tdd$ also compatible with parabolic induction.
It suffice to consider the case where $\ckfll$ is a maximal parabolic of type
$A_{l}\times C_{n}$ and
the orbit is trivial on the $A_{l}$ factor.
Suppose $l$ is the bad parity, the claim is clear by our computation for the bad
parity case.

Now suppose $l=2m$ has good parity. If there is a $i$ such that
$\bfrr_{2i+1}(\ckcO)=\bfrr_{2i+2}(\ckcO)=2m$. Then
$\bfcc_{2i+1}(\cO)=\bfcc_{2i+2}(\cO)=2m$ and the claim follows.

Otherwise, we can assume
\[
R_{2i-1}:=\bfrr_{2i-1}(\ckcO)> \bfrr_{2i}(\ckcO)=\bfrr_{2i+1}(\ckcO) > \bfrr_{2i+2}(\ckcO)=:R_{2i+2}.
\]
and then
\[
\cO = (\cdots, R_{2i-1}-1, 2m+1, 2m-1, R_{2i+2}, \cdots)
\]
One check again that $\cO = \ind_{\fll}^{\fgg}\cO_{\fll}$.
(Note that the induction operation is add two length $2m$ columns and then apply
C-collapsing.)

Now it suffice to check that
$\tdd(\ckcO) = \tdBV(\ckcO)$ for every orbits $\ckcO$ whose
rows are multiplicity free) (i.e. $\ckcO$ is distinguished).
This is clear by the explicit formula for the both sides.
}

\section{Matching specail shape and non-special shape painted bipartitions}
\label{app:comb}

\subsection{Proof of ...}
\begin{proof}
  Let $i$ be the minimal integer such that $(2i-1, 2i)\in \sP$.
  We define $\sP' = \sP - \set{(2i-1,2i)}$.
  We will establish a bijection
  \[
    \PBP_{\star}(\ckcO_{g}, \sP')\longrightarrow \PBP_{\star}(\ckcO_{g},\sP).
  \]
\end{proof}

% \subsection{Embedding Harish-Chandra cells to cells in $\sO$}
% Let $H$ be a Cartan subgroup of $G$ and $T$ is the maximal compact subgroup of
% $H$. %Let $(\fgg, T)$-module.
% Let $\cO_{H}$ be the category


\begin{bibdiv}
  \begin{biblist}
% \bib{AB}{article}{
%   title={Genuine representations of the metaplectic group},
%   author={Adams, Jeffrey},
%   author = {Barbasch, Dan},
%   journal={Compositio Mathematica},
%   volume={113},
%   number={01},
%   pages={23--66},
%   year={1998},
% }

\bib{Ad83}{article}{
  author = {Adams, J.},
  title = {Discrete spectrum of the reductive dual pair $(O(p,q),Sp(2m))$ },
  journal = {Invent. Math.},
  number = {3},
 pages = {449--475},
 volume = {74},
 year = {1983}
}

%\bib{Ad07}{article}{
%  author = {Adams, J.},
%  title = {The theta correspondence over R},
%  journal = {Harmonic analysis, group representations, automorphic forms and invariant theory,  Lect. Notes Ser. Inst. Math. Sci. Natl. Univ. Singap., 12},
% pages = {1--39},
% year = {2007}
% publisher={World Sci. Publ.}
%}


\bib{ABV}{book}{
  title={The Langlands classification and irreducible characters for real reductive groups},
  author={Adams, J.},
  author={Barbasch, D.},
  author={Vogan, D. A.},
  series={Progress in Math.},
  volume={104},
  year={1991},
  publisher={Birkhauser}
}

\bib{AC}{article}{
  title={Algorithms for representation theory of
    real reductive groups},
  volume={8},
  DOI={10.1017/S1474748008000352},
  number={2},
  journal={Journal of the Institute of Mathematics of Jussieu},
  publisher={Cambridge University Press},
  author={Adams, Jeffrey},
  author={du Cloux, Fokko},
  year={2009},
  pages={209-259}
}

\bib{ArPro}{article}{
  author = {Arthur, J.},
  title = {On some problems suggested by the trace formula},
  journal = {Lie group representations, II (College Park, Md.), Lecture Notes in Math. 1041},
 pages = {1--49},
 year = {1984}
}


\bib{ArUni}{article}{
  author = {Arthur, J.},
  title = {Unipotent automorphic representations: conjectures},
  %booktitle = {Orbites unipotentes et repr\'esentations, II},
  journal = {Orbites unipotentes et repr\'esentations, II, Ast\'erisque},
 pages = {13--71},
 volume = {171-172},
 year = {1989}
}

\bib{AK}{article}{
  author = {Auslander, L.},
  author = {Kostant, B.},
  title = {Polarizations and unitary representations of solvable Lie groups},
  journal = {Invent. Math.},
 pages = {255--354},
 volume = {14},
 year = {1971}
}


\bib{B.Uni}{article}{
  author = {Barbasch, D.},
  title = {Unipotent representations for real reductive groups},
 %booktitle = {Proceedings of ICM, Kyoto 1990},
 journal = {Proceedings of ICM (1990), Kyoto},
   % series = {Proc. Sympos. Pure Math.},
 %   volume = {68},
     pages = {769--777},
 publisher = {Springer-Verlag, The Mathematical Society of Japan},
      year = {2000},
}


\bib{BMSZ1}{article}{
      title={On the notion of metaplectic Barbasch-Vogan duality},
      year={2020},
      author={Barbasch, Dan M.},
      author = {Ma, Jia-jun},
      author = {Sun, Binyong},
      author = {Zhu, Chen-Bo},
      eprint={2010.16089},
      archivePrefix={arXiv},
      primaryClass={math.RT}
}

\bib{BMSZ2}{article}{
      title={Special unipotent representations: orthogonal and symplectic groups},
      author={Barbasch, Dan M.},
      author = {Ma, Jia-jun},
      author = {Sun, Binyong},
      author = {Zhu, Chen-Bo},
      year={2021},
      eprint={1712.05552},
      archivePrefix={arXiv},
      primaryClass={math.RT}
}

\bib{BV1}{article}{
   author={Barbasch, Dan},
   author={Vogan, David},
   title={Primitive ideals and orbital integrals in complex classical
   groups},
   journal={Math. Ann.},
   volume={259},
   date={1982},
   number={2},
   pages={153--199},
   issn={0025-5831},
   review={\MR{656661}},
   doi={10.1007/BF01457308},
}

\bib{BV2}{article}{
   author={Barbasch, Dan},
   author={Vogan, David},
   title={Primitive ideals and orbital integrals in complex exceptional
   groups},
   journal={J. Algebra},
   volume={80},
   date={1983},
   number={2},
   pages={350--382},
   issn={0021-8693},
   review={\MR{691809}},
   doi={10.1016/0021-8693(83)90006-6},
}

\bib{BV.W}{article}{
  author={Barbasch, Dan},
  author={Vogan, David},
  editor={Trombi, P. C.},
  title={Weyl Group Representations and Nilpotent Orbits},
  bookTitle={Representation Theory of Reductive Groups:
    Proceedings of the University of Utah Conference 1982},
  year={1983},
  publisher={Birkh{\"a}user Boston},
  address={Boston, MA},
  pages={21--33},
  %doi={10.1007/978-1-4684-6730-7_2},
}



\bib{B.Orbit}{article}{
  author = {Barbasch, D.},
  title = {Orbital integrals of nilpotent orbits},
 %booktitle = {The mathematical legacy of {H}arish-{C}handra ({B}altimore,{MD}, 1998)},
    journal = {The mathematical legacy of {H}arish-{C}handra, Proc. Sympos. Pure Math.},
    %series={The mathematical legacy of {H}arish-{C}handra, Proc. Sympos. Pure Math},
    volume = {68},
     pages = {97--110},
 publisher = {Amer. Math. Soc., Providence, RI},
      year = {2000},
}



\bib{B10}{article}{
  author = {Barbasch, D.},
  title = {The unitary spherical spectrum for split classical groups},
  journal = {J. Inst. Math. Jussieu},
% number = {9},
 pages = {265--356},
 volume = {9},
 year = {2010}
}



\bib{B17}{article}{
  author = {Barbasch, D.},
  title = {Unipotent representations and the dual pair correspondence},
  journal = {J. Cogdell et al. (eds.), Representation Theory, Number Theory, and Invariant Theory, In Honor of Roger Howe. Progress in Math.},
  %series ={Progress in Math.},
  volume = {323},
  pages = {47--85},
  year = {2017},
}

\bib{BVUni}{article}{
 author = {Barbasch, D.},
 author = {Vogan, D. A.},
 journal = {Annals of Math.},
 number = {1},
 pages = {41--110},
 title = {Unipotent representations of complex semisimple groups},
 volume = {121},
 year = {1985}
}

\bib{BB}{article}{
   author={Beilinson, Alexandre},
   author={Bernstein, Joseph},
   title={Localisation de $\mathfrak g$-modules},
   language={French, with English summary},
   journal={C. R. Acad. Sci. Paris S\'{e}r. I Math.},
   volume={292},
   date={1981},
   number={1},
   pages={15--18},
   issn={0249-6291},
   review={\MR{610137}},
}

\bib{Br}{article}{
  author = {Brylinski, R.},
  title = {Dixmier algebras for classical complex nilpotent orbits via Kraft-Procesi models. I},
  journal = {The orbit method in geometry and physics (Marseille, 2000). Progress in Math.}
  volume = {213},
  pages = {49--67},
  year = {2003},
}

\bib{BK}{article}{
   author={Brylinski, J.-L.},
   author={Kashiwara, M.},
   title={Kazhdan-Lusztig conjecture and holonomic systems},
   journal={Invent. Math.},
   volume={64},
   date={1981},
   number={3},
   pages={387--410},
   issn={0020-9910},
   review={\MR{632980}},
   doi={10.1007/BF01389272},
}

\bib{Carter}{book}{
   author={Carter, Roger W.},
   title={Finite groups of Lie type},
   series={Wiley Classics Library},
   %note={Conjugacy classes and complex characters;
   %Reprint of the 1985 original;
   %A Wiley-Interscience Publication},
   publisher={John Wiley \& Sons, Ltd., Chichester},
   date={1993},
   pages={xii+544},
   isbn={0-471-94109-3},
   %review={\MR{1266626}},
}

\bib{Cas}{article}{
   author={Casian, Luis G.},
   title={Primitive ideals and representations},
   journal={J. Algebra},
   volume={101},
   date={1986},
   number={2},
   pages={497--515},
   issn={0021-8693},
   review={\MR{847174}},
   doi={10.1016/0021-8693(86)90208-5},
}

\bib{Ca89}{article}{
 author = {Casselman, W.},
 journal = {Canad. J. Math.},
 pages = {385--438},
 title = {Canonical extensions of Harish-Chandra modules to representations of $G$},
 volume = {41},
 year = {1989}
}



\bib{Cl}{article}{
  author = {Du Cloux, F.},
  journal = {Ann. Sci. \'Ecole Norm. Sup.},
  number = {3},
  pages = {257--318},
  title = {Sur les repr\'esentations diff\'erentiables des groupes de Lie alg\'ebriques},
  url = {http://eudml.org/doc/82297},
  volume = {24},
  year = {1991},
}

\bib{CM}{book}{
  title = {Nilpotent orbits in semisimple Lie algebra: an introduction},
  author = {Collingwood, D. H.},
  author = {McGovern, W. M.},
  year = {1993},
  publisher = {Van Nostrand Reinhold Co.},
}


% \bib{Dieu}{book}{
%    title={La g\'{e}om\'{e}trie des groupes classiques},
%    author={Dieudonn\'{e}, Jean},
%    year={1963},
% 	publisher={Springer},
%  }

\bib{DKPC}{article}{
title = {Nilpotent orbits and complex dual pairs},
journal = {J. Algebra},
volume = {190},
number = {2},
pages = {518 - 539},
year = {1997},
author = {Daszkiewicz, A.},
author = {Kra\'skiewicz, W.},
author = {Przebinda, T.},
}

\bib{DKP2}{article}{
  author = {Daszkiewicz, A.},
  author = {Kra\'skiewicz, W.},
  author = {Przebinda, T.},
  title = {Dual pairs and Kostant-Sekiguchi correspondence. II. Classification
	of nilpotent elements},
  journal = {Central European J. Math.},
  year = {2005},
  volume = {3},
  pages = {430--474},
}


\bib{DM}{article}{
  author = {Dixmier, J.},
  author = {Malliavin, P.},
  title = {Factorisations de fonctions et de vecteurs ind\'efiniment diff\'erentiables},
  journal = {Bull. Sci. Math. (2)},
  year = {1978},
  volume = {102},
  pages = {307--330},
}

%\bibitem[DM]{DM}
%J. Dixmier and P. Malliavin, \textit{Factorisations de fonctions et de vecteurs ind\'efiniment diff\'erentiables}, Bull. Sci. Math. (2), 102 (4),  307-330 (1978).



\bib{Du77}{article}{
  author = {Duflo, M.},
  journal = {Annals of Math.},
  number = {1},
  pages = {107-120},
  title = {Sur la Classification des Ideaux Primitifs Dans
    L'algebre Enveloppante d'une Algebre de Lie Semi-Simple},
  volume = {105},
  year = {1977}
}

\bib{Du82}{article}{
 author = {Duflo, M.},
 journal = {Acta Math.},
  volume = {149},
 number = {3-4},
 pages = {153--213},
 title = {Th\'eorie de Mackey pour les groupes de Lie alg\'ebriques},
 year = {1982}
}



\bib{GZ}{article}{
author={Gomez, R.},
author={Zhu, C.-B.},
title={Local theta lifting of generalized Whittaker models associated to nilpotent orbits},
journal={Geom. Funct. Anal.},
year={2014},
volume={24},
number={3},
pages={796--853},
}

\bib{EGAIV2}{article}{
  title = {\'El\'ements de g\'eom\'etrie alg\'brique IV: \'Etude locale des
    sch\'emas et des morphismes de sch\'emas. II},
  author = {Grothendieck, A.},
  author = {Dieudonn\'e, J.},
  journal  = {Inst. Hautes \'Etudes Sci. Publ. Math.},
  volume = {24},
  year = {1965},
}


\bib{EGAIV3}{article}{
  title = {\'El\'ements de g\'eom\'etrie alg\'brique IV: \'Etude locale des
    sch\'emas et des morphismes de sch\'emas. III},
  author = {Grothendieck, A.},
  author = {Dieudonn\'e, J.},
  journal  = {Inst. Hautes \'Etudes Sci. Publ. Math.},
  volume = {28},
  year = {1966},
}


\bib{HLS}{article}{
    author = {Harris, M.},
    author = {Li, J.-S.},
    author = {Sun, B.},
     title = {Theta correspondences for close unitary groups},
 %booktitle = {Arithmetic Geometry and Automorphic Forms},
    %series = {Adv. Lect. Math. (ALM)},
    journal = {Arithmetic Geometry and Automorphic Forms, Adv. Lect. Math. (ALM)},
    volume = {19},
     pages = {265--307},
 publisher = {Int. Press, Somerville, MA},
      year = {2011},
}

\bib{HS}{book}{
 author = {Hartshorne, R.},
 title = {Algebraic Geometry},
publisher={Graduate Texts in Mathematics, 52. New York-Heidelberg-Berlin: Springer-Verlag},
year={1983},
}

\bib{He}{article}{
author={He, H.},
title={Unipotent representations and quantum induction},
journal={arXiv:math/0210372},
year = {2002},
}

\bib{HL}{article}{
author={Huang, J.-S.},
author={Li, J.-S.},
title={Unipotent representations attached to spherical nilpotent orbits},
journal={Amer. J. Math.},
volume={121},
number = {3},
pages={497--517},
year={1999},
}


\bib{HZ}{article}{
author={Huang, J.-S.},
author={Zhu, C.-B.},
title={On certain small representations of indefinite orthogonal groups},
journal={Represent. Theory},
volume={1},
pages={190--206},
year={1997},
}



\bib{Howe79}{article}{
  title={$\theta$-series and invariant theory},
  author={Howe, R.},
  book = {
    title={Automorphic Forms, Representations and $L$-functions},
    series={Proc. Sympos. Pure Math},
    volume={33},
    year={1979},
  },
  pages={275-285},
}

\bib{HoweRank}{article}{
author={Howe, R.},
title={On a notion of rank for unitary representations of the classical groups},
journal={Harmonic analysis and group representations, Liguori, Naples},
pages={223-331},
year={1982},
}

\bib{Howe89}{article}{
author={Howe, R.},
title={Transcending classical invariant theory},
journal={J. Amer. Math. Soc.},
volume={2},
pages={535--552},
year={1989},
}

\bib{Howe95}{article}{,
  author = {Howe, R.},
  title = {Perspectives on invariant theory: Schur duality, multiplicity-free actions and beyond},
  journal = {Piatetski-Shapiro, I. et al. (eds.), The Schur lectures (1992). Ramat-Gan: Bar-Ilan University, Isr. Math. Conf. Proc. 8,},
  year = {1995},
  pages = {1-182},
}

\bib{H}{book}{
   author={Humphreys, James E.},
   title={Representations of semisimple Lie algebras in the BGG category
   $\scr{O}$},
   series={Graduate Studies in Mathematics},
   volume={94},
   publisher={American Mathematical Society, Providence, RI},
   date={2008},
   pages={xvi+289},
   isbn={978-0-8218-4678-0},
   review={\MR{2428237}},
   doi={10.1090/gsm/094},
}

\bib{J1}{article}{
   author={Joseph, A.},
   title={Goldie rank in the enveloping algebra of a semisimple Lie algebra.
   I},
   journal={J. Algebra},
   volume={65},
   date={1980},
   number={2},
   pages={269--283},
   issn={0021-8693},
   review={\MR{585721}},
   doi={10.1016/0021-8693(80)90217-3},
}

\bib{J2}{article}{
   author={Joseph, A.},
   title={Goldie rank in the enveloping algebra of a semisimple Lie algebra.
   II},
   journal={J. Algebra},
   volume={65},
   date={1980},
   number={2},
   pages={284--306},
   issn={0021-8693},
   review={\MR{585721}},
   doi={10.1016/0021-8693(80)90217-3},
}

\bib{J3}{article}{
   author={Joseph, A.},
   title={Goldie rank in the enveloping algebra of a semisimple Lie algebra.
   III},
   journal={J. Algebra},
   volume={73},
   date={1981},
   number={2},
   pages={295--326},
   issn={0021-8693},
   review={\MR{640039}},
   doi={10.1016/0021-8693(81)90324-0},
}
	


\bib{J.av}{article}{
   author={Joseph, Anthony},
   title={On the associated variety of a primitive ideal},
   journal={J. Algebra},
   volume={93},
   date={1985},
   number={2},
   pages={509--523},
   issn={0021-8693},
   review={\MR{786766}},
   doi={10.1016/0021-8693(85)90172-3},
}

\bib{J.ann}{article}{
   author={Joseph, Anthony},
   title={Annihilators and associated varieties of unitary highest weight
   modules},
   journal={Ann. Sci. \'{E}cole Norm. Sup. (4)},
   volume={25},
   date={1992},
   number={1},
   pages={1--45},
   issn={0012-9593},
   review={\MR{1152612}},
}

\bib{J.hw}{article}{
   author={Joseph, Anthony},
   title={On the variety of a highest weight module},
   journal={J. Algebra},
   volume={88},
   date={1984},
   number={1},
   pages={238--278},
   issn={0021-8693},
   review={\MR{741942}},
   doi={10.1016/0021-8693(84)90100-5},
}
	

\bib{JLS}{article}{
author={Jiang, D.},
author={Liu, B.},
author={Savin, G.},
title={Raising nilpotent orbits in wave-front sets},
journal={Represent. Theory},
volume={20},
pages={419--450},
year={2016},
}

\bib{Ki62}{article}{
author={Kirillov, A. A.},
title={Unitary representations of nilpotent Lie groups},
journal={Uspehi Mat. Nauk},
volume={17},
issue ={4},
pages={57--110},
year={1962},
}


\bib{Ko70}{article}{
author={Kostant, B.},
title={Quantization and unitary representations},
journal={Lectures in Modern Analysis and Applications III, Lecture Notes in Math.},
volume={170},
pages={87--208},
year={1970},
}


\bib{KP}{article}{
author={Kraft, H.},
author={Procesi, C.},
title={On the geometry of conjugacy classes in classical groups},
journal={Comment. Math. Helv.},
volume={57},
pages={539--602},
year={1982},
}

\bib{KR}{article}{
author={Kudla, S. S.},
author={Rallis, S.},
title={Degenerate principal series and invariant distributions},
journal={Israel J. Math.},
volume={69},
pages={25--45},
year={1990},
}


\bib{Ku}{article}{
author={Kudla, S. S.},
title={Some extensions of the Siegel-Weil formula},
journal={In: Gan W., Kudla S., Tschinkel Y. (eds) Eisenstein Series and Applications. Progress in Mathematics, vol 258. Birkh\"auser Boston},
%volume={69},
pages={205--237},
year={2008},
}





\bib{LZ1}{article}{
author={Lee, S. T.},
author={Zhu, C.-B.},
title={Degenerate principal series and local theta correspondence II},
journal={Israel J. Math.},
volume={100},
pages={29--59},
year={1997},
}

\bib{LZ2}{article}{
author={Lee, S. T.},
author={Zhu, C.-B.},
title={Degenerate principal series of metaplectic groups and Howe correspondence},
journal = {D. Prasad at al. (eds.), Automorphic Representations and L-Functions, Tata Institute of Fundamental Research, India,},
year = {2013},
pages = {379--408},
}

\bib{Li89}{article}{
author={Li, J.-S.},
title={Singular unitary representations of classical groups},
journal={Invent. Math.},
volume={97},
number = {2},
pages={237--255},
year={1989},
}

\bib{LiuAG}{book}{
  title={Algebraic Geometry and Arithmetic Curves},
  author = {Liu, Q.},
  year = {2006},
  publisher={Oxford University Press},
}

\bib{LM}{article}{
   author = {Loke, H. Y.},
   author = {Ma, J.},
    title = {Invariants and $K$-spectrums of local theta lifts},
    journal = {Compositio Math.},
    volume = {151},
    issue = {01},
    year = {2015},
    pages ={179--206},
}

\bib{DL}{article}{
   author={Deligne, P.},
   author={Lusztig, G.},
   title={Representations of reductive groups over finite fields},
   journal={Ann. of Math. (2)},
   volume={103},
   date={1976},
   number={1},
   pages={103--161},
   issn={0003-486X},
   review={\MR{393266}},
   doi={10.2307/1971021},
}

\bib{KL}{article}{
   author={Kazhdan, David},
   author={Lusztig, George},
   title={Representations of Coxeter groups and Hecke algebras},
   journal={Invent. Math.},
   volume={53},
   date={1979},
   number={2},
   pages={165--184},
   issn={0020-9910},
   review={\MR{560412}},
   doi={10.1007/BF01390031},
}

\bib{Lu}{book}{
   author={Lusztig, George},
   title={Characters of reductive groups over a finite field},
   series={Annals of Mathematics Studies},
   volume={107},
   publisher={Princeton University Press, Princeton, NJ},
   date={1984},
   pages={xxi+384},
   isbn={0-691-08350-9},
   isbn={0-691-08351-7},
   review={\MR{742472}},
   doi={10.1515/9781400881772},
}


\bib{LS}{article}{
   author = {Lusztig, G.},
   author = {Spaltenstein, N.},
    title = {Induced unipotent classes},
    journal = {j. London Math. Soc.},
    volume = {19},
    year = {1979},
    pages ={41--52},
}

\bib{Lu.I}{article}{
   author={Lusztig, G.},
   title={Intersection cohomology complexes on a reductive group},
   journal={Invent. Math.},
   volume={75},
   date={1984},
   number={2},
   pages={205--272},
   issn={0020-9910},
   review={\MR{732546}},
   doi={10.1007/BF01388564},
}
	

\bib{Ma}{article}{
   author = {Mackey, G. W.},
    title = {Unitary representations of group extentions},
    journal = {Acta Math.},
    volume = {99},
    year = {1958},
    pages ={265--311},
}


\bib{Mc}{article}{
   author = {McGovern, W. M},
    title = {Cells of Harish-Chandra modules for real classical groups},
    journal = {Amer. J.  of Math.},
    volume = {120},
    issue = {01},
    year = {1998},
    pages ={211--228},
}

\bib{Mo96}{article}{
 author={M{\oe}glin, C.},
    title = {Front d'onde des repr\'esentations des groupes classiques $p$-adiques},
    journal = {Amer. J. Math.},
    volume = {118},
    issue = {06},
    year = {1996},
    pages ={1313--1346},
}

\bib{Mo17}{article}{
  author={M{\oe}glin, C.},
  title = {Paquets d'Arthur Sp\'eciaux Unipotents aux Places Archim\'ediennes et Correspondance de Howe},
  journal = {J. Cogdell et al. (eds.), Representation Theory, Number Theory, and Invariant Theory, In Honor of Roger Howe. Progress in Math.}
  %series ={Progress in Math.},
  volume = {323},
  pages = {469--502}
  year = {2017}
}

\bib{MR19}{article}{
   author={M\oe glin, Colette},
   author={Renard, David},
   title={Sur les paquets d'Arthur des groupes unitaires et quelques
   cons\'{e}quences pour les groupes classiques},
   language={French, with English and French summaries},
   journal={Pacific J. Math.},
   volume={299},
   date={2019},
   number={1},
   pages={53--88},
   issn={0030-8730},
   review={\MR{3947270}},
   doi={10.2140/pjm.2019.299.53},
}


\bib{MVW}{book}{
  volume={1291},
  title={Correspondances de Howe sur un corps $p$-adique},
  author={M{\oe}glin, C.},
  author={Vign\'eras, M.-F.},
  author={Waldspurger, J.-L.},
  series={Lecture Notes in Mathematics},
  publisher={Springer}
  ISBN={978-3-540-18699-1},
  date={1987},
}

\bib{NOTYK}{article}{
   author = {Nishiyama, K.},
   author = {Ochiai, H.},
   author = {Taniguchi, K.},
   author = {Yamashita, H.},
   author = {Kato, S.},
    title = {Nilpotent orbits, associated cycles and Whittaker models for highest weight representations},
    journal = {Ast\'erisque},
    volume = {273},
    year = {2001},
   pages ={1--163},
}

\bib{NOZ}{article}{
  author = {Nishiyama, K.},
  author = {Ochiai, H.},
  author = {Zhu, C.-B.},
  journal = {Trans. Amer. Math. Soc.},
  title = {Theta lifting of nilpotent orbits for symmetric pairs},
  volume = {358},
  year = {2006},
  pages = {2713--2734},
}


\bib{NZ}{article}{
   author = {Nishiyama, K.},
   author = {Zhu, C.-B.},
    title = {Theta lifting of unitary lowest weight modules and their associated cycles},
    journal = {Duke Math. J.},
    volume = {125},
    number= {03},
    year = {2004},
   pages ={415--465},
}



\bib{Ohta}{article}{
  author = {Ohta, T.},
  %doi = {10.2748/tmj/1178227492},
  journal = {Tohoku Math. J.},
  number = {2},
  pages = {161--211},
  publisher = {Tohoku University, Mathematical Institute},
  title = {The closures of nilpotent orbits in the classical symmetric
    pairs and their singularities},
  volume = {43},
  year = {1991}
}

\bib{Ohta2}{article}{
  author = {Ohta, T.},
  journal = {Hiroshima Math. J.},
  number = {2},
  pages = {347--360},
  title = {Induction of nilpotent orbits for real reductive groups and associated varieties of standard representations},
  volume = {29},
  year = {1999}
}

\bib{Ohta4}{article}{
  title={Nilpotent orbits of $\mathbb{Z}_4$-graded Lie algebra and geometry of
    moment maps associated to the dual pair $(\mathrm{U} (p, q), \mathrm{U} (r, s))$},
  author={Ohta, T.},
  journal={Publ. RIMS},
  volume={41},
  number={3},
  pages={723--756},
  year={2005}
}

\bib{PT}{article}{
  title={Some small unipotent representations of indefinite orthogonal groups and the theta correspondence},
  author={Paul, A.},
  author={Trapa, P.},
  journal={University of Aarhus Publ. Series},
  volume={48},
  pages={103--125},
  year={2007}
}


\bib{PV}{article}{
  title={Invariant Theory},
  author={Popov, V. L.},
  author={Vinberg, E. B.},
  book={
  title={Algebraic Geometry IV: Linear Algebraic Groups, Invariant Theory},
  series={Encyclopedia of Mathematical Sciences},
  volume={55},
  year={1994},
  publisher={Springer},}
}




%\bib{PPz}{article}{
%author={Protsak, V.} ,
%author={Przebinda, T.},
%title={On the occurrence of admissible representations in the real Howe
%    correspondence in stable range},
%journal={Manuscr. Math.},
%volume={126},
%number={2},
%pages={135--141},
%year={2008}
%}


\bib{PrzInf}{article}{
      author={Przebinda, T.},
       title={The duality correspondence of infinitesimal characters},
        date={1996},
     journal={Colloq. Math.},
      volume={70},
       pages={93--102},
}


\bib{Pz}{article}{
author={Przebinda, T.},
title={Characters, dual pairs, and unitary representations},
journal={Duke Math. J. },
volume={69},
number={3},
pages={547--592},
year={1993}
}

\bib{Ra}{article}{
author={Rallis, S.},
title={On the Howe duality conjecture},
journal={Compositio Math.},
volume={51},
pages={333--399},
year={1984}
}

\bib{RT1}{article}{
   author={Renard, David A.},
   author={Trapa, Peter E.},
   title={Irreducible genuine characters of the metaplectic group:
   Kazhdan-Lusztig algorithm and Vogan duality},
   journal={Represent. Theory},
   volume={4},
   date={2000},
   pages={245--295},
   review={\MR{1795754}},
   doi={10.1090/S1088-4165-00-00105-9},
}

\bib{RT2}{article}{
   author={Renard, David A.},
   author={Trapa, Peter E.},
   title={Irreducible characters of the metaplectic group. II.
   Functoriality},
   journal={J. Reine Angew. Math.},
   volume={557},
   date={2003},
   pages={121--158},
   issn={0075-4102},
   review={\MR{1978405}},
   doi={10.1515/crll.2003.028},
}

\bib{Sa}{article}{
author={Sahi, S.},
title={Explicit Hilbert spaces for certain unipotent representations},
journal={Invent. Math.},
volume={110},
number = {2},
pages={409--418},
year={1992}
}

\bib{Se}{article}{
author={Sekiguchi, J.},
title={Remarks on real nilpotent orbits of a symmetric pair},
journal={J. Math. Soc. Japan},
%publisher={The Mathematical Society of Japan},
year={1987},
volume={39},
number={1},
pages={127--138},
}

\bib{SV}{article}{
  author = {Schmid, W.},
  author = {Vilonen, K.},
  journal = {Annals of Math.},
  number = {3},
  pages = {1071--1118},
  %publisher = {Princeton University, Mathematics Department, Princeton, NJ; Mathematical Sciences Publishers, Berkeley},
  title = {Characteristic cycles and wave front cycles of representations of reductive Lie groups},
  volume = {151},
year = {2000},
}


\bib{Soergel}{article}{
   author={Soergel, Wolfgang},
   title={Kategorie $\scr O$, perverse Garben und Moduln \"{u}ber den
   Koinvarianten zur Weylgruppe},
   language={German, with English summary},
   journal={J. Amer. Math. Soc.},
   volume={3},
   date={1990},
   number={2},
   pages={421--445},
   issn={0894-0347},
   review={\MR{1029692}},
   doi={10.2307/1990960},
}


\bib{So}{article}{
author = {Sommers, E.},
title = {Lusztig's canonical quotient and generalized duality},
journal = {J. Algebra},
volume = {243},
number = {2},
pages = {790--812},
year = {2001},
}

\bib{SS}{book}{
  author = {Springer, T. A.},
  author = {Steinberg, R.},
  title = {Seminar on algebraic groups and related finite groups; Conjugate classes},
  series = {Lecture Notes in Math.},
  volume = {131},
publisher={Springer},
year={1970},
}

\bib{SZ1}{article}{
title={A general form of Gelfand-Kazhdan criterion},
author={Sun, B.},
author={Zhu, C.-B.},
journal={Manuscripta Math.},
pages = {185--197},
volume = {136},
year={2011}
}


%\bib{SZ2}{article}{
%  title={Conservation relations for local theta correspondence},
%  author={Sun, B.},
%  author={Zhu, C.-B.},
%  journal={J. Amer. Math. Soc.},
%  pages = {939--983},
%  volume = {28},
%  year={2015}
%}



\bib{Tr}{article}{
  title={Special unipotent representations and the Howe correspondence},
  author={Trapa, P.},
  year = {2004},
  journal={University of Aarhus Publication Series},
  volume = {47},
  pages= {210--230}
}

% \bib{Wa}{article}{
%    author = {Waldspurger, J.-L.},
%     title = {D\'{e}monstration d'une conjecture de dualit\'{e} de Howe dans le cas $p$-adique, $p \neq 2$ in Festschrift in honor of I. I. Piatetski-Shapiro on the occasion of his sixtieth birthday},
%   journal = {Israel Math. Conf. Proc., 2, Weizmann, Jerusalem},
%  year = {1990},
% pages = {267-324},
% }

\bib{VGK}{article}{
   author={Vogan, David A., Jr.},
   title={Gel\cprime fand-Kirillov dimension for Harish-Chandra modules},
   journal={Invent. Math.},
   volume={48},
   date={1978},
   number={1},
   pages={75--98},
   issn={0020-9910},
   review={\MR{506503}},
   doi={10.1007/BF01390063},
}

\bib{Vg}{book}{
   author={Vogan, David A.},
   title={Representations of real reductive Lie groups},
   series={Progress in Mathematics},
   volume={15},
   publisher={Birkh\"{a}user, Boston, Mass.},
   date={1981},
   pages={xvii+754},
   isbn={3-7643-3037-6},
   review={\MR{632407}},
}

\bib{V4}{article}{
   author={Vogan, D. A. },
   title={Irreducible characters of semisimple Lie groups. IV.
   Character-multiplicity duality},
   journal={Duke Math. J.},
   volume={49},
   date={1982},
   number={4},
   pages={943--1073},
   issn={0012-7094},
   review={\MR{683010}},
}

\bib{VoBook}{book}{
author = {Vogan, D. A. },
  title={Unitary representations of reductive Lie groups},
  year={1987},
  series = {Ann. of Math. Stud.},
 volume={118},
  publisher={Princeton University Press}
}


\bib{Vo89}{article}{
  author = {Vogan, D. A. },
  title = {Associated varieties and unipotent representations},
 %booktitle ={Harmonic analysis on reductive groups, Proc. Conf., Brunswick/ME (USA) 1989,},
  journal = {Harmonic analysis on reductive groups, Proc. Conf., Brunswick/ME
    (USA) 1989, Prog. Math.},
 volume={101},
  publisher = {Birkh\"{a}user, Boston-Basel-Berlin},
  year = {1991},
pages={315--388},
  editor = {W. Barker and P. Sally},
}

\bib{Vo98}{article}{
  author = {Vogan, D. A. },
  title = {The method of coadjoint orbits for real reductive groups},
 %booktitle ={Representation theory of Lie groups (Park City, UT, 1998)},
 journal = {Representation theory of Lie groups (Park City, UT, 1998). IAS/Park City Math. Ser.},
  volume={8},
  publisher = {Amer. Math. Soc.},
  year = {2000},
pages={179--238},
}

\bib{Vo00}{article}{
  author = {Vogan, D. A. },
  title = {Unitary representations of reductive Lie groups},
 %booktitle ={Mathematics towards the Third Millennium (Rome, 1999)},
 journal ={Mathematics towards the Third Millennium (Rome, 1999). Accademia Nazionale dei Lincei, (2000)},
  %series = {Accademia Nazionale dei Lincei, 2000},
 %volume={9},
pages={147--167},
}


\bib{Wa1}{book}{
  title={Real reductive groups I},
  author={Wallach, N. R.},
  year={1988},
  publisher={Academic Press Inc. }
}

\bib{Wa2}{book}{
  title={Real reductive groups II},
  author={Wallach, N. R.},
  year={1992},
  publisher={Academic Press Inc. }
}


\bib{Weyl}{book}{
  title={The classical groups: their invariants and representations},
  author={Weyl, H.},
  year={1947},
  publisher={Princeton University Press}
}

\bib{Ya}{article}{
  title={Degenerate principal series representations for quaternionic unitary groups},
  author={Yamana, S.},
  year = {2011},
  journal={Israel J. Math.},
  volume = {185},
  pages= {77--124}
}



% \bib{EGAIV4}{article}{
%   title = {\'El\'ements de g\'eom\'etrie alg\'brique IV 4: \'Etude locale des
%     sch\'emas et des morphismes de sch\'emas},
%   author = {Grothendieck, Alexandre},
%   author = {Dieudonn\'e, Jean},
%   journal  = {Inst. Hautes \'Etudes Sci. Publ. Math.},
%   volume = {32},
%   year = {1967},
%   pages = {5--361}
% }



\end{biblist}
\end{bibdiv}


\end{document}


%%% Local Variables:
%%% coding: utf-8
%%% mode: latex
%%% TeX-engine: tex
%%% ispell-local-dictionary: "en_US"
%%% End:
