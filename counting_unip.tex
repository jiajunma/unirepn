\documentclass[12pt,a4paper]{amsart}
\usepackage[margin=2.5cm,marginpar=2cm]{geometry}



% \usepackage{showkeys}
% \makeatletter
%   \SK@def\Cref#1{\SK@\SK@@ref{#1}\SK@Cref{#1}}
%   \SK@def\cref#1{\SK@\SK@@ref{#1}\SK@cref{#1}}
% \makeatother
\usepackage[bookmarksopen,bookmarksdepth=3]{hyperref}
\usepackage[nameinlink]{cleveref}

%% FONTS
\usepackage{amssymb}
%\usepackage{amsmath}
\usepackage{mathrsfs}
%\usepackage{amsrefs}
\usepackage{mathbbol,mathabx}
\usepackage{amsthm}
\usepackage{graphicx}
\usepackage{braket}
%\usepackage[pointedenum]{paralist}
%\usepackage{paralist}


\usepackage{amsrefs}

\usepackage[all,cmtip]{xy}
\usepackage{rotating}
\usepackage{leftidx}
%\usepackage{arydshln}

%\DeclareSymbolFont{bbold}{U}{bbold}{m}{n}
%\DeclareSymbolFontAlphabet{\mathbbold}{bbold}


%\usepackage[dvipdfx,rgb,table]{xcolor}
\usepackage[rgb,table]{xcolor}
%\usepackage{mathrsfs}

\setcounter{tocdepth}{1}
\setcounter{secnumdepth}{3}

%\usepackage[abbrev,shortalphabetic]{amsrefs}


\usepackage[normalem]{ulem}

% circled number
\usepackage{pifont}
\makeatletter
\newcommand*{\circnuma}[1]{%
  \ifnum#1<1 %
    \@ctrerr
  \else
    \ifnum#1>20 %
      \@ctrerr
    \else
      \mbox{\ding{\numexpr 171+(#1)\relax}}%
     \fi
  \fi
}
\makeatother

\usepackage[centertableaux]{ytableau}

%\usepackage[mathlines,pagewise]{lineno}
%\linenumbers

\usepackage{enumitem}
%% Enumitem
\newlist{enumC}{enumerate}{1} % Conditions in Lemma/Theorem/Prop
\setlist[enumC,1]{label=(\alph*),wide,ref=(\alph*)}
\crefname{enumCi}{condition}{conditions}
\Crefname{enumCi}{Condition}{Conditions}
\newlist{enumT}{enumerate}{3} % "Theorem"=conclusions in Lemma/Theorem/Prop
\setlist[enumT]{label=(\roman*),wide}
\setlist[enumT,1]{label=(\roman*),wide}
\setlist[enumT,2]{label=(\alph*),ref ={(\roman{enumTi}.\alph*)}}
\setlist[enumT,3]{label=(\arabic*), ref ={(\roman{enumTi}.\alph{enumTii}.\alph*)}}
\crefname{enumTi}{}{}
\Crefname{enumTi}{Item}{Items}
\crefname{enumTii}{}{}
\Crefname{enumTii}{Item}{Items}
\crefname{enumTiii}{}{}
\Crefname{enumTiii}{Item}{Items}
\newlist{enumPF}{enumerate}{3}
\setlist[enumPF]{label=(\alph*),wide}
\setlist[enumPF,1]{label=(\roman*),wide}
\setlist[enumPF,2]{label=(\alph*)}
\setlist[enumPF,3]{label=\arabic*).}
\newlist{enumS}{enumerate}{3} % Statement outside Lemma/Theorem/Prop
\setlist[enumS]{label=\roman*)}
\setlist[enumS,1]{label=\roman*)}
\setlist[enumS,2]{label=\alph*)}
\setlist[enumS,3]{label=\arabic*.}
\newlist{enumI}{enumerate}{3} % items
\setlist[enumI,1]{label=\roman*),leftmargin=*}
\setlist[enumI,2]{label=\alph*), leftmargin=*}
\setlist[enumI,3]{label=\arabic*), leftmargin=*}
\newlist{enumIL}{enumerate*}{1} % inline enum
\setlist*[enumIL]{label=\roman*)}
\newlist{enumR}{enumerate}{1} % remarks
\setlist[enumR]{label=\arabic*.,wide,labelwidth=!, labelindent=0pt}
\crefname{enumRi}{remark}{remarks}

\crefname{equation}{}{}
\Crefname{equation}{Equation}{Equations}
\Crefname{lem}{Lemma}{Lemma}
\Crefname{thm}{Theorem}{Theorem}

\newlist{des}{description}{1}
\setlist[des]{font=\sffamily\bfseries}

% editing macros.
\blendcolors{!80!black}
\long\def\okay#1{\ifcsname highlightokay\endcsname
{\color{red} #1}
\else
{#1}
\fi
}
\long\def\editc#1{{\color{red} #1}}
\long\def\mjj#1{{{\color{blue}#1}}}
\long\def\mjjr#1{{\color{red} (#1)}}
\long\def\mjjd#1#2{{\color{blue} #1 \sout{#2}}}
\def\mjjb{\color{blue}}
\def\mjje{\color{black}}
\def\mjjcb{\color{green!50!black}}
\def\mjjce{\color{black}}

\long\def\sun#1{{{\color{cyan}#1}}}
\long\def\sund#1#2{{\color{cyan}#1  \sout{#2}}}
\long\def\mv#1{{{\color{red} {\bf move to a proper place:} #1}}}
\long\def\delete#1{}

%\reversemarginpar
\newcommand{\lokec}[1]{\marginpar{\color{blue}\tiny #1 \mbox{--loke}}}
\newcommand{\mjjc}[1]{\marginpar{\color{green}\tiny #1 \mbox{--ma}}}

\newcommand{\trivial}[2][]{\if\relax\detokenize{#1}\relax
  {%\hfill\break
   % \begin{minipage}{\textwidth}
      \color{orange} \vspace{0em} $[$  #2 $]$
  %\end{minipage}
  %\break
      \color{black}
  }
  \else
\ifx#1h
\ifcsname showtrivial\endcsname
{%\hfill\break
 % \begin{minipage}{\textwidth}
    \color{orange} \vspace{0em}  $[$ #2 $]$
%\end{minipage}
%\break
    \color{black}
}
\fi
\else {\red Wrong argument!} \fi
\fi
}

\newcommand{\byhide}[2][]{\if\relax\detokenize{#1}\relax
{\color{orange} \vspace{0em} Plan to delete:  #2}
\else
\ifx#1h\relax\fi
\fi
}



\newcommand{\Rank}{\mathrm{rk}}
\newcommand{\cqq}{\mathscr{D}}
\newcommand{\rsym}{\mathrm{sym}}
\newcommand{\rskew}{\mathrm{skew}}
\newcommand{\fraksp}{\mathfrak{sp}}
\newcommand{\frakso}{\mathfrak{so}}
\newcommand{\frakm}{\mathfrak{m}}
\newcommand{\frakp}{\mathfrak{p}}
\newcommand{\pr}{\mathrm{pr}}
\newcommand{\rhopst}{\rho'^*}
\newcommand{\Rad}{\mathrm{Rad}}
\newcommand{\Res}{\mathrm{Res}}
\newcommand{\Hol}{\mathrm{Hol}}
\newcommand{\AC}{\mathrm{AC}}
%\newcommand{\AS}{\mathrm{AS}}
\newcommand{\WF}{\mathrm{WF}}
\newcommand{\AV}{\mathrm{AV}}
\newcommand{\AVC}{\mathrm{AV}_\bC}
\newcommand{\VC}{\mathrm{V}_\bC}
\newcommand{\bfv}{\mathbf{v}}
\newcommand{\depth}{\mathrm{depth}}
\newcommand{\wtM}{\widetilde{M}}
\newcommand{\wtMone}{{\widetilde{M}^{(1,1)}}}

\newcommand{\nullpp}{N(\fpp'^*)}
\newcommand{\nullp}{N(\fpp^*)}
%\newcommand{\Aut}{\mathrm{Aut}}

\def\mstar{{\medstar}}


\newcommand{\bfone}{\mathbf{1}}
\newcommand{\piSigma}{\pi_\Sigma}
\newcommand{\piSigmap}{\pi'_\Sigma}


\newcommand{\sfVprime}{\mathsf{V}^\prime}
\newcommand{\sfVdprime}{\mathsf{V}^{\prime \prime}}
\newcommand{\gminusone}{\mathfrak{g}_{-\frac{1}{m}}}

\newcommand{\eva}{\mathrm{eva}}

% \newcommand\iso{\xrightarrow{
%    \,\smash{\raisebox{-0.65ex}{\ensuremath{\scriptstyle\sim}}}\,}}

\def\Ueven{{U_{\rm{even}}}}
\def\Uodd{{U_{\rm{odd}}}}
\def\ttau{\tilde{\tau}}
\def\Wcp{W}
\def\Kur{{K^{\mathrm{u}}}}

\def\Im{\operatorname{Im}}

\providecommand{\bcN}{{\overline{\cN}}}



\makeatletter

\def\gen#1{\left\langle
    #1
      \right\rangle}
\makeatother

\makeatletter
\def\inn#1#2{\left\langle
      \def\ta{#1}\def\tb{#2}
      \ifx\ta\@empty{\;} \else {\ta}\fi ,
      \ifx\tb\@empty{\;} \else {\tb}\fi
      \right\rangle}
\def\binn#1#2{\left\lAngle
      \def\ta{#1}\def\tb{#2}
      \ifx\ta\@empty{\;} \else {\ta}\fi ,
      \ifx\tb\@empty{\;} \else {\tb}\fi
      \right\rAngle}
\makeatother

\makeatletter
\def\binn#1#2{\overline{\inn{#1}{#2}}}
\makeatother


\def\innwi#1#2{\inn{#1}{#2}_{W_i}}
\def\innw#1#2{\inn{#1}{#2}_{\bfW}}
\def\innv#1#2{\inn{#1}{#2}_{\bfV}}
\def\innbfv#1#2{\inn{#1}{#2}_{\bfV}}
\def\innvi#1#2{\inn{#1}{#2}_{V_i}}
\def\innvp#1#2{\inn{#1}{#2}_{\bfV'}}
\def\innp#1#2{\inn{#1}{#2}'}

% choose one of then
\def\simrightarrow{\iso}
\def\surj{\twoheadrightarrow}
%\def\simrightarrow{\xrightarrow{\sim}}

\newcommand\iso{\xrightarrow{
   \,\smash{\raisebox{-0.65ex}{\ensuremath{\scriptstyle\sim}}}\,}}

\newcommand\riso{\xleftarrow{
   \,\smash{\raisebox{-0.65ex}{\ensuremath{\scriptstyle\sim}}}\,}}









\usepackage{xparse}
\def\usecsname#1{\csname #1\endcsname}
\def\useLetter#1{#1}
\def\usedbletter#1{#1#1}

% \def\useCSf#1{\csname f#1\endcsname}

\ExplSyntaxOn

\def\mydefcirc#1#2#3{\expandafter\def\csname
  circ#3{#1}\endcsname{{}^\circ {#2{#1}}}}
\def\mydefvec#1#2#3{\expandafter\def\csname
  vec#3{#1}\endcsname{\vec{#2{#1}}}}
\def\mydefdot#1#2#3{\expandafter\def\csname
  dot#3{#1}\endcsname{\dot{#2{#1}}}}

\def\mydefacute#1#2#3{\expandafter\def\csname a#3{#1}\endcsname{\acute{#2{#1}}}}
\def\mydefbr#1#2#3{\expandafter\def\csname br#3{#1}\endcsname{\breve{#2{#1}}}}
\def\mydefbar#1#2#3{\expandafter\def\csname bar#3{#1}\endcsname{\bar{#2{#1}}}}
\def\mydefhat#1#2#3{\expandafter\def\csname hat#3{#1}\endcsname{\hat{#2{#1}}}}
\def\mydefwh#1#2#3{\expandafter\def\csname wh#3{#1}\endcsname{\widehat{#2{#1}}}}
\def\mydeft#1#2#3{\expandafter\def\csname t#3{#1}\endcsname{\tilde{#2{#1}}}}
\def\mydefu#1#2#3{\expandafter\def\csname u#3{#1}\endcsname{\underline{#2{#1}}}}
\def\mydefr#1#2#3{\expandafter\def\csname r#3{#1}\endcsname{\mathrm{#2{#1}}}}
\def\mydefb#1#2#3{\expandafter\def\csname b#3{#1}\endcsname{\mathbb{#2{#1}}}}
\def\mydefwt#1#2#3{\expandafter\def\csname wt#3{#1}\endcsname{\widetilde{#2{#1}}}}
%\def\mydeff#1#2#3{\expandafter\def\csname f#3{#1}\endcsname{\mathfrak{#2{#1}}}}
\def\mydefbf#1#2#3{\expandafter\def\csname bf#3{#1}\endcsname{\mathbf{#2{#1}}}}
\def\mydefc#1#2#3{\expandafter\def\csname c#3{#1}\endcsname{\mathcal{#2{#1}}}}
\def\mydefsf#1#2#3{\expandafter\def\csname sf#3{#1}\endcsname{\mathsf{#2{#1}}}}
\def\mydefs#1#2#3{\expandafter\def\csname s#3{#1}\endcsname{\mathscr{#2{#1}}}}
\def\mydefcks#1#2#3{\expandafter\def\csname cks#3{#1}\endcsname{{\check{
        \csname s#2{#1}\endcsname}}}}
\def\mydefckc#1#2#3{\expandafter\def\csname ckc#3{#1}\endcsname{{\check{
      \csname c#2{#1}\endcsname}}}}
\def\mydefck#1#2#3{\expandafter\def\csname ck#3{#1}\endcsname{{\check{#2{#1}}}}}

\cs_new:Npn \mydeff #1#2#3 {\cs_new:cpn {f#3{#1}} {\mathfrak{#2{#1}}}}

\cs_new:Npn \doGreek #1
{
  \clist_map_inline:nn {alpha,beta,gamma,Gamma,delta,Delta,epsilon,varepsilon,zeta,eta,theta,vartheta,Theta,iota,kappa,lambda,Lambda,mu,nu,xi,Xi,pi,Pi,rho,sigma,varsigma,Sigma,tau,upsilon,Upsilon,phi,varphi,Phi,chi,psi,Psi,omega,Omega,tG} {#1{##1}{\usecsname}{\useLetter}}
}

\cs_new:Npn \doSymbols #1
{
  \clist_map_inline:nn {otimes,boxtimes} {#1{##1}{\usecsname}{\useLetter}}
}

\cs_new:Npn \doAtZ #1
{
  \clist_map_inline:nn {A,B,C,D,E,F,G,H,I,J,K,L,M,N,O,P,Q,R,S,T,U,V,W,X,Y,Z} {#1{##1}{\useLetter}{\useLetter}}
}

\cs_new:Npn \doatz #1
{
  \clist_map_inline:nn {a,b,c,d,e,f,g,h,i,j,k,l,m,n,o,p,q,r,s,t,u,v,w,x,y,z} {#1{##1}{\useLetter}{\usedbletter}}
}

\cs_new:Npn \doallAtZ
{
\clist_map_inline:nn {mydefsf,mydeft,mydefu,mydefwh,mydefhat,mydefr,mydefwt,mydeff,mydefb,mydefbf,mydefc,mydefs,mydefck,mydefcks,mydefckc,mydefbar,mydefvec,mydefcirc,mydefdot,mydefbr,mydefacute} {\doAtZ{\csname ##1\endcsname}}
}

\cs_new:Npn \doallatz
{
\clist_map_inline:nn {mydefsf,mydeft,mydefu,mydefwh,mydefhat,mydefr,mydefwt,mydeff,mydefb,mydefbf,mydefc,mydefs,mydefck,mydefbar,mydefvec,mydefdot,mydefbr,mydefacute} {\doatz{\csname ##1\endcsname}}
}

\cs_new:Npn \doallGreek
{
\clist_map_inline:nn {mydefck,mydefwt,mydeft,mydefwh,mydefbar,mydefu,mydefvec,mydefcirc,mydefdot,mydefbr,mydefacute} {\doGreek{\csname ##1\endcsname}}
}

\cs_new:Npn \doallSymbols
{
\clist_map_inline:nn {mydefck,mydefwt,mydeft,mydefwh,mydefbar,mydefu,mydefvec,mydefcirc,mydefdot} {\doSymbols{\csname ##1\endcsname}}
}



\cs_new:Npn \doGroups #1
{
  \clist_map_inline:nn {GL,Sp,rO,rU,fgl,fsp,foo,fuu,fkk,fuu,ufkk,uK} {#1{##1}{\usecsname}{\useLetter}}
}

\cs_new:Npn \doallGroups
{
\clist_map_inline:nn {mydeft,mydefu,mydefwh,mydefhat,mydefwt,mydefck,mydefbar} {\doGroups{\csname ##1\endcsname}}
}


\cs_new:Npn \decsyms #1
{
\clist_map_inline:nn {#1} {\expandafter\DeclareMathOperator\csname ##1\endcsname{##1}}
}

\decsyms{Mp,id,SL,Sp,SU,SO,GO,GSO,GU,GSp,PGL,Pic,Lie,Mat,Ker,Hom,Ext,Ind,reg,res,inv,Isom,Det,Tr,Norm,Sym,Span,Stab,Spec,PGSp,PSL,tr,Ad,Br,Ch,Cent,End,Aut,Dvi,Frob,Gal,GL,Gr,DO,ur,vol,ab,Nil,Supp,rank,Sign}

\def\abs#1{\left|{#1}\right|}
\def\norm#1{{\left\|{#1}\right\|}}


% \NewDocumentCommand\inn{m m}{
% \left\langle
% \IfValueTF{#1}{#1}{000}
% ,
% \IfValueTF{#2}{#2}{000}
% \right\rangle
% }
\NewDocumentCommand\cent{o m }{
  \IfValueTF{#1}{
    \mathop{Z}_{#1}{(#2)}}
  {\mathop{Z}{(#2)}}
}


\def\fsl{\mathfrak{sl}}
\def\fsp{\mathfrak{sp}}


%\def\cent#1#2{{\mathrm{Z}_{#1}({#2})}}


\doallAtZ
\doallatz
\doallGreek
\doallGroups
\doallSymbols
\ExplSyntaxOff


% \usepackage{geometry,amsthm,graphics,tabularx,amssymb,shapepar}
% \usepackage{amscd}
% \usepackage{mathrsfs}


\usepackage{diagbox}
% Update the information and uncomment if AMS is not the copyright
% holder.
%\copyrightinfo{2006}{American Mathematical Society}
%\usepackage{nicematrix}
\usepackage{arydshln}

\usepackage{tikz}
\usetikzlibrary{matrix,arrows,positioning,cd,backgrounds}
\usetikzlibrary{decorations.pathmorphing,decorations.pathreplacing}

\usepackage{upgreek}

\usepackage{listings}
\lstset{
    basicstyle=\ttfamily\tiny,
    keywordstyle=\color{black},
    commentstyle=\color{white}, % white comments
    stringstyle=\ttfamily, % typewriter type for strings
    showstringspaces=false,
    breaklines=true,
    emph={Output},emphstyle=\color{blue},
} 

\newcommand{\BA}{{\mathbb{A}}}
%\newcommand{\BB}{{\mathbb {B}}}
\newcommand{\BC}{{\mathbb {C}}}
\newcommand{\BD}{{\mathbb {D}}}
\newcommand{\BE}{{\mathbb {E}}}
\newcommand{\BF}{{\mathbb {F}}}
\newcommand{\BG}{{\mathbb {G}}}
\newcommand{\BH}{{\mathbb {H}}}
\newcommand{\BI}{{\mathbb {I}}}
\newcommand{\BJ}{{\mathbb {J}}}
\newcommand{\BK}{{\mathbb {U}}}
\newcommand{\BL}{{\mathbb {L}}}
\newcommand{\BM}{{\mathbb {M}}}
\newcommand{\BN}{{\mathbb {N}}}
\newcommand{\BO}{{\mathbb {O}}}
\newcommand{\BP}{{\mathbb {P}}}
\newcommand{\BQ}{{\mathbb {Q}}}
\newcommand{\BR}{{\mathbb {R}}}
\newcommand{\BS}{{\mathbb {S}}}
\newcommand{\BT}{{\mathbb {T}}}
\newcommand{\BU}{{\mathbb {U}}}
\newcommand{\BV}{{\mathbb {V}}}
\newcommand{\BW}{{\mathbb {W}}}
\newcommand{\BX}{{\mathbb {X}}}
\newcommand{\BY}{{\mathbb {Y}}}
\newcommand{\BZ}{{\mathbb {Z}}}
\newcommand{\Bk}{{\mathbf {k}}}

\newcommand{\CA}{{\mathcal {A}}}
\newcommand{\CB}{{\mathcal {B}}}
\newcommand{\CC}{{\mathcal {C}}}

\newcommand{\CE}{{\mathcal {E}}}
\newcommand{\CF}{{\mathcal {F}}}
\newcommand{\CG}{{\mathcal {G}}}
\newcommand{\CH}{{\mathcal {H}}}
\newcommand{\CI}{{\mathcal {I}}}
\newcommand{\CJ}{{\mathcal {J}}}
\newcommand{\CK}{{\mathcal {K}}}
\newcommand{\CL}{{\mathcal {L}}}
\newcommand{\CM}{{\mathcal {M}}}
\newcommand{\CN}{{\mathcal {N}}}
\newcommand{\CO}{{\mathcal {O}}}
\newcommand{\CP}{{\mathcal {P}}}
\newcommand{\CQ}{{\mathcal {Q}}}
\newcommand{\CR}{{\mathcal {R}}}
\newcommand{\CS}{{\mathcal {S}}}
\newcommand{\CT}{{\mathcal {T}}}
\newcommand{\CU}{{\mathcal {U}}}
\newcommand{\CV}{{\mathcal {V}}}
\newcommand{\CW}{{\mathcal {W}}}
\newcommand{\CX}{{\mathcal {X}}}
\newcommand{\CY}{{\mathcal {Y}}}
\newcommand{\CZ}{{\mathcal {Z}}}


\newcommand{\RA}{{\mathrm {A}}}
\newcommand{\RB}{{\mathrm {B}}}
\newcommand{\RC}{{\mathrm {C}}}
\newcommand{\RD}{{\mathrm {D}}}
\newcommand{\RE}{{\mathrm {E}}}
\newcommand{\RF}{{\mathrm {F}}}
\newcommand{\RG}{{\mathrm {G}}}
\newcommand{\RH}{{\mathrm {H}}}
\newcommand{\RI}{{\mathrm {I}}}
\newcommand{\RJ}{{\mathrm {J}}}
\newcommand{\RK}{{\mathrm {K}}}
\newcommand{\RL}{{\mathrm {L}}}
\newcommand{\RM}{{\mathrm {M}}}
\newcommand{\RN}{{\mathrm {N}}}
\newcommand{\RO}{{\mathrm {O}}}
\newcommand{\RP}{{\mathrm {P}}}
\newcommand{\RQ}{{\mathrm {Q}}}
%\newcommand{\RR}{{\mathrm {R}}}
\newcommand{\RS}{{\mathrm {S}}}
\newcommand{\RT}{{\mathrm {T}}}
\newcommand{\RU}{{\mathrm {U}}}
\newcommand{\RV}{{\mathrm {V}}}
\newcommand{\RW}{{\mathrm {W}}}
\newcommand{\RX}{{\mathrm {X}}}
\newcommand{\RY}{{\mathrm {Y}}}
\newcommand{\RZ}{{\mathrm {Z}}}

\DeclareMathOperator{\absNorm}{\mathfrak{N}}
\DeclareMathOperator{\Ann}{Ann}
\DeclareMathOperator{\LAnn}{L-Ann}
\DeclareMathOperator{\RAnn}{R-Ann}
\DeclareMathOperator{\ind}{ind}
%\DeclareMathOperator{\Ind}{Ind}



\newcommand{\cod}{{\mathrm{cod}}}
\newcommand{\cont}{{\mathrm{cont}}}
\newcommand{\cl}{{\mathrm{cl}}}
\newcommand{\cusp}{{\mathrm{cusp}}}

\newcommand{\disc}{{\mathrm{disc}}}
\renewcommand{\div}{{\mathrm{div}}}



\newcommand{\Gm}{{\mathbb{G}_m}}



\newcommand{\I}{{\mathrm{I}}}

\newcommand{\Jac}{{\mathrm{Jac}}}
\newcommand{\PM}{{\mathrm{PM}}}


\newcommand{\new}{{\mathrm{new}}}
\newcommand{\NS}{{\mathrm{NS}}}
\newcommand{\N}{{\mathrm{N}}}

\newcommand{\ord}{{\mathrm{ord}}}

%\newcommand{\rank}{{\mathrm{rank}}}

\newcommand{\rk}{{\mathrm{k}}}
\newcommand{\rr}{{\mathrm{r}}}
\newcommand{\rh}{{\mathrm{h}}}

\newcommand{\Sel}{{\mathrm{Sel}}}
\newcommand{\Sim}{{\mathrm{Sim}}}

\newcommand{\wt}{\widetilde}
\newcommand{\wh}{\widehat}
\newcommand{\pp}{\frac{\partial\bar\partial}{\pi i}}
\newcommand{\pair}[1]{\langle {#1} \rangle}
\newcommand{\wpair}[1]{\left\{{#1}\right\}}
\newcommand{\intn}[1]{\left( {#1} \right)}
\newcommand{\sfrac}[2]{\left( \frac {#1}{#2}\right)}
\newcommand{\ds}{\displaystyle}
\newcommand{\ov}{\overline}
\newcommand{\incl}{\hookrightarrow}
\newcommand{\lra}{\longrightarrow}
\newcommand{\imp}{\Longrightarrow}
%\newcommand{\lto}{\longmapsto}
\newcommand{\bs}{\backslash}

\newcommand{\cover}[1]{\widetilde{#1}}

\renewcommand{\vsp}{{\vspace{0.2in}}}

\newcommand{\Norma}{\operatorname{N}}
\newcommand{\Ima}{\operatorname{Im}}
\newcommand{\con}{\textit{C}}
\newcommand{\gr}{\operatorname{gr}}
\newcommand{\ad}{\operatorname{ad}}
\newcommand{\der}{\operatorname{der}}
\newcommand{\dif}{\operatorname{d}\!}
\newcommand{\pro}{\operatorname{pro}}
\newcommand{\Ev}{\operatorname{Ev}}
% \renewcommand{\span}{\operatorname{span}} \span is an innernal command.
%\newcommand{\degree}{\operatorname{deg}}
\newcommand{\Invf}{\operatorname{Invf}}
\newcommand{\Inv}{\operatorname{Inv}}
\newcommand{\slt}{\operatorname{SL}_2(\mathbb{R})}
%\newcommand{\temp}{\operatorname{temp}}
%\newcommand{\otop}{\operatorname{top}}
\renewcommand{\small}{\operatorname{small}}
\newcommand{\HC}{\operatorname{HC}}
\newcommand{\lef}{\operatorname{left}}
\newcommand{\righ}{\operatorname{right}}
\newcommand{\Diff}{\operatorname{DO}}
\newcommand{\diag}{\operatorname{diag}}
\newcommand{\sh}{\varsigma}
\newcommand{\sch}{\operatorname{sch}}
%\newcommand{\oleft}{\operatorname{left}}
%\newcommand{\oright}{\operatorname{right}}
\newcommand{\open}{\operatorname{open}}
\newcommand{\sgn}{\operatorname{sgn}}
\newcommand{\triv}{\operatorname{triv}}
\newcommand{\Sh}{\operatorname{Sh}}
\newcommand{\oN}{\operatorname{N}}

\newcommand{\oc}{\operatorname{c}}
\newcommand{\od}{\operatorname{d}}
\newcommand{\os}{\operatorname{s}}
\newcommand{\ol}{\operatorname{l}}
\newcommand{\oL}{\operatorname{L}}
\newcommand{\oJ}{\operatorname{J}}
\newcommand{\oH}{\operatorname{H}}
\newcommand{\oO}{\operatorname{O}}
\newcommand{\oS}{\operatorname{S}}
\newcommand{\oR}{\operatorname{R}}
\newcommand{\oT}{\operatorname{T}}
%\newcommand{\rU}{\operatorname{U}}
\newcommand{\oZ}{\operatorname{Z}}
\newcommand{\oD}{\textit{D}}
\newcommand{\oW}{\textit{W}}
\newcommand{\oE}{\operatorname{E}}
\newcommand{\oP}{\operatorname{P}}
\newcommand{\PD}{\operatorname{PD}}
\newcommand{\oU}{\operatorname{U}}

\newcommand{\g}{\mathfrak g}
\newcommand{\gC}{{\mathfrak g}_{\C}}
\renewcommand{\k}{\mathfrak k}
\newcommand{\h}{\mathfrak h}
\newcommand{\p}{\mathfrak p}
%\newcommand{\q}{\mathfrak q}
\renewcommand{\a}{\mathfrak a}
\renewcommand{\b}{\mathfrak b}
\renewcommand{\c}{\mathfrak c}
\newcommand{\n}{\mathfrak n}
\renewcommand{\u}{\mathfrak u}
\renewcommand{\v}{\mathfrak v}
\newcommand{\e}{\mathfrak e}
\newcommand{\f}{\mathfrak f}
\renewcommand{\l}{\mathfrak l}
\renewcommand{\t}{\mathfrak t}
\newcommand{\s}{\mathfrak s}
\renewcommand{\r}{\mathfrak r}
\renewcommand{\o}{\mathfrak o}
\newcommand{\m}{\mathfrak m}
\newcommand{\z}{\mathfrak z}
%\renewcommand{\sl}{\mathfrak s \mathfrak l}
\newcommand{\gl}{\mathfrak g \mathfrak l}


\newcommand{\re}{\mathrm e}

\renewcommand{\rk}{\mathrm k}

\newcommand{\Z}{\mathbb{Z}}
\DeclareDocumentCommand{\C}{}{\mathbb{C}}
\newcommand{\R}{\mathbb R}
\newcommand{\Q}{\mathbb Q}
\renewcommand{\H}{\mathbb{H}}
%\newcommand{\N}{\mathbb{N}}
\newcommand{\K}{\mathbb{K}}
%\renewcommand{\S}{\mathbf S}
\newcommand{\M}{\mathbf{M}}
\newcommand{\A}{\mathbb{A}}
\newcommand{\B}{\mathbf{B}}
%\renewcommand{\G}{\mathbf{G}}
\newcommand{\V}{\mathbf{V}}
\newcommand{\W}{\mathbf{W}}
\newcommand{\F}{\mathbf{F}}
\newcommand{\E}{\mathbf{E}}
%\newcommand{\J}{\mathbf{J}}
\renewcommand{\H}{\mathbf{H}}
\newcommand{\X}{\mathbf{X}}
\newcommand{\Y}{\mathbf{Y}}
%\newcommand{\RR}{\mathcal R}
\newcommand{\FF}{\mathcal F}
%\newcommand{\BB}{\mathcal B}
\newcommand{\HH}{\mathcal H}
%\newcommand{\UU}{\mathcal U}
%\newcommand{\MM}{\mathcal M}
%\newcommand{\CC}{\mathcal C}
%\newcommand{\DD}{\mathcal D}
\def\eDD{\mathrm{d}^{e}}
\def\DD{\nabla}
\def\DDc{\boldsymbol{\nabla}}
\def\gDD{\nabla^{\mathrm{gen}}}
\def\gDDc{\boldsymbol{\nabla}^{\mathrm{gen}}}
%\newcommand{\OO}{\mathcal O}
%\newcommand{\ZZ}{\mathcal Z}
\newcommand{\ve}{{\vee}}
\newcommand{\aut}{\mathcal A}
\newcommand{\ii}{\mathbf{i}}
\newcommand{\jj}{\mathbf{j}}
\newcommand{\kk}{\mathbf{k}}

\newcommand{\la}{\langle}
\newcommand{\ra}{\rangle}
\newcommand{\bp}{\bigskip}
\newcommand{\be}{\begin {equation}}
\newcommand{\ee}{\end {equation}}

\newcommand{\LRleq}{\stackrel{LR}{\leq}}

\numberwithin{equation}{section}


\def\flushl#1{\ifmmode\makebox[0pt][l]{${#1}$}\else\makebox[0pt][l]{#1}\fi}
\def\flushr#1{\ifmmode\makebox[0pt][r]{${#1}$}\else\makebox[0pt][r]{#1}\fi}
\def\flushmr#1{\makebox[0pt][r]{${#1}$}}


%\theoremstyle{Theorem}
% \newtheorem*{thmM}{Main Theorem}
% \crefformat{thmM}{main theorem}
% \Crefformat{thmM}{Main Theorem}
\newtheorem*{thm*}{Theorem}
\newtheorem{thm}{Theorem}[section]
\newtheorem{thml}[thm]{Theorem}
\newtheorem{lem}[thm]{Lemma}
\newtheorem{obs}[thm]{Observation}
\newtheorem{lemt}[thm]{Lemma}
\newtheorem*{lem*}{Lemma}
\newtheorem{whyp}[thm]{Working Hypothesis}
\newtheorem{prop}[thm]{Proposition}
\newtheorem{prpt}[thm]{Proposition}
\newtheorem{prpl}[thm]{Proposition}
\newtheorem{cor}[thm]{Corollary}
%\newtheorem*{prop*}{Proposition}
\newtheorem{claim}{Claim}
\newtheorem*{claim*}{Claim}
%\theoremstyle{definition}
\newtheorem{defn}[thm]{Definition}
\newtheorem{dfnl}[thm]{Definition}
\newtheorem*{IndH}{Induction Hypothesis}

\newtheorem*{eg*}{Example}
\newtheorem{eg}[thm]{Example}

\theoremstyle{remark}
\newtheorem*{remark}{Remark}
\newtheorem*{remarks}{Remarks}


\def\cpc{\sigma}
\def\ccJ{\epsilon\dotepsilon}
\def\ccL{c_L}

\def\wtbfK{\widetilde{\bfK}}
%\def\abfV{\acute{\bfV}}
\def\AbfV{\acute{\bfV}}
%\def\afgg{\acute{\fgg}}
%\def\abfG{\acute{\bfG}}
\def\abfV{\bfV'}
\def\afgg{\fgg'}
\def\abfG{\bfG'}

\def\half{{\tfrac{1}{2}}}
\def\ihalf{{\tfrac{\mathbf i}{2}}}
\def\slt{\fsl_2(\bC)}
\def\sltr{\fsl_2(\bR)}

% \def\Jslt{{J_{\fslt}}}
% \def\Lslt{{L_{\fslt}}}
\def\slee{{
\begin{pmatrix}
 0 & 1\\
 0 & 0
\end{pmatrix}
}}
\def\slff{{
\begin{pmatrix}
 0 & 0\\
 1 & 0
\end{pmatrix}
}}\def\slhh{{
\begin{pmatrix}
 1 & 0\\
 0 & -1
\end{pmatrix}
}}
\def\sleei{{
\begin{pmatrix}
 0 & i\\
 0 & 0
\end{pmatrix}
}}
\def\slxx{{\begin{pmatrix}
-\ihalf & \half\\
\phantom{-}\half & \ihalf
\end{pmatrix}}}
% \def\slxx{{\begin{pmatrix}
% -\sqrt{-1}/2 & 1/2\\
% 1/2 & \sqrt{-1}/2
% \end{pmatrix}}}
\def\slyy{{\begin{pmatrix}
\ihalf & \half\\
\half & -\ihalf
\end{pmatrix}}}
\def\slxxi{{\begin{pmatrix}
+\half & -\ihalf\\
-\ihalf & -\half
\end{pmatrix}}}
\def\slH{{\begin{pmatrix}
   0   & -\mathbf i\\
\mathbf i & 0
\end{pmatrix}}
}

\ExplSyntaxOn
\clist_map_inline:nn {J,L,C,X,Y,H,c,e,f,h,}{
  \expandafter\def\csname #1slt\endcsname{{\mathring{#1}}}}
\ExplSyntaxOff


\def\Mop{\fT}

\def\fggJ{\fgg_J}
\def\fggJp{\fgg'_{J'}}

\def\NilGC{\Nil_{\bfG}(\fgg)}
\def\NilGCp{\Nil_{\bfG'}(\fgg')}
\def\Nilgp{\Nil_{\fgg'_{J'}}}
\def\Nilg{\Nil_{\fgg_{J}}}
%\def\NilP'{\Nil_{\fpp'}}
\def\nNil{\Nil^{\mathrm n}}
\def\eNil{\Nil^{\mathrm e}}


\NewDocumentCommand{\NilP}{t'}{
\IfBooleanTF{#1}{\Nil_{\fpp'}}{\Nil_\fpp}
}

\def\KS{\mathsf{KS}}
\def\MM{\bfM}
\def\MMP{M}

\NewDocumentCommand{\KTW}{o g}{
  \IfValueTF{#2}{
    \left.\varsigma_{\IfValueT{#1}{#1}}\right|_{#2}}{
    \varsigma_{\IfValueT{#1}{#1}}}
}
\def\IST{\rho}
\def\tIST{\trho}

\NewDocumentCommand{\CHI}{o g}{
  \IfValueTF{#1}{
    {\chi}_{\left[#1\right]}}{
    \IfValueTF{#2}{
      {\chi}_{\left(#2\right)}}{
      {\chi}}
  }
}
\NewDocumentCommand{\PR}{g}{
  \IfValueTF{#1}{
    \mathop{\pr}_{\left(#1\right)}}{
    \mathop{\pr}}
}
\NewDocumentCommand{\XX}{g}{
  \IfValueTF{#1}{
    {\cX}_{\left(#1\right)}}{
    {\cX}}
}
\NewDocumentCommand{\PP}{g}{
  \IfValueTF{#1}{
    {\fpp}_{\left(#1\right)}}{
    {\fpp}}
}
\NewDocumentCommand{\LL}{g}{
  \IfValueTF{#1}{
    {\bfL}_{\left(#1\right)}}{
    {\bfL}}
}
\NewDocumentCommand{\ZZ}{g}{
  \IfValueTF{#1}{
    {\cZ}_{\left(#1\right)}}{
    {\cZ}}
}

\NewDocumentCommand{\WW}{g}{
  \IfValueTF{#1}{
    {\bfW}_{\left(#1\right)}}{
    {\bfW}}
}




\def\gpi{\wp}
\NewDocumentCommand\KK{g}{
\IfValueTF{#1}{K_{(#1)}}{K}}
% \NewDocumentCommand\OO{g}{
% \IfValueTF{#1}{\cO_{(#1)}}{K}}
\NewDocumentCommand\XXo{d()}{
\IfValueTF{#1}{\cX^\circ_{(#1)}}{\cX^\circ}}
\def\bfWo{\bfW^\circ}
\def\bfWoo{\bfW^{\circ \circ}}
\def\bfWg{\bfW^{\mathrm{gen}}}
\def\Xg{\cX^{\mathrm{gen}}}
\def\Xo{\cX^\circ}
\def\Xoo{\cX^{\circ \circ}}
\def\fppo{\fpp^\circ}
\def\fggo{\fgg^\circ}
\NewDocumentCommand\ZZo{g}{
\IfValueTF{#1}{\cZ^\circ_{(#1)}}{\cZ^\circ}}

% \ExplSyntaxOn
% \NewDocumentCommand{\bcO}{t' E{^_}{{}{}}}{
%   \overline{\cO\sb{\use_ii:nn#2}\IfBooleanTF{#1}{^{'\use_i:nn#2}}{^{\use_i:nn#2}}
%   }
% }
% \ExplSyntaxOff

\NewDocumentCommand{\bcO}{t'}{
  \overline{\cO\IfBooleanT{#1}{'}}}

\NewDocumentCommand{\oliftc}{g}{
\IfValueTF{#1}{\boldsymbol{\vartheta} (#1)}{\boldsymbol{\vartheta}}
}
\NewDocumentCommand{\oliftr}{g}{
\IfValueTF{#1}{\vartheta_\bR(#1)}{\vartheta_\bR}
}
\NewDocumentCommand{\olift}{g}{
\IfValueTF{#1}{\vartheta(#1)}{\vartheta}
}
% \NewDocumentCommand{\dliftv}{g}{
% \IfValueTF{#1}{\ckvartheta(#1)}{\ckvartheta}
% }
\def\dliftv{\vartheta}
\NewDocumentCommand{\tlift}{g}{
\IfValueTF{#1}{\wtvartheta(#1)}{\wtvartheta}
}

\def\slift{\cL}

\def\BB{\bB}


\def\thetaO#1{\vartheta\left(#1\right)}

\def\bbThetav{\check{\mathbbold{\Theta}}}
\def\Thetav{\check{\Theta}}
\def\thetav{\check{\theta}}

\DeclareDocumentCommand{\NN}{g}{
\IfValueTF{#1}{\fN(#1)}{\fN}
}
\DeclareDocumentCommand{\RR}{m m}{
\fR({#1},{#2})
}

%\DeclareMathOperator*{\sign}{Sign}

\NewDocumentCommand{\lsign}{m}{
{}^l\mathrm{Sign}(#1)
}



\NewDocumentCommand\lnn{t+ t- g}{
  \IfBooleanTF{#1}{{}^l n^+\IfValueT{#3}{(#3)}}{
    \IfBooleanTF{#2}{{}^l n^-\IfValueT{#3}{(#3)}}{}
  }
}


% Fancy bcO, support feature \bcO'^a_b = \overline{\cO'^a_b}
\makeatletter
\def\bcO{\def\O@@{\cO}\@ifnextchar'\@Op\@Onp}
\def\@Opnext{\@ifnextchar^\@Opsp\@Opnsp}
\def\@Op{\afterassignment\@Opnext\let\scratch=}
\def\@Opnsp{\def\O@@{\cO'}\@Otsb}
\def\@Onp{\@ifnextchar^\@Onpsp\@Otsb}
\def\@Opsp^#1{\def\O@@{\cO'^{#1}}\@Otsb}
\def\@Onpsp^#1{\def\O@@{\cO^{#1}}\@Otsb}
\def\@Otsb{\@ifnextchar_\@Osb{\@Ofinalnsb}}
\def\@Osb_#1{\overline{\O@@_{#1}}}
\def\@Ofinalnsb{\overline{\O@@}}

% Fancy \command: \command`#1 will translate to {}^{#1}\bfV, i.e. superscript on the
% lift conner.

\def\defpcmd#1{
  \def\nn@tmp{#1}
  \def\nn@np@tmp{@np@#1}
  \expandafter\let\csname\nn@np@tmp\expandafter\endcsname \csname\nn@tmp\endcsname
  \expandafter\def\csname @pp@#1\endcsname`##1{{}^{##1}{\csname @np@#1\endcsname}}
  \expandafter\def\csname #1\endcsname{\,\@ifnextchar`{\csname
      @pp@#1\endcsname}{\csname @np@#1\endcsname}}
}

% \def\defppcmd#1{
% \expandafter\NewDocumentCommand{\csname #1\endcsname}{##1 }{}
% }



\defpcmd{bfV}
\def\KK{\bfK}\defpcmd{KK}
\defpcmd{bfG}
\def\A{\!A}\defpcmd{A}
\def\K{\!K}\defpcmd{K}
\def\G{G}\defpcmd{G}
\def\J{\!J}\defpcmd{J}
\def\L{\!L}\defpcmd{L}
\def\eps{\epsilon}\defpcmd{eps}
\def\pp{p}\defpcmd{pp}
\defpcmd{wtK}
\makeatother

\def\fggR{\fgg_\bR}
\def\rmtop{{\mathrm{top}}}
\def\dimo{\dim^\circ}

\NewDocumentCommand\LW{g}{
\IfValueTF{#1}{L_{W_{#1}}}{L_{W}}}
%\def\LW#1{L_{W_{#1}}}
\def\JW#1{J_{W_{#1}}}

\def\floor#1{{\lfloor #1 \rfloor}}

\def\KSP{K}
\def\UU{\rU}
\def\UUC{\rU_\bC}
\def\tUUC{\widetilde{\rU}_\bC}
\def\OmegabfW{\Omega_{\bfW}}


\def\BB{\bB}


\def\thetaO#1{\vartheta\left(#1\right)}

\def\Thetav{\check{\Theta}}
\def\thetav{\check{\theta}}

\def\Thetab{\bar{\Theta}}

\def\cKaod{\cK^{\mathrm{aod}}}

%G_V's or G
%%%%%%%%%%%%%%%%%%%%%%%%%%%
% \def\GVr{G_{\bfV}}
% \def\tGVr{\wtG_{\bfV}}
% \def\GVpr{G_{\bfV'}}
% \def\tGVpr{\wtG_{\bfV'}}
% \def\GVpr{G_{\abfV}}
% \def\tGVar{\wtG_{\abfV}}
% \def\GV{\bfG_{\bfV}}
% \def\GVp{\bfG_{\bfV'}}
% \def\KVr{K_{\bfV}}
% \def\tKVr{\wtK_{\bfV}}
% \def\KV{\bfK_{\bfV}}
% \def\KaV{\bfK_{\acute{V}}}

% \def\KV{\bfK}
% \def\KaV{\acute{\bfK}}
% \def\acO{\acute{\cO}}
% \def\asO{\acute{\sO}}
%%%%%%%%%%%%%%%%%%%%%%%%%%%
\def\GVr{G}
\def\tGVr{\wtG}
\def\GVpr{G'}
\def\tGVpr{\widetilde{G'}}
\def\GVar{G'}
\def\tGVar{\wtG'}
\def\GV{\bfG}
\def\GVp{\bfG'}
\def\KVr{K_{\bfV}}
\def\tKVr{\wtK_{\bfV}}
\def\KV{\bfK_{\bfV}}
\def\KaV{\bfK_{\acute{V}}}

\def\KV{\bfK}
\def\KaV{\acute{\bfK}}
\def\acO{{\cO'}}
\def\asO{{\sO'}}

\DeclareMathOperator{\sspan}{span}

%%%%%%%%%%%%%%%%%%%%%%%%%%%%

\def\sp{{\mathrm{sp}}}

\def\bfLz{\bfL_0}
\def\sOpe{\sO^\perp}
\def\sOpeR{\sO^\perp_\bR}
\def\sOR{\sO_\bR}

\def\ZX{\cZ_{X}}
\def\gdliftv{\vartheta}
\def\gdlift{\vartheta^{\mathrm{gen}}}
\def\bcOp{\overline{\cO'}}
\def\bsO{\overline{\sO}}
\def\bsOp{\overline{\sO'}}
\def\bfVpe{\bfV^\perp}
\def\bfEz{\bfE_0}
\def\bfVn{\bfV^-}
\def\bfEzp{\bfE'_0}

\def\totimes{\widehat{\otimes}}
\def\dotbfV{\dot{\bfV}}

\def\aod{\mathrm{aod}}
\def\unip{\mathrm{unip}}


\def\ssP{{\ddot\cP}}
\def\ssD{\ddot{\bfD}}
\def\ssdd{\ddot{\bfdd}}
\def\phik{\phi_{\fkk}}
\def\phikp{\phi_{\fkk'}}
%\def\bbfK{\breve{\bfK}}
\def\bbfK{\wtbfK}
\def\brrho{\breve{\rho}}

\def\whAX{\widehat{A_X}}
\def\mktvvp{\varsigma_{{\bf V},{\bf V}'}}

\def\Piunip{\Pi^{\mathrm{unip}}}
\def\cf{\emph{cf.} }
\def\Groth{\mathrm{Groth}}
\def\Irr{\mathrm{Irr}}

\def\edrc{\mathrm{DRC}^{\mathrm e}}
\def\drc{\mathrm{DRC}}
\def\LS{\mathrm{LS}}
\def\Unip{\mathrm{Unip}}

\newcommand{\noticed}{noticed }
\newcommand{\ess}{essential }

% Ytableau tweak
\makeatletter
\pgfkeys{/ytableau/options,
  noframe/.default = false,
  noframe/.is choice,
  noframe/true/.code = {%
    \global\let\vrule@YT=\vrule@none@YT
    \global\let\hrule@YT=\hrule@none@YT
  },
  noframe/false/.code = {%
    \global\let\vrule@YT=\vrule@normal@YT
    \global\let\hrule@YT=\hrule@normal@YT
  },
  noframe/on/.style = {noframe/true},
  noframe/off/.style = {noframe/false},
}
\makeatother 

\begin{document}


\title[]{Combinatorics for unipotent representations}

\author [D. Barbasch] {Dan M. Barbasch}
\address{the Department of Mathematics\\
  310 Malott Hall, Cornell University, Ithaca, New York 14853 }
\email{dmb14@cornell.edu}

\author [J.-J. Ma] {Jia-jun Ma}
\address{School of Mathematical Sciences\\
  Shanghai Jiao Tong University\\
  800 Dongchuan Road, Shanghai, 200240, China} \email{hoxide@sjtu.edu.cn}

\author [B. Sun] {Binyong Sun}
% MCM, HCMS, HLM, CEMS, UCAS,
\address{Academy of Mathematics and Systems Science\\
  Chinese Academy of Sciences\\
  Beijing, 100190, China} \email{sun@math.ac.cn}

\author [C.-B. Zhu] {Chen-Bo Zhu}
\address{Department of Mathematics\\
  National University of Singapore\\
  10 Lower Kent Ridge Road, Singapore 119076} \email{matzhucb@nus.edu.sg}




\subjclass[2000]{22E45, 22E46} \keywords{orbit method, unitary dual, unipotent
  representation, classical group, theta lifting, moment map}

% \thanks{Supported by NSFC Grant 11222101}
\maketitle


\tableofcontents
\section{Relevent nilpotent orbits}

Let
\[\cO=(C_{2a},C_{2a-1}, \cdots, C_1,C_0,C_{-1}=0)\]
be a special orbit.  The dual orbit
\[ \ckcO = [R_{2a}\geq R_{2a-1}\geq \cdots \geq R_1\geq R_0].\] The $\ckcO$ is
given by add a box on the longest row of $\cO^T$ and then do $B$-collapse.


\begin{defn}
  Suppose $\bfG$ is type $C$.  An special orbit $\cO$ is called purely even if
  $\ckcO$ only has odd rows.
  This is equivalent to the condition on columns of $\cO$ such that (see \Cref{lem:C.even})
  \begin{enumC}
  \item $C_{2i} \equiv C_{2i-1} \pmod{2}$; 
  \item 
   If $C_{2i-1}$ is even,  $C_{2i-1} > C_{2i-2}$; 
  \item
    If $C_{2i}$ is odd, $C_{2i}=C_{2i-1}$. 
  \end{enumC}
  We call a purely even orbit $\cO$ is stablelly trivial when only even columns
  occures. This is equivalent to  
  \[
    C_{2i} \equiv C_{2i-1} \equiv 0 \pmod{2}  \text{ and } C_{2i-1}>
    C_{2i-2} \quad \forall i.
  \]
  {\color{red} In this case all LS in the unipotent representations are
    irreducible. }
  This is also equivalent to
  \[
    R_{2i}>R_{2i-1} \quad \forall i
  \]

  We call a purely even orbit $\cO$ is theta-admissible when each row columns
  appears at most $2$-times.
  {\color{red} All unipotent representations of these orbits are expected to
    constructed by theta correspondence. Moreover, they are parameterized by the
    their local systems. 
  }
  This is also equivalent to each row lenght appears at most $3$-times in
  $\ckcO$. 
\end{defn}

Suppose $\cO$ is purely even. 
The dual orbit 
$\ckcO = [R_{2a}\geq R_{2a-1}\geq \cdots \geq R_1\geq R_0]$
is calculated in the following way: 
\[
  [R_{2i},R_{2i-1}] = \begin{cases}
    [C_0+1, 0] & i=0\\
    [C_{2i}+1, C_{2i-1}-1] & i>0, C_{2i} \text{ is even} \\
    [C_{2i},C_{2i-1}] & i>0, C_{2i} \text{ is odd}
  \end{cases}
\]


\begin{defn}
  Suppose $\bfG$ is type $D$.  An special orbit
  \[
    \cO=(C_{2a-1}\ge\dots \ge C_0\geq C_{-1}=0) \quad C_{2i}\equiv
    C_{2i-1}\pmod{2}, C_{2a-1} \equiv 0 \pmod{2}
  \]
  is called purley even if
  $\ckcO$ only has odd rows.
  This is equivalent to the condition on columns of $\cO$ such that (see \Cref{lem:C.even})
  \begin{enumC}
  \item $C_{2i} \equiv C_{2i-1} \pmod{2}$; 
  \item 
   If $C_{2i-1}$ is even,  $C_{2i-1} > C_{2i-2}$; 
  \item
    If $C_{2i}$ is odd, $C_{2i}=C_{2i-1}$. 
  \end{enumC}
  Note that this is the same condition as the type C case. 
\end{defn}


Suppose $\cO$ is purely even. 
The dual orbit 
$\ckcO = [R_{2a-1}\geq \cdots \geq R_1\geq R_0, R_{-1}=0]$
is calculated in the following way: 
\[
  [R_{2i},R_{2i-1}] = \begin{cases}
    [C_0+1, 0] & i=0\\
    [0, C_{2a-1}-1] & i = a \\
    [C_{2i}+1, C_{2i-1}-1] & a> i>0, C_{2i} \text{ is even} \\
    [C_{2i},C_{2i-1}] & a>i>0, C_{2i} \text{ is odd}
  \end{cases}
\]

\section{Parameterize of Unipotent representations}
We fix an abstract complex Cartan subgroup $\bfH_a$ and $\fhh_a$ in $\bfG$ and a
set of simple roots $\Pi_a$.  Let $\cP(\bfG)$ be the set of all Langlands
parameters of $G$-modules with character $\rho$ (i.e. the infinitesimal
character of the trivial representation). For $\gamma\in \cP(\bfG)$, let
$\cL(\gamma)$, $\cS(\gamma)$ and $\Phi_\gamma$ be the corresponding Langlands
quotient, standard module and coherent family such that
$\Phi_\gamma(\rho) = \cL(\gamma)$. Let $\cM(\bfG)$ be the span of $\cL(\gamma)$.
Let $\set{\bB}$ be the set of all blocks. Then $\cP(\bfG) = \bigsqcup_\cB \cB$.
The Weyl group $W = W(G)$ acts on $\cM(\bfG)$ by coherent continuation.  Let
$\cM_{\cB}$ be the submodule of $\cM(\bfG)$ spand by $\gamma\in\cB$, then
\[
  \cM(\bfG) = \bigoplus_\cB \cM_{\cB}
\]
Let $\tau(\gamma)\subset \Pi_a$ be the $\tau$-invariant of $\gamma$.

Let $\ckcO$ be even orbit. $\lambda= \half \ckhh$.  Define
\[
  S(\lambda) = \set{\alpha\in \Pi_a| \inn{\alpha}{\lambda}=0}.
\]
Let $\cP_{\lambda}(\bfG)$ be the set of all Langlands parameters with
infinitesimal character $\lambda$. Let $T_{\lambda,\rho}$ be the translation
functor.  Let
\[
  \cB(S) = \set{\gamma\in \cB|S\cap \tau(\gamma)=\emptyset}
\]
and
\[
  \cP(\bfG,S) = \bigsqcup_{\cB} \cB(S)
\]


Then
\[
  \begin{tikzcd}[row sep=0em]
    \cP(\bfG,S) \ar[r] & \cP_{\lambda}(\bfG)\\
    \gamma \ar[r, maps to]& T_{\lambda, \rho}(\gamma)
  \end{tikzcd}
\]

Let $\cO$ be a complex nilpotent orbit in $\fgg$.  Let
\[
  \cB(S,\cO) = \set{\gamma\in \cB(S)|\AVC(\cL(\gamma))\subset \bcO}
\]

Let
\[
  \begin{aligned}
    m_S(\sigma) &= [\sigma: \Ind_{W(S)}^{W}\bfone]\\
    m_\cB(\sigma)& = [\sigma: \cM_\cB]
  \end{aligned}
\]

Barbasch \cite[Theorem~9.1]{B10} established the following theorem.
\begin{thm}
  \[
    |\cB(S,\cO)| = \sum_{\sigma} m_\cB(\sigma)m_S(\sigma)
  \]
  Here $\sigma\times \sigma$ running over the $W\times W$ appears in the double
  cell $\cC(\cO)$.
\end{thm}
\begin{proof}
  We need to take the graded module of $\cM(\bfG)$ with respect to the
  $\LRleq$. By abuse of notation, we identify the basis $\cP(\bfG)$ with its
  image in the graded module.  Note that $S\cap \tau(\lambda)=\emptyset$ if and
  only if $W(S)$ acts on $\gamma$ trivially by \cite[Lemma~14.7]{V4}.  On the
  other hand, by \cite[Theorem~14.10, and page 58]{V4},
  $\AVC(\cL(\gamma))\subset \bcO$ only if $\gamma$ generate a $W$-module in the
  double cell of $\cO$.
\end{proof}

Now assume $S=S(\lambda)$. By \cite[Cor~5.30 b) and c)]{BVUni},
$[\sigma: \Ind_{W(S)}^{W}\bfone]=[\bfone|_{W(S)}:\sigma]\leq 1$.

\section{Combinatorics of Weyl group representations}
The irreducible representations of $S_n$ are parameterized by Young diagrams.
We use the notation that the trivial representation corresponds to a row and the
sign representation corresponds to a column.
\[
  triv \leftrightarrow \ytableaushort{n \ b o x e s} \quad \sgn \leftrightarrow
  \ytableaushort{n,\ ,b,o,x,e,s}
\]
\trivial{ This notation coincide with the Springer correspondence, where a row
  of boxes represents the regular nilpotent orbit and the corresponding
  representation is the trivial representation.  The trivial orbit yeilds the
  sign representation.  }

Under the above paramterization, $\tau\mapsto \tau\otimes \sgn$ is described by
the transpose of Young diagram. The branching rule is given by
Littlewoods-Richardson.  Let $\cK = \bigoplus_n \Groth(S_n)$ be the graded ring
of the Grothendieck groups of $S_n$.  This yeilds well defined ring structure on
the graded algebra
\[
  \mu \nu :=\Ind_{S_{|\mu|}\times S_{|\nu|}}^{S_{|\mu|+|\nu|}}\mu\otimes \nu.
\]

Notation, we will use $[r_1, r_2, \cdots, r_k]$ or $(c_1,c_2, \cdots, c_k)$
denote the Young diagram, where $\set{r_i}$ denote the lengthes of its rows and
$\set{c_i}$ denote the lengthes of its columns.

Let $W_n$ denote the Weyl group of type BC: $W_n = S_n \ltimes \set{\pm 1}^n$.
Now $\Irr(W_n)$ is paramterized by a pair of Young diagram, we use the notation
$\mu\times \nu$.

As the tensor algebra, $\bigoplus_n \Groth(W_n)\cong \cK\otimes \cK$ via
$\mu\times \nu \mapsto \mu \otimes \nu$.
   
The branching rule is given by:
$(\mu_1\times \nu_1)(\mu_2\times \nu_2) = (\mu_1\mu_2)\times (\nu_1 \nu_2)$.
The sgn repersentation of $W_n$ is parameterized by $\emptyset\times (n)$.  From
the definitioin of the identification of bipartition with $\Irr(W_n)$, we also
have
\[
  (\mu\times \nu) \otimes \sgn = \nu^t\times \mu^t.
\]

We also will use the following branching formula:
\[
  \Ind_{S_n}^{W_n} \triv = \sum_{\substack{a,b,\\ a+b=n}} [a]\times [b],
\]
and
\begin{equation}\label{eq:SW2}
  \Ind_{S_n}^{W_n} \sgn = \sum_{\substack{a,b,\\ a+b=n}}
  (a)\times (b).
\end{equation}
\trivial{ Sketch of the proof of the first formula, the dimension of LHS is
  $n!2^n/n! = 2^n$.  On the other hand, by Mackey theory,
  $[LHS:\Ind_{W_a\times W_b}^{W_n} \triv \otimes \chi ] \geq 1$ since
  $S_n\cap W_a\times W_b = S_a\times S_b$.  Therefore, $LHS\supset RHS$.  On the
  other hand, dimension of RHS is $\sum_{a} n!/a!b! = 2^n$.  This finished the
  proof.  The proof of the second formula is similar. One also can obtain it by
  tensoring with $\sgn$ of the first formula.  }

 

\section{Combinatorics for type C}


\subsection{The coherent continuation representation of type C}
The detail of the coherent continuation representation was discussed in
\cite[p221 (7)]{Mc}.

\trivial{ In McGovern's paper, the coherent continuation representation is
  described as:
  \[
    \sum_{t,s,a,b}\Ind_{W_t\times (W_s\ltimes W(A_1)^s)\times W_a\times
      W_b}^{W_{t+2s+a+b}} \sgn\otimes (\triv \otimes \sgn)\otimes \triv\otimes
    \triv
  \]
  Now \eqref{eq:CC.C} was obtained by the following branching formula:
  \cite[p220 (6)]{Mc}
  \[
    I_n:= \Ind_{(W_s\ltimes W(A_1)^s)}^{W_{2s}}\triv\otimes \sgn = \sum
    \lambda\times \lambda
  \]
  where $\lambda$ running over all Young diagrams of size $s$.  As McGovern
  claimed the proof of the above formula is similar to Barbasch's proof of
  \cite[Lemma~4.1]{B.W}:
  \[
    \Ind_{W_n}^{S_{2n}} \triv = \sum \sigma \quad \text{where $\sigma$ has even
      rows only}.
  \]

  Sketch of the proof (use branching rule and dimension counting): Note that
  $\dim I_n = \frac{(2p)! 2^{2p}}{p! 2^{2p}} = (2p)!/p!  =\sum_\lambda \dim
  \lambda\times \lambda$ (For the last equality:
  $\dim \lambda\times \lambda = (2p)! (\dim \lambda)^2/(p!)^2$ where
  $\dim \lambda$ is the dimension of $S_n$ representation determined by
  $\lambda$; But $\sum (\dim \lambda)^2 = p!$).  On the other hand,
  $H :=W_s\ltimes W(A_1)^s\cap W_s\times W_s = \Delta W_s \subset W_{2s}$.
  $\triv \otimes \sgn|_H = \sgn$ of $\Delta W_s$ Therefore,
  $\lambda\times \lambda$ appears in $I_n$ by Mackey formula. Now by dimension
  counting, we get the formula.  }

The Cartan subgroups are parametrized by three integers $(a,2s,b)$ where
$a+2s+b=n.$ For the coherent continuation representation we need four integers
$(t,2s,p,q),$ satisfying $t+2s+p+q=n.$ The corresponding representation is
\begin{equation}
  \sum_{\substack{p,q,t,s,\\
      \tau\in \widehat{S_s}}} \Ind_{S_t\times W_{2s}\times W_p\times W_q}
  ^{W_n}[\sgn\otimes(\tau\times \tau)\otimes \triv\otimes \triv].
  \label{eq:CC.C}
\end{equation}

The notation is set up to take the duality in \cite{V4} of types $B$ and $C$
into account.  So we write $r$ for the sign representation of $S_t,$ and $c$ and
$c'$ for the trivial representation of $W_p$, $W_q$.

Now representations in the coherent continuation are parameterized by the
labeled diagram $\tau_L\times \tau_R$: dots are filled in both sides forming the
same shape of Young diagram (coesponding to $\tau\times \tau$ in
\eqref{eq:CC.C}); a column of totaly $t$ ``$r$''s then added next to dots on the
both sides (by \eqref{eq:SW2}), i.e. at most one $r$ is added in each row of the
existing dots diagram (corresponding to $S_t$ in \eqref{eq:CC.C}); then add $p$
``$c$''s and finally add $q$ ``$c'$''s on the left side of the diagram; each $c$
or $c'$ is added to at most one column.  Through this procedure, one must make
sure that the shape of the diagram is a valid Young diagram after each step.


{\color{blue} Not in use right now.  We label the rows of $\tau_L$ as
  $r_0,r_2,\dots ,r_{2m}$ and the rows of $\tau_R$ as $r_1,r_3,\dots ,r_{2m-1}$
  to conform to the notation of the special symbol }

\subsection{Barbasch-Vogan dual}
Let $\cO=(C_{2a},C_{2a-1}, \cdots, C_1,C_0,C_{-1}=0)$ be a special orbit.



The dual orbit $\ckcO = [R_{2a}\geq R_{2a-1}\geq \cdots \geq R_1\geq R_0]$.

{\color{red} The $\ckcO$ is given by add a box on the longest row of $\cO^T$ and
  then do $B$-collapse.}

\begin{lem}\label{lem:C.even}
  $\ckcO$ is not even if and only if there is $i$ such that $C_{2i-1}= C_{2i-2}$
  is an even number $>0$.  In particular, there is no special unipotent
  representations attached to $\cO$.
\end{lem}
\begin{proof}
  Note that $R_{2a}$ must be an odd number. Let $i$ be the largest number
  satisfies the condition above, then $R_{2i-1}=C_{2i-1}$ is an even
  number. Therefore $\ckcO$ is not even.

  If there is on such kind of columns, then all rows in $\ckcO$ have odd
  lengths.
\end{proof}

\trivial{ Note that $C_0,C_{-1}=0$ is even,
  $(C_0,C_{-1})=(C_0,0)\leftrightarrow [C_0+1,0]=:[R_0,0]$. \\
  Suppose that $C_{2i},C_{2i-1}$ are even and $C_{2i-1}>C_{2i-2}$, then
  $(C_{2i},C_{2i-1})\leftrightarrow [C_{2i}+1,C_{2i-1}-1]=:[R_{2i},R_{2i-1}]$. \\
  Suppose that $C_{2i}=C_{2i-1}$ are odd, then
  $(C_{2i},C_{2i-1})\leftrightarrow [C_{2i},C_{2i-1}]=:[R_{2i},R_{2i-1}]$.  }

\begin{defn}
  Suppose $\bfG$ is type $C$.  An special orbit $\cO$ is called purley even if
  $\ckcO$ only has odd rows.
  
\end{defn}

\subsection{Nilpotent orbit of type C and Springer correspondence}
Nilpotent orbits of type C are parameterized by partitions
\[
  \bfpp:= (C_{2a},C_{2a-1},\cdots, C_1, C_0, C_{-1}=0) \quad \text{such that
    $C_{2i}\equiv C_{2i-1}\pmod{2}$}.
\]

Spacial orbits are those orbits satisfing $C_{2i}=C_{2i-1}$ when $C_{2i}$ is
odd. \trivial{ This is exactly the condition that the transpose of the Young
  diagram is also type C.}

The Springer correspondence restricted on the set of special orbit is given as
the following: Write $C_0=2c_0$, $C_{2i}=2c_{2i}+\epsilon_i$ and
$C_{2i-1}=2c_{2i-1}-\epsilon_i.$ The (special) Weyl group representation
associated to the special orbit $\cO$ has columns
\begin{equation}
  \label{eq:tableauc}
  \tau_L\times\tau_R=(c_{2a-1},\dots , c_1)\times (c_{2a},\dots ,c_0).
\end{equation}
\trivial{ The above formula could be deduced from Lusztig's interpration of
  Springer correspondence.  Which is computed as the following: (We follow
  Lusztig's description of ``generalized'' Springer correspondence in
  \cite[\S~11, \S~12]{Lu.I}, see also McGovern's description on p80 of his Left
  cells and domino tableaux paper. Warning: the algorithm of Lusztig and
  McGovern are different but essentially the same, we use McGovern's version. )

  0. Suppose $\cO^t$ is a Young diagram of type C has rows (later $\cO^t$ will
  be the dual of a special orbit $\cO$ of type C).
  \[
    [C_{2a},\cdots, C_0].
  \]

  1. Form a string of numbers
  \[
    C_0, C_1+1, \cdots, C_i+i, \cdots, C_{2a}+2a.
  \]
  
  Note that if $C_{2i}=C_{2i-1} = 2 c_{2i}+1$ is odd, then
  $C_{2i} +2i = 2c_{2i}+1+2i = 2 c_{2i-1}+(2i-1)$ and
  $C_{2i-1} +2i-1 = 2 c_{2i} +2i $; if $C_{2i}=2c_{2i}, C_{2i-1} = 2c_{2i-1}$
  are both even, then $C_{2i} +2i = 2c_{2i}+2i$,
  $C_{2i-1}+2i-1 = 2c_{2i-1}+(2i-1)$.
  
  Therefore, the even parts of the above sequence is
  \[2c_0,\cdots, 2c_{2i}+2i, \cdots 2 c_{2a}+2a,\] the odd part of the above
  sequence is
  \[
    2c_1 +1, \cdots, 2 c_{2i-1}+ (2i-1), \cdots, 2 c_{2a-1}+(2a-1).
  \]

  2. Note that the odd parts is shorter than the even parts, we get the symbol
  \[
    \begin{pmatrix}
      c_0 & & c_{2} +1 & \cdots & c_{2i}+i &\cdots &&\cdots & c_{2a}+a\\
      & c_1 && c_{3}+1 & \cdots & c_{2i+1}+i &\cdots& c_{2a-1}+a-1
    \end{pmatrix}
  \]

  3. This symbol corresponds to the representation
  \[
    [c_{2a}, \cdots, c_{2},c_0]\times [c_{2a-1}, \cdots, c_3, c_1].
  \]
  

  4. There is one flaw of the above calculation, we use columns instead of the
  row.  Use duality to fix the problem which yeilds Barbasch's answer: Let
  $\pi_\cO$ be the Special representation attached to the special orbit $\cO$;
  then
  \[
    \pi_\cO = \pi_{\cO^t} \otimes \sgn = (c_{2a-1}, \cdots, c_3, c_1)\times
    (c_{2a},\cdots, c_2,c_0).
  \]
}

  
  \subsubsection{Some left cells in type C}
  Now recall some results about cells of Weyl group representation.  A symbol
  (of defact 1) is a pair of integer sequence such that $0< a_0< \cdots < a_s$
  and $b_0< b_1< \cdots <b_{s-1}$
  \[
    \sigma :=
    \begin{pmatrix}
      a_0 && a_1 && \cdots && a_{s}\\
      & b_0 & & b_1& \cdots & b_{s-1}
    \end{pmatrix}.
  \]
  Now $\sigma$ correspond to irreducible $W_n$ representation with rows:
  \[
    [a_{s}-s,\cdots, a_1-1, a_0]\times [b_{s-1}-(s-1), \cdots, b_1-1, b_0]
  \]
  $\sigma$ is called special if
  $a_0\leq b_0\leq a_1\leq b_1 \cdots \leq b_{s-1}\leq a_{S}$.  Define
  \[\rank(\sigma) := \sum a_i +\sum b_i - (s(s+1)/2+ s(s-1)/2) = \sum a_i
    +\sum b_i - s^2.
  \]

  On the set of symbols, one define the shift operation:
  \[
    \begin{pmatrix}
      a_0 && a_1 && \cdots && a_{s}\\
      & b_0 & & b_1& \cdots & b_{s-1}
    \end{pmatrix}\mapsto
    \begin{pmatrix}
      0&& a_0+1 && a_1+1  && \cdots && a_{s}+1 \\
      &0& & b_0+1 & & b_1+1 & \cdots & b_{s-1}+1
    \end{pmatrix}
  \]
  For a symbol $\sigma$, let $[\sigma]$ denote the equivalent class of symbols
  under the shift operation.  {\color{red} Warning: Lusztig's definition of rank
    may be different in his different papers.  } \trivial{ In general symbol is
    a pair of integers $\begin{pmatrix}A \\ B\end{pmatrix}$.
    $\rank(\begin{pmatrix}A \\ B\end{pmatrix}):= \sum a_i + \sum b_i
    -(\frac{|A|+|B|-1}{2})^2$ which equals to $\sum a_i +\sum b_i -s^2$ }
  
  Double cells of $\whW_n$ are parameterized by the special symbols class of
  rank $n$.  The dobule cell $\cC_\sigma$ of a given special symbol $\sigma$
  consists all representations corresponding to symbols with the same set of
  entries of the special symbol $\sigma$.
  
  Now recall the definition of $L$. Fix a special symbol $\sigma$ as above, let
  $T = \set{t_1, t_2, \cdots, t_p}$ be the set of elements in $\sigma$ only
  appears in the top row, and $B = \set{b_1, b_2, \cdots, b_{p-1}}$ be the set
  of elements only appears in the bottom row.  \trivial{ (It is obvious that
    $|T| = |B|+1$ by the definition of symbol since each number appears at most
    twice.)  }
  $L := \set{(t_i,b_i)| i=1,2,\cdots,p-1} \subset \bF_2^T\times \bF_2^B$ (Here
  $|\bF_2^S|$ is naturally identified with the power set of the set $S$).
  $R := \set{(t_{i+1},b_i)|i = 1,2,\cdots, p-1}$.

  
  
  Left cells are more complicated, they are parameterized by supersmooth
  subspace of $L$, where $L$ is a $\bF_2$ vector space attached to the spacial
  symbol $\sigma$.  There are two very important supersmooth subspaces $L$ and
  $0$.

  The left cell $\cC_L$ consists of all symbols obtained by transfering some
  $t_i$ to the bottom row and corresponding $b_i$ to the top row.  The left cell
  $\cC_0$ consists of all symbols obtained by transfering some $t_{i+1}$ to the
  bottom row and corresponding $b_i$ to the top row.

  Another way to obtain $\cC_L$: it consists of symbols obtained by
  interchanging some of the entries which only appears in the bottom row to
  their closest neighbors to the \emph{left} which only appears in the top row.

  Similarly $\cC_0$ consists of symbols obtained by interchanging some of the
  entries which only appears in the bottom row to their closest neighbors to the
  \emph{right} which only appears in the top row.

  Note that a symbol is in certain $\cO_L$ if only if its entry satisfies
  $a_{i-1} \leq b_{i}\leq a_{i+1} \ \forall i$.

  A symbol is in certain $\cO_0$ if only if its entry satisfies
  $a_i \leq b_{i}\leq a_{i+2} \ \forall i$.
  
  {\color{orange} This is equivalent to \cite[p 219 (4)]{Mc}.  Note that a
    special symbol may have configuration
    \[
      \begin{tikzcd}[row sep=1em,column sep=1em]
        \cdots & & a_i \ar[dr,equal] & & \cdots & & a_{i+s} \ar[dr,equal] & & \cdots\\
        & \cdots & & b_i & & \cdots & & b_{i+s}&
      \end{tikzcd}
    \]
    This type of segment is irrelavent to the ``interchange'' operation.
   
    Suppose we have a segement as the following:
    \[
      \begin{tikzcd}[row sep=1em,column sep=1em]
        \cdots & & top_j \ar[drrrrrrr,dashed,leftrightarrow] &&  a_{i+1} \ar[dl,equal] & & \cdots & & a_{i+s+1} \ar[dl,equal] & & \cdots\\
        & \cdots & & b_i & & \cdots & & b_{i+s}& & bot_j
      \end{tikzcd}
    \]
    After the interchange, we will have the configuration:
    \[
      \begin{tikzcd}[row sep=1em,column sep=1em]
        \cdots && a_{i+1} \ar[drrr,equal] & & \cdots & & a_{i+s+1}
        \ar[drrr,equal]
        & & bot_j & &\cdots\\
        & \cdots & & top_j&& b_i & & \cdots & & b_{i+s}& &
      \end{tikzcd}
    \]
    By observing the above pictures, a moment though yields the claim.  }


 
  View $W_n$ as the Weyl group of type C, then $\cC_L$ is called Lusztig and
  $\cC_0$ is called Springer.

  {\color{red} McGovern claims that: The set of all ``Springer'' symbols
    coincide with all symbols obtain by Springer correspondence from a nilpotent
    orbit of type C (with trivial local system).

    I can not varify that. It seem that this dose not agree with Lusztig's
    results, see Carter's book p420 paragraph-3 on Springer correspondence.  }
  \trivial{ This is McGovern's claim, haven't checked yet.}


  Let
  $\tau :=[\lambda_1\geq \lambda_2\geq \cdots \lambda_{r+1}\geq 0]\times
  [\nu_1\geq \nu_2\cdots \geq \nu_r]\in \whW_n$ then $\tau\in \cC_L$ if and only
  if
  \begin{equation}\label{eq:rep-Lusztig}
    \lambda_i+1\geq \mu_i \geq \lambda_{i+2}-1.
  \end{equation}
  \trivial{ $\tau$ has symbol
    \[
      \begin{pmatrix}
        \lambda_{r+1} & & \lambda_r +1 & & \cdots && \lambda_{2}+r-1 && \lambda_1+r\\
        & \nu_r && \nu_{r-1}+1 &&\cdots && \nu_1+r-1 &\\
      \end{pmatrix}
    \]
    Now the condition on symbol reads
    \[
      \lambda_{i+1}+r-i-1 =\lambda_{i+2}+r+1-(i+2) \leq \mu_{i}+r-i \leq
      \lambda_i+r+1-i.
    \]
  
  }


  Similarly, $\tau \in \cC_0$ if and only if
  \[
    \lambda_{i-1}+2 \geq \nu_i \geq \lambda_{i+1}.
  \]

  From now on, let $\tau\in \whW_n$ be a irreducible representation occure in
  \eqref{eq:CC.C}.  Then $\tau$ are Lusztig.  \trivial{ It is to varity the
    \eqref{eq:rep-Lusztig}.  It follows from the dot-r-c diagram construction.
    $\nu_i\leq \lambda_{i}+1$ is clear since we will add at most one ``r'' on
    the right young digram.  The inequality $\lambda_{i+2}-1\leq\mu_i$ follows
    form the consideration of ``exterme cases'' for the row $i,i+1,i+2$:
    \[
      \begin{matrix}
        \cdots & \bullet & r & c\\
        \cdots & \bullet & * & c'\\
        \cdots &  * & *\\
      \end{matrix}
      \times
      \begin{matrix}
        \cdots & \bullet\\
        \cdots\\
        \cdots\\
      \end{matrix}
    \]
    where $*$ could be $\bullet$, $r$,$c$, or $c'$ when ever it is proper.
    Anyway, for the ``exterme case'', $\lambda_{i+1} -1 = \mu_i+1-1 \leq \mu_i$.
  } By duality, $\tau\otimes \sgn$ are all Springer (in the sense of type C
  representation).

  {\color{red} To be check:} Now we reach the conclusion of Barabasch/McGovern
  that: The other representations in the primitive ideal cell are obtained by
  interchanging pairs coming from even columns,
  \begin{equation}
    \label{eq:wccell}
    (c_{2i}, c_{2i-1})\longleftrightarrow (c_{2i-1}-1,c_{2i}+1).  
  \end{equation}
  \subsection{Reductions in type C}
  Let $\cO = (C_{2a}, C_{2a-1}, \cdots, C_{1}, C_{0}\geq 0)$ be a special orbit.

  Now we start to count the number of unipotent representations.  In this
  section, we reduce the counting problem to the certain orbits with restricted
  shapes.

  \subsubsection{Delete odd rows}


\begin{prop}\label{prop:reduction.C}
  Let $\cO'$ be the special orbit obtained by deleteing all odd rows of $\cO$.
  There is a bijection from the dot-c-r diagrams attached to $\cO$ to that of
  $\cO'$.  In particular, $\unip(\cO)$ and $\unip(\cO')$ has the same number of
  elements.
\end{prop}
\begin{proof}
  Let $\tau_\cO :=\tau_L\times \tau_R$ be the special representation attached to
  $\cO$.  We refer to \eqref{eq:tableauc} for the shape of
  $\tau_L\times \tau_R$.  All other representations are obtain from $\tau_\cO$
  by ``switching columns'' using recipy \eqref{eq:wccell}. We use
  $\tau' = \tau'_L\times \tau'_R$ denote an arbitary $W$-representation in the
  cell of $\tau_\cO$.

  \textbf{Delete the longest odd rows: } Suppose the longest rows of $\cO$ have
  odd size. This is equivalent to $C_0 = 2c_0 >0$. Note that the largest row
  occur an even number of times.  Now all $\tau'$ has the shape that $\tau'_R$
  has one more column (the column $c_0$) than $\tau'_L$.  Therefore, the column
  for $c_0$ must be filled with $r$'s. The remainder of the rows corresponding
  to the column of size $c_0$ in both $\tau_L$ and $\tau_R$ must be filled with
  $\bullet'$s.

  % Note that the rows in $\tau'_L$ must be of size one less than those of
  % $\tau'_R$.


  The multiplicity is not affected if the first $c_0$ rows are removed from both
  $\tau'_L$ and $\tau'_R$. The computation of multiplicities is reduced to the
  case where the first $C_0$ rows removed $c_0=0$.

  The situation is illustrated below, delete the boxed parts.

    \begin{tikzpicture}
      \matrix [matrix of math nodes,text width=.7em, text height=.7em,text badly
      centered] (m) {
        \cdots  & \bullet & \bullet &        & \cdots & \bullet & \bullet & r      \\
        \cdots  & \vdots  & \vdots  &        & \cdots & \vdots  & \vdots  & \vdots \\
        \cdots  & \bullet & \bullet &        & \cdots & \bullet & \bullet & r      \\
        \cdots  & *       &         & \times & \cdots & *       &         &        \\
        \cdots & *       &         &        & \cdots & *       &         &        \\
        \cdots  & \vdots  &         &        & \cdots & \vdots  &         &        \\
      };
      % simple rectangle
      \draw (m-3-1.south west) rectangle (m-1-3.north east); \draw (m-3-5.south
      west) rectangle (m-1-8.north east);
      % fancy blue rectangle
    \end{tikzpicture}

    \textbf{Delete the odd rows of length $2a-1$} Now assume $C_0=0$ but
    $C_2>C_1$. Note that $\cO$ is a special orbit. So $C_2$ and $C_1$ must be
    even columns, and $c_2>c_1$.

    First consider $\tau_\cO = \tau_L\times \tau_R$, it must be filled as the
    following: The first $c_1$ rows in the column for $c_2$ must be filled by
    $\bullet'$s or $r's$. The remaining rows in $c_2$ are filled with $r'$s. The
    remainder of rows $c_1+1$ to $c_2$ in both $\tau_L$ and $\tau_R$ (columns
    $c_3, c_4\dots$) must be filled with $\bullet'$s.  Therefore, delete the
    boxed part do not affect the multiplicity, see below.


\begin{tikzpicture}
  \matrix [matrix of math nodes,text width=.7em, text height=.7em,text badly
  centered] (m) {
    &         & c_1       &        &        &         & c_2       \\
    \cdots & *       & \bullet/r &        & \cdots & *       & \bullet/r \\
    \cdots & \vdots  & \vdots    &        & \cdots & \vdots  & \vdots    \\
    \cdots & *       & \bullet/r &        & \cdots & *       & \bullet/r \\
    \cdots & \bullet &           &        & \cdots & \bullet & r         \\
    \cdots & \vdots  &           &        & \cdots & \vdots  & \vdots    \\
    \cdots & \bullet &           &        & \cdots & \bullet & r         \\
    \cdots & *       &           & \times & \cdots & *       &           \\
    \cdots & *       &           &        & \cdots & *       &           \\
    \cdots & \vdots  &           &        & \cdots & \vdots  &           \\
  };
  % simple rectangle
  \draw (m-7-1.south west) rectangle (m-5-2.north east); \draw (m-7-5.south
  west) rectangle (m-5-7.north east);
  % fancy blue rectangle
\end{tikzpicture}


For the other representations obtained from $\tau_\cO$ without switching $c_1$
and $c_2$, the situation similar.

Now consider the representations switching
$(c_1,c_2)\leftrightarrow (c_2+1,c_1-1)$. The shape of the dot-r-c diagram must
be as the following. Now delete the boxed part will not affect the
multiplicities.

\begin{tikzpicture}
  \matrix [matrix of math nodes,text width=.7em, text height=.7em,align=flush
  center ] (m) {
    &         & c_2+1     &        &        &         & c_1-1     \\
    \cdots & *       & \bullet/r &        & \cdots & *       & \bullet/r \\
    \cdots & \vdots  & \vdots    &        & \cdots & \vdots  & \vdots    \\
    \cdots & *       & \bullet/r &        & \cdots & *       & \bullet/r \\
    \cdots & \bullet & r         &        & \cdots & \bullet &           \\
    \cdots & \vdots  & \cdots    &        & \cdots & \vdots  &           \\
    \cdots & \bullet & r         &        & \cdots & \bullet &           \\
    \cdots & \bullet & r/c       &        & \cdots & \bullet &           \\
    \cdots & \bullet & r/c'      &        & \cdots & \bullet &           \\
    \cdots & *       &           & \times & \cdots & *       &           \\
    \cdots & *       &           &        & \cdots & *       &           \\
    \cdots & \vdots  &           &        & \cdots & \vdots  &           \\
  };
  % simple rectangle
  \draw (m-7-1.south west) rectangle (m-5-3.north east); \draw (m-7-5.south
  west) rectangle (m-5-6.north east);
  % fancy blue rectangle
\end{tikzpicture}   

They correspond to the next largest odd rows of $\cO$ and do
not affect the number of dot-r-c diagrams.

In summary, delete the odd row of length $2a-1$
do not affect the number of dot-r-c diagrams.

\textbf{The general case} Argue exactly in the same way, we see that delete odd rows of any length in $\cO$ will
not affect the number of dot-r-c diagrams.


\end{proof}

\begin{thm}[{Type C}]\label{t:1}
The multiplicities of a $\tau_L\times\tau_R$ coincide with the case
when $C_0=0$ and $c_{2i}=c_{2i-1}$ when $C_{2i},C_{2i-1}$ are both even, and
{\color{red} $c_{2i}=c_{2i-1}-1$} in case $C_{2i}C_{2i-1}$ are both odd. 
\end{thm}

\subsection{Some examples}
For each complex nilpotent orbit $\cO$.
Let $\drc(\cO)$ be the set of dot-r-c diagrams attached to $\cO$.
Let $\LS(\cO)$ be the set of local systems obtained by theta correspondence. 

Here is some examples about counting.


\section{Parameterization of unipotent representations}

\subsection{Type D}
Combinatorics of type D. 

\noindent\textbf{Type D.\ } Label the columns $C_{2a-1}\ge\dots \ge
C_0\ge 0$ satisfying $C_{2i}\equiv C_{2i-1} \pmod{2},$ and
$C_{2a-1}\equiv C_0\equiv 0 \pmod{2}.$ As before, write $C_{2i}=2c_{2i}+\epsilon_i$,
$C_{2i-1}=2c_{2i-1}-\epsilon_i$ and $C_0=2c_0,\ C_{2a-1}=2c_{2a-1}.$ Special
nilpotent orbits satisfy $C_{2i}=C_{2i-1}$ whenever they are odd. We will reduce
to the case $C_{2i}=C_{2i-1}$ which is the case of $\cO$ having no even sized
rows.  The special Weyl group representation associated to $\cO$ has columns
\begin{equation}
  \label{eq:drep}
\tau_L\times\tau_R=(c_{2a-1},\dots ,c_1)\times (c_{2a-2},\dots
,c_0).  
\end{equation}
The other representations in the primitive ideal cell are obtained by
interchanging pairs coming from even columns, 
$$
(c_{2i},c_{2i-1})\longleftrightarrow (c_{2i-1}+1,c_{2i}-1).
$$
This is again as in  \cite{Mc}.  
The infinitesimal character $\lambda_{\cO}$ is as in type C, \eqref{eq:inflc}.
\begin{remark}
Adding an even sized column of length at least $2c_{2a-1}$ gives a
nilpotent orbit of type C, and implements the $\Theta-$correspondence. 

In the case when all rows of odd size occur an even number of times,
the corresponding unipotent representations can be parametrized by the
local systems on the \textit{real} orbits.
\end{remark}


The unipotent representations of $\SO$ are parameterized by 
filling $\bullet$, $r'$, $r$, $c$, $c'$ in the relevent representations.

In the must of the cases, induce to the full orthogonal group will yields $2$
different representations (Lets swich the left and right diagram to mark the
difference). The only exception is that the diagram only contains ``$\bullet$''
in which case, the induce yields an irreducible $\rO$-moudle.

To compare with theta correspondence,
we made the following choice. The theta lift from $\Sp$ always correspondes to
the case where ``$r'$, $r$, $c$, $c'$'' is filled on the left diagram.

\begin{eg}
 We make the following choice of parameterization of $\rO(p,q)$
 representations.
 The advantage of the following choice is that, the occurance of ``$c'$'' is
 equalent to 
 ``spliting/doubling'' of associate varieties. 
 \[
   \begin{tikzpicture}[baseline={([yshift=-.8ex]current bounding box.center)}]
     \matrix [matrix of math nodes,nodes in empty cells] (m)
     {r'  \\
       \vdots   \\
       r'   & \\ 
       r        \\
       \vdots \\
       r\\
       r\\
     };
     \draw [decorate,decoration={brace, amplitude=4pt, raise=5pt}]
     (m-3-1.west) -- (m-1-1.west) node [black,midway,xshift=-2em]
     {$s_0$};
     \draw [decorate,decoration={brace, amplitude=4pt, raise=5pt}]
     (m-7-1.west) -- (m-4-1.west) node [black,midway,xshift=-2em]
     {$r_0$};
   \end{tikzpicture}   
   \longleftrightarrow    \bfone_{2r_0,2s_0}^{+,-}
   \qquad
   \begin{tikzpicture}[baseline={([yshift=-.8ex]current bounding box.center)}]
     \matrix [matrix of math nodes,nodes in empty cells] (m)
     {r'  \\
       \vdots   \\
       r'   & \\ 
       r        \\
       \vdots \\
       r\\
       c\\
     };
     \draw [decorate,decoration={brace, amplitude=4pt, raise=5pt}]
     (m-3-1.west) -- (m-1-1.west) node [black,midway,xshift=-2em]
     {$s_0$};
     \draw [decorate,decoration={brace, amplitude=4pt, raise=5pt}]
     (m-6-1.west) -- (m-4-1.west) node [black,midway,xshift=-2em]
     {$r_0$};
   \end{tikzpicture}   
   \longleftrightarrow    \bfone_{2r_0+1,2s_0+1}^{+,-}
 \]

 \[
   \begin{tikzpicture}[baseline={([yshift=-.8ex]current bounding box.center)}]
     \matrix [matrix of math nodes,nodes in empty cells] (m)
     {r'  \\
       \vdots   \\
       r'   & \\ 
       r        \\
       \vdots \\
       r\\
       c\\
       c'\\
     };
     \draw [decorate,decoration={brace, amplitude=4pt, raise=5pt}]
     (m-3-1.west) -- (m-1-1.west) node [black,midway,xshift=-2em]
     {$s_0$};
     \draw [decorate,decoration={brace, amplitude=4pt, raise=5pt}]
     (m-6-1.west) -- (m-4-1.west) node [black,midway,xshift=-2em]
     {$r_0$};
   \end{tikzpicture}   
   \longleftrightarrow    \bfone_{2r_0+2,2s_0+2}
   \qquad
   \begin{tikzpicture}[baseline={([yshift=-.8ex]current bounding box.center)}]
     \matrix [matrix of math nodes,nodes in empty cells] (m)
     {r'  \\
       \vdots   \\
       r'   & \\ 
       r        \\
       \vdots \\
       r\\
       r\\
       c'\\
     };
     \draw [decorate,decoration={brace, amplitude=4pt, raise=5pt}]
     (m-3-1.west) -- (m-1-1.west) node [black,midway,xshift=-2em]
     {$s_0$};
     \draw [decorate,decoration={brace, amplitude=4pt, raise=5pt}]
     (m-7-1.west) -- (m-4-1.west) node [black,midway,xshift=-2em]
     {$r_0$};
   \end{tikzpicture}   
   \longleftrightarrow    \bfone_{2r_0+1,2s_0+1}
 \]

 Now the lifting algorithm of diagrams is that
 \begin{enumerate}
  \item replace $r'$ by $\bullet$, 
  \item lifting case: add a colums of lenght $c_{2a-1}$ to make a valid dot-r-c diagram.
  \item lifting the determinant twist:
    do the usual lift, then move the left bottom corner ``r'' (if it exists) from the right diagram 
    to the most left column of the left diagram. 
    Otherwise, remove the left bottom corner ``$\bullet$'' and replace the most
    left column of the left diagram by $(\bullet\cdots\bullet c c')^T$.
  \item generalized lifting case: add a colums of lenght $c_{2a-1}-1$ and
    make a valid dot-r-c diagram. If there is no valid dot-r-c diagram, this
    representation dose not lift.
  \end{enumerate}

  \ytableausetup{noframe=true,centertableaux} 
 \[
   \begin{tikzcd}[baseline={([yshift=-.8ex]current bounding box.center)}]
     \pi & \drc(\pi)& \drc(\theta(\pi)) & \cV(\theta(\pi))\\  
     \bfone_{2r_0,2s_0}^{+,-}
     \ar[r, <->]& 
\ytableaushort{
  {r'}    ,  
  \vdots  , 
  {r'}    , 
  {r}     , 
  \vdots  , 
  {r}      
}
     \ar[r] &
\ytableaushort{
  {\bullet} \none {\bullet},
  \vdots \none \vdots,
  {\bullet} {\none[\times]} {\bullet},
  {r} \none {r},
  \vdots \none \vdots,
  {r} \none {r}
}
\ar[r]&   
\ytableaushort{
  +-,
  +-,
  \vdots\vdots,
  +-,
  +-,
  -+,
  -+,
  \vdots\vdots,
  -+,
  -+
}
\end{tikzcd}   
 \]

\end{eg}

\subsection{}
Lift drc diagram $\tau$, corresponding to $\pi$ from type C to type D

A type C special unipotent orbit attach a column of even length $C_{2a+1} = 2
c_{2a+1}$. 

Now the attach a column of length $c_{2a+1}$ on the left of the $\tau_L$.
Allow all possible lables and change ``$\bullet$'' into ``$r'$'' to make a valid
diagram, denote by $\tau' = \tau'_L\times \tau'_R$.

Count number of symbols to determine the signature of the orthogonal group.
If ``$c'$'' is not presented in the new diagram, attach the diagram to
the representation $\theta(\pi)\otimes \bfone^{+,-} $.
If ``$c'$'' is presented, the new diagram is attached to 
$\theta(\pi)$.

By this choice, it is clear again that if no ``$r$/$c$/$c'$'' occure in the
longest column of $\tau'_L$ then $\theta(\pi)\otimes \bfone^{+,-}$ will be
killed 
by generalized lift.  
Moreover the appearance of $c'$ implies a ``splitting'' of orbit in the
generalized lift. 



\subsection{The strategy of matching}
Let $G$ and $G'$ form a dual pair.  Suppose a special nilpotent orbit $\cO'$ is
obtained by lift/generalized lift of $\cO$ (i.e. by adding an even length
column, or adding an odd length column and then delete a box on the previous
column when is a symplectic group)

What we have to do is construct maps to form the following commutative diagram:
\[
  \begin{tikzcd}
    \drc(\cO) \ar[r,dashed,"\eta"] \ar[d,"\cL"] & \drc(\cO') \ar[d,"\cL"]  \\
    \LS(\cO) \ar[r,dashed,"\vartheta"] & \LS(\cO')
  \end{tikzcd}
\]

If $G$ is an orthogonal group, and $\cO'$ is a normal lift of $\cO$, the
following additional diagram on the twisting of determinant also need to be
establish: ($\cD$ is the operation which change special shape diagram to the
non-special shape)
\[
  \begin{tikzcd}
    \drc(\cO)\ar[drr,"\eta"] \ar[d,"\cL"]& & \\
    \LS(\cO) \ar[r,"\vartheta"] \ar[d,"\det\otimes"] & \LS(\cO') \ar[r,<-,"\cL"] &
    \drc(\cO')
    \ar[d,<->,"\cD"]\\
    \LS(\cO) \ar[r,->,"\vartheta"] & \LS(\cO') \ar[r,<-,"\cL"]
    & \drc(\cO')
  \end{tikzcd}
\]

The map $\eta$ and $\vartheta$ are the lifting map (running over all possible
real forms of the real orthogonal group).
The map $\cL$ calculates the local system for each dot-r-c diagram. 

The $\eta$ maps is already defined in your notes, i.e. the algorithm of
changing ``$r'$'' to ``$\bullet$'' (although I prefer a slightly different
choice for the exceptional cases).
From the definition of $\eta$, we could see all elements in $\drc(\cO')$ could
be obtained from $\drc(\cO)$ up to character twisting. 

The key statement we must establish is that $\cL$ is an injection.


\delete{
\subsection{D-module for $\rO(2n)$}


Consider $\bfG = \rO(2n,\bC)$.

Now
\begin{eg}
  Consider the trivial and sign representations of $\rO(2n)$.  Let $\pi = \bfone$
  or $\det$.  Then $\bfK = \bfG$. Let $w_c$ be the element in $\bfK$ fixing
  $\bfB$.  The $\bfK$-sheaf on $\bfG/\bfB$ is determined by the representation
  of $\Stab_{\bfK}(\bfB) = \gen{w_c}\ltimes \bfT$.  We put the trivial
  representation on $\bfT$ and the trivial or sign on $\gen{w_c}$.  This
  realizes the correspondence of $\bfK$-sheaf on $\bfG/\bfB$ with
  $\rO(2n)$-module $\bfone$ and $\det$.
\end{eg}
}


\section{Counting}
In this section,  we use $(c_1,c_2, ..., \star)$ denote the object obtained by adding
columns $c_1, c_2, ...$ to $\star$ where $\star$ could be a Young diagram or a
dot-r-c diagram. 

Let $\drc(\tau)$ denote all drc diagrams of shape to
$\tau = \tau_L\times \tau_R$.  Let
$\nabla \tau = \nabla \tau_L\times \nabla \tau_R$ be the ``descent'' of
representation where $\nabla\tau_L$ (resp. $\nabla\tau_R$) is obtained from
$\tau_L$ (resp. $\tau_R$) by reomoving the first column.

Clearly, $\nabla$ induces a welldefined map
\begin{tikzcd}
\nabla \colon \drc(\tau) \ar[r] & \drc(\nabla \tau). 
\end{tikzcd}
The map $\nabla$ extends to a map
\[
\begin{tikzcd}
\nabla_\cO \colon \drc(\cO) \ar[r] & \drc(\nabla^2 \cO).
\end{tikzcd}
\]

\subsection{Add even columns}
In this section we consider the process of adding two even length colums. 

  
\begin{eg}
  Suppose $\cO=(C_{2a},C_{2a-1}, \cdots, C_1,C_0)$
  satisfies  $C_{2a},C_{2a-1}$ are even.
  Attaching a even column $(C_{2a}+2d,C_{2a-1}+2d)$ yields a unipotent orbit
  \[
    \cO'' = (C_{2a}+2d,C_{2a-1}+2d)+\cO
  \]
  
  We have
  \[
    \# \drc(\cO'') = 8 \#\drc(\cO)
  \]
 %# only has even columns. 
  In particular, when $\cO$ only has even length columns, we have
  \[
    \# \drc(\cO) = \#\LS(\cO).
  \]
  This is easy by induction. 
\end{eg}  

\begin{eg}
  Assume $n$ is odd. Consider $\cO = (n,n)$ of $\Sp(2n)$.
  We have $\drc(\cO) = 2n+1$.
  {\color{red} Check: \[ \# \LS(\cO) = 2n+1.\]}
  In $\LS(\cO)$ there are $n$ reducible local systems (the associated variety has
  two components). The rest $n+1$ local systems are irreducible.  

  Suppose
  \[
    \cO' = (n+2d-1, n+2d-1, n,n)
  \]
  Then
  \[
    \#\drc(\cO) = 4(2d)n + 4(2d-1)(n+1).
  \]
\end{eg}


\begin{eg}
  Assume $n$ is odd. Consider $\cO = (C_{2a},C_{2a}, \cdots )$ of $\Sp(2n)$
  with $C_{2a}$ even. 
  Suppose
  \[
    \cO' = (c_{2a}+n,c_{2a}+n)+\cO
  \]
  Then
  \[
    \#\drc(\cO') = (2n+1) \#\drc(\cO).
  \]
  In $\LS(\cO')$ there are $n$ doubled local systems (the associated variety has
  two components). The rest $n+1$ local systems are preserved.  
\end{eg}

\begin{eg}
  Assume $n$ is odd. Consider $\cO_0 = (C_{2a},C_{2a}, \cdots )$,
  $C_{2a}$ 
  $\cO = (n,n)$ of $\Sp(2n)$.
  We have $\drc(\cO) = 2n+1$.
  {\color{red} Check: \[ \# \LS(\cO) = 2n+1.\]}
  In $\LS(\cO)$ there are $n$ reducible local systems (the associated variety has
  two components). The rest $n+1$ local systems are irreducible.  

  Suppose
  \[
    \cO' = (n+2d-1, n+2d-1, n,n)
  \]
  Then
  \[
    \#\drc(\cO) = 4(2d)n + 4(2d-1)(n+1).
  \]
\end{eg}

\section{Combinatorics for type B }
In this section $\bfG = \SO(2n+1,\bC)$.

\subsection{Nilpotent orbit in type B}
Nilpotent orbits of $\bfG$ is parameterized Young diagrams with columns
\[
  \cO = (C_{2a}\geq C_{2a-1} \geq\cdots C_{1} \geq C_0 > 0)
\]
satisfying
\begin{enumC}
\item $C_{2i+1} \equiv C_{2i } \pmod{2}$ (This corresponds to the condition even rows occur
  even times.) 
\item $C_{2a} \equiv 1 \pmod{2}$ (This corresponds to the total number of box is odd)  
\end{enumC}

Note that  $\cO$ is  special if and only if 
\[
  C_{2i+1} = C_{2i} \text{ when $C_{2i}$ is even. }
\]



This correspond to the condition that the transpose $\cO^t$ is also type B.

Now let $\cO = (C_{2a}, C_{2a-1}, \cdots, C_1, C_0)$ be a special nilpotent
orbit of type B.

Then the Springer representation attached to $\cO$ is given by
(see\cite[p~421]{Carter})

\trivial{
  We follow Carter's notation.
  Note that $\cO^t$ is also type B, with rows $[C_{2a}, C_{2a-1}, \cdots, C_1,
  C_0]$.
  Now $\lambda^* = [C_0, C_1+1, \cdots, C_{2a}+2a]$.
  Write $C_{2i} = 2 c_{2i}+\epsilon_i$, $C_{2i+1} = 2 c_{2i+1}-\epsilon_i$,
  $\epsilon_i = 0$ or  $1$.
  
  Now the odd parts of $\lambda^*$ are 
  So
  \[
    \begin{split}
    (2\xi^*_1+1, \cdots, 2\xi^*_{a+1}+1) &= (2c_0 + 1, \cdots ,2 c_{2(a-1)}+1+2(a-1), 2
    c_{2a} +1 + 2 a)\\
    (2\eta^*_1, \cdots, 2\eta^*_{a}) &= (2c_1 + 1, \cdots ,2 c_{2(a-1)}+1+2(a-1), 2
    c_{2a} +1 + 2 a)
  \end{split}
\]
  ( The argument: $C_{2a}$ must be odd. Note that, the parity of
  $(C_{2i}+2i, C_{2i+1}+2i+1)$ are different. Moreover, $C_{2i}=C_{2i+1}$ when
  they are even. So sppose $C_{2i}$ is even,
  $2\xi^*_{i+1}+1 = 2 c_{2i+1}+2i+1 = 2 c_{2i} + 2i+1$,
  $2\eta^*_{i+1}= 2 c_{2i}+2i = 2 c_{2i+1}+2i$.  Suppose $C_{2i}$ is odd,
  $2\xi^*_{i+1} +1= (2 c_{2i}+1 )+2i$ and $2\eta^*_{i+1} = 2 c_{2i+1}-1 +2i+1 =
  2 c_{2i+1}+2i$)

  Therefore, $\xi^*_{i+1} = c_{2i}+i$, $\eta^*{i+1} = c_{2i+i}+ i$.  
  Hence, $\xi_{i+1} = c_{2i}$, $\eta_{i+1} = c_{2i+1}$.
  Therefore, the Springer representation of $\cO^t$ is
  \[
     [c_0, c_2, \cdots, c_{2a}] \times [c_1, c_3, \cdots, c_{2a-1}] .
  \]
  Now tensor $\sgn$ yeilds that
  $\cO$ corresponds to the special representation
  \[
    \tau_L\times \tau_R = (c_{2a-1}, \cdots, c_3, c_1)\times (c_{2a}, \cdots,
    c_2, c_0).
  \]
}

\begin{equation}
  \label{eq:tableaub}
\tau_L\times\tau_R=(c_{2a-1},\dots ,c_1)\times (c_{2a},c_{2a-2},\dots ,c_0).  
\end{equation}
The other representations in the primitive ideal cell are given by
interchanging 
\[
  (c_{2i+1}, c_{2i})\longleftrightarrow
(c_{2i},c_{2i+1}).
\]
This corresponds to the Springer representation
(associated to the trivial local system) for another nilpotent orbit.


Now assume $\cO = (C_{2a}, C_{2a-1}, \cdots, C_0>0)$ satisfies
\[
  C_{2i} >  C_{2i-1} \quad \forall i.
\]
Note that $C_{2a}$ is always odd.
The Barbasch-Vogan duality map is given by (for special orbit)
the type C-collapse of $\cO^t$ then remove the last box. 

Then $\ckcO = (R_{2a}, \cdots, R_0)$ is given by
\[
  \begin{split}
    R_{2a} &  = C_{2a}-1\\
    (R_{2i+1}, R_{2i}) & =
    \begin{cases}
      (C_{2i+1},C_{2i}) & \text{if } C_{2i} \text{ is even}\\
      (C_{2i+1}+1,C_{2i}-1) & \text{if } C_{2i} \text{ is odd}\\
    \end{cases}
  \end{split} 
\]

Note that the infinitesimal character of $\bfG$ is given by
\[
  \chi_\ckcO = (\rho_{R_{2a}}, \cdots, \rho_{R_0})
\]
where
\[
  \rho_{R} = (\frac{R-1}{2},\frac{R-3}{2}\cdots, \frac{1}{2})\in \half + \bZ^{R/2}.
\]


\begin{remark}
  Recall that a type B orbit $\cO$ is called quasi-distinguish if it has columns
  \[
    \cO = (C_{2a}, \cdots, C_{0})
  \]
  where all $C_k$ are odd integers and $C_{2i} > C_{2i-1}$.  Now
  $\ckcO = (R_{2a}, \cdots, R_0) = (C_{2a}-1, C_{2a-1}+1, C_{2a-2}-1, \cdots,
  C_1+1, C_0-1)$ .
  Clearly, $R_{k}$ are all even integers and $R_{2i+1}>R_{2i}$.
\end{remark}

\subsection{dot-c-r diagram}
We fix the embedding $\iota \colon \GL(n,\bC)\rightarrow \SO(2n+1,\bC)$ in the
standard way (by fixing a polarization).  The strong real form of $\bG$ is given
by the involution
\[
  x_{t} = \iota(\diag(\underbrace{1,\cdots, 1}_{n-t},\underbrace{-1}_{t}))
\]
Note that $x_{t}$ corresponds to $\SO(2n-2t+1, 2t)$.
Therefore, the following counting procedure only counts $\SO(p,q)$ with $q$
even (or equivalently only counts representations for $\SO(p,q)$ with $p>q$.  


The Cartan subgroups are parametrized by
four integers $(p,q,2s,t),$ satisfying $p+q+2s+t=n.$ The corresponding
representation is 
\begin{equation} \label{eq:CC.B}
\sum_{\tau\in\wtS_s}Ind_{W_p\times W_q\times W_{2s}\times S_t}^{W_n}
[\sgn\otimes \sgn\otimes(\tau\times \tau)\otimes \triv].
\end{equation}
The sum is over the $\tau\times\tau$ where $\tau$ is a partition
of $S_s$.
The representation $\tau\times \tau$ is labelled by dots, sign by $r$ or $r',$ and $triv$
by $c$. Recall also the well known formula
\begin{equation}
Ind_{S_n}^{W_n}(\triv)=\sum_{a+b=n} (a)\times(b)
\label{2.1.1}\end{equation}
To induce we add $r$ and $r'$ at most one to each row to $\tau_R$, and
$c$ at most one to each column to both $\tau_L$ and $\tau_R$ the total
number being $r$.



\section{Combinatorics for type M }
In this section $\bfG = \Sp(2n,\bC)$ and $G = \Mp(2n,\bR)$.
Note that the dual (real) group is $\ckG= \Mp(2n,\bR)$ according to Renard-Trapa.

\subsection{Nilpotent orbit in type M}
Nilpotent orbits of $\bfG$ is parameterized Young diagrams with columns
\[
  \cO = (C_{2a+1} \geq\cdots C_{1} \geq C_0 > 0)
\]
satisfying
\begin{enumC}
\item [] $C_{2i+1} \equiv C_{2i } \pmod{2}$ (This corresponds to the condition odd rows occur
  even times.) 
\end{enumC}

We call $\cO$ \emph{metaplectic special} if 
\[
  C_{2i+1} = C_{2i} \text{ when $C_{2i}$ is even. }
\]
Note that $\cO$ is not a special orbit in general. 
The condition is equivalent to that adding a column of size $2n+1$, the orbit
become special in type B (This operation implements the stable range theta lifting.) 


Now let $\cO = (C_{2a+1}, \cdots, C_1, C_0)$ be a special nilpotent
orbit of type M.
We define 
\[
  \ckcO = (R_{2a+1}, \cdots, R_1, R_0)
\]
where
\[
    (R_{2i+1}, R_{2i})  =
    \begin{cases}
      (C_{2i+1},C_{2i}) & \text{if } C_{2i} \text{ is even}\\
      (C_{2i+1}+1,C_{2i}-1) & \text{if } C_{2i} \text{ is odd}\\
    \end{cases}
\]
This definition matches the duality of type B.

Then the Springer representation attached to $\cO$ is given by
\[
  \tau_L\times \tau_R = (c_{2a+1}, c_{2a-1}, \cdots, c_1)\times (c_{2a},
  c_{2a-2}, \cdots, c_0)
\]
where
\[
  \begin{split}
    C_{2i+1} & = 2 c_{2i+1} - \epsilon_i\\
    C_{2i} & = 2 c_{2i} + \epsilon_i\\
    \epsilon_i & = 0 \text{ or } 1
  \end{split}
\]
The non-special representations are given by
\[
  (c_{2i+1}, c_{2i}) \leftrightarrow (c_{2i}, c_{2i+1})
\]


\section{Matching doc-r-c diagrams and local systems}
Let $\bfG = \Sp, \Mp, \rO$, and $\cO\in \Nil(\bfG)$ is a complex nilpotent orbit.

Let $\drc(\bfG,\cO)$ be the set of dot-r-c diagrams parameterizing unipotent
representations of the real forms of $\bfG$.


The following set of orbit are essential to us:
\begin{defn}
  A nilpotent orbit $\cO\in \Nil(\bfG)$ is called \noticed if
  \begin{itemize}
    \item when $G$ is type C,
          $\cO$ is special and it has columns
          \[
            [C_{2k},C_{2k-1},\cdots, C_{1},C_{0}, C_{-1}=0]
            \quad \text{such that } C_{2i+1}>C_{2i} \text{ for } i=0, \cdots, k-1.
          \]
           Note that, by the
          definition of special orbit,
          we have $C_{2i}\equiv C_{2i-1} \pmod{2}$, $C_{2i} = C_{2i-1}$ if
          $C_{2i}$ is odd for $i = 1, \cdots k$.
    \item when $G$ is type M,
          $\cO$ is metaplecitic special and it has columns
          \[
            [C_{2k},C_{2k-1},\cdots, C_{1},C_{0}=0]
            \quad \text{such that } C_{2i+1}>C_{2i} \text{ for } i=0, \cdots, k-1.
          \]
          Note that, by the
          definition of metaplectic special,
          we have $C_{2i}\equiv C_{2i-1} \pmod{2}$, $C_{2i} = C_{2i-1}$ if
          $C_{2i}$ is even
          for $i = 1, \cdots k$.
    \item when $G$ is type D, $\cO$ is special and it has columns
          \[
            [C_{2k+1},C_{2k},C_{2k-1},\cdots, C_{1},C_{0}, C_{-1}=0]
            \quad \text{such that } C_{2i+1}>C_{2i} \text{ for } i=0, \cdots, k.
          \]
          Note that, the
          condition is equivalent to $\DD(\cO)$ is \noticed of type C, i.e.
           $C_{2i}\equiv C_{2i-1} \pmod{2}$ and $C_{2i} = C_{2i-1}$ if $C_{2i}$
           is odd for $i = 1, \cdots k$.
    \item when $G$ is type B, $\cO$ is special and it has columns
          \[
            [C_{2k+1},C_{2k},C_{2k-1},\cdots, C_{1},C_{0}, C_{-1}=0]
            \quad \text{such that } C_{2i+1}>C_{2i} \text{ for } i=0, \cdots, k.
          \]
          Note that, the
          condition is equivalent to $\DD(\cO)$ is \noticed of type M, i.e.
           $C_{2i}\equiv C_{2i-1} \pmod{2}$ and $C_{2i} = C_{2i-1}$ if $C_{2i}$
           is even for $i=1, \cdots, k$.
  \end{itemize}
  Under the above notation, $\cO$ is called \ess if it is \noticed and
  $C_{2i}=C_{2i-1}$ in addition.
  Let $\nNil(\star)$ and $\eNil(\star)$ be all the sets of \noticed and \ess
  orbits for type $\star=$ B,C,D,M respectively.

  The \ess descent of orbit $\cO\in \eNil(\star)$ is defined in the following
  way (under the above notation):
  \begin{itemize}
    \item when $\star$ is B or D, $\eDD(\cO):= \DD(\cO)$;
    \item when $\star$ is C or M,
          \[
          \eDD(\cO):= \begin{cases}
            \DD(\cO) & \text{if $C_{2k}$ is even and $\star=B$},\\
            \DD(\cO) & \text{if $C_{2k}$ is odd and $\star=M$},\\
            [C_{2k-1}+1, C_{2k-2},C_{2k-3},\cdots, C_{0}]  & \text{otherwise}.
          \end{cases}
          \]
          Note that $\cO \mapsto \eDD(\cO)$ is a good generalized descent in the last case.
  \end{itemize}
\end{defn}

Let $\LS(\cO)$ denote the Grothendieck group of $\wtbfK$-equivariant local
systems on $\cO$ for various real forms of $\bfG$.
The following is our main theorem regards of the counting.
\begin{thm} \label{thm:count}
  Suppose $\cO$ is \noticed. Then the map
  \[
    \begin{tikzcd}
      \Ch\colon \Unip(\cO) \ar[r] & \LS(\cO)
    \end{tikzcd}
  \]
  is injective.
  Moreover, every element in $\Unip(\cO)$ can be obtained by iterated theta lifting.
\end{thm}

By the counting argument above, it is suffice to establish the above theorem for
\ess orbits.


Note that type B dot-r-c diagram is extend by mark with $A$ or $B$.
Suppose $\bfG$ is of type B or D, we define a map
\[
  \begin{tikzcd}[column sep=0em, row sep=0em]
    \Sign \colon&   \edrc(B)\sqcup \edrc(D) \ar[rr]& \hspace{3em} & \bN\times \bN \\
    & \uptau \ar[rr, maps to]& & (p,q)
  \end{tikzcd}
\]
Here $(p,q)$ is the signature of the orthogonal group $\rO(p,q)$ corresponding
to the dot-r-c diagram $\uptau$, which is calculated by the following formula
\[
  \begin{split}
    p &= \# \bullet+ 2 \# r + \# c + \# d + \# A\\
    q &= \# \bullet+ 2 \# s + \# c + \# d + \# B\\
  \end{split}
\]

For $\set{\star,\star'} = \set{C,D}, \set{B,M}$,
we define maps
\[
  \begin{tikzcd}[column sep=0em, row sep=0em]
    \eDD \colon & \edrc(\star)\sqcup \edrc(\star') \ar[rr] & \hspace{3em} &
    \left(\edrc(\star')\sqcup \edrc(\star) \right)\times \bZ/2\bZ\\
    & \uptau \ar[rr, maps to] &  & (\uptau', \upepsilon)
  \end{tikzcd}
\]
such that
\begin{enumC}
  \item $\uptau$ is in $\edrc(\cO)$ where $\cO$ is essential;
  \item $\uptau'\in \eNil(\cO')$ where $\cO': = \eDD(\cO)$;
  \item $\upepsilon = 0$ if $\cO\mapsto \cO'$.
\end{enumC}

Using $\eDD$, we now define two maps $\cL$ and $\uppi$ which are fitted into the
following diagram.
\[
  \begin{tikzcd}[column sep=0em, row sep=4em]
    & & & & \LS \\[-4em]
    & & &   \cL_{\uptau}& \\
    &  \uptau \ar[rr,maps to] \ar[rru,maps to]& \hspace{3em} & \pi_{\uptau}  \ar[u,maps to]&  \\[-4em]
    \edrc \ar[rrrr,"\uppi"] \ar[rrrruuu, "\cL", bend left]  & & & & \Unip \ar[uuu, "\Ch"]
  \end{tikzcd}
\]
Here
\begin{enumC}
  \item $\uptau$ is in $\edrc(\cO)$ where $\cO$ is essential;
  \item $\pi_{\uptau}$ is a unipotent representation of $\wtG$ attached to $\cO$,
  \item $\wtG$ is $\Mp(2n,\bR)$ or $\Sp(2n,\bR)$ if $\uptau$ is a diagram
  of type C or M;
  \item $\wtG$ is $\rO(p,q)$ if $\uptau$ is of type B or C and $(p,q)=\Sign(\uptau)$;
  \item $\cL_{\uptau} = \Ch(\pi_{\uptau})$.
\end{enumC}
The definition of $\cL$ and $\uppi$ are inductive with respect to the number of
columns of $\cO$:
\begin{enumS}
  \item Suppose $\cO=[C_{0}=0]$ is type C or M. Then
  $\edrc(\cO) = \set{\uptau_{0}}$ is a singleton with $\uptau_{0}=(\emptyset,\emptyset)$.
  Define $\pi_{\uptau} = \bfone$ the trivial representation of the trivial group
  $\Sp(0,\bR)=\Mp(0,\bR) = \bfone$. Here we assume the theta lift of $\bfone$ to
  $\rO(p,q)$ for any signature $p,q$ is the trivial representation $\bfone_{p,q}^{+,+}$ of $\rO(p,q)$.
  \item We now assume $\uptau \in \edrc(\cO)$ and
  $(\uptau', \upepsilon) = \eDD(\tau)\in \edrc(\cO')$ where $\cO' = \eDD(\cO)$.
  \begin{enumS}
    \item Suppose $\cO$ is type C or M. Then $\wtG = \Sp(2n,\bR)$ or
    $\Mp(2n,\bR)$, and $\pi_{\uptau'}$ is a unipotent
    representation of $\rO(p,q)$ attached to $\uptau'$ with $n=\abs{\uptau}$ and $(p,q) = \Sign(\uptau')$.
    \[
      \pi_{\uptau} := \Thetab_{\wtG',\wtG}(\pi_{\uptau'}\otimes (\bfone_{p,q}^{-,-})^{\upepsilon}).
    \]
    Here $\bfone_{p,q}^{-,-}$ is the determinant character of $\rO(p,q)$.
    \item  Suppose $\cO$ is type D or B. Then $\wtG = \rO(p,q)$ and $\pi_{\uptau'}$ is a unipotent
    representation of $\wtG' = \Sp(2n,\bR)$ or $\Mp(2n,\bR)$ attached to
    $\uptau'$ with $n=\abs{\uptau'}$ and $(p,q) = \Sign(\uptau)$.
    \[
      \pi_{\uptau} := \Thetab_{\wtG',\wtG}(\pi_{\uptau'})\otimes (\bfone_{p,q}^{+,-})^{\upepsilon}.
    \]
  \end{enumS}
\end{enumS}

\Cref{thm:count} is a immediate consequence of the following
\begin{prop}
  %The map $\cL$ is a well-defined injection. More precisely, we have
  We have
  \begin{enumS}
    \item $\cL_{\uptau}$ is non-zero for each $\uptau\in \edrc$.
    \item For each \ess orbit $\cO$ of type C or M, the following is an
    injection
    \[
      \begin{tikzcd}[row sep=0em]
        \edrc(\cO) \ar[r] & \LS(\cO) \\
        \uptau \ar[r,maps to] & \Ch(\pi_{\uptau}) = \cL_{\uptau}
      \end{tikzcd}
    \]
    \item For each \ess orbit $\cO$ of type D or B, the following is an
    injection
    \[
      \begin{tikzcd}[row sep=0em]
        \edrc(\cO)\times \bZ/2\bZ \ar[r] & \LS(\cO)\\
        (\uptau,\upepsilon) \ar[r,maps to] & \Ch(\pi_{\uptau}\otimes (\bfone_{p,q}^{-,-})^{\upepsilon})
        = \cL_{\uptau}\otimes (\bfone_{p,q}^{-,-})^{\upepsilon}
      \end{tikzcd}
    \]
  \end{enumS}
\end{prop}

\subsection{The algorithm to define $\eDD$}
\def\taulf{\uptau_{L,0}}
\def\tauls{\uptau_{L,1}}
\def\taurf{\uptau_{R,0}}
\def\taurs{\uptau_{R,1}}

\def\tauplf{\uptau'_{L,0}}
\def\taupls{\uptau'_{L,1}}
\def\tauprf{\uptau'_{R,0}}
\def\tauprs{\uptau'_{R,1}}
\def\tail{\mathrm{tail}}

From now on, we assume
\[
  \uptau = ([\taulf,\tauls,\cdots],[\taurf,\taurs,\cdots])
\]
and $(\uptau', \epsilon) := \eDD(\uptau)$
\[
  \uptau' = ([\tauplf,\taupls,\cdots],[\tauprf,\tauprs,\cdots])
\]

The definition of sign $\upepsilon$
\begin{enumS}
  \item Suppose $\uptau$ is type D or B, $\upepsilon=0$ if and only if
  $\# d (\uptau)> \#d(\uptau')$. This means:
  \begin{enumS}
    \item When $\uptau$ is type D,  $\tail(\taulf) = d$.
    \item When $\uptau$ is type B,   $\tail(\taurf) = d$ or
    $(\tail(\taulf),\tail(\taurf),\tail(\taurs)) = (c,r,d)$.
  \end{enumS}
  \item Suppose $\uptau$ is type C or M, $\upepsilon=0$ if and only if
  $\abs{\taulf}\geq \abs{\taurf}$.
\end{enumS}



\appendix
\section{The coherent continuation representation}
We specialize the atlas setting to the classical group case.
Let $\Gamma = \Gal(\bC/\bR) = \gen{\gamma}$.

\subsection{Type C}
References \cite[Example~5.11]{AC}.  The complex group is
$\bfG=\Sp(2n,\bC)$. Use the standard coordiante of characters and cocharacters.
Fix a Cartan subgroup $\bfH$.  The coweight lattice $X_*=\bZ^n$.  The coroot
lattice
\[ P^\vee = \bZ^n \cup (\bZ^n+\half)^n = \set{(a_1, \cdots, a_n)| a_i \text{ are
      all integers or are all half integers}}
\]

The outer involution $\gamma=1$. The extended group is
$\bfG^\Gamma = \bfG\times \Gamma$ ($\Gamma=\gen{\gamma}$).

$Z(\bfG)= \set{\pm 1}$.  The strong real forms are parameterized by
\[
  ( (\half [\bZ^n\cup (\bZ^n+\half)])/\bZ^n )/W
\]
The representative $\frac{1}{4}(1,\cdots, 1)$ corresponds to the split real form
$G=\Sp(2n,\bR)$.

The representative
\[\frac{1}{2}(\underbrace{1, \cdots, 1}_p,\underbrace{0, \cdots, 0}_q)
\]
correspondes to $\Sp(p,q)$ \def\tcX{\widetilde{\cX}} Let
\begin{align*}
  \bfN^\Gamma &= \Norm_{\bfG^\Gamma}(\bfH)\\
  \cI  & = \set{\xi \in \bfG^\Gamma\setminus \bfG | \xi^2 \in Z(G)}\\
  \tcX &= \cI \cap \bfN^\Gamma\\
  \cX & = \tcX / \bfH \quad \text{\color{red} (The $\bfH$ action is by conjugation.)} \\
  \cI_W & = \set{\tau\in W^\Gamma - W | \tau^2 =1}\\
\end{align*}

The set $\cI/G$ parameterized the strong inner forms of $\bfG$ (corresponding to
the outer automorphism $\gamma$).
$\Sp(2n,\bR)$ corresponds to $\xi_0=\diag(i,\cdots, i, -i, \cdots, -i)$.
Let $z_0 = \xi_0^2 = -1$. 

\subsubsection{}
Note that we have $\gamma$ act on $\bfG$ trivially.  The set $\cI_W$ is
identified with Cartan involutions of $H$ (for the inner class $\gamma$)\cite[9.13b]{AC}
\[
  \begin{tikzcd}
  \tcX\ar[r]&  \cX\ar[r,"p"] & \cI_W\ar[r,<->,"1-1"] & \set{\theta_{x,H} := \Ad_x}
  \end{tikzcd}
\]
We may identify $\cI_W$ with $\set{\tau \in W| \tau^2=1}$.
Then $\cI_W/W$ is identified with the $K$-conjugacy class of $\theta$-stable
Cartan in $G=\Sp(2n,\bR)$. 

Note $W = W_n = S_n\ltimes \set{\pm 1}^n$.


Let $s_{i,j} = s_{e_i -e_j}$, $s_i = s_{2e_i}$ in the Weyl group $W$.  Define $w_{i,j}$ and $w_{i}$ (lift of $s_i$) also represent its lift in
$\bfN = \Norm_{\bfG}(\bfH)$.
\[
  w_{i,j}=
  \left(
    \begin{array}{cccc;{2pt/2pt}cccc}
      %& i      & j  &   &  &  &   &    & \\
      \ddots &    &   &        &        &    &   &        \\
             & 0  & 1 &        &        &    &   &        \\
             & -1 & 0 &        &        &    &   &        \\
             &    &   & \ddots &        &    &   &        \\
      \hdashline[2pt/2pt]
             &    &   &        & \ddots &    &   &        \\
             &    &   &        &        & 0  & -1          \\
             &    &   &        &        & 1 & 0 &        \\
             &    &   &        &        &    &   & \ddots \\
    \end{array} 
  \right)
\]

\[
  w_{i}=
  \left(
    \begin{array}{ccc;{2pt/2pt}ccc}
      %& i      & j  &   &  &  &   &    & \\
        1   &   &    &    &        &           \\
            & 0   &  &      &      1  &           \\
            &   & 1  &    &        &           \\
      \hdashline[2pt/2pt]
            &   &    &  1 &        &           \\
            &  -1  &  &      &      0  &           \\
            &   &    &    &        & 1         \\
    \end{array} 
  \right)
\]

Therefore, $\cI_W$ has the following types of representatives: For each set of
numbers $p,t, r$ such that $2p+t+r =n$.
\[
  w_{p,t}= w_{1,2}w_{3,4}\cdots w_{2p-1,2p}\ltimes  w_{2p+1}\cdots w_{2p+t}.
\]

The $\theta$-stable Cartan corresponding to $w_{p,t}$ is isomorphic to
\[
  H(\bR):=H^{w_{p,t}, \mathrm{conj}}= (\bC^\times )^p\times (\bR^\times)^t\times \bT^r.
\]
Now $H^\vee(\bR) = (\bC^\times)^p \times \bT^t\times (\bR^\times)^r$. 

For $\tau\in \cI_w$, $z\in Z(\bfG)$, let  
\begin{align*}
  \cX_0 &= \set{x\in \cX | \text{$x$ is $\bfG$-conjugate to $\xi_0$}} \\
        & = \cX(-1) = \set{x\in\cX | x^2 = -1}\\
  \cX_\tau(z) & = \set{x\in \cX | p(x) = \tau, x^2= z}
\end{align*}
Note that $\cX_\tau(z)$ has a simply transitive action by $\bfH_{-\tau}/A_\tau$
\cite[Prop~11.2]{AC}
\[
  |\cX_\tau(z)| = 2^r
\]

Since $\overline{p}\colon \cX_0/W \to \cI_W/W$ is bijective\cite[Prop~12.12]{AC}, we have  
\[
  \cX_0 = \bigcup_{\tau \in \cI_W/W} W\cdot \cX_\tau(z_0).
\]

Let $W^\tau = \Stab_W(\tau)$. 
Let $h_k = \diag(1, \cdots, 1, \underbrace{i}_{k-th},1,
\cdot, 1)\in (\bC^\times)^n = \bfH \subset \bfG$, then $h_k^{-1}\in \bfH$ has the same
shape with $i$ replaced by $-i$.
Now fix $\tau = w_{p,t}$. We can list elements in $\cX_\tau(z_0)$ as the
following: Fix $a+b=r$, 
let
\[
  x_{p,t,a,b} = w_{1,2}w_{3,4}\cdots, w_{2p-1,2p} w_{2p+1}\cdots w_{2p+t}
  h_{2p+t+1}\cdots h_{2p+t+a} h_{2p+t+a+1}^{-1}h_{2p+t+a+b=n}^{-1},
\]
then
\[
  \cX_\tau(z_0) = \bigcup_{a+b=r} W^\tau \cdot x_{p,t,a,b}
\]

Fix the strong real form $x = x_{p,t,a,b}$. Then $\bfK= \bfG^\xi$ is the
complexification of the maximal compact subgroup of $G$.  Now the real Weyl
group
\[
  \begin{split}
    W(\bfG(\bR), \bfH(\bR)) &\cong  W(K,H) = \Stab_W(x) \\ 
    &\cong (W_p \ltimes W(A_1)^p)\times S_r\times W_{t}.
  \end{split}
\]
Here $W(A_i)^p = \gen{w_{1,2}, w_{3,4},\cdots, w_{2p-1, 2p}}$
Here $W_p$ is identified with $\triangle W_p \subset W_p\times W_p$ where
$W_p\times W_p$ is the subgroup of $W_{2p}$ via embedding
\[
  (a_1,a_3, \cdots, a_{2p-1})\times (a_2, a_4, \cdots, a_{2p}) \mapsto (a_1, a_2,
a_3, a_4, \cdots, a_{2p-1},a_{2p}).
\]

There is the list of real, imaginary, and complex root systems.
\begin{des}
\item [Real roots ] % $\pm (e_1-e_2), \cdots, \pm(e_{2p-1}-e_{2p}), \pm
  % (e_{2p+1}- e_{2p+2}), \cdots, \pm (e_{2p+t-1}-e_{2p+t})  $
  \[
   R^\bR= \set{\pm (e_{2i-1}+e_{2i}) | i=1, \cdots, p}\cup \set{\pm
      (e_{2p+i}-e_{2p+j}), \pm (e_{2p+i}+e_{2p+j})| i,j =1, \cdots, t}
  \]
  It is type $(A_1)^p\times C_t$.
  \[
    \ckrho^\bR = \half(\underbrace{1,-1,\cdots, 1,-1}_{2p\text{-terms}},
   \underbrace{2t-1,2t-3, \cdots, 1}_{t\text{-terms}}, 0,\cdots, 0  )
  \]
\item [Imaginary roots]
  \[
   R^{i\bR}= \set{\pm (e_{2i-1}-e_{2i}) | i=1, \cdots, p}\cup \set{\pm
      (e_{2p+t+i}-e_{2p+t+j}), \pm (e_{2p+t+i}+e_{2p+t+j})| i,j =1, \cdots, r}
  \]
  It is type $(A_1)^p\times C_r$
  \[
    \ckrho^{i\bR} = \half(\underbrace{1,1,\cdots, 1,1}_{2p\text{-terms}} ,
    0,\cdots, 0, \underbrace{2r-1,2r-3, \cdots, 1}_{r\text{-terms}} )
  \]
  Take $x_{p,t,a,b}$,
  The imaginary compact roots are:
  \[
    \begin{split}
    R_{ic} = & \set{e_{2p+t+i}-e_{2p+t+j}| 1\leq i,j \leq a}
    \cup \set{e_{2p+t+i}-e_{2p+t+j}| a+1\leq i,j \leq a+b=r}\\
    & 
    \cup \set{\pm ( e_{2p+t+i}+e_{2p+t+j})| 1\leq i\leq a, a+1\leq j \leq a+b=r}
  \end{split}
\]
Clearly $R_{ic}$ is of type $A_{r}$.
$W(R_{ic}) = S_r$
See \cite[3.13]{V4}
\[
  W_2(R_{ic}):= \set{w\in W(R^{i\bR})| w R_{ic}=R_{ic}} = W(A_1)^p\times
W(R_{ic}).\] 
\item [Complex root system $R^\bC$]
  Note that $R^\bC  := \set{\alpha\in R|
    \inn{\alpha}{\ckrho^\bR}=\inn{\alpha}{\ckrho^{i\bR}}=0}$
  Then
  \[
    R^\bC = \set{\pm \alpha^C_{i,j}:=\pm (e_{2i-1}-e_{2j-1})|1\leq i<j\leq p }
    \cup \set{\pm \tau\alpha^C_{i,j} = \mp (e_{2i}-e_{2j})|1\leq i<j\leq p }.
  \]
   Now $W(R^\bC)^\tau = \gen{s_{\alpha_{i,i+1}}s_{\tau \alpha_{i,i+1}}| i=1,
     \cdots, p}$ is of type $A_{p-1}$.
\end{des}

Now the real Weyl group is of the form
\[
  W(\bfK,\bfH) = W(R^\bC)^\tau\ltimes (W(R^{\bR})\times W(\bfG^\bfA,\bfH)) 
\]
Here $W_{ic} < W(\bfG^\bfA,\bfH)< W(R^{i\bR})$, and
\[
  W(\bfG^\bfA,\bfH) = W_{ic}\ltimes A(H). 
\]
Note that $G=\Sp(2n,\bR)$ is large, $W(\bfG^\bfA,\bfH) = W_2(R_{ic})$.

In summary, 
\begin{equation}\label{eq:W_KH.C}
  W(\bfK,\bfH) =\underbrace {\overbrace{ S_p }^{(W^C)^\tau} \ltimes
    \overbrace{W(A_1)^p}^{\text{real}}}_{W_p }\ltimes
  \underbrace{\overbrace{W(A_1)^p}^{im}}_{A(H)}\times
  \overbrace{S_r}^{ic} \times \overbrace{W_{t}}^{real}
\end{equation}

\subsection{Coherent continuation representation}
The full parameter space
\[
  \cZ= \set{(x,y)|x\in \cX, y\in \ckcX, \theta_y=-\theta_x^T}.
\]
Since we consider the catogery of $G=\Sp(2n,\bR)$-modules at infinitesimal character
$\rho(G)$, we choose $\ckzz_0 = 1$.  
Now the full parameter space of $\Pi(G,\rho(G))$ is given by
\[
  \begin{split}
    \cZ_0 = & \set{(x,y)| x\in \cX(-1), y\in \ckcX(1), \theta_y = -\theta_x^T }\\
    = & \bigcup_{\tau\in \cI_W/W} \set{(x,y)| x \in \cX_\tau(-1), y\in \ckcX(1),
      \theta_y=-\theta_x^T}\\
  \end{split}
\]

We now decompose $\cZ_0$ into $W$-orbits. 
Note that $\cX_0/W\rightarrow \cI_W/W$ is bijective
We choose $x_{p,t,r,0}$ as the representative of $\cX_0/W\cong \cI_W/W$. 
Note that $\Stab_W(x_{p,t,r,0})$ is calculated in \eqref{eq:W_KH.C}.

Let
\[
  \ckcX^{x} = \set{y\in \ckcX(1)| \theta_y = -\theta_x^T}= \ckcX_{\cktau}(1).
\]
Here that $\cktau\in \cI_\ckW$ is determined by $\theta_y$.
Therefore, $|\ckcX^{x}| = 2^r$ (here there real
tori $H(\bR)$ has $t$ copies of $\bR^\times$).  Note that
$W^{\cktau} = (W^C)^\tau\ltimes(W(R^{i\bR})\times W(R^{\bR})$, and except the
factor  
$W(R^{\bR})$ (the dual imaginary root group), other factors has trivial action
on $\cX_\cktau(1)$. 
We could choose the following representatives of $\ckcX^x/W(R^{\bR})$ (it has
$t+1$ members): for $a+b=t$, define 
\[
  y_{p,a,b,r} := s_{1,2}\cdots s_{2p-1,2p} \ckhh_{2p+1}\cdots
  \ckhh_{2p+a}\ckss_{2p+t+1}\cdots \ckss_{n}.    
\]
Here $\ckhh_{j} = \diag(1, \cdots,1, \underbrace{-1}_{j},1, \cdots, 1) $.
Now
\[
  \Stab_{W(R^{\bR})}(y_{p,a,b,r}) = W_a\times W_b.
\]
Therefore, 
\[
  \begin{split}
  \cZ_0 = & \bigcup_{\tau\in \cI_W/W} W \cdot \set{(x_{p,t,r,0},y)| y \in \ckcX_\cktau(-1), }  \\
  = & \bigcup_{p,a,b,r} W\cdot (x_{p,t,r,0},y_{p,a,b,r})
  \end{split}
\]
Moreover, the stabilizer of $(x_{p,t,r,0},y_{p,a,b,r})$ is
\[
  (W^C)^\tau \ltimes (W(\bfG^\bfA,\bfH)\times \Stab_{W(R^{\bR})}(y_{p,a,b,r}))
   = W_p\ltimes \underbrace{W(A_1)^p\times S_r}_{W(\bfG^\bfA,\bfH)}\times   W_a\times W_b
\]

By taking filtration according to the dimension of real root system, the Cayley
transform could be ignored.  Then the coherent continuation representation is
essentially given by the cross action.  For the stabilizer, the
$W(\bfG^\bfA,\bfH)$ factor acts by $-1$ and other factors act trivially.  This
leads to the formula \eqref{eq:CC.C}. {\color{red} Though, the notations are not
  matching.}

\section{Type D}
Now let $\bfG_0=\SO(2n,\bC)$. Let $\bfG = \rO(2n,\bC)$.  Fix a Cartan subgroup
$\bfH$ and Borel $\bfB$.  Let $w_c$ be the element in $\Norm_{\bfG}(\bfH)$
fixing $\bfB$.  It is the element sending $e_n$ to $-e_n$. (The upper triangler
matrices form the Borel.)
\[
  \begin{pmatrix}
    I & & &  \\
    &0 & 1 & \\
    &1 &0  & \\
    & & & I\\
  \end{pmatrix}
\]

Now $W := W(\bfG)$ and $W' := W(\bfG_0)$ is the subgroup of $W$ of index $2$.

The cocharacter lattice $X_*=\bZ^n$.  The coweight lattice
\[ P^\vee = \bZ^n \cup (\bZ^n+\half)^n = \set{(a_1, \cdots, a_n)| a_i \text{ are
      all integers or are all half integers}}
\]
\trivial{ Note that the root lattice of $\SO(2n,\bC)$ and $\Sp(2n,\bC)$ are the
  same $=\set{(a_i)|\sum a_i \equiv 0 \pmod{2}}$.  }

Note that $Z(\bfG)=Z(\bfG_0)=\set{\pm 1}$

We consider the real forms $\SO(p,q)$.  When $p,q$ are even integers, we take
the outer involution $\gamma = \bfone$, there group with $\SO^*(2n)$ form an
inner class.  When $p,q$ are odd integers, the outer involution $\gamma=1$,
these gives another inner class.

The set elements $\set{\xi|\xi^2\in Z(\bfG)}$ in $\SO(2n,\bC)$ and
$\SO(2n,\bC)w_c$ could be identified with the set of inolutions in
$\rO(2n,\bC)$.

When $\xi^2 =1$, the $\bfG_0$-conjugacy class is the same as the
$\bfG$-conjugacy class.

When $\xi^2=-1$, there is only one $\bfG$-orbit which breaks into two
$\bfG_0$-orbits.  In fact, for each $\xi$ under the suitable orthonormal basis,
it has the form
$\xi_{p,q}=\diag(\underbrace{1, \cdots, 1}_p,\underbrace{-1,\cdots,-1}_q )$
($p+q=2n$, $\xi_{p,q}^2=1$) or
$\xi_0 =\diag(\underbrace{i,\cdots, i}_n,\underbrace{-i,\cdots, -i}_n)$ under a
``hyperbolic'' basis. So there is an element in $\rO(2n,\bC)-\SO(2n,\bC)$ fixing
$\xi$ under the conjugation ation.


Suppose $\gamma=e$.  The representative $\xi_0$ corresponds to the real form
$G=\SO^*(2n)$. Clearly $z_0 = \xi_0^2 = -1$ in this case.  The representative
$\xi_{2s,2t}$ correspondes to $\SO(2s,2t)$. Clearly, $z = \xi_{2s,2t}^2=1$ in
this case.

Define
\[
  \cI(\bfG) = \cI(\bfG_0,e)\sqcup \cI(\bfG_0, w_c) = \set{\xi\in \bfG| \xi^2\in
    Z(\bfG)}
\]
By the above discussion, we have
\[
  \begin{tikzcd}
    \cI(\bfG)/\bfG \ar[r,<->,"1-1"]& \Set{\rO(p,q), \rO^*(2n)| p+q=2n}
  \end{tikzcd}
\]

Define
\[
  \begin{split}
    \tcX &= \cI(\bfG)\cap \Norm_{\bfG}(\bfH) = \cX(\bfG) = \cX(\bfG_0,e)\sqcup
    \cI(\bfG_0,w_c)\\
    & = \set{x\in \Norm_{\bfG}(\bfH)|x^2\in Z(G)}\\
    \cX &= \tcX/\bfH
  \end{split}
\]

Define
\[
  \begin{split}
    \cI_W &= \set{w\in W|w^2 =1} = \cI_{W'}(\gamma=e)\sqcup \cI_{W'}(\gamma=w_c)
  \end{split}
\]
Define $w_{i,j}$ and $w_{i}$ be elements in $\bfN = \Norm_{\bfG}(\bfH)$, let
$s_{i,j}$ and $s_i$ be their image in $W$:
\[
  w_{i,i+1}= \left(
    \begin{array}{cccc;{2pt/2pt}cccc}
      % & i & j & & & & & & \\
      \ddots &    &   &        &        &    &   &        \\
        & 0  & 1 &        &        &    &   &        \\
        & 1 & 0 &        &        &    &   &        \\
        &    &   & \ddots &        &    &   &        \\
      \hdashline[2pt/2pt]
        &    &   &        & \ddots &    &   &        \\
        &    &   &        &        & 0  & 1          \\
        &    &   &        &        & 1 & 0 &        \\
        &    &   &        &        &    &   & \ddots \\
    \end{array} 
  \right)
\]

\[
  w_{i}= \left(
    \begin{array}{ccc;{2pt/2pt}ccc}
      % & i & j & & & & & & \\
      1   &   &    &    &        &           \\
        & 0   &  &      &      1  &           \\
        &   & 1  &    &        &           \\
      \hdashline[2pt/2pt]
        &   &    &  1 &        &           \\
        &  1  &  &      &      0  &           \\
        &   &    &    &        & 1         \\
    \end{array} 
  \right)
\]

Let
\[
  w'_{i,i+1}= \left(
    \begin{array}{cccc;{2pt/2pt}cccc}
      % & i & j & & & & & & \\
      \ddots &    &   &        &        &    &   &        \\
        & 0  & -1 &        &        &    &   &        \\
        & -1 & 0 &        &        &    &   &        \\
        &    &   & \ddots &        &    &   &        \\
      \hdashline[2pt/2pt]
        &    &   &        & \ddots &    &   &        \\
        &    &   &        &        & 0  & -1          \\
        &    &   &        &        & -1 & 0 &        \\
        &    &   &        &        &    &   & \ddots \\
    \end{array}.
  \right)
\]
Then $w'_{i,i+1}= w_i w_{i,i+1} w_i^{-1}$.

Under $W$-action, $\cI_W/W$ is parameterized by $s,a,b$ such that $2s+a+b=n$.
Representatives are given by:
\[
  w_{p,a,b}= w_{1,2}w_{3,4}\cdots w_{2s-1,2s}\ltimes w_{2s+1}\cdots w_{2s+a}.
\]


Let $\tau = w_{s,a,b}$.  Then
\[
  \begin{split}
    W^\tau & = S_s \ltimes ( \gen{w_{2k-1}w_{2k}|k=1,\cdots, s}\times
    \gen{w_{2k-1,2k}|k=1,\cdots, s} )\times W_a\times W_b \\
    &\cong S_s\times(\bZ_2^s\times \bZ_2^s) \times W_a\times W_b.
  \end{split}
\]
Therefore, the twisted involution orbit $W\cdot \tau$ has size
\[
  \frac{2^n n!}{2^{2s} s! 2^a a! 2^b b!} = \frac{n!}{s!a!b!}
\]

When $a=b=0$, the $W$-orbit $W\cdot w_{p,a,b}$ splits into two $W'$ orbits
$W' \cdot w_{p,a,b}$ and $W'\cdot w'_{p,a,b}$ where
\[
  w'_{p,a,b}= w'_{1,2}w_{3,4}\cdots w_{2s-1,2s}\ltimes w_{2s+1}\cdots w_{2s+a}.
\]

The $\theta$-stable Cartan corresponding to $w_{s,a,b}$ is isomorphic to
\[
  H(\bR):=H^{w_{s,a,b}, \mathrm{conj}}= (\bC^\times )^s\times
  (\bR^\times)^a\times \bT^b.
\]
Now $H^\vee(\bR) = (\bC^\times)^s \times \bT^a\times (\bR^\times)^b$.

Consider the map
\[
  \begin{tikzcd}
    \tcX\ar[r]& \cX := \tcX/\bfH \ar[r, "p"] & \cI_W
  \end{tikzcd}
\]

Let
\[
  h_i = (1,\dots,1,\underbrace{-1}_i,1,\cdots,1) \in (\bC^\times)^n = \bfH
\]


Now the $W$-orbit on $\cX$ is given by the representatives
\[
  x_{s,a,p,q} = w_{1,2}w_{3,4}\cdots w_{2s-1,2s} w_{2s+1}\cdots w_{2s+a}
  h_{2s+a+1}\cdots h_{2s+a+p}.
\]
Now
\begin{equation}\label{eq:Xtau.D}
\cX_{\tau} = \bigsqcup_{p+q=b} W^\tau \cdot x_{s,a,p,q}.
\end{equation}
Moreover, only $W_b$ factor of $W^\tau$ acts on $\cX_{\tau}$ non-trivially. 

Observe that $x=x_{s,a,p,q}$ gives the realform $\rO(2s+a+2p,2s+a+2q)$.

Here is the list of real, imaginary, and complex root systems.
\begin{des}
\item [Real roots ] % $\pm (e_1-e_2), \cdots, \pm(e_{2p-1}-e_{2p}), \pm
  % (e_{2p+1}- e_{2p+2}), \cdots, \pm (e_{2p+t-1}-e_{2p+t}) $
  \[
    R^\bR= \set{\pm (e_{2k-1}-e_{2k}) | k=1, \cdots, s}\cup \set{e_{2s+k}\pm
      e_{2s+l}| k\neq l =1, \cdots, a}
  \]
  It is type $(A_1)^s\times D_a$.
  \[
    \ckrho^\bR = \half(\underbrace{1,-1,\cdots, 1,-1}_{2s\text{-terms}},
    \underbrace{2a-2,2a-4, \cdots, 0}_{a\text{-terms}}, 0,\cdots, 0 )
  \]
\item [Imaginary roots]
  \[
    R^{i\bR}= \set{\pm (e_{2k-1}+e_{2k}) | k=1, \cdots, s}\cup
    \set{e_{2s+a+k}\pm e_{2s+a+l}| k\neq l =1, \cdots, b}
  \]
  It is type $(A_1)^s\times D_b$
  \[
    \ckrho^{i\bR} = \half(\underbrace{1,1,\cdots, 1,1}_{2s\text{-terms}} ,
    0,\cdots, 0, \underbrace{2b-2,2b-4, \cdots, 0}_{b\text{-terms}} )
  \]
  The imaginary compact roots are:
  \[
    \begin{split}
      R_{ic} = & \set{e_{2s+a+i}\pm e_{2s+a+j}| 1\leq i,j \leq p}\\
      & \cup \set{e_{2s+a+i}\pm e_{2s+a+j}| p+1\leq i,j \leq p+q=b}
    \end{split}
  \]
  Clearly $R_{ic}$ is of type $D_p\times D_q$.  $W(R_{ic}) = W_p\times W_q$
\item [Complex root system $R^\bC$] Note that
  $R^\bC := \set{\alpha\in R|
    \inn{\alpha}{\ckrho^\bR}=\inn{\alpha}{\ckrho^{i\bR}}=0}$ Then
  \[
    \begin{split}
      R^\bC &= \set{\pm \alpha^C_{i,j}:=\pm (e_{2i-1}-e_{2j-1})|1\leq i<j\leq s
      }
      \cup \set{\pm \tau\alpha^C_{i,j} =  (e_{2i}-e_{2j})|1\leq i<j\leq s }\\
      & \cup \set{\alpha_0=\pm (e_{2s+a}-e_{2n}),
        \tau(\alpha_0)=\mp(e_{2s+a}+e_{2n})}.  \\
      &  \quad \text{This term only appear when
        $a,b$ are both non-zero}
    \end{split}
  \]
  Now
  \[
    W(R^\bC)^\tau =
    \begin{cases}
      \gen{s_{\alpha_{i,i+1}}s_{\tau \alpha_{i,i+1}}| i=1,
        \cdots, s} & \text{type $A_{s-1}$ if $ab=0$}\\
      \gen{s_{\alpha_{i,i+1}}s_{\tau \alpha_{i,i+1}}| i=1,
        \cdots, s} \gen{w_{cx} := w_{2s+a,n}w'_{2s+a,n}}& \text{type $A_{s-1}\cdot A_1$ if $ab\neq 0$}\\
    \end{cases}
  \]
  Here $w_{cx} = w_{2s+a,n}w'_{2s+a,n} = w_{2s+a} w_n$.
\end{des}


In summary,
\begin{equation}\label{eq:W_KH.C}
  W(\bfK,\bfH)= W(\bfG(\bR),\bfH(\bR))=\underbrace {\overbrace{ S_s }^{(W^C)^\tau} \ltimes
    \overbrace{W(A_1)^s}^{\text{real}}}_{W_s }\ltimes
  \underbrace{\overbrace{W(A_1)^s}^{im}}_{A(H)}\times
  \overbrace{W_p\times W_q}^{ic} \times \overbrace{W_{a}}^{real}
\end{equation}
\begin{equation}\label{eq:W_KH.C'}
  \begin{split}
    W(\bfK_0,\bfH) &= W(\bfG_0(\bR),\bfH(\bR)) \\
    & =\underbrace {\overbrace{ S_s \cdot P_2}^{(W^C)^\tau} \ltimes
      \overbrace{W(A_1)^s}^{\text{real}}}_{W_s }\ltimes
    \underbrace{\overbrace{W(A_1)^s}^{im}}_{A(H)}\times \overbrace{(W_p\times
      W_q)'}^{ic} \times \overbrace{W'_{a}}^{real} \\
    & = S_s \ltimes ( W(A_1)^s \times W(A_1)^s) \times (W_p\times W_q \times
    W_{a})'
  \end{split}
\end{equation}
Here $P_2 = \gen{w_{2s+a,n}w'_{2s+a,n}}$ if $ab\neq 0$, and trivial otherwise.


\subsection{Coherent continuation representation}
We consider the catgory of modules with infinitesimal character $\rho(\bfG)$.
Thefore we choose $\ckzz_0 = 1$.  The set
$\bigcup_{p+q=2n} \Pi(\SO(p,q),\rho(\bfG))$ is paramterized by
\[
  \begin{split}
    \cZ &= \set{(x,y)\in \cX(1)\times \ckcX(1) |\theta_x = -\theta_y^T}
  \end{split}
\]
Note that we allow $y\in \rO(2n,\bC)$. In fact, if $n$ is even $x,y$ are both in
$\SO$ or not (i.e. $\gamma=\ckgamma\in \mathrm{Out}(\SO(2n,\bC))$). If $n$ is
odd, exactly one of $x,y$ is in $\SO$ (i.e. , $\gamma$ and $\ckgamma$ are
different elemnts in $\mathrm{Out}(\SO(2n,\bC))$).

We now decompose $\cZ$ into $W$-orbits.  Note that $\theta_x$ is determined by
$\tau_{s,a,b}$ and $-\theta_x^T$ gives the element $\cktau$ which is conjugate
to $\tau_{s,b,a}$.
By \eqref{eq:Xtau.D}
\[
  \begin{split}
    \set{(x,y)\in \cZ| x = x_{s,a,p,q}} & = \set{(x_{s,a,p,q},y)|y\in
    \ckcX(1)_{\cktau}} \\
  &= \bigsqcup_{t+u=a, w\in W_a} (x_{s,a,p,q},w \cdot y_{s,t,u,b}) 
\end{split}
\]
with
\[
  y_{s,t,u,b} = w_{1,2}w_{3,4}\cdots w_{2s-1,2s} w_{2s+a+1}\cdots w_{2s+a+b}
  h_{2s+1}\cdots h_{2s+t}.
\]
Clearly,
\[
  \Stab_W((x_{s,a,p,q},y_{s,t,u,b})) = S_s \ltimes ( W(A_1)^s \times W(A_1)^s) \times W_t\times W_u \times
    W_p\times W_q
\]

Let $\sgn'$ denote the inflation of the sign character of $S_n$ to $W_n$.  Note
that $\sgn' = (n)\times \empty$ (Here $(n)$ denote a column of lenght $n$)k. On
the other hand, the sign charactor of $W_n$ corresponds to bipartition
$\empty\times (n)$.
Let $\sgn$ denote the sign charactor of $W(A_1)^s = (\bZ/2\bZ)^s$.

We claim that restrict of the $W_n$-representation to $W'_n$ yeilds the coherent
continuation representation at the infintesimal character $\rho(G)$:
\[
  \bC \cX = \bigoplus_{2s+t+u+p+q=n}\Ind_{W_s\ltimes W(A_1)^s\times W_t\times
    W_u\times W_p\times W_q}^{W_{n}} \triv \otimes \sgn \otimes \triv \otimes \triv
  \otimes \sgn'\otimes \sgn'
\]

Note that $W_s\ltimes W(A_1)^s\subset W'_n$.
Note that $\sgn'|_{W'_a} =\sgn|_{W'_a}$.
\begin{des}
\item [$t=u=p=q=0$] In this case, the generaters of $W(A_1)^s$ are reflections
  given by a (conjugate) of simple
  imaginary root. 
  Therefore, they act by $-1$. 
\item [$p=q=0$] $W'_n \cap (W_s \ltimes W(A_1)^s \times W_t\times W_u) = W_s \ltimes W(A_1)^s \times (W_t\times W_u)'$  
  Since the real roots acts by cross action, the stabilizer $(W_t\times
  W_u)'\subset W_a$
\item [$r=u=0$] $W'_n \cap (W_s \ltimes W(A_1)^s \times W_p\times W_q) = W_s \ltimes W(A_1)^s \times (W_p\times W_q)'$  
  Since the imaginary roots acts by $-1$*cross action, the stabilizer $(W_p\times
  W_q)' = (W'_p\times W'_q) \gen{w_{ic}}$. Here $w_{ic} = w_{2s+p}w_{2s+p+q}$ is
  a product of two reflections of conjugations of simple imaginary root.
  Therefore $w_{ic}$ acts trivially.
  So on $W_p\times W_q$ we must put either $\sgn'\otimes \sgn'$ or
  $\sgn\otimes \sgn$.
\item [$p+q$ and $r+u$ are non-zero]
  In this case, $W'_n \cap (W_s \ltimes W(A_1)^s \times W_t\times W_u \times
  W_p\times W_q)$ contain the element $w_{cx} = w_{2s+a}w_{2s+a+b}$.
  Since $w_{cx}$ is in the complex roots group, it must act trivially.
  This constrain forces us to put $\sgn'\otimes \sgn'$ on $W_p\times w_q$.
  (In this case $\triv\otimes \triv \otimes \sgn'\otimes\sgn'(w_{2s+a}w_{2s+a+b})=1$.)
\end{des}

\clearpage
\section*{\color{blue} Progress reports and questions}
I have written a program to compute the combinatoric objects
numerically.
It can compute the following things:
\begin{enumerate}
\item For each special nilpotent orbit of type B (odd special orthogonal group) and type C (symplectic group),
  it can generate all the ``dot-c-r'' diagrams.
\item It can compute the theta lift of local systems on the nilpotent orbits
  for symplectic-even orthogonal groups dual pair and metaplectic-odd orthogonal
  groups dual pair.   
\item For each nilpotent orbit of the orthogonal, symplectic and metaplectic group,
  it can list all local systems obtained via iterated theta lift and
  twisting of characters.   
\end{enumerate}

Here are findings of the program. They {\color{red} agree } with Barbasch's notes.

\begin{enumerate}
\item Using stable range theta lift, we have an injective map from $\Unip(G)$
  into $\Unip(G')$ where $G$ is the smaller group,  $G'$ is split with 
  split rank $\geq 2 \rank G$.  Therefore it is enough to understand the
  counting of unipotent representations for the symplectic groups (type C) and
  odd special orthogonal groups (type B).

  The countings for type B and C are similar due to Vogan duality. (In my
  program, the generation of dot-c-r diagram of type B is translated to the case
  of type C).
  
\item {\bfseries Symplectic group} Nilpotent orbits are parameterized by columns
  $\cO := (C_{2a}\geq C_{2a-1}\cdots C_1\geq C_0 \geq C_{-1}=0)$ satisfying
  $C_{2i]}\equiv C_{2i-1} \pmod{2}$.  We only consider the special orbits which
  satisfy $C_{2i}=C_{2i-1}$ when $C_{2i}$ is odd.
  \begin{enumerate}
  \item Reduction theorem~\Cref{prop:reduction.C} imples that all odd rows in
    $\cO$ could be deleted.  Example:
    \begin{lstlisting}[language={Python}]
      partO = [16,14,13,13,6,4,1,1]
      partO1 = remove_odd_rows(partO)
      print("deleting all odd rows of %s yields %s\n"%(partO, partO1))
      print_drc_diag_C(partO)
      print_list_LS(partO, "C")
      print_drc_diag_C(partO1)
      print_list_LS(partO1, "C")
      Output
      deleting all odd rows of [16, 14, 13, 13, 6, 4, 1, 1] yields [12, 12, 11, 11, 4, 4, 1, 1]
      Type C partition: [16, 14, 13, 13, 6, 4, 1, 1]
      Number of drc diagrams: 3520
      Number of LS: 3520
      Type C partition: [12, 12, 11, 11, 4, 4, 1, 1]
      Number of drc diagrams: 3520 Type C
      Number of LS: 3520
    \end{lstlisting}
%    {\color{red} How to put the odd rows back?}

  \item By the assumption of integral infinitesimal character, even length
    columns are only allowed to appear at most two times. Moreover, there is a
    reduction to the case that odd columns only appear twice.  {\color{blue}
      Warning: the reduction is not a one-one correspondence (See Barbasch's
      notes Theorem~2 on p9). } Moreover, the number of local systems obtained
    by theta lifting is less than the number of unipotent representations, see
    example:
    \begin{lstlisting}[language=Python]
      partO = [3,3,3,3,1,1,1,1]
      print_drc_diag_C(partO)
      print_list_LS(partO,"C")
      Output
      Type C partition: [3, 3, 3, 3, 1, 1, 1, 1]
      Number of drc diagrams: 49
      Number of LS: 45
    \end{lstlisting}
    I guess this phenomenon is due to that double theta lifting only relizes 
    parabolic inductions(r-induction). (See Barbasch's notes p10)
  \item {\bfseries Match the local systems.} The following is verified
    numerically  
    (its' proof should be contained in Section~5 of
    Barbasch's notes, which I dose not fully understand.)
    \begin{thm}\label{thm:count.C}
      Suppose $\cO$ is a nilpotent orbit of type $C$ such that even
      columns appear at most twice, and odd columns appear exactly twice.  Then
      iterated theta lifting yields all unipotent representations attached to
      $\cO$. Moreover, these representations are characterized by their local
      systems (note that these local systems could be reducible)
    \end{thm}
    An example for a ``big orbit'':
    \begin{lstlisting}[language=Python]
      partO = [21,21,18,16,12,12,11,11,6,4,1,1]
      print_drc_diag_C(partO)
      print_list_LS(partO, "C")
      Output
      Type C partition: [21, 21, 18, 16, 12, 12, 11, 11, 6, 4, 1, 1]
      Number of drc diagrams: 286720
      Number of LS: 286720
    \end{lstlisting}
  \end{enumerate}

\item {\bfseries Odd orthogonal group} Nilpotent orbits are parameterized by
  columns
  \[
    \cO := (C_{2a}\geq C_{2a-1}\cdots C_1\geq C_0 \geq 0 )
  \]
  satisfying $C_{2i+1}\equiv C_{2i} \pmod{2}$ and $C_{2a}$ is odd.  We only
  consider the special orbits satisfying $C_{2i+1}=C_{2i}$ when $C_{2i}$ is
  even.  The dot-r-c diagram only counts unipotent representations of
  $\SO(2n+1 -2k,2k)$ with $k = 0, \cdots, n$ (all strong real forms). So
  \[
    \# \bigcup_{p+q=2n+1}\Unip_\cO(\rO(p,q)) = 4 \# \drc(\cO)
  \]
  \begin{enumerate}
  \item {\bfseries Remove even rows} Following operations will not change the
    number of unipotent representations: (This is also in Barbasch's notes
    Section~3.2)
    \begin{itemize}
    \item If the column $C_{2i+1} = C_{2i}+2r$ is odd, delete the $2r$ rows of
      length $2(a-i)$.  (This also agrees with the theta correspondence)
    \item If the column $C_{2i+1} = C_{2i}+2r$ is even, then delete {\color{red}
        $2r-2$} rows of length $2(a-i)$.  {\color{blue} When $C_{2i+1} =
        C_{2i}+2$ is even, it can not be
        treated by theta lifting. Example:}
\begin{lstlisting}[language=Python]
partO = [21,8,4,1,1]
partO1 = [17,4,4,1,1] # deleting all even rows of partO
partO2 = [19,6,4,1,1] # deleting 2(a-i)-2 even rows of partO
print_drc_diag_B(partO)
print_drc_diag_B(partO1)
print_drc_diag_B(partO2)
Output
Type B partition: [21, 8, 4, 1, 1]
The total number of drc diagrams: 896
Type B partition: [17, 4, 4, 1, 1]
The total number of drc diagrams: 376
Type B partition: [19, 6, 4, 1, 1]
The total number of drc diagrams: 896
\end{lstlisting}
    \end{itemize}
  \item {\bfseries Repeated columns} When a column occurs more than twice, theta
    lifting will not yields enough local systems.

  \item The case (a) and (b) seems already discussed in Section~3.2 of Barbasch
    notes (the reducition in metaplectic case).
\vspace{5em}  
  \item {\bfseries Match the local systems.} The following is verified numerically.
    \begin{thm}\label{thm:count.B}
      Suppose $\cO$ is a nilpotent orbit of type $B$ such that odd columns
      appear at most twice, and even columns appear exactly twice.  Then
      iterated theta lifting yields all unipotent representations attached to
      $\cO$. Moreover, these representations are characterized by their local
      systems.
    \end{thm}
    An example for a ``big orbit'':
    \begin{lstlisting}[language=Python]
      partO = [21,16,16,12,12,11,11,4,4,1,1]
      print_drc_diag_B(partO)
      print_list_LS(partO, "B")
      Output
      Number of drc diagrams: 131456
      Unipotent repn. of all forms: 525824
      Number of LS: 525824
    \end{lstlisting}
  \item {\bfseries Metaplectic group to odd orthogonal groups} Let $\cO'$ be a
    relevent nilpotent orbit of $\Mp(2m,\bC)$.  By stable range theta lifting,
    the set of unipotent representations $\Unip_{\cO'}(\Mp(2m,\bR))$ is embedded
    to $\Unip_{\cO}(\SO(2m+3,2m+2))$.  $\cO$ is the orbit of $\SO(4m+5,\bC)$
    obtained by attaching a column of length
    $2m+5$.  % When $\cO'$ is ``good'' enough,
    I expect that the twisting of the strange character of $\SO(2m+3,2m+2)$ is a
    free action on $\Unip_{\cO}(\SO(2m+3,2m+2))$.  Then we will have:
    \[
      2 \# \Unip_{\cO'}(\Mp(2m,\bR)) = \# \Unip_{\cO}(\SO(2m+3,2m+2))
    \]
    Note that $\# \Unip_{\cO}(\SO(2m+3,2m+2))$ is counted by the dot-r-c
    diagram having the same number of $r$ and $r'$.
  \item {\color{red} A question to Dan: } In p2 of your notes 10\_13\_19, type B
    paragraph, shall we count the representations obtained from the special ones
    by interchanging $c_{2i+1}$ and $c_{2i}$?  According to my calculation, we
    should include these representations to make the numbers to match.
  \end{enumerate}

\item {\bfseries A remark:} In the formula of theta lifting of local systems,
  there is a twisting of character according to the signature of the orthogonal
  group. This twisting is crucial. Only the correct twisting leads to the
  matching of numbers of dot-c-r diagram and local systems.

\item {\bfseries TODO:}
  \begin{enumerate}
  \item There are inductive structures of dot-r-c diagrams and local systems
    counting.  I will try to prove \Cref{thm:count.B,thm:count.C} by reducing it
    to small rank cases. (This should be down by the same method in Barbasch's
    notes)
  \item Understand the relationship between our construction and
    Moeglin-Renard's work on Arthur packet.  They have a notion of ``good
    parity'' which seems relevant.
  \item Understand the coherent continuation representations of type D and type M
    (the metaplectic group).  
  \item Understand the cohomological and parabolic induction that appeared in
    the drc diagram reduction. {\color{red} To Dan: Are these inductions
      preserve unitarity?}
  \item Deduce the Langlands parameter of unipotent representations from the drc
    diagram (It should not be too difficult, Adams already had an Atlas program
    to compute the L-parameters of all unipotent representations. But the program
    was not very efficient for classical groups.)  
  \item Describe the L-parameter correspondence in our theta lifting construction.  
    (Induction principle may be the guideline.)
  \end{enumerate}
\item {\bfseries The goal of our project.}  Maybe we should write a series of
  papers on the subject.  I propose only deal with the cases in
  \Cref{thm:count.B,thm:count.C} in our first paper. Then let us deal with the
  reductions in the second paper.
\end{enumerate}

\clearpage
\delete{
\section*{Some questions}
I am trying to understand the computation for the symplectic-even orthogonal
dual pair right now. I would like to ask some questions.

The following is a list of questions:

\begin{enumerate}
\item[] {\color{red} The coherent continuation representation of type D.
    I am confused by  (2.4.1).
  }
  \item
  First the formula is not exactly McGovern's
  \cite[(11)]{Mc}. 
  $W_t\times W_u$ corresponds to $W(A_a)\times W(A_b)$. But $W(A_a) =
  S_a$. I guess this is a typo of McGovern.
\item 
  On the other hand, $W_p\times W_q\times W'_{2s}
  \times W_t \times W_u$ is not a subgroup of the Weyl group $W'_n$ of type $D$. 
  I guess in the correct group should be $W'_{n}\cap (W_p\times W_q\times W'_{2s}
  \times W_t \times W_u)$.
\item What is the meaning of $\sgn$ for $W_p$ and $W_q$ (which are called
  $W_{p'}$ and $W_{q'}$ in \cite[(11)]{Mc}).
  If it is the sign representation of $W_p$, it corresponds to the bi-partition
  $\emptyset\times (1^p)$ ($(1^p)$ represents a column of length $p$).
  This dose not agree with your  formula for type D but agrees with the type B.  

  Is ``$\sgn$'' in fact the ``determinate'', which corresponds to
  $(1^p)\times \emptyset$?  This agrees with your notes and also McGovern's
  claim after \cite[(11)]{Mc}.
  
\item Now turn to the branching rule.  In the type D case, the following
  branching rule is used in the computation: Suppose
  $\sigma = \sigma_L\times \sigma_R$ with $\sigma_R=\emptyset$ as $W_m$-module
  (they are obtained by adding $r,r',c,c'$'s), $\tau' = \tau'_L\times \tau'_R$ as
  $W_{2s}$-module, with $\tau'_L = \tau'_R$.  Let $\tau'_{I/II}$ be the two
  irreducible components of $\tau'|_{W'_{2s}}$.  Then
  \[
    \Ind_{W'_{2s}\times W'_{m}}^{W_{2s+m}} \tau'_{I/II}\otimes \sigma = (\tau'_L\cdot
    \sigma_L)\times \tau'_R.
  \]
  Here
  $\tau'_L\cdot \sigma_L = \Ind_{S_s\times S_m}^{S_{s+m}}\tau'_L \otimes
  \sigma_L$ is calculated by Littlewood-Richardson rule.
  
  Here are two questions:
  \begin{enumerate}
  \item If $m=0$, how to distinguish $I$ and $II$? (This may not be a big problem
    for us).  
  \item If $|\sigma_R| = |\sigma_L|$, the branching rule seems more
    complicated, since a bipartition of the same left and right components may
    occur.
    The we should choose the same sign?
  \end{enumerate}
\item About the parameterization of type D.
  On page~8, it is claim that ``this seems to count for $\rO(a,b)$, as well as
  for $\SO(a,b)$.''
  But on page~16, Section~4.3 Remark, about the ``embedding''. Since additional
  non-special representations of type C must be accounted, the parameterization
  seems only count $\SO(a,b)$. 

  Assuming the description of the coherent continuation representation of type D
  in the notes is correct, the embedding works perfect. Apart from the
  combinatoric, is there any representation theory behind? 
\item A small question, what is ``$A(\pi)$'' in the remark? the associated
  variety? 
\item[] {\color{red} The reduction theorem in type C}
\item Theorem~1 on p9 proved that deleting the odd length rows will not affect
  the number of unipotent representations.  It corresponds to the
  ``$\theta$-induction'' (cohomological induction?) on the dual side(Now I fully
  understand the proof). But what dose it correspond to on the group side? Will
  this procedure preserve the unitarity?  ( Will Vogan duality preserve
  unitarity?)
\item The problem is further reduced to the case that each even/odd length column
  appears at most two times. (In fact, I think the condition should be
  $C_{2i-1}\neq C_{2i-2}$ if they are even. Because, in this case, the
  infinitesimal character is not integral.)

  I have no question about the reduction. Just a comment:
  if we use theta correspondence to construct unipotent representations,
  the reduction for odd length seems can be implemented by a double theta lift,
  with a twisting of a correct character of the orthogonal group. 
  There are only two possible choices of the character which correspond to
  $\theta$ and $r$-induction ($r$/$\rho$-induction means the usual parabolic
  induction?). 
\end{enumerate}
}

\begin{bibdiv}
  \begin{biblist}
% \bib{AB}{article}{
%   title={Genuine representations of the metaplectic group},
%   author={Adams, Jeffrey},
%   author = {Barbasch, Dan},
%   journal={Compositio Mathematica},
%   volume={113},
%   number={01},
%   pages={23--66},
%   year={1998},
% }

\bib{Ad83}{article}{
  author = {Adams, J.},
  title = {Discrete spectrum of the reductive dual pair $(O(p,q),Sp(2m))$ },
  journal = {Invent. Math.},
  number = {3},
 pages = {449--475},
 volume = {74},
 year = {1983}
}

%\bib{Ad07}{article}{
%  author = {Adams, J.},
%  title = {The theta correspondence over R},
%  journal = {Harmonic analysis, group representations, automorphic forms and invariant theory,  Lect. Notes Ser. Inst. Math. Sci. Natl. Univ. Singap., 12},
% pages = {1--39},
% year = {2007}
% publisher={World Sci. Publ.}
%}


\bib{ABV}{book}{
  title={The Langlands classification and irreducible characters for real reductive groups},
  author={Adams, J.},
  author={Barbasch, D.},
  author={Vogan, D. A.},
  series={Progress in Math.},
  volume={104},
  year={1991},
  publisher={Birkhauser}
}

\bib{AC}{article}{
  title={Algorithms for representation theory of
    real reductive groups},
  volume={8},
  DOI={10.1017/S1474748008000352},
  number={2},
  journal={Journal of the Institute of Mathematics of Jussieu},
  publisher={Cambridge University Press},
  author={Adams, Jeffrey}
  author={du Cloux,
    Fokko},
  year={2009},
  pages={209-259}
}

\bib{ArPro}{article}{
  author = {Arthur, J.},
  title = {On some problems suggested by the trace formula},
  journal = {Lie group representations, II (College Park, Md.), Lecture Notes in Math. 1041},
 pages = {1--49},
 year = {1984}
}


\bib{ArUni}{article}{
  author = {Arthur, J.},
  title = {Unipotent automorphic representations: conjectures},
  %booktitle = {Orbites unipotentes et repr\'esentations, II},
  journal = {Orbites unipotentes et repr\'esentations, II, Ast\'erisque},
 pages = {13--71},
 volume = {171-172},
 year = {1989}
}

\bib{AK}{article}{
  author = {Auslander, L.},
  author = {Kostant, B.},
  title = {Polarizations and unitary representations of solvable Lie groups},
  journal = {Invent. Math.},
 pages = {255--354},
 volume = {14},
 year = {1971}
}

\bib{B.Uni}{article}{
  author = {Barbasch, D.},
  title = {Unipotent representations for real reductive groups},
 %booktitle = {Proceedings of ICM, Kyoto 1990},
 journal = {Proceedings of ICM (1990), Kyoto},
   % series = {Proc. Sympos. Pure Math.},
 %   volume = {68},
     pages = {769--777},
 publisher = {Springer-Verlag, The Mathematical Society of Japan},
      year = {2000},
}

\bib{B.W}{article}{
  author={Barbasch, Dan},
  author={Vogan, David},
  editor={Trombi, P. C.},
  title={Weyl Group Representations and Nilpotent Orbits},
  bookTitle={Representation Theory of Reductive Groups:
    Proceedings of the University of Utah Conference 1982},
  year={1983},
  publisher={Birkh{\"a}user Boston},
  address={Boston, MA},
  pages={21--33},
  %doi={10.1007/978-1-4684-6730-7_2},
}



\bib{B.Orbit}{article}{
  author = {Barbasch, D.},
  title = {Orbital integrals of nilpotent orbits},
 %booktitle = {The mathematical legacy of {H}arish-{C}handra ({B}altimore,{MD}, 1998)},
    journal = {The mathematical legacy of {H}arish-{C}handra, Proc. Sympos. Pure Math.},
    %series={The mathematical legacy of {H}arish-{C}handra, Proc. Sympos. Pure Math},
    volume = {68},
     pages = {97--110},
 publisher = {Amer. Math. Soc., Providence, RI},
      year = {2000},
}



\bib{B10}{article}{
  author = {Barbasch, D.},
  title = {The unitary spherical spectrum for split classical groups},
  journal = {J. Inst. Math. Jussieu},
% number = {9},
 pages = {265--356},
 volume = {9},
 year = {2010}
}



\bib{B17}{article}{
  author = {Barbasch, D.},
  title = {Unipotent representations and the dual pair correspondence},
  journal = {J. Cogdell et al. (eds.), Representation Theory, Number Theory, and Invariant Theory, In Honor of Roger Howe. Progress in Math.}
  %series ={Progress in Math.},
  volume = {323},
  pages = {47--85},
  year = {2017},
}

\bib{BVUni}{article}{
 author = {Barbasch, D.},
 author = {Vogan, D. A.},
 journal = {Annals of Math.},
 number = {1},
 pages = {41--110},
 title = {Unipotent representations of complex semisimple groups},
 volume = {121},
 year = {1985}
}

\bib{Br}{article}{
  author = {Brylinski, R.},
  title = {Dixmier algebras for classical complex nilpotent orbits via Kraft-Procesi models. I},
  journal = {The orbit method in geometry and physics (Marseille, 2000). Progress in Math.}
  volume = {213},
  pages = {49--67},
  year = {2003},
}

\bib{Carter}{book}{
   author={Carter, Roger W.},
   title={Finite groups of Lie type},
   series={Wiley Classics Library},
   %note={Conjugacy classes and complex characters;
   %Reprint of the 1985 original;
   %A Wiley-Interscience Publication},
   publisher={John Wiley \& Sons, Ltd., Chichester},
   date={1993},
   pages={xii+544},
   isbn={0-471-94109-3},
   %review={\MR{1266626}},
}
\bib{Ca89}{article}{
 author = {Casselman, W.},
 journal = {Canad. J. Math.},
 pages = {385--438},
 title = {Canonical extensions of Harish-Chandra modules to representations of $G$},
 volume = {41},
 year = {1989}
}



\bib{Cl}{article}{
  author = {Du Cloux, F.},
  journal = {Ann. Sci. \'Ecole Norm. Sup.},
  number = {3},
  pages = {257--318},
  title = {Sur les repr\'esentations diff\'erentiables des groupes de Lie alg\'ebriques},
  url = {http://eudml.org/doc/82297},
  volume = {24},
  year = {1991},
}

\bib{CM}{book}{
  title = {Nilpotent orbits in semisimple Lie algebra: an introduction},
  author = {Collingwood, D. H.},
  author = {McGovern, W. M.},
  year = {1993}
  publisher = {Van Nostrand Reinhold Co.},
}


% \bib{Dieu}{book}{
%    title={La g\'{e}om\'{e}trie des groupes classiques},
%    author={Dieudonn\'{e}, Jean},
%    year={1963},
% 	publisher={Springer},
%  }

\bib{DKPC}{article}{
title = {Nilpotent orbits and complex dual pairs},
journal = {J. Algebra},
volume = {190},
number = {2},
pages = {518 - 539},
year = {1997},
author = {Daszkiewicz, A.},
author = {Kra\'skiewicz, W.},
author = {Przebinda, T.},
}

\bib{DKP2}{article}{
  author = {Daszkiewicz, A.},
  author = {Kra\'skiewicz, W.},
  author = {Przebinda, T.},
  title = {Dual pairs and Kostant-Sekiguchi correspondence. II. Classification
	of nilpotent elements},
  journal = {Central European J. Math.},
  year = {2005},
  volume = {3},
  pages = {430--474},
}


\bib{DM}{article}{
  author = {Dixmier, J.},
  author = {Malliavin, P.},
  title = {Factorisations de fonctions et de vecteurs ind\'efiniment diff\'erentiables},
  journal = {Bull. Sci. Math. (2)},
  year = {1978},
  volume = {102},
  pages = {307--330},
}

%\bibitem[DM]{DM}
%J. Dixmier and P. Malliavin, \textit{Factorisations de fonctions et de vecteurs ind\'efiniment diff\'erentiables}, Bull. Sci. Math. (2), 102 (4),  307-330 (1978).



%\bib{Du77}{article}{
% author = {Duflo, M.},
% journal = {Annals of Math.},
% number = {1},
% pages = {107-120},
% title = {Sur la Classification des Ideaux Primitifs Dans
%   L'algebre Enveloppante d'une Algebre de Lie Semi-Simple},
% volume = {105},
% year = {1977}
%}

\bib{Du82}{article}{
 author = {Duflo, M.},
 journal = {Acta Math.},
  volume = {149},
 number = {3-4},
 pages = {153--213},
 title = {Th\'eorie de Mackey pour les groupes de Lie alg\'ebriques},
 year = {1982}
}



\bib{GZ}{article}{
author={Gomez, R.},
author={Zhu, C.-B.},
title={Local theta lifting of generalized Whittaker models associated to nilpotent orbits},
journal={Geom. Funct. Anal.},
year={2014},
volume={24},
number={3},
pages={796--853},
}

\bib{EGAIV2}{article}{
  title = {\'El\'ements de g\'eom\'etrie alg\'brique IV: \'Etude locale des
    sch\'emas et des morphismes de sch\'emas. II},
  author = {Grothendieck, A.},
  author = {Dieudonn\'e, J.},
  journal  = {Inst. Hautes \'Etudes Sci. Publ. Math.},
  volume = {24},
  year = {1965},
}


\bib{EGAIV3}{article}{
  title = {\'El\'ements de g\'eom\'etrie alg\'brique IV: \'Etude locale des
    sch\'emas et des morphismes de sch\'emas. III},
  author = {Grothendieck, A.},
  author = {Dieudonn\'e, J.},
  journal  = {Inst. Hautes \'Etudes Sci. Publ. Math.},
  volume = {28},
  year = {1966},
}


\bib{HLS}{article}{
    author = {Harris, M.},
    author = {Li, J.-S.},
    author = {Sun, B.},
     title = {Theta correspondences for close unitary groups},
 %booktitle = {Arithmetic Geometry and Automorphic Forms},
    %series = {Adv. Lect. Math. (ALM)},
    journal = {Arithmetic Geometry and Automorphic Forms, Adv. Lect. Math. (ALM)},
    volume = {19},
     pages = {265--307},
 publisher = {Int. Press, Somerville, MA},
      year = {2011},
}

\bib{HS}{book}{
 author = {Hartshorne, R.},
 title = {Algebraic Geometry},
publisher={Graduate Texts in Mathematics, 52. New York-Heidelberg-Berlin: Springer-Verlag},
year={1983},
}

\bib{He}{article}{
author={He, H.},
title={Unipotent representations and quantum induction},
journal={arXiv:math/0210372},
year = {2002},
}

\bib{HL}{article}{
author={Huang, J.-S.},
author={Li, J.-S.},
title={Unipotent representations attached to spherical nilpotent orbits},
journal={Amer. J. Math.},
volume={121},
number = {3},
pages={497--517},
year={1999},
}


\bib{HZ}{article}{
author={Huang, J.-S.},
author={Zhu, C.-B.},
title={On certain small representations of indefinite orthogonal groups},
journal={Represent. Theory},
volume={1},
pages={190--206},
year={1997},
}



\bib{Howe79}{article}{
  title={$\theta$-series and invariant theory},
  author={Howe, R.},
  book = {
    title={Automorphic Forms, Representations and $L$-functions},
    series={Proc. Sympos. Pure Math},
    volume={33},
    year={1979},
  },
  pages={275-285},
}

\bib{HoweRank}{article}{
author={Howe, R.},
title={On a notion of rank for unitary representations of the classical groups},
journal={Harmonic analysis and group representations, Liguori, Naples},
pages={223-331},
year={1982},
}

\bib{Howe89}{article}{
author={Howe, R.},
title={Transcending classical invariant theory},
journal={J. Amer. Math. Soc.},
volume={2},
pages={535--552},
year={1989},
}

\bib{Howe95}{article}{,
  author = {Howe, R.},
  title = {Perspectives on invariant theory: Schur duality, multiplicity-free actions and beyond},
  journal = {Piatetski-Shapiro, I. et al. (eds.), The Schur lectures (1992). Ramat-Gan: Bar-Ilan University, Isr. Math. Conf. Proc. 8,},
  year = {1995},
  pages = {1-182},
}

\bib{JLS}{article}{
author={Jiang, D.},
author={Liu, B.},
author={Savin, G.},
title={Raising nilpotent orbits in wave-front sets},
journal={Represent. Theory},
volume={20},
pages={419--450},
year={2016},
}

\bib{Ki62}{article}{
author={Kirillov, A. A.},
title={Unitary representations of nilpotent Lie groups},
journal={Uspehi Mat. Nauk},
volume={17},
issue ={4},
pages={57--110},
year={1962},
}


\bib{Ko70}{article}{
author={Kostant, B.},
title={Quantization and unitary representations},
journal={Lectures in Modern Analysis and Applications III, Lecture Notes in Math.},
volume={170},
pages={87--208},
year={1970},
}


\bib{KP}{article}{
author={Kraft, H.},
author={Procesi, C.},
title={On the geometry of conjugacy classes in classical groups},
journal={Comment. Math. Helv.},
volume={57},
pages={539--602},
year={1982},
}

\bib{KR}{article}{
author={Kudla, S. S.},
author={Rallis, S.},
title={Degenerate principal series and invariant distributions},
journal={Israel J. Math.},
volume={69},
pages={25--45},
year={1990},
}


\bib{Ku}{article}{
author={Kudla, S. S.},
title={Some extensions of the Siegel-Weil formula},
journal={In: Gan W., Kudla S., Tschinkel Y. (eds) Eisenstein Series and Applications. Progress in Mathematics, vol 258. Birkh\"auser Boston},
%volume={69},
pages={205--237},
year={2008},
}





\bib{LZ1}{article}{
author={Lee, S. T.},
author={Zhu, C.-B.},
title={Degenerate principal series and local theta correspondence II},
journal={Israel J. Math.},
volume={100},
pages={29--59},
year={1997},
}

\bib{LZ2}{article}{
author={Lee, S. T.},
author={Zhu, C.-B.},
title={Degenerate principal series of metaplectic groups and Howe correspondence},
journal = {D. Prasad at al. (eds.), Automorphic Representations and L-Functions, Tata Institute of Fundamental Research, India,},
year = {2013},
pages = {379--408},
}

\bib{Li89}{article}{
author={Li, J.-S.},
title={Singular unitary representations of classical groups},
journal={Invent. Math.},
volume={97},
number = {2},
pages={237--255},
year={1989},
}

\bib{LiuAG}{book}{
  title={Algebraic Geometry and Arithmetic Curves},
  author = {Liu, Q.},
  year = {2006},
  publisher={Oxford University Press},
}

\bib{LM}{article}{
   author = {Loke, H. Y.},
   author = {Ma, J.},
    title = {Invariants and $K$-spectrums of local theta lifts},
    journal = {Compositio Math.},
    volume = {151},
    issue = {01},
    year = {2015},
    pages ={179--206},
}

\bib{LS}{article}{
   author = {Lusztig, G.},
   author = {Spaltenstein, N.},
    title = {Induced unipotent classes},
    journal = {j. London Math. Soc.},
    volume = {19},
    year = {1979},
    pages ={41--52},
}

\bib{Lu.I}{article}{
   author={Lusztig, G.},
   title={Intersection cohomology complexes on a reductive group},
   journal={Invent. Math.},
   volume={75},
   date={1984},
   number={2},
   pages={205--272},
   issn={0020-9910},
   review={\MR{732546}},
   doi={10.1007/BF01388564},
}
	

\bib{Ma}{article}{
   author = {Mackey, G. W.},
    title = {Unitary representations of group extentions},
    journal = {Acta Math.},
    volume = {99},
    year = {1958},
    pages ={265--311},
}


\bib{Mc}{article}{
   author = {McGovern, W. M},
    title = {Cells of Harish-Chandra modules for real classical groups},
    journal = {Amer. J.  of Math.},
    volume = {120},
    issue = {01},
    year = {1998},
    pages ={211--228},
}

\bib{Mo96}{article}{
 author={M{\oe}glin, C.},
    title = {Front d'onde des repr\'esentations des groupes classiques $p$-adiques},
    journal = {Amer. J. Math.},
    volume = {118},
    issue = {06},
    year = {1996},
    pages ={1313--1346},
}

\bib{Mo17}{article}{
  author={M{\oe}glin, C.},
  title = {Paquets d'Arthur Sp\'eciaux Unipotents aux Places Archim\'ediennes et Correspondance de Howe},
  journal = {J. Cogdell et al. (eds.), Representation Theory, Number Theory, and Invariant Theory, In Honor of Roger Howe. Progress in Math.}
  %series ={Progress in Math.},
  volume = {323},
  pages = {469--502}
  year = {2017}
}


\bib{MVW}{book}{
  volume={1291},
  title={Correspondances de Howe sur un corps $p$-adique},
  author={M{\oe}glin, C.},
  author={Vign\'eras, M.-F.},
  author={Waldspurger, J.-L.},
  series={Lecture Notes in Mathematics},
  publisher={Springer}
  ISBN={978-3-540-18699-1},
  date={1987},
}

\bib{NOTYK}{article}{
   author = {Nishiyama, K.},
   author = {Ochiai, H.},
   author = {Taniguchi, K.},
   author = {Yamashita, H.},
   author = {Kato, S.},
    title = {Nilpotent orbits, associated cycles and Whittaker models for highest weight representations},
    journal = {Ast\'erisque},
    volume = {273},
    year = {2001},
   pages ={1--163},
}

\bib{NOZ}{article}{
  author = {Nishiyama, K.},
  author = {Ochiai, H.},
  author = {Zhu, C.-B.},
  journal = {Trans. Amer. Math. Soc.},
  title = {Theta lifting of nilpotent orbits for symmetric pairs},
  volume = {358},
  year = {2006},
  pages = {2713--2734},
}


\bib{NZ}{article}{
   author = {Nishiyama, K.},
   author = {Zhu, C.-B.},
    title = {Theta lifting of unitary lowest weight modules and their associated cycles},
    journal = {Duke Math. J.},
    volume = {125},
    issue = {03},
    year = {2004},
   pages ={415--465},
}



\bib{Ohta}{article}{
  author = {Ohta, T.},
  %doi = {10.2748/tmj/1178227492},
  journal = {Tohoku Math. J.},
  number = {2},
  pages = {161--211},
  publisher = {Tohoku University, Mathematical Institute},
  title = {The closures of nilpotent orbits in the classical symmetric
    pairs and their singularities},
  volume = {43},
  year = {1991}
}

\bib{Ohta2}{article}{
  author = {Ohta, T.},
  journal = {Hiroshima Math. J.},
  number = {2},
  pages = {347--360},
  title = {Induction of nilpotent orbits for real reductive groups and associated varieties of standard representations},
  volume = {29},
  year = {1999}
}

\bib{Ohta4}{article}{
  title={Nilpotent orbits of $\mathbb{Z}_4$-graded Lie algebra and geometry of
    moment maps associated to the dual pair $(\mathrm{U} (p, q), \mathrm{U} (r, s))$},
  author={Ohta, T.},
  journal={Publ. RIMS},
  volume={41},
  number={3},
  pages={723--756},
  year={2005}
}

\bib{PT}{article}{
  title={Some small unipotent representations of indefinite orthogonal groups and the theta correspondence},
  author={Paul, A.},
  author={Trapa, P.},
  journal={University of Aarhus Publ. Series},
  volume={48},
  pages={103--125},
  year={2007}
}


\bib{PV}{article}{
  title={Invariant Theory},
  author={Popov, V. L.},
  author={Vinberg, E. B.},
  book={
  title={Algebraic Geometry IV: Linear Algebraic Groups, Invariant Theory},
  series={Encyclopedia of Mathematical Sciences},
  volume={55},
  year={1994},
  publisher={Springer},}
}




%\bib{PPz}{article}{
%author={Protsak, V.} ,
%author={Przebinda, T.},
%title={On the occurrence of admissible representations in the real Howe
%    correspondence in stable range},
%journal={Manuscr. Math.},
%volume={126},
%number={2},
%pages={135--141},
%year={2008}
%}


\bib{PrzInf}{article}{
      author={Przebinda, T.},
       title={The duality correspondence of infinitesimal characters},
        date={1996},
     journal={Colloq. Math.},
      volume={70},
       pages={93--102},
}


\bib{Pz}{article}{
author={Przebinda, T.},
title={Characters, dual pairs, and unitary representations},
journal={Duke Math. J. },
volume={69},
number={3},
pages={547--592},
year={1993}
}

\bib{Ra}{article}{
author={Rallis, S.},
title={On the Howe duality conjecture},
journal={Compositio Math.},
volume={51},
pages={333--399},
year={1984}
}

\bib{Sa}{article}{
author={Sahi, S.},
title={Explicit Hilbert spaces for certain unipotent representations},
journal={Invent. Math.},
volume={110},
number = {2},
pages={409--418},
year={1992}
}

\bib{Se}{article}{
author={Sekiguchi, J.},
title={Remarks on real nilpotent orbits of a symmetric pair},
journal={J. Math. Soc. Japan},
%publisher={The Mathematical Society of Japan},
year={1987},
volume={39},
number={1},
pages={127--138},
}

\bib{SV}{article}{
  author = {Schmid, W.},
  author = {Vilonen, K.},
  journal = {Annals of Math.},
  number = {3},
  pages = {1071--1118},
  %publisher = {Princeton University, Mathematics Department, Princeton, NJ; Mathematical Sciences Publishers, Berkeley},
  title = {Characteristic cycles and wave front cycles of representations of reductive Lie groups},
  volume = {151},
year = {2000},
}

\bib{So}{article}{
author = {Sommers, E.},
title = {Lusztig's canonical quotient and generalized duality},
journal = {J. Algebra},
volume = {243},
number = {2},
pages = {790--812},
year = {2001},
}

\bib{SS}{book}{
  author = {Springer, T. A.},
  author = {Steinberg, R.},
  title = {Seminar on algebraic groups and related finite groups; Conjugate classes},
  series = {Lecture Notes in Math.}
  volume = {131}
publisher={Springer},
year={1970},
}

\bib{SZ1}{article}{
title={A general form of Gelfand-Kazhdan criterion},
author={Sun, B.},
author={Zhu, C.-B.},
journal={Manuscripta Math.},
pages = {185--197},
volume = {136},
year={2011}
}


%\bib{SZ2}{article}{
%  title={Conservation relations for local theta correspondence},
%  author={Sun, B.},
%  author={Zhu, C.-B.},
%  journal={J. Amer. Math. Soc.},
%  pages = {939--983},
%  volume = {28},
%  year={2015}
%}



\bib{Tr}{article}{
  title={Special unipotent representations and the Howe correspondence},
  author={Trapa, P.},
  year = {2004},
  journal={University of Aarhus Publication Series},
  volume = {47},
  pages= {210--230}
}

% \bib{Wa}{article}{
%    author = {Waldspurger, J.-L.},
%     title = {D\'{e}monstration d'une conjecture de dualit\'{e} de Howe dans le cas $p$-adique, $p \neq 2$ in Festschrift in honor of I. I. Piatetski-Shapiro on the occasion of his sixtieth birthday},
%   journal = {Israel Math. Conf. Proc., 2, Weizmann, Jerusalem},
%  year = {1990},
% pages = {267-324},
% }

\bib{V4}{article}{
   author={Vogan, D. A. },
   title={Irreducible characters of semisimple Lie groups. IV.
   Character-multiplicity duality},
   journal={Duke Math. J.},
   volume={49},
   date={1982},
   number={4},
   pages={943--1073},
   issn={0012-7094},
   review={\MR{683010}},
}
\bib{VoBook}{book}{
author = {Vogan, D. A. },
  title={Unitary representations of reductive Lie groups},
  year={1987},
  series = {Ann. of Math. Stud.},
 volume={118},
  publisher={Princeton University Press}
}


\bib{Vo89}{article}{
  author = {Vogan, D. A. },
  title = {Associated varieties and unipotent representations},
 %booktitle ={Harmonic analysis on reductive groups, Proc. Conf., Brunswick/ME (USA) 1989,},
  journal = {Harmonic analysis on reductive groups, Proc. Conf., Brunswick/ME
    (USA) 1989, Prog. Math.},
 volume={101},
  publisher = {Birkh\"{a}user, Boston-Basel-Berlin},
  year = {1991},
pages={315--388},
  editor = {W. Barker and P. Sally},
}

\bib{Vo98}{article}{
  author = {Vogan, D. A. },
  title = {The method of coadjoint orbits for real reductive groups},
 %booktitle ={Representation theory of Lie groups (Park City, UT, 1998)},
 journal = {Representation theory of Lie groups (Park City, UT, 1998). IAS/Park City Math. Ser.},
  volume={8},
  publisher = {Amer. Math. Soc.},
  year = {2000},
pages={179--238},
}

\bib{Vo00}{article}{
  author = {Vogan, D. A. },
  title = {Unitary representations of reductive Lie groups},
 %booktitle ={Mathematics towards the Third Millennium (Rome, 1999)},
 journal ={Mathematics towards the Third Millennium (Rome, 1999). Accademia Nazionale dei Lincei, (2000)},
  %series = {Accademia Nazionale dei Lincei, 2000},
 %volume={9},
pages={147--167},
}


\bib{Wa1}{book}{
  title={Real reductive groups I},
  author={Wallach, N. R.},
  year={1988},
  publisher={Academic Press Inc. }
}

\bib{Wa2}{book}{
  title={Real reductive groups II},
  author={Wallach, N. R.},
  year={1992},
  publisher={Academic Press Inc. }
}


\bib{Weyl}{book}{
  title={The classical groups: their invariants and representations},
  author={Weyl, H.},
  year={1947},
  publisher={Princeton University Press}
}

\bib{Ya}{article}{
  title={Degenerate principal series representations for quaternionic unitary groups},
  author={Yamana, S.},
  year = {2011},
  journal={Israel J. Math.},
  volume = {185},
  pages= {77--124}
}



% \bib{EGAIV4}{article}{
%   title = {\'El\'ements de g\'eom\'etrie alg\'brique IV 4: \'Etude locale des
%     sch\'emas et des morphismes de sch\'emas},
%   author = {Grothendieck, Alexandre},
%   author = {Dieudonn\'e, Jean},
%   journal  = {Inst. Hautes \'Etudes Sci. Publ. Math.},
%   volume = {32},
%   year = {1967},
%   pages = {5--361}
% }



\end{biblist}
\end{bibdiv}


\end{document}


%%% Local Variables:
%%% coding: utf-8
%%% mode: latex
%%% TeX-engine: xetex
%%% ispell-local-dictionary: "en_US"
%%% End:
