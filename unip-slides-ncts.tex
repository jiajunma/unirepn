% !TEX program = xelatex
%\documentclass[t,serif,11pt,handout,usenames,dvipsnames]{beamer}
%\documentclass[t,11pt,usenames,dvipsnames]{beamer}
\documentclass[t,mathserif,11pt]{beamer} 
%\usetheme{Berlin}
%\usecolortheme{seahorse}
%
%\usetheme{CambridgeUS}
%\usecolortheme{rose}
%\usecolortheme{dolphin}
\usecolortheme{seahorse}
%
\usefonttheme{serif}
%\usefonttheme{professionalfonts}
%\usepackage{mathpazo}
%\usepackage{pxfonts}

\setbeamertemplate{itemize items}[square]
\setbeamertemplate{itemize subitem}[circle]
\setbeamertemplate{itemize subsubitem}[triangle]

\setbeamertemplate{enumerate items}[default]




\usepackage{verbatim}
%\usepackage{multirow}
\usepackage[normalem]{ulem}

%\newcommand\hmmax{0}
\usepackage{hyperref}
%\usepackage[abbrev]{amsrefs}

%\usepackage{bbold}
%\usepackage{bbm}
%\def\mathbb{\mathbbm}
\usepackage{amssymb}
\usepackage{amsmath}
%\usepackage{unicode-math}
%\usepackage{amsthm}
\usepackage{graphicx}
\usepackage{braket}
%\usepackage{paralist}
%\usepackage{rotating}
%\usepackage{arydshln}

%\usepackage{color}
\usepackage{xcolor}
%\usepackage[all,color]{xy}
%\UseCrayolaColors

%\usepackage{mathabx}
\usepackage{mathrsfs}
%\usepackage{MnSymbol}
%\usepackage{mathbbol}
%\usepackage{psnfss}



\usepackage{youngtab}
\Yautoscale1
\Yvcentermath1

\usepackage[centertableaux,smalltableaux]{ytableau}

%\usepackage{cleveref}

\theoremstyle{plain}
\newtheorem{thm}{Theorem}
\newtheorem{claim}{Claim}
\newtheorem{conj}{Conjecture}

%\newtheorem{lemA}[thmA]{Lemma}
%\newtheorem{propA}[thmA]{Proposition}
%\newtheorem{corA}[thmA]{Corollary}
%\newtheorem{thm}[subsection]{Theorem}
%\newtheorem{lemma}{Lemma}
\newtheorem{prop}{Proposition}
\newtheorem{cor}{Corollary}
%\newtheorem*{prop*}{Proposition}

\theoremstyle{definition}
\newtheorem{defn}{Definition}



\definecolor{refkey}{gray}{.3}
\definecolor{labelkey}{gray}{.3}


% \def\cone{\ding{172}}
% \def\ctwo{\ding{173}}
% \def\cthree{\ding{174}}
% \def\cfour{\ding{175}}
% \def\cfive{\ding{176}}



\newcommand{\mjjc}[1]{\marginpar{\color{green}\tiny #1 \mbox{--ma}}}


\newcommand{\rk}{\mathrm{rk}}
\newcommand{\cqq}{\mathscr{D}}
\newcommand{\Sym}{\mathrm{Sym}}
\newcommand{\rSp}{\mathrm{Sp}}
\newcommand{\rsym}{\mathrm{sym}}
\newcommand{\rO}{\mathrm{O}}
\newcommand{\SO}{\mathrm{SO}}
\newcommand{\rskew}{\mathrm{skew}}
\newcommand{\fraksp}{\mathfrak{sp}}
\newcommand{\frakso}{\mathfrak{so}}
\newcommand{\frakm}{\mathfrak{m}}
\newcommand{\frakp}{\mathfrak{p}}
\newcommand{\pr}{\mathrm{pr}}
\newcommand{\rhopst}{\rho'^*}
\newcommand{\Rad}{\mathrm{Rad}}
\newcommand{\Res}{\mathrm{Res}}
\newcommand{\Hol}{\mathrm{Hol}}
\newcommand{\AC}{\mathrm{AC}}
\newcommand{\AV}{\mathrm{AV}}
\newcommand{\VC}{\mathrm{V}_\bC}
\newcommand{\bfv}{\mathbf{v}}
\newcommand{\depth}{\mathrm{depth}}
\newcommand{\wtM}{\widetilde{M}}
\newcommand{\wtMone}{{\widetilde{M}^{(1,1)}}}

\newcommand{\nullpp}{N(\fpp'^*)}
\newcommand{\nullp}{N(\fpp^*)}
\newcommand{\Aut}{\mathrm{Aut}}



\newcommand{\bfone}{\mathbf{1}}
\newcommand{\bbone}{\mathbb{1}}




\newcommand{\sfVprime}{\mathsf{V}^\prime}
\newcommand{\sfVdprime}{\mathsf{V}^{\prime \prime}}
\newcommand{\gminusone}{\mathfrak{g}_{-\frac{1}{m}}}

\newcommand{\eva}{\mathrm{eva}}

\newcommand\iso{\xrightarrow{
\,\smash{\raisebox{-0.65ex}{\ensuremath{\scriptstyle\sim}}}\,}}

\def\tmu{\tilde{\mu}}

\def\tr{\mathrm{tr}}
\def\trD{{\tr_{D/F}}}
\def\trF{{\tr_F}}

\def\Ueven{{U_{\rm{even}}}}
\def\Uodd{{U_{\rm{odd}}}}
\def\ttau{\tilde{\tau}}
\def\Wcp{W}
\def\Kur{{K^{\mathrm{u}}}}


\providecommand{\bcN}{{\overline{\cN}}}
\providecommand{\bcO}{{\overline{\cO}}}

\def\Mpii{M'^{(1,1)}}
\def\Mptz{M'^{(2,0)}}
\def\Mpzt{M'^{(0,2)}}
\def\mpii{\fmm'^{(1,1)}}
\def\mptz{\fmm'^{(2,0)}}
\def\mpzt{\fmm'^{(0,2)}}
\def\bcOp{\overline{\cO'}}

\def\frakN{\mathfrak{N}}
\def\topform{\mbox{$\bigwedge^{\! \mathrm{top} \, }$}}

\def\barxi{{\overline{\xi}}}
\def\Stab{{\rm Stab}}
\def\ad{{\rm ad}}
\def\Ad{{\rm Ad}}
\def\id{{\rm id}}
\def\sgn{{\rm sgn}}
\def\gcd{{\rm gcd}}

\makeatletter
\def\inn#1#2{\left\langle 
\def\ta{#1}\def\tb{#2}
\ifx\ta\@empty{\;} \else {\ta}\fi ,
\ifx\tb\@empty{\;} \else {\tb}\fi
\right\rangle} 
\makeatother

\def\innw#1#2{\inn{#1}{#2}_{W}}
\def\innv#1#2{\inn{#1}{#2}_{V}}
\def\innvp#1#2{\inn{#1}{#2}_{V'}}
\def\innGa#1#2{\inn{#1}{#2}_{\Gamma}}
\def\innGap#1#2{\inn{#1}{#2}_{\Gamma'}}

\def\abs#1{\left|{#1}\right|}
\def\norm#1{{\left\|{#1}\right\|}}



\def\mydefhat#1{\expandafter\def\csname hat#1\endcsname{\hat{#1}}}
\def\mydefwh#1{\expandafter\def\csname wh#1\endcsname{\widehat{#1}}}
\def\mydeft#1{\expandafter\def\csname t#1\endcsname{\tilde{#1}}}
\def\mydefr#1{\expandafter\def\csname r#1\endcsname{\mathrm{#1}}}
\def\mydefb#1{\expandafter\def\csname b#1\endcsname{\mathbb{#1}}}
\def\mydefwt#1{\expandafter\def\csname wt#1\endcsname{\widetilde{#1}}}
\def\mydeff#1{\expandafter\def\csname f#1\endcsname{\mathfrak{#1}}}
\def\mydefbf#1{\expandafter\def\csname bf#1\endcsname{\mathbf{#1}}}
\def\mydefc#1{\expandafter\def\csname c#1\endcsname{\mathcal{#1}}}
\def\mydefsf#1{\expandafter\def\csname sf#1\endcsname{\mathsf{#1}}}
\def\mydefs#1{\expandafter\def\csname s#1\endcsname{\mathscr{#1}}}
\def\mydefcks#1{\expandafter\def\csname cks#1\endcsname{{\check{
\csname s#1\endcsname}}}}
\def\mydefck#1{\expandafter\def\csname ck#1\endcsname{{\check{
#1}}}}

\def\mydefbfdowncase#1{
\expandafter\def\csname bf#1#1\endcsname{\mathbf{#1}}
}
\def\mydefckdowncase#1{
\expandafter\def\csname ck#1#1\endcsname{\check{#1}}
}
\def\mydefrmdowncase#1{
\expandafter\def\csname r#1#1\endcsname{\mathrm{#1}}
}
\def\mydefsfdowncase#1{
\expandafter\def\csname sf#1#1\endcsname{\mathsf{#1}}
}
\def\mydeffdowncase#1{
\expandafter\def\csname f#1#1\endcsname{\mathfrak{#1}}
}

\def\doAZ#1{#1A #1B #1C #1D #1E #1F #1G #1H #1I #1J #1K #1L 
#1M #1N #1O #1P #1Q #1R #1S #1T #1U #1V #1W #1X #1Y #1Z}
\def\doaz#1{#1a #1b #1c #1d #1e #1f #1g #1h #1i #1j #1k #1l 
#1m #1n #1o #1p #1q #1r #1s #1t #1u #1v #1w #1x #1y #1z}

\doAZ{\mydefsf}
\doAZ{\mydeft}
\doAZ{\mydefwh}
\doAZ{\mydefhat}
\doAZ{\mydefr}
\doAZ{\mydefwt}
\doAZ{\mydeff}
\doAZ{\mydefb}
\doAZ{\mydefbf}
\doAZ{\mydefc}
\doAZ{\mydefs}
\doAZ{\mydefck}
\doAZ{\mydefcks}

\doaz{\mydefbfdowncase}
\doaz{\mydeffdowncase}
\doaz{\mydefrmdowncase}
\doaz{\mydefsfdowncase}
\doaz{\mydefckdowncase}




\def\wtGL{\widetilde{\GL}}
\def\wtSp{\widetilde{\rSp}}
\def\tSp{\tilde{\rSp}}


\def\trho{{\widetilde{\rho}}}
\def\tiota{{\widetilde{\iota}}}
\def\biota{{\overline{\iota}}}



\def\GL{\mathrm{GL}}
\def\wtSp{\widetilde{\mathrm{Sp}}}
\def\fsp{{\mathfrak{sp}}}
\def\fgl{{\mathfrak{gl}}}
\def\fsl{{\mathfrak{sl}}}
\DeclareMathOperator{\Wh}{Wh}
\DeclareMathOperator{\WF}{WF}

\def\Id{\mathrm{Id}}
\def\Ann{{\mathrm{Ann}\,}}
\def\Ker{{\rm Ker}\,}
\def\Lie{{\rm Lie}}
\def\Im{{\rm Im\,}}
\def\Hom{{\rm Hom}}
\def\End{{\rm End}}
\def\Mat{{\rm Mat}}
\def\Ind{{\rm Ind}}
\def\ind{{\rm ind}}
\def\Spec{{\rm Spec\,}}
\def\Supp{\mathrm{Supp}}
\def\Sym{\mathrm{Sym}}
\def\Alt{\mathrm{Alt}}
\def\Gr{\mathrm{Gr\,}}
\def\codim{\mathrm{codim}}
\def\rank{\mathrm{rank}}
\def\Gal{\mathrm{Gal}}

\def\cAV{{\cA\cV}}
\def\supp{{\rm{supp}\,}}
\def\wtMoo{{{\wtM^{(1,1)}}}}

\def\nosizesub#1#2{{{#1}_{\makebox[0pt][l]{$\scriptstyle {#2}$}}}}

% editing macros. 
\long\def\mjj#1{{{\color{magenta}#1}}}
\long\def\delete#1{}
\long\def\mjjd#1#2{{\color{blue}#1 \sout{#2}}}

\newcommand{\trivial}[2][]{\if\relax\detokenize{#1}\relax
{\green {\medskip The following is trivial. } #2}
\else 
\ifx#1h\relax  \else {\red Wrong argument!} \fi
\fi
}


\def\floor#1{{\lfloor #1 \rfloor}}
\def\ceil#1{{\lceil #1 \rceil}}
\def\foo{{\mathfrak{o}}}
\def\sV{{\mathscr{V}}}
\def\tjj{{\tilde{j}}}
\def\wtA{{\widetilde{A}}}
\def\iinn#1#2{\ll#1,#2\gg}
\def\fF{{\mathfrak{F}}}
\def\fM{{\mathfrak{M}}}
\def\Hom{{\mathrm{Hom}}}

\def\tww{{\tilde{w}}}
\def\tgg{{\tilde{g}}}
\def\hG{{\hat{G}}}
\def\LG{{{}^LG}}
\def\LN{{{}^LN}}
\def\hn{{\hat{n}}}
\def\hns{{\hat{n}_s}}
\def\hnt{{\hat{n}_t}}

\def\fppC{\fpp_\bC}
\def\fppCp{\fpp'_{\bC}}


\def\sW{{\mathscr{W}}}
\def\Stab{{\mathrm{Stab}}}
\def\Mp{{\mathrm{Mp}}}
\def\Sp{{\mathrm{Sp}}}
\def\SL{{\mathrm{SL}}}
\def\Irr{{\mathrm{Irr}}}
\def\cX{{\mathcal{X}}}
\def\cW{{\mathcal{W}}}
\def\rG{{\mathrm{G}}}
\def\Unip{{\mathrm{Unip}}}
\def\unip{{\mathrm{unip}}}
\def\Ch{{\mathrm{Ch}}}

\def\Xllp{{{\mathsf{X}}_{\lambda,\lambda'}}}
\def\Pil{{\pi_{\lambda,\check{\rho}}}}
\def\Pilp{{\pi'_{\lambda',\rho'}}}

\def\piSigma{{\pi_\Sigma}}
\def\piSigmap{{\pi'_{\Sigma'}}}
\def\piSigmapp{{\pi'_{\Sigma''}}}


\def\wteta{{\widetilde{\eta}}}
\def\wtpi{{\widetilde{\pi}}}
\def\tpiSigma{{\wtpi_\Sigma}}
\def\tpiSigmap{{\wtpi'_{\Sigma'}}}
\def\tpiSigmapp{{\wtpi'_{\Sigma''}}}

\def\Latto{{\mathrm{Latt^1}}}


\def\half{{\frac{1}{2}}}
\def\halfm{{\frac{m}{2}}}
\def\halfmm{{\frac{m-1}{2}}}
\def\fracmm{{\frac{1}{m}}}
\def\fracdmm{{\frac{1}{2m}}}

\def\cdt{{c}}
\def\bPsi{{\overline{\Psi}}}
\def\bpsi{{\overline{\psi}}}

\def\vG{{\overrightarrow{G}}}
\def\vrr{{\overrightarrow{r}}}
\def\vphi{{\overrightarrow{\phi}}}

\def\Cent#1#2{{\mathrm{Z}_{#1}({#2})}}

\def\bomega{{\overline{\omega}}}
\def\bomegaS{{\bomega_S}}
\def\bomegadgS{{\bomega_{\dgS}}}

%\def\cInd{{\mathrm{c-Ind}}}
\DeclareMathOperator{\cInd}{c-Ind}

\def\istar{{\filledstar}}
\def\mstar{{\medstar}}
\def\dstar{{\medstarofdavid}}

\def\skewinv{{\mathrm{skew-inv}}}
\def\btt{{\mathbbm{t}}}

\def\diagJ{{J^\triangle}}

\def\dd{{\rm{d}}}
\def\dalpha{{\dd\alpha}}
\def\dalphap{{\dd\alpha^\perp}}
\def\odalpha{\overline{\dd\alpha}}
\def\odalphap{\overline{\dd\alpha^\perp}}

\def\vol{{\mathrm{vol}}}

\def\Jump{{\mathrm{Jump}}}

\def\CO{\cO}
\def\bTheta{\overline{\Theta}}
\def\ckcO{{\check{\cO}}}
\def\ckfgg{{\check{\fgg}}}
\def\ckfgl{{\check{\fgl}}}
\def\ckGamma{{\check{\Gamma}}}
\def\ckww{{\check{w}}}
\def\ckD{{\check{D}}}

\def\sspan{{\mathrm{Span}}}

\def\dbK{{\breve{K}}}
\def\dbKp{{\dbK_+}}
\def\dbKzp{{\dbK_{0^+}}}
\def\dbpsi{{\breve{\psi}}}
\def\dbrho{{\breve{\rho}}}
\def\dbkappa{{\breve{\kappa}}}
\def\dbeta{{\breve{\eta}}}

\def\TTidx#1#2{\,{}^{#1}\hspace{-0.1em}#2}

\def\mydefTT#1#2#3{
\expandafter\def\csname ii#3{#1}\endcsname{\TTidx{i}{#2{#1}}}
\expandafter\def\csname zz#3{#1}\endcsname{\TTidx{0}{#2{#1}}}
\expandafter\def\csname ll#3{#1}\endcsname{\TTidx{l}{#2{#1}}}
\expandafter\def\csname aa#3{#1}\endcsname{\TTidx{a}{#2{#1}}}
\expandafter\def\csname bb#3{#1}\endcsname{\TTidx{b}{#2{#1}}}
\expandafter\def\csname oo#3{#1}\endcsname{\TTidx{1}{#2{#1}}}
\expandafter\def\csname ss#3{#1}\endcsname{\TTidx{\boxslash}{#2{#1}}}
\expandafter\def\csname dg#3{#1}\endcsname{\TTidx{\boxbackslash}{#2{#1}}}
}
\def\usecsname#1{\csname #1\endcsname}
\def\useLetter#1{#1}
\def\usedbletter#1{#1#1}

\def\Vp{V'}
\def\sLp{\sL'}
\def\Sigmap{\Sigma'}
\def\fggp{\fgg'}

\mydefTT{sB}{\usecsname}{\useLetter}
\mydefTT{Sigma}{\usecsname}{\useLetter}
\mydefTT{Sigmap}{\usecsname}{\useLetter}
\mydefTT{Gamma}{\usecsname}{\useLetter}
\mydefTT{Omega}{\usecsname}{\useLetter}
\mydefTT{phi}{\usecsname}{\useLetter}
\mydefTT{eta}{\usecsname}{\useLetter}
\mydefTT{kappa}{\usecsname}{\useLetter}
\mydefTT{dbkappa}{\usecsname}{\useLetter}
\mydefTT{bomega}{\usecsname}{\useLetter}
\mydefTT{rho}{\usecsname}{\useLetter}
\mydefTT{dbrho}{\usecsname}{\useLetter}
\mydefTT{bfbb}{\usecsname}{\useLetter}
\mydefTT{sL}{\usecsname}{\useLetter}
\mydefTT{sLp}{\usecsname}{\useLetter}
\mydefTT{G}{\useLetter}{\useLetter}
\mydefTT{K}{\useLetter}{\useLetter}
\mydefTT{W}{\useLetter}{\useLetter}
\mydefTT{S}{\useLetter}{\useLetter}
\mydefTT{V}{\useLetter}{\useLetter}
\mydefTT{Vp}{\usecsname}{\useLetter}
\mydefTT{dbK}{\usecsname}{\useLetter}
\mydefTT{dbG}{\usecsname}{\useLetter}
\mydefTT{End}{\usecsname}{\useLetter}

\mydefTT{x}{\useLetter}{\usedbletter}
\mydefTT{v}{\useLetter}{\usedbletter}
\mydefTT{w}{\useLetter}{\usedbletter}
\mydefTT{fgg}{\usecsname}{\useLetter}
\mydefTT{fggp}{\usecsname}{\useLetter}


\def\zzll{\ell}

\def\bbsSB{\sS(\bbsB_0)}
\def\aasSB{\sS(\aasB_0)}
\def\iisSB{\sS(\iisB_0)}
\def\ssdbkappa{\TTidx{\boxslash}{\dbkappa}}

\def\ssbasB{\TTidx{\boxslash}{\sB^{ba}}}
\def\ssabsB{\TTidx{\boxslash}{\sB^{ab}}}
\def\ssiota{\TTidx{\boxslash}{\iota}}

\def\ssbabfbb{\TTidx{\boxslash}{\bfbb^{ba}}}
\def\ssabbfbb{\TTidx{\boxslash}{\bfbb^{ab}}}


\def\ssbaW{\TTidx{\boxslash}{W^{ba}}}
\def\ssabW{\TTidx{\boxslash}{W^{ab}}}

\def\ggs{\fgg_{x,s}}
\def\ggsp{\fgg'_{x',s}}
\def\ggss{\fgg_{x,s:s^+}}
\def\ggssp{\fgg'_{x',s:s^+}}


\def\nuD{{\nu_D}}
\def\nuF{{\nu_F}}
\def\fooD{{\foo_D}}
\def\fppD{{\fpp_D}}
\def\fffD{{\fff_D}}

\def\tkappa{{\widetilde{\kappa}}}
\def\tpsi{{\widetilde{\psi}}}

\def\sfGz{\sfG^0_x}
\def\sfGzp{\sfG'^0_{x'}}

\def\MM{\sfM}
\def\MMp{\sfM'}
\def\Nil{\mathrm{Nil}}

\def\barX{\bar{X}}

\def\psiGa{\psi_\Gamma}


\def\Sec#1{\S~#1}

\def\blue{\color{blue}}
\def\gray{\color{gray}}
\def\red{\color{red}}
\def\lblue{\color{blue}}

\def\vV{\bfV^\vee}
\def\bSb{\bS(\bfbb)}
\def\vcO{\widehat{\cO}}


\def\bbF{\overline{\bF}}

\def\PBP{\mathrm{PBP}}
\def\LS{\mathrm{LS}}
\def\AOD{\mathrm{AOD}}

\def\gen#1{\langle #1 \rangle}

\let\oldemph\emph
\def\emph#1{\oldemph{\blue #1}}
%\def\emp#1{\blue #1}

\allowdisplaybreaks

\let\ytb=\ytableaushort

\setbeamertemplate{section in toc}[sections numbered]

\def\TTPG{
\begin{frame}[c]{}
    \tableofcontents[currentsection,hideallsubsections,subsubsectionstyle=hide]
\end{frame}
}

\usepackage{listings}
% \lstset{
%     basicstyle=\ttfamily\tiny,
%     keywordstyle=\color{black},
%     commentstyle=\color{white}, % white comments
%     stringstyle=\ttfamily, % typewriter type for strings
%     showstringspaces=false,
%     breaklines=true,
%     emph={Output},emphstyle=\color{blue},
% } 

\usepackage{tikz}
\usetikzlibrary{matrix,arrows,positioning,cd,backgrounds}
\usetikzlibrary{decorations.pathmorphing,decorations.pathreplacing}

\title[Uni. Repn.]{Unipotent representations \\
and theta correspondence}

\author[Ma, Jia-Jun]{Ma, Jia-Jun}

\institute[XMU]{School of Mathematics\\
Xiamen University\\[1.5em]
Department of Mathematics\\
Xiamen University Malaysia Campus
}



%\setbeamercovered{invisible}
%\setbeamercovered{transparent}
\setbeamertemplate{headline}{}
\setbeamertemplate{navigation symbols}{}
\defbeamertemplate{footline}{my footline}{%
\vskip1pt%
%\hrule
%\makebox[0pt][l]{\,\insertsection}%
\hspace*{\fill}%\insertshorttitle\hspace*{\fill}%
%\hfill
\llap{\insertpagenumber\,/\,\insertpresentationendpage\,}
%\hspace*{\fill}
\hspace*{\fill}
}
%\setbeamertemplate{footline}[my footline]
\setbeamertemplate{footline}[frame number]


\makeatletter
\newcommand\xleftrightarrow[2][]{%
  \ext@arrow 9999{\longleftrightarrowfill@}{#1}{#2}}
\newcommand\longleftrightarrowfill@{%
  \arrowfill@\leftarrow\relbar\rightarrow}
\makeatother

%\date{Dec. 13, 2024\\ at NCTS}

%\def\ckcO{\widehat{\cO}}
\def\dBV{d_{BV}}

\def\Witt{\mathrm{Witt}}
\def\Quad{\mathrm{Quad}}


\begin{document}

\begin{frame}[plain,label=tt]
    
    \titlepage
    \vspace{-3em}
% \centering{
%     (Zhejiang University)
% }
\end{frame}




% \begin{frame}[c]{Outline}
    %   \tableofcontents[hideallsubsections,subsubsectionstyle=hide]
    % \end{frame}

\def\vgraph#1#2{\ensuremath{\vcenter{\hbox{\includegraphics[width=#1]{#2}}}}}
\def\hgraph#1#2{\ensuremath{\vcenter{\hbox{\includegraphics[height=#1]{#2}}}}}
\def\hhgraph#1#2#3{\ensuremath{\begin{array}{c}\text{#3}\\
        \hbox{\includegraphics[height=#1]{#2}}
        \end{array}}}
    \section{Motivation}

    \begin{frame}{Jordan-Chevalley decomposition}
     \begin{itemize}[<+->]
       \item For each $g\in G:=\GL_{n}(\bC)$, $\exists !$ pair $(s,u)$
             such that
             \[
              g=su=us,
             \]
       \item $s$ is semisimple, and $u$ is unipotent ($u-1$ is nilpotent.)
       \item \hhgraph{0.15\textwidth}{Jordan.jpg}{Camille Jordan}
             \hhgraph{0.15\textwidth}{Chevalley.jpg}{Claude Chevalley}
       \item Classification of conjugation classes in $G$:
             \[
             G/\sim = \bigsqcup_{s\in G_{\mathrm{s.s.}}/\sim} \set{s u|u\in G^{s}}/\sim \;
             \xleftrightarrow{\ \ bij. \ \ } \bigsqcup_{s\in G_{\mathrm{s.s.}}/\sim} \unip(G^{s})
             \]
     \end{itemize}
   \end{frame}

    \begin{frame}{Finite group of Lie type: $G:=\bfG(\bF_{q})$}
      \begin{itemize}[<+->]
        \item \emph{Jordan decomposition of representations}:
        \item[] Langlands Dual group $\check{\bfG}$.
            E.g. $\Sp_{2n} = \bfG\leftrightsquigarrow \check{\bfG} = \SO_{2n+1}$\\
        \[
             \hhgraph{0.14\textwidth}{Deligne}{Deligne}
              \hspace{2em}
              \Irr(G) = \displaystyle \bigsqcup_{s \in \check{G}_{\text{s.s.}}/\sim
              %\substack{\text{semisimple class}\\ s \in \widehat{G}_{\text{s.s.}}/\sim}
            } \cE(G, s).
             \hspace{2em}
              \hhgraph{0.14\textwidth}{Lusztig}{Lusztig}
            \]
        \item[] Lusztig's map to the unipotent packet.
        \[
          \cE(G,s)\xleftrightarrow{\ \  bij. \ \ } \cE(\check{G}_{s},1)
        \]
        \item[] Preserve cuspidality. 
      \end{itemize}
    \end{frame}

    \begin{frame}
      \frametitle{Unipotent cuspidal representations}
      \begin{itemize}
        \item[] Let $q$ be an odd prime power.
        \item[] Unipotent cuspidal representations (for classical groups) exist only in the following cases:
        \item $\rU_n(\bF_q)$  has one $\pi_k$, when $n= k(k+1)/2$; 
        \item $\Sp_{2n}(\bF_q)$ has one $\pi_k^{\Sp}$, when $n = k(k+1)$
        \item $\rO^{\epsilon}_{2n}(\bF_q)$ has two $\pi_{k,a/b}^{e}$, when $n=k^2$ and $\epsilon = (-1)^k$. 
         \item $\rO^{\epsilon}_{2n+1}(\bF_q)$ has two $\pi_{k,a/b}^{o}$, when $n=k(k+1)$. 
      \end{itemize}
      By Adams-Moy, all the unipotent cuspidal representations can be consturced by theta lifting.
    \end{frame}

    \begin{frame}{Rational nilpotent orbits of orthogonal/symplectic groups}
      \begin{itemize}
      \item $\Quad(k) $ be the isometric classes of quadratic spaces over $k$.
      \item $\Witt(k)$ be the Witt group of a field $k$.
      \item[] We identify $\Witt(k)\times \bN =\Quad(k)$ via
      \[
      ([V_0], n)\mapsto V_0\oplus \bH^n \quad \text{with $V_0$ anisotropic}
      \]
      \item \pause The rational nilpotent oribts are parameterized by 
        formed Yong-diagram : 
        \[[(Q_1,r_1),(Q_2,r_2), \cdots (Q_l,r_l)]\] such that
      \begin{itemize}
      \item $Q_i \in \Quad(k)$,
      \item  $r_1> r_2> \cdots r_l>0$
      \item and $Q_i$ is split if $r_i$ is even for orthogonal group 
      \item[] $r_i$ is odd for symplectic group.  
      \end{itemize}
      \end{itemize}
    \end{frame}

    \def\ytb#1{{\tiny\ytableaushort{#1}}}
    \begin{frame}{}
      The chain/descent sequence of unipotent cuspdial representations \\
      \centering{
      \begin{tikzcd}[column sep=4em, row sep=0em, ampersand replacement=\&]
      \emptyset \&  \pi^e_{0,a}  \arrow[r] \&  \pi^{\Sp}_0 \ar[ld,""] \& \emptyset\\
      \ytb{Q,Q}\&  \pi^e_{1,a} \arrow[d,"\det\otimes"] \& \\
      \&  \pi^e_{1,b} \arrow[r,""] \& \pi^{\Sp}_1 \ar[ld] \& \ytb{Q\ ,Q\ }\\
      \ytb{Q\ \ ,Q\ \ ,Q,Q}\&  \pi^e_{2,a} \arrow[d,"\det\otimes"] \& \\
      \&  \pi^e_{2,b} \arrow[r,""] \& \pi^{\Sp}_2 \ar[ld] \& \ytb{Q\ \ \ ,Q\ \ \ ,Q\ ,Q\ } \\
       \ytb{Q\ \ \ \ ,Q\ \ \ \ ,Q\ \ ,Q\ \ ,Q,Q}\&  \pi^e_{3,a} \arrow[d,"\det\otimes"] \& \\
      \&  \pi^e_{3,b} \arrow[r,""] \& \pi^{\Sp}_3 \ar[ld] \\
      \&  \cdots \&
        \end{tikzcd}
      }\\
        \pause

      (The complete result on the rational Wavefront of finite classical group has been worked out by Z.-C. Wang)
    \end{frame}
    \let\ytb=\ytableaushort

\begin{frame}
  %\frametitle{Geometry of the \underline{moment maps}}
  \frametitle{Lifting of cycles}
  \begin{itemize}
  \item Example $(G,G')=(\Sp_{2n}, \rO_m)$ 
    \[
      \begin{tikzcd}[ampersand replacement=\&,column sep=3em, row sep=1em]
        \& M_{m,2n}\ar[rd,"\varphi"]
        \ar[ld,"\varphi'"'] \& \\
      \Alt^2_m\cong {\mathfrak{g}'} \&  \text{\footnotesize $A$} \ar[rd,mapsto]\ar[ld,mapsto]\& {\mathfrak g}\cong
        \Sym^2_{2n}\\
       \text{\footnotesize $X'=AJ A^T$} \& \& \text{\footnotesize $X = A^T A$}\\
      \end{tikzcd}
    \] \pause
  \item $\cO := G \cdot X$ is called the lift of $\cO' := G' \cdot X'$ 
  \item[] if $A$ is full rank and $m\leq 2n$.  \pause
  \item  One can define lifting of cycles use the geometry of moment maps 
    \[\vartheta^{\text{geo}}\colon
       \cK_{\cO'}(G') \longrightarrow \cK_{\cO}(G)\]
   \item[] {\lblue Theorem (Gomez-Zhu)} Theta lift of generalized Whittaker models 
    \[
       \Wh_{\cO}(\Theta(\pi')) = \vartheta^{\text{geo}} (\Wh_{\cO'}(\pi')),
     \]
  \end{itemize}
\end{frame}


    \begin{frame}
      \frametitle{Representations of Real Lie groups}
      \begin{itemize}[<+->]
        \item $G$ real reductive group\\[-1em]
        \item[]\
              \vspace{-1em}
        \[
          \set{\text{discrete series}}\subset\set{\text{tempered}}
          \subset\set{\text{unitary}}\subset\Irr(G)
        \]
        \item Unitary dual of $\SL_{2}(\bR)$:
        \[
          \vgraph{0.6\textwidth}{unitary_sl2.png}
        \]
        \item \emph{Open problem:} Structure of the unitary dual!
      \end{itemize}
    \end{frame}

    % \def\vG{\widehat{G}}
    % \begin{frame}
    %   \frametitle{Weakly unipotent representation}
    %   \begin{itemize}[<+->]
    %     \item \emph{Arthur} $\Irr_{\text{temp}}(G) \subset \Irr_{A}(G)\subset
    %     \Irr_{\text{unit}}(G)$
    %     \[
    %       \vgraph{0.15\textwidth}{Arthur.jpg}
    %       \hspace{1em}
    %       \Irr_{A}(G)\xrightarrow{\text{\ finite to finite\
    %         }} \set{\psi\colon WD_{k}\times \SL_{2}(\bC)\rightarrow {}^{L}G}/\vG.
    %     \]
    %     \item Barabasch-Vogan 
    %   \end{itemize}
    % \end{frame}

    \begin{frame}
      \frametitle{Arthur's unipotent representations (for real classical groups)}
      \begin{itemize}[<+->]
        \item $W_{\bR} = \bC^{\times} \cup j\, \bC^{\times}$
        \item $\psi:W_{\bR}\times \SL_{2}(\bC)\rightarrow {}^{L} G $
        \item Unipotent parameter: % $\Leftrightarrow$
              $\psi|_{\bC^\times}=$ trivial
        \item[] $\Leftrightarrow$ nilpotent orbit of a real from of $\vG$.
        \item \vgraph{0.15\textwidth}{Arthur2.jpg}  \emph{Conjecture:} $\exists$ unipotent Arthur
        packets
        \item[] ``On some problems suggested
        by the trace formula'' 1980's
        \item \hhgraph{0.15\textwidth}{Moeglin3.png}{M\oe glin}
        \hhgraph{0.15\textwidth}{Renard.jpg}{Renard}
        % \item[]
              \emph{Reduction to unipotent Arthur packet}
        \item[] ``Sur Les paquets d'Arthur des groupes classiques r\'eels''\\
        (2020)
      \end{itemize}
    \end{frame}



    \begin{frame}
      \frametitle{Barbasch-Vogan's definition of  special unip. repn.}
      \begin{itemize}[<+->]
            \item[] $G$: a real reductive group.
            \item  $\ckcO$ : a nilpotent orbit in $\ckG$ $\leadsto$ inf. char. $\chi_{\ckcO}$.
            \item[] \hspace{1em} $\leadsto$  the maximal primitive ideal
            $\cI_{\ckcO}\subset \cU(\fgg)$. %attached to $\ckcO$.
              %\hhgraph{0.11\textwidth}{Duflo}{Duflo}
              %\hhgraph{0.11\textwidth}{Joseph2}{Joseph}
            \item  \emph{Definition} (Barbasch-Vogan):
            \item [] An irr. adm. $G$-module is called
            \emph{special unipotent} if
            \[
            \Ann_{\cU(\fgg)}(\pi) = \cI_{\ckcO}.
            \]
            %\item  \emph{Theorem} (Barbasch-Vogan):
            \item[]
            $\Longleftrightarrow$ $\pi$ has inf. char. $\chi_{\ckcO}$ and
            $\AV_\bC(\pi) \leq  \dBV(\ckcO)$
            %\item $\Nil^{\text{special}}(\bfG)\ni \cO:=$
            %the \hhgraph{0.08\textwidth}{Lusztig2}{\ }
            %\hhgraph{0.08\textwidth}{Spaltenstein1}{\tiny Spaltenstein}
            %\hhgraph{0.08\textwidth}{Barbasch2}{\ }
            %\hhgraph{0.08\textwidth}{Vogan2}{\ }
            %dual of $\ckcO$.
            \item \emph{Weak Unipotent Packet:}
            \item[] $\Unip_{\ckcO}(G):= \{ $ special unipotent repn. attached to $\ckcO\}$.
            \item Conjecture 1:
            $\Unip_{\ckcO}(G)$ consists of {\red unitary} repn.
            \item Conjecture 2:
            $\Unip_{\ckcO}(G)$ is the union of certain {\red Arthur packets}.

      \end{itemize}
    \end{frame}
    \def\SU{\mathrm{SU}}
    \def\Spin{\mathrm{Spin}}
    \begin{frame}[label=UN]
        \frametitle{Special unip. repn. of simply conn. classical groups} 
        \begin{block}{Theorem (Barbasch-M.-Sun-Zhu)}
        Suppose $G$ is a simply connected real classical group, i.e. one of the following groups
        \[\SU(p,q), \Spin(p,q), \Spin(2n,\bH), \Sp(2n,\bR), \Mp(2n,\bR), \Sp(p,q)
        \] 
            Arthur-Barbasch-Vogan's unitarity conj. for special unipotent repn. holds:\\[1em]
            \pause
            \centering{All \emph{special unipotent repn.} of $G$  are \emph{unitarizable}.}
        \end{block}
        The Arthur parameter of these representations are determined by recent  work of Sun-Xu. 
    \end{frame}

    \begin{frame}{Reduction to the ``good parity''}
        \begin{itemize}
            \item Consider $G= \Sp(2n,\bR)$ for example. 
            \item  $\ckcO$ decompose into two parts $\ckcO_{g}$ (good parity)  and $\ckcO_{b}$ (bad parity).
            \item Assume $\ckcO_{b} = \{r_1, r_1, \cdots, r_k, r_k\}$ and
            \item Set $\ckcO'_b = \set{r_1,\cdots, r_k}\in \Nil_{\GL}$.\pause \\
            %  ,  $\abs{\ckcO_b}=2n_1$ and 
            % $|\ckcO_{g}| = 2n_0$
            \begin{block}{ Theorem  (Barbasch-M.-Sun-Zhu) }
              \hspace{2em}
              $
            \begin{array}{ccc}
                {\tiny\Unip_{\ckcO'_b}(\GL)
                \times \Unip_{\ckcO_g}(\Sp)} &\xrightarrow{\ \ bij.\ \ }&
                {\tiny \Unip_{\ckcO}(\Sp)} \\[.6em]
                (\pi',\pi_0 ) &\mapsto & 
                \Ind_{\raisebox{-0.5em}{\tiny$\GL_{\abs{\ckcO'_b}}\times \Sp(2n_0,\bR)\ltimes U$}}^{\tiny\raisebox{1em}{$\Sp(2n,\bR)$}}
                \hspace{-5em}\pi'\otimes \pi_0
            \end{array}
            $
          \end{block}
          \vspace{-1em}
            \[
            \begin{split}
                {\tiny\Unip_{\ckcO'_b}(\GL)}
                = 
                \Set{\Ind \mathop{\otimes}_{j=1}^k \sgn^{\epsilon_j}_{\GL(r_j,\bR)}
                % \Set{\Ind_{\raisebox{-0.7em}{\tiny$\GL(r_1)\times\cdots\times \GL(r_k)$}}^{\tiny\raisebox{1em}{$\GL(2n_1,\bR)$}} \hspace{-6.8em}\raisebox{0em}{ $\sgn^{\epsilon_1}\otimes \cdots \otimes\sgn^{\epsilon_k}$}
                % | 
                |\epsilon_j \in \bZ/2\bZ}
            \end{split}
            \]
            \pause
            \item Use \emph{theta correspondence}  to study 
            $ \Unip_{\ckcO_{g}}(G)$.
            \item We assume  $\ckcO$ has {\red good parity} from now on.
        \end{itemize}
    \end{frame}


    \begin{frame}{}
      \frametitle{Inductive structure of nilpotent orbits}
        \[
        \begin{tikzcd}[ampersand replacement=\&,column sep=1em,row sep=.5em]
            \check{\bfG}_i \&  \SO(15,\bC) \&  \rO(10,\bC) \&  \SO(7,\bC) \& \rO(4,\bC)\\
            \ckcO_i \& \ydiagram{5,3,3,3,1} \&
            \ydiagram{0,3,3,3,1} \& \ydiagram{0,0,3,3,1} \& \ydiagram{0,0,0,3,1}    \\
            \cO_i \& \ydiagram{4,4,4,2}\& \ydiagram{3,3,3,1} \& \ydiagram{2,2,2,0} \&
            \ydiagram{1,1,1,1}\\
            \bfG_i \&  \Sp(14,\bC) \&  \rO(10,\bC) \&  \Sp(6,\bC) \& \rO(4,\bC)\\
        \end{tikzcd}
      \]
      \\
        \pause
        Relate to
        Kraft-Procesi and Ohta's 
          study of singularities of
          a nilpotent orbit closure.
    \end{frame}


    \begin{frame}
      \frametitle{Construction of elements in $\Unip_{\ckcO}(G)$}
      \begin{itemize}
          \item 
          $\chi=\displaystyle\mathop{\otimes}_{j=0}^{a}\chi_{j}$, a character 
          of
          $\displaystyle\mathop{\textstyle\prod}_{j=0}^{a}G_{j}$.
          \item $\chi_j \in \{ \bfone, \sgn^{+,-}, \sgn^{-,+},  {\det}\}$ when $G_j$ is an orthogonal group.
          \item 
          Define a smooth repn. of $G = G_{a}$ 
          \[
          {\lblue \pi_{\chi}}:=(\omega_{G_{a},G_{a-1}}\widehat \otimes
          \omega_{G_{a-1},G_{a-2}} \widehat \otimes \cdots \widehat \otimes
          \omega_{G_1,G_0} \otimes \chi)_{G_{a-1}\times G_{a-2}\times \cdots \times G_{0}}
          \]
          The AC of $\pi_\chi$ is computable by an algorithm of lift of AC (Nishiyama-Zhu, Loke-M., BMSZ) 
          \pause 
          \begin{thm}[Barbasch-M.-Sun-Zhu]
              Suppose $\ckcO$ is an orbit with good parity.
              Then $\WF(\pi_\chi) = \Wh(\pi_\chi)$ 
              \begin{itemize}
                  \item either $\pi_\chi = 0$ or
                  $\lblue \pi_{\chi}\in  \Unip_{\ckcO}(G)$ and   
                  {\lblue unitarizable}.\pause
                  \item Moreover, 
                  \[\lblue
                  \Unip_{\ckcO}(G) = \set{\pi_{\chi} | \pi_{\chi}\neq 0}. 
                  \]
              \end{itemize}
          \end{thm}
      \end{itemize}
      %Moreover, 
  \end{frame}
  
  \begin{frame}
      \frametitle{Example 1: }
      Lift to $G = \Sp(8,\bR)$ from real forms of $\bfG = \rO(4,\bC)$.\\
      $\ckcO = 5 3 1$ and  $\cO = 2222$.
      Then 
      \[
      \begin{split}
      & \Unip_\ckcO(G) \\
      &= \Set{\pi_{p,q}^{\pm } := \text{ theta lift of trivial and sign of $\rO(p,q)$} | p+q=4} 
      \end{split}
      \]
      Then $\WF(\pi_{p,q}^{\pm}) = \Wh(\pi_{p,q}^{\pm})$ 
      consists of the single orbit:\\[2em]

      \centering{
      \ytb{
        \pm\mp,
        \pm\mp,
        \vdots\vdots,
        \pm\mp
      }
      }
  \end{frame}    
  
  
  \begin{frame}
      \frametitle{Example 2: Coincidences of theta liftings}
      Lift to $G = \Sp(6,\bR)$ from real forms of $\bfG = \rO(4,\bC)$.\\
      $\ckcO = 3^2 1^1$ and  $\cO = 2^3$.
      \[
      \begin{tikzcd}[ampersand replacement=\&,column sep=2em, row sep=1em,execute at end picture={
          \draw (-7em,-7em)--(-7em,8em);
        }]
          \& \& \Sp(6,\bR)\\[-1em]
          \hline
          \rO(4,0) \& \  \& \theta(\sgn^{+,-}) \ar[dr,equal]\& \   \\
          \rO(3,1) \& \theta(\bfone) \& \theta(\sgn^{+,-}) \ar[dr,equal] \& \theta(\sgn^{-,+}) \\
          \rO(2,2) \& \theta(\bfone) \& \theta(\sgn^{+,-}) \ar[dr,equal] \& \theta(\sgn^{-,+}) \\
          \rO(1,3) \& \theta(\bfone) \& \theta(\sgn^{+,-}) \ar[dr,equal] \& \theta(\sgn^{-,+}) \\
          \rO(0,4) \& \  \&\  \& \theta(\sgn^{-,+}) \\
      \end{tikzcd}
      \]
  \end{frame}    
    
  \begin{frame}{Example 2 (cont.)}
      All $\theta(\bfone)$ has reducible associated cycle. 
      \[
      \WF(\theta_{\rO(3,1)}^{\Sp_6(\bR)}(\bfone)) = 
      \ytb{
        -+, -+, -+ } \cup \ytb{ -+, -+, +- }
      \] 
      \[
      \WF(\theta_{\rO(2,2)}^{\Sp_6(\bR)}(\bfone)) = 
      \ytb{
        -+, -+, +- } \cup \ytb{ -+, +-, +- }
      \] 
      \[
      \WF(\theta_{\rO(1,3)}^{\Sp_6(\bR)}(\bfone)) = 
      \ytb{
        -+, +-, +- } \cup \ytb{ +-, +-, +- }
      \] 
  \end{frame}  
  \begin{frame}{Weak unipotent packet for $p$-adic group }
    \begin{itemize}
    \item $G$ : a split orthogonal group or symplectic group  defined over a $p$-adic field. 
    \item $\ckcO \in \Nil(\ckG)$, and $\ckhh \in \ckfgg$ is the semisimple element attached to $\ckcO$ \pause
    \item Weak unipotent packet of $\ckcO$ (Ciubotaru-Mason-Brown-Okada) 
    \item[] 
    \[
      \Unip_\ckcO(G):= \Set{X := X(q^{\half\ckhh},n, \rho) | 
    \WF(X) \subseteq \dBV(\ckcO)}
    \] 
    Here 
    \begin{itemize}
    \item $n\in \fgg^\vee$ such that $[\ckhh,n] = 2n$; 
    \item $\rho$ is an irreducible character of $A^1_{\ckG}(s,n)$;
    (more or less the componet group of $Z_\ckG(\set{\ckhh,n})$)  
    \item $X(q^{\half\ckhh},n, \rho)$ is Lusztig's unipotent. representation. 
    \end{itemize}
    \end{itemize}
  \end{frame}

  \def\AZ{\mathrm{AZ}}

  \begin{frame}{Elements in a weak unipotent packet}
    \begin{itemize}
      \item By Ciubotaru-Mason-Brown-Okada, 
      \[ 
      \Unip_\ckcO(G) = \Set{AZ(X(q^{\half \ckhh},n,\rho))| n \in \text{Special piece of $\ckcO$}}.  
      \]
      \item \pause\emph{Question:} Can we use theta lifting to construct all elements in $\Unip_\ckcO(G)$? 
    \end{itemize}
  \end{frame}

  \begin{frame}{Unipotent supercuspidal repn. (over a $p$-adic field $k$)}
    \begin{itemize}[<+->]
      \item Unipotent supercuspidal representation of symplectic group is parameterized by a pair of natural number $(k_1,k_2)$: 
      \item \emph{$\pi_{k_1,k_2}:=\cInd_P^{\Sp} \pi^\Sp_{k_1}\otimes \pi^\Sp_{k_2}$} here $P$ is a paraholic subgroup with reductive quotient $\Sp_{2k_1(k_1+1)}(\bF_q)\times \Sp_{2k_2(k_2+1)}(\bF_q)$. 
      \item $\Witt(\bF_q)\times \Witt(\bF_q)\stackrel{1-1}{\longleftrightarrow} \Witt(k)$. %\\ 
      %where $k$ is a $p$-adic field with residual field $\bF_q$. 
      \item descent sequence of $\pi^\Sp_{k_1}$ and $\pi^\Sp_{k_2}$ $\leadsto$ descent sequence of $\pi_{k_1,k_2}$.  
      \item Apply Gomez-Zhu $\leadsto$ the wavefront set of $\pi_{k_1,k_2}$. 
      \item $\WF(\pi_{k_1,k_2})$ contains a single orbit of ``triangular shape''  
      %\item[] Not every unipotent supercuspidal repn. is weakly unipotent (since infinitesimal character may not be real).  
    \end{itemize}
  \end{frame}


  \begin{frame}{Example 3: $\ckcO = \ytb{\ \ \ , \ \ \ ,\ }$  $\quad \dBV(\ckcO) = \ytb{\ \ ,\ \ , \ \ }$}
   \begin{itemize}
    \item $\Unip_\ckcO(\Sp_6)$ has two elements.
    \item They are the theta lifts.
    \item[] $\theta_{\rO^+_4}(1)$ 
    \item[] $\theta_{\rO^-_4}(1)$
    \item they have reducible wavefront set. 
   \end{itemize} 
  \end{frame}

  \begin{frame}{Example 4: $\ckcO = \ytb{\ \ \ \ \ , \ \ \ ,\ }$  $\quad \dBV(\ckcO) = \ytb{\ \ ,\ \ , \ \ ,\ \ }$}
   \begin{itemize}
    \item $\Unip_\ckcO(\Sp_8)$ has 5 elements.
    \item It is the union of two Arthur packets. 
    \item The anti-tempered packet 
    \[
    \Set{\delta=\theta_{\rO^+_4}(1), \pi_1=\theta_{\rO^+_4}(\det),  \pi_2=\theta_{\rO^-_4}(1), \pi_{sc}=\theta_{\rO^-_4}(\det)}
    \]
    \item They has irreducible wavefront set. 
    \item One non-anti-tempered packet 
      \[
      \Set{\pi_1, \pi_{sc},\tau^t }. 
      \]
   \end{itemize} 
  \end{frame}

  \begin{frame}{}
       \vfill
        \centering{\bf\Large\color{blue} Thank you for your attention!}
        \vfill
      \end{frame}

  \end{document}
    


%%% Local Variables: 
%%% coding: utf-8
%%% mode: latex
%%% TeX-engine: pdfletex
%%% TeX-master: t
%%% End:
