% !TeX program = xelatex
\documentclass[12pt,a4paper]{amsart}
\usepackage[margin=2.5cm,marginpar=2cm]{geometry}

\usepackage[bookmarksopen,bookmarksdepth=2,hidelinks,colorlinks=false]{hyperref}
\usepackage[nameinlink]{cleveref}

% \usepackage[color]{showkeys}
% \makeatletter
%   \SK@def\Cref#1{\SK@\SK@@ref{#1}\SK@Cref{#1}}%
% \makeatother
%% FONTS

\usepackage{amssymb}
%\usepackage{amsmath}
\usepackage{mathrsfs}
\usepackage{mathtools}
%\usepackage{amsrefs}
%\usepackage{mathbbol,mathabx}
\usepackage{amsthm}
\usepackage{graphicx}
\usepackage{braket}
%\usepackage[pointedenum]{paralist}
%\usepackage{paralist}
\usepackage{amscd}

\usepackage[alphabetic]{amsrefs}

\usepackage[all,cmtip]{xy}
\usepackage{rotating}
\usepackage{leftidx}
%\usepackage{arydshln}

%\DeclareSymbolFont{bbold}{U}{bbold}{m}{n}
%\DeclareSymbolFontAlphabet{\mathbbold}{bbold}


%\usepackage[dvipdfx,rgb,table]{xcolor}
\usepackage[rgb,table]{xcolor}
%\usepackage{mathrsfs}

\setcounter{tocdepth}{1}
\setcounter{secnumdepth}{2}

%\usepackage[abbrev,shortalphabetic]{amsrefs}


\usepackage[normalem]{ulem}

% circled number
\usepackage{pifont}
\makeatletter
\newcommand*{\circnuma}[1]{%
  \ifnum#1<1 %
    \@ctrerr
  \else
    \ifnum#1>20 %
      \@ctrerr
    \else
      \mbox{\ding{\numexpr 171+(#1)\relax}}%
     \fi
  \fi
}
\makeatother

\usepackage[centertableaux]{ytableau}


% Ytableau tweak
\makeatletter
\pgfkeys{/ytableau/options,
  noframe/.default = false,
  noframe/.is choice,
  noframe/true/.code = {%
    \global\let\vrule@YT=\vrule@none@YT
    \global\let\hrule@YT=\hrule@none@YT
  },
  noframe/false/.code = {%
    \global\let\vrule@YT=\vrule@normal@YT
    \global\let\hrule@YT=\hrule@normal@YT
  },
  noframe/on/.style = {noframe/true},
  noframe/off/.style = {noframe/false},
}

\def\hrule@enon@YT{%
  \hrule width  \dimexpr \boxdim@YT + \fboxrule *2 \relax
  height 0pt
}
\def\vrule@enon@YT{%
  \vrule height \dimexpr  \boxdim@YT + \fboxrule\relax
     width \fboxrule
}

\def\enon{\omit\enon@YT}
\newcommand{\enon@YT}[2][clear]{%
  \def\thisboxcolor@YT{#1}%
  \let\hrule@YT=\hrule@enon@YT
  \let\vrule@YT=\vrule@enon@YT
  \startbox@@YT#2\endbox@YT
  \nullfont
}

\makeatother
%\ytableausetup{noframe=on,smalltableaux}
\ytableausetup{noframe=off,boxsize=1.3em}
\let\ytb=\ytableaushort

\newcommand{\tytb}[1]{{\tiny\ytb{#1}}}


%\usepackage[mathlines,pagewise]{lineno}
%\linenumbers

\usepackage{enumitem}
%% Enumitem
\newlist{enumC}{enumerate}{1} % Conditions in Lemma/Theorem/Prop
\setlist[enumC,1]{label=(\alph*),wide,ref=(\alph*)}
\crefname{enumCi}{condition}{conditions}
\Crefname{enumCi}{Condition}{Conditions}
\newlist{enumT}{enumerate}{3} % "Theorem"=conclusions in Lemma/Theorem/Prop
\setlist[enumT]{label=(\roman*),wide}
\setlist[enumT,1]{label=(\roman*),wide}
\setlist[enumT,2]{label=(\alph*),ref ={(\roman{enumTi}.\alph*)}}
\setlist[enumT,3]{label=(\arabic*), ref ={(\roman{enumTi}.\alph{enumTii}.\alph*)}}
\crefname{enumTi}{}{}
\Crefname{enumTi}{Item}{Items}
\crefname{enumTii}{}{}
\Crefname{enumTii}{Item}{Items}
\crefname{enumTiii}{}{}
\Crefname{enumTiii}{Item}{Items}
\newlist{enumPF}{enumerate}{3}
\setlist[enumPF]{label=(\alph*),wide}
\setlist[enumPF,1]{label=(\roman*),wide}
\setlist[enumPF,2]{label=(\alph*)}
\setlist[enumPF,3]{label=\arabic*).}
\newlist{enumS}{enumerate}{3} % Statement outside Lemma/Theorem/Prop
\setlist[enumS]{label=\roman*)}
\setlist[enumS,1]{label=\roman*)}
\setlist[enumS,2]{label=\alph*)}
\setlist[enumS,3]{label=\arabic*.}
\newlist{enumI}{enumerate}{3} % items
\setlist[enumI,1]{label=\roman*),leftmargin=*}
\setlist[enumI,2]{label=\alph*), leftmargin=*}
\setlist[enumI,3]{label=\arabic*), leftmargin=*}
\newlist{enumIL}{enumerate*}{1} % inline enum
\setlist*[enumIL]{label=\roman*)}
\newlist{enumR}{enumerate}{1} % remarks
\setlist[enumR]{label=\arabic*.,wide,labelwidth=!, labelindent=0pt}
\crefname{enumRi}{remark}{remarks}

\crefname{equation}{}{}
\Crefname{equation}{Equation}{Equations}
\Crefname{lem}{Lemma}{Lemma}
\Crefname{thm}{Theorem}{Theorem}

\newlist{des}{description}{1}
\setlist[des]{font=\sffamily\bfseries}

% editing macros.
\blendcolors{!80!black}
\long\def\okay#1{\ifcsname highlightokay\endcsname
{\color{red} #1}
\else
{#1}
\fi
}
\long\def\editc#1{{\color{red} #1}}
\long\def\mjj#1{{{\color{blue}#1}}}
\long\def\mjjr#1{{\color{red} (#1)}}
\long\def\mjjd#1#2{{\color{blue} #1 \sout{#2}}}
\def\mjjb{\color{blue}}
\def\mjje{\color{black}}
\def\mjjcb{\color{green!50!black}}
\def\mjjce{\color{black}}

\long\def\sun#1{{{\color{cyan}#1}}}
\long\def\sund#1#2{{\color{cyan}#1  \sout{#2}}}
\long\def\mv#1{{{\color{red} {\bf move to a proper place:} #1}}}
\long\def\delete#1{}

%\reversemarginpar
\newcommand{\lokec}[1]{\marginpar{\color{blue}\tiny #1 \mbox{--loke}}}
\newcommand{\mjjc}[1]{\marginpar{\color{green}\tiny #1 \mbox{--ma}}}

\newcommand{\trivial}[2][]{\if\relax\detokenize{#1}\relax
  {%\hfill\break
   % \begin{minipage}{\textwidth}
      \color{orange} \vspace{0em} $[$  #2 $]$
  %\end{minipage}
  %\break
      \color{black}
  }
  \else
\ifx#1h
\ifcsname showtrivial\endcsname
{%\hfill\break
 % \begin{minipage}{\textwidth}
    \color{orange} \vspace{0em}  $[$ #2 $]$
%\end{minipage}
%\break
    \color{black}
}
\fi
\else {\red Wrong argument!} \fi
\fi
}

\newcommand{\byhide}[2][]{\if\relax\detokenize{#1}\relax
{\color{orange} \vspace{0em} Plan to delete:  #2}
\else
\ifx#1h\relax\fi
\fi
}



\newcommand{\Rank}{\mathrm{rk}}
\newcommand{\cqq}{\mathscr{D}}
\newcommand{\rsym}{\mathrm{sym}}
\newcommand{\rskew}{\mathrm{skew}}
\newcommand{\fraksp}{\mathfrak{sp}}
\newcommand{\frakso}{\mathfrak{so}}
\newcommand{\frakm}{\mathfrak{m}}
\newcommand{\frakp}{\mathfrak{p}}
\newcommand{\pr}{\mathrm{pr}}
\newcommand{\rhopst}{\rho'^*}
\newcommand{\Rad}{\mathrm{Rad}}
\newcommand{\Res}{\mathrm{Res}}
\newcommand{\Hol}{\mathrm{Hol}}
\newcommand{\AC}{\mathrm{AC}}
%\newcommand{\AS}{\mathrm{AS}}
\newcommand{\WF}{\mathrm{WF}}
\newcommand{\AV}{\mathrm{AV}}
\newcommand{\AVC}{\mathrm{AV}_\bC}
\newcommand{\VC}{\mathrm{V}_\bC}
\newcommand{\bfv}{\mathbf{v}}
\newcommand{\depth}{\mathrm{depth}}
\newcommand{\wtM}{\widetilde{M}}
\newcommand{\wtMone}{{\widetilde{M}^{(1,1)}}}

\newcommand{\nullpp}{N(\fpp'^*)}
\newcommand{\nullp}{N(\fpp^*)}
%\newcommand{\Aut}{\mathrm{Aut}}

\def\mstar{{\medstar}}


\newcommand{\bfone}{\mathbf{1}}
\newcommand{\piSigma}{\pi_\Sigma}
\newcommand{\piSigmap}{\pi'_\Sigma}


\newcommand{\sfVprime}{\mathsf{V}^\prime}
\newcommand{\sfVdprime}{\mathsf{V}^{\prime \prime}}
\newcommand{\gminusone}{\mathfrak{g}_{-\frac{1}{m}}}

\newcommand{\eva}{\mathrm{eva}}

\def\subset{\subseteq}


% \newcommand\iso{\xrightarrow{
%    \,\smash{\raisebox{-0.65ex}{\ensuremath{\scriptstyle\sim}}}\,}}

\def\Ueven{{U_{\rm{even}}}}
\def\Uodd{{U_{\rm{odd}}}}
\def\ttau{\tilde{\tau}}
\def\Wcp{W}
\def\Kur{{K^{\mathrm{u}}}}

\def\Im{\operatorname{Im}}

\providecommand{\bcN}{{\overline{\cN}}}



\makeatletter

\def\gen#1{\left\langle
    #1
      \right\rangle}
\makeatother

\makeatletter
\def\inn#1#2{\left\langle
      \def\ta{#1}\def\tb{#2}
      \ifx\ta\@empty{\;} \else {\ta}\fi ,
      \ifx\tb\@empty{\;} \else {\tb}\fi
      \right\rangle}
\def\binn#1#2{\left\lAngle
      \def\ta{#1}\def\tb{#2}
      \ifx\ta\@empty{\;} \else {\ta}\fi ,
      \ifx\tb\@empty{\;} \else {\tb}\fi
      \right\rAngle}
\makeatother

\makeatletter
\def\binn#1#2{\overline{\inn{#1}{#2}}}
\makeatother


\def\innwi#1#2{\inn{#1}{#2}_{W_i}}
\def\innw#1#2{\inn{#1}{#2}_{\bfW}}
\def\innv#1#2{\inn{#1}{#2}_{\bfV}}
\def\innbfv#1#2{\inn{#1}{#2}_{\bfV}}
\def\innvi#1#2{\inn{#1}{#2}_{V_i}}
\def\innvp#1#2{\inn{#1}{#2}_{\bfV'}}
\def\innp#1#2{\inn{#1}{#2}'}

% choose one of then
\def\simrightarrow{\iso}
\def\surj{\twoheadrightarrow}
%\def\simrightarrow{\xrightarrow{\sim}}

\newcommand\iso{\xrightarrow{
   \,\smash{\raisebox{-0.65ex}{\ensuremath{\scriptstyle\sim}}}\,}}

\newcommand\riso{\xleftarrow{
   \,\smash{\raisebox{-0.65ex}{\ensuremath{\scriptstyle\sim}}}\,}}









\usepackage{xparse}
\def\usecsname#1{\csname #1\endcsname}
\def\useLetter#1{#1}
\def\usedbletter#1{#1#1}

% \def\useCSf#1{\csname f#1\endcsname}

\ExplSyntaxOn

\def\mydefcirc#1#2#3{\expandafter\def\csname
  circ#3{#1}\endcsname{{}^\circ {#2{#1}}}}
\def\mydefvec#1#2#3{\expandafter\def\csname
  vec#3{#1}\endcsname{\vec{#2{#1}}}}
\def\mydefdot#1#2#3{\expandafter\def\csname
  dot#3{#1}\endcsname{\dot{#2{#1}}}}

\def\mydefacute#1#2#3{\expandafter\def\csname a#3{#1}\endcsname{\acute{#2{#1}}}}
\def\mydefbr#1#2#3{\expandafter\def\csname br#3{#1}\endcsname{\breve{#2{#1}}}}
\def\mydefbar#1#2#3{\expandafter\def\csname bar#3{#1}\endcsname{\bar{#2{#1}}}}
\def\mydefhat#1#2#3{\expandafter\def\csname hat#3{#1}\endcsname{\hat{#2{#1}}}}
\def\mydefwh#1#2#3{\expandafter\def\csname wh#3{#1}\endcsname{\widehat{#2{#1}}}}
\def\mydeft#1#2#3{\expandafter\def\csname t#3{#1}\endcsname{\tilde{#2{#1}}}}
\def\mydefu#1#2#3{\expandafter\def\csname u#3{#1}\endcsname{\underline{#2{#1}}}}
\def\mydefr#1#2#3{\expandafter\def\csname r#3{#1}\endcsname{\mathrm{#2{#1}}}}
\def\mydefb#1#2#3{\expandafter\def\csname b#3{#1}\endcsname{\mathbb{#2{#1}}}}
\def\mydefwt#1#2#3{\expandafter\def\csname wt#3{#1}\endcsname{\widetilde{#2{#1}}}}
%\def\mydeff#1#2#3{\expandafter\def\csname f#3{#1}\endcsname{\mathfrak{#2{#1}}}}
\def\mydefbf#1#2#3{\expandafter\def\csname bf#3{#1}\endcsname{\mathbf{#2{#1}}}}
\def\mydefc#1#2#3{\expandafter\def\csname c#3{#1}\endcsname{\mathcal{#2{#1}}}}
\def\mydefsf#1#2#3{\expandafter\def\csname sf#3{#1}\endcsname{\mathsf{#2{#1}}}}
\def\mydefs#1#2#3{\expandafter\def\csname s#3{#1}\endcsname{\mathscr{#2{#1}}}}
\def\mydefcks#1#2#3{\expandafter\def\csname cks#3{#1}\endcsname{{\check{
        \csname s#2{#1}\endcsname}}}}
\def\mydefckc#1#2#3{\expandafter\def\csname ckc#3{#1}\endcsname{{\check{
      \csname c#2{#1}\endcsname}}}}
\def\mydefck#1#2#3{\expandafter\def\csname ck#3{#1}\endcsname{{\check{#2{#1}}}}}

\cs_new:Npn \mydeff #1#2#3 {\cs_new:cpn {f#3{#1}} {\mathfrak{#2{#1}}}}

\cs_new:Npn \doGreek #1
{
  \clist_map_inline:nn {alpha,beta,gamma,Gamma,delta,Delta,epsilon,varepsilon,zeta,eta,theta,vartheta,Theta,iota,kappa,lambda,Lambda,mu,nu,xi,Xi,pi,Pi,rho,sigma,varsigma,Sigma,tau,upsilon,Upsilon,phi,varphi,Phi,chi,psi,Psi,omega,Omega,tG} {#1{##1}{\usecsname}{\useLetter}}
}

\cs_new:Npn \doSymbols #1
{
  \clist_map_inline:nn {otimes,boxtimes} {#1{##1}{\usecsname}{\useLetter}}
}

\cs_new:Npn \doAtZ #1
{
  \clist_map_inline:nn {A,B,C,D,E,F,G,H,I,J,K,L,M,N,O,P,Q,R,S,T,U,V,W,X,Y,Z} {#1{##1}{\useLetter}{\useLetter}}
}

\cs_new:Npn \doatz #1
{
  \clist_map_inline:nn {a,b,c,d,e,f,g,h,i,j,k,l,m,n,o,p,q,r,s,t,u,v,w,x,y,z} {#1{##1}{\useLetter}{\usedbletter}}
}

\cs_new:Npn \doallAtZ
{
\clist_map_inline:nn {mydefsf,mydeft,mydefu,mydefwh,mydefhat,mydefr,mydefwt,mydeff,mydefb,mydefbf,mydefc,mydefs,mydefck,mydefcks,mydefckc,mydefbar,mydefvec,mydefcirc,mydefdot,mydefbr,mydefacute} {\doAtZ{\csname ##1\endcsname}}
}

\cs_new:Npn \doallatz
{
\clist_map_inline:nn {mydefsf,mydeft,mydefu,mydefwh,mydefhat,mydefr,mydefwt,mydeff,mydefb,mydefbf,mydefc,mydefs,mydefck,mydefbar,mydefvec,mydefdot,mydefbr,mydefacute} {\doatz{\csname ##1\endcsname}}
}

\cs_new:Npn \doallGreek
{
\clist_map_inline:nn {mydefck,mydefwt,mydeft,mydefwh,mydefbar,mydefu,mydefvec,mydefcirc,mydefdot,mydefbr,mydefacute} {\doGreek{\csname ##1\endcsname}}
}

\cs_new:Npn \doallSymbols
{
\clist_map_inline:nn {mydefck,mydefwt,mydeft,mydefwh,mydefbar,mydefu,mydefvec,mydefcirc,mydefdot} {\doSymbols{\csname ##1\endcsname}}
}



\cs_new:Npn \doGroups #1
{
  \clist_map_inline:nn {GL,Sp,rO,rU,fgl,fsp,foo,fuu,fkk,fuu,ufkk,uK} {#1{##1}{\usecsname}{\useLetter}}
}

\cs_new:Npn \doallGroups
{
\clist_map_inline:nn {mydeft,mydefu,mydefwh,mydefhat,mydefwt,mydefck,mydefbar} {\doGroups{\csname ##1\endcsname}}
}


\cs_new:Npn \decsyms #1
{
\clist_map_inline:nn {#1} {\expandafter\DeclareMathOperator\csname ##1\endcsname{##1}}
}

\decsyms{Mp,id,SL,Sp,SU,SO,GO,GSO,GU,GSp,PGL,Pic,Lie,Mat,Ker,Hom,Ext,Ind,reg,res,inv,Isom,Det,Tr,Norm,Sym,Span,Stab,Spec,PGSp,PSL,tr,Ad,Br,Ch,Cent,End,Aut,Dvi,Frob,Gal,GL,Gr,DO,ur,vol,ab,Nil,Supp,rank,Sign}

\def\abs#1{\left|{#1}\right|}
\def\norm#1{{\left\|{#1}\right\|}}


% \NewDocumentCommand\inn{m m}{
% \left\langle
% \IfValueTF{#1}{#1}{000}
% ,
% \IfValueTF{#2}{#2}{000}
% \right\rangle
% }
\NewDocumentCommand\cent{o m }{
  \IfValueTF{#1}{
    \mathop{Z}_{#1}{(#2)}}
  {\mathop{Z}{(#2)}}
}


\def\fsl{\mathfrak{sl}}
\def\fsp{\mathfrak{sp}}


%\def\cent#1#2{{\mathrm{Z}_{#1}({#2})}}


\doallAtZ
\doallatz
\doallGreek
\doallGroups
\doallSymbols
\ExplSyntaxOff


% \usepackage{geometry,amsthm,graphics,tabularx,amssymb,shapepar}
% \usepackage{amscd}
% \usepackage{mathrsfs}


\usepackage{diagbox}
% Update the information and uncomment if AMS is not the copyright
% holder.
%\copyrightinfo{2006}{American Mathematical Society}
%\usepackage{nicematrix}
\usepackage{arydshln}

\usepackage{tikz}
\usetikzlibrary{matrix,arrows,positioning,cd,backgrounds}
\usetikzlibrary{decorations.pathmorphing,decorations.pathreplacing}

\usepackage{upgreek}

\usepackage{listings}
\lstset{
    basicstyle=\ttfamily\tiny,
    keywordstyle=\color{black},
    commentstyle=\color{white}, % white comments
    stringstyle=\ttfamily, % typewriter type for strings
    showstringspaces=false,
    breaklines=true,
    emph={Output},emphstyle=\color{blue},
}

\newcommand{\BA}{{\mathbb{A}}}
%\newcommand{\BB}{{\mathbb {B}}}
\newcommand{\BC}{{\mathbb {C}}}
\newcommand{\BD}{{\mathbb {D}}}
\newcommand{\BE}{{\mathbb {E}}}
\newcommand{\BF}{{\mathbb {F}}}
\newcommand{\BG}{{\mathbb {G}}}
\newcommand{\BH}{{\mathbb {H}}}
\newcommand{\BI}{{\mathbb {I}}}
\newcommand{\BJ}{{\mathbb {J}}}
\newcommand{\BK}{{\mathbb {U}}}
\newcommand{\BL}{{\mathbb {L}}}
\newcommand{\BM}{{\mathbb {M}}}
\newcommand{\BN}{{\mathbb {N}}}
\newcommand{\BO}{{\mathbb {O}}}
\newcommand{\BP}{{\mathbb {P}}}
\newcommand{\BQ}{{\mathbb {Q}}}
\newcommand{\BR}{{\mathbb {R}}}
\newcommand{\BS}{{\mathbb {S}}}
\newcommand{\BT}{{\mathbb {T}}}
\newcommand{\BU}{{\mathbb {U}}}
\newcommand{\BV}{{\mathbb {V}}}
\newcommand{\BW}{{\mathbb {W}}}
\newcommand{\BX}{{\mathbb {X}}}
\newcommand{\BY}{{\mathbb {Y}}}
\newcommand{\BZ}{{\mathbb {Z}}}
\newcommand{\Bk}{{\mathbf {k}}}

\newcommand{\CA}{{\mathcal {A}}}
\newcommand{\CB}{{\mathcal {B}}}
\newcommand{\CC}{{\mathcal {C}}}

\newcommand{\CE}{{\mathcal {E}}}
\newcommand{\CF}{{\mathcal {F}}}
\newcommand{\CG}{{\mathcal {G}}}
\newcommand{\CH}{{\mathcal {H}}}
\newcommand{\CI}{{\mathcal {I}}}
\newcommand{\CJ}{{\mathcal {J}}}
\newcommand{\CK}{{\mathcal {K}}}
\newcommand{\CL}{{\mathcal {L}}}
\newcommand{\CM}{{\mathcal {M}}}
\newcommand{\CN}{{\mathcal {N}}}
\newcommand{\CO}{{\mathcal {O}}}
\newcommand{\CP}{{\mathcal {P}}}
\newcommand{\CQ}{{\mathcal {Q}}}
\newcommand{\CR}{{\mathcal {R}}}
\newcommand{\CS}{{\mathcal {S}}}
\newcommand{\CT}{{\mathcal {T}}}
\newcommand{\CU}{{\mathcal {U}}}
\newcommand{\CV}{{\mathcal {V}}}
\newcommand{\CW}{{\mathcal {W}}}
\newcommand{\CX}{{\mathcal {X}}}
\newcommand{\CY}{{\mathcal {Y}}}
\newcommand{\CZ}{{\mathcal {Z}}}


\newcommand{\RA}{{\mathrm {A}}}
\newcommand{\RB}{{\mathrm {B}}}
\newcommand{\RC}{{\mathrm {C}}}
\newcommand{\RD}{{\mathrm {D}}}
\newcommand{\RE}{{\mathrm {E}}}
\newcommand{\RF}{{\mathrm {F}}}
\newcommand{\RG}{{\mathrm {G}}}
\newcommand{\RH}{{\mathrm {H}}}
\newcommand{\RI}{{\mathrm {I}}}
\newcommand{\RJ}{{\mathrm {J}}}
\newcommand{\RK}{{\mathrm {K}}}
\newcommand{\RL}{{\mathrm {L}}}
\newcommand{\RM}{{\mathrm {M}}}
\newcommand{\RN}{{\mathrm {N}}}
\newcommand{\RO}{{\mathrm {O}}}
\newcommand{\RP}{{\mathrm {P}}}
\newcommand{\RQ}{{\mathrm {Q}}}
%\newcommand{\RR}{{\mathrm {R}}}
\newcommand{\RS}{{\mathrm {S}}}
\newcommand{\RT}{{\mathrm {T}}}
\newcommand{\RU}{{\mathrm {U}}}
\newcommand{\RV}{{\mathrm {V}}}
\newcommand{\RW}{{\mathrm {W}}}
\newcommand{\RX}{{\mathrm {X}}}
\newcommand{\RY}{{\mathrm {Y}}}
\newcommand{\RZ}{{\mathrm {Z}}}

\DeclareMathOperator{\absNorm}{\mathfrak{N}}
\DeclareMathOperator{\Ann}{Ann}
\DeclareMathOperator{\LAnn}{L-Ann}
\DeclareMathOperator{\RAnn}{R-Ann}
\DeclareMathOperator{\ind}{ind}
%\DeclareMathOperator{\Ind}{Ind}



\newcommand{\cod}{{\mathrm{cod}}}
\newcommand{\cont}{{\mathrm{cont}}}
\newcommand{\cl}{{\mathrm{cl}}}
\newcommand{\cusp}{{\mathrm{cusp}}}

\newcommand{\disc}{{\mathrm{disc}}}
\renewcommand{\div}{{\mathrm{div}}}



\newcommand{\Gm}{{\mathbb{G}_m}}



\newcommand{\I}{{\mathrm{I}}}

\newcommand{\Jac}{{\mathrm{Jac}}}
\newcommand{\PM}{{\mathrm{PM}}}


\newcommand{\new}{{\mathrm{new}}}
\newcommand{\NS}{{\mathrm{NS}}}
\newcommand{\N}{{\mathrm{N}}}

\newcommand{\ord}{{\mathrm{ord}}}

%\newcommand{\rank}{{\mathrm{rank}}}

\newcommand{\rk}{{\mathrm{k}}}
\newcommand{\rr}{{\mathrm{r}}}
\newcommand{\rh}{{\mathrm{h}}}

\newcommand{\Sel}{{\mathrm{Sel}}}
\newcommand{\Sim}{{\mathrm{Sim}}}

\newcommand{\wt}{\widetilde}
\newcommand{\wh}{\widehat}
\newcommand{\pp}{\frac{\partial\bar\partial}{\pi i}}
\newcommand{\pair}[1]{\langle {#1} \rangle}
\newcommand{\wpair}[1]{\left\{{#1}\right\}}
\newcommand{\intn}[1]{\left( {#1} \right)}
\newcommand{\sfrac}[2]{\left( \frac {#1}{#2}\right)}
\newcommand{\ds}{\displaystyle}
\newcommand{\ov}{\overline}
\newcommand{\incl}{\hookrightarrow}
\newcommand{\lra}{\longrightarrow}
\newcommand{\imp}{\Longrightarrow}
%\newcommand{\lto}{\longmapsto}
\newcommand{\bs}{\backslash}

\newcommand{\cover}[1]{\widetilde{#1}}

\renewcommand{\vsp}{{\vspace{0.2in}}}

\newcommand{\Norma}{\operatorname{N}}
\newcommand{\Ima}{\operatorname{Im}}
\newcommand{\con}{\textit{C}}
\newcommand{\gr}{\operatorname{gr}}
\newcommand{\ad}{\operatorname{ad}}
\newcommand{\der}{\operatorname{der}}
\newcommand{\dif}{\operatorname{d}\!}
\newcommand{\pro}{\operatorname{pro}}
\newcommand{\Ev}{\operatorname{Ev}}
% \renewcommand{\span}{\operatorname{span}} \span is an innernal command.
%\newcommand{\degree}{\operatorname{deg}}
\newcommand{\Invf}{\operatorname{Invf}}
\newcommand{\Inv}{\operatorname{Inv}}
\newcommand{\slt}{\operatorname{SL}_2(\mathbb{R})}
%\newcommand{\temp}{\operatorname{temp}}
%\newcommand{\otop}{\operatorname{top}}
\renewcommand{\small}{\operatorname{small}}
\newcommand{\HC}{\operatorname{HC}}
\newcommand{\lef}{\operatorname{left}}
\newcommand{\righ}{\operatorname{right}}
\newcommand{\Diff}{\operatorname{DO}}
\newcommand{\diag}{\operatorname{diag}}
\newcommand{\sh}{\varsigma}
\newcommand{\sch}{\operatorname{sch}}
%\newcommand{\oleft}{\operatorname{left}}
%\newcommand{\oright}{\operatorname{right}}
\newcommand{\open}{\operatorname{open}}
\newcommand{\sgn}{\operatorname{sgn}}
\newcommand{\triv}{\operatorname{triv}}
\newcommand{\Sh}{\operatorname{Sh}}
\newcommand{\oN}{\operatorname{N}}

\newcommand{\oc}{\operatorname{c}}
\newcommand{\od}{\operatorname{d}}
\newcommand{\os}{\operatorname{s}}
\newcommand{\ol}{\operatorname{l}}
\newcommand{\oL}{\operatorname{L}}
\newcommand{\oJ}{\operatorname{J}}
\newcommand{\oH}{\operatorname{H}}
\newcommand{\oO}{\operatorname{O}}
\newcommand{\oS}{\operatorname{S}}
\newcommand{\oR}{\operatorname{R}}
\newcommand{\oT}{\operatorname{T}}
%\newcommand{\rU}{\operatorname{U}}
\newcommand{\oZ}{\operatorname{Z}}
\newcommand{\oD}{\textit{D}}
\newcommand{\oW}{\textit{W}}
\newcommand{\oE}{\operatorname{E}}
\newcommand{\oP}{\operatorname{P}}
\newcommand{\PD}{\operatorname{PD}}
\newcommand{\oU}{\operatorname{U}}

\newcommand{\g}{\mathfrak g}
\newcommand{\gC}{{\mathfrak g}_{\C}}
\renewcommand{\k}{\mathfrak k}
\newcommand{\h}{\mathfrak h}
\newcommand{\p}{\mathfrak p}
%\newcommand{\q}{\mathfrak q}
\renewcommand{\a}{\mathfrak a}
\renewcommand{\b}{\mathfrak b}
\renewcommand{\c}{\mathfrak c}
\newcommand{\n}{\mathfrak n}
\renewcommand{\u}{\mathfrak u}
%\renewcommand{\v}{\mathfrak v}
\newcommand{\e}{\mathfrak e}
\newcommand{\f}{\mathfrak f}
\renewcommand{\l}{\mathfrak l}
\renewcommand{\t}{\mathfrak t}
\newcommand{\s}{\mathfrak s}
\renewcommand{\r}{\mathfrak r}
\renewcommand{\o}{\mathfrak o}
\newcommand{\m}{\mathfrak m}
\newcommand{\z}{\mathfrak z}
%\renewcommand{\sl}{\mathfrak s \mathfrak l}
\newcommand{\gl}{\mathfrak g \mathfrak l}


\newcommand{\re}{\mathrm e}

\renewcommand{\rk}{\mathrm k}

\newcommand{\Z}{\mathbb{Z}}
\DeclareDocumentCommand{\C}{}{\mathbb{C}}
\newcommand{\R}{\mathbb R}
\newcommand{\Q}{\mathbb Q}
\renewcommand{\H}{\mathbb{H}}
%\newcommand{\N}{\mathbb{N}}
\newcommand{\K}{\mathbb{K}}
%\renewcommand{\S}{\mathbf S}
\newcommand{\M}{\mathbf{M}}
\newcommand{\A}{\mathbb{A}}
\newcommand{\B}{\mathbf{B}}
%\renewcommand{\G}{\mathbf{G}}
\newcommand{\V}{\mathbf{V}}
\newcommand{\W}{\mathbf{W}}
\newcommand{\F}{\mathbf{F}}
\newcommand{\E}{\mathbf{E}}
%\newcommand{\J}{\mathbf{J}}
\renewcommand{\H}{\mathbf{H}}
\newcommand{\X}{\mathbf{X}}
\newcommand{\Y}{\mathbf{Y}}
%\newcommand{\RR}{\mathcal R}
\newcommand{\FF}{\mathcal F}
%\newcommand{\BB}{\mathcal B}
\newcommand{\HH}{\mathcal H}
%\newcommand{\UU}{\mathcal U}
%\newcommand{\MM}{\mathcal M}
%\newcommand{\CC}{\mathcal C}
%\newcommand{\DD}{\mathcal D}
%


 \def\tnaive{\mathrm{naive}}

 \def\Ass{A_{\mathrm{ss}}}
 \def\Ans{A_{\mathrm{ns}}}
 \def\wpu{\wp_{\uparrow}}
 \def\wpm{\wp_{\downarrow}}
 \def\wpd{\wp} % define the done-wp to be \wp
 \def\uptauu{\uptau_{\uparrow}}
 \def\uptaud{\uptau} % define the done-tau to be \uptau

\def\PPm{\wp_{\downarrow}}
\def\tauwpp{\tau'_{\wp'}}
\def\tauwppp{\tau''_{\wp''}}
\def\uptaum{\uptau_{\downarrow}}
\def\uptaupn{\uptau'_{\tnaive}}
\def\alphapn{\alpha'_{\tnaive}}
\def\imathpn{\imath'_{\tnaive}}
\def\jmathpn{\jmath'_{\tnaive}}

\def\eDD{\mathrm{d}^{e}}
\def\ckDD{{\check\DD}}
\def\DD{\nabla}
\def\DDn{\nabla_{\tnaive}}
\def\ckDDn{{\ckDD}_{\tnaive}}
\def\DDD{{\check\nabla}}
\def\DDc{\boldsymbol{\nabla}}
\def\gDD{\nabla^{\mathrm{gen}}}
\def\gDDc{\boldsymbol{\nabla}^{\mathrm{gen}}}
%\newcommand{\OO}{\mathcal O}
%\newcommand{\ZZ}{\mathcal Z}
\newcommand{\ve}{{\vee}}
\newcommand{\aut}{\mathcal A}
\newcommand{\ii}{\mathbf{i}}
\newcommand{\jj}{\mathbf{j}}
\newcommand{\kk}{\mathbf{k}}

\newcommand{\la}{\langle}
\newcommand{\ra}{\rangle}
\newcommand{\bp}{\bigskip}
\newcommand{\be}{\begin {equation}}
\newcommand{\ee}{\end {equation}}

\newcommand{\LRleq}{\stackrel{LR}{\leq}}

\numberwithin{equation}{section}


\def\flushl#1{\ifmmode\makebox[0pt][l]{${#1}$}\else\makebox[0pt][l]{#1}\fi}
\def\flushr#1{\ifmmode\makebox[0pt][r]{${#1}$}\else\makebox[0pt][r]{#1}\fi}
\def\flushmr#1{\makebox[0pt][r]{${#1}$}}


%\theoremstyle{Theorem}
% \newtheorem*{thmM}{Main Theorem}
% \crefformat{thmM}{main theorem}
% \Crefformat{thmM}{Main Theorem}
\newtheorem*{thm*}{Theorem}
\newtheorem{thm}{Theorem}[section]
\newtheorem{thml}[thm]{Theorem}
\newtheorem{lem}[thm]{Lemma}
\newtheorem{obs}[thm]{Observation}
\newtheorem{lemt}[thm]{Lemma}
\newtheorem*{lem*}{Lemma}
\newtheorem{whyp}[thm]{Working Hypothesis}
\newtheorem{prop}[thm]{Proposition}
\newtheorem{prpt}[thm]{Proposition}
\newtheorem{prpl}[thm]{Proposition}
\newtheorem{cor}[thm]{Corollary}
%\newtheorem*{prop*}{Proposition}
\newtheorem{claim}{Claim}
\newtheorem*{claim*}{Claim}
%\theoremstyle{definition}
\newtheorem{defn}[thm]{Definition}
\newtheorem{dfnl}[thm]{Definition}
\newtheorem*{IndH}{Induction Hypothesis}

\newtheorem*{eg*}{Example}
\newtheorem{eg}[thm]{Example}
\newtheorem{Example}[thm]{Example}
\newtheorem{conj}[thm]{Conjecture}

\theoremstyle{remark}
\newtheorem{remark}[thm]{Remark}
\newtheorem{remarks}[thm]{Remarks}


\def\cpc{\sigma}
\def\ccJ{\epsilon\dotepsilon}
\def\ccL{c_L}

\def\wtbfK{\widetilde{\bfK}}
%\def\abfV{\acute{\bfV}}
\def\AbfV{\acute{\bfV}}
%\def\afgg{\acute{\fgg}}
%\def\abfG{\acute{\bfG}}
\def\abfV{\bfV'}
\def\afgg{\fgg'}
\def\abfG{\bfG'}

\def\half{{\tfrac{1}{2}}}
\def\ihalf{{\tfrac{\mathbf i}{2}}}
\def\slt{\fsl_2(\bC)}
\def\sltr{\fsl_2(\bR)}

% \def\Jslt{{J_{\fslt}}}
% \def\Lslt{{L_{\fslt}}}
\def\slee{{
\begin{pmatrix}
 0 & 1\\
 0 & 0
\end{pmatrix}
}}
\def\slff{{
\begin{pmatrix}
 0 & 0\\
 1 & 0
\end{pmatrix}
}}\def\slhh{{
\begin{pmatrix}
 1 & 0\\
 0 & -1
\end{pmatrix}
}}
\def\sleei{{
\begin{pmatrix}
 0 & i\\
 0 & 0
\end{pmatrix}
}}
\def\slxx{{\begin{pmatrix}
-\ihalf & \half\\
\phantom{-}\half & \ihalf
\end{pmatrix}}}
% \def\slxx{{\begin{pmatrix}
% -\sqrt{-1}/2 & 1/2\\
% 1/2 & \sqrt{-1}/2
% \end{pmatrix}}}
\def\slyy{{\begin{pmatrix}
\ihalf & \half\\
\half & -\ihalf
\end{pmatrix}}}
\def\slxxi{{\begin{pmatrix}
+\half & -\ihalf\\
-\ihalf & -\half
\end{pmatrix}}}
\def\slH{{\begin{pmatrix}
   0   & -\mathbf i\\
\mathbf i & 0
\end{pmatrix}}
}

\ExplSyntaxOn
\clist_map_inline:nn {J,L,C,X,Y,H,c,e,f,h,}{
  \expandafter\def\csname #1slt\endcsname{{\mathring{#1}}}}
\ExplSyntaxOff


\def\Mop{\fT}

\def\fggJ{\fgg_J}
\def\fggJp{\fgg'_{J'}}

\def\NilGC{\Nil_{\bfG}(\fgg)}
\def\NilGCp{\Nil_{\bfG'}(\fgg')}
\def\Nilgp{\Nil_{\fgg'_{J'}}}
\def\Nilg{\Nil_{\fgg_{J}}}
%\def\NilP'{\Nil_{\fpp'}}
\def\nNil{\Nil^{\mathrm n}}
\def\eNil{\Nil^{\mathrm e}}


\NewDocumentCommand{\NilP}{t'}{
\IfBooleanTF{#1}{\Nil_{\fpp'}}{\Nil_\fpp}
}

\def\KS{\mathsf{KS}}
\def\MM{\bfM}
\def\MMP{M}

\NewDocumentCommand{\KTW}{o g}{
  \IfValueTF{#2}{
    \left.\varsigma_{\IfValueT{#1}{#1}}\right|_{#2}}{
    \varsigma_{\IfValueT{#1}{#1}}}
}
\def\IST{\rho}
\def\tIST{\trho}

\NewDocumentCommand{\CHI}{o g}{
  \IfValueTF{#1}{
    {\chi}_{\left[#1\right]}}{
    \IfValueTF{#2}{
      {\chi}_{\left(#2\right)}}{
      {\chi}}
  }
}
\NewDocumentCommand{\PR}{g}{
  \IfValueTF{#1}{
    \mathop{\pr}_{\left(#1\right)}}{
    \mathop{\pr}}
}
\NewDocumentCommand{\XX}{g}{
  \IfValueTF{#1}{
    {\cX}_{\left(#1\right)}}{
    {\cX}}
}
\NewDocumentCommand{\PP}{g}{
  \IfValueTF{#1}{
    {\fpp}_{\left(#1\right)}}{
    {\fpp}}
}
\NewDocumentCommand{\LL}{g}{
  \IfValueTF{#1}{
    {\bfL}_{\left(#1\right)}}{
    {\bfL}}
}
\NewDocumentCommand{\ZZ}{g}{
  \IfValueTF{#1}{
    {\cZ}_{\left(#1\right)}}{
    {\cZ}}
}

\NewDocumentCommand{\WW}{g}{
  \IfValueTF{#1}{
    {\bfW}_{\left(#1\right)}}{
    {\bfW}}
}




\def\gpi{\wp}
\NewDocumentCommand\KK{g}{
\IfValueTF{#1}{K_{(#1)}}{K}}
% \NewDocumentCommand\OO{g}{
% \IfValueTF{#1}{\cO_{(#1)}}{K}}
\NewDocumentCommand\XXo{d()}{
\IfValueTF{#1}{\cX^\circ_{(#1)}}{\cX^\circ}}
\def\bfWo{\bfW^\circ}
\def\bfWoo{\bfW^{\circ \circ}}
\def\bfWg{\bfW^{\mathrm{gen}}}
\def\Xg{\cX^{\mathrm{gen}}}
\def\Xo{\cX^\circ}
\def\Xoo{\cX^{\circ \circ}}
\def\fppo{\fpp^\circ}
\def\fggo{\fgg^\circ}
\NewDocumentCommand\ZZo{g}{
\IfValueTF{#1}{\cZ^\circ_{(#1)}}{\cZ^\circ}}

% \ExplSyntaxOn
% \NewDocumentCommand{\bcO}{t' E{^_}{{}{}}}{
%   \overline{\cO\sb{\use_ii:nn#2}\IfBooleanTF{#1}{^{'\use_i:nn#2}}{^{\use_i:nn#2}}
%   }
% }
% \ExplSyntaxOff

\NewDocumentCommand{\bcO}{t'}{
  \overline{\cO\IfBooleanT{#1}{'}}}

\NewDocumentCommand{\oliftc}{g}{
\IfValueTF{#1}{\boldsymbol{\vartheta} (#1)}{\boldsymbol{\vartheta}}
}
\NewDocumentCommand{\oliftr}{g}{
\IfValueTF{#1}{\vartheta_\bR(#1)}{\vartheta_\bR}
}
\NewDocumentCommand{\olift}{g}{
\IfValueTF{#1}{\vartheta(#1)}{\vartheta}
}
% \NewDocumentCommand{\dliftv}{g}{
% \IfValueTF{#1}{\ckvartheta(#1)}{\ckvartheta}
% }
\def\dliftv{\vartheta}
\NewDocumentCommand{\tlift}{g}{
\IfValueTF{#1}{\wtvartheta(#1)}{\wtvartheta}
}

\def\slift{\cL}

\def\BB{\bB}


\def\PhiO#1{\vartheta\left(#1\right)}

\def\bbThetav{\check{\mathbbold{\Phi}}}
\def\Phiv{\check{\Phi}}
\def\Phiv{\check{\Phi}}

\DeclareDocumentCommand{\NN}{g}{
\IfValueTF{#1}{\fN(#1)}{\fN}
}
\DeclareDocumentCommand{\RR}{m m}{
\fR({#1},{#2})
}

%\DeclareMathOperator*{\sign}{Sign}

\NewDocumentCommand{\lsign}{m}{
{}^l\mathrm{Sign}(#1)
}

% \NewDocumentCommand{\Sign}{m}{
% \mathrm{Sign}(#1)
% }


\NewDocumentCommand\lnn{t+ t- g}{
  \IfBooleanTF{#1}{{}^l n^+\IfValueT{#3}{(#3)}}{
    \IfBooleanTF{#2}{{}^l n^-\IfValueT{#3}{(#3)}}{}
  }
}


% % Fancy bcO, support feature \bcO'^a_{\mathrm b} = \overline{\cO'^a_{\mathrm b}}
% \makeatletter
% \def\bcO{\def\O@@{\cO}\@ifnextchar'\@Op\@Onp}
% \def\@Opnext{\@ifnextchar^\@Opsp\@Opnsp}
% \def\@Op{\afterassignment\@Opnext\let\scratch=}
% \def\@Opnsp{\def\O@@{\cO'}\@Otsb}
% \def\@Onp{\@ifnextchar^\@Onpsp\@Otsb}
% \def\@Opsp^#1{\def\O@@{\cO'^{#1}}\@Otsb}
% \def\@Onpsp^#1{\def\O@@{\cO^{#1}}\@Otsb}
% \def\@Otsb{\@ifnextchar_\@Osb{\@Ofinalnsb}}
% \def\@Osb_#1{\overline{\O@@_{#1}}}
% \def\@Ofinalnsb{\overline{\O@@}}

% Fancy \command: \command`#1 will translate to {}^{#1}\bfV, i.e. superscript on the
% lift conner.

% \def\defpcmd#1{
%   \def\nn@tmp{#1}
%   \def\nn@np@tmp{@np@#1}
%   \expandafter\let\csname\nn@np@tmp\expandafter\endcsname \csname\nn@tmp\endcsname
%   \expandafter\def\csname @pp@#1\endcsname`##1{{}^{##1}{\csname @np@#1\endcsname}}
%   \expandafter\def\csname #1\endcsname{\,\@ifnextchar`{\csname
%       @pp@#1\endcsname}{\csname @np@#1\endcsname}}
% }

% \def\defppcmd#1{
% \expandafter\NewDocumentCommand{\csname #1\endcsname}{##1 }{}
% }



% \defpcmd{bfV}
% \def\KK{\bfK}\defpcmd{KK}
% \defpcmd{bfG}
% \def\A{\!A}\defpcmd{A}
% \def\K{\!K}\defpcmd{K}
% \def\G{G}\defpcmd{G}
% \def\J{\!J}\defpcmd{J}
% \def\L{\!L}\defpcmd{L}
% \def\eps{\epsilon}\defpcmd{eps}
% \def\pp{p}\defpcmd{pp}
% \defpcmd{wtK}
% \makeatother

\def\fggR{\fgg_\bR}
\def\rmtop{{\mathrm{top}}}
\def\dimo{\dim^\circ}
\def\GKdim{\text{GK-}\dim}

\NewDocumentCommand\LW{g}{
\IfValueTF{#1}{L_{W_{#1}}}{L_{W}}}
%\def\LW#1{L_{W_{#1}}}
\def\JW#1{J_{W_{#1}}}

\def\floor#1{{\lfloor #1 \rfloor}}

\def\KSP{K}
\def\UU{\rU}
\def\UUC{\rU_\bC}
\def\tUUC{\widetilde{\rU}_\bC}
\def\OmegabfW{\Omega_{\bfW}}


\def\BB{\bB}


\def\PhiO#1{\vartheta\left(#1\right)}

\def\Phiv{\check{\Phi}}
\def\Phiv{\check{\Phi}}

\def\Phib{\bar{\Phi}}

\def\cKaod{\cK^{\mathrm{aod}}}

\DeclareMathOperator{\sspan}{span}


\def\sp{{\mathrm{sp}}}

\def\bfLz{\bfL_0}
\def\sOpe{\sO^\perp}
\def\sOpeR{\sO^\perp_\bR}
\def\sOR{\sO_\bR}

\def\ZX{\cZ_{X}}
\def\gdliftv{\vartheta}
\def\gdlift{\vartheta^{\mathrm{gen}}}
\def\bcOp{\overline{\cO'}}
\def\bsO{\overline{\sO}}
\def\bsOp{\overline{\sO'}}
\def\bfVpe{\bfV^\perp}
\def\bfEz{\bfE_0}
\def\bfVn{\bfV^-}
\def\bfEzp{\bfE'_0}

\def\totimes{\widehat{\otimes}}
\def\dotbfV{\dot{\bfV}}

\def\aod{\mathrm{aod}}
\def\unip{\mathrm{unip}}
\def\IC{\mathfrak{I}}

\def\PI#1{\Pi_{\cI_{#1}}}
\def\Piunip{\Pi^{\mathrm{unip}}}
\def\cf{\emph{cf.} }
\def\Groth{\mathrm{Groth}}
\def\Irr{\mathrm{Irr}}
\def\Irrsp{\mathrm{Irr}^{\mathrm{sp}}}

\def\edrc{\mathrm{DRC}^{\mathrm e}}
\def\drc{\mathrm{DRC}}
\def\LS{\mathrm{LS}}
\def\Unip{\mathrm{Unip}}


% Ytableau tweak
\makeatletter
\pgfkeys{/ytableau/options,
  noframe/.default = false,
  noframe/.is choice,
  noframe/true/.code = {%
    \global\let\vrule@YT=\vrule@none@YT
    \global\let\hrule@YT=\hrule@none@YT
  },
  noframe/false/.code = {%
    \global\let\vrule@YT=\vrule@normal@YT
    \global\let\hrule@YT=\hrule@normal@YT
  },
  noframe/on/.style = {noframe/true},
  noframe/off/.style = {noframe/false},
}
\makeatother


\def\wAV{\AV^{\mathrm{weak}}}
\def\ckG{\check{G}}
\def\ckGc{\check{G}_{\bC}}
\def\dBV{d_{\mathrm{BV}}}
\def\CP{\mathsf{CP}}
\def\YD{\mathsf{YD}}
\def\SYD{\mathsf{SYD}}
\def\DD{\nabla}

\def\lamck{\lambda_\ckcO}
\def\Lamck{[\lambda_\ckcO]}
\def\lamckb{\lambda_{\ckcO_{\mathrm b}}}
\def\lamckg{\lambda_{\ckcO_{\mathrm g}}}
\def\Wint#1{W_{[#1]}}
\def\CLam{\Coh_{\Lambda}}
\def\Cint#1{\Coh_{[#1]}}
\def\PP{\mathsf{PAP}}
\def\PAP{\mathsf{PAP}}
%\def\BOX#1{\mathrm{Box}(#1)}
\newcommand{\BOX}[1]{\mathrm{Box}(#1)}

\DeclareDocumentCommand{\bigtimes}{}{\mathop{\scalebox{1.7}{$\times$}}}
\providecommand\mapsfrom{\scalebox{-1}[1]{$\mapsto$}}

\def\ihh{{i_\fhh}}

\def\Gc{G_\bC}
\def\Gcad{G_\bC^{\text{ad}}}
\def\Gad{\Inn(\fgg)}

\def\hha{{}^a\fhh}
\def\ahh{\hha}
\def\aSR{{}^a\Sigma}
\def\aRp{{}^a\Delta^+}
\def\aX{{}^aX}
\def\aQ{{}^aQ}
\def\aP{{}^aP}
\def\aR{{}^aR}
\def\aRp{{}^aR^+}
\def\asRp{{}^a \Delta^+}
\def\Gfin{\cG(\Gc)}
\def\PiGfin{\Pi_{\mathrm{fin}}( \Gc )}
\def\PiGlfin{\Pi_{\Lambda_0}( \Gc )}
\def\adGfin{\cG_{\mathrm{ad}}(\Gc)}
\def\Ggk{\cG(\fgg,K)}
\def\WT#1{\Delta(#1)}
\def\WG{W(\Gc)}
\def\ch{\mathrm{ch}\,}
\def\Wlam{W_{[\lambda]}}
\def\aLam{a_{\Lambda}}
\def\WLam{W_{\Lam}}
\def\WLamck{W_{[\lambda_{\ckcO}]}}
\def\Wlamck{W_{\lamck}}
\def\Rlam{R_{[\lambda]}}
\def\RLam{R_\Lambda}
\def\RLamp{R_\Lambda^+}
\def\Rplam{R^+(\lambda)}
\def\Glfin{\cG_{\Lambda}(\Gc)}
\def\CL{{\sC}^{\scriptscriptstyle L}}
\def\CR{{\sC}^{\scriptscriptstyle R}}
\def\CLR{{\sC}^{\scriptscriptstyle LR}}
\def\LV{{}^{\scriptscriptstyle L}\sV}
\def\LC{{}^{\scriptscriptstyle L}\sC}
\def\RC{{}^{\scriptscriptstyle R}\sC}
\def\LRC{{}^{\scriptscriptstyle LR}\sC}
\def\ckLC{{}^{\scriptscriptstyle L}\check{\sC}}

\def\LV{{}^{\scriptscriptstyle L}\sV}
\def\ckLV{{}^{\scriptscriptstyle L}\check\sV}
\def\ckLC{{}^{\scriptscriptstyle L}\check\sC}

\def\tLV{{}^{\scriptscriptstyle L}\widetilde{\sV}}
\def\tLC{{}^{\scriptscriptstyle L}\widetilde{\sC}}

\def\brsgn{\breve{\sgn}}
\def\bsgn{\overline{\sgn}}

\def\Wb{W_{\mathrm b}}
\def\Wg{W_{\mathrm g}}


\def\nbb{n_{\mathrm b}}
\def\ngg{n_{\mathrm g}}
\def\tU{\widetilde{\rU}}

\newcommand{\cross}{\times}
\newcommand{\crossa}{\times^a}

\def\bVL{{\overline{\sV}}^{\scriptscriptstyle L}}
\def\bVR{{\overline{\sV}}^{\scriptscriptstyle R}}
\def\bVLR{{\overline{\sV}}^{\scriptscriptstyle LR}}
\def\VL{{\sV}^{\scriptscriptstyle L}}
\def\VR{{\sV}^{\scriptscriptstyle R}}
\def\VLR{{\sV}^{\scriptscriptstyle LR}}

\def\Con{\sfC}
\def\bCon{\overline{\sfC}}
\def\Re{\mathrm{Re}}
\def\Im{\mathrm{Im}}
\def\AND{\quad \text{and} \quad}
\def\Coh{\mathrm{Coh}}
\def\Cohlm{\Coh_{\Lambda}(\cM)}
\def\ev#1{{\mathrm{ev}_{#1}}}

\def\ppp{\times}
\def\mmm{\slash}


\def\cuprow{{\stackrel{r}{\sqcup}}}
\def\cupcol{{\stackrel{c}{\sqcup}}}

\def\Spr{\mathrm{Springer}}
\def\Prim{\mathrm{Prim}}



\def\imathp{\imath_{\wp}}
\def\jmathp{\jmath_{\wp}}

\def\imathwpp{\imath_{\wp'}}
\def\jmathwpp{\jmath_{\wp'}}
\def\cPpn{\cP'_\mathrm{naive}}
\def\cQpn{\cQ'_\mathrm{naive}}


\def\CQ{\overline{\sfA}}% Lusztig's canonical quotient
\def\CPP{\mathrm{PP}}
\def\CPPs{\mathrm{PP}_{\star}}
%\def\CPP{\mathfrak{P}}
%\def\CPPs{\mathfrak{P}_{\star}}


\def\ceil#1{\lceil #1 \rceil}
\def\symb#1#2{{\left(\substack{{#1}\\{#2}}\right)}}
\def\cboxs#1{\mbox{\scalebox{0.25}{\ytb{\ ,\vdots,\vdots,\ }}}_{#1}}

\def\hsgn{\widetilde{\mathrm{sgn}}}

\def\tPBP{\widetilde{\mathsf{PBP}}}
\def\PBPe{\mathsf{PBP}^{\mathrm{ext}}}
\def\PBPes{\mathsf{PBP}^{\mathrm{ext}}_{\star}}
\def\PBPsb{\mathsf{PBP}_{\star,b}}

\def\bev#1{\overline{\mathrm{ev}}_{#1}}

\def\Prim{\mathrm{Prim}}
% \def\leqL{\stackrel{L}{\leq}}
% \def\leqR{\stackrel{R}{\leq}}
% \def\leqLR{\stackrel{LR}{\leq}}

% \def\leqL{{\leq_L}}
% \def\leqR{{\leq_R}}
% \def\leqLR{{\leq_{LR}}}


\def\Dsp{\cD^{\text{sp}}}
\def\Csp{\sfC^{\text{sp}}}


\def\lneqL{\mathrel{\mathop{\lneq}\limits_{\scriptscriptstyle L}}}
\def\lneqR{\mathrel{\mathop{\lneq}\limits_{\scriptscriptstyle R}}}
\def\lneqLR{\mathrel{\mathop{\lneq}\limits_{\scriptscriptstyle LR}}}

\def\leqL{\mathrel{\mathop{\leq}\limits_{\scriptscriptstyle L}}}
\def\leqR{\mathrel{\mathop{\leq}\limits_{\scriptscriptstyle R}}}
\def\leqLR{\mathrel{\mathop{\leq}\limits_{\scriptscriptstyle LR}}}


\def\approxL{\mathrel{\mathop{\approx}\limits_{\scriptscriptstyle L}}}
\def\approxR{\mathrel{\mathop{\approx}\limits_{\scriptscriptstyle R}}}
\def\approxLR{\mathrel{\mathop{\approx}\limits_{\scriptscriptstyle LR}}}


\def\dphi{\rdd \phi}
\def\CPH{C(H)}
\def\whCPH{\widehat{C(H)}}

\def\Greg{G_{\text{reg}}}
\def\Hnreg{H^-_{\text{reg}}}
\def\Hireg{H_{i,\text{reg}}}
\def\Hnireg{H^-_{i,\text{reg}}}


\def\tsgn{\widetilde{\sgn}}
\def\PBP{\mathsf{PBP}}

\def\ckstar{{\check \star}}
\def\ckfgg{{\check \fgg}}

\def\Inn{\mathrm{Inn}}

\providecommand{\nsubset}{\not\subset}

\def\cuprow{{\,\stackrel{r}{\sqcup}\,}}
\def\cupcol{{\,\stackrel{c}{\sqcup}\,}}

\def\ckcOp{\ckcO'}
\def\ckcOpp{\ckcO''}

\def\ckcOb{\ckcO_{\mathrm b}}
\def\ckcOpb{\ckcO'_{\mathrm b}}
\def\cOpb{\cO'_{\mathrm b}}
\def\ckcOg{\ckcO_{\mathrm g}}

\def\nng{n_{\mathrm g}}
\def\nnb{n_{\mathrm b}}
\def\Gb{G_{\mathrm b}}
\def\Gpb{G'_{\mathrm b}}
\def\Pb{P_{\mathrm b}}
\def\Gg{G_{\mathrm g}}
\def\ckGb{\ckG_{\mathrm b}}
\def\ckGg{\ckG_{\mathrm g}}

\def\bcOb{\overline{\cO_{\mathrm b}}}
\def\bcOg{\overline{\cO_{\mathrm g}}}

\def\lamb{\lambda_{\mathrm b}}
\def\lamg{\lambda_{\mathrm g}}


\def\tPBP{\widetilde{\mathsf{PBP}}}
\def\PBPs{\mathsf{PBP}_{\star}}
\def\PBPe{\mathsf{PBP}^{\mathrm{ext}}}
\def\PBPes{\mathsf{PBP}^{\mathrm{ext}}_{\star}}
\def\PBPsb{\mathsf{PBP}_{\star,b}}

\def\PBPop#1#2#3#4{\PBP_{#1}^{#2}(#3,#4)}
\newcommand{\PBPOP}[1][]{\PBPop{\star}{#1}{\ckcO}{\wp}}
% \def\PBPdOP{\PBPop{\star}{\mathrm{d}}{\ckcO}{\wp}}
% \def\PBPrcOP{\PBPop{\star}{\mathrm{rc}}{\ckcO}{\wp}}
% \def\PBPsOP{\PBPop{\star}{\mathrm{s}}{\ckcO}{\wp}}
\def\PBPOPp{\PBPop{\star'}{}{\ckcO'}{\wp'}}
%\def\PBPOPpp{\PBPop{\star}{}{\ckcO''}{\wp''}}
\newcommand{\PBPOPpp}[1][]{\PBPop{\star}{#1}{\ckcO''}{\wp''}}

% \def\PBPdOPpp{\PBPop{\star}{\mathrm{d}}{\ckcO''}{\wp}}
% \def\PBPrcOPpp{\PBPop{\star}{\mathrm{rc}}{\ckcO''}{\wp}}
% \def\PBPsOPpp{\PBPop{\star}{\mathrm{s}}{\ckcO''}{\wp}}

\def\PBPGOP{\PBPop{G}{}{\ckcO}{\wp}}
\def\PBPGOPp{\PBPop{G'}{}{\ckcO'}{\wp'}}
\def\PBPGOPpp{\PBPop{G''}{}{\ckcO''}{\wp''}}

\def\yrow#1{\left[#1\right]_{\mathrm{row}}}
\def\ycol#1{\left[#1\right]_{\mathrm{col}}}

%\newcommand{\DDn}{\DD_{\mathrm{naive}}}

\newcommand{\Lam}{{[\lambda]}}

\newcommand{\Rg}{\cR(\fgg)}
\newcommand{\Grt}{\cK}
\newcommand{\nckG}{n_{\ckG}}
%\newcommand{\nb}{n_{\mathrm b}}
%\newcommand{\ng}{n_{\mathrm g}}

\usepackage{xr}
\usepackage{subfiles} % Best loaded last in the preamble



\begin{document}


\title[]{Special unipotent representations of real classical groups: counting}

\author [D. Barbasch] {Dan M. Barbasch}
\address{the Department of Mathematics\\
  310 Malott Hall, Cornell University, Ithaca, New York 14853 }
\email{dmb14@cornell.edu}

\author [J.-J. Ma] {Jia-jun Ma}
\address{School of Mathematical Sciences\\
  Xiamen University\\
  Xiamen, China} \email{hoxide@xmu.edu.cn}

\author [B. Sun] {Binyong Sun}
% MCM, HCMS, HLM, CEMS, UCAS,
\address{Institute for Advanced Study in Mathematics\\
 Zhejiang University\\
  Hangzhou, China} \email{sunbinyong@zju.edu.cn}
  %310058

%\address{Academy of Mathematics and Systems Science\\
  %Chinese Academy of Sciences\\
  %Beijing, 100190, China} \email{sun@math.ac.cn}

\author [C.-B. Zhu] {Chen-Bo Zhu}
\address{Department of Mathematics\\
  National University of Singapore\\
  10 Lower Kent Ridge Road, Singapore 119076} \email{matzhucb@nus.edu.sg}




\subjclass[2010]{22E46, 22E47} \keywords{Special unipotent representation, associated variety, coherent continuation, primitive ideal, cell, classical group}

% \thanks{Supported by NSFC Grant 11222101}

\begin{abstract} Let $G$ be a real reductive group in the Harish-Chandra class. We derive some consequences of theory of coherent continuation representations, primitive ideals and cells to the counting of irreducible representations of $G$ with a given infinitesimal character and a given bound in the complex associated variety. When $G$ is a real classical group (including the real metaplectic group), we give a precise count for the number of special unipotent representations of $G$ attached to $\check \CO$, in the sense of Barbasch and Vogan. Here $\check \CO$ is a nilpotent adjoint orbit in the Langlands dual of $G$ (or the metaplectic dual of $G$ when $G$ is a real metaplectic group).
\end{abstract}




\maketitle



\tableofcontents



\section{Introduction and the main results}\label{sec:intro}

Let $G$ be a real reductive group in the Harish-Chandra class (which may be
linear or non-linear). Write $\fgg$ for the complexified Lie algebra of $G$ and
let $\hha$ denote the universal Cartan subalgebra of $\g$ (also called the abstract Cartan subalgbra in \cite{V4}).
Let $\lambda \in \hha^*$ (a superscript $*$ indicates the dual space). By Harish-Chandra isomorphism, it
determines an algebraic character $\chi_\lambda: \CZ(\g)\rightarrow \C$. Here
$\CZ(\g)$ denotes the center of the universal enveloping algebra
$\mathcal U(\g)$. Denote by $\Irr(G)$ the set of isomorphism classes of
irreducible Casselman-Wallach representations of $G$ (see \cite[Chapter 11]{Wa2}), and by $\Irr_\lambda(G)$
its subset consisting of the representations with infinitesimal character
$\chi_\lambda$ (or simply $\lambda$). The latter set has finite cardinality.


Let $\Nil(\g^*)$ denote the set of nilpotent elements in $\g^*$. It has only
finitely many orbits under the coadjoint action of the inner automorphism group
$\mathrm{Inn}(\g)$ of $\g$. Let $\sfS$ be an $\mathrm{Inn}(\g)$-stable Zariski
closed subset of $\mathrm{Nil}(\g^*)$. Put
\[
  \Irr_{\lambda,\sfS}(G):=\Set{\pi \in \Irr_{\lambda}(G)| \text{$\AVC(\pi)\subset \sfS$} }.
\]
Here $\AVC(\pi)$ denotes the complex associated variety of $\pi$, namely the
associated variety of the annihilate ideal of $\pi$. It is an
$\mathrm{Inn}(\g)$-stable Zariski closed subset of $\mathrm{Nil}(\g^*)$. An
interesting problem of representation theory is to count the finite set $\Irr_{\lambda,\sfS}(G)$. The coherent continuation
representation (of the integral Weyl group) provides a powerful tool for this problem. The first goal of the paper is to give a systematic treatment of the problem in this general set-up, by building on earlier ideas of several authors including Vogan \cite{Vg}, Joseph \cite{J1,J2},  King \cite{King}, Barbasch-Vogan \cite{BVUni}, Casian \cite{Cas}, as well as Soergel \cite{Soergel}.

\subsection{The coherent continuation representation}\label{sec11}


Write $\mathrm{Rep}(G)$ for the category of Casselman-Wallach representations of $G$, and write $\CK(G)$ for the
Grothendieck group of this category.  Throughout this article we take $\C$ as the coefficient ring to define  Grothendieck groups.
When no confusion is possible, for every object $O$ in an abelian category, we  still use the same symbol to indicate the Grothendieck group element represented by the object $O$.


Write $\mathrm{Rep}_{\lambda, \sfS}(G)$ for the full subcategory of $\mathrm{Rep}(G)$ whose objects are the representations that have
generalized infinitesimal character $\lambda$ and whose complex associated
variety is contained in $\sfS$. Write $\CK_{\lambda,\sfS}(G)$ for the
Grothendieck group of this category. Then
\[
  \sharp (\Irr_{\lambda,\sfS}(G))=\dim \CK_{\lambda,\sfS}(G)\qquad(\sharp\textrm{
    indicates the cardinality of a finite set}).
\]
We also have that
\[
  \CK_\sfS(G)=\bigoplus_{\mu\in W\backslash \hha^*} \CK_{\mu,\sfS}(G)\qquad (W\textrm{
    denotes the Weyl group}),
\]
where $\CK_{\sfS}(G)$ is the Grothendieck group of $\mathrm{Rep}_\sfS(G)$, and latter is the
category of Casselman-Wallach representations of $G$ whose complex associated
variety is contained in $\sfS$.


Let $\Rg$ be the Grothendieck group of the category of finite-dimensional algebraic
representations of $\mathrm{Inn}(\fgg)$. It is
 a commutative $\bC$-algebra under the tensor
product of representations.
Write \[
\Delta\subset Q \quad (\subset \hha^*)
\] for the root system and the root lattice of
$\fgg$, respectively.
%and identified with $\bC[Q/W]$.
By pulling back through the adjoint representation
$G\rightarrow \mathrm{Inn}(\g)$, every algebraic representation of $\mathrm{Inn}(\g)$ is viewed as a representation of $G$.
Under the tensor product of representations, $\CK_\sfS(G)$ is naturally a $\Rg$-module.


Put
\[
\Lam:=\lambda+Q\subset \hha^*,
\]
 and write $W_\Lam$
for its stabilizer in $W$. Then $W_\Lam$ equals the Weyl group of the root
system  (\cite[Section 1.3]{Jan})
\[
\Delta_\Lam:=  \{\alpha \in \Delta\mid \langle \lambda, \alpha^\vee\rangle \in \Z\}\qquad (\alpha^\vee \textrm{ denotes the coroot corresponding to $\alpha$}).
\]
We will often refer to $W_\Lam$ as the integral Weyl group.

\begin{defn}\label{defcoh}
  Let $\CK$ be a $\mathcal R(\g)$-module equipped with a family
  $\{\CK_\mu\}_{\mu\in \Lam}$ of subspaces such that $\CK_{w \cdot \mu}=\CK_\mu$
  for all $w\in W_\Lam$ and $\mu\in \Lam$. A $\CK$-valued coherent family on
  $\Lam$ is a map
  \[
    \Phi: \Lam\rightarrow \CK%, \qquad \mu\mapsto \Phi_\mu
  \]
  satisfying the following two conditions:
  \begin{itemize}
    \item for all $\mu\in \Lam$, $\Phi(\mu)\in \CK_\mu$;
    \item for all finite-dimensional algebraic representations $F$ of $\mathrm{Inn}(\g)$
          and all $\mu\in \Lam$,
          \[
          F \cdot (\Phi(\mu)) = \sum_{\nu} \Phi(\mu+\nu),
          \]
          where $\nu$ runs over all weights of $F$, counted with multiplicities.%  and $F$ is viewed as an element of $\mathcal R(\g)$.
  \end{itemize}
\end{defn}


In the notation of \Cref{defcoh}, let $\Coh_{\Lam}(\CK)$ denote the
vector space of all $\mathcal K$-valued coherent families on $\Lam$. It is a
representation of $W_{\Lam}$ under the action
\[
  (w \cdot \Phi)(\mu) = \Phi(w^{-1}\cdot \mu), \qquad \textrm{for all
  }\ w\in W_\Lam, \ \mu\in \Lam.
\]
This is called a coherent continuation representation. When specifying a coherent continuation representation $\Coh_{\Lam}(\CK)$, we will often explicitly describe $\CK$ as a Grothendieck group, while the $\mathcal R(\g)$-module structure and the $W_\Lam$-invariant family  $\{\CK_\mu\}_{\mu\in \Lam}$ are the ones which are clear from the context.
For example, $\CK_\sfS(G)$ is a $\Rg$-module as described previously, and it is equipped with the family $\{\CK_{\mu, \sfS}(G)\}_{\mu\in \Lam}$ of subspaces. We thus have the coherent continuation representation $\Coh_{\Lam}(\CK_\sfS(G))$ of $W_\Lam$.


%To shorten the notation, put
%\[
%  \Coh_{\Lam,\sfS}(G):=\Coh_{\Lam}(\CK_\sfS(G)).
%\]


\subsection{Counting irreducible representations with a bounded complex
  associated variety} %\label{sec12}

Denote by $W_\lambda$ the stabilizer of
$\lambda$ in $W$. Then $W_\lambda\subset W_\Lam$. Write $1_{W_\lambda}$ for the
trivial representation of $W_{\lambda}$.

Our starting point is the following theorem of Vogan. We will provide a proof due to lack of a convenient reference.
\begin{thm}[Vogan]\label{count1}
  The equality
  \[
    \sharp(\Irr_{\lambda,\sfS}(G)) = [1_{W_{\lambda}}:\Coh_{\Lam}(\CK_\sfS(G))]
  \]
  holds.
  % \[
  %   \dim {\barmu} = \dim (\cohm)_{W_\mu} = [\cohm, 1_{W_\mu}].
  % \]
\end{thm}
Here and henceforth, $[\ : \ ]$ indicates the multiplicity of the first
(irreducible) representation in the second one. Theorem \ref{count1} implies
that
\[
%\begin{equation}\label{countlg}
  \sharp(\Irr_{\lambda,\sfS}(G)) = \sum_{\sigma\in \Irr(W_\Lam)} [1_{W_{\lambda}}: \sigma]\cdot [\sigma: \Coh_{\Lam}(\CK_\sfS(G))].
%\end{equation}
\]
Thus it suffices to understand the multiplicity $ [\sigma: \Coh_{\Lam,\sfS}(G)]$
for every $\sigma\in \Irr(W_\Lam)$.

Let $\sigma\in \Irr(W_\Lam)$. Define the
nilpotent orbit
\[
  \CO_\sigma:=\mathrm{Springer}^{-1}
  (j_{W_\Lam}^W \sigma_0)\subset \mathrm{Nil}(\g^*),
  \]
where $\sigma_0$ denotes the special irreducible representation of $W_\Lam$ that
lies in the same double cell as $\sigma$, $j_{W_\Lam}^W \sigma_0\in \Irr(W)$ denotes
the $j$-induction of $\sigma _0$, and  ``Springer"  indicates the Springer correspondence. See \cite[Chapter 11]{Carter} or Section \ref{secGoldie} for the notion of $j$-induction, and Section \ref{seccell} on special representations and double cells.


Let
\be\label{sfc}
  \Irr_\sfS(W_\Lam):= \Set{\sigma\in \Irr(\WLam)| \cO_{\sigma}\subset \sfS}.
\ee

\begin{thm}\label{count2}
  Suppose that $\sigma\in \Irr(W_\Lam)\setminus \Irr_\sfS(W_\Lam)$. Then
  %$\CO_\sigma\nsubset \sfS$.
  \[
    [\sigma:\Coh_{\Lam}(\CK_\sfS(G))]=0.
  \]
Consequently we have
\begin{equation}\label{leq002}
  \sharp(\Irr_{\lambda,\sfS}(G)) = \sum_{\sigma \in \Irr_\sfS(W_\Lam)} [1_{W_{\lambda}}: \sigma]\cdot [\sigma:\Coh_{\Lam}(\CK_\sfS(G))].
  \end{equation}
  % \[
  %   \dim {\barmu} = \dim (\cohm)_{W_\mu} = [\cohm, 1_{W_\mu}].
  % \]
\end{thm}


Theorem \ref{count2} clearly implies that
\begin{equation}\label{leq2}  \sharp(\Irr_{\lambda,\sfS}(G)) \leq  \sum_{\sigma \in \Irr_\sfS(W_\Lam)} [1_{W_{\lambda}}: \sigma]\cdot [\sigma:\Coh_{\Lam}(\CK(G))].
\end{equation}

Recall the notion of a Harish-Chandra cell representation in $\Coh_{\Lam}(\CK(G))$ (which is a subquotient of $\Coh_{\Lam}(\CK(G))$). See \cite{V4}*{Section 14} or Section \ref{seccell}.

%Also recall the notion of Lustig double cells in $ \Irr(W_{[\lambda]}$ (see Section \ref{seccell}).
% For every Harish-Chandra cell $C$  in $\Coh_{\Lam}(\CK(G))$, write $\CV(C)$ for the  Harish-Chandra cell representation attached to $C$, which is a subquotient representation of $\Coh_{\Lam}(\CK(G))$.

 \begin{thm}\label{counteq}
   % Under the notation of \Cref{lem:lcell.BV}, we have
   Assume that for every Harish-Chandra cell representation $V$ in $\Coh_{\Lam}(\CK(G))$, the set $\{\sigma\in \Irr(W_{[\lambda]}) \,|\, [\sigma: V]\neq 0\}$ is contained in a single double cell. Then
  \begin{equation*}%\label{boundc}
    \sharp(\Irr_{\lambda,\sfS}(G)) = \sum_{\sigma \in \Irr_\sfS(W_\Lam)} [1_{W_{\lambda}}: \sigma]\cdot [\sigma:\Coh_{\Lam}(\CK(G))].
  \end{equation*}
    \end{thm}

% the equality always holds. Combining Theorem \ref{count1}, Theorem
% \ref{count2} and \eqref{leq1}, we conclude that
% \begin{equation}\label{leq2}
%   \sharp(\Irr_{\lambda,\sfS}(G)) \leq \sum_{\sigma\in \Irr(W_\Lam), \CO_\sigma\subset \sfS} [1_{W_{\lambda}}: \sigma]\cdot [\sigma: \Coh_{\Lam}(G)].
% \end{equation}





\subsection{Counting irreducible representations annihilated by a maximal primitive ideal}\label{sec13}
Write $I_\lambda$ for the maximal ideal of $\mathcal U(\g)$ with infinitesimal
character $\lambda$. Its associated variety equals the Zariski closure
$\overline{\CO_\lambda}$ of an $\mathrm{Inn}(\g)$-orbit
$\CO_\lambda\subset\mathrm{Nil}(\g^*) $. Note that an irreducible
Casselman-Wallach representation of $G$ lies in
$\Irr_{\lambda,\overline{\CO_\lambda}}(G)$ if and only if it is annihilated by
$I_\lambda$.


Let
\begin{equation}\label{eq:LC} \LC_{\lambda}:= \Set{\sigma\in \Irr(W_\Lam) | \sigma \text{ occurs in $(J_{W_{\lambda}}^{W_{\Lam}} \sgn )\otimes \sgn$}}, \end{equation}
called the Lusztig left cell attached to $\lambda$. Here $J_{W_{\lambda}}^{W_{\Lam}} $ indicates the $J$-induction (see \cite[Chapter 12]{Carter}), and $\sgn$
denotes the sign character (of an appropriate Weyl group).

%Let $\LC_{\lambda}\subset \Irr(W_\Lam)$ be the subset consisting of all the
%irreducible representations that occur in
%\[
%  (J_{W_{\lambda}}^{W_{\Lam}} \sgn )\otimes \sgn,
%\]
%where $J_{W_{\lambda}}^{W_{\Lam}} $ indicates the $J$-induction (see \cite[Chapter 12]{Carter}), and $\sgn$
%denotes the sign character (of an appropriate Weyl group).



 \begin{prop}[{\cite{BVUni}*{(5.26), Proposition~5.28}}]\label{lem:lcell.BV0}
  The following equality of sets holds:
   \[
     \LC_{\lambda} = \Set{\sigma\in \Irr_{\overline{\CO_\lambda}}(W_\Lam)\, |\,   [1_{W_{\lambda}}:\sigma]\neq 0}.
   \]
   Moreover, $[1_{W_{\lambda}}:\sigma]=1$ when
   $\sigma\in \LC_\lambda$.
 \end{prop}


In view of \Cref{lem:lcell.BV0}, Theorems \ref{count2} and \ref{counteq} have the following consequence.
%nonumber  Combining \eqref{leq2}, \eqref{leq111} and  Propositions \ref{lem:lcell.BV0}, we obtain the following inequality.

 \begin{cor}
   \label{cor:bound} The equality
   \[
     \sharp(\Irr_{\lambda,\overline{\CO_\lambda}}(G)) =\sum_{\sigma\in \LC_\lambda} [\sigma: \Coh_{\Lam}(\CK_{\overline{\CO_\lambda}}(G))]
   \]
   holds.
   % Under the notation of \Cref{lem:lcell.BV}, we have
   Consequently,
   \begin{equation}\label{boundc}
     \sharp(\Irr_{\lambda,\overline{\CO_\lambda}}(G)) \leq \sum_{\sigma\in \LC_\lambda} [\sigma: \Coh_{\Lam}(\CK(G))].
   \end{equation}
   If for every Harish-Chandra cell representation $V$  in $\Coh_{\Lam}(\CK(G))$, the set $\{\sigma\in \Irr(W_{[\lambda]}\,|\, [\sigma: V]\neq 0\}$ is contained in a single double cell, then the equality holds in
   \eqref{boundc}.

 \end{cor}


 \subsection{Special unipotent representations of real classical groups}
 \label{sec:defunip}

 We are particularly interested in counting special unipotent representations of
 real classical groups.

 Let $\star$ be one of the 10 symbols
 \[
   \textrm{ $A^\R$, $A^\bH$, $A$, $\widetilde A$,  $B$, $D$, $C$, $\wtC$,
     $D^*$, $C^*$. }
 \]
 Suppose that $G$ is a classical Lie group of type $\star$, namely $G$
 respectively equals one of the following Lie groups:
 \[
   \begin{array}{c}
     \GL_n(\R),  \  \GL_n(\bH),\  \oU(p,q),\  \widetilde \oU(p,q), \smallskip\\
     \SO(p,q)\ (p+q\, \textrm{ is odd}),  \  \SO(p,q)\  (p+q\, \textrm{ is even}),\smallskip\\
     \Sp_{2n}(\R),  \   \widetilde \Sp_{2n}(\R), \  \oO^*(2n), \  \Sp(p,q),\qquad (n, p, q\geq 0).
   \end{array}
 \]
 Here $\wtSp_{2n}(\R)$ denotes the metaplectic double cover of the symplectic
 group $\Sp_{2n}(\R)$ that does not split unless $n=0$, and $\tU(p,q)$ is the double cover of $\rU(p,q)$ defined by a square root of the determinant character.

 Define the Langlands dual $\ckG$ of $G$ to be respectively the complex group
 \[
   \begin{array}{c}
     \GL_n(\C),  \  \GL_{2n}(\C),\  \GL_{p+q}(\C), \  \GL_{p+q}(\C),\smallskip\\
     \Sp_{p+q-1}(\C)\ (p+q\, \textrm{ is odd}),  \  \SO_{p+q}(\C)\  (p+q\, \textrm{ is even}),\smallskip\\
     \SO_{2n+1}(\C),  \ \Sp_{2n}(\C), \  \SO_{2n}(\C), \ \textrm{or } \   \SO_{2p+2q+1}(\C).
   \end{array}
 \]
 Write $\check \g$ for the Lie algebra of $\ckG$, and let $\check \CO\subset\mathrm{Nil}(\check \g)$ be a nilpotent $\ckG$-orbit.

  Let $\lambda_{\ckcO}\in \check \g$ be half of the neutral element in any
 $\fsl_{2}$ triple attached to $\ckcO$, as in \cite[Section 5]{BVUni}. It is a semisimple element and is uniquely determined up to conjugation by $\ckG$.
Using the identification
 \be\label{sse}
   \ckG\backslash  \{\textrm{semisimple element in $\check \g$}\}=W\backslash \hha^*,
 \ee
we view $\lambda_{\check \CO}$ as an element of $ W\backslash \hha^*$, and write $I_{\check \CO}:=I_{\star, \check \CO}$ for the maximal ideal of $\mathcal U(\g)$ with infinitesimal character $\lambda_{\check \CO}$. We remark that in the metaplectic case, namely $G=\widetilde \Sp_{2n}(\R)$, both $\hha$ and $\hha^*$ are identified with $\C^n$ in the usual way and hence \eqref{sse} still holds.

 %By using Harish-Chandra isomorphism, we view $\lambda_{\check \CO}$ as a character $\lambda_{\check \CO}: \mathcal Z(\g)\rightarrow \C$.


Following Barbasch-Vogan \cite{BVUni}, define the set of the special unipotent representations of $G$
 attached to $\ckcO$ by
\[
 %\begin{equation}\label{eq:defuni}
   \begin{split}
     \Unip_{\ckcO}(G):=&  \Unip_{\star, \ckcO}(G) \\
     :=& \begin{cases}
       % \{\pi\in \Irr(G)\mid \pi \textrm{ is annihilated by $ I_{\check \CO}$ or $I'_{\check \CO}$}\}, & \text{if } \star \in \set{D, D^\C, D^*};\\
       \{\pi\in \Irr(G)\mid \pi \textrm{ is genuine  and annihilated by } I_{\check \CO}\}, & \text{if } \star\in \{\widetilde A, \widetilde C\};\\
       \{\pi\in \Irr(G)\mid \pi \textrm{ is annihilated by } I_{\check \CO}\}, & otherwise.\\
     \end{cases}
   \end{split}
% \end{equation}
\]
 Here ``genuine" means that the central subgroup $\{\pm 1\}$ of $G$, which is the kernel of the covering homomorphism $\widetilde \oU(p,q)\rightarrow  \oU(p,q)$ or $\widetilde \Sp_{2n}(\R)\rightarrow \Sp_{2n}(\R)$, acts on $\pi$ through the nontrivial character.
 % of $G$ representation $\pi$ of $\widetilde \oU(p,q)$ or $\widetilde \Sp_{2n}(\R)$ does not descend to $\oU(p,q)$ or $\Sp_{2n}(\R)$, respectively.

The main goal of the paper is to count the set $\Unip_{\check \CO}(G)$. In view of \Cref{cor:bound}, we will explicitly determine both $\LC_\lambda$ and $\Coh_{\Lam}(\CK(G))$ (for $\lambda =\lambda_{\ckcO}$) and will express the sum $\sum_{\sigma\in \LC_\lambda} [\sigma: \Coh_{\Lam}(\CK(G))]$ as the count of certain combinatorial constructs. These combinatorial constructs provide the key linkage with the authors' second paper \cite{BMSZ2}, whose main goal is to construct all the representations in $\Unip_{\check \CO}(G)$.


\subsection{The cases of general linear groups and unitary groups}


  For a Young diagram $\imath$, write
 \[
   \mathbf r_1(\imath)\geq \mathbf r_2(\imath)\geq \mathbf r_3(\imath)\geq \cdots
 \]
 for its row lengths, and similarly, write
 \[
   \mathbf c_1(\imath)\geq \mathbf c_2(\imath)\geq \mathbf c_3(\imath)\geq \cdots
 \]
 for its column lengths. Denote by
 $\abs{\imath}:=\sum_{i=1}^\infty \mathbf r_i(\imath)$ the total size of
 $\imath$.


When no confusion is possible, we still use $\check \CO$ to denote the Young diagram attached to the nilpotent orbit $\check \CO$. Note that the Young diagram determines the nilpotent orbit unless $\check G=\SO_{4n}(\C)$ ($n\geq 1$) and all the row lengths are even.

Let $\bN^+$ denote the set of positive integers. For any Young diagram $\imath$, we introduce the set $\mathrm{Box}(\imath)$ of
boxes of $\imath$ as the following subset of $\bN^+\times \bN^+$:
\[
% \begin{equation}\label{eq:BOX}
  \mathrm{Box}(\imath):=\Set{(i,j)\in\bN^+\times \bN^+| j\leq \bfrr_i(\imath)}.
%\end{equation}
\]

% We will also call a subset of $\bN^+\times \bN^+$ of the form \eqref{eq:BOX} a
% Young diagram.

% We say that a Young diagram $\imath'$ is contained in $\imath$ (and write
% $\imath'\subset \imath$) if
% \[
%   \mathbf r_i(\imath')\leq \mathbf r_i(\imath)\qquad \textrm{for all
% } i=1,2, 3, \cdots.
% \]
% When this is the case, $\mathrm{Box}(\imath')$ is viewed as a subset of
% $\mathrm{Box}(\imath)$ concentrating on the upper-left corner. We say that a
% subset of $\mathrm{Box}(\imath)$ is a Young subdiagram if it equals
% $\mathrm{Box}(\imath')$ for a Young diagram $\imath'\subset \imath$. In this
% case, we call $\imath'$ the Young diagram corresponding to this Young
% subdiagram.

\renewcommand{\CP}{\mathcal{P}} We also introduce five symbols $\bullet$, $s$,
$r$, $c$ and $d$, and make the following definitions.
\begin{defn}
  A painting on a Young diagram $\imath$ is a map
  \[
    \mathcal P: \mathrm{Box}(\imath) \rightarrow \{\bullet, s, r, c, d \}
  \]
  with the following properties:
  \begin{itemize}
    \item $\mathcal P^{-1}(S)$ is the set of boxes of a Young diagram when
          $S=\{\bullet\}, \{\bullet, s \}, \{\bullet, s, r\}$ or
          $\{\bullet, s, r, c \} $;
    \item when $S=\{s\}$ or $ \{r\}$, every row of $\imath$ has at most one box
          in $\CP^{-1}(S)$;
    \item when $S=\{c\}$ or $ \{d \}$, every column of $\imath$ has at most one
          box in $\CP^{-1}(S)$.
  \end{itemize}
A painted Young diagram is a pair $(\imath, \CP)$ consisting of a Young diagram $\imath$ and a painting $\CP$ on $\imath$.
\end{defn}



\begin{defn}\label{defpbp0}
  Suppose that $\star\in \{A^\R, A^\bH, A, \widetilde A\}$. A painting $\CP$ on a Young diagram
  $\imath$ has type $\star$ if
  \begin{itemize}
    \item the image of $\CP$ is contained in
          \[
          \left\{
          \begin{array}{ll}
            \{\bullet, c, d\}, &\hbox{if $\star=A^\R$}; \smallskip\\
            \{\bullet\}, &\hbox{if $\star=A^\bH$}; \smallskip\\
            \{\bullet, s, r\}, &\hbox{if $\star\in \{A, \widetilde A\}$},            \end{array}
        \right.
          \]
    \item if $\star\in \{A^\R, A^\bH\}$, then $\CP^{-1}(\bullet)$ has even number of
          boxes in every column of $\imath$,
    \item if $\star\in \{A, \widetilde A\}$, then $\CP^{-1}(\bullet)$ has even number of boxes in
          every row of $\imath$.
  \end{itemize}
  Denote by $\PAP_\star(\imath)$ the set of paintings on $\imath^{t}$ that has type $\star$, where $\imath^{t}$
  is the transpose of $\imath$.
   \end{defn}

%Note that in the definition of $\PAP_\star(\imath^{t})$, we have incorporated the transpose map in order to reconcile with the Barbasch-Vogan duality.

The middle letter $A$ in $\PAP$ refers to the common $A$ in $\{A^\R, A^\bH, A, \widetilde A\}$.

Special unipotent representations of general linear groups are
well-understood (see \cite{V.GL}*{Page 450}). In particular, we have the following counting result for general linear groups.

\begin{thm}\label{GLcase}
 Suppose that  $\star\in\{A^\R, A^\bH\}$. Then
  \[
    \sharp(\Unip_{\check \CO}(G))=        \sharp(\PAP_\star(\check \CO)).
    \]

\end{thm}
\begin{remark}
  If $\star=A^\R$, then
  \[
    \sharp(\PAP_\star(\check \CO))=\prod_{i\in \bN^+} (1+\textrm{the
      number of rows of length $i$ in $\check \CO$}).
  \]
  If $\star=A^\bH$, then
  \[
    \sharp(\PAP_\star(\check \CO))= \left\{
      \begin{array}{ll}
        1, &\hbox{if all row lengths of $\check \CO$ are even}; \smallskip\\
        0, &\hbox{otherwise}.  \end{array}
    \right.
  \]

\end{remark}

Now suppose that $\imath$ is a Young diagram and $\CP$ is a painting on $\imath$
that has type $A$ or $\widetilde A$. Define the signature of $\CP$ to be the pair
\begin{equation}\label{eq:signature}
    (p_\CP, q_\cP): = \left (\frac{\sharp(\cP^{-1}(\bullet))}{2}+\sharp(\cP^{-1}(r)),\,
    \frac{ \sharp(\cP^{-1}(\bullet))}{2}+\sharp(\cP^{-1}(s))\right).
\end{equation}
\trivial[h]{ The first equation is the true definition of signature. The second
  one is an easy consequence of the definition of $\AC_\cP$. }

\begin{eg}
  Suppose
  that \[ \check \CO=\ytb{\ \ \ \ \ , \ \ \ , \ , \ , \ }\quad \textrm{and}\quad \CP=\ytb{\bullet\bullet\bullet\bullet r,\bullet\bullet , sr,s,r}\in \mathrm{PAP}_{A}(\check \CO) .
  \]
  Then $(p_\CP, q_\cP)=(6,5)$.

\end{eg}

% Given two Young diagrams $\imath$ and $\jmath$, write $\imath\cuprow \jmath$ for
% the Young diagram whose multiset of nonzero row lengths equals the union of
% those of $\imath$ and $\jmath$. Also write $2\imath =\imath\cuprow \imath$.


Given two Young diagrams $\imath$ and $\jmath$, write $\imath\cuprow \jmath$ for
the Young diagram whose multiset of nonzero row lengths equals the union of
those of $\imath$ and $\jmath$. Also write $2\imath =\imath\cuprow \imath$.
Similarly, we write $\imath\cupcol \jmath$ for
the Young diagram whose multiset of nonzero column lengths equals the union of
those of $\imath$ and $\jmath$.


For unitary groups, we have the following counting result.
\begin{thm}
  Suppose that $\star=A$ or $\widetilde A$ so that $G=\oU(p,q)$ or $\widetilde \oU(p,q)$, respectively. Assume that there is a decomposition
  \[
    \ckcO=\ckcOg \cuprow 2\ckcOpb
  \]
  with the following property:
  \begin{itemize}
  \item if $\star=A$, then all nonzero row lengths of $\ckcOg$ have the same parity as $p+q$,
  and all nonzero row lengths of $\ckcOpb$ have different parity as $p+q$;
  \item if $\star=\widetilde A$, then all nonzero row lengths of $\ckcOpb$ have the same parity as $p+q$,
  and all nonzero row lengths of $\ckcOg$ have different parity as $p+q$.
  \end{itemize}
  Then
  \[
    \sharp(\Unip_{\ckcO}(G))= \sharp \set{\CP\in \mathrm{PAP}_\star(\ckcOg)|(p_\CP+\abs{\ckcOpb}, q_\CP+\abs{\ckcOpb})=(p,q)}.
  \]
  If there is no such decomposition, then $\sharp(\Unip_{\check \CO}(G))=0$.

\end{thm}

In particular, when $\star=\widetilde A$ and $p+q$ is odd, the set $\Unip_{\check \CO}(\widetilde \oU(p,q))$ is empty.

\subsection{Orthogonal and symplectic groups: reduction to good parity}

Now we assume that
$\star\in \Set{ B, D, C, \wtC, D^*, C^*}$.
Then there is a unique decomposition
\[
  \ckcO=\ckcOg \cuprow 2\ckcOpb
\]
such that $\ckcOg$ has $\star$-good parity in the sense that all its nonzero row
lengths are
\[
  \left\{
    \begin{array}{ll}
      \textrm{even}, &\hbox{if $\star\in \set{B, \widetilde C}$}; \smallskip\\
      \textrm{odd}, &\hbox{if $\star\in \set{C, D, D^*, C^*}$},
    \end{array}
  \right.
\]
and $\check \CO'_{\mathrm b}$ has $\star$-bad parity in the sense that all its
nonzero row lengths are
\[
  \left\{
    \begin{array}{ll}
      \textrm{odd}, &\hbox{if  $\star\in \set{B, \widetilde C}$}; \smallskip\\
      \textrm{even}, &\hbox{if  $\star\in \set{C, D, D^*, C^*}$}.
    \end{array}
  \right.
\]

For simplicity, put
\[
  l:=\abs{\ckcOpb},
\]
and
\begin{equation}\label{Gpb}
  \Gpb := \begin{cases}
    \GL_{l}(\bR), & \text{if } \star \in \set{B,C,D}; \\
       \widetilde{ \GL}_{l}(\bR), & \text{if } \star =\wtC; \\
    \GL_{\frac{l}{2}}(\bH), & \text{if } \star \in \set{C^{*},D^{*}}. \\
  \end{cases}
\end{equation}
Here $ \widetilde{ \GL}_{l}(\bR)$ is the double cover of $ \GL_{l}(\bR)$ that fits the following Cartesian diagram of Lie groups:
\begin{equation}\label{wgll}
\begin{CD}
 \widetilde{ \GL}_{l}(\bR)@>>>  \GL_{l}(\bR)\\
  @VVV @VV g\mapsto \textrm{ sign of $\det(g)$} V\\
  \{\pm 1, \pm \sqrt{-1}\} @> x\mapsto x^2 >> \{\pm 1\}. \\
\end{CD}
\end{equation}

Define
 \[
      \Unip_{\ckcOpb}(\widetilde{ \GL}_{l}(\bR)):=
       \{\pi\in \Irr(\widetilde{ \GL}_{l}(\bR))\mid \pi \textrm{ is genuine  and annihilated by } I_{\ckcOpb}:= I_{A^\R, \ckcOpb}\}.
       \]
        Here and as before, ``genuine" means that the central subgroup $\{\pm 1\}$ acts through the nontrivial character.  Then we have a bijective map
 \[
    \Unip_{\ckcOpb}(\GL_{l}(\bR))\rightarrow  \Unip_{\ckcOpb}(\widetilde{ \GL}_{l}(\bR)), \quad \pi\mapsto \pi\otimes \tilde \chi_l,
 \]
 where $\tilde \chi_l$ is the character given by the left vertical arrow of \eqref{wgll}.

Note that $G$ has a closed subgroup isomorphic to $\Gpb$ (as Lie groups) if and only if
\[
  \begin{cases}
    p,q\geq l, & \text{if $G = \SO(p,q)$};\\
    p,q\geq \frac{l}{2}, &  \text{if $G = \Sp(p,q)$};\\
    \text{no condition,} & \text{otherwise}.
  \end{cases}
\]
In such cases, $G$ has a Levi subgroup that is identified with $\Gpb\times \Gg$ (or   $(\Gpb\times \Gg)/\{\pm 1\} $ when $\star=\wtC$), where
\be\label{gg00}
  \Gg :=
  \begin{cases}
    \SO(p-l,q-l), & \textrm{if $\star\in \set{B,D}$};\\
  %  \SO_{n-2l}(\bC) &\textrm{if $\star\in \set{B^{\bC},D^{\bC}}$},\\
    \rO^{*}(2n-2l), &\textrm{if $\star = D^{*}$};\\
    \Sp_{2n-2l}(\bR), &\textrm{if $\star = C$};\\
    \wtSp_{2n-2l}(\bR), &\textrm{if $\star = \wtC$};\\
  %  \Sp_{2n-2l}(\bC) &\textrm{if $\star \in \set{C^{\bC},\wtC^{\bC}}$},\\
    \Sp(p-\frac{l}{2},q-\frac{l}{2}), &\textrm{if $\star = C^{*}$}.\\
  \end{cases}
\ee
% and respectively put
% \[
%   \begin{array}{rl}
%     \Gg:=  & \SO(p-l,q-l)\ \  (\textrm{when $p,q\geq l$}),   \ \     \SO_{n-2l}(\C),  \  \   \Sp_{2n-2l}(\R), \  \ \Sp_{2n-2l}(\C), \smallskip \\
%     %   & \oO^*(2n-2l), \ \  \Sp(p-\frac{l}{2},q-\frac{l}{2}) \ \  (\textrm{when $p,q\geq 2l$}),  \ \   \widetilde \Sp_{2n-2l}(\R) \ \ \textrm{or }  \ \  \Sp_{2n-2l}(\C),
%      \end{array}
%    \]
%    when \[
%      \begin{array}{rl}
%     G=  & \SO(p,q)   \ \     \SO_{n}(\C),  \  \   \Sp_{2n}(\R), \  \ \Sp_{2n}(\C), \smallskip \\
%     %   & \oO^*(2n), \ \  \Sp(p,q),  \ \   \widetilde \Sp_{2n}(\R) \ \ \textrm{or }  \ \  \Sp_{2n}(\C).
%      \end{array}
%    \]

\begin{thm}\label{reduction}
 If  $G$ has a closed subgroup isomorphic to $\Gpb$, then parabolic induction yields
   a bijection
   \[
 % \begin{equation}\label{eq:IND}
    \begin{array}{rccc}
      \fI\colon &   \Unip_{\ckcO'_{\mathrm b}}( G'_{\mathrm b})\times \Unip_{\ckcO_{\mathrm g}}( G_{\mathrm g})&         \longrightarrow &\Unip_{\ckcO }(G) \\
                &   (\pi',\pi_{0}) & \mapsto & \pi'\rtimes \pi_{0}.
    \end{array}
 % \end{equation}
 \]
  Otherwise,
  \[
    \Unip_{\ckcO}(G)=\emptyset.
  \]
\end{thm}


Combining with the counting result for general linear groups (Theorem \ref{GLcase}), we list the more specific results as follows:
\begin{enumerate}[label=(\alph*)]
  \item Assume that $\star\in \{B,D\}$ so that $G=\SO(p,q)$. Then
        \[
        \sharp(\Unip_{\check \CO}(G))=
        \begin{cases}
          \sharp(\Unip_{\check \CO_{\mathrm g}}(G_{\mathrm g}))\times \sharp(\Unip_{\check \CO'_{\mathrm b}}(\GL_l(\R)) ), &\hbox{if $p,q\geq l$}; \smallskip\\
          0, &\hbox{otherwise.}
        \end{cases}
        \]
  \item Assume that $\star=C^*$ so that $G=\Sp(p,q)$. Then
        \[
        \sharp(\Unip_{\check \CO}(G))=
        \begin{cases}
          \sharp(\Unip_{\check \CO_{\mathrm g}}(G_{\mathrm g} )), &\hbox{if $p,q\geq \frac{l}{2}$}; \smallskip\\
          0, &\hbox{otherwise.}
        \end{cases}
        \]

  \item Assume that $\star\in \{C,\widetilde C\}$ so that $G=\Sp_{2n}(\R)$ or
        $\widetilde \Sp_{2n}(\R)$. Then
        \[
        \sharp(\Unip_{\check \CO}(G))= \sharp(\Unip_{\check \CO_{\mathrm g}}(G_{\mathrm g}))\times \sharp(\Unip_{\check \CO'_{\mathrm b}}(\GL_l(\R)) ). \]
  \item Assume that $\star =D^*$ so that $G=\oO^*(2n)$. Then
        \[
          \sharp(\Unip_{\check \CO}(G))= \sharp(\Unip_{\check \CO_{\mathrm g}}(G_{\mathrm g})).
        \]
\end{enumerate}


 \subsection{Orthogonal and symplectic groups: the case of good parity}\label{secorgp0}
 We now assume that $\check \CO$ has $\star$-good parity, namely
 $\check \CO=\check \CO_{\mathrm g}$. By Theorem \ref{reduction}, the counting
 problem in general is reduced to this case.



 \delete{
   \begin{defn}
     A $\star$-pair is a pair $(i,i+1)$ of consecutive positive integers such
     that
     \[
       \left\{
         \begin{array}{ll}
           i\textrm{ is odd}, \quad &\textrm{if $\star\in\{C, \widetilde{C}, C^*, C^\C, \widetilde C^\C\}$};  \\
           i \textrm{ is even}, \quad &\textrm{if $\star\in\{B, D, D^*, B^\C, D^\C\}$}. \\
         \end{array}
       \right.
     \]
     A $\star$-pair $(i,i+1)$ is said to be primitive in $\check \CO$ if
     $\mathbf r_i(\check \CO)-\mathbf r_{i+1}(\check \CO)$ is positive and even.
     Denote $\mathrm{PP}_\star(\check \CO)$ the set of all $\star$-pairs that
     are primitive in $\check \CO$.
   \end{defn}
 }



\begin{defn}\label{defn:PP}
  A $\star$-pair is a pair $(i,i+1)$ of consecutive positive integers such that
  \[
    \left\{
      \begin{array}{ll}
        i\textrm{ is odd}, \quad &\textrm{if $\star\in\{C, \widetilde{C}, C^*\}$};  \\
        i \textrm{ is even}, \quad &\textrm{if $\star\in\{B, D, D^*\}$}. \\
      \end{array}
    \right.
  \]
  A $\star$-pair $(i,i+1)$ is said to be
  \begin{itemize}
    \item vacant in $\check \CO$, if
          $\mathbf r_i(\check \CO)=\mathbf r_{i+1}(\check \CO)=0$;
    \item balanced in $\check \CO$, if
          $\mathbf r_i(\check \CO)=\mathbf r_{i+1}(\check \CO)>0$;
    \item tailed in $\check \CO$, if
          $\mathbf r_i(\check \CO)-\mathbf r_{i+1}(\check \CO)$ is positive and
          odd;
    \item primitive in $\check \CO$, if
          $\mathbf r_i(\check \CO)-\mathbf r_{i+1}(\check \CO)$ is positive and
          even.
  \end{itemize}
  Denote $\CPP_\star(\check \CO)$ the set of all $\star$-pairs that are
  primitive in $\check \CO$.
\end{defn}

%We continue with the counting problem of $\Unip_{\check \CO}(G)$, when $\check \CO$ has $\star$-good parity.

We attach to $\check \CO$ a pair of Young diagrams
\be\label{ijo}
  (\imath_{\check \CO}, \jmath_{\check \CO}):=(\imath_\star(\check \CO), \jmath_\star(\check \CO)),
\ee
as follows.

\medskip

\noindent {\bf The case when $\star=B$.} In this case,
\[
  \mathbf c_{1}(\jmath_{\check \CO})=\frac{\mathbf r_1(\check \CO)}{2},
\]
and for all $i\geq 1$,
\[
  \left (\mathbf c_{i}(\imath_{\check \CO}), \mathbf c_{i+1}(\jmath_{\check \CO})\right )= \left (\frac{\mathbf r_{2i}(\check \CO)}{2}, \frac{\mathbf r_{2i+1}(\check \CO)}{2}\right ).
\]

\medskip

\noindent {\bf The case when $\star=\widetilde C$.} In
this case, for all $i\geq 1$,
\[
  (\mathbf c_{i}(\imath_{\check \CO}), \mathbf c_{i}(\jmath_{\check \CO}))= \left (\frac{\mathbf r_{2i-1}(\check \CO)}{2}, \frac{\mathbf r_{2i}(\check \CO)}{2}\right).
\]

\medskip

\noindent {\bf The case when $\star=\{ C,C^*\}$.} In this case, for all
$i\geq 1$,
\[
  (\mathbf c_{i}(\jmath_{\check \CO}), \mathbf c_{i}(\imath_{\check \CO}))= \left\{
    \begin{array}{ll}
      (0,  0), &\hbox{if $(2i-1, 2i)$ is vacant  in $\check \CO$};\smallskip\\
      (\frac{\mathbf r_{2i-1}(\check \CO)-1}{2},  0), & \hbox{if $(2i-1, 2i)$ is tailed in $\check \CO$};\smallskip\\
      (\frac{\mathbf r_{2i-1}(\check \CO)-1}{2},  \frac{\mathbf r_{2i}(\check \CO)+1}{2}), &\hbox{otherwise}.\\
    \end{array}
  \right.
\]
\medskip

\noindent {\bf The case when $\star\in \{D,D^*\}$.} In this case,
\[
  \mathbf c_{1}(\imath_{\check \CO})= \left\{
    \begin{array}{ll}
      0,  &\hbox{if $\mathbf r_1(\check \CO)=0$}; \smallskip\\
      \frac{\mathbf r_1(\check \CO)+1}{2},   &\hbox{if $\mathbf r_1(\check \CO)>0$},\\
    \end{array}
  \right.
\]
and for all $i\geq 1$,
\[
  (\mathbf c_{i}(\jmath_{\check \CO}), \mathbf c_{i+1}(\imath_{\check \CO}))= \left\{
    \begin{array}{ll}
      (0,  0), &\hbox{if $(2i, 2i+1)$ is vacant in $\check \CO$};\smallskip\\
      \left  (\frac{\mathbf r_{2i}(\check \CO)-1}{2},  0\right ), & \hbox{if $(2i, 2i+1)$ is tailed in $\check \CO$};\smallskip\\
      \left  (\frac{\mathbf r_{2i}(\check \CO)-1}{2},  \frac{\mathbf r_{2i+1}(\check \CO)+1}{2}\right ), &\hbox{otherwise}.\\
    \end{array}
  \right.
\]




\begin{eg} Suppose that $\star=C$, and $\check \CO$ is the following Young
  diagram which has $\star$-good parity.
  \begin{equation*}%\label{eq:sp-nsp.C}
    \tytb{\ \ \ \ \  , \ \ \  , \ \ \ , \ \ \  , \ \ \ , \  ,\  }
  \end{equation*}
  Then
  \[
    \CPP_\star(\check \CO)=\{(1,2), (5,6)\}
  \]
  and
  \[
    (\imath_{\check \CO}, \jmath_{\check \CO})= \tytb{\ \ \ ,\ \ } \times \tytb{\ \ \ , \ }.
  \]


\end{eg}



\delete{
  \begin{eg} Suppose that $\star=C$, and $\check \CO$ is the following Young
    diagram which has $\star$-good parity.
    \begin{equation*}%\label{eq:sp-nsp.C}
      \tytb{\ \ \ \ \  , \ \ \  , \ \ \ , \ \ \  , \ \ \ , \  ,\  }
    \end{equation*}
    Then
    \[
      \mathrm{PP}_\star(\check \CO)=\{(1,2), (5,6)\}.
    \]
    and $(\imath_\star(\check \CO, \wp), \jmath_\star(\check \CO,
    \wp))$ %\in \mathrm{BP}_\star(\check \CO)$
    has the following form.

    \begin{equation*}%\label{eq:sp-nsp.C}
      \begin{array}{rclcrcl}
        \wp=\emptyset & : & \tytb{\ \ \ ,\ \  } \times \tytb{\ \ \ , \  }  & \qquad \quad &  \wp=\{(1,2)\}& : & \tytb{\ \ \  , \ \ , \   } \times \tytb{\ \ \  } \medskip \medskip \medskip \\
        \wp=\{(5,6)\} & : & \tytb{\ \ \ ,\ \ \ } \times \tytb{\ \ , \   }  & \qquad \quad &  \wp=\{(1,2), (5,6)\}  & : & \tytb{\ \ \  , \ \ \ ,  \ } \times \tytb{\ \   } \\
      \end{array}
    \end{equation*}

\end{eg}
}

Here and henceforth, when no confusion is possible, we write
$\alpha\times \beta$ for a pair $(\alpha, \beta)$. We will also write
$\alpha\times \beta\times \gamma$ for a triple $(\alpha, \beta, \gamma)$.


We introduce two more symbols $B^+$ and $B^-$, and make the following
definition.
\begin{defn} \label{def:pbp1}
  A painted bipartition is a triple
  $\tau=(\imath, \CP)\times (\jmath, \cQ)\times \alpha$, where $(\imath, \CP)$
  and $ (\jmath, \mathcal Q)$ are painted Young diagrams, and
  $\alpha\in \{B^+,B^-, C,D,\widetilde {C}, C^*, D^*\}$, subject to the
  following conditions:
  \begin{itemize}
          \delete{\item $(\imath, \jmath)\in \mathrm{BP}_\alpha$ if
          $\alpha\notin\{B^+,B^-\}$, and $(\imath, \jmath)\in \mathrm{BP}_{B}$
          if $\alpha\in\{B^+,B^-\}$;}

    \item $\CP^{-1}(\bullet)=\mathcal Q^{-1}(\bullet)$;
    \item the image of $\CP$ is contained in
          \[
          \left\{
          \begin{array}{ll}
            \{\bullet, c\}, &\hbox{if $\alpha=B^+$ or $B^-$}; \smallskip\\
            \{\bullet,  r, c,d\}, &\hbox{if $\alpha=C$}; \smallskip\\
            \{\bullet, s, r, c,d\}, &\hbox{if $\alpha=D$}; \smallskip\\
            \{\bullet, s, c\}, &\hbox{if $\alpha=\widetilde{ C}$}; \smallskip \\
            \{\bullet\}, &\hbox{if $\alpha=C^*$}; \smallskip \\
            \{\bullet, s\}, &\hbox{if $\alpha=D^*$},\\
          \end{array}
          \right.
          \]
    \item the image of $\mathcal Q$ is contained in
          \[
          \left\{
          \begin{array}{ll}
            \{\bullet, s, r, d\}, &\hbox{if $\alpha=B^+$ or $B^-$}; \smallskip\\
            \{\bullet, s\}, &\hbox{if $\alpha=C$}; \smallskip\\
            \{\bullet\}, &\hbox{if $\alpha=D$}; \smallskip\\
            \{\bullet, r, d\}, &\hbox{if $\alpha=\widetilde{ C}$}; \smallskip\\
            \{\bullet, s,r\}, &\hbox{if $\alpha=C^*$}; \smallskip \\
            \{\bullet, r\}, &\hbox{if $\alpha=D^*$}.
          \end{array}
          \right.
          \]

  \end{itemize}
\end{defn}

% \begin{remark}
%   The set of painted bipartition counts the multiplicities of an irreducible
%   representation of $W_{r_{\fgg}}$ occurs in the coherent continuation
%   representation at the infinitesimal character of the trivial representation.
%   For the relationship between painted bipartitions and the coherent
%   continuation representations of Harish-Chandra modules, see \cite{Mc}.
% \end{remark}

For any painted bipartition $\tau$ as in Definition \ref{defpbp0}, we write
\[
  \imath_\tau:=\imath,\ \cP_\tau:=\cP,\ \jmath_\tau:=\jmath,\ \cQ_\tau:=\cQ,\ \alpha_\tau:=\alpha,\ \abs{\tau}:=\abs{\imath}+\abs{\jmath},
\]
and
\[
  \star_\tau:= \left\{
    \begin{array}{ll}
      B, &\hbox{if $\alpha=B^+$ or $B^-$}; \smallskip\\
      \alpha, & \hbox{otherwise}.           \end{array}
  \right.
\]
% Its leading column is then defined to be the first column of $(\jmath, \CQ)$
% when $\star_\tau\in \{B, C,C^*\}$, and the first column of $(\imath, \CP)$
% when $\star_\tau\in \{\widetilde C, D, D^*\}$.

We further define the signature of the painted bipartition
\[
\Sign(\uptau):=(p_{\tau}, q_{\tau})
\]
to be a pair  of natural numbers given by the
following recipe.
\begin{itemize}
  \item If $\star_\tau\in \{B, D, C^*\}$, then $(p_\tau, q_\tau)$ is given by
        counting the various symbols appearing in $(\imath, \CP)$,
        $(\jmath, \cQ)$ and $\{\alpha\}$ :
        \begin{equation*}%\label{ptqt}
          \left\{
            \begin{array}{l}
              p_\tau :=( \# \bullet)+ 2 (\# r) +(\# c )+ (\# d) + (\# B^+);\smallskip\\
              q_\tau :=( \# \bullet)+ 2 (\# s) + (\# c) + (\# d) + (\# B^-).\\
            \end{array}
          \right.
        \end{equation*}
        Here
        \[
        \#\bullet:=\#(\cP^{-1}(\bullet))+\#(\cQ^{-1}(\bullet))
        %\qquad (\textrm{$\#$        indicates the cardinality of a finite set}),
        \]
        and the other terms are similarly defined.
  \item If $\star_\tau\in \{C, \widetilde C, D^*\}$, then
        $p_\tau:=q_\tau:=\abs{\tau}$.
\end{itemize}
\smallskip

We also define a classical group
\begin{equation*}%\label{def:Gt}
  G_\tau:=
  \begin{cases}
    \SO(p_\tau, q_\tau), &\hbox{if $\star_\tau=B$ or $D$}; \smallskip\\
    \Sp_{2\abs{\tau}}(\R), &\hbox{if $\star_\tau=C$}; \smallskip\\
    \widetilde{\Sp}_{2\abs{\tau}}(\R), &\hbox{if $\star_\tau=\widetilde{ C}$}; \smallskip \\
    \Sp(\frac{p_\tau}{2}, \frac{q_\tau}{2}), &\hbox{if $\star_\tau=C^*$}; \smallskip \\
    \oO^*(2\abs{\tau}), &\hbox{if $\star_\tau=D^*$}.\\
  \end{cases}
\end{equation*}


Define
\begin{equation}\label{defpbp2222}
  \PBP_\star(\check \CO) :=\set{ \uptau\textrm{ is a painted
      bipartition} \mid \star_\uptau = \star, \text{ and
    } (\imath_\tau,\jmath_\tau) = (\imath_{\check \CO}, \jmath_{\check \CO})},
\end{equation}
and
\begin{equation*} %\label{defpbp3}
    \PBP_{G}(\ckcO) :=\set{\uptau\in \PBP_{\star}(\ckcO)| G_{\uptau} = G}.
\end{equation*}

\delete{
  \[
    \begin{array}{rl}
      \mathrm{PBP}_\star(\check \CO):=\{ &
                                           \tau\textrm{ is a painted bipartition}  \mid    \star_\tau = \star,
                                           \text{ and } \\  & (\imath_\tau,\jmath_\tau) = (\imath_{\check \CO}, \jmath_{\check \CO})\}.
    \end{array}
  \]
}


\begin{eg} Suppose that $\star=B$ and
  \[
    \check \CO =\tytb{\ \ \ \ \ \ , \ \ \ \ \ \ , \ \ , \ \ , \ \ }
  \]
  Then
  \[
    \tau:= \tytb{\bullet \bullet ,\bullet , c } \times \tytb{\bullet \bullet d ,\bullet , d }\times B^+\in \mathrm{PBP}_{\star}(\check \CO),
  \]
  and
  \[
    G_\tau=\SO(10,9).
  \]
\end{eg}


We now state our final result on the counting of special unipotent representations.


\begin{thm}\label{countup}
  Assume that $\star\in \{B, C,D,\widetilde {C}, C^*, D^*\}$, and $\check \CO$ has $\star$-good parity. Then
 \[
   \sharp(\Unip_{\check \CO}(G)) \leq
    \left\{
    \begin{array}{ll}
       \sharp (\PBP_{G}(\ckcO)),  & \hbox{if $\star\in \{C^*,D^*\}$}; \smallskip\\
       2^{\sharp(\CPPs(\check \CO))} \cdot \sharp (\PBP_{G}(\ckcO)),  &\hbox{if $\star\in \{B, C,D,\widetilde {C}\}$}.
    \end{array}
  \right.
  \]
\end{thm}

In \cite{BMSZ2}, the authors construct a set of representations in
$\Unip_{\check \CO}(G)$ whose cardinality equals the upper bound in Theorem \ref{countup}, when $\check \CO$ has $\star$-good parity. See \cite[Theorem 4.1]{BMSZ2}. Thus the equality holds in \Cref{countup}.


\trivial[h]{
\begin{thm}\label{countup}
  Assume that $\star\in \{B, C,D,\widetilde {C}, C^*, D^*\}$, and $\check \CO$ has $\star$-good parity. Then
  \[
    \sharp(\Unip_{\ckcO}(G))\leq 2^{\sharp(\CPP_\star(\check \CO))} \cdot \sharp (\PBP_{\mathrm g}(\ckcO)).
  \]
\end{thm}

In \cite{BMSZ2}, the authors have constructed $2^{\sharp(\CPP_\star(\check \CO))} \cdot \sharp (\PBP_{\mathrm g}(\ckcO))$ number of representations in
$\Unip_{\check \CO}(G)$, when $\check \CO$ has $\star$-good parity. See \cite[Theorem 4.1]{BMSZ2}. Thus the equality holds in \Cref{countup}.

\medskip}

%\begin{remark}

\subsection{The case of complex classical groups}
Special unipotent representations of complex classical groups are well-understood (\cite{BVUni}, \cite{B89}). We briefly review their counting and constructions in what follows. As the methods of this paper and \cite{BMSZ2} work for complex classical groups, we will present the results in the complex case parallel to those of this paper and \cite{BMSZ2}. For this subsection, we introduce five more symbols $A^\C, B^\C,D^\C, C^\C$, and $\widetilde C^\C$, and let $\star$ be one of them. Let $G$ be a complex classical group of type $\star$, namely $G=\GL_n(\C)$, $\SO_{2n+1}(\C)$, $\SO_{2n}(\C)$, $\Sp_{2n}(\C)$, or $\Sp_{2n}(\C)$ ($n\geq 0$), respectively. The Langlands dual $\check G$ of $G$ is respectively defined to be $\GL_n(\C)$, $\Sp_{2n}(\C)$, $\SO_{2n}(\C)$, $\SO_{2n+1}(\C)$, or $\Sp_{2n}(\C)$. Let $\check \CO$ be a $\check G$-orbit in $\Nil(\check \g)$ where $\check \g$ is the Lie algebra of $\check G$. As in the real case we have a maximal ideal $I_{\check \CO}:=I_{\star, \check \CO}$ of $\CU(\g_0)$, where $\g_0$ is the Lie algebra of $G$ (viewed as a complex Lie group).


 Write $\overline \g_0$ for the complex Lie algebra equipped with a conjugate linear isomorphism $\bar{\phantom a} :\g_0\rightarrow \overline{\g_0}$. The latter induces a  conjugate linear isomorphism $\bar{\phantom a} :\CU(\g_0)\rightarrow \CU( \overline{\g_0})$. Note that $\g_0\times \overline{\g_0}$ equals the complexified Lie algebra $\g$ of $G$.  Define the set of special unipotent representations of $G$
 attached to $\ckcO$ by
 \[
     \Unip_{\ckcO}(G):=  \Unip_{\star, \ckcO}(G)
     :=
       \{\pi\in \Irr(G)\mid \pi \textrm{ is annihilated by } I_{\check \CO}\otimes \CU(\overline{\g_0}) + \CU(\g_0)\otimes \overline{I_{\check \CO}}\, \}.
       \]

 If $\star=A^\C$ so that $G=\GL_n(\C)$, then $\Unip_{\ckcO}(G)$ is a singleton whose unique element is given by the normalized parabolic induction
 $\Ind_{P}^{G} 1_P$, where $P$ is the standard parabolic subgroup whose Levi component equals
 \[
 \GL_{\mathbf r_1(\check \CO)}(\C)\times \GL_{\mathbf r_2(\check \CO)}(\C)\times \dots \times \GL_{\mathbf r_{\mathbf c_1(\check \CO)}(\check \CO)}(\C),
 \]
 and $1_P$ denotes the trivial representation of $P$.
 %See \cite{V.GL}.

Now suppose that $\star\in \{B^\C,D^\C, C^\C, \widetilde C^\C\}$. As in the real case, write
 \[
   \ckcO=\ckcOg \cuprow 2\ckcOpb \quad\textrm{and}\quad l:=\abs{\ckcOpb}
 \]
 so that $G$ has a Levi subgroup  that is identified with $\Gpb\times \Gg$, where $\Gpb=\GL_l(\C)$ and
\[
  \Gg :=
  \begin{cases}
    \SO_{2n-2l+1}(\C), & \textrm{if $\star=B^\C$};\\
    \SO_{2n-2l}(\C), & \textrm{if $\star=D^\C$};\\
    \Sp_{2n-2l}(\C), &\textrm{if $\star \in\{ C^\C, \widetilde C^\C \}$}.
      \end{cases}
\]

Define the set $\CPP_\star(\ckcOg)$ as in the real case. Then by the work of
Barbasch-Vogan \cite[Corollary 5.29]{BVUni} (integral case) and Moeglin-Renard
\cite{MR.C} (general case), we have that
   \[
    \sharp(\Unip_{\check \CO}(G))=\sharp(\Unip_{\ckcOg}(\Gg))=2^{\sharp(\CPP_\star(\ckcOg))}.
  \]
   As in the real case, every representation in $\Unip_{\check \CO}(G)$ is
   obtained through irreducible parabolic induction via those of
   $\Unip_{\ckcO'_{\mathrm b}}( G'_{\mathrm b})\times \Unip_{\ckcO_{\mathrm g}}( G_{\mathrm g})$
   (see Theorem \ref{reduction}), and every representation in
   $\Unip_{\ckcOg}(\Gg)$ is obtained through iterated theta lifting (see
   \cite[Theorem 3.5.1]{B17}, \cite{Mo17} and \cite{BMSZ2}).


\vskip.25in

Here are some words on the contents and the organization of this article. In Section 2, we develop some generalities on the coherent continuation representation, which lead to the proofs of Theorems \ref{count1}, \ref{count2} and \ref{counteq}. The generalities include coherent continuation representations for highest weight modules, primitive ideals and Goldie rank polynomials, as well as cell representations in the coherent continuation setting. As mentioned earlier, we build on previous works of several authors.
In Section 3, we give explicit formulas for the coherent continuation representation $\Coh_{\Lam}(\CK(G))$, based on an unpublished result of Barbasch and Vogan. Sections 4 to 7 are devoted to the main concern of the article, which is to give a precise count of special unipotent representations of all real classical groups, using results of Sections 2 and 3. We first deal with the general linear groups and the unitary groups, and then real classical groups of type $\mathrm{BCD}$. All answers are given in terms of combinatorial constructs described earlier in this section. It is worthwhile to note, while the algebraic theory developed in Sections 2 and 3 yields ultimately an upper bound of the count, we are unable to demonstrate the precise count using the algebraic theory alone, due to a certain technical issue on the relationship of a Harish-Chandra cell and a Lusztig double cell, which we have formerly stated as Conjecture \ref{conjcell}. In the case at hand, namely for the real classical groups, we rely on the analytic theory of theta lifting to construct the right number of special unipotent representations (\cite{BMSZ2}), thus arriving at the precise count. It will be clearly desirable to demonstrate the precise count, without recourse to the analytic theory.




\section{Counting of special unipotent representations in type BCD}


%\begin{remark}
%If
%\end{remark}

In this section, we assume that $\star \in \set{B,C,\wtC,C^{*},D,D^{*}}$. Proposition \ref{prop:cohBCD44} and \eqref{boundc22} and  imply that the set $\Unip_{\check \CO}(G)$ is empty unless
 $p_\mathrm g, q_\mathrm g\geq 0$. For this reason we assume that  $p_\mathrm g, q_\mathrm g\geq 0$ throughout this section. 
 
 

To ease the notation, for every sequence $a_1\geq a_2\geq \dots \geq a_k\geq 0$ ($k\geq 0$) of integers,   we let $[a_1, a_2, \cdots, a_k]_{\mathrm{col}}$ denote the Young diagram
whose $i$-th column has length $a_i$ if $1 \leq i \leq k$ and length $0$
otherwise. Likewise, we let $[a_1, a_2, \cdots, a_k]_{\mathrm{row}}$ denote the Young diagram
whose $i$-th  row has length $a_i$ if $1 \leq i \leq k$ and length $0$ otherwise.

 As usual, we identify $\Irr(\sfW_{n})$ ($n\geq 0$) with the set of bipartitions $\tau =(\tau_{L},\tau_{R})$ of total size $n$ (\cite[Section 11.4]{Carter}). Here the total size refers to
$\abs{\tau_{L}}+\abs{\tau_{R}}$.
%Given the chosen embedding of $\sfS_{n}$ into $\sfW'_{n}$,
We also let $(\tau_{L},\tau_{R})_{I}\in \Irr(\sfW'_n)$ denote the  irreducible representation  given by
  \begin{itemize}
    \item the restriction of $(\tau_{L},\tau_{R})\in \Irr(\sfW_{n})$  if
    $\tau_{L}\neq \tau_{R}$, and
    \item
    the induced representation
    $\Ind_{\sfS_{n}}^{\sfW'_{n}} \tau_{L}$ if $\tau_{L}=\tau_{R}$.
  \end{itemize}
  Note that
  \[
    (\tau_{L},\tau_{R})_{I}=(\tau_{R},\tau_{L})_{I}\in \Irr(\sfW'_{n}).
  \]



  \subsection{The left cells}
  \label{sec:LCBCD}
  In this subsection, we describe the Lusztig left cell $\LC_{\lambda_{\ckcO}}$
  attached to $\lambda_{\ckcO}$. 
 

Define two Young diagrams
 \begin{equation}\label{eq:taub}
    \begin{split}
      \tau_{L,\mathrm b} := \begin{cases}
        \big[\half(\bfrr_{1}(\ckcO'_{\mathrm b})+1), \half(\bfrr_{2}(\ckcO'_{\mathrm b})+1), \cdots, \half(\bfrr_{c}(\ckcO'_{\mathrm b})+1)\big]_{\mathrm{col}},
               &\quad \text{if } \star \in \set{B,\wtC}; \\% \smallskip \\
         \big[\half\bfrr_{1}(\ckcO'_{\mathrm b}), \half\bfrr_{2}(\ckcO'_{\mathrm b}),\cdots, \half\bfrr_{c}(\ckcO'_{\mathrm b})\big]_{\mathrm{col}},
        &\quad  \text{if } \star \in \set{C,C^{*}, D,D^{*}},\\
      \end{cases}
    \end{split}
  \end{equation}
  and
   \begin{equation}\label{eq:taub2}
    \begin{split}
      \tau_{R,\mathrm b} := \begin{cases}
        \big(\half(\bfrr_{1}(\ckcO'_{\mathrm b})-1), \half(\bfrr_{2}(\ckcO'_{\mathrm b})-1), \cdots, \half(\bfrr_{c}(\ckcO'_{\mathrm b})-1)\big)_{\mathrm{col}},
               &\quad \text{if } \star \in \set{B,\wtC}; \\% \smallskip \\
         \big(\half\bfrr_{1}(\ckcO'_{\mathrm b}), \half\bfrr_{2}(\ckcO'_{\mathrm b}),\cdots, \half\bfrr_{c}(\ckcO'_{\mathrm b})\big)_{\mathrm{col}},
        &\quad  \text{if } \star \in \set{C,C^{*}, D,D^{*}},\\
      \end{cases}
    \end{split}
  \end{equation}
 where $c:= \bfcc_{1}(\ckcO'_{\mathrm b})$.

 Recall  from \eqref{woc} that  $
    W_{[\lamck]} =W_{[\lamckg]}\times W_{[\lamckb]}=\Wg\times \Wb$.
 Define an irreducible  representation $\tau_{\mathrm b}\in \Irr(W_{\mathrm b})$ attached to $\ckcO_{\mathrm b}$ by
\begin{equation}\label{eq:taub}
 \tau_{\mathrm b}:= \left\{
     \begin{array}{ll}
       ( \tau_{L,\mathrm b}, \tau_{R,\mathrm b}), \qquad
       & \text{if } \star \in \set{B,\wtC}; \medskip\\
         ( \tau_{L,\mathrm b}, \tau_{R,\mathrm b})_I, \qquad & \text{if } \star \in \set{C,C^{*}, D,D^{*}}.
\end{array}
  \right.
\end{equation}



Recall the set  $\CPPs(\ckcO_{\mathrm g})$ from   \Cref{defn:PP}.   Put
  \[
    {\mathrm A}(\ckcO) := {\mathrm A}(\ckcO_{\mathrm g}):= \textrm{the power set of $\CPPs(\ckcO_{\mathrm g})$},
    \]
    which is identified with the free $\bF_2$-vector space with free basis $\CPP(\ckcO_{\mathrm g})$. Here $\bF_{2}:=\bZ/2\bZ$ is the field with two elements only.
Note that   $\{\emptyset, \CPP(\ckcO_{\mathrm g})\}$ is a subgroup of ${\mathrm A}(\ckcO)$.  Define
   \begin{equation*}%\label{def:barA}
  \bar{\mathrm A}(\ckcO):= \bar{\mathrm A}(\ckcO_{\mathrm g}):=
  \begin{cases}
 {\mathrm A}(\ckcO)/\{\emptyset, \CPP(\ckcO_{\mathrm g})\}, & \quad \text{if  } \star =\wtC;\\
 {\mathrm A}(\ckcO),  & \quad \text{otherwise.}
  \end{cases}
    % \begin{cases}
    %   \wtA(\ckcO)/\wp\sim\wp^{c} & \text{when } \star \in \set{\wtC,D,D^{*}}.\\
    %   \wtA(\ckcO) & \text{when } \star \in \set{B,C,C^{*}},\\
    % \end{cases}
  \end{equation*}

  Generalizing \eqref{ijo}, for each $\wp\in  {\mathrm A}(\ckcO)$,    we define a pair \[
(\imath_\wp, \jmath_\wp):=(\imath_\star(\check \CO, \wp), \jmath_\star(\check \CO, \wp))
\]
 of Young diagrams  as in what follows.

If $\star=B$, then
 \[
   \mathbf c_{1}(\jmath_\wp)=\frac{\mathbf r_1(\check \CO_{\mathrm g})}{2},
\]
and for all $i\geq 1$,
\[
(\mathbf c_{i}(\imath_\wp), \mathbf c_{i+1}(\jmath_\wp))=
   \left\{
     \begin{array}{ll}
           (\frac{\mathbf r_{2i+1}(\check \CO_{\mathrm g})}{2},  \frac{\mathbf r_{2i}(\check \CO_{\mathrm g})}{2}), &\hbox{if $(2i, 2i+1)\in \wp$}; \smallskip\\
            (\frac{\mathbf r_{2i}(\check \CO_{\mathrm g})}{2},  \frac{\mathbf r_{2i+1}(\check \CO_{\mathrm g})}{2}), &\hbox{otherwise}.\\
            \end{array}
   \right.
\]


If $\star=\widetilde{C}$, then for all $i\geq 1$,
\[
(\mathbf c_{i}(\imath_\wp), \mathbf c_{i}(\jmath_\wp))=
   \left\{
     \begin{array}{ll}
           (\frac{\mathbf r_{2i}(\check \CO_{\mathrm g})}{2},  \frac{\mathbf r_{2i-1}(\check \CO_{\mathrm g})}{2}), &\hbox{if $(2i-1, 2i)\in \wp$}; \smallskip\\
            (\frac{\mathbf r_{2i-1}(\check \CO_{\mathrm g})}{2},  \frac{\mathbf r_{2i}(\check \CO_{\mathrm g})}{2}), &\hbox{otherwise}.\\
            \end{array}
   \right.
\]


If $\star\in\{D,D^*\}$, then
 \[
   \mathbf c_{1}(\imath_\wp)= \left\{
     \begin{array}{ll}
            \frac{\mathbf r_1(\check \CO_{\mathrm g})+1}{2},   &\hbox{if $\mathbf r_1(\check \CO_{\mathrm g})>0$}; \smallskip\\
       0,  &\hbox{if $\mathbf r_1(\check \CO_{\mathrm g})=0$},\\
            \end{array}
   \right.
 \]
and for all $i\geq 1$,
\[
(\mathbf c_{i}(\jmath_\wp), \mathbf c_{i+1}(\imath_\wp))=
   \left\{
     \begin{array}{ll}
            (\frac{\mathbf r_{2i+1}(\check \CO_{\mathrm g})-1}{2},  \frac{\mathbf r_{2i}(\check \CO_{\mathrm g})+1}{2}), &\hbox{if $(2i, 2i+1)\in \wp$}; \smallskip\\
        (\frac{\mathbf r_{2i}(\check \CO_{\mathrm g})-1}{2},  0), & \hbox{if $(2i, 2i+1)$ is tailed in $\check \CO_{\mathrm g}$};\smallskip\\
         (0,  0), &\hbox{if $(2i, 2i+1)$ is empty in $\check \CO_{\mathrm g}$};\\
         (\frac{\mathbf r_{2i}(\check \CO)-1}{2},  \frac{\mathbf r_{2i+1}(\check \CO_{\mathrm g})+1}{2}), &\hbox{otherwise}.\\
            \end{array}
   \right.
\]


If $\star\in\{C,C^*\}$, then for all $i\geq 1$,
\[
(\mathbf c_{i}(\jmath_\wp), \mathbf c_{i}(\imath_\wp))=
   \left\{
     \begin{array}{ll}
            (\frac{\mathbf r_{2i}(\check \CO_{\mathrm g})-1}{2},  \frac{\mathbf r_{2i-1}(\check \CO_{\mathrm g})+1}{2}), &\hbox{if $(2i-1, 2i)\in \wp$}; \smallskip\\
        (\frac{\mathbf r_{2i-1}(\check \CO_{\mathrm g})-1}{2},  0), & \hbox{if $(2i-1, 2i)$ is tailed in $\check \CO_{\mathrm g}$};\smallskip\\
         (0,  0), &\hbox{if $(2i-1, 2i)$ is empty in $\check \CO_{\mathrm g}$};\\
         (\frac{\mathbf r_{2i-1}(\check \CO_{\mathrm g})-1}{2},  \frac{\mathbf r_{2i}(\check \CO_{\mathrm g})+1}{2}), &\hbox{otherwise}.\\
            \end{array}
   \right.
\]


We define an element $\tau_{\wp}\in \Irr(\Wg)$ by
  \begin{equation}\label{eq:tauwp}
    \tau_{\wp} :=
    \begin{cases}
      (\imathp,\jmathp),  & \quad \text{if } \star \in \set{B,C, C^{*} }; \\
      (\imathp,\jmathp)_{I},  & \quad \text{if } \star \in \set{\wtC,D,D^{*}}.
    \end{cases}
  \end{equation}
  Note that  if $\star=\wtC$, then $\tau_{\wp} = \tau_{\wp^{c}}$, where $\wp^{c}$ is the complement of $\wp$ in $\CPPs(\ckcO_{\mathrm g})$.
Therefore in all cases,   $\tau_{\bar \wp}\in \Irr(\Wg)$ is obviously defined for every $\bar \wp\in \bar{\mathrm A}(\check \CO)$.



\begin{remark} When $\star\neq \wtC$, $\mathrm A(\ckcO)$ gives another description of Lusztig's canonical
  quotient attached to $\ckcO$. The set $\CPP_{\star}(\ckcO)$ appears implicitly in
\cite{So}*{Section~5}.

  \trivial[h]{ This can be seen from the following
    lemma, c.f. \cite{BVUni}*{Proposition~5.28}. }
\end{remark}

  To simplify the notation, we write $\LC_{\ckcO}:= \LC_{\lambda_{\ckcO}}$. Recall that $\LC_{\ckcO}$ is the set of all $\sigma\in \Irr(W_{[\lambda_{\check \CO}]})$  that occurs in  \[
    \LV_{\ckcO}:= \left(J_{\Wlamck}^{\WLamck} \sgn\right) \otimes \sgn.
  \]


  \begin{lem}[\cf Barbasch-Vogan {\cite{BVUni}*{Proposition~5.28}}]
    \label{lem:Lcell}
    The representation   $\LV_{\ckcO}$ of $W_{[\lambda_{\check \CO}]}$ is multiplicity free, and  the
  map    \[
      \begin{array}{rcl}
        \bar{\mathrm A}(\ckcO) & \rightarrow & \LC_{\ckcO},\\
                       \bar \wp & \mapsto & \tau_{\bar \wp} \otimes \tau_{\mathrm b}
      \end{array}
    \]
    is well-defined and bijective.
    Moreover,
    \[
      \tau_{\ckcO}:=\tau_{\emptyset}\otimes \tau_{\mathrm b}
    \] is the unique special representation in $\LC_{\ckcO}$,
    % \begin{equation}\label{eq:dBV.W}
    %   \Spr ^{-1}(j_{\WLamck}^{W}(\tau_{\ckcO})) = \textrm{the unique Zariski open orbit in $\mathrm{AV}(I_{\star, \ckcO})$}.
    %   \end{equation} Here
    %   $\mathrm{AV}$ indicates the associated variety.
    \begin{equation}\label{eq:dBV.W}
      \Spr ^{-1}(j_{\WLamck}^{W}(\tau_{\ckcO}))
      = \dBV(\ckcO),
       \end{equation}
      and
  \begin{equation}\label{eq:dBV.W2}
     \dBV(\ckcO)=  \dBV(\ckcOg) \cupcol (\ckcOpb)^{t} \cupcol (\ckcOpb)^{t}
   \end{equation}
as Young diagrams.
        \end{lem}

          % \[
          %   (\bfcc_{i}(\imathp), \bfcc_{i}(\jmathp)):=
          %   \begin{cases}
          %     (\half (\bfrr_{2i-1}(\ckcO_{\mathrm g})+1),\half (\bfrr_{2i}(\ckcO_{\mathrm g})-1)) &\text{if } (2i-1,2i)\in \wp, \\
          %     (\half (\bfrr_{2i}(\ckcO_{\mathrm g})+1), \half (\bfrr_{2i-1}(\ckcO_{\mathrm g})-1))
          %     & \text{if } (2i-1,2i)\notin \wp\\
          %     & \text{ and }\bfrr_{2i}(\ckcO_{\mathrm g})\neq 0,
          %     \\
          %     (0,0)
          %     & \text{if } \bfrr_{2i-1}(\ckcO_{\mathrm g})=0,\\
          %     (0, \half (\bfrr_{2i-1}(\ckcO_{\mathrm g})-1)) & \text{otherwise}
          %   \end{cases}
          % \]

          % \[
          %   (\bfcc_{l+1}(\imathp), \bfcc_{l+1}(\jmathp)) := (0,\half(\bfrr_{2l+1}(\ckcO_{\mathrm g})-1))
          % \]
          % and for all $1\leq i\leq l$

    \trivial[h]{ The last equality could be checked using Sommer's formula on
      Springer correspondence directly: double columns $(2c+1,2c+1)$ corresponds
      to
      $ B_{c=\alpha_{2i-1}}\times D_{c+1=\alpha_{2i}+1}=D_{c+1=\alpha_{2i-1}+1}\times C_{c=\alpha_{2i}}$
      factor in type $B,\wtC$. double columns $(2c,2c)$ corresponds to factor
      $D_{c=\beta_{2i-1}}\times C_{c=\beta_{2i}}=D_{c=\beta_{2i-1}}\times B_{c=\beta_{2i}}$
      in type $C,C^{*},D,D^{*}$. Here $\dBV$ is the metaplectic dual if
    $\star=\wtC$ and is the Barbasch-Vogan dual otherwise.
}

  \begin{proof}
    When  $\ckcO=\ckcO_{\mathrm g}$ and $\star\neq \wtC$, the lemma is proved in \cite{BVUni}*{Proposition~5.28}.  When $\star\neq \wtC$, the equalities
    \eqref{eq:dBV.W} and \eqref{eq:dBV.W2} are proved in \cite{BVUni}*{Proposition~A2}. In general, the lemma follows from
    %from an induction on number of columns of $\ckcO_{\mathrm g}$ using
    Lusztig's formula of $J$-induction in \cite{Lu}*{\S 4.4-4.6}.

    \trivial[h]{{ {\bf Suppose $\star=C$.}
      In this case, bad parity is even and each row length occur with even
      multiplicity. Suppose
      $\ckcO_{\mathrm b} = (C_{1}, C_{1}, C_{2},C_{2}, \cdots, C_{k'},C_{k'})$ with
      $c_{1}=2k$ and $k' = \bfrr_{1}(\ckcO_{\mathrm b})$.
      \[
        W_{\lamckb} = S_{C_{1}}\times S_{C_{2}}\times \cdots S_{C_{k'}}.
      \]
      The symbol of trivial representation of trivial group of type D is
      \[
        \binom{0,1, \cdots, k-1}{0,1, \cdots, k-1}.
      \]
      Now it is easy to see that (use the similar computation as below)
      \[
        J_{W_{\lamckb}}^{W_{\mathrm b}}\sgn = ((\half C_{1}, \half C_{2},\cdots, \half C_{k'}),(\half C_{1}, \half C_{2},\cdots, \half C_{k'})).
      \]


      For the good parity part. Let
      $r'_{i} = \floor{\half(\bfrr_{i}(\ckcO_{\mathrm g})-\bfrr_{i+1}(\ckcO_{\mathrm g}))}$.
      Suppose $\ckcO_{\mathrm g}$ has $2l+1$ columns (superscripts denote the
      multiplicity)
      \[
        \ckcO_{\mathrm g} = ((2l+1)^{2r'_{2l+1}+1}, 2l^{2r'_{2l}}, (2l-1)^{2r'_{2l-1}}, \cdots, 2^{2r'_{2}}, 1^{2r'_{1}} )
      \]
      and
      % $\ckcO_{\mathrm g} = (2c_{1}+1, C_{2}, C_{2},C_{3},C_{3},\cdots, C_{k'},C_{k'})$
      % with $2c_{1}+1=2l+1$ and $2k'+1 = \bfrr_{1}(\ckcO_{\mathrm g})$.
      \[
        W_{\lamckg} = W_{l}\times \underbrace{S_{2l+1}\times \cdots \times S_{2l+1}}_{2r'_{2l+1}\text{-terms}} \times \prod_{i<2l+1} \underbrace{S_{i}\times \cdots\times S_{i}}_{r'_{i}\text{-terms}}
      \]

      The symbol of sign representation of $W_{l}$ is
      \[
        \binom{0,1,2, \cdots, l}{1,2, \cdots, l}.
      \]
      The induction begins with the longest columns to the shorter columns

      Induce to include all $2l+1$-length columns yields
      \[
        \binom{r'_{2l+1}+0,r'_{2l+1}+1,r'_{2l+1}+2, \cdots, r'_{2l+1}+l}{ r'_{2l+1}+1,r'_{2l+1}+2, \cdots, r'_{2l+1}+l}.
      \]
      Now move the shorter columns, we see that when even columns
      $(2i)^{2r'_{2i}}$ occurs, it adds $(i)^{r'_{2i}}$ columns on the both
      sides of the bipartition; when odd columns $(2i+1)^{r'_{2i+1}}$ occur, the
      bifurcation happens: one can
      \begin{itemize}
        \item attach columns $(i+1)^{r'_{2i+1}}$ on the left and columns
              $(i)^{r'_{2i+1}}$ on the right, which corresponds to
              $(2i+1,2i+2)\neq \wp$, or
        \item attach columns $(i)^{r'_{2i+1}}$ on the left and columns
              $(i+1)^{r'_{2i+1}}$ on the right, which corresponds to
              $(2i+1,2i+2)\in \wp$,
      \end{itemize}

      Therefore,
      \[
        \begin{array}{ccc}
          J_{W_{\lamckg}}^{W_{\mathrm g}} \sgn
          &\leftrightarrow&  \bF_{2}(\CPP(\ckcO_{\mathrm g}))\\
          (\cktau_{L},\cktau_{R}) =:\cktau_{\wp}&\leftrightarrow & \wp
        \end{array}
      \]
      where
      \[
        \bfrr_{l+1}(\cktau_{L}) = r'_{2l+1} = \half (\bfrr_{2l+1}(\ckcO_{\mathrm g})-1)
      \]
      and, if $(2i-1,2i)\notin \wp$,
      \[
        \begin{split}
          \bfrr_{i}(\cktau_{L}) & = \sum_{l\geq 2i-1} r'_{l}
          = \half(\bfrr_{2i-1}(\ckcO)-1)\\
          \bfrr_{i}(\cktau_{R}) & = 1 + \sum_{l\geq 2i} r'_{l} = \half(\bfrr_{2i}(\ckcO)+1)
        \end{split}
      \]
      if $(2i-1,2i)\in \wp$,
      \[
        \begin{split}
          \bfrr_{i}(\cktau_{L}) & = \sum_{l\geq 2i} r'_{l}
          = \half(\bfrr_{2i}(\ckcO)-1)\\
          \bfrr_{i}(\cktau_{R}) & = 1 + \sum_{l\geq 2i-1} r'_{l} = \half(\bfrr_{2i-1}(\ckcO)+1)
        \end{split}
      \]

      % \[
      %   \begin{split}
      %     \bfrr_{l+1}(\cktau_{L}) & = r'_{2l+1} =
      %     \half (\bfrr_{2l+1}(\ckcO_{\mathrm g})-1)\\
      %     (\bfrr_{i}(\cktau_{L}), \bfrr_{i}(\cktau_{R})) & =
      %     \begin{cases}
      %       (\half(\bfrr_{2i-1}(\ckcO_{\mathrm g})-1), \half(\bfrr_{2i}(\ckcO_{\mathrm g})+1)) & (2i-1,2i)\notin \wp\\
      %       (\half(\bfrr_{2i}(\ckcO_{\mathrm g})-1), \half(\bfrr_{2i-1}(\ckcO_{\mathrm g})+1)) & (2i-1,2i)\in \wp
      %     \end{cases}
      %   \end{split}
      % \]

      Since $\tau_{\wp} = \cktau_{\wp}\otimes \sgn$, we get the claim.

      We adopt the convention that
      \[
        \sfS_{\cO} := \prod_{i\in \bN^{+}}\sfS_{\bfcc_{i}(\cO)}
      \]
      so that $j_{\sfS_{\cO}}^{\sfS_{\abs{\cO}}}\sgn = \cO$ for each partition
      $\cO$.

      Now consider the orbit under the Springer correspondence.

      Let
      $\ckcO'_{\mathrm b}: = [\bfrr_{2}(\ckcO_{\mathrm b}), \bfrr_{4}(\ckcO_{\mathrm b}),\cdots, \bfrr_{2k}(\ckcO_{\mathrm b})]$,
      $\cO'_{\mathrm b}:=(\ckcO'_{\mathrm b})^{t}$ and $\cO_{\mathrm b}:=\cO'_{\mathrm b}\cupcol \cO'_{\mathrm b}$.
      Clearly, $\ckcO_{\mathrm b} = \ckcO'_{\mathrm b}\cuprow \ckcO'_{\mathrm b}$. Note that
      $\tau_{\mathrm b} = j_{S_{\cO'_{\mathrm b}}}^{W'_{\mathrm b}} \sgn$ (by the formula of fake degree
      see Lusztig or Carter's book). So, by induction by stage of $j$-induction,
      we have
      \[
        \wttau_{\cO}:= j_{W'_{\mathrm b}\times W_{\mathrm g}}^{W_{n}} (\tau_{\mathrm b}\otimes \tau_{\emptyset}) = j_{S_{\cO'_{\mathrm b}}\times W_{\mathrm g}}^{W_{n}} \sgn\otimes \tau_{\wp}.
      \]
      By Barbasch-Vogan, $\cO_{\mathrm g}:=\Spr(\tau_{\emptyset}) = d_{BV}(\ckcO_{\mathrm g})$,
      which is well know how to calculate. (In fact, one can deduce the result
      by our computation. )

      Since the Springer correspondence commutes with parabolic induction, we
      get
      $\Spr(\wttau) = \Ind_{\GL_{\cO'_{\mathrm b}}\times \Sp(2g)}^{\Sp(2n)} 0\times \cO_{\mathrm g} = \cO_{\mathrm b}\cupcol \cO_{\mathrm g}$.


      \medskip

      {\bf Suppose $\star=D$.}

      The bad parity part is the same as that of the case when $\star = C$.

      Now consider the good parity part.
      \[
        \ckcO_{\mathrm g} = ((2l)^{2r'_{2l}+1}, (2l-1)^{2r'_{2l-1}}, (2l-2)^{2r'_{2l-2}}, \cdots, 2^{2r'_{2}}, 1^{2r'_{1}} )
      \]
      and
      \[
        W_{\lamckg} = W'_{l}\times \underbrace{S_{2l}\times \cdots \times S_{2l}}_{2r'_{2l}\text{-terms}} \times \prod_{i<2l} \underbrace{S_{i}\times \cdots\times S_{i}}_{r'_{i}\text{-terms}}
      \]

      The symbol of sign representation of $W'_{l}$ is
      \[
        \binom{0,1, \cdots, l-1}{1,2, \cdots, l\phantom{-1}}.
      \]
      (Here we always made the choice of the top and bottom row to compatible
      with the type $C$ case. )

      Induce to include all $2l$-length columns yields
      \[
        \binom{r'_{2l}+0,r'_{2l}+1, \cdots, r'_{2l}+l-1}{ r'_{2l}+1,r'_{2l}+2, \cdots, r'_{2l}+l\phantom{-1}}.
      \]
      Now move the shorter columns. When odd columns $(2i+1)^{2r'_{2i+1}}$
      occurs, it adds $(i)^{r'_{2i+1}}$ columns on the left and
      $(i+1)^{r'_{2i+1}}$ on the right. When even columns $(2i)^{r'_{2i}}$
      occur, the bifurcation happens: one can
      \begin{itemize}
        \item attach columns $(i)^{r'_{2i}}$ on the left and columns
              $(i)^{r'_{2i}}$ on the right, which corresponds to
              $(2i,2i+1)\neq \wp$, or
        \item attach columns $(i-1)^{r'_{2i}}$ on the left and columns
              $(i+1)^{r'_{2i}}$ on the right, which corresponds to
              $(2i,2i+ 1)\in \wp$,
      \end{itemize}

      Therefore,
      \[
        \begin{array}{ccc}
          \bF_{2}(\CPP(\ckcO_{\mathrm g}))&\longrightarrow
          & J_{W_{\lamckg}}^{W_{\mathrm g}} \sgn \\
          \wp&\mapsto&    (\cktau_{L},\cktau_{R}) =:\cktau_{\wp}
        \end{array}
      \]
      where
      \[
        \bfrr_{l}(\cktau_{L}) = r'_{2l} = \half (\bfrr_{2l}(\ckcO_{\mathrm g})-1)
      \]
      \[
        \bfrr_{1}(\cktau_{R}) = 1+ \sum_{i} r'_{i} = \half (\bfrr_{1}(\ckcO_{\mathrm g})+1)
      \]
      and, if $(2i,2i+1)\notin \wp$,
      \[
        \begin{split}
          \bfrr_{i}(\cktau_{L}) & = \sum_{l\geq 2i} r'_{l}
          = \half(\bfrr_{2i}(\ckcO)-1)\\
          \bfrr_{i+1}(\cktau_{R}) & = 1 + \sum_{l\geq 2i+1} r'_{l} = \half(\bfrr_{2i+1}(\ckcO)+1)
        \end{split}
      \]
      if $(2i,2i+1)\in \wp$,
      \[
        \begin{split}
          \bfrr_{i}(\cktau_{L}) & = \sum_{l\geq 2i+1} r'_{l}
          = \half(\bfrr_{2i+1}(\ckcO)-1)\\
          \bfrr_{i}(\cktau_{R}) & = 1 + \sum_{l\geq 2i} r'_{l} = \half(\bfrr_{2i}(\ckcO)+1)
        \end{split}
      \]

      Also note that $\cktau_{\wp}=\cktau_{\wp^{c}}$. The rest parts are the
      same as that of type $C$.

      {\bf Suppose $\star=B$. }

      In this case, bad parity is odd and every odd row occurs with even
      times.

      We can write
      $r'_{i} := \floor{\half(\bfrr_{i}(\ckcO_{\mathrm b})-\bfrr_{i-1}(\ckcO_{\mathrm b}))}$
      \[
        \ckcO_{\mathrm b} % = [2r_{1}+1, 2r_{1}+1, \cdots, 2r_{k}+1,2r_{k}+1]
        % = (2c_{0},2c_{1},2c_{1}, \cdots, 2c_{l}, 2c_{l}).
        = ((2l)^{2r'_{2l}+1}, (2l-1)^{2r'_{2l-1}},\cdots, 1^{2r'_{1}})
      \]
            %             with $k = c_{0}$ and $l = r_{1}$.
      Then
      \[
        W_{\lamckb} = W_{l} \times \underbrace{S_{2l}\times \cdots \times S_{2l}}_{2r'_{2l}\text{-terms}} \times \prod_{i<2l} \underbrace{S_{i}\times \cdots\times S_{i}}_{r'_{i}\text{-terms}}
      \]
      (Note that in the product, $r'_{i}=0$ if $i$ is odd.) The computation of
      $\cksigma_{\mathrm b} = J_{W_{\lamckb}}^{W_{\mathrm b}} \sgn$ is similar to that of the
      good parity for type $C$ with no bifurcating, one deduce that
      $J$-induction and $j$-induction gives the same result.
      \[
        \begin{split}
          \cksigma_{\mathrm b} &=
          \binom{0, 1+r_{l}, 2+r_{l-1}\cdots, l+r_{1}}{1+r_{l},2+r_{l-1}, \cdots, l+r_{1}}\\
          & = ([r_{1},r_{2},\cdots, r_{l}],[r_{1}+1,r_{2}+1,\cdots,r_{l}+1])\\
        \end{split}
      \]
      with $r_{i} = \half\bfrr_{2i-1}(\ckcO_{\mathrm b}) = \half\bfrr_{2i}(\ckcO_{\mathrm b})$.
      Now
      \[
        \sigma_{\mathrm b} = ((r_{1}+1,r_{2}+1,\cdots,r_{l}+1), (r_{1},r_{2},\cdots, r_{l})) = j_{S_{\cO'_{\mathrm b}}}^{W_{\mathrm b}}\sgn
      \]
      where
      $\cO'_{\mathrm b}=(\bfrr_{2}(\ckcO_{\mathrm b}),\bfrr_{4}(\ckcO_{\mathrm b}),\cdots, \bfrr_{2l}(\ckcO_{\mathrm b}))$.
      Under the Springer correspondence of type $B$, it corresponds to
      $\Ind_{\GL_{\mathrm b}}^{\SO(2b+1)}\cO'_{\mathrm b} = \cO'_{\mathrm b}\cuprow \cO'_{\mathrm b}\cuprow (1)$.

      % \[
      %   \begin{split}
      %     \cksigma_{\mathrm b} &:= \sigma_{\mathrm b}\otimes \sgn = j_{W_{\lamckb}}^{W_{\mathrm b}} \sgn \\
      %     & = %\dagger_{2c_{l}}\cdots \dagger_{2c_{1}}
      %     \sigma_{\mathrm b}\otimes \sgn = j_{W_{\lamckb}}^{W_{\mathrm b}} \sgn\otimes
      %     \binom{0, 1, \cdots, c_{0}}{1, \cdots, c_{0}}\\
      %     & =
      %     \binom{0, 1+r_{k}, 2+r_{k-1}\cdots, c_{0}+r_{1}}{1+r_{k},2+r_{k-1}, \cdots, c_{0}+r_{1}}\\
      %     & = ([r_{1},r_{2},\cdots, r_{k}],[r_{1}+1,r_{2}+1,\cdots,r_{k}+1])\\
      %     &= ((c_{1},c_{2},\cdots, c_{k}),(c_{0},c_{1}, \cdots, c_{l}))\\
      %   \end{split}
      % \]


      % We take the convention that $\dagger \cO = [r_{i}+1]$. By abuse of
      % notation, let $\dagger_{n} \sigma$ denote the
      % $j_{S_{n} \times W_{\abs{\sigma}}}^{W_{n+\abs{\sigma}}} \sgn\otimes \sigma$.
      % We can write
      % \[
      %   \ckcO_{\mathrm b} = [2r_{1}+1, 2r_{1}+1, \cdots, 2r_{k}+1,2r_{k}+1] = (2c_{0},2c_{1},2c_{1}, \cdots, 2c_{l}, 2c_{l})
      % \]
      % with $k = c_{0}$ and $l = r_{1}$.

      % \[
      %   \begin{split}
      %     W_{\lamckb} &= W_{c_{0}} \times S_{2c_{1}} \times S_{2c_{2}}\times \cdots \times S_{2c_{l}}\\
      %     \cksigma_{\mathrm b} &:= \sigma_{\mathrm b}\otimes \sgn = j_{W_{\lamckb}}^{W_{\mathrm b}} \sgn \\
      %     & = \dagger_{2c_{l}}\cdots \dagger_{2c_{1}}
      %     \binom{0, 1, \cdots, c_{0}}{1, \cdots, c_{0}}\\
      %     & =
      %     \binom{0, 1+r_{k}, 2+r_{k-1}\cdots, c_{0}+r_{1}}{1+r_{k},2+r_{k-1}, \cdots, c_{0}+r_{1}}\\
      %     & = ([r_{1},r_{2},\cdots, r_{k}],[r_{1}+1,r_{2}+1,\cdots,r_{k}+1])\\
      %     &= ((c_{1},c_{2},\cdots, c_{k}),(c_{0},c_{1}, \cdots, c_{l}))\\
      %   \end{split}
      % \]

      % Therefore
      % \[
      %   \begin{split}
      %     \sigma_{\mathrm b} &= \cksigma_{\mathrm b}\otimes \sgn = ((r_{1}+1,r_{2}+1,\cdots,r_{k}+1),(r_{1},r_{2},\cdots, r_{k})) \\
      %     & = j_{S_{2r_{1}+1}\times \cdots S_{2r_{k}+1}}^{W_{\mathrm b}} \sgn\\
      %     & = j_{S_{\mathrm b}}^{W_{\mathrm b}} (2r_{1}+1, 2r_{2}+1, \cdots, 2r_{k}+1)
      %   \end{split}
      % \]
      % which corresponds to the orbit
      % \[
      %   \cO_{\mathrm b} = (2r_{1}+1, 2r_{1}+1,2r_{2}+1, 2r_{2}+1, \cdots,2r_{k}+1, 2r_{k}+1 ) = \ckcO_{\mathrm b}^{t}.
      % \]
      % (Note that $\cO'_{\mathrm b} = (2r_{1}+1,2r_{2}+1, \cdots, 2r_{k}+1)$ which
      % corresponds to $j_{W_{L_{\mathrm b}}}^{S_{\mathrm b}}\sgn$ and
      % $\ind_{L}^{G} \cO'_{\mathrm b} = \cO_{\mathrm b}$. ) This implies the unique special
      % representation is
      % \[
      %   \sigma_{\mathrm b} = (j_{W_{\lamckb}}^{W_{\mathrm b}}\sgn), \quad \text{where
      % } W_{L,b} = \prod_{i=1}^{k} S_{2r_{i}+1}.
      % \]
      % The $J$-induction is calculated by \cite{Lu}*{(4.5.4)}. It is easy to
      % see that in our case $J_{W_{\lamckb}}^{W_{\mathrm b}} \sgn$ consists of the
      % single special representation by induction.


      Now we consider the good parity part, where each row of $\ckcO_{\mathrm g}$ has
      even length.

      Assume $r'_{i} := \half(\bfrr_{i}(\ckcO_{\mathrm g})-\bfrr_{i-1}(\ckcO_{\mathrm g}))$ and
      so
      \[
        \ckcO_{\mathrm g} % = [2r_{1}+1, 2r_{1}+1, \cdots, 2r_{k}+1,2r_{k}+1]
        % = (2c_{0},2c_{1},2c_{1}, \cdots, 2c_{l}, 2c_{l}).
        = ((2l+1)^{2r'_{2l+1}}, (2l)^{2r'_{2l}},\cdots, 1^{2r'_{1}})
      \]
      % Consider
      % \[
      %   \cO_{\mathrm g} = [2r_{1},2r_{2}, \cdots, 2r_{2k-1},2r_{2k}] = (C_{1},C_{1}, C_{2},C_{2},\cdots, C_{l}, C_{l}).
      % \]
      with $l =\min\set{i|\bfrr_{2i+2}(\ckcO_{\mathrm g}) = 0}$.

      Then
      \[
        W_{\lamckg} = \times \prod_{i\leq 2l+1} \underbrace{S_{i}\times \cdots\times S_{i}}_{r'_{i}\text{-terms}}
      \]

      Note that the trivial representation of the trivial group has symbol
      \[
        \binom{0,1, 2, \cdots, l\phantom{-1}}{0,1, \cdots, l-1}.
      \]


      Induce to include all $2l+1$-length columns yields
      \[
        \binom{r'_{2l+1}+0,r'_{2l+1}+1,r'_{2l+1}+2,\cdots, r'_{2l+1}+l\phantom{-1}}{ r'_{2l+1}+0,r'_{2l+1}+1, \cdots, r'_{2l+1}+l-1}.
      \]
      Now move the shorter columns. When odd columns $(2i+1)^{2r'_{2i+1}}$
      occurs, it adds $(i+1)^{r'_{2i+1}}$ columns on the left and
      $(i)^{r'_{2i+1}}$ on the right. When even columns $(2i)^{r'_{2i}}$ occur,
      the bifurcation happens: one can
      \begin{itemize}
        \item attach columns $(i)^{r'_{2i}}$ on the left and columns
              $(i)^{r'_{2i}}$ on the right, which corresponds to
              $(2i,2i+1)\neq \wp$, or
        \item attach columns $(i-1)^{r'_{2i}}$ on the left and columns
              $(i+1)^{r'_{2i}}$ on the right, which corresponds to
              $(2i,2i+1)\in \wp$.
      \end{itemize}


      Therefore,
      \[
        \begin{array}{ccc}
          \bF_{2}(\CPP(\ckcO_{\mathrm g}))&\longrightarrow
          & J_{W_{\lamckg}}^{W_{\mathrm g}} \sgn \\
          \wp&\mapsto&    (\cktau_{L},\cktau_{R}) =:\cktau_{\wp}
        \end{array}
      \]
      where
      % \[
      %   \bfrr_{2l+1}(\cktau_{L}) = r'_{2l+1} = \half \bfrr_{2l+1}(\ckcO_{\mathrm g})
      % \]
      \[
        \bfrr_{1}(\cktau_{L}) = \sum_{i} r'_{i} = \half \bfrr_{1}(\ckcO_{\mathrm g})
      \]
      and, if $(2i,2i+1)\notin \wp$,
      \[
        \begin{split}
          \bfrr_{i+1}(\cktau_{L}) & = \sum_{l\geq 2i+1} r'_{l}
          = \half\bfrr_{2i+1}(\ckcO)\\
          \bfrr_{i}(\cktau_{R}) & = \sum_{l\geq 2i} r'_{l} = \half\bfrr_{2i}(\ckcO)
        \end{split}
      \]
      if $(2i,2i+1)\in \wp$,
      \[
        \begin{split}
          \bfrr_{i+1}(\cktau_{L}) & = \sum_{l\geq 2i} r'_{l}
          = \half\bfrr_{2i}(\ckcO)\\
          \bfrr_{i}(\cktau_{R}) & = \sum_{l\geq 2i+1} r'_{l} = \half\bfrr_{2i+1}(\ckcO)
        \end{split}
      \]

      Some remarks on the BV-dual. The calculation of $\cO_{\mathrm g}$ from
      $\tau_{\emptyset}$ can be reduced to the case of quasi-distinguished
      orbits (other case are deduced from this by parabolic induction,
      corresponds to attach two even columns for the balanced pairs). Compare
      Sommer's description of Springer correspondence with ours, we deduce that
      \[
        \cO_{\mathrm g} = (\bfrr_{1}(\ckcO_{1})+1,\bfrr_{2}(\ckcO_{2})-1,\bfrr_{3}(\ckcO_{3})+1, \cdots, \bfrr_{2l}(\ckcO_{2l})-1,\bfrr_{2l+1}(\ckcO_{2l+1})+1)
      \]
      The rest parts are similar to that of type $D$.
    }


    We give the main steps of the proof for the case when $\star = \wtC$.


    We first consider the good parity part $\ckcO_{\mathrm g}$, where each row has
    even length.

    Set $r'_{i} := \half(\bfrr_{i}(\ckcO_{\mathrm g})-\bfrr_{i-1}(\ckcO_{\mathrm g}))$,
    $l =\min\set{i|\bfrr_{2i+1}(\ckcO_{\mathrm g})=0}$, and write
    \[
      \ckcO_{\mathrm g} % = [2r_{1}+1, 2r_{1}+1, \cdots, 2r_{k}+1,2r_{k}+1]
      % = (2c_{0},2c_{1},2c_{1}, \cdots, 2c_{l}, 2c_{l}).
     % = ((2l)^{2r'_{2l}}, (2l-1)^{2r'_{2l-1}},\cdots, 1^{2r'_{1}})
      = (\underbrace{2l,\cdots, 2l}_{2r'_{2l}}, \underbrace{2l-1,\cdots, 2l-1}_{2r'_{2l-1}},\cdots,
      \underbrace{1,\cdots, 1}_{2r'_{1}})
    \]
    where $i^{r'}$ denotes $r'$-copies of length $i$ columns.
    % Consider
    % \[
    %   \cO_{\mathrm g} = [2r_{1},2r_{2}, \cdots, 2r_{2k-1},2r_{2k}] = (C_{1},C_{1}, C_{2},C_{2},\cdots, C_{l}, C_{l}).
    % \]
    The Weyl group $W_{\mathrm g}$ of good parity is $\sfW'_{n_{\mathrm g}}$ with
    $n_{\mathrm g} = \half\abs{\ckcO_{\mathrm g}}$. For $1\leq k\leq l$, let
    \[
      % S_{r,s} = \prod_{i=r}^{s} \underbrace{\sfS_{i}\times \cdots\times \sfS_{i}}_{r'_{i}\text{-terms}}
      \vec{S}_{i} = \underbrace{\sfS_{i}\times \cdots\times \sfS_{i}}_{r'_{i}\text{-times}} \AND n_{k} = \sum_{i=k}^{2l} i\cdot r'_{i}.
      % \AND n_{r,s} = \sum_{i=r}^{s} i\cdot r'_{i}.
    \]

    Then $W_{\lamckg}=\prod_{i=1}^{l} \vec{S}_{i}$ and
    \[
      \begin{split}
        \ckLV_{\ckcO_{\mathrm g}}& :=J_{W_{\lamckg}}^{\Wg}\sgn\\
        & = J_{\vec{S}_{1}\times \sfW'_{n_{2}}}^{\sfW'_{n_{1}}} \Big(\sgn \otimes J_{\vec{S}_{2}\times \sfW'_{n_{3}}}^{\sfW'_{n_{2}}}\Big(\sgn
        \otimes \cdots\big(J_{\vec{S}_{l}} \sgn\big)\cdots \Big)\Big) \\
      \end{split}
    \]
    Applying \cite{Lu}*{(4.6.2)} inductively, we see that the operation
    $J_{\vec{S}_{i}\times \sfW'_{n_{i+1}}}^{\sfW'_{n_{i}}}(\sgn \otimes \underline{\ \ \ })$ yields a multiplicity-free representation and
    doubles (resp. keeps) the number of irreducible constituents if $i$ is odd (resp. even). Informally we will say that a bifurcation occurs when attaching an odd length column.
    Denote by $\ckLC_{\ckcO_{\mathrm g}}$ the multiset of irreducible constituents of $\ckLV_{\ckcO_{\mathrm g}}$.

    First assume $\CPPs(\ckcO_{\mathrm g}) = \emptyset$. Define
    \[
      \tA'(\ckcO):= \bZ_{2}[\emptyset]
    \]
    to be the trivial group.
    Then $\ckLV_{\ckcO_{\mathrm g}}$ is
    irreducible (with the corresponding bipartition marked by the label $I$).  Hence, $\LC_{\ckcO_{\mathrm g}}$
    %$\LC_{\ckcO)}$
    and $\tA'(\ckcO)$ can be obviously identified.


    Now assume $\CPPs(\ckcO_{\mathrm g})\neq \emptyset$. Then the two parts of the
    bipartition of an irreducible constituent after each operation
    $J_{\vec{S}_{i}\times \sfW'_{n_{i+1}}}^{\sfW'_{n_{i}}}(\sgn \otimes \underline{\ \ \ })$ are different. Let
    \[i_{0}:= \min\Set{i| (2i-1,2i)\in \CPPs(\ckcO_{\mathrm g})}.\]
    % $(2i_{0}-1,2i_{0})$ be the element in such that $i_{0}$ is minimal.
    Then we will have a bijection
    \[
      \tA'(\ckcO):= \set{\wp\in \bZ_{2}[\CPPs(\ckcO_{\mathrm g})]|(2i_{0}-1,2i_{0})\notin \wp} \longrightarrow \ckLC_{\ckcO_{\mathrm g}}
    \]
    which records the bifurcation when attaching odd length columns. More precisely, by Lusztig's formula of $J$-induction, a bijection is given by sending
      $\wp$ to $\cktau_{\wp}:= (\cktau_{L},\cktau_{R})$, where
    \[
      (\bfrr_{i}(\cktau_{L}),\bfrr_{i}(\cktau_{R})) := \begin{cases} (\half\bfrr_{2i}(\ckcO_{\mathrm g}),\half\bfrr_{2i-1}(\ckcO_{\mathrm g}))
        & \text{if } (2i-1,2i)\notin \wp ,\\
        (\half\bfrr_{2i-1}(\ckcO_{\mathrm g}),\half\bfrr_{2i}(\ckcO_{\mathrm g}))
        & \text{if } (2i-1,2i)\in \wp .\\
      \end{cases}
    \]
    By interchanging the two parts of a bipartition belonging to $\ckLC_{\ckcO_{\mathrm g}}$, we thus obtain a bijection of $\tA'(\ckcO)$ with $\LC_{\ckcO_{\mathrm g}}$, which sends $\wp$ to $\tau_{\wp}$.


    \trivial[h]{ Note that the trivial representation of the trivial group is
      represented by the symbol
      \[
        \binom{0,1, 2, \cdots, l-1}{0,1,2, \cdots, l-1}_{I}.
      \]
      % Induce to include all $2l$-length columns yields
      % \[
      %   \binom{r'_{2l+1}+0,r'_{2l+1}+1,r'_{2l+1}+2,\cdots, r'_{2l+1}+l\phantom{-1}}{ r'_{2l+1}+0,r'_{2l+1}+1, \cdots, r'_{2l+1}+l-1}.
      % \]

      Now move the shorter columns. When even columns $(2i)^{2r'_{2i}}$
      occurs, it adds $(i)^{r'_{2i}}$ columns on the left and $(i)^{r'_{2i}}$ on
      the right. When odd columns $(2i-1)^{r'_{2i-1}}$ occur, the bifurcation
      happens: one can
      \begin{itemize}
        \item attach columns $(i-1)^{r'_{2i-1}}$ on the left and columns
              $(i)^{r'_{2i-1}}$ on the right, which corresponds to
              $(2i-1,2i)\neq \wp$, or
        \item attach columns $(i)^{r'_{2i-1}}$ on the left and columns
              $(i-1)^{r'_{2i-1}}$ on the right, which corresponds to
              $(2i-1,2i)\in \wp$.
      \end{itemize}
      Note that when we first encounter the longest odd column, we make the
      choice that the size of left part is larger than that of the right part.
      Now If $(2i-1,2i)\notin \wp$,
      \[
        \begin{split}
          \bfrr_{i}(\cktau_{L}) & = \sum_{l\geq 2i} r'_{l}
          = \half\bfrr_{2i}(\ckcO_{\mathrm g})\\
          \bfrr_{i}(\cktau_{R}) & = \sum_{l\geq 2i-1} r'_{l} = \half\bfrr_{2i-1}(\ckcO_{\mathrm g})
        \end{split}
      \]
      if $(2i-1,2i)\in \wp$,
      \[
        \begin{split}
          \bfrr_{i}(\cktau_{L}) & = \sum_{l\geq 2i-1} r'_{l}
          = \half\bfrr_{2i-1}(\ckcO_{\mathrm g})\\
          \bfrr_{i}(\cktau_{R}) & = \sum_{l\geq 2i} r'_{l} = \half\bfrr_{2i}(\ckcO_{\mathrm g})
        \end{split}
      \]
    }


    Now we consider the bad parity part, where each row has
    odd length.
    %For a partition $\cO$, we set
    %\[
    % \sfS_{\cO} := \prod_{i\in \bN^{+}}\sfS_{\bfcc_{i}(\cO)}
    %\]
    %so that $j_{\sfS_{\cO}}^{\sfS_{\abs{\cO}}}\sgn = \cO$.



    Suppose $\ckcOpb$ is nonempty.
    % such that
    % \[
    %   \ckcO_{\mathrm b} = (c_{0},c_{1}, c_{1}, c_{2},c_{2}, \cdots, c_{k}, c_{k})
    % \]
    % where
    Let $2k+1=\bfrr_{1}(\ckcOpb)$ and $c_{i} = \bfcc_{2i+1}(\ckcOpb)$.
    Now
    \[
      W_{\lamckb} = \sfW_{c_{0}} \times \sfS_{2c_{1}} \times \sfS_{2c_{2}}\times \cdots \times \sfS_{2c_{k}}
    \]
    and %$J_{W_{\lamckb}}^{W_{\mathrm b}}\sgn$ is irreducible by
    \[
      \begin{split}
        \cktau_{\mathrm b} &:= J_{W_{\lamckb}}^{\Wb} \sgn
        = \big((c_{1},c_{2},\cdots, c_{k}),(c_{0},c_{1}, \cdots, c_{l})\big)\\
        & = \big([\half(\bfrr_{1}(\ckcOpb)-1),\half(\bfrr_{2}(\ckcOpb)-1),\cdots, \half(\bfrr_{c_{0}}(\ckcOpb)-1)],\\
        & \hspace{2em} [\half(\bfrr_{1}(\ckcOpb)+1),\half(\bfrr_{2}(\ckcOpb)+1),\cdots, \half(\bfrr_{c_{0}}(\ckcOpb)+1)]\big)
      \end{split}
    \]
    is irreducible by \cite{Lu}*{(4.5.4)}. Tensoring with sign yields the
    formula of $\tau_{\mathrm b}$. Moreover, by the fake degree formula (see
    \cite{Carter}*{p~376}), we have
    \[
      \tau_{\mathrm b} = \cktau_{\mathrm b}\otimes \sgn = j_{\sfS_{\cO'_{\mathrm b}}}^{\sfW_{n_{\mathrm b}}} \sgn,
    \]
    where
    $ \cOpb:= (\ckcOpb)^{t}$
    % $:=(\bfrr_{1}(\ckcO),\bfrr_{4}(\ckcO),\cdots , \bfrr_{2c_{0}}(\ckcO))$,
    and $\sfS_{\cOpb} := \prod_{1\leq i\leq c_{0}}\sfS_{\bfrr_{i}(\ckcOpb)}$. The proof for the first part is now complete.

    \medskip \def\ckfll{\check{\fll}}

    Next we give the main steps of the proof for the second part, when $\star = \wtC$.
    %the proof of \eqref{eq:dBV.W}.

    Recall the definition of the metaplectic Barbasch-Vogan dual in
    \cite{BMSZ1}. A key property is that this duality map commutes with parabolic induction: Suppose
    $\ckfll\subset \ckfgg$ is a parabolic subalgebra of $\ckfgg$ and
    $\fll$ is the corresponding parabolic subalgebra in $\fgg$, then
    \begin{equation*}%\label{eq:inddBV}
      \dBV(\ckcO) =  \Ind_{\fll}^{\fgg}(\dBV(\ckcO_{\ckfll}))
    \end{equation*}
    for each nilpotent orbit $\ckcO$ in $\ckfgg$ such that
    $\ckcO_{\ckfll}:=\ckcO\cap \ckfll\neq \emptyset$. This is clear by reducing
    to the type $B$ case, as in \cite{BMSZ1}*{Proposition~3.8}. By removing pairs
    of rows with the same lengths in $\ckcO$, we are reduced to check the equality in \eqref{eq:dBV.W}
    in the case when $\ckcOpb=\emptyset$ and
    $\bfrr_{2i-1}(\ckcO_{\mathrm g})>\bfrr_{2i}(\ckcO_{\mathrm g})$ for all $i$ such that
    $i\leq \bfcc_{1}(\ckcO_{\mathrm g})$. In this case, both sides of \eqref{eq:dBV.W}
    can be easily computed and are equal to
    \[
      (\bfrr_{1}(\ckcOg)-1, \bfrr_{2}(\ckcOg)+1, \cdots,\bfrr_{2c-1}(\ckcOg)-1,\bfrr_{2c}(\ckcOg)+1),
    \]
    where $c = \min\set{i|\bfrr_{2i+1}(\ckcOg)=0}$. The general case follows
    from the aforementioned compatibility with parabolic induction.
    }
  \end{proof}



    % Now we compare the metaplectic dual defined in \cite{BMSZ1} with the Weyl
    % group representations.
    \trivial[h]{ Compare Sommer's description of Springer correspondence we
      deduce that the RHS is
      \[
        \cO_{\mathrm g} = (\bfrr_{1}(\ckcO_{1})-1,\bfrr_{2}(\ckcO_{2})+1,\bfrr_{3}(\ckcO_{3})+1, \cdots, \bfrr_{2l-1}(\ckcO_{2l-1})-1,\bfrr_{2l}(\ckcO_{2l})+1)
      \]
      The LHS is calculated by $((((\ckcO^{t})_{D})^{+})^{-})_{C}$. We write
      $R_{i}=\bfrr_{i}(\ckcO)=2r_{i}$. Now under our assumption,
      $R_{2i-1}>R_{2i}$, we have
      \[
        \begin{split}
          ((((\ckcO^{t})_{D})^{+})^{-})_{C} &=
          ((((R_{1},R_{2}, \cdots, R_{2l-1},R_{2l})_{D})^{+})^{-})_{C}\\
          &=((R_{1}-1,R_{2}, \cdots, R_{2l-1},R_{2l},1))_{C}\\
          &=(R_{1}-1,R_{2}+1, \cdots, R_{2l-1}-1,R_{2l}+1)\\
        \end{split}
      \]
      So the proof is done.

    }

    %At last, one can see that
    %$\dBV(\ckcO_{\mathrm b}\cuprow \ckcO_{\mathrm g}) = \ckcO_{\mathrm b}^{t} \cupcol \dBV(\ckcO_{\mathrm g})$
    %using \eqref{eq:inddBV}.


  % \begin{remark}
  %   When $\star=\wtC$, one can see that
  %   $\dBV(\ckcO_{\mathrm b}\cuprow \ckcO_{\mathrm g}) = \ckcO_{\mathrm b}^{t} \cupcol \dBV(\ckcO_{\mathrm g})$
  %   using \eqref{eq:inddBV}. \trivial[]{ This could be checked using the
  %   formula of Springer correspondence directly, see Sommer's. }
  % \end{remark}
%

Recall from \eqref{eq:Wbg} that $\Wg =\sfW'_{n_{\mathrm g}}$, when $\star \in \set{\wtC,D,D^{*}}$. Since the representation theory of $\sfW_{n}$ is more elementary than that of
$\sfW'_{n}$, we prefer to work with $\sfW_n$ instead of $\sfW_n'$ in some situations. For this reason, in all cases we also define
\begin{equation}\label{eq:ttauwp}
\wttau_{\wp} = (\imath_{\wp},\jmath_{\wp})\in \Irr(\sfW_{n_\mathrm g}), \qquad \wp\in {\mathrm A}(\ckcO).
\end{equation}
See \eqref{eq:tauwp} for the description of $\imath_{\wp}$ and $\jmath_{\wp}$.

For later use, we record the following lemma, which follows immediately from our explicit descriptions of $\tau_{\wp}$ and $\tau_{\mathrm b}$.

\begin{lem}\label{lem:WLcell}
Let $\wp\in {\mathrm A}(\ckcO)$. If $\star=\wtC$, then
  \[
    \Ind_{\sfW'_{n_{\mathrm g}}\times \sfW_{n_{\mathrm b}}}^{\sfW_{n_{\mathrm g}}\times \sfW_{n_{\mathrm b}}} \tau_{\wp}\otimes \tau_{\mathrm b} =
    \begin{cases}
       \wttau_{\emptyset}\otimes \tau_{\mathrm b}, &\quad  \text{if } n_{\mathrm g}=0; \\
      (\wttau_{\wp}\otimes \tau_{\mathrm b}) \oplus (\wttau_{\wp^{c}}\otimes \tau_{\mathrm b}),
      &\quad \text{otherwise}.
    \end{cases}
  \]
If $\star\in\set{D,D^{*}}$, then
  \[
    \Ind_{\sfW'_{n_{\mathrm g}}\times \sfW'_{n_{\mathrm b}}}^{\sfW_{n_{\mathrm g}}\times \sfW_{n_{\mathrm b}}} \tau_{\wp}\otimes \tau_{\mathrm b} \cong
    \begin{cases}
      \wttau_{\mathrm b}, & \quad \text{if } n_{\mathrm g}=0; \\
      (\wttau_{\wp}\otimes \wttau_{\mathrm b})  \oplus (\wttau_{\wp}^{s}\otimes \wttau_{\mathrm b}),
      &\quad \text{otherwise}.
    \end{cases}
  \]
  Here $\wttau_{\mathrm b} = \Ind_{\sfW'_{n_{\mathrm b}}}^{\sfW_{n_{\mathrm b}}}\tau_{\mathrm b}$ and $\wttau_{\wp}^{s}:= \wttau_{\wp}\otimes \varepsilon$ (recall the quadratic character $\varepsilon$ from \eqref{defep}).

  %\brsgn \neq \wttau_{\wp}$.

  % Then we have a bijection
  % \[
  %     \begin{array}{lccccccc}
  %       \tA(\ckcO)&=&\tA(\ckcO_{\mathrm g}) & \longrightarrow & \tLC(\ckcO_{\mathrm g})
  %       & \longrightarrow & \tLC(\ckcO)\\
  %                   &  &\wp & \mapsto & \wttau_{\wp} &
  %                                                    \mapsto & \tau_{\mathrm b}\otimes \wttau_{\wp}.
  %     \end{array}
  % \]

 \trivial[h]{
  Let
\[
  \tLV_{\ckcO}:= \Ind_{\sfW'_{n_{\mathrm g}}\times \sfW_{n_{\mathrm b}}}^{\sfW_{n_{\mathrm g}}\times \sfW_{n_{\mathrm b}}}\LV_{\ckcO}
\]
and $\tLC_{\ckcO}$ be the set of irreducible constituents of the (multiplicity free) $\sfW_{n_{\mathrm g}}\times \sfW_{n_{\mathrm b}}$-module $\tLV_{\ckcO}$.
Then we have a
bijection:
  \[
      \begin{array}{lccccccc}
        \tA(\ckcO)&\cong&\tA(\ckcO_{\mathrm g}) & \longrightarrow & \tLC(\ckcO_{\mathrm g})
        & \longrightarrow & \tLC_{\ckcO}\\
                  &  &\wp & \mapsto & \wttau_{\wp}
        & \mapsto & \wttau_{\wp}\otimes \tau_{\mathrm b}.
      \end{array}
  \]
  }
\end{lem}

\subsection{From coherent continuation representation to counting}




We have defined in \eqref{defpbp2222} the set $\PBPs(\ckcO)$ when $\ckcO$ has good parity. Similarly, we make the following definition in the bad parity case.
\begin{defn}
  Let $\PBPs(\ckcOb)$ be the set of all triples
  $\uptau = (\imath,\cP)\times(\jmath,\cQ)\times \star $ where $(\imath,\cP)$ and
  $(\jmath,\cQ)$ are painted partitions such that
  \begin{itemize}
    \item $(\imath,\jmath) = (\tau_{L,\mathrm b},\tau_{R,\mathrm b})$ (see \eqref{eq:taub});
    \item the image of $\cP$ is contained in
          \[
          \begin{cases}
            \set{\bullet, c,d},  & \text{if } \star\in \set{B,\wtC}; \\
            \set{\bullet, d},  & \text{if } \star\in \set{C,D};\\
            \set{\bullet},  & \text{if } \star\in \set{C^{*},D^{*}};\\
          \end{cases}
          \]
    \item the image of $\cQ$ is contained in

          \[
          \begin{cases}
            \set{\bullet, c},  & \text{if } \star\in \set{C,D};\\
            \set{\bullet},  & \text{if } \star\in \set{B,\wtC, C^{*},D^{*}}.\\
          \end{cases}
          \]
  \end{itemize}
\end{defn}

%We define the partition $\ckcOpb$ by $\ckcOb = 2\ckcOpb$. To be more precise, $\bfrr_{i}(\ckcO'_{\mathrm b}):= \bfrr_{2i}(\ckcO_{\mathrm b})$, for all $i\in \bN^{+}$. Let $\cO'_{\mathrm b}:= (\ckcO'_{\mathrm b})^{t}$.
  %\item $\cO_{\mathrm b}:= \cO'_{\mathrm b}\cupcol \cO'_{\mathrm b}$.
 %$\ckcOpb$ is a partition of $\half\abs{\ckcOb}$.

% Recall from \eqref{Gpb} the group
% \[
%   G'_{\mathrm b} := \begin{cases}
%     \GL_{n_{\mathrm b}}(\bR), & \text{if } \star \in \set{B,C,D}; \\
%     \widetilde{\GL}_{n_{\mathrm b}}(\bR), & \text{if } \star = \wtC; \\
%     \GL_{\frac{n_{\mathrm b}}{2}}(\bH), & \text{if } \star \in \set{C^{*},D^{*}}.\\
%   \end{cases}
% \]
%When $\star=\wtC$, let $\Unip_{\ckcOpb}(\Gpb)$ be the set of genuine unipotent representations of $\Gpb$ which is naturally identified with $\Unip_{\ckcOpb}(\GL(n_{\mathrm b},\bR))$.

% \begin{prop}\label{prop:BP.PP} In all cases,
% \[
%     \begin{split}
%       \sharp(\PBP_{\star}(\ckcO_{\mathrm b})) = \sharp(\PP_{\star '}(\ckcO'_{\mathrm b})) = \sharp(\Unip_{\ckcO'_{\mathrm b}}(G'_{\mathrm b})),
%     \end{split}
%   \]
% where
% \begin{equation}\label{def:star'}
% \star ':= \begin{cases}
%     A^{\bR}, & \text{if } \star \in \set{B,C,\wtC,D}; \\
%     A^{\bH}, & \text{if } \star \in \set{C^{*},D^{*}}.\\
%   \end{cases}
%   \end{equation}
% %is the set of painted partitions of type $A$ or $A^{\bH}$ attached to $\ckcO'_{\mathrm b}$.
% \end{prop}

% \begin{proof} Suppose that $\star \in \Set{C^{*},D^{*}}$. Then
% \[
%     \begin{split}
%       \sharp(\PBP_{\star}(\ckcO_{\mathrm b};\tau_{\mathrm b}))= \sharp(\PP_{A^{\bH}}(\ckcO'_{\mathrm b}))= 1.
%     \end{split}
% \]
%   Suppose that $\star \in \Set{B,C,\wtC,D}$.
%   %Recall the Young diagrams $\tau_{L,b}$ and $\tau_{R,b})$ from \eqref{eq:taub} and \eqref{eq:taub2}.
%    It is easy to see that we have a bijection
%   \[
%     \begin{array}{ccc}
%       \PBP_{\star}(\ckcO_{\mathrm b}) &  \rightarrow & \PP_{A^{\bR}}(\ckcO'_{\mathrm b}),\\
%       (\tau_{L,b},\cP)\times (\tau_{R,b},\cQ)\times & \mapsto & ((\ckcO'_{\mathrm b})^t,\cP'),
%     \end{array}
%   \]
%   where $\cP'$ is defined by the condition that
%   \[
%     \cP(\bfcc_{j}(\tau_{L,b}),j)=d \Longleftrightarrow \cP'(\bfrr_{j}(\ckcO'_{\mathrm b}),j)=d, \quad \textrm{ for all } j=1,2,\cdots, \bfcc_{1}(\ckcO'_{\mathrm b}).
%   \]
%   The last equality is in Theorems \ref{thm:mainR} and \ref{thm:mainH}.   \end{proof}


%   \trivial[h]{
%     % Let $\tau' = \ckcO'^{t}_{\mathrm b}$ and $\tau_{\mathrm b}=(\tau_{L,b}, \tau_{R,b})$. Here
%     % $\tau_{L,b}, \tau_{R,b}$.
%     Now the claim follows for the fact that the bottom rows in $\uptau_{L}$ can
%     be filled by $\bullet/c$ or $d$ and
%     \[
%       \bfcc_{i}(\tau_{L,b}) = \bfcc_{j}(\tau_{L,b}) \Leftrightarrow \bfcc_{i}(\cOpb) = \bfcc_{j}(\cOpb) \quad \textrm{ for all } i,j\in \bN^{+}.
%     \]
%   }




 
 We introduce some additional notation. For each bipartition $\tau$, let
\[
  \PBP_{\star}(\tau) := \Set{ \uptau \text{ is a painted bipartition }\mid  \star_{\uptau}=\star, (\imath_{\uptau},\jmath_{\uptau}) = \tau}
  % \uptau=(\imath, \cP)\times (\jmath,\cP)\times \alpha|}
\]
and
\[
  \PBP_{G}(\tau) := \Set{\uptau \text{ is a painted bipartition }\mid G_{\uptau}=G, (\imath_{\uptau},\jmath_{\uptau}) = \tau}.
  % \uptau=(\imath, \cP)\times (\jmath,\cP)\times \alpha|}
\]
Put
\[
  \tPBP_{\star}(\ckcO) := %\bigsqcup_{\tau\in\LC(\ckcOg)}\PBP_{\star}(\tau).
  \bigsqcup_{\wp \subseteq \CPP(\ckcO)}\PBP_{\star}(\wttau_{\wp})
\]
and
\[
  \tPBP_{G}(\check \CO):=   \bigsqcup_{\wp \subseteq \CPP(\ckcO)}\PBP_{G}(\wttau_{\wp}),
 \] 
where $\wttau_{\wp} := (\imath_{\wp},\jmath_{\wp})$ (see \eqref{eq:ttauwp}). 


Recall the group $G_\mathrm g$ from \eqref{gg00} (with $l=\nnb$).  Its Langlands dual group is identified with $\check G_\mathrm g$. 




\section{Combinatorics of painted bipartitions}


In this section, we assume that $\star \in \{B,C,\wtC,C^{*},D,D^{*}\}$, and $\ckcO = \ckcOg$, namely $\ckcO $ has $\star$-good parity.
Recall the set  $\CPPs(\ckcO)$ of primitive $\star$-pairs in $\ckcO$. For each subset $\wp$ of $\CPPs(\ckcO)$, we have defined a bipartition $\tau_{\wp}=(\imath_{\wp},\jmath_{\wp})$ in \Cref{sec:LCBCD}.

In this section, we study the set
\[
\PBPGOP := \set{\uptau \text{ is a painted bi
    partition}| G_{\uptau}=G,  (\imath_{\uptau},\jmath_{\uptau}) = \tau_{\wp}}
\]
for each $\wp\in \CPP(\ckcO)$


% The following two combinatorial results follow by induction on $\mathbf c_1(\check \CO)$. As the proof is quite tedious, we omit the details.
%
\subsection{Non-existence of painted bipartition in quaternionic group cases}
\begin{prop} \label{prop:PBP1} Suppose that $\star\in \set{C^{*}, D^{*}}$. Then
\[
    \PBP_{G}(\tau_{\wp}) = \emptyset, \quad \text{for all nonempty $\wp\subset \CPPs(\ckcO)$.}
  \]
 Consequently,
     \[
     \sharp(\tPBP_{G}(\ckcO)) = \sharp(\PBP_{G}(\ckcO)).
  \]
\end{prop}

\begin{proof}
Suppose that
  $\emptyset\neq \wp\subset \CPPs(\ckcO)$, and there was an element $\uptau = (\imath_{\wp}, \cP)\times (\jmath_{\wp},\cQ)\times \star\in \PBP_{G}(\tau_{\wp})$.

   First assume that  $\star = C^{*}$.  Pick an element $(2i-1, 2i)\in \wp$. Then we have that
  \begin{equation}\label{eq:res.C*}
    \bfcc_{i}(\imath_{\wp}) = \half(\bfrr_{2i-1}(\ckcO)+1)>
    \half(\bfrr_{2i}(\ckcO)-1) = \bfcc_{i}(\jmath_{\wp}).
      \end{equation}
   By the requirements of a painted bipartition, we also have that
  \begin{eqnarray*}
    \bfcc_{i}(\imath_{\wp})& = &\sharp\set{j\in \BN^+ \mid (i,j)\in \BOX{\imath_\wp}, \, \cP(i,j)=\bullet} \\
    &= &\sharp\set{j\in \BN^+\mid  (i,j)\in \BOX{\jmath_\wp}, \, \cQ(i,j)=\bullet} \\
    &\leq & \bfcc_{i}(\jmath_{\wp}).
  \end{eqnarray*}
 This contradicts \eqref{eq:res.C*} and therefore  $\PBP_{G}(\tau_{\wp})= \emptyset$.



  Now we assume that $\star = D^{*}$. Pick an element  $(2i, 2i+1)\in \wp$.
  Then we have that
  \begin{equation}\label{eq:res.D*}
    \bfcc_{i+1}(\imath_{\wp}) = \half(\bfrr_{2i}(\ckcO)+1)>
    \half(\bfrr_{2i+1}(\ckcO)-1) = \bfcc_{i}(\jmath_{\wp}).  \end{equation}
      By the requirements of a painted bipartition, we also have that
  \begin{eqnarray*}
  \bfcc_{i+1}(\imath_{\wp})& \leq & \sharp\set{j\in \BN^+ \mid (i,j)\in \BOX{\imath_\wp}, \,  \cP(i,j)=\bullet} \\
  &=&\sharp\set{j\in \BN^+ \mid (i,j)\in \BOX{\jmath_\wp}, \,  \cQ(i,j)=\bullet} \\
  &\leq & \bfcc_{i}(\jmath_{\wp}).  \end{eqnarray*}
 This contradicts \eqref{eq:res.D*} and therefore  $\PBP_{G}(\tau_{\wp})= \emptyset$.
\end{proof}



%The purpose of this section is to prove the   following combinatorial result.


%We shall deal with the two quaternionic cases first, which are simple. When $\star \in \set{B, C, \wtC, D}$, the proof of the main statement of the above proposition involves an elaborate reduction argument (by removing elements from $\wp$ one-by-one), and will be handled separately in \Cref{lem:down} below.
%[Proof of {\Cref{prop:PBP}} the quaternionic case]



Because of the following vanishing proposition, we will only need to consider
quasi-distinguished nilpotent orbits $\ckcO$ when $\star\in \set{C^*, D^*}$.

\begin{prop}\label{prop:CD*}
  Suppose that $\star\in \set{C^*, D^*}$. If the set $\PBP_\star(\ckcO)$ is nonempty, then $\check \CO$ is quasi-distinguished.
\end{prop}
\begin{proof}
  Suppose that $\tau=(\imath,\cP)\times(\jmath,\cQ)\times \alpha \in  \mathrm{PBP}_\star(\check \CO)$. If  $\star=C^*$, then  the definition of painted bipartitions implies that
 \[
 \bfcc_i(\imath)\leq \bfcc_i(\jmath) \qquad \textrm{for all } i=1,2,3, \cdots.
 \]
This forces that $\check \CO$ is quasi-distinguished.

 If  $\star=D^*$, then  the definition of painted bipartitions implies that
 \[
 \bfcc_{i+1}(\imath)\leq \bfcc_i(\jmath) \qquad \textrm{for all } i=1,2,3, \cdots.
 \]
This  also forces that   $\check \CO$ is quasi-distinguished.
 \end{proof}

%\subsection{Descent of painted bi-partition}
The following proposition is clear by \Cref{lem:gf.C*}.
\begin{prop}\label{prop:count.CD*}
  Suppose that $\star\in \set{C^*, D^*}$ and $\ckcO$ is quasi-distinguished.
  Then $\# \PBP_{G}(\ckcO)$ equals to the number of real nilpotent orbits of $G$
  whose complexification is $\dBV(\ckcO)$.
\end{prop}

\subsection{Counting in type $B,C,\wtC,D$}


%The following combinatorial result follows by induction on $\mathbf c_1(\check \CO)$.
In the rest of the section, we will establish \Cref{lem:gf.BD} and \Cref{lem:gf.C} which
imply the following counting result.

\begin{prop} \label{prop:PBP2} Suppose that   $\star\in \set{B,C,\wtC,D}$. Then
  \[
    \sharp(\PBP_{G}(\tau_{\wp})) = \sharp(\PBP_{G}(\tau_{\emptyset})), \quad \textrm{for all } \wp \subset \CPPs(\ckcO).
  \]
 Consequently,
     \[
     \sharp(\tPBP_{G}(\ckcO)) = 2^{\sharp(\CPPs(\ckcO))}\cdot \sharp(\PBP_{G}(\ckcO)).
  \]
\end{prop}



 \subsection{Shape shifting in type $C$ and $\wtC$}
 In this subsection, we assume $\star \in \set{C,\wtC}$ and $\ckcO$ is an orbit
 with good parity.



 Moreover, we assume that $(1,2)$ is a primitive pair, i.e.
 $(1,2)\in \PP(\ckcO)$. We have the following decomposition of the power set
 $A(\ckcO)$ of $\PP(\ckcO)$
 \[
   A(\ckcO) = \Ass(\ckcO) \sqcup \Ans(\ckcO)
 \]
 where
 \[
 \Ass(\ckcO): = \set{\wp | (1,2)\notin \wp}, \AND
 \Ans(\ckcO): = \set{\wp | (1,2)\in \wp}.
 \]
 For each $\wp\in A(\ckcO)$, we
 define $\wp_{\uparrow}:= \wp\cup \set{(1,2)}$. Clearly, the map
 $\wp \mapsto \wp_{\uparrow}$ defines bijection from $\Ass(\ckcO)$ to
 $\Ans(\ckcO)$.
 % For each $\wp\in A(\ckcO)$, we define
 % $\wp_{\leftrightarrow}:= (\wp-\set{(1,2)})\cup (\set{(1,2)} - \wp)$. Clearly,
 % the map $\wp \mapsto \wp_{\leftrightarrow}$ defines bijection between
 % $\Ass(\ckcO)$ and $\Ans(\ckcO)$.
 %
 %
 Let $\wp\in \Ass(\ckcO)$. We have
  \[
    \begin{split}
      (\bfcc_{1}(\imath_{\wpu}), \bfcc_{t}(\jmath_{\wpu})) &=
      \begin{cases}
        (\bfcc_{1}(\jmath_{\wpd})+1, \bfcc_{t}(\imath_{\wpd})-1) & \text{when
        } \star = C\\
        (\bfcc_{1}(\jmath_{\wpd}), \bfcc_{t}(\imath_{\wpd}))& \text{when
        } \star = \wtC\\
      \end{cases}
      \\
     %  &= (b_{2},b_{1}),  \\
     % &= (b_{1}-1,b_{2}+1),\AND\\
      % (\bfcc_{t}(\imath_{\PPm}), \bfcc_{t}(\jmath_{\PPm})) &= (b_{2},b_{1}),  \\
      % (\bfcc_{t}(\imath_{\wp}), \bfcc_{t}(\jmath_{\wp})) &= (b_{1}-1,b_{2}+1),\AND\\
      \AND
      (\bfcc_{i}(\imath_{\wpu}),\bfcc_{i}(\jmath_{\wpu})) &=(\bfcc_{i}(\imath_{\wpd}),\bfcc_{i}(\jmath_{\wpd}))\quad \text{for $i\neq 1$}.
    \end{split}
  \]

 For
 $\uptau := (\imath_{\wp},\cP_{\uptau})\times (\jmath_{\wp},\cQ_{\uptau})\times \alpha \in \PBPOP $,
 we
 define
 \[
   \uptauu:= (\imath_{\wpu}, \cP_{\uptauu})\times (\jmath_{\wpu},\cQ_{\uptauu})\times \alpha
 \]
 by the following recipe.

 {\bfseries Suppose $\star = C$.}

 \newcommand{\localtextbulletone}{\textcolor{black}{\raisebox{.45ex}{\rule{.6ex}{.6ex}}}}

  For all $(i,j)\in \BOX{\jmath_{\wpu}}$, we define $\cP_{\uptauu}(i,j)$ case by
  case:
 \begin{enumerate}[label=(\alph*)]
   \item Suppose
   $\cP_{\uptaud}(\bfcc_{1}(\imath_{\wpd}),1)\neq \bullet$.
   \begin{enumerate}[label={\localtextbulletone}]
     \item If $\bfcc_{1}(\imath_{\wpd})\geq 2$ and
     $\cP_{\uptau}(\bfcc_{1}(\imath_{\wpd})-1,1) = c$,
     we define
     \[
       \cP_{\uptauu}(i,j) := \begin{cases}
         r ,& \text{if $j=1$ and $\bfcc_{1}(\imath_{\wpd})-1
           \leq i \leq \bfcc_{1}(\imath_{\wpu})-2$},\\
         c ,& \text{if $(i,j)=(\bfcc_{t}(\imath_{\wpu})-1,1)$},\\
         d ,& \text{if $(i,j)=(\bfcc_{t}(\imath_{\wpu}),1)$},\\
         \cP_{\uptaud}(i,j) ,&\text{otherwise};
       \end{cases}
     \]
     \item Otherwise, we define
     \[
       \cP_{\uptauu}(i,j) := \begin{cases}
         r ,& \text{if $j=1$ and $\bfcc_{1}(\imath_{\wpu})
           \leq i \leq \bfcc_{t}(\imath_{\wpu})-1$},\\
         \cP_{\uptaud}(\bfcc_{t}(\imath_{\wpd}),t) ,&
         \text{if $(i,j)=(\bfcc_{t}(\imath_{\wpu}),t)$},\\
         \cP_{\uptaud}(i,j) ,&\text{otherwise};
       \end{cases}
     \]
   \end{enumerate}
   \item Suppose $\cP_{\uptaud}(\bfcc_{1}(\imath_{\wpd}),1)=\bullet$.
   \begin{enumerate}[label={\localtextbulletone}]
     \item If $\bfcc_{2}(\imath_{\wpd}) = \bfcc_{1}(\imath_{\wpd})$
     and
     $\cP_{\uptaud}(\bfcc_{1}(\imath_{\wpd}),2) = r$,
     we define
     \[
       \cP_{\uptauu}(i,j) := \begin{cases}
         r ,& \text{if $j=1$ and $\bfcc_{t}(\imath_{\wpd})\leq i \leq \bfcc_{t}(\imath_{\wpu})-1$},\\
         c ,& \text{if $(i,j)=(\bfcc_{t+1}(\imath_{\wpd}),t+1)$},\\
         d ,& \text{if $(i,j)=(\bfcc_{t}(\imath_{\wpu}),t)$},\\
         \cP_{\uptaud}(i,j) ,&\text{otherwise}.
       \end{cases}
     \]
     \item Otherwise, we define
     \[
       \cP_{\uptauu}(i,j) := \begin{cases}
         r ,& \text{if $j=1$ and $\bfcc_{1}(\imath_{\wpd})\leq i \leq \bfcc_{1}(\imath_{\wpu})-2$},\\
         c ,& \text{if $(i,j)=(\bfcc_{t}(\imath_{\wpu})-1,t)$},\\
         d ,& \text{if $(i,j)=(\bfcc_{t}(\imath_{\wpu}),t)$},\\
         \cP_{\uptaud}(i,j) ,&\text{otherwise}.
       \end{cases}
     \]
   \end{enumerate}
 \end{enumerate}

  For all $(i,j)\in \BOX{\jmath_{\wpu}}$,
   \[
     \cQ_{\uptauu}(i,j) := \cQ_{\uptau}(i,j).
   \]

{\bfseries Suppose $\star = \wtC$.}

  For all $(i,j)\in \BOX{\imath_{\wpu}}$,
   \[
     \cP_{\uptauu}(i,j) := \cP_{\uptau}(i,j).
   \]

  For all $(i,j)\in \BOX{\jmath_{\wpu}}$,
   % For $\uptaum\in\PBPs(\tau_{\PPm})$, define $\uptau=:T_{\PPm,\wp}(\uptaum)$
   % by the following formula:
   \[
     \cQ_{\uptauu}(i,j) :=
     \begin{cases}
       r& \text{if $j=1$ and  $\cP_{\uptau}(i,j)=s$,}\\
       d& \text{if $j=1$ and  $\cP_{\uptau}(i,j)=c$,}\\
       \cQ_{\uptau}(i,j) &\text{otherwise.}
     \end{cases}
   \]


 % to an element $\uptauu\in \PBP_{\star}(\ttau_{\wpu})$ such that

\begin{lem}\label{lem:sn}
  Let $\wp\in \Ass(\ckcO)$. The map $\uptau\mapsto \uptauu$ defined above
  gives a well defined
  bijection
  \[
    T_{\wp,\wpu}\colon \PBP_{\star}(\ttau_{\wp}) \rightarrow
    \PBP_{\star}(\ttau_{\wpu}) %. \quad \uptau \mapsto \uptauu
  \]
\end{lem}
\begin{proof}
 It is routine to check that $\uptauu$ is a valid painted bipartition
 and to construct the inverse map $T_{\wp,\PPm}$ by reversing the above steps.

 \trivial[h]{
   The inverse map $T_{\wp,\PPm}$ is given by the following algorithm:
   \begin{description}
     \item[STEP 1] We first recover $\cP'$.
           If $t=1$ or $\cP'(\bfcc_{t}(\imath_{\wp})-1,t-1)=\bullet$, then
           $\cP':= \cP_{\wp}$.
           Otherwise,
           $\cP'$ is given by $\cP_{\wp}$ except the $2\times 2$ square in
           \eqref{eq:modP} which is given by reversing the formula cited.
     \item[STEP 2]

           \def\xxn{\cP_{\uptaum}(\bfcc_t(\imath_{\PPm})-1,t)} %x_0
           \def\xxo{\cP_{\uptaum}(\bfcc_t(\imath_{\wp}),t)} %x_1
           \def\xxd{\cP_{\uptaum}(\bfcc_t(\imath_{\wp}),t+1)} %x_2
           \def\yyn{\cP'(\bfcc_t(\imath_{\PPm})-1,t)} %y_0
           \def\yyo{\cP'(\bfcc_t(\imath_{\wp})-1,t)} %y_1
           \def\yyt{\cP'(\bfcc_t(\imath_{\wp}),t)} %y_3
           \def\yyd{\cP'(\bfcc_t(\imath_{\wp}),t+1)} %y_2
           We have the following cases:
           \begin{enumerate}[label=(\alph*)]
             \item Suppose $\yyo=r$.
             \begin{itemize}
               \item If $\bfcc_{t+1}(\imath_{\wp}) = \bfcc_{t}(\imath_{\PPm})$
               and
               \[
                 (\yyd,\yyt) = (c,d),
               \]
               let
               \[
                 (\xxo,\xxd):=(\bullet, r)
               \]
               \item Otherwise, let \[
                 \xxo:=\yyt.
               \]
             \end{itemize}
             \item Suppose $\yyo=c$
             \begin{itemize}
               \item If $\bfcc_{t}(\imath_{\PPm})\geq 2$ and $\xxn=r$,
               then let
               \[
                 (\xxn,\xxo):=(c,d).
               \]
               \item Otherwise, let
               \[
                 \xxo :=\bullet.
               \]
             \end{itemize}
           \end{enumerate}
           For the boxes $(i,j)$ in $\BOX{\imath_{\uptaum}}$ which are not specified
           in the above procedure, set
           \[
           \cP_{\uptaum}(i,j):=\cP'(i,j).
           \]
     \item[STEP 3]
           Now $\cP_{\uptaum}$ uniquely determine the painted bipartition
           $\uptaum$.
   \end{description}
   The construction of the inverse map implies that $T_{\PPm,\wp}$ is a
   bijection.
 }

\end{proof}

In the next section, we will define the descent map of painted biparttion and reduce the proof of
\Cref{prop:PBP2} to the basic case.
% % The rest of this section is devoted to the proof of the following lemma, which
% % clearly implies Proposition \ref{prop:PBP}.




% % Suppose $\star = \wtC$, $\wp\neq \emptyset$, and
% % $t:=\min{t|(2t-1,2t)\in \wp}$. Let $\PPm:=\wp - \set{(2t-1,2t)}$. Let
% % $\PPm:=\wp - \set{(2t-1,2t)}$.

% \begin{lem}\label{lem:down}
%   Suppose $\star \in \set{B, C, \wtC, D}$ and $\wp$ is a non-empty subset of
%   $\CPPs(\ckcO)$. Let
%   \[
%     t:=
%     \begin{cases}
%       \min\set{i|(2i-1,2i)\in \wp}, & \text{if $\star \in \set{C,\wtC}$};\\
%       \min\set{i|(2i,2i+1)\in \wp}, & \text{if $\star \in \set{B,D}$},\\
%     \end{cases}
%   \]
%   and
%   \[
%     \PPm:=
%     \begin{cases}
%       \wp \setminus \set{(2t-1,2t)},  & \text{if $\star \in \set{C,\wtC}$};\\
%       \wp \setminus   \set{(2t,2t+1)}, & \text{if $\star \in \set{B,D}$}.\\
%     \end{cases}
%   \]
%   Then
%   \[
%     \sharp(\PBP_{\star}(\tau_{\PPm})) = \sharp(\PBP_{\star}(\tau_{\wp})).
%   \]
% \end{lem}

% \begin{proof}
%   We prove the equality by defining a bijection
%   \[
%     T_{\PPm,\wp}\colon \PBP_{\star}(\tau_{\PPm}) \rightarrow \PBP_{\star}(\tau_{\wp})\quad \uptaum \mapsto \uptau
%   \]
%   % and its inverse $T_{\wp,\PPm}$
%   explicitly case by case. In the following,
%   $\uptau = (\imath_{\wp},\cP_{\uptau})\times (\jmath_{\wp},\cQ_{\uptau})$ will
%   always denote an element in $\PBP_{\star}(\tau_{\wp})$ and
%   $\uptaum = (\imath_{\PPm},\cP_{\uptaum})\times (\jmath_{\PPm},\cQ_{\uptaum})$
%   an element in $\PBP_{\star}(\tau_{\PPm})$.

%   \medskip
% \end{proof}

\subsection{Tails of painted bipartitions of type $B$ and $D$}
\label{sec:tail}
In this subsection, we assume that $\abs{\ckcO}>0$ and
$\star\in\set{B, D, C^*}$. We further assume that $\ckcO$ is quasi-distinguished
if $\star = C^{*}$.

We define the notion of ``tail'' of a painted bipartition.

%Let $(\imath,\jmath) = (\imath_{\star}(\ckcO),\jmath_{\star}(\ckcO))$.
%Note that $l\geq l'$ if $\star\in \{B,C^*\}$,   and $l\geq l'+1$ if $\star=D$.
\def\startt{\star_{\mathrm t}}
Put
\[
  \startt:= \begin{cases}
  D, & \ \text{ if $\star\in \set{B,D}$}; \\
  C^*, &\  \text{ if $\star=C^*$}.
\end{cases}
\quad
\text{and}
\quad
k := \begin{cases}
  \frac{\bfrr_{1}(\ckcO)-\bfrr_{2}(\ckcO)}{2} + 1 &
    \text{if $\star\in \{B,D\}$}; \\
\frac{\bfrr_{1}(\ckcO)-\bfrr_{2}(\ckcO)}{2} -1 &  \text{if $\star=C^*$}.
  \end{cases}
\]
In all the cases, $k$ is a positive integer when $\star \in \set{B,D}$ and $k$ is
non-negative.

\trivial[h]{
Note that $\bfrr_{2}(\ckcO)>0$ is odd in our assumption.
}

% We have $k = \bfcc_{1}(\jmath)-\bfcc_{1}(\imath)+1$,
% $\bfcc_{1}(\jmath)-\bfcc_{1}(\imath)$,
% and $\bfcc_{1}(\imath)-\bfcc_{1}(\jmath)$
% when $\star = B,D,C^{*}$, respectively.

From the pair $(\star, \ckcO)$, we define a Young diagram $\ckcO_{\bftt}$ as follows:
\begin{itemize}
    \item If $\star \in \set{B,D}$,
then $\ckcO_{\bftt}$  consists of two rows with lengths $2k-1$ and $1$.
\item
If $\star =C^*$, then $\ckcO_{\bftt}$ consists of one row
with length  $2k+1$.
\end{itemize}
Note that in all three cases
 $\check \CO_{\mathbf t}$ has $\star_{\mathbf t}$-good parity and every element in $\PBP_{\star_\bftt}(\ckcO_\bftt)$ has the form
 \begin{equation}
 \label{tail0}
  \ytb{{x_1} , {x_2} , {\enon\vdots},{\enon{\vdots}},{x_k}  } \times \emptyset \times
  D,\qquad \qquad  \ytb{{x_1} , {x_2} , {\enon\vdots},{\enon{\vdots}},{x_k}  } \times \emptyset \times
  D\qquad\textrm{or}\qquad \emptyset \times  \ytb{{x_1} , {x_2} , {\enon\vdots},{\enon{\vdots}},{x_k}  } \times
 C^*,
\end{equation}
according to $\star=B, D$ or $C^*$, respectively. %Here $k$ can be zero if $\star = C^*$.

% Here $k=l-l'+1, l-l'$ or $l-l'$ respectively.
%\subsubsection{The case when $\star = B$}

\medskip


Let $ \tau=(\imath,\cP)\times(\jmath,\cQ)\times \alpha \in
\mathrm{PBP}_\star(\check \CO) $. The tail $\tau_\bftt$ of $\tau$ will be a painted bipartition in
$\PBP_{\star_\bftt}(\ckcO_\bftt)$, which is defined below case by case.

{\bfseries The case when $\star = B$:}

In this case, we define the tail $\tau_\bftt$ to be the first painted bipartition in \eqref{tail0} such that the multiset $\{x_1, x_2, \cdots, x_k\}$ is the
union of the multiset
\[
  \set{\cQ(j,1)| \bfcc_{1}(\imath)+1 \leq j \leq  \bfcc_{1}(\jmath) }
    % \cQ(\bfcc_{1}(\imath)+1,1),\cQ(\bfcc_{1}(\imath)+2,1),\cdots, \cQ(\bfcc_{1}(\jmath),1)}
\]
with the set
\[
  \begin{cases}
 \set{c}, &
 \qquad
  \text{if $\alpha = B^+$, and either $\bfcc_{1}(\imath)=0$ or $\cQ(\bfcc_{1}(\imath),1)\in \set{\bullet,s}$};  \\
 \set{s},&
  \qquad \text{if $\alpha = B^-$, and either $\bfcc_{1}(\imath)=0$ or $\cQ(\bfcc_{1}(\imath),1)\in \set{\bullet,s}$}; \\
%  \qquad\text{when } \alpha_\tau = B^-, \text{ and, } l'=0 \textrm{ or } \cQ_\tau(l',1)\in \set{\bullet,s},  \\
\set{\cQ(\bfcc_{1}(\imath),1)},&
\qquad
\text{otherwise.}
%\text{if $\bfcc_{1}(\imath)>0$ and $\cQ(\bfcc_{1}(\imath),1)\in \{r,d\}$.}
\end{cases}
\]

{\bfseries The case when $\star = D$:}
In this case, we define the tail $\tau_\bftt$ to be the second painted
bipartition in \eqref{tail0} such that the multiset $\{x_1, x_2, \cdots, x_k\}$
is the union of the multiset
\[
\set{\cP(j,1)| \bfcc_{1}(\jmath)+2 \leq j \leq \bfcc_{1}(\imath)}
\]
with the set
\[
\begin{cases}
    \set{c},                          &
    \ \text{if $\bfrr_2(\ckcO)=\bfrr_3(\ckcO)$,}                                                                         \\
                                      & \quad \text{$(\cP(\bfcc_{1}(\jmath)+1,1) ,\cP(\bfcc_{1}(\jmath)+1,2)) = (r,c)$ } \\
                                      & \quad \text{ and $\cP(\bfcc_1(\imath),1)\in \set{r,d}$};                                       \\
    \set{\cP(\bfcc_{1}(\jmath)+1,1)}, &
    \    \text{otherwise.}
  \end{cases}
\]


{\bfseries The case $\star = C^*$:}

If $k=0$, we define the tail $\tau_{\bftt}$ to be
$\emptyset\times \emptyset \times C^{*}$.
If $k> 0$, we define the tail $\tau_\bftt$ to be the third painted bipartition in \eqref{tail0} such that
\[
  (x_1, x_2, \cdots, x_k)= (\cQ(\bfcc_{1}(\imath)+1,1),\cQ(\bfcc_{1}(\imath)+2,1),\cdots, \cQ(\bfcc_{1}(\jmath),1)).
\]


 When $\star \in \set{B,D}$, the symbol in the last box of the tail $\tau_\bftt\in \PBP_{\star_\bftt}(\ckcO_\bftt)$ will be important for us. We write $x_\tau$ for it, namely
\[
x_\tau := \cP_{\tau_\bftt}(k,1).
\]


\trivial[]{

 The following technical lemma is easy to check.

\begin{lem}\label{tailtip}
If $\star=B$, then
\[
\hspace{9em} x_\tau=s\Longleftrightarrow
\begin{cases}
  \alpha=B^-;\\
  \bfcc_1(\jmath)=0 \text{ or }\cQ(\bfcc_{1}(\jmath),1) \in\set{\bullet, s},
  \end{cases}
%\quad \textrm{if and only if}\quad \alpha=B^- \ \textrm{ and }\  \cQ(l,1) = s,
\]
and
\[
x_\tau=d \Longleftrightarrow
%\quad \textrm{if and only if}\quad
\cQ(\bfcc_{1}(\jmath),1) =d.
\]
If $\star=D$, then
\[
x_\tau=s\Longleftrightarrow \cP(\bfcc_{1}(\imath),1) = s,
\]
and
\[
x_\tau=d\Longleftrightarrow \cP(\bfcc_{1}(\imath),1) =d.
\]
\qed
\end{lem}
}

\subsection{Descents of painted bipartitions}\label{sec:comb}


In this subsection, we define the descent of a painted bipartition.
%as alluded to in Section \ref{subsec:comTOrep}.
As before, let  $\star\in \{ B, C,  D, \widetilde{C},  C^*, D^*\}$ and let $\check \CO$ be a Young diagram that has $\star$-good parity.


Define a symbol
\[
\star':=\widetilde{C}, \ D, \  C, \ B, \ D^*\  \textrm{ or } \ C^*
\]
respectively if
\[
\star=B,\  C, \ D, \ \widetilde{C}, \ C^* \ \textrm{ or }\  D^*.
\]
We call $\star'$
 the Howe dual of $\star$.

 Define the naive dual descent of a Young diagram $\ckcO$ to be
 \[
  \ckDDn(\ckcO):= \textrm{the Young diagram obtained from $\check \CO$ by removing the first row}.
  \]
   By convention, $\check \nabla_{\mathrm{naive}}(\emptyset)=\emptyset$.

   The dual descent of $\check \CO$ is defined to be
  \[
   \check \CO':=\check \nabla(\check \CO):=\check \nabla_\star( \check \CO):=\begin{cases}
   \tytb{\   }\, , \quad& \textrm{if $\star\in \{D, D^*\}$ and $\check \CO=\emptyset$};\smallskip\\
   \check \nabla_{\mathrm{naive}}(\check \CO), \quad & \textrm{otherwise},
    \end{cases}
  \]
  where $\tytb{\   }$ denotes the Young diagram that has total size $1$.

  % Here $\emptyset$ stands for the empty Young diagram, and $\tytb{\ }$ stands for the Young diagram of total size $1$.

\delete{Put
\begin{equation}\label{lstarco}
  l:=l_{\star, \check \CO}:=\begin{cases}
 \frac{\bfrr_1(\ckcO)}{2}; & \quad \textrm{if } \star\in \{B, \widetilde C\};\smallskip\\
 \frac{\bfrr_1(\ckcO)-1}{2}, &\quad \textrm{if } \star\in \{C, C^* \};\smallskip\\
 \frac{\bfrr_1(\ckcO)+1}{2}, &\quad \textrm{if } \star\in \{ D, D^*\}.\\
\end{cases}
\end{equation}
This is the length of the leading column of every element of $\mathrm{PBP}_\star(\check \CO)$.
}

For a Young diagram $\imath$, its naive descent, which is denoted by $\nabla_\mathrm{naive}(\imath)$, is defined to be the Young diagram obtained from $\imath$ by removing the first column. By convention, $\nabla_\mathrm{naive}(\emptyset)=\emptyset$.

%In the rest of this section, we assume that $\check \CO\neq \emptyset$.

 %Put\[
%l':=l_{\star', \check \CO'}
%\]


\def\bipartl{\mathrm{bi\cP_L}}
\def\bipartr{\mathrm{bi\cP_R}}
\def\dsdiagl{\mathrm{DS_L}}
\def\dsdiagr{\mathrm{DS_R}}
\def\DDl{\eDD_\mathrm{L}}
\def\DDr{\eDD_\mathrm{R}}

We first define the naive descent of a painted bipartition. Let $\uptau=(\imath,\cP)\times (\jmath,\cQ)\times \alpha$ be a painted bipartition such that $\star_\tau=\star$. Put
\delete{\begin{equation} \label{eq:def.alphap}
\alpha'=\begin{cases} B^+,
& \textrm{if $\alpha = \wtC$ and $\cP_\tau(l_{\star,\ckcO},1),1) \neq c$;}\\
B^-,
& \textrm{if $\alpha = \wtC$ and $\cP_\tau(l_{\star,\ckcO},1),1)  = c$;}\\
\star', & \textrm{if $\alpha\neq \widetilde C$}.
\end{cases}
\end{equation}
}
  \begin{equation} \label{eq:def.alphap}
    \alphapn=\begin{cases} B^+,
  & \textrm{if $\alpha=\widetilde{C}$ and $c$ does not occur in the first column of $(\imath,\cP)$}; \smallskip \\
  B^-,
  & \textrm{if $\alpha=\widetilde{C}$ and  $c$ occurs in the first column of $(\imath,\cP)$}; \smallskip \\
  \star', & \textrm{if $\alpha\neq \widetilde C$}.
  \end{cases}
  \end{equation}

\begin{lem}\label{lemDDn1}
  If $\star \in \set{B,C,C^*}$, then there is a unique painted bipartition of the form $\uptaupn= (\imathpn,\cPpn)\times (\jmathpn,\cQpn)\times \alphapn$ with the following properties:
  \begin{itemize}
        \item $
   (\imathpn,\jmathpn)= (\imath,\DD_\mathrm{naive}(\jmath)); \smallskip
   $
   \item for all $(i,j)\in \BOX{\imathpn}$,
   \[
     \cPpn(i,j)=\begin{cases}
    \bullet \textrm{ or } s,&\textrm{ if  $\ \cP(i,j)\in \{\bullet, s\}$;} \smallskip \\
  \cP(i,j),& \textrm{ if $\ \cP(i,j)\notin \{\bullet, s\}$};\end{cases}
   \]
   \item for all $(i,j)\in \BOX{\jmathpn}$,
   \[
     \cQpn(i,j)=\begin{cases}
    \bullet \textrm{ or } s,&\textrm{ if  $\ \cQ(i,j+1)\in \{\bullet, s\}$;} \smallskip \\
  \cQ(i,j+1), & \textrm{ if $\ \cQ(i,j+1)\notin \{\bullet, s\}$}.  \end{cases}
   \]
    \end{itemize}
    \end{lem}




   \begin{proof}
    First assume that the images of $\cP$ and $\cQ$ are both contained in $\{\bullet, s\}$. Then  the image of $\cP$  is in fact contained in $\{\bullet\}$, and $(\imath, \jmath)$ is  right interlaced in the sense that
 \[
 \mathbf{c}_1(\jmath)\geq \mathbf{c}_1(\imath)\geq \mathbf{c}_2(\jmath)\geq \mathbf{c}_2(\imath)\geq \mathbf{c}_3(\jmath)\geq \mathbf{c}_3(\imath) \geq \cdots.
 \]
 Hence $ (\imath',\jmath'):= (\imath,\DD(\jmath))$ is left interlaced in the sense that
 \[
 \mathbf{c}_1(\imath')\geq \mathbf{c}_1(\jmath')\geq \mathbf{c}_2(\imath')\geq \mathbf{c}_2(\jmath')\geq \mathbf{c}_3(\imath')\geq \mathbf{c}_3(\jmath') \geq \cdots.
 \]
 Then it is clear that there is a unique painted bipartition of the form  $\uptaupn=(\imathpn,\cPpn)\times (\jmathpn,\cQpn)\times \alphapn$ such that images of $\cPpn$ and $\cQpn$ are both contained in $\{\bullet, s\}$. This proves the lemma in the special case when the images of $\cP$ and $\cQ$ are both contained in $\{\bullet, s\}$.

 The proof of the lemma in the general case is easily reduced to this special case.
   \end{proof}
    \begin{lem}\label{lemDDn2}
    If $\star \in \set{ \wtC, D,D^*}$, then there is a unique painted bipartition of the form $\uptaupn= (\imathpn,\cPpn)\times (\jmathpn,\cQpn)\times \alphapn$ with the following properties:
  \begin{itemize}
        \item $
   (\imathpn,\jmathpn)= (\DD_\mathrm{naive}(\imath),\jmath); \smallskip
   $
   \item for all $(i,j)\in \BOX{\imathpn}$,
   \[
     \cPpn(i,j)=\begin{cases}
    \bullet \textrm{ or } s,&\textrm{ if  $\ \cP(i,j+1)\in \{\bullet, s\}$;} \smallskip \\
  \cP(i,j+1),& \textrm{ if $\ \cP(i,j+1)\notin \{\bullet, s\}$};\end{cases}
   \]
   \item for all $(i,j)\in \BOX
          {\jmathpn}$,
   \[
     \cQpn(i,j)=\begin{cases}
    \bullet \textrm{ or } s,&\textrm{ if  $\ \cQ(i,j)\in \{\bullet, s\}$;} \smallskip \\
  \cQ(i,j), & \textrm{ if $\ \cQ(i,j)\notin \{\bullet, s\}$}.  \end{cases}
   \]

    \end{itemize}
\end{lem}
\begin{proof}
  The proof is similar to that of \Cref{lemDDn1}.
\end{proof}

\begin{defn}
 In the notation of \Cref{lemDDn1,lemDDn2}, we call $\uptaupn$ the naive descent of $\uptau$, to be denoted by $\DDn(\uptau)$.
\end{defn}




 \begin{eg} If
    \[
     \uptau = \ytb{\bullet\bullet\bullet {c},\bullet {s} {c},{s},{c}}
    \times \ytb{\bullet\bullet\bullet ,\bullet {r} {d},{d}{d}, \none}
    \times \widetilde C, \]
   then
   \[
    \nabla_{\mathrm{naive}}(\uptau) =\ytb{\bullet\bullet{c} ,\bullet{c},\none }
    \times  \ytb{\bullet\bullet {s} ,\bullet {r} {d},{d}{d}}\times B^-.
    \]

\end{eg}


From now on, we suppose that $\ckcO$ is non-empty.
Let $\ckcO' := \DD(\ckcO)$, $\wp':=\DD(\wp)$ and $\tauwpp = (\imathwpp,\jmathwpp)$ be the
bipartition defined by \eqref{eq:ttauwp} with respect to $\ckcO'$.

Suppose that $\wp\in A(\ckcO)$.
\[
\uptau=(\imath,\cP)\times(\jmath,\cQ)\times \alpha \in  \PBPOP.
\]
% and write
% \[
%   \uptaupn=(\imath'_{\mathrm{naive}}, \cP'_{\mathrm{naive}})\times (\jmath'_{\mathrm{naive}}, \cQ'_{\mathrm{naive}})\times \alpha'
% \]
% for the naive descent of $\uptau$.


We define
\[
  \uptau' := (\imathwpp, \cP')\times (\jmathwpp, \cQ')\times \alpha'
\]
such that $\cP'$ and $\cQ'$ are paintings on $\BOX{\imath_{\wp'}}$ and
$\BOX{\jmath_{\wp'}}$  % determined
by the following algorithm respectively:

%the descent of the painted bipartition $\uptau$ by to be


% This is clearly an element of $  \PBP(\check \CO')$.
%Put
%\begin{equation}\label{lstarco}
%  l:=l_{\star, \check \CO}:=\begin{cases}
% \frac{\bfrr_2(\ckcO)}{2}; & \quad \textrm{if } \star\in \{B, \widetilde C\};\\
% \frac{\bfrr_2(\ckcO)+1}{2}, &\quad \textrm{if } \star\in \{C, C^* \};\\
% \frac{\bfrr_2(\ckcO)-1}{2}, &\quad \textrm{if } \star\in \{ D, D^*\}.\\
%\end{cases}
%\end{equation}

{\bfseries Case $\star = B$. }
We define $\alpha' := \alphapn$.
%\begin{enumerate}[label={\localtextbulletone}]
\begin{enumerate}[label=(\alph*)]
  \item  Suppose
                \[
                \begin{cases}
                  \alpha = B^+; & \\
                  (2,3) \notin \wp & \\
                  \bfrr_2(\ckcO)>0; & \\
                  \cQ(\bfcc_1(\imath_{\wp}),1)\in \set{r,d}.
                \end{cases}
                \]
                We define %
                \[
                \cP'(i,j) := \begin{cases}
                  s, & \ \text{ if $(i,j) = (\bfcc_1(\imathwpp),1)$;}\\
                  \cPpn(i,j), & \ \text{ otherwise},
                \end{cases}
                % \quad \text{for all $(i,j)\in \BOX{\imathwpp}$,}
                \] for all $(i,j)\in \BOX{\imathwpp}$, and
                $\cQ' := \cQpn $.
                \trivial[h]{
                  Note that $\bfcc_{1}(\imathwpp) = \bfcc_{1}(\imath_{\wp})$
                }
  \item Suppose
                \[
                \begin{cases}
                  \alpha = B^+; & \\
                  (2,3)\in \wp & \\
                  \cQ(\bfcc_2(\jmath_{\wp}),1)\in \set{r,d},
                \end{cases}
                \]
                We define
                $\cP' := \cPpn $ and
                \[
                \cQ'(i,j) := \begin{cases}
                  r, & \ \text{ if $(i,j) = (\bfcc_1(\jmathwpp),1)$;}\\
                  \cQpn(i,j), & \ \text{ otherwise},
                \end{cases}
                \] for all $(i,j)\in \BOX{\jmathwpp}$.
                \trivial[h]{
                  Note that $\bfcc_{1}(\jmathwpp) = \bfcc_{2}(\jmath_{\wp})$
                }
        \item Otherwise, we define $\cP' := \cPpn$ and $\cQ':= \cQpn$.
\end{enumerate}

{\bfseries Case $\star = D$. }
We define $\alpha' := \alphapn$.
\begin{enumerate}[label=(\alph*)]
  \item  Suppose
  \[
    \begin{cases}
      \bfrr_2(\ckcO)=\bfrr_{3}(\ckcO)>0; & \\
      \cP(\bfcc_{2}(\imath_{\wp}),2) = c;  &\\
      \cP(i,1)\in \set{r,d}, & \text{for all
        $\bfcc_{2}(\imath_{\wp})\leq i\leq \bfcc_{1}(\imath_{\wp})$}.
    \end{cases}
  \]
  We define %
  \[
    \cP'(i,j) := \begin{cases}
      r, & \ \text{ if $(i,j) = (\bfcc_1(\imathwpp),1)$;}\\
      \cPpn(i,j), & \ \text{ otherwise},
    \end{cases}
    % \quad \text{for all $(i,j)\in \BOX{\imathwpp}$,}
  \] for all $(i,j)\in \BOX{\imathwpp}$, and
  $\cQ' := \cQpn $.
  \item Suppose
  \[
    \begin{cases}
      (2,3)\in \wp & \\
      \cP(\bfcc_2(\imath_{\wp})-1,1)\in \set{r,c},
    \end{cases}
  \]
  We define
  \[
    \cP'(i,j) := \begin{cases}
      r, & \ \text{ if $(i,j) = (\bfcc_1(\imathwpp)-1,1)$;}\\
      \cP(\bfcc_2(\imath_{\wp})-1,1) & \ \text{ if $(i,j) = (\bfcc_1(\imathwpp),1)$;}\\
      \cPpn(i,j), & \ \text{ otherwise},
    \end{cases}
  \] for all $(i,j)\in \BOX{\imathwpp}$.
  $\cQ' := \cQpn $ and
  \item Otherwise, we define $\cP' := \cPpn$ and $\cQ':= \cQpn$.
\end{enumerate}

{\bfseries Case $\star \in \set{C,\wtC,C^{*},D^{*}}$}

\begin{enumerate}[label=(\alph*)]
  \item Suppose $(1,2)\notin \wp$. We define
  \[
    \uptau' := \DDn(\uptau).
  \]
  % \[
  %   \alpha' = \alphapn, \quad
  %   \cP' := \cPpn \AND
  %   \cQ' := \cQpn.
  % \]
  \item Suppose $(1,2)\in \wp$. We define
  \[
  \uptau' = \DDn(\uptau_{\wpm})
  \]
  where $\wpm := \wp - \set{(1,2)}$ and
  $\uptau_{\wpm}  := T_{\wpm,\wp}^{-1}(\uptau)$ (see \Cref{lem:sn}).
\end{enumerate}


It is routine to check that the above definition do gives valid painted
bipartition and we record the following lemma.
\begin{lem}
  Suppose that $\ckcO$ is non-empty and $\ckcO' := \DD(\ckcO)$.
  For each $\wp\in \PP(\ckcO)$, the map $\uptau\mapsto \uptau'$ defined above
  gives a well defined map
  \begin{equation}
    \label{eq:DD.CC}
    \DD \colon \PBPOP \rightarrow \PBPOPp.
  \end{equation}
  \qed
\end{lem}

Note that when $\star\in \set{C^{*}, D^{*}}$, we will only consider the case
when $\ckcO$ is quasi-distinguished and $\wp=\emptyset$ since
$\PBPOP = \emptyset $ otherwise.


The following proposition summaries the key properties of the descent map.
\begin{prop} \label{lem:PBPd.C}
  Suppose that $\star\in \set{C,\wtC, D^*}$. Let $\ckcO$ be a non-empty partition of good
  parity and $\ckcO' := \DD(\ckcO)$.
  \begin{enumerate}[label=(\alph*)]
    \item When $\bfrr_1(\ckcO)>\bfrr_2(\ckcO)$,
    the map \eqref{eq:DD.CC} is bijective.
    \item When $\star\in \{C,\widetilde C\}$ and $\bfrr_1(\ckcO)=\bfrr_2(\ckcO)$,
    then the  map \eqref{eq:DD.CC} is injective and its image equals
    \[
      \Set{\uptau'\in \PBPOPp |  x_{\uptau'}\neq s}.
    \]
  \end{enumerate}
\end{prop}



\begin{prop}
\label{lem:delta}
Suppose that $\star \in \set{D,B,C^*}$ and $\bfrr_2(\ckcO)>0$. Write
$\ckcOpp := \ckDD(\ckcO')$, $\wp'':= \DD(\DD(\wp))$ and let $\tauwppp$ be the
bipartition attached to $\wp''$ with respect to $\ckcO''$. Consider the map
\begin{equation}\label{eq:delta}
  \delta  \colon \PBPOP \longrightarrow
    \PBPOPp \times \PBP_{\star_\bftt}(\ckcO_\bftt),
    \qquad \uptau \mapsto (\DD^2(\uptau),\uptau_\bftt).
\end{equation}
\begin{enumerate}[label=(\alph*)]
\item Suppose that
$\star = C^*$ or $\bfrr_2(\ckcO)>\bfrr_3(\ckcO)$. Then the map \eqref{eq:delta} is bijective, and for every $\uptau\in  \PBP_\star(\ckcO) $,
    % We have the following equation of signatures.
\begin{equation}\label{eq:sign.D}
\Sign(\uptau)
=(\bfcc_2(\cO),\bfcc_2(\cO))+\Sign(\DD^2(\uptau))+\Sign(\uptau_\bftt).
\end{equation}

\item Suppose that  $\star \in \set{B,D}$ and $\bfrr_2(\ckcO)=\bfrr_3(\ckcO)>0$.
Then the map \eqref{eq:delta} is an  injection and its  image equals
\begin{equation}\label{eq:delta.I}
    \Set{ (\uptau'',\uptau_0)  \in \PBPOPpp \times \PBP_D(\ckcO_\bftt)  |
    \begin{array}{l}
        \text{either
    $x_{\uptau''} = d$, or} \\
    \text{$x_{\uptau''}\in \set{r,c}$  and
    $\cP_{\uptau_0}^{-1}(\set{s,c})\neq \emptyset$}
    \end{array}
}.
\end{equation}
Moreover,  for every $\uptau\in  \PBPOP$,
\begin{equation}\label{eq:sign.GD}
\Sign(\uptau)
=(\bfcc_2(\cO)-1,\bfcc_2(\cO)-1)+\Sign(\DD^2(\uptau))+\Sign(\uptau_\bftt).
\end{equation}
\end{enumerate}
\end{prop}
\begin{proof}
  One can deduce the inverse of $\delta$ (implemented in \cite{MU}) and signature formula via a patient analysis % The proposition can be verified
  % by a patient analysis
  of the descent algorithm.
\end{proof}


We have the following immediate corollary.

\begin{cor}\label{prop:DD.BDinj}
  Suppose $\star \in \set{B, D,C^*}$.
  We define
  \[
    \varepsilon_{\uptau} = \begin{cases}
      0 & \text{if $\star\in \set{B,D}$, and $x_{\uptau}=d$,}\\
      1 & \text{otherwise.}
    \end{cases}
  \]
  Then the map
\begin{equation}
  \begin{array}{rcl}
   \PBPOP&\rightarrow&
   \PBPOPp \times \BN\times \bN\times \Z/2\Z, \smallskip\\
   \uptau& \mapsto & (\DD(\uptau), p_\uptau, q_\uptau, \varepsilon_\uptau)
   \end{array}
\end{equation}
is injective.
\end{cor}
\begin{proof}
  It is easy to verify the the basic case where $\bfrr_{3}(\ckcO)=0$ and
    \[
      (\bfrr_{1}(\ckcO), \bfrr_{2}(\ckcO))
      =
      \begin{cases}
        (2k-2,0) & \text{if } \star=B,\\
        (2k-1,1) & \text{if } \star=D,\\
        (2k-1,0) & \text{if } \star=C^{*}.\\
      \end{cases}
    \]
    \trivial[h]{
    Then $\ckcO$ is the regular nilpotent orbit.
    % % in $\check \fgg$
    and $\cO =\dBV(\ckcO) = (1^{\bfrr_{1}(\ckcO)+1})_{\star}$.
    }
    The other cases follows from the injectivity of $\DD$ in \Cref{lem:PBPd.C}
    and the signature formula in \Cref{lem:delta}.
\end{proof}



Now we define some generating functions of painted bipartition. %$(\ckcO,\wp)$


Let $\star \in \set{B,D}$.
For each $\wp\in \CPPs(\ckcO)$ and a subset $S\subset \set{c,d,r,s}$, we define
\[
  \PBPOP[S] = \set{\uptau\in \PBPOP|x_{\uptau}\in S}
\]
and the corresponding generating function
\[
   f_{\star,\ckcO,\wp}^{S}(\bfpp,\bfqq) := \sum_{\uptau \in \PBPOP[S]} \bfpp^{p_{\uptau}} \bfqq^{q_{\uptau}}
\]
in $\bZ[\bfpp,\bfqq]$ where $(p_{\uptau}, q_{\uptau})$ is the signature of
$\uptau$ and $\bfpp$
and $\bfqq$ are indeterminants.
By to ease the notation, we write
\[
  f_{\star,\ckcO,\wp} := f_{\star,\ckcO,\wp}^{\set{c,d,r,s}} =
  f_{\star,\ckcO,\wp}^{\set{s}}
  + f_{\star,\ckcO,\wp}^{\set{c,r}}
  + f_{\star,\ckcO,\wp}^{\set{d}}
\]
whose coefficient of $\bfpp^{p}\bfqq^{q}$ equals the cardinality of
$\PBPop{\SO(p,q)}{}{\ckcO}{\wp}$.
% $\PBPGOP$
% with $\Sign(G)=(p,q)$.

\def\CSk#1#2{h_{#1}^{#2}}
\def\TSk#1#2{g_{#1}^{#2}}
\def\RS{\nu}

For an integer $k$,
we define
\[
\RS_{k} := \sum_{i=0}^{k} \bfpp^{i}\bfqq^{k-i}
\]
if $k\geq 0$ and $0$ otherwise.
\trivial[]{
This is the generating function of all PBP of type D with only one row filled
by r or s.
}

We make the following definition:
\[
  \begin{split}
  \TSk{k}{\set{d}} &:= f_{D,\yrow{2k-1,1},\emptyset}^{\set{d}}= \bfpp\bfqq\RS_{k-1} + \bfpp^{2}\bfqq^{2}\RS_{k-2}, \\
  \TSk{k}{\set{c,r}} &:= f_{D,\yrow{2k-1,1},\emptyset}^{\set{rc}}= (\bfpp\bfqq+\bfpp^{2})\RS_{k-1},  \\
  \TSk{k}{\set{s}}& :=\CSk{k}{\set{s}} := f_{D,\yrow{2k-1,1},\emptyset}^{\set{s}}= \bfqq^{2n},\\
  \CSk{k}{\set{d}} &:= \bfpp^{2}\bfqq^{2}\RS_{k-2} + \bfpp\bfqq^{3}\RS_{k-2}, \AND \\
  \CSk{k}{\set{c,r}} &:= \bfpp\bfqq\RS_{k-1}+\bfpp^{2}\bfqq^{2}\RS_{k-2}.\\
  \end{split}
\]
\trivial[h]{
  For reference, rest = column filled by r and s.
\[
  \begin{split}
  \TSk{k}{\set{d}} &:=  rest + d\cup rest+cd \\
  \TSk{k}{\set{c,r}} &:= rest+c \cup rest+r\\
  \TSk{k}{\set{s}}& :=\CSk{k}{\set{s}} := ss\cdots s,\\
  \CSk{k}{\set{d}} &:= rest+cd \cup s+rest + d\\
  \CSk{k}{\set{c,r}} &:= rest+c \cup s+rest+r \\
  \end{split}
\]
}

It is also straight to see that, for non-negative  integer $k$,
\[
  \begin{split}
   f_{B,\yrow{2k},\emptyset}^{\set{d}} & = (\bfpp^{2}\bfqq+ \bfpp\bfqq^{3})\RS_{k-1}, \\
   f_{B,\yrow{2k},\emptyset}^{\set{rc}} & = \bfpp\RS_{k} + \bfpp^{2}\bfqq\RS_{k-1},  \\
  f_{B,\yrow{2k},\emptyset}^{\set{s}} & = \bfqq^{2k+1},\\
  \end{split}
\]
\trivial[h]{
  For reference, rest := count columns filled by r and s.
\[
  \begin{split}
    f_{B,\yrow{2k},\emptyset}^{\set{d}} &:=  B^{+} + rest + d \cup B^{-}+rest+d, \\
    f_{B,\yrow{2k},\emptyset}^{\set{rc}}   &:= B^{+} + rest \cup B^{-}+rest+r, \\
    f_{B,\yrow{2k},\emptyset}^{\set{s}} & := B^{-} + ss\cdots s.\\
  \end{split}
\]
}

By convention, we set
\[
  f_{D,\yrow{0}, \emptyset}^{\set{d}} := 1, \quad
  f_{D,\yrow{0}, \emptyset}^{\set{r,c}} := 0,
  \AND
  f_{D,\yrow{0}, \emptyset}^{\set{s}} := 0,
\]

\begin{lem}\label{lem:gf.BD}
  Suppose $\star\in\set{B,D}$ and $\ckcO$ is a good parity partition such that $\bfrr_{2}(\ckcO)>0$.

  Let $k:=\frac{\bfrr_{1}(\ckcO)-\bfrr_{2}(\ckcO)}{2}+1$.
  Then we have the following  recursive formulas of the generating functions
  with $S \in \set{\set{d}, \set{c,r}, \set{s}}$:
  \begin{enumerate}[label=(\alph*)]
    \item If $(2,3)\in \CPPs(\ckcO)$, then
    \[
      f_{D,\ckcO, \wp}^{S} := \TSk{k}{S} \cdot  f_{D,\DD^{2}(\ckcO),\DD^{2}(\wp)}.
    \]
    \item If $(2,3)\notin \CPPs(\ckcO)$,
    then
    \[
      f_{D,\ckcO, \wp}^{S} := \TSk{k}{S} \cdot f_{D,\DD^{2}(\ckcO),\DD^{2}(\wp)}^{\set{d}}
      + \CSk{k}{S} \cdot f_{D,\DD^{2}(\ckcO),\DD^{2}(\wp)}^{\set{r,c}}.
    \]
  \end{enumerate}
  In particular, the generating function is independent of $\wp$ and so
  \[
    f_{D,\ckcO,\wp} = f_{D,\ckcO,\emptyset} \quad \text{for all $\wp\in \CPPs(\ckcO)$.}
  \]
\end{lem}
\begin{proof}
  This follows immediately from \Cref{lem:delta}.
\end{proof}


\begin{lem}\label{lem:gf.C}
  Suppose $\star\in\set{C,\wtC}$ and $\ckcO$ is a good parity partition such that $\bfrr_{1}(\ckcO)>0$.
  Then % we have the following  formulas of the generating functions
  % with $S \in \set{\set{d}, \set{c,r}, \set{s}}$:
  \begin{enumerate}[label=(\alph*)]
    \item If $(1,2)\in \CPPs(\ckcO)$, then
    \[
      \# \PBPOP =  f_{\star',\DD(\ckcO), \DD(\wp)}(1,1).
    \]
    \item If $(1,2)\notin \CPPs(\ckcO)$,
    then
    \[
      \# \PBPOP = f_{\star',\DD(\ckcO), \DD(\wp)}^{\set{c,r}}(1,1) + f_{\star',\DD(\ckcO), \DD(\wp)}^{\set{d}}(1,1).
    \]
  \end{enumerate}
  In particular, the cardinality of $\PBPOP$ is independent of $\wp$ and so
  \[
    \# \PBPOP = \# \PBPop{\star}{}{\ckcO}{\emptyset} \quad \text{for all $\wp\in \CPPs(\ckcO)$.}
  \]
\end{lem}
\begin{proof}
  This follows immediately from \Cref{lem:delta} and \Cref{lem:gf.BD}.
\end{proof}

% In this section, we exhibit the inductive structure of painted bipartitions.
% The key is to define the descent of a painted bipartition.
% As before, let  $\star\in \set{B, C,  D, \wtC,  C^*, D^*}$ and let $\check \CO$ be a Young diagram that has $\star$-good parity.

\begin{lem}\label{lem:gf.C*}
  We have the following recursive formula
\end{lem}







\begin{bibdiv}
  \begin{biblist}
% \bib{AB}{article}{
%   title={Genuine representations of the metaplectic group},
%   author={Adams, Jeffrey},
%   author = {Barbasch, Dan},
%   journal={Compositio Mathematica},
%   volume={113},
%   number={01},
%   pages={23--66},
%   year={1998},
% }

% \bib{Ad83}{article}{
%   author = {Adams, J.},
%   title = {Discrete spectrum of the reductive dual pair $(O(p,q),Sp(2m))$ },
%   journal = {Invent. Math.},
%   number = {3},
%  pages = {449--475},
%  volume = {74},
%  year = {1983}
% }

%\bib{Ad07}{article}{
%  author = {Adams, J.},
%  title = {The theta correspondence over R},
%  journal = {Harmonic analysis, group representations, automorphic forms and invariant theory,  Lect. Notes Ser. Inst. Math. Sci. Natl. Univ. Singap., 12},
% pages = {1--39},
% year = {2007}
% publisher={World Sci. Publ.}
%}


\bib{ABV}{book}{
  title={The Langlands classification and irreducible characters for real reductive groups},
  author={Adams, J.},
  author={Barbasch, D.},
  author={Vogan, D. A.},
  series={Progress in Math.},
  volume={104},
  year={1991},
  publisher={Birkhauser}
}

\bib{AC}{article}{
  title={Algorithms for representation theory of
    real reductive groups},
  volume={8},
  DOI={10.1017/S1474748008000352},
  number={2},
  journal={Journal of the Institute of Mathematics of Jussieu},
  publisher={Cambridge University Press},
  author={Adams, J.},
  author={du Cloux, F.},
  year={2009},
  pages={209-259}
}

% \bib{ArPro}{article}{
%   author = {Arthur, J.},
%   title = {On some problems suggested by the trace formula},
%   journal = {Lie group representations, II (College Park, Md.), Lecture Notes in Math. 1041},
%  pages = {1--49},
%  year = {1984}
% }


% \bib{ArUni}{article}{
%   author = {Arthur, J.},
%   title = {Unipotent automorphic representations: conjectures},
%   %booktitle = {Orbites unipotentes et repr\'esentations, II},
%   journal = {Orbites unipotentes et repr\'esentations, II, Ast\'erisque},
%  pages = {13--71},
%  volume = {171-172},
%  year = {1989}
% }

% \bib{AK}{article}{
%   author = {Auslander, L.},
%   author = {Kostant, B.},
%   title = {Polarizations and unitary representations of solvable Lie groups},
%   journal = {Invent. Math.},
%  pages = {255--354},
%  volume = {14},
%  year = {1971}
% }


% \bib{B.Uni}{article}{
%   author = {Barbasch, D.},
%   title = {Unipotent representations for real reductive groups},
%  %booktitle = {Proceedings of ICM, Kyoto 1990},
%  journal = {Proceedings of ICM (1990), Kyoto},
%    % series = {Proc. Sympos. Pure Math.},
%  %   volume = {68},
%      pages = {769--777},
%  publisher = {Springer-Verlag, The Mathematical Society of Japan},
%       year = {2000},
% }



\bib{B.Orbit}{article}{
  author = {Barbasch, D.},
  title = {Orbital integrals of nilpotent orbits},
 %booktitle = {The mathematical legacy of {H}arish-{C}handra ({B}altimore,{MD}, 1998)},
    journal = {The mathematical legacy of {H}arish-{C}handra, Proc. Sympos. Pure Math.},
    %series={The mathematical legacy of {H}arish-{C}handra, Proc. Sympos. Pure Math},
    volume = {68},
     pages = {97--110},
 publisher = {Amer. Math. Soc., Providence, RI},
      year = {2000},
}

\bib{B89}{article}{
  author = {Barbasch, D.},
  title = {The unitary dual for complex classical Lie groups},
  journal = {Invent. Math.},
  volume = {96},
  number = {1},
 pages = {103--176},
 year = {1989}
}


\bib{B10}{article}{
  author = {Barbasch, D.},
  title = {The unitary spherical spectrum for split classical groups},
  journal = {J. Inst. Math. Jussieu},
% number = {9},
 pages = {265--356},
 volume = {9},
 year = {2010}
}



\bib{B17}{article}{
  author = {Barbasch, D.},
  title = {Unipotent representations and the dual pair correspondence},
  journal = {J. Cogdell et al. (eds.), Representation Theory, Number Theory, and Invariant Theory, In Honor of Roger Howe. Progress in Math.},
  %series ={Progress in Math.},
  volume = {323},
  pages = {47--85},
  year = {2017},
}


\bib{Bo}{article}{
   author={Bo\v{z}i\v{c}evi\'{c}, M.},
   title={Double cells for unitary groups},
   journal={J. Algebra},
   volume={254},
   date={2002},
   number={1},
   pages={115--124},
   issn={0021-8693},
   review={\MR{1927434}},
   doi={10.1016/S0021-8693(02)00070-4},
}



\bib{BMSZ1}{article}{
      title={On the notion of metaplectic Barbasch-Vogan duality},
      year={2020},
      author={Barbasch, D.},
      author = {Ma, J.-J.},
      author = {Sun, B.},
      author = {Zhu, C.-B.},
      eprint={2010.16089},
      archivePrefix={arXiv},
      primaryClass={math.RT}
}

\bib{BMSZ2}{article}{
      title={Special unipotent representations of real classical groups: construction and unitarity},
      author={Barbasch, D.},
      author = {Ma, J.-J.},
      author = {Sun, B.},
      author = {Zhu, C.-B.},
      year={2021},
      eprint={arXiv:1712.05552},
      archivePrefix={arXiv},
      primaryClass={math.RT}
}



\bib{BV1}{article}{
   author={Barbasch, D.},
   author={Vogan, D. A.},
   title={Primitive ideals and orbital integrals in complex classical
   groups},
   journal={Math. Ann.},
   volume={259},
   date={1982},
   number={2},
   pages={153--199},
   issn={0025-5831},
   review={\MR{656661}},
   doi={10.1007/BF01457308},
}

\bib{BV2}{article}{
   author={Barbasch, D.},
   author={Vogan, D. A.},
   title={Primitive ideals and orbital integrals in complex exceptional
   groups},
   journal={J. Algebra},
   volume={80},
   date={1983},
   number={2},
   pages={350--382},
   issn={0021-8693},
   review={\MR{691809}},
   doi={10.1016/0021-8693(83)90006-6},
}

\bib{BV.W}{article}{
  author={Barbasch, D.},
  author={Vogan, D. A.},
  editor={Trombi, P. C.},
  title={Weyl Group Representations and Nilpotent Orbits},
  bookTitle={Representation Theory of Reductive Groups:
    Proceedings of the University of Utah Conference 1982},
  year={1983},
  publisher={Birkh{\"a}user Boston},
  address={Boston, MA},
  pages={21--33},
  %doi={10.1007/978-1-4684-6730-7_2},
}


\bib{BVUni}{article}{
 author = {Barbasch, D.},
 author = {Vogan, D. A.},
 journal = {Annals of Math.},
 number = {1},
 pages = {41--110},
 title = {Unipotent representations of complex semisimple groups},
 volume = {121},
 year = {1985}
}

% \bib{BB}{article}{
%    author={Beilinson, Alexandre},
%    author={Bernstein, Joseph},
%    title={Localisation de $\mathfrak g$-modules},
%    language={French, with English summary},
%    journal={C. R. Acad. Sci. Paris S\'{e}r. I Math.},
%    volume={292},
%    date={1981},
%    number={1},
%    pages={15--18},
%    issn={0249-6291},
%    review={\MR{610137}},
% }

\bib{BGG.M}{article}{
   author={Bernstein, I. N.},
   author={Gel'fand, I. M.},
   author={Gel'fand, S. I.},
   title={Models of representations of compact Lie groups},
   language={Russian},
   journal={Funkcional. Anal. i Prilo\v{z}en.},
   volume={9},
   date={1975},
   number={4},
   pages={61--62},
   %issn={0374-1990},
   %review={\MR{0414792}},
}


\bib{BK}{article}{
author={Borho, W.},
author={Kraft, H.},
title={\"{U}ber die Gelfand-Kirillov-Dimension},
journal={Math. Ann.},
volume={220},
date={1976},
number={1},
pages={1--24},
issn={0025-5831},
review={\MR{412240}},
doi={10.1007/BF01354525},
}


% \bib{Br}{article}{
%   author = {Brylinski, R.},
%   title = {Dixmier algebras for classical complex nilpotent orbits via Kraft-Procesi models. I},
%   journal = {The orbit method in geometry and physics (Marseille, 2000). Progress in Math.}
%   volume = {213},
%   pages = {49--67},
%   year = {2003},
% }

\bib{BK}{article}{
   author={Brylinski, J.-L.},
   author={Kashiwara, M.},
   title={Kazhdan-Lusztig conjecture and holonomic systems},
   journal={Invent. Math.},
   volume={64},
   date={1981},
   number={3},
   pages={387--410},
   issn={0020-9910},
   review={\MR{632980}},
   doi={10.1007/BF01389272},
}

\bib{Carter}{book}{
   author={Carter, R. W.},
   title={Finite groups of Lie type},
   series={Wiley Classics Library},
   %note={Conjugacy classes and complex characters;
   %Reprint of the 1985 original;
   %A Wiley-Interscience Publication},
   publisher={John Wiley \& Sons, Ltd., Chichester},
   date={1993},
   pages={xii+544},
   isbn={0-471-94109-3},
   %review={\MR{1266626}},
}

\bib{Cas}{article}{
   author={Casian, L. G.},
   title={Primitive ideals and representations},
   journal={J. Algebra},
   volume={101},
   date={1986},
   number={2},
   pages={497--515},
   issn={0021-8693},
   review={\MR{847174}},
   doi={10.1016/0021-8693(86)90208-5},
}

% \bib{Ca89}{article}{
%  author = {Casselman, W.},
%  journal = {Canad. J. Math.},
%  pages = {385--438},
%  title = {Canonical extensions of Harish-Chandra modules to representations of $G$},
%  volume = {41},
%  year = {1989}
% }



% \bib{Cl}{article}{
%   author = {Du Cloux, F.},
%   journal = {Ann. Sci. \'Ecole Norm. Sup.},
%   number = {3},
%   pages = {257--318},
%   title = {Sur les repr\'esentations diff\'erentiables des groupes de Lie alg\'ebriques},
%   url = {http://eudml.org/doc/82297},
%   volume = {24},
%   year = {1991},
% }

\bib{CM}{book}{
  title = {Nilpotent orbits in semisimple Lie algebra: an introduction},
  author = {Collingwood, D. H.},
  author = {McGovern, W. M.},
  year = {1993},
  publisher = {Van Nostrand Reinhold Co.},
}


% \bib{Dieu}{book}{
%    title={La g\'{e}om\'{e}trie des groupes classiques},
%    author={Dieudonn\'{e}, Jean},
%    year={1963},
% 	publisher={Springer},
%  }

% \bib{DKPC}{article}{
% title = {Nilpotent orbits and complex dual pairs},
% journal = {J. Algebra},
% volume = {190},
% number = {2},
% pages = {518 - 539},
% year = {1997},
% author = {Daszkiewicz, A.},
% author = {Kra\'skiewicz, W.},
% author = {Przebinda, T.},
% }

% \bib{DKP2}{article}{
%   author = {Daszkiewicz, A.},
%   author = {Kra\'skiewicz, W.},
%   author = {Przebinda, T.},
%   title = {Dual pairs and Kostant-Sekiguchi correspondence. II. Classification
% 	of nilpotent elements},
%   journal = {Central European J. Math.},
%   year = {2005},
%   volume = {3},
%   pages = {430--474},
% }


\bib{DM}{article}{
  author = {Dixmier, J.},
  author = {Malliavin, P.},
  title = {Factorisations de fonctions et de vecteurs ind\'efiniment diff\'erentiables},
  journal = {Bull. Sci. Math. (2)},
  year = {1978},
  volume = {102},
  pages = {307--330},
}

%\bibitem[DM]{DM}
%J. Dixmier and P. Malliavin, \textit{Factorisations de fonctions et de vecteurs ind\'efiniment diff\'erentiables}, Bull. Sci. Math. (2), 102 (4),  307-330 (1978).



\bib{Du77}{article}{
  author = {Duflo, M.},
  journal = {Annals of Math.},
  number = {1},
  pages = {107-120},
  title = {Sur la Classification des Ideaux Primitifs Dans
    L'algebre Enveloppante d'une Algebre de Lie Semi-Simple},
  volume = {105},
  year = {1977}
}

% \bib{Du82}{article}{
%  author = {Duflo, M.},
%  journal = {Acta Math.},
%   volume = {149},
%  number = {3-4},
%  pages = {153--213},
%  title = {Th\'eorie de Mackey pour les groupes de Lie alg\'ebriques},
%  year = {1982}
% }



% \bib{GZ}{article}{
% author={Gomez, R.},
% author={Zhu, C.-B.},
% title={Local theta lifting of generalized Whittaker models associated to nilpotent orbits},
% journal={Geom. Funct. Anal.},
% year={2014},
% volume={24},
% number={3},
% pages={796--853},
% }

% \bib{EGAIV2}{article}{
%   title = {\'El\'ements de g\'eom\'etrie alg\'brique IV: \'Etude locale des
%     sch\'emas et des morphismes de sch\'emas. II},
%   author = {Grothendieck, A.},
%   author = {Dieudonn\'e, J.},
%   journal  = {Inst. Hautes \'Etudes Sci. Publ. Math.},
%   volume = {24},
%   year = {1965},
% }


% \bib{EGAIV3}{article}{
%   title = {\'El\'ements de g\'eom\'etrie alg\'brique IV: \'Etude locale des
%     sch\'emas et des morphismes de sch\'emas. III},
%   author = {Grothendieck, A.},
%   author = {Dieudonn\'e, J.},
%   journal  = {Inst. Hautes \'Etudes Sci. Publ. Math.},
%   volume = {28},
%   year = {1966},
% }

\bib{GI}{article}{
   author={Gan, W. T.},
   author={Ichino, A.},
   title={On the irreducibility of some induced representations of real
   reductive Lie groups},
   journal={Tunis. J. Math.},
   volume={1},
   date={2019},
   number={1},
   pages={73--107},
   issn={2576-7658},
   review={\MR{3907735}},
   doi={10.2140/tunis.2019.1.73},
}

\bib{GW}{book}{
   author={Goodman, Roe},
   author={Wallach, Nolan R.},
   title={Symmetry, representations, and invariants},
   series={Graduate Texts in Mathematics},
   volume={255},
   publisher={Springer, Dordrecht},
   date={2009},
   pages={xx+716},
   isbn={978-0-387-79851-6},
   %review={\MR{2522486}},
   %doi={10.1007/978-0-387-79852-3},
}
% \bib{HLS}{article}{
%     author = {Harris, M.},
%     author = {Li, J.-S.},
%     author = {Sun, B.},
%      title = {Theta correspondences for close unitary groups},
%  %booktitle = {Arithmetic Geometry and Automorphic Forms},
%     %series = {Adv. Lect. Math. (ALM)},
%     journal = {Arithmetic Geometry and Automorphic Forms, Adv. Lect. Math. (ALM)},
%     volume = {19},
%      pages = {265--307},
%  publisher = {Int. Press, Somerville, MA},
%       year = {2011},
% }

% \bib{HS}{book}{
%  author = {Hartshorne, R.},
%  title = {Algebraic Geometry},
% publisher={Graduate Texts in Mathematics, 52. New York-Heidelberg-Berlin: Springer-Verlag},
% year={1983},
% }

% \bib{He}{article}{
% author={He, H.},
% title={Unipotent representations and quantum induction},
% journal={arXiv:math/0210372},
% year = {2002},
% }



\bib{Ho}{article}{
author={Hotta, R.},
title={On Joseph's construction of Weyl group representations},
journal={Tohoku Math. J.},
volume={36},
%number = {3},
pages={49--74 },
year={1984},
}




% \bib{Howe79}{article}{
%   title={$\Phi$-series and invariant theory},
%   author={Howe, R.},
%   book = {
%     title={Automorphic Forms, Representations and $L$-functions},
%     series={Proc. Sympos. Pure Math},
%     volume={33},
%     year={1979},
%   },
%   pages={275-285},
% }

% \bib{HoweRank}{article}{
% author={Howe, R.},
% title={On a notion of rank for unitary representations of the classical groups},
% journal={Harmonic analysis and group representations, Liguori, Naples},
% pages={223-331},
% year={1982},
% }

% \bib{Howe89}{article}{
% author={Howe, R.},
% title={Transcending classical invariant theory},
% journal={J. Amer. Math. Soc.},
% volume={2},
% pages={535--552},
% year={1989},
% }

% \bib{Howe95}{article}{,
%   author = {Howe, R.},
%   title = {Perspectives on invariant theory: Schur duality, multiplicity-free actions and beyond},
%   journal = {Piatetski-Shapiro, I. et al. (eds.), The Schur lectures (1992). Ramat-Gan: Bar-Ilan University, Isr. Math. Conf. Proc. 8,},
%   year = {1995},
%   pages = {1-182},
% }



% \bib{HL}{article}{
% author={Huang, J.-S.},
% author={Li, J.-S.},
% title={Unipotent representations attached to spherical nilpotent orbits},
% journal={Amer. J. Math.},
% volume={121},
% number = {3},
% pages={497--517},
% year={1999},
% }


% \bib{HZ}{article}{
% author={Huang, J.-S.},
% author={Zhu, C.-B.},
% title={On certain small representations of indefinite orthogonal groups},
% journal={Represent. Theory},
% volume={1},
% pages={190--206},
% year={1997},
% }


% \bib{H}{book}{
%    author={Humphreys, James E.},
%    title={Representations of semisimple Lie algebras in the BGG category
%    $\scr{O}$},
%    series={Graduate Studies in Mathematics},
%    volume={94},
%    publisher={American Mathematical Society, Providence, RI},
%    date={2008},
%    pages={xvi+289},
%    isbn={978-0-8218-4678-0},
%    review={\MR{2428237}},
%    doi={10.1090/gsm/094},
% }



\bib{Jan}{book}{
   author={Jantzen, J. C.},
   title={Moduln mit einem h\"{o}chsten Gewicht},
   series={Lecture notes in Mathematics},
   volume={750},
   publisher={Springer-Verlag, Berlin/Heidelberg/New York},
   date={1979},
  % pages={xvi+289},
   % isbn={978-0-8218-4678-0},
   % review={\MR{2428237}},
   % doi={10.1090/gsm/094},
}




\bib{J1}{article}{
   author={Joseph, A.},
   title={Goldie rank in the enveloping algebra of a semisimple Lie algebra. I},
   journal={J. Algebra},
   volume={65},
   date={1980},
   number={2},
   pages={269--283},
   issn={0021-8693},
   review={\MR{585721}},
   doi={10.1016/0021-8693(80)90217-3},
}

\bib{J2}{article}{
   author={Joseph, A.},
   title={Goldie rank in the enveloping algebra of a semisimple Lie algebra. II},
   journal={J. Algebra},
   volume={65},
   date={1980},
   number={2},
   pages={284--306},
   issn={0021-8693},
   review={\MR{585721}},
   doi={10.1016/0021-8693(80)90217-3},
}

% \bib{J3}{article}{
%    author={Joseph, A.},
%    title={Goldie rank in the enveloping algebra of a semisimple Lie algebra.
%    III},
%    journal={J. Algebra},
%    volume={73},
%    date={1981},
%    number={2},
%    pages={295--326},
%    issn={0021-8693},
%    review={\MR{640039}},
%    doi={10.1016/0021-8693(81)90324-0},
% }

\bib{J.hw}{article}{
   author={Joseph, A.},
   title={On the variety of a highest weight module},
   journal={J. Algebra},
   volume={88},
   date={1984},
   number={1},
   pages={238--278},
   issn={0021-8693},
   review={\MR{741942}},
   doi={10.1016/0021-8693(84)90100-5},
}


\bib{J.av}{article}{
   author={Joseph, A.},
   title={On the associated variety of a primitive ideal},
   journal={J. Algebra},
   volume={93},
   date={1985},
   number={2},
   pages={509--523},
   issn={0021-8693},
   review={\MR{786766}},
   doi={10.1016/0021-8693(85)90172-3},
}

% \bib{J.ann}{article}{
%    author={Joseph, Anthony},
%    title={Annihilators and associated varieties of unitary highest weight
%    modules},
%    journal={Ann. Sci. \'{E}cole Norm. Sup. (4)},
%    volume={25},
%    date={1992},
%    number={1},
%    pages={1--45},
%    issn={0012-9593},
%    review={\MR{1152612}},
% }


% \bib{JLS}{article}{
% author={Jiang, D.},
% author={Liu, B.},
% author={Savin, G.},
% title={Raising nilpotent orbits in wave-front sets},
% journal={Represent. Theory},
% volume={20},
% pages={419--450},
% year={2016},
% }

\bib{King}{article}{
author={King, D. R.},
title={The character polynomial of the annihilator of an irreducible Harish-Chandra module},
journal={Amer. J. Math.},
volume={103},
%issue ={4},
pages={1195--1240},
year={1981},
}


% \bib{Ki62}{article}{
% author={Kirillov, A. A.},
% title={Unitary representations of nilpotent Lie groups},
% journal={Uspehi Mat. Nauk},
% volume={17},
% issue ={4},
% pages={57--110},
% year={1962},
% }

% \bib{Ko70}{article}{
% author={Kostant, B.},
% title={Quantization and unitary representations},
% journal={Lectures in Modern Analysis and Applications III, Lecture Notes in Math.},
% volume={170},
% pages={87--208},
% year={1970},
% }


% \bib{KP}{article}{
% author={Kraft, H.},
% author={Procesi, C.},
% title={On the geometry of conjugacy classes in classical groups},
% journal={Comment. Math. Helv.},
% volume={57},
% pages={539--602},
% year={1982},
% }

% \bib{KR}{article}{
% author={Kudla, S. S.},
% author={Rallis, S.},
% title={Degenerate principal series and invariant distributions},
% journal={Israel J. Math.},
% volume={69},
% pages={25--45},
% year={1990},
% }


% \bib{Ku}{article}{
% author={Kudla, S. S.},
% title={Some extensions of the Siegel-Weil formula},
% journal={In: Gan W., Kudla S., Tschinkel Y. (eds) Eisenstein Series and Applications. Progress in Mathematics, vol 258. Birkh\"auser Boston},
% %volume={69},
% pages={205--237},
% year={2008},
% }





% \bib{LZ1}{article}{
% author={Lee, S. T.},
% author={Zhu, C.-B.},
% title={Degenerate principal series and local theta correspondence II},
% journal={Israel J. Math.},
% volume={100},
% pages={29--59},
% year={1997},
% }

% \bib{LZ2}{article}{
% author={Lee, S. T.},
% author={Zhu, C.-B.},
% title={Degenerate principal series of metaplectic groups and Howe correspondence},
% journal = {D. Prasad at al. (eds.), Automorphic Representations and L-Functions, Tata Institute of Fundamental Research, India,},
% year = {2013},
% pages = {379--408},
% }

% \bib{Li89}{article}{
% author={Li, J.-S.},
% title={Singular unitary representations of classical groups},
% journal={Invent. Math.},
% volume={97},
% number = {2},
% pages={237--255},
% year={1989},
% }

% \bib{LiuAG}{book}{
%   title={Algebraic Geometry and Arithmetic Curves},
%   author = {Liu, Q.},
%   year = {2006},
%   publisher={Oxford University Press},
% }

% \bib{LM}{article}{
%    author = {Loke, H. Y.},
%    author = {Ma, J.},
%     title = {Invariants and $K$-spectrums of local theta lifts},
%     journal = {Compositio Math.},
%     volume = {151},
%     issue = {01},
%     year = {2015},
%     pages ={179--206},
% }

% \bib{DL}{article}{
%    author={Deligne, P.},
%    author={Lusztig, G.},
%    title={Representations of reductive groups over finite fields},
%    journal={Ann. of Math. (2)},
%    volume={103},
%    date={1976},
%    number={1},
%    pages={103--161},
%    issn={0003-486X},
%    review={\MR{393266}},
%    doi={10.2307/1971021},
% }

% \bib{KL}{article}{
%    author={Kazhdan, David},
%    author={Lusztig, George},
%    title={Representations of Coxeter groups and Hecke algebras},
%    journal={Invent. Math.},
%    volume={53},
%    date={1979},
%    number={2},
%    pages={165--184},
%    issn={0020-9910},
%    review={\MR{560412}},
%    doi={10.1007/BF01390031},
% }

\bib{Lu}{book}{
   author={Lusztig, G.},
   title={Characters of reductive groups over a finite field},
   series={Annals of Mathematics Studies},
   volume={107},
   publisher={Princeton University Press, Princeton, NJ},
   date={1984},
   pages={xxi+384},
   isbn={0-691-08350-9},
   isbn={0-691-08351-7},
   review={\MR{742472}},
   doi={10.1515/9781400881772},
}


% \bib{Lu.I}{article}{
%    author={Lusztig, G.},
%    title={Intersection cohomology complexes on a reductive group},
%    journal={Invent. Math.},
%    volume={75},
%    date={1984},
%    number={2},
%    pages={205--272},
%    issn={0020-9910},
%    review={\MR{732546}},
%    doi={10.1007/BF01388564},
% }


% \bib{LS}{article}{
%    author = {Lusztig, G.},
%    author = {Spaltenstein, N.},
%     title = {Induced unipotent classes},
%     journal = {J. London Math. Soc.},
%     volume = {19},
%     year = {1979},
%     pages ={41--52},
% }


% \bib{Ma}{article}{
%    author = {Mackey, G. W.},
%     title = {Unitary representations of group extentions},
%     journal = {Acta Math.},
%     volume = {99},
%     year = {1958},
%     pages ={265--311},
% }

\bib{Mat96}{article}{
   author={Matumoto, H.},
   title={On the representations of ${\rm U}(m,n)$ unitarily induced from
   derived functor modules},
   journal={Compos. Math.},
   volume={100},
   date={1996},
   number={1},
   pages={1--39},
   issn={0010-437X},
   review={\MR{1377407}},
}

\bib{Mat}{article}{
   author={Matumoto, H.},
   title={On the representations of ${\rm Sp}(p,q)$ and ${\rm SO}^*(2n)$
   unitarily induced from derived functor modules},
   journal={Compos. Math.},
   volume={140},
   date={2004},
   number={4},
   pages={1059--1096},
   issn={0010-437X},
   review={\MR{2059231}},
   doi={10.1112/S0010437X03000629},
}

\bib{Mc}{article}{
   author = {McGovern, W. M.},
    title = {Cells of Harish-Chandra modules for real classical groups},
    journal = {Amer. J.  of Math.},
    volume = {120},
    issue = {01},
    year = {1998},
    pages ={211--228},
}


\bib{Mil}{article}{
   author = {Mili\v{c}i\'c, D.},
    title = {Localizations and representation theory of reductive Lie groups},
    journal = {preprint},
    eprint = {http://www.math.utah.edu/~milicic/Eprints/book.pdf},
   % volume = {120},
    %issue = {01},
    %year = {1998},
   % pages ={211--228},
}


% \bib{Mo96}{article}{
%  author={M{\oe}glin, C.},
%     title = {Front d'onde des repr\'esentations des groupes classiques $p$-adiques},
%     journal = {Amer. J. Math.},
%     volume = {118},
%     issue = {06},
%     year = {1996},
%     pages ={1313--1346},
% }


\bib{MU}{webpage}{
  author={Ma, Jia-Jun},
  title = {Python codes for unipotent representations of classical groups},
  url = {https://github.com/jiajunma/unipotentrepn}
}

\bib{Mo17}{article}{
  author={M{\oe}glin, C.},
  title = {Paquets d'Arthur Sp\'eciaux Unipotents aux Places Archim\'ediennes et Correspondance de Howe},
  journal = {J. Cogdell et al. (eds.), Representation Theory, Number Theory, and Invariant Theory, In Honor of Roger Howe. Progress in Math.}
  %series ={Progress in Math.},
  volume = {323},
  pages = {469--502}
  year = {2017}
}

\bib{MR.C}{article}{
   author={M{\oe}glin, C.},
   author={Renard, D.},
   title={Paquets d'Arthur des groupes classiques complexes},
   language={French, with English and French summaries},
   conference={
      title={Around Langlands correspondences},
   },
   book={
      series={Contemp. Math.},
      volume={691},
      publisher={Amer. Math. Soc., Providence, RI},
   },
   date={2017},
   pages={203--256},
   review={\MR{3666056}},
   doi={10.1090/conm/691/13899},
}

\bib{MR.U}{article}{
   author={M{\oe}glin, C.},
   author={Renard, D.},
   title={Sur les paquets d'Arthur des groupes unitaires et quelques
   cons\'{e}quences pour les groupes classiques},
   language={French, with English and French summaries},
   journal={Pacific J. Math.},
   volume={299},
   date={2019},
   number={1},
   pages={53--88},
   issn={0030-8730},
   review={\MR{3947270}},
   doi={10.2140/pjm.2019.299.53},
}


% \bib{MVW}{book}{
%   volume={1291},
%   title={Correspondances de Howe sur un corps $p$-adique},
%   author={M{\oe}glin, C.},
%   author={Vign\'eras, M.-F.},
%   author={Waldspurger, J.-L.},
%   series={Lecture Notes in Mathematics},
%   publisher={Springer}
%   ISBN={978-3-540-18699-1},
%   date={1987},
% }

% \bib{NOTYK}{article}{
%    author = {Nishiyama, K.},
%    author = {Ochiai, H.},
%    author = {Taniguchi, K.},
%    author = {Yamashita, H.},
%    author = {Kato, S.},
%     title = {Nilpotent orbits, associated cycles and Whittaker models for highest weight representations},
%     journal = {Ast\'erisque},
%     volume = {273},
%     year = {2001},
%    pages ={1--163},
% }

% \bib{NOZ}{article}{
%   author = {Nishiyama, K.},
%   author = {Ochiai, H.},
%   author = {Zhu, C.-B.},
%   journal = {Trans. Amer. Math. Soc.},
%   title = {Theta lifting of nilpotent orbits for symmetric pairs},
%   volume = {358},
%   year = {2006},
%   pages = {2713--2734},
% }


% \bib{NZ}{article}{
%    author = {Nishiyama, K.},
%    author = {Zhu, C.-B.},
%     title = {Theta lifting of unitary lowest weight modules and their associated cycles},
%     journal = {Duke Math. J.},
%     volume = {125},
%     number= {03},
%     year = {2004},
%    pages ={415--465},
% }



% \bib{Ohta}{article}{
%   author = {Ohta, T.},
%   %doi = {10.2748/tmj/1178227492},
%   journal = {Tohoku Math. J.},
%   number = {2},
%   pages = {161--211},
%   publisher = {Tohoku University, Mathematical Institute},
%   title = {The closures of nilpotent orbits in the classical symmetric
%     pairs and their singularities},
%   volume = {43},
%   year = {1991}
% }

% \bib{Ohta2}{article}{
%   author = {Ohta, T.},
%   journal = {Hiroshima Math. J.},
%   number = {2},
%   pages = {347--360},
%   title = {Induction of nilpotent orbits for real reductive groups and associated varieties of standard representations},
%   volume = {29},
%   year = {1999}
% }

% \bib{Ohta4}{article}{
%   title={Nilpotent orbits of $\mathbb{Z}_4$-graded Lie algebra and geometry of
%     moment maps associated to the dual pair $(\mathrm{U} (p, q), \mathrm{U} (r, s))$},
%   author={Ohta, T.},
%   journal={Publ. RIMS},
%   volume={41},
%   number={3},
%   pages={723--756},
%   year={2005}
% }

% \bib{PT}{article}{
%   title={Some small unipotent representations of indefinite orthogonal groups and the theta correspondence},
%   author={Paul, A.},
%   author={Trapa, P.},
%   journal={University of Aarhus Publ. Series},
%   volume={48},
%   pages={103--125},
%   year={2007}
% }


% \bib{PV}{article}{
%   title={Invariant Theory},
%   author={Popov, V. L.},
%   author={Vinberg, E. B.},
%   book={
%   title={Algebraic Geometry IV: Linear Algebraic Groups, Invariant Theory},
%   series={Encyclopedia of Mathematical Sciences},
%   volume={55},
%   year={1994},
%   publisher={Springer},}
% }




%\bib{PPz}{article}{
%author={Protsak, V.} ,
%author={Przebinda, T.},
%title={On the occurrence of admissible representations in the real Howe
%    correspondence in stable range},
%journal={Manuscr. Math.},
%volume={126},
%number={2},
%pages={135--141},
%year={2008}
%}


% \bib{PrzInf}{article}{
%       author={Przebinda, T.},
%        title={The duality correspondence of infinitesimal characters},
%         date={1996},
%      journal={Colloq. Math.},
%       volume={70},
%        pages={93--102},
% }


% \bib{Pz}{article}{
% author={Przebinda, T.},
% title={Characters, dual pairs, and unitary representations},
% journal={Duke Math. J. },
% volume={69},
% number={3},
% pages={547--592},
% year={1993}
% }

% \bib{Ra}{article}{
% author={Rallis, S.},
% title={On the Howe duality conjecture},
% journal={Compositio Math.},
% volume={51},
% pages={333--399},
% year={1984}
% }

\bib{RT1}{article}{
   author={Renard, D.},
   author={Trapa, P.},
   title={Irreducible genuine characters of the metaplectic group:
   Kazhdan-Lusztig algorithm and Vogan duality},
   journal={Represent. Theory},
   volume={4},
   date={2000},
   pages={245--295},
   review={\MR{1795754}},
   doi={10.1090/S1088-4165-00-00105-9},
}

\bib{RT2}{article}{
   author={Renard, D.},
   author={Trapa, P.},
   title={Irreducible characters of the metaplectic group. II.
   Functoriality},
   journal={J. Reine Angew. Math.},
   volume={557},
   date={2003},
   pages={121--158},
   issn={0075-4102},
   review={\MR{1978405}},
   doi={10.1515/crll.2003.028},
}

% \bib{RT3}{article}{
%    author={Renard, David A.},
%    author={Trapa, Peter E.},
%    title={Kazhdan-Lusztig algorithms for nonlinear groups and applications
%    to Kazhdan-Patterson lifting},
%    journal={Amer. J. Math.},
%    volume={127},
%    date={2005},
%    number={5},
%    pages={911--971},
%    issn={0002-9327},
%    review={\MR{2170136}},
% }


% \bib{Sa}{article}{
% author={Sahi, S.},
% title={Explicit Hilbert spaces for certain unipotent representations},
% journal={Invent. Math.},
% volume={110},
% number = {2},
% pages={409--418},
% year={1992}
% }

% \bib{Se}{article}{
% author={Sekiguchi, J.},
% title={Remarks on real nilpotent orbits of a symmetric pair},
% journal={J. Math. Soc. Japan},
% %publisher={The Mathematical Society of Japan},
% year={1987},
% volume={39},
% number={1},
% pages={127--138},
% }

\bib{SV}{article}{
author = {Schmid, W.},
author = {Vilonen, K.},
journal = {Annals of Math.},
number = {3},
pages = {1071--1118},
%publisher = {Princeton University, Mathematics Department, Princeton, NJ; Mathematical Sciences Publishers, Berkeley},
title = {Characteristic cycles and wave front cycles of representations of reductive Lie groups},
volume = {151},
year = {2000},
}


\bib{Soergel}{article}{
   author={Soergel, W.},
   title={Kategorie $\scr O$, perverse Garben und Moduln \"{u}ber den
   Koinvarianten zur Weylgruppe},
   language={German, with English summary},
   journal={J. Amer. Math. Soc.},
   volume={3},
   date={1990},
   number={2},
   pages={421--445},
   issn={0894-0347},
   review={\MR{1029692}},
   doi={10.2307/1990960},
}


\bib{So}{article}{
author = {Sommers, E.},
title = {Lusztig's canonical quotient and generalized duality},
journal = {J. Algebra},
volume = {243},
number = {2},
pages = {790--812},
year = {2001},
}

% \bib{SS}{book}{
%   author = {Springer, T. A.},
%   author = {Steinberg, R.},
%   title = {Seminar on algebraic groups and related finite groups; Conjugate classes},
%   series = {Lecture Notes in Math.},
%   volume = {131},
% publisher={Springer},
% year={1970},
% }

% \bib{SZ1}{article}{
% title={A general form of Gelfand-Kazhdan criterion},
% author={Sun, B.},
% author={Zhu, C.-B.},
% journal={Manuscripta Math.},
% pages = {185--197},
% volume = {136},
% year={2011}
% }


%\bib{SZ2}{article}{
%  title={Conservation relations for local theta correspondence},
%  author={Sun, B.},
%  author={Zhu, C.-B.},
%  journal={J. Amer. Math. Soc.},
%  pages = {939--983},
%  volume = {28},
%  year={2015}
%}

\bib{Tr.U}{article}{
   author={Trapa, P.},
   title={Annihilators and associated varieties of $A_{\mathfrak q}(\lambda)$
   modules for $\mathrm U(p,q)$},
   journal={Compositio Math.},
   volume={129},
   date={2001},
   number={1},
   pages={1--45},
   issn={0010-437X},
   review={\MR{1856021}},
   doi={10.1023/A:1013115223377},
}

\bib{Tr.H}{article}{
  title={Special unipotent representations and the Howe correspondence},
  author={Trapa, P.},
  year = {2004},
  journal={University of Aarhus Publication Series},
  volume = {47},
  pages= {210--230}
}

% \bib{Wa}{article}{
%    author = {Waldspurger, J.-L.},
%     title = {D\'{e}monstration d'une conjecture de dualit\'{e} de Howe dans le cas $p$-adique, $p \neq 2$ in Festschrift in honor of I. I. Piatetski-Shapiro on the occasion of his sixtieth birthday},
%   journal = {Israel Math. Conf. Proc., 2, Weizmann, Jerusalem},
%  year = {1990},
% pages = {267-324},
% }

\bib{VGK}{article}{
   author={Vogan, D. A.},
   title={Gel\cprime fand-Kirillov dimension for Harish-Chandra modules},
   journal={Invent. Math.},
   volume={48},
   date={1978},
   number={1},
   pages={75--98},
   issn={0020-9910},
   review={\MR{506503}},
   doi={10.1007/BF01390063},
}



\bib{Vg}{book}{
   author={Vogan, D. A.},
   title={Representations of real reductive Lie groups},
   series={Progress in Mathematics},
   volume={15},
   publisher={Birkh\"{a}user, Boston, Mass.},
   date={1981},
   pages={xvii+754},
   isbn={3-7643-3037-6},
   review={\MR{632407}},
}

\bib{V1}{article}{
   author={Vogan, D. A.},
   title={Irreducible characters of semisimple Lie groups. I},
   journal={Duke Math. J.},
   volume={46},
   date={1979},
   number={1},
   pages={61--108},
   issn={0012-7094},
   review={\MR{523602}},
}

\bib{V3}{article}{
   author={Vogan, D. A.},
   title={Irreducible characters of semisimple Lie groups. III. Proof of Kazhdan-Lusztig conjecture in the integral case},
   journal={Invent. Math.},
   volume={71},
   date={1983},
   number={2},
   pages={381--417},
}

\bib{V4}{article}{
   author={Vogan, D. A.},
   title={Irreducible characters of semisimple Lie groups. IV.
   Character-multiplicity duality},
   journal={Duke Math. J.},
   volume={49},
   date={1982},
   number={4},
   pages={943--1073},
   issn={0012-7094},
   review={\MR{683010}},
}


\bib{V.GL}{article}{
   author={Vogan, D. A.},
   title={The unitary dual of ${\rm GL}(n)$ over an Archimedean field},
   journal={Invent. Math.},
   volume={83},
   date={1986},
   number={3},
   pages={449--505},
   issn={0020-9910},
   review={\MR{827363}},
   doi={10.1007/BF01394418},
}

\bib{VoBook}{book}{
author = {Vogan, D. A.},
  title={Unitary representations of reductive Lie groups},
  year={1987},
  series = {Ann. of Math. Stud.},
 volume={118},
  publisher={Princeton University Press}
}


\bib{Vo89}{article}{
  author = {Vogan, D. A.},
  title = {Associated varieties and unipotent representations},
 %booktitle ={Harmonic analysis on reductive groups, Proc. Conf., Brunswick/ME (USA) 1989,},
  journal = {Harmonic analysis on reductive groups, Proc. Conf., Brunswick/ME
    (USA) 1989, Prog. Math.},
 volume={101},
  publisher = {Birkh\"{a}user, Boston-Basel-Berlin},
  year = {1991},
pages={315--388},
  editor = {W. Barker and P. Sally},
}

% \bib{Vo98}{article}{
%   author = {Vogan, D. A. },
%   title = {The method of coadjoint orbits for real reductive groups},
%  %booktitle ={Representation theory of Lie groups (Park City, UT, 1998)},
%  journal = {Representation theory of Lie groups (Park City, UT, 1998). IAS/Park City Math. Ser.},
%   volume={8},
%   publisher = {Amer. Math. Soc.},
%   year = {2000},
% pages={179--238},
% }




% \bib{Vo00}{article}{
%   author = {Vogan, D. A. },
%   title = {Unitary representations of reductive Lie groups},
%  %booktitle ={Mathematics towards the Third Millennium (Rome, 1999)},
%  journal ={Mathematics towards the Third Millennium (Rome, 1999). Accademia Nazionale dei Lincei, (2000)},
%   %series = {Accademia Nazionale dei Lincei, 2000},
%  %volume={9},
% pages={147--167},
% }


% \bib{Wa1}{book}{
%   title={Real reductive groups I},
%   author={Wallach, N. R.},
%   year={1988},
%   publisher={Academic Press Inc. }
% }

\bib{Wa2}{book}{
title={Real reductive groups II},
author={Wallach, N. R.},
year={1992},
publisher={Academic Press Inc. }
}


% \bib{Weyl}{book}{
%   title={The classical groups: their invariants and representations},
%   author={Weyl, H.},
%   year={1947},
%   publisher={Princeton University Press}
% }

% \bib{Ya}{article}{
%   title={Degenerate principal series representations for quaternionic unitary groups},
%   author={Yamana, S.},
%   year = {2011},
%   journal={Israel J. Math.},
%   volume = {185},
%   pages= {77--124}
% }


\bib{Zu}{article}{
  title={Tensor products of finite and infinite dimensional representations of semisimple Lie groups},
  author={Zuckerman, G.},
  year = {1977},
  journal={Ann. of Math. 106},
  volume = {106},
  pages= {295--308}
}


% \bib{EGAIV4}{article}{
%   title = {\'El\'ements de g\'eom\'etrie alg\'brique IV 4: \'Etude locale des
%     sch\'emas et des morphismes de sch\'emas},
%   author = {Grothendieck, Alexandre},
%   author = {Dieudonn\'e, Jean},
%   journal  = {Inst. Hautes \'Etudes Sci. Publ. Math.},
%   volume = {32},
%   year = {1967},
%   pages = {5--361}
% }



\end{biblist}
\end{bibdiv}


\end{document}


%%% Local Variables:
%%% coding: utf-8
%%% mode: latex
%%% TeX-engine: tex
%%% ispell-local-dictionary: "en_US"
%%% End:
