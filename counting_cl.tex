\documentclass[counting_main.tex]{subfiles}
\begin{document}

\section{Counting in type BCD}

In this section, we consider the case when %$\ckstar \in \set{B,C,D}$, i.e
$\star \in \set{B,\wtC, C,D,C^{*}, D^{*}}$.

% We identify $\fhh^{*}$ with $\bZ^{n}$ where $n = \rank(\Gc)$
% and let $\rho$ be the half sum of all positive roots.

Recall that
\[
  \text{good parity} =
\begin{cases}
 %\text{odd} & \text{when } \ckstar\in \set{B,D}\\
 %\text{even} & \text{when } \ckstar = C\\
 \text{odd} & \text{when } \star \in \set{C,C^{*},D,D^{*}}\\
 \text{even} & \text{when } \star \in \set{B,\wtC}\\
\end{cases}
\]
For $\ckcO\in \Nil(\ckcG)$, we have the decomposition
\[
  \ckcO = \ckcOb\cuprow \ckcOg \AND \ckcOb = \ckcOpb\cuprow \ckcOpb
\]
where $\ckcOpb$ is a partition of $\half\abs{\ckcOb}$.

%We have $\ckcG_{\lamck} = \ckcG_{b}\otimes \ckcG_{g}$.
Let $n_{b}$ and $n_{g}$ be the rank of $\ckcG_{b}$ and $\ckcG_{g}$
respectively. We have
  \[
    (\nbb,\ngg) =
    \begin{cases}
      (\half \abs{\ckcO_{b}}, \half(\abs{\ckcO_{g}}-1)) & \text{when
      } \star \in \set{C,C^{*}}\\
      (\half \abs{\ckcO_{b}}, \half\abs{\ckcO_{g}}) & \text{when
      } \star \in \set{B,\wtC,D,D^{*}}\\
    \end{cases}
  \]
  and integral Weyl group $\WLamck$is a product of two factors
  \[
    W_{[\lamck]} =\Wb\times \Wg
  \]
  where
  \begin{equation}\label{eq:Wbg}
    \begin{split}
    \Wb & := \begin{cases}
      \sfW_{n_{b}}  & \text{when } \star \in \set{B, \wtC} \\
      \sfW'_{n_{b}} & \text{when } \star \in \set{C,C^{*},D,D^{*}}
      \end{cases}\\
    \Wg & := \begin{cases}
      \sfW_{n_{g}}  & \text{when } \star \in \set{B,C, C^{*} } \\
      \sfW'_{n_{g}} & \text{when } \star \in \set{\wtC,D,D^{*}}
      \end{cases}
    \end{split}
  \end{equation}

  When $\Wb$ or $\Wg$ is a Weyl group of type $D_{n}$, we always have the
  preferred embedding of $\sfS_{n}$ into $\sfW'_{n}$ given by the root system of
  $\ckfgg$. The label $I$ on the irreducible character of $\sfW_{n}$ is refer to
  this particular embedding.

  More precisely, we identify bipartition with $n$ parts with $\Irr(\sfW_{n})$.
  To ease the notations, we let $(\tau_{L},\tau_{R})_{I}$ denote
  the unique irreducible character of $\sfW'_{n}$ given by
  \begin{itemize}
    \item the restriction of
    the irreducible character of $\sfW_{n}$ attached to $(\tau_{L},\tau_{R})$ if
    $\tau_{L}\neq \tau_{R}$, and
    \item
    the character
    $\Ind_{\sfS_{\frac{n}{2}}}^{\sfW_{n}} \tau_{L}$ if $\tau_{L}=\tau_{R}$.
  \end{itemize}
  We remark that we always have
  \[
    (\tau_{L},\tau_{R})_{I}=(\tau_{R},\tau_{L})_{I}
  \]
  as $\sfW'_{n}$-character.


  \subsection{The left cell}
  \label{sec:LCBCD}
  In this subsection, we described the Lusztig left cell attached to
  $\lambda_{\ckcO}$ in each cases, where
  $\star \in \set{B,C,\wtC,C^{*},D,D^{*}}$.


  To state the results, we made some definitions first. Define the irreducible
  $W_{b}$ representation attached to $\ckcO_{b}$ by the following formula
  \begin{equation}\label{eq:taub}
    \begin{split}
      \tau_{b} & := (\tau_{L,b},\tau_{R,b})\\
      &:= \begin{cases}
        \Big(\big(\half(\bfrr_{2}(\ckcO_{b})+1), \half(\bfrr_{4}(\ckcO_{b})+1), \cdots, \half(\bfrr_{2c}(\ckcO_{b})+1)\big),\\
        \hspace{1em}\big(\half(\bfrr_{2}(\ckcO_{b})-1), \half(\bfrr_{4}(\ckcO_{b})-1), \cdots, \half(\bfrr_{2c}(\ckcO_{b})-1) \big)\Big)
        & \text{if } \star \in \set{B,\wtC},\\
        \Big( \big(\half\bfrr_{2}(\ckcO_{b}), \half\bfrr_{4}(\ckcO_{b}),\cdots, \half\bfrr_{2c}(\ckcO_{b})\big), \\
        \hspace{1em} \big(\half\bfrr_{2}(\ckcO_{b}), \half\bfrr_{4}(\ckcO_{b}),\cdots, \half\bfrr_{2c}(\ckcO_{b}) \big)\Big)_{I}
        & \text{if } \star \in \set{C,C^{*}, D,D^{*}},\\
      \end{cases}
    \end{split}
  \end{equation}
  with $2c = \bfcc_{1}(\ckcO_{b})$. Define partitions
  \begin{itemize}
    \item $\ckcO'_{b}$ such that $\bfrr_{i}(\ckcO'_{b}):= \bfrr_{2i}(\ckcO_{b})$
          for all $i\in \bN^{+}$,
    \item $\cO'_{b}:= (\ckcO'_{b})^{t}$, and
    \item $\ckcO_{b}:= \cO'_{b}\cupcol \cO'_{b}$.
  \end{itemize}

  Set
  \[
    \CPPs(\ckcO_{g}) =
    \begin{cases}
      \set{(2i-1,2i)| \bfrr_{2i-1}(\ckcO_{g})- \bfrr_{2i}(\ckcO_{g})\geq
        2, %\text{and}
        i\in \bN^{+}} & \text{if $\star\in \Set{C,\wtC,C^{*}}$}\\
      \set{(2i,2i+1)| \bfrr_{2i}(\ckcO_{g})- \bfrr_{2i+1}(\ckcO_{g})\geq
        2, %\text{and }
        i\in \bN^{+}} & \text{if $\star\in \Set{B,D,D^{*}}$}.
    \end{cases}
  \]
  Let
  \[
    \wtA(\ckcO) := \bF_{2}[\CPP(\ckcO_{g})]
  \] be the power set of $\CPPs(\ckcO_{g})$.

  For each $\wp\in \wtA(\ckcO)$ we define an element $\tau_{\wp}$ in
  $\Irr(\Wg)$. Here
  \[
    \tau_{\wp} :=
    \begin{cases}
      (\imathp,\jmathp) & \text{when } \star \in \set{B,C, C^{*} } \\
      (\imathp,\jmathp)_{I} & \text{when } \star \in \set{\wtC,D,D^{*}}
    \end{cases}
  \]
  and $(\imathp, \jmathp)$ are given by the following formulas:
  \begin{itemize}
    \item Suppose $\star\in \set{C,C^{*}}$ and let
          $l=\min\set{i| \bfrr_{2i}(\ckcO_{g})=0}$. Then
          \[
          (\bfcc_{l}(\imathp), \bfcc_{l}(\jmathp)) := (0,\half(\bfrr_{2l+1}(\ckcO_{g})-1))
          \]
          and, for all $1\leq i< l$
          \[
          (\bfcc_{i}(\imathp), \bfcc_{i}(\jmathp)):=
          \begin{cases}
            (\half (\bfrr_{2i}(\ckcO_{g})+1), \half (\bfrr_{2i-1}(\ckcO_{g})-1))
            & \text{if } (2i-1,2i)\notin \wp,\\
            (\half (\bfrr_{2i-1}(\ckcO_{g})+1),\half (\bfrr_{2i}(\ckcO_{g})-1)) & \text{otherwise.}
          \end{cases}
          \]
    \item Suppose $\star\in \set{D,D^{*}}$ and let
          $l=\min\set{i| \bfrr_{2i+1}(\ckcO_{g})=0}$. Then
          \[
          \begin{split}
            \bfcc_{1}(\imathp) &:=
            \half(\bfrr_{1}(\ckcO_{g})+1)\\
            (\bfcc_{l+1}(\imathp), \bfcc_{l}(\jmathp)) &:= (0,\half(\bfrr_{2l}(\ckcO_{g})-1))
          \end{split}
          \]
          and, for all $1\leq i<l$
          \[
          (\bfcc_{i+1}(\imathp), \bfcc_{i}(\jmathp)):=
          \begin{cases}
            \left(\half (\bfrr_{2i+1}(\ckcO_{g})+1), \half (\bfrr_{2i}(\ckcO_{g})-1)\right)
            & \text{if } (2i,2i+1)\notin \wp,\\
            (\half (\bfrr_{2i}(\ckcO_{g})+1),\half (\bfrr_{2i+1}(\ckcO_{g})-1)) & \text{otherwise.}
          \end{cases}
          \]
          % \[
          %   (\bfcc_{i}(\imathp), \bfcc_{i}(\jmathp)):=
          %   \begin{cases}
          %     (\half (\bfrr_{2i-1}(\ckcO_{g})+1),\half (\bfrr_{2i}(\ckcO_{g})-1)) &\text{if } (2i-1,2i)\in \wp, \\
          %     (\half (\bfrr_{2i}(\ckcO_{g})+1), \half (\bfrr_{2i-1}(\ckcO_{g})-1))
          %     & \text{if } (2i-1,2i)\notin \wp\\
          %     & \text{ and }\bfrr_{2i}(\ckcO_{g})\neq 0,
          %     \\
          %     (0,0)
          %     & \text{if } \bfrr_{2i-1}(\ckcO_{g})=0,\\
          %     (0, \half (\bfrr_{2i-1}(\ckcO_{g})-1)) & \text{otherwise}
          %   \end{cases}
          % \]

          % \[
          %   (\bfcc_{l+1}(\imathp), \bfcc_{l+1}(\jmathp)) := (0,\half(\bfrr_{2l+1}(\ckcO_{g})-1))
          % \]
          % and for all $1\leq i\leq l$
    \item Suppose $\star=B$. Then
          \[
          \bfcc_{1}(\jmathp) := \half\bfrr_{1}(\ckcO_{g})
          \]
          and for all $i\geq 1$
          \[
          (\bfcc_{i}(\imathp), \bfcc_{i+1}(\jmathp)):=
          \begin{cases}
            (\half \bfrr_{2i}(\ckcO_{g}), \half \bfrr_{2i+1}(\ckcO_{g}))
            & \text{if } (2i,2i+1)\notin \wp,\\
            (\half \bfrr_{2i+1}(\ckcO_{g}),\half \bfrr_{2i}(\ckcO_{g})) & \text{otherwise.}
          \end{cases}
          \]
    \item Suppose $\star = \wtC$. Then for all $i\geq 1$
          \[
          (\bfcc_{i}(\imathp), \bfcc_{i}(\jmathp)):=
          \begin{cases}
            (\half \bfrr_{2i-1}(\ckcO_{g}), \half \bfrr_{2i}(\ckcO_{g}))
            & \text{if } (2i-1,2i)\notin \wp,\\
            (\half \bfrr_{2i}(\ckcO_{g}),\half \bfrr_{2i-1}(\ckcO_{g})) & \text{otherwise.}
          \end{cases}
          \]
  \end{itemize}

  For $\wp\subset\CPPs(\ckcO_{g})$, let $\wp^{c}$ be the complement of $\wp$ in
  $\CPPs(\ckcO_{g})$ and we have $\tau_{\wp} = \tau_{\wp^{c}}$ if
  % $\star \in \Set{\wtC,D,D^{*}}$. ### This is not correct!
  $\star = \wtC$.

  We define
  \[
    \barA(\ckcO)=
    \begin{cases}
      \wtA(\ckcO)/\wp\sim\wp^{c} & \text{when } \star =\wtC,\\
      \wtA(\ckcO) & \text{otherwise.}\\
    \end{cases}
    % \begin{cases}
    %   \wtA(\ckcO)/\wp\sim\wp^{c} & \text{when } \star \in \set{\wtC,D,D^{*}}.\\
    %   \wtA(\ckcO) & \text{when } \star \in \set{B,C,C^{*}},\\
    % \end{cases}
  \]
  Here $\wtA(\ckcO)/\wp\sim\wp^{c}$ denote the quotient of $\wtA(\ckcO)$ by
  identifying $\wp$ with its complement $\wp^{c}$.

  When $\star\neq \wtC$, $\barA(\ckcO)$ is nothing but the Lusztig canonical
  quotient attached to $\ckcO$. \trivial{ This can be seem from the following
    lemma, c.f. \cite{BVUni}*{Proposition~5.28}. }

  Note that by definition, we have $\wtA(\ckcO)=\wtA(\ckcO_{g})$ and
  $\barA(\ckcO)=\barA(\ckcO_{g})$.

  Recall that
  \[
    \LV_{\ckcO}:= \left(J_{\Wlamck}^{\WLamck} \sgn\right) \otimes \sgn.
  \]
  and $\LC_{\ckcO}$ is the multiset of irreducible components

  \begin{lem}[c.f. Barbasch-Vogan{\cite{BVUni}*{Proposition~5.28}}]
    \label{lem:Lcell}
    In all the cases, $\LC_{\ckcO}$ is multiplicity free and we have the
    following bijections
    \[
      \begin{array}{lccccccc}
        \barA(\ckcO)&=&\barA(\ckcO_{g}) & \longrightarrow & \LC(\ckcO_{g})
        & \longrightarrow & \LC(\ckcO)\\
                    &  &\wp & \mapsto & \tau_{\wp} &
                                                     \mapsto & \tau_{b}\otimes \tau_{\wp}.
      \end{array}
    \]
    Moreover,
    \[
      \tau_{\ckcO}=\tau_{b}\otimes \tau_{\emptyset}
    \] is the unique special representation in $\LC_{\ckcO}$ and
    \begin{equation}\label{eq:dBV.W}
      \Spr(j_{\WLamck}^{W}(\tau_{\ckcO})) = \dBV(\ckcO) = \cO_{b}\cupcol \dBV(\ckcO_{g}).
    \end{equation}
    \trivial[]{ The last equality could be checked using Sommer's formula on
      Springer correspondence directly: double columns $(2c+1,2c+1)$ corresponds
      to
      $ B_{c=\alpha_{2i-1}}\times D_{c+1=\alpha_{2i}+1}=D_{c+1=\alpha_{2i-1}+1}\times C_{c=\alpha_{2i}}$
      factor in type $B,\wtC$. double columns $(2c,2c)$ corresponds to factor
      $D_{c=\beta_{2i-1}}\times C_{c=\beta_{2i}}=D_{c=\beta_{2i-1}}\times B_{c=\beta_{2i}}$
      in type $C,C^{*},D,D^{*}$. } Here $\dBV$ is the metaplectic dual if
    $\star=\wtC$ and is the Barbasch-Vogan dual otherwise.
  \end{lem}
  \begin{proof}
    For $\ckcO_{g}$, the lemma is given by \cite{BVUni}*{Proposition~5.28}. For
    all the cases, the lemma follow from an induction on number of columns using
    Lusztig's formula of $J$-induction in \cite{Lu}*{\S 4.4-4.6}. The equality
    \eqref{eq:dBV.W} is due to Barbasch-Vogan for linear groups
    \cite{BVUni}*{Proposition~A2}.

    \trivial[h]{ {\bf Suppose $\star=C$.}

      In this case, bad parity is even and each row length occur with even
      multiplicity. Suppose
      $\ckcO_{b} = (C_{1}, C_{1}, C_{2},C_{2}, \cdots, C_{k'},C_{k'})$ with
      $c_{1}=2k$ and $k' = \bfrr_{1}(\ckcO_{b})$.
      \[
        W_{\lamckb} = S_{C_{1}}\times S_{C_{2}}\times \cdots S_{C_{k'}}.
      \]
      The symbol of trivial representation of trivial group of type D is
      \[
        \binom{0,1, \cdots, k-1}{0,1, \cdots, k-1}.
      \]
      Now it is easy to see that (use the similar computation as below)
      \[
        J_{W_{\lamckb}}^{W_{b}}\sgn = ((\half C_{1}, \half C_{2},\cdots, \half C_{k'}),(\half C_{1}, \half C_{2},\cdots, \half C_{k'})).
      \]


      For the good parity part. Let
      $r'_{i} = \floor{\half(\bfrr_{i}(\ckcO_{g})-\bfrr_{i+1}(\ckcO_{g}))}$.
      Suppose $\ckcO_{g}$ has $2l+1$ columns (superscripts denote the
      multiplicity)
      \[
        \ckcO_{g} = ((2l+1)^{2r'_{2l+1}+1}, 2l^{2r'_{2l}}, (2l-1)^{2r'_{2l-1}}, \cdots, 2^{2r'_{2}}, 1^{2r'_{1}} )
      \]
      and
      % $\ckcO_{g} = (2c_{1}+1, C_{2}, C_{2},C_{3},C_{3},\cdots, C_{k'},C_{k'})$
      % with $2c_{1}+1=2l+1$ and $2k'+1 = \bfrr_{1}(\ckcO_{g})$.
      \[
        W_{\lamckg} = W_{l}\times \underbrace{S_{2l+1}\times \cdots \times S_{2l+1}}_{2r'_{2l+1}\text{-terms}} \times \prod_{i<2l+1} \underbrace{S_{i}\times \cdots\times S_{i}}_{r'_{i}\text{-terms}}
      \]

      The symbol of sign representation of $W_{l}$ is
      \[
        \binom{0,1,2, \cdots, l}{1,2, \cdots, l}.
      \]
      The induction begins with the longest columns to the shorter columns

      Induce to include all $2l+1$-length columns yields
      \[
        \binom{r'_{2l+1}+0,r'_{2l+1}+1,r'_{2l+1}+2, \cdots, r'_{2l+1}+l}{ r'_{2l+1}+1,r'_{2l+1}+2, \cdots, r'_{2l+1}+l}.
      \]
      Now move the the shorter columns, we see that when even columns
      $(2i)^{2r'_{2i}}$ occurs, it adds $(i)^{r'_{2i}}$ columns on the both
      sides of the bipartition; when odd columns $(2i+1)^{r'_{2i+1}}$ occur, the
      bifurcation happens: one can
      \begin{itemize}
        \item attach columns $(i+1)^{r'_{2i+1}}$ on the left and columns
              $(i)^{r'_{2i+1}}$ on the right, which corresponds to
              $(2i+1,2i+2)\neq \wp$, or
        \item attach columns $(i)^{r'_{2i+1}}$ on the left and columns
              $(i+1)^{r'_{2i+1}}$ on the right, which corresponds to
              $(2i+1,2i+2)\in \wp$,
      \end{itemize}

      Therefore,
      \[
        \begin{array}{ccc}
          J_{W_{\lamckg}}^{W_{g}} \sgn
          &\leftrightarrow&  \bF_{2}(\CPP(\ckcO_{g}))\\
          (\cktau_{L},\cktau_{R}) =:\cktau_{\wp}&\leftrightarrow & \wp
        \end{array}
      \]
      where
      \[
        \bfrr_{l+1}(\cktau_{L}) = r'_{2l+1} = \half (\bfrr_{2l+1}(\ckcO_{g})-1)
      \]
      and, if $(2i-1,2i)\notin \wp$,
      \[
        \begin{split}
          \bfrr_{i}(\cktau_{L}) & = \sum_{l\geq 2i-1} r'_{l}
          = \half(\bfrr_{2i-1}(\ckcO)-1)\\
          \bfrr_{i}(\cktau_{R}) & = 1 + \sum_{l\geq 2i} r'_{l} = \half(\bfrr_{2i}(\ckcO)+1)
        \end{split}
      \]
      if $(2i-1,2i)\in \wp$,
      \[
        \begin{split}
          \bfrr_{i}(\cktau_{L}) & = \sum_{l\geq 2i} r'_{l}
          = \half(\bfrr_{2i}(\ckcO)-1)\\
          \bfrr_{i}(\cktau_{R}) & = 1 + \sum_{l\geq 2i-1} r'_{l} = \half(\bfrr_{2i-1}(\ckcO)+1)
        \end{split}
      \]

      % \[
      %   \begin{split}
      %     \bfrr_{l+1}(\cktau_{L}) & = r'_{2l+1} =
      %     \half (\bfrr_{2l+1}(\ckcO_{g})-1)\\
      %     (\bfrr_{i}(\cktau_{L}), \bfrr_{i}(\cktau_{R})) & =
      %     \begin{cases}
      %       (\half(\bfrr_{2i-1}(\ckcO_{g})-1), \half(\bfrr_{2i}(\ckcO_{g})+1)) & (2i-1,2i)\notin \wp\\
      %       (\half(\bfrr_{2i}(\ckcO_{g})-1), \half(\bfrr_{2i-1}(\ckcO_{g})+1)) & (2i-1,2i)\in \wp
      %     \end{cases}
      %   \end{split}
      % \]

      Since $\tau_{\wp} = \cktau_{\wp}\otimes \sgn$, we get the claim.

      We adopt the convention that
      \[
        \sfS_{\cO} := \prod_{i\in \bN^{+}}\sfS_{\bfcc_{i}(\cO)}
      \]
      so that $j_{\sfS_{\cO}}^{\sfS_{\abs{\cO}}}\sgn = \cO$ for each partition
      $\cO$.

      Now consider the orbit under the Springer correspondence.

      Let
      $\ckcO'_{b}: = [\bfrr_{2}(\ckcO_{b}), \bfrr_{4}(\ckcO_{b}),\cdots, \bfrr_{2k}(\ckcO_{b})]$,
      $\cO'_{b}:=(\ckcO'_{b})^{t}$ and $\cO_{b}:=\cO'_{b}\cupcol \cO'_{b}$.
      Clearly, $\ckcO_{b} = \ckcO'_{b}\cuprow \ckcO'_{b}$. Note that
      $\tau_{b} = j_{S_{\cO'_{b}}}^{W'_{b}} \sgn$ (by the formula of fake degree
      see Lusztig or Carter's book). So, by induction by stage of $j$-induction,
      we have
      \[
        \wttau_{\cO}:= j_{W'_{b}\times W_{g}}^{W_{n}} (\tau_{b}\otimes \tau_{\emptyset}) = j_{S_{\cO'_{b}}\times W_{g}}^{W_{n}} \sgn\otimes \tau_{\wp}.
      \]
      By Barbasch-Vogan, $\cO_{g}:=\Spr(\tau_{\emptyset}) = d_{BV}(\ckcO_{g})$,
      which is well know how to calculate. (In fact, one can deduce the result
      by our computation. )

      Since the Springer correspondence commutes with parabolic induction, we
      get
      $\Spr(\wttau) = \Ind_{\GL_{\cO'_{b}}\times \Sp(2g)}^{\Sp(2n)} 0\times \cO_{g} = \cO_{b}\cupcol \cO_{g}$.


      \medskip

      {\bf Suppose $\star=D$.}

      The bad parity part is the same as that of the case when $\star = C$.

      Now consider the good parity part.
      \[
        \ckcO_{g} = ((2l)^{2r'_{2l}+1}, (2l-1)^{2r'_{2l-1}}, (2l-2)^{2r'_{2l-2}}, \cdots, 2^{2r'_{2}}, 1^{2r'_{1}} )
      \]
      and
      \[
        W_{\lamckg} = W'_{l}\times \underbrace{S_{2l}\times \cdots \times S_{2l}}_{2r'_{2l}\text{-terms}} \times \prod_{i<2l} \underbrace{S_{i}\times \cdots\times S_{i}}_{r'_{i}\text{-terms}}
      \]

      The symbol of sign representation of $W'_{l}$ is
      \[
        \binom{0,1, \cdots, l-1}{1,2, \cdots, l\phantom{-1}}.
      \]
      (Here we always made the choice of the top and bottom row to compatible
      with the type $C$ case. )

      Induce to include all $2l$-length columns yields
      \[
        \binom{r'_{2l}+0,r'_{2l}+1, \cdots, r'_{2l}+l-1}{ r'_{2l}+1,r'_{2l}+2, \cdots, r'_{2l}+l\phantom{-1}}.
      \]
      Now move the the shorter columns. When odd columns $(2i+1)^{2r'_{2i+1}}$
      occurs, it adds $(i)^{r'_{2i+1}}$ columns on the left and
      $(i+1)^{r'_{2i+1}}$ on the right. When even columns $(2i)^{r'_{2i}}$
      occur, the bifurcation happens: one can
      \begin{itemize}
        \item attach columns $(i)^{r'_{2i}}$ on the left and columns
              $(i)^{r'_{2i}}$ on the right, which corresponds to
              $(2i,2i+1)\neq \wp$, or
        \item attach columns $(i-1)^{r'_{2i}}$ on the left and columns
              $(i+1)^{r'_{2i}}$ on the right, which corresponds to
              $(2i,2i+ 1)\in \wp$,
      \end{itemize}

      Therefore,
      \[
        \begin{array}{ccc}
          \bF_{2}(\CPP(\ckcO_{g}))&\longrightarrow
          & J_{W_{\lamckg}}^{W_{g}} \sgn \\
          \wp&\mapsto&    (\cktau_{L},\cktau_{R}) =:\cktau_{\wp}
        \end{array}
      \]
      where
      \[
        \bfrr_{l}(\cktau_{L}) = r'_{2l} = \half (\bfrr_{2l}(\ckcO_{g})-1)
      \]
      \[
        \bfrr_{1}(\cktau_{R}) = 1+ \sum_{i} r'_{i} = \half (\bfrr_{1}(\ckcO_{g})+1)
      \]
      and, if $(2i,2i+1)\notin \wp$,
      \[
        \begin{split}
          \bfrr_{i}(\cktau_{L}) & = \sum_{l\geq 2i} r'_{l}
          = \half(\bfrr_{2i}(\ckcO)-1)\\
          \bfrr_{i+1}(\cktau_{R}) & = 1 + \sum_{l\geq 2i+1} r'_{l} = \half(\bfrr_{2i+1}(\ckcO)+1)
        \end{split}
      \]
      if $(2i,2i+1)\in \wp$,
      \[
        \begin{split}
          \bfrr_{i}(\cktau_{L}) & = \sum_{l\geq 2i+1} r'_{l}
          = \half(\bfrr_{2i+1}(\ckcO)-1)\\
          \bfrr_{i}(\cktau_{R}) & = 1 + \sum_{l\geq 2i} r'_{l} = \half(\bfrr_{2i}(\ckcO)+1)
        \end{split}
      \]

      Also note that $\cktau_{\wp}=\cktau_{\wp^{c}}$. The rest parts are the
      same as that of type $C$.

      {\bf Suppose $\star=B$. }

      In this case, bad parity is odd and every odd row occurs with with even
      times.

      We can write
      $r'_{i} := \floor{\half(\bfrr_{i}(\ckcO_{b})-\bfrr_{i-1}(\ckcO_{b}))}$
      \[
        \ckcO_{b} % = [2r_{1}+1, 2r_{1}+1, \cdots, 2r_{k}+1,2r_{k}+1]
        % = (2c_{0},2c_{1},2c_{1}, \cdots, 2c_{l}, 2c_{l}).
        = ((2l)^{2r'_{2l}+1}, (2l-1)^{2r'_{2l-1}},\cdots, 1^{2r'_{1}})
      \]
            %             with $k = c_{0}$ and $l = r_{1}$.
      Then
      \[
        W_{\lamckb} = W_{l} \times \underbrace{S_{2l}\times \cdots \times S_{2l}}_{2r'_{2l}\text{-terms}} \times \prod_{i<2l} \underbrace{S_{i}\times \cdots\times S_{i}}_{r'_{i}\text{-terms}}
      \]
      (Note that in the product, $r'_{i}=0$ if $i$ is odd.) The computation of
      $\cksigma_{b} = J_{W_{\lamckb}}^{W_{b}} \sgn$ is similar to that of the
      good parity for type $C$ with no bifurcating, one deduce that
      $J$-induction and $j$-induction gives the same result.
      \[
        \begin{split}
          \cksigma_{b} &=
          \binom{0, 1+r_{l}, 2+r_{l-1}\cdots, l+r_{1}}{1+r_{l},2+r_{l-1}, \cdots, l+r_{1}}\\
          & = ([r_{1},r_{2},\cdots, r_{l}],[r_{1}+1,r_{2}+1,\cdots,r_{l}+1])\\
        \end{split}
      \]
      with $r_{i} = \half\bfrr_{2i-1}(\ckcO_{b}) = \half\bfrr_{2i}(\ckcO_{b})$.
      Now
      \[
        \sigma_{b} = ((r_{1}+1,r_{2}+1,\cdots,r_{l}+1), (r_{1},r_{2},\cdots, r_{l})) = j_{S_{\cO'_{b}}}^{W_{b}}\sgn
      \]
      where
      $\cO'_{b}=(\bfrr_{2}(\ckcO_{b}),\bfrr_{4}(\ckcO_{b}),\cdots, \bfrr_{2l}(\ckcO_{b}))$.
      Under the Springer correspondence of type $B$, it corresponds to
      $\Ind_{\GL_{b}}^{\SO(2b+1)}\cO'_{b} = \cO'_{b}\cuprow \cO'_{b}\cuprow (1)$.

      % \[
      %   \begin{split}
      %     \cksigma_{b} &:= \sigma_{b}\otimes \sgn = j_{W_{\lamckb}}^{W_{b}} \sgn \\
      %     & = %\dagger_{2c_{l}}\cdots \dagger_{2c_{1}}
      %     \sigma_{b}\otimes \sgn = j_{W_{\lamckb}}^{W_{b}} \sgn\otimes
      %     \binom{0, 1, \cdots, c_{0}}{1, \cdots, c_{0}}\\
      %     & =
      %     \binom{0, 1+r_{k}, 2+r_{k-1}\cdots, c_{0}+r_{1}}{1+r_{k},2+r_{k-1}, \cdots, c_{0}+r_{1}}\\
      %     & = ([r_{1},r_{2},\cdots, r_{k}],[r_{1}+1,r_{2}+1,\cdots,r_{k}+1])\\
      %     &= ((c_{1},c_{2},\cdots, c_{k}),(c_{0},c_{1}, \cdots, c_{l}))\\
      %   \end{split}
      % \]


      % We take the convention that $\dagger \cO = [r_{i}+1]$. By abuse of
      % notation, let $\dagger_{n} \sigma$ denote the
      % $j_{S_{n} \times W_{\abs{\sigma}}}^{W_{n+\abs{\sigma}}} \sgn\otimes \sigma$.
      % We can write
      % \[
      %   \ckcO_{b} = [2r_{1}+1, 2r_{1}+1, \cdots, 2r_{k}+1,2r_{k}+1] = (2c_{0},2c_{1},2c_{1}, \cdots, 2c_{l}, 2c_{l})
      % \]
      % with $k = c_{0}$ and $l = r_{1}$.

      % \[
      %   \begin{split}
      %     W_{\lamckb} &= W_{c_{0}} \times S_{2c_{1}} \times S_{2c_{2}}\times \cdots \times S_{2c_{l}}\\
      %     \cksigma_{b} &:= \sigma_{b}\otimes \sgn = j_{W_{\lamckb}}^{W_{b}} \sgn \\
      %     & = \dagger_{2c_{l}}\cdots \dagger_{2c_{1}}
      %     \binom{0, 1, \cdots, c_{0}}{1, \cdots, c_{0}}\\
      %     & =
      %     \binom{0, 1+r_{k}, 2+r_{k-1}\cdots, c_{0}+r_{1}}{1+r_{k},2+r_{k-1}, \cdots, c_{0}+r_{1}}\\
      %     & = ([r_{1},r_{2},\cdots, r_{k}],[r_{1}+1,r_{2}+1,\cdots,r_{k}+1])\\
      %     &= ((c_{1},c_{2},\cdots, c_{k}),(c_{0},c_{1}, \cdots, c_{l}))\\
      %   \end{split}
      % \]

      % Therefore
      % \[
      %   \begin{split}
      %     \sigma_{b} &= \cksigma_{b}\otimes \sgn = ((r_{1}+1,r_{2}+1,\cdots,r_{k}+1),(r_{1},r_{2},\cdots, r_{k})) \\
      %     & = j_{S_{2r_{1}+1}\times \cdots S_{2r_{k}+1}}^{W_{b}} \sgn\\
      %     & = j_{S_{b}}^{W_{b}} (2r_{1}+1, 2r_{2}+1, \cdots, 2r_{k}+1)
      %   \end{split}
      % \]
      % which corresponds to the orbit
      % \[
      %   \cO_{b} = (2r_{1}+1, 2r_{1}+1,2r_{2}+1, 2r_{2}+1, \cdots,2r_{k}+1, 2r_{k}+1 ) = \ckcO_{b}^{t}.
      % \]
      % (Note that $\cO'_{b} = (2r_{1}+1,2r_{2}+1, \cdots, 2r_{k}+1)$ which
      % corresponds to $j_{W_{L_{b}}}^{S_{b}}\sgn$ and
      % $\ind_{L}^{G} \cO'_{b} = \cO_{b}$. ) This implies the unique special
      % representation is
      % \[
      %   \sigma_{b} = (j_{W_{\lamckb}}^{W_{b}}\sgn), \quad \text{where
      % } W_{L,b} = \prod_{i=1}^{k} S_{2r_{i}+1}.
      % \]
      % The $J$-induction is calculated by \cite{Lu}*{(4.5.4)}. It is easy to
      % see that in our case $J_{W_{\lamckb}}^{W_{b}} \sgn$ consists of the
      % single special representation by induction.


      Now we consider the good parity parts, where each row of $\ckcO_{g}$ has
      even length.

      Assume $r'_{i} := \half(\bfrr_{i}(\ckcO_{g})-\bfrr_{i-1}(\ckcO_{g}))$ and
      so
      \[
        \ckcO_{g} % = [2r_{1}+1, 2r_{1}+1, \cdots, 2r_{k}+1,2r_{k}+1]
        % = (2c_{0},2c_{1},2c_{1}, \cdots, 2c_{l}, 2c_{l}).
        = ((2l+1)^{2r'_{2l+1}}, (2l)^{2r'_{2l}},\cdots, 1^{2r'_{1}})
      \]
      % Consider
      % \[
      %   \cO_{g} = [2r_{1},2r_{2}, \cdots, 2r_{2k-1},2r_{2k}] = (C_{1},C_{1}, C_{2},C_{2},\cdots, C_{l}, C_{l}).
      % \]
      with $l =\min\set{i|\bfrr_{2i+2}(\ckcO_{g}) = 0}$.

      Then
      \[
        W_{\lamckg} = \times \prod_{i\leq 2l+1} \underbrace{S_{i}\times \cdots\times S_{i}}_{r'_{i}\text{-terms}}
      \]

      Note that the trivial representation of the trivial group has symbol
      \[
        \binom{0,1, 2, \cdots, l\phantom{-1}}{0,1, \cdots, l-1}.
      \]


      Induce to include all $2l+1$-length columns yields
      \[
        \binom{r'_{2l+1}+0,r'_{2l+1}+1,r'_{2l+1}+2,\cdots, r'_{2l+1}+l\phantom{-1}}{ r'_{2l+1}+0,r'_{2l+1}+1, \cdots, r'_{2l+1}+l-1}.
      \]
      Now move the the shorter columns. When odd columns $(2i+1)^{2r'_{2i+1}}$
      occurs, it adds $(i+1)^{r'_{2i+1}}$ columns on the left and
      $(i)^{r'_{2i+1}}$ on the right. When even columns $(2i)^{r'_{2i}}$ occur,
      the bifurcation happens: one can
      \begin{itemize}
        \item attach columns $(i)^{r'_{2i}}$ on the left and columns
              $(i)^{r'_{2i}}$ on the right, which corresponds to
              $(2i,2i+1)\neq \wp$, or
        \item attach columns $(i-1)^{r'_{2i}}$ on the left and columns
              $(i+1)^{r'_{2i}}$ on the right, which corresponds to
              $(2i,2i+1)\in \wp$.
      \end{itemize}


      Therefore,
      \[
        \begin{array}{ccc}
          \bF_{2}(\CPP(\ckcO_{g}))&\longrightarrow
          & J_{W_{\lamckg}}^{W_{g}} \sgn \\
          \wp&\mapsto&    (\cktau_{L},\cktau_{R}) =:\cktau_{\wp}
        \end{array}
      \]
      where
      % \[
      %   \bfrr_{2l+1}(\cktau_{L}) = r'_{2l+1} = \half \bfrr_{2l+1}(\ckcO_{g})
      % \]
      \[
        \bfrr_{1}(\cktau_{L}) = \sum_{i} r'_{i} = \half \bfrr_{1}(\ckcO_{g})
      \]
      and, if $(2i,2i+1)\notin \wp$,
      \[
        \begin{split}
          \bfrr_{i+1}(\cktau_{L}) & = \sum_{l\geq 2i+1} r'_{l}
          = \half\bfrr_{2i+1}(\ckcO)\\
          \bfrr_{i}(\cktau_{R}) & = \sum_{l\geq 2i} r'_{l} = \half\bfrr_{2i}(\ckcO)
        \end{split}
      \]
      if $(2i,2i+1)\in \wp$,
      \[
        \begin{split}
          \bfrr_{i+1}(\cktau_{L}) & = \sum_{l\geq 2i} r'_{l}
          = \half\bfrr_{2i}(\ckcO)\\
          \bfrr_{i}(\cktau_{R}) & = \sum_{l\geq 2i+1} r'_{l} = \half\bfrr_{2i+1}(\ckcO)
        \end{split}
      \]

      Some remarks on the BV-dual. The calculation of $\cO_{g}$ from
      $\tau_{\emptyset}$ can be reduced to the case of quasi-distinguished
      orbits (other case are deduced from this by parabolic induction,
      corresponds to attach two even columns for the balanced pairs). Compare
      Sommer's description of Springer correspondence with ours, we deduce that
      \[
        \cO_{g} = (\bfrr_{1}(\ckcO_{1})+1,\bfrr_{2}(\ckcO_{2})-1,\bfrr_{3}(\ckcO_{3})+1, \cdots, \bfrr_{2l}(\ckcO_{2l})-1,\bfrr_{2l+1}(\ckcO_{2l+1})+1)
      \]
      The rest parts are similar to that of type $D$.
    }


    We sketch the proof for the case when $\star = \wtC$.


    For a partition $\cO$, we set
    \[
      \sfS_{\cO} := \prod_{i\in \bN^{+}}\sfS_{\bfcc_{i}(\cO)}
    \]
    so that $j_{\sfS_{\cO}}^{\sfS_{\abs{\cO}}}\sgn = \cO$.


    We first consider the good parity (even) part.

    Now we consider the good parity parts, where each row of $\ckcO_{g}$ has
    even length.

    We set $r'_{i} := \half(\bfrr_{i}(\ckcO_{g})-\bfrr_{i-1}(\ckcO_{g}))$,
    $l =\min\set{i|\bfrr_{2i+1}(\ckcO_{g})=0}$, and write
    \[
      \ckcO_{g} % = [2r_{1}+1, 2r_{1}+1, \cdots, 2r_{k}+1,2r_{k}+1]
      % = (2c_{0},2c_{1},2c_{1}, \cdots, 2c_{l}, 2c_{l}).
      = ((2l)^{2r'_{2l}}, (2l-1)^{2r'_{2l-1}},\cdots, 1^{2r'_{1}})
    \]
    where $i^{r'}$ denotes $r'$-copies of length $i$ columns.
    % Consider
    % \[
    %   \cO_{g} = [2r_{1},2r_{2}, \cdots, 2r_{2k-1},2r_{2k}] = (C_{1},C_{1}, C_{2},C_{2},\cdots, C_{l}, C_{l}).
    % \]
    The Weyl group of good parity is $\sfW'_{n_{g}}$ with
    $n_{g} = \half\abs{\ckcO_{g}}$. For $1\leq k\leq l$, let
    \[
      % S_{r,s} = \prod_{i=r}^{s} \underbrace{\sfS_{i}\times \cdots\times \sfS_{i}}_{r'_{i}\text{-terms}}
      \vec{S}_{i} = \underbrace{\sfS_{i}\times \cdots\times \sfS_{i}}_{r'_{i}\text{-terms}} \AND n_{k} = \sum_{i=k}^{l} i\cdot r'_{i}.
      % \AND n_{r,s} = \sum_{i=r}^{s} i\cdot r'_{i}.
    \]

    Then $W_{\lamckg}=\prod_{i=1}^{l} \vec{S}_{i}$ and
    \[
      \begin{split}
        \ckLV_{\ckcO_{g}}& :=J_{W_{\lamckg}}^{\Wg}\sgn\\
        & = J_{\vec{S}_{1}\times \sfW'_{n_{2}}}^{W_{n_{1}}} \Big(\sgn \otimes J_{\vec{S}_{2}\times \sfW'_{n_{3}}}^{\sfW'_{n_{2}}}\Big(\sgn
        \otimes \cdots\big(J_{\vec{S}_{l}} \sgn\big)\cdots \Big)\Big) \\
      \end{split}
    \]
    Applying \cite{Lu}*{(4.6.2)} inductively, we see that the operation
    $J_{\vec{S}_{i}\times \sfW'_{n_{i+1}}}^{\sfW'_{n_{i}}}(\sgn \otimes \underline{\ \ \ })$
    doubles (resp. keeps) the number of irreducible components if $i$ is odd
    (resp. even).

    Suppose $\CPPs(\ckcO_{g}) = \emptyset$.
    We define
    \[
      \tA'(\ckcO):= \bF_{2}[\emptyset]
    \]
    to be the trivial group.
    Then $\ckLV_{\ckcO_{g}}$ is
    irreducible and marked by label $I$.  Hence, $\LC(\ckcO_{g})$,
    $\LC(\ckcO)$ and $\tA'(\ckcO)$ are naturally identified.


    Now assume $\CPPs(\ckcO_{g})\neq \emptyset$. Then the two parts of the
    bipartition of an irreducible component are different. Let
    $i_{0}:= \min\Set{i| (2i-1,2i)\in \CPPs(\ckcO_{g})}$.
    % $(2i_{0}-1,2i_{0})$ be the element in such that $i_{0}$ is minimal.
    Then we have a bijection
    \[
      \tA'(\ckcO):= \set{\wp\in \bF_{2}[\CPPs(\ckcO_{g})]|(2i_{0}-1,2i_{0})\notin \wp} \longrightarrow \ckLV_{\ckcO_{g}}
    \]
    to record the bifurcation when attaching odd length columns. Here we send
    $\wp$ to $\cktau_{\wp}:= (\cktau_{L},\cktau_{R})$ such that
    \[
      (\bfrr_{i}(\cktau_{L}),\bfrr_{i}(\cktau_{R})) := \begin{cases} (\half\bfrr_{2i}(\ckcO_{g}),\half\bfrr_{2i-1}(\ckcO_{g}))
        & \text{if } (2i-1,2i)\notin \wp\\
        (\half\bfrr_{2i-1}(\ckcO_{g}),\half\bfrr_{2i}(\ckcO_{g}))
        & \text{if } (2i-1,2i)\in \wp\\
      \end{cases}
    \]
    We obtain the structure of $\LC_{\ckcO_{g}}$ by tensoring the sign
    representation.

    \trivial[]{ Note that the trivial representation of the trivial group is
      represented by the symbol
      \[
        \binom{0,1, 2, \cdots, l-1}{0,1,2, \cdots, l-1}_{I}.
      \]
      % Induce to include all $2l$-length columns yields
      % \[
      %   \binom{r'_{2l+1}+0,r'_{2l+1}+1,r'_{2l+1}+2,\cdots, r'_{2l+1}+l\phantom{-1}}{ r'_{2l+1}+0,r'_{2l+1}+1, \cdots, r'_{2l+1}+l-1}.
      % \]

      Now move the the shorter columns. When even columns $(2i)^{2r'_{2i}}$
      occurs, it adds $(i)^{r'_{2i}}$ columns on the left and $(i)^{r'_{2i}}$ on
      the right. When odd columns $(2i-1)^{r'_{2i-1}}$ occur, the bifurcation
      happens: one can
      \begin{itemize}
        \item attach columns $(i-1)^{r'_{2i-1}}$ on the left and columns
              $(i)^{r'_{2i-1}}$ on the right, which corresponds to
              $(2i-1,2i)\neq \wp$, or
        \item attach columns $(i)^{r'_{2i-1}}$ on the left and columns
              $(i-1)^{r'_{2i-1}}$ on the right, which corresponds to
              $(2i-1,2i)\in \wp$.
      \end{itemize}
      Note that when we first encounter the longest odd column, we make the
      choice that the size of left part is larger than that of the right part.
      Now If $(2i-1,2i)\notin \wp$,
      \[
        \begin{split}
          \bfrr_{i}(\cktau_{L}) & = \sum_{l\geq 2i} r'_{l}
          = \half\bfrr_{2i}(\ckcO_{g})\\
          \bfrr_{i}(\cktau_{R}) & = \sum_{l\geq 2i-1} r'_{l} = \half\bfrr_{2i-1}(\ckcO_{g})
        \end{split}
      \]
      if $(2i-1,2i)\in \wp$,
      \[
        \begin{split}
          \bfrr_{i}(\cktau_{L}) & = \sum_{l\geq 2i-1} r'_{l}
          = \half\bfrr_{2i-1}(\ckcO_{g})\\
          \bfrr_{i}(\cktau_{R}) & = \sum_{l\geq 2i} r'_{l} = \half\bfrr_{2i}(\ckcO_{g})
        \end{split}
      \]
    }


    Now we consider the bad parity (odd) part. Suppose $\ckcO_{b}$ is nonempty
    such that
    \[
      \ckcO_{b} = (2c_{0},2c_{1}, 2c_{1}, 2c_{2},2c_{2}, \cdots, 2c_{k}, 2c_{k})
    \]
    where $2k+1=\bfrr_{1}(\ckcO_{b})$ and $2c_{i} = \bfcc_{2i+1}(\ckcO_{b})$.
    Now
    \[
      W_{\lamckb} = \sfW_{c_{0}} \times \sfS_{2c_{1}} \times \sfS_{2c_{2}}\times \cdots \times \sfS_{2c_{k}}
    \]
    and %$J_{W_{\lamckb}}^{W_{b}}\sgn$ is irreducible by
    \[
      \begin{split}
        \cktau_{b} &:= J_{W_{\lamckb}}^{W_{b}} \sgn
        = ((c_{1},c_{2},\cdots, c_{k}),(c_{0},c_{1}, \cdots, c_{l}))\\
        & = \big([\half(\bfrr_{2}(\ckcO)-1),\half(\bfrr_{4}(\ckcO)-1),\cdots, \half(\bfrr_{2c_{0}}(\ckcO)-1)],\\
        & \hspace{2em} [\half(\bfrr_{2}(\ckcO)+1),\half(\bfrr_{4}(\ckcO)+1),\cdots, \half(\bfrr_{2c_{0}}(\ckcO)+1)]\big)
      \end{split}
    \]
    is irreducible by \cite{Lu}*{(4.5.4)}. Tensoring with sign yields the
    formula of $\tau_{b}$. Moreover, by the fake degree formula (see
    \cite{Carter}*{p~376}), we have
    \[
      \tau_{b} = \cktau_{b}\otimes \sgn = j_{\sfS_{\cO'_{b}}}^{\sfW_{n_{b}}} \sgn.
    \]
    where
    $ \cO'_{b} := (\ckcO'_{b})^{t}:=(\bfrr_{2}(\ckcO),\bfrr_{4}(\ckcO),\cdots , \bfrr_{2c_{0}}(\ckcO))$.

    \medskip \def\ckfll{\check{\fll}}

    Now we sketch the proof of \eqref{eq:dBV.W}.

    Recall the definition of the metaplectic Barbasch-Vogan dual in
    \cite{BMSZ1}. The duality map commutes with parabolic induction: Suppose
    $\ckfll\subset \check \ckfgg$ is a parabolic subalgebra of $\ckfgg$ and
    $\fll$ is the corresponding parabolic subalgebra in $\fgg$, then
    \begin{equation}\label{eq:inddBV}
      \dBV(\ckcO) =  \Ind_{\fll}^{\fgg}(\dBV(\ckcO_{\ckfll}))
    \end{equation}
    for each nilpotent orbit $\ckcO$ in $\ckfgg$ such that
    $\ckcO_{\ckfll}:=\ckcO\cap \ckfll\neq \emptyset$. This is clear by reducing
    to the type $B$ case, see \cite{BMSZ1}*{Proposition~3.8}. By removing pairs
    of rows with the same lengths in $\ckcO$, we reduced to check the equality
    in the case when $\ckcO_{b}=\emptyset$ and
    $\bfrr_{2i-1}(\ckcO_{g})>\bfrr_{2i}(\ckcO_{g})$ for all $i$ such that
    $i\leq \bfcc_{1}(\ckcO_{g})$. In this case, both sides of \eqref{eq:dBV.W}
    can be easily computed directly, which equals to
    \[
      (\bfrr_{1}(\ckcO)-1, \bfrr_{2}(\ckcO)+1, \cdots,\bfrr_{2c-1}(\ckcO)-1,\bfrr_{2c}(\ckcO)+1)
    \]
    with $c = \min\set{i|\bfrr_{2i+1}(\ckcO)=0}$.

    % Now we compare the metaplectic dual defined in \cite{BMSZ1} with the Weyl
    % group representations.
    \trivial[]{ Compare Sommer's description of Springer correspondence we
      deduce that the RHS is
      \[
        \cO_{g} = (\bfrr_{1}(\ckcO_{1})-1,\bfrr_{2}(\ckcO_{2})+1,\bfrr_{3}(\ckcO_{3})+1, \cdots, \bfrr_{2l-1}(\ckcO_{2l-1})-1,\bfrr_{2l}(\ckcO_{2l})+1)
      \]
      The LHS is calculated by $((((\ckcO^{t})_{D})^{+})^{-})_{C}$. We write
      $R_{i}=\bfrr_{i}(\ckcO)=2r_{i}$. Now under our assumption,
      $R_{2i-1}>R_{2i}$, we have
      \[
        \begin{split}
          ((((\ckcO^{t})_{D})^{+})^{-})_{C} &=
          ((((R_{1},R_{2}, \cdots, R_{2l-1},R_{2l})_{D})^{+})^{-})_{C}\\
          &=((R_{1}-1,R_{2}, \cdots, R_{2l-1},R_{2l},1))_{C}\\
          &=(R_{1}-1,R_{2}+1, \cdots, R_{2l-1}-1,R_{2l}+1)\\
        \end{split}
      \]
      So the proof is done.

    }

    At last, one can see that
    $\dBV(\ckcO_{b}\cuprow \ckcO_{g}) = \ckcO_{b}^{t} \cupcol \dBV(\ckcO_{g})$
    using \eqref{eq:inddBV}.
  \end{proof}

  % \begin{remark}
  %   When $\star=\wtC$, one can see that
  %   $\dBV(\ckcO_{b}\cuprow \ckcO_{g}) = \ckcO_{b}^{t} \cupcol \dBV(\ckcO_{g})$
  %   using \eqref{eq:inddBV}. \trivial[]{ This could be checked using the
  %   formula of Springer correspondence directly, see Sommer's. }
  % \end{remark}
%
Since the representation theory of $\sfW_{n}$ is more elementary than that of
$\sfW'_{n}$, we prefer to induces every thing to $\sfW_{n}$.
We record the following simple lemma for the later use.

\begin{lem}\label{lem:WLcell}
  Suppose $\star=\wtC$. For any $\wp\in \tA(\ckcO)$, let
  $\wttau_{\wp} = (\imath_{\wp},\jmath_{\wp})$. Then
  \[
    \Ind_{\sfW_{n_{b}}\times \sfW'_{n_{g}}}^{\sfW_{n_{b}}\times \sfW_{n_{g}}} \tau_{b}\boxtimes \tau_{\wp} =
    \begin{cases}
      \tau_{b}\boxtimes \wttau_{\emptyset} & \text{if } \tA(\ckcO) =\emptyset,\\
      \tau_{b}\boxtimes \wttau_{\wp} \oplus \tau_{b}\boxtimes \wttau_{\wp^{c}}
      &\text{otherwise}.
    \end{cases}
  \]
Let
\[
  \tLV_{\ckcO}:= \Ind_{W_{b}\times W_{g}}^{\sfW_{n_{b}}\times \sfW_{n_{g}}}\LV_{\ckcO}
\]
and $\tLC_{\ckcO}$ be the set of irreducible components of the
$\sfW_{n_{b}}\times \sfW_{n_{g}}$-module $\tLV_{\ckcO}$. Then we have a
bijection
  \[
      \begin{array}{lccccccc}
        \tA(\ckcO)&=&\tA(\ckcO_{g}) & \longrightarrow & \tLC(\ckcO_{g})
        & \longrightarrow & \tLC(\ckcO)\\
                  &  &\wp & \mapsto & \wttau_{\wp}
        & \mapsto & \tau_{b}\otimes \wttau_{\wp}.
      \end{array}
  \]
  Suppose $\star\in\set{D,D^{*}}$. For any $\wp\in \tA(\ckcO)$, let
  $\wttau_{\wp} = (\imath_{\wp},\jmath_{\wp})$ and $\wttau_{b} = \Ind_{\sfW'_{n_{b}}}^{\sfW_{n_{b}}}\tau_{b}$.
  Then
  \[
    \Ind_{\sfW'_{n_{b}}\times \sfW'_{n_{g}}}^{\sfW_{n_{b}}\times \sfW_{n_{g}}} \tau_{b}\boxtimes \tau_{\wp} =
    \begin{cases}
      \wttau_{b} & \text{if } n_{g}=0\\
      \wttau_{b}\boxtimes \wttau_{\wp} \oplus \wttau_{b}\boxtimes \wttau_{\wp}^{s}
      &\text{otherwise}.
    \end{cases}
  \]
  where $\wttau_{\wp}^{s}:= \wttau_{\wp}\otimes \brsgn \neq \wttau_{\sP}$.

  % Then we have a bijection
  % \[
  %     \begin{array}{lccccccc}
  %       \tA(\ckcO)&=&\tA(\ckcO_{g}) & \longrightarrow & \tLC(\ckcO_{g})
  %       & \longrightarrow & \tLC(\ckcO)\\
  %                   &  &\wp & \mapsto & \wttau_{\wp} &
  %                                                    \mapsto & \tau_{b}\otimes \wttau_{\wp}.
  %     \end{array}
  % \]
\end{lem}
\begin{proof}
  The lemma follows immediately from the above lemma on the explicit
  descriptions of the left cells.
\end{proof}


\subsection{Coherent continuation representations}


% \subsubsection{Some subgroups of the Weyl groups}
Before we start to describe the coherent continuation representations we first
recall some subgroups of the Weyl group and the related branching rule.


In the following $a,b,c,d,n,p,q,r,s,t\in \bN$. We view $\sfW_{2t}$ and
$\sfS_{2t}$ as the reflection group acts on $\bC^{t}$ as usual. Let
$\sfH_{t} := \sfW_t\ltimes \set{\pm 1}^t$ be the subgroup in $\sfW_{2t}$ such
that
\begin{itemize}
  \item the first factor $\sfW_{t}$ sits in $\sfS_{2t}$ commuting with the
        involution
        \[
        (12)(34)\cdots ((2t-1)(2t)).
        \]
  \item The element $(1,\cdots,1, \underbrace{-1}_{i\text{-th
        term}}, 1, \cdots, 1)\in \set{\pm 1}^{t}$ acts on $\bC^{2t}$ by
        \[
          % (x_{1},\cdots, x_{2i-2}, x_{2i-1}, x_{2i},x_{2i+1},\cdots, x_{2t} )
        (x_{1},x_{2},\cdots, x_{2t} ) \mapsto (x_{1},\cdots, x_{2i-2}, -x_{2i},-x_{2i-1},x_{2i+1},\cdots, x_{2t}).
        \]
\end{itemize}
Note that $\sfH_{t}$ is also a subgroup of $\sfW'_{2t}$. Define the quadratic
character
\[
  \begin{array}{rccc}
    \hsgn := 1\otimes \sgn\colon & \sfH_{t}=  \sfW_{t}\ltimes \set{\pm 1}^{t}& \longrightarrow & \set{\pm 1}\\
                                 & (g,(a_{1},a_{2},\cdots, a_{t})) & \mapsto & a_{1}a_{2}\cdots a_{t}.
  \end{array}
\]
The most important formulas are
\begin{equation}\label{eq:CC.C}
  \Ind_{\sfH_{t}}^{\sfW_{2t}} \hsgn = \sum_{\sigma\in \Irr(\sfS_{t})} (\sigma,\sigma)
  \AND
  \Ind_{\sfH_{t}}^{\sfW'_{2t}} \hsgn = \sum_{\sigma\in \Irr(\sfS_{t})} (\sigma,\sigma)_{I},
\end{equation}
see \cite{Mc}*{p220 (6)}.

We denote $\bsgn$ the inflation of the sign representation of $\sfS_{n}$ to
$\sfW_{n}$. Then
\[
  \Ind_{\sfS_{n}}^{\sfW_{n}}\sgn = \bigoplus_{a+b=n}\Ind_{W_{a}\times W_{b}}^{W_{n}} \bsgn\boxtimes \sgn =\bigoplus_{a+b=n} ((a,),(b,)) .
\]
Let $\brsgn := \sgn \otimes \bsgn$ which is the quadratic character of
$\sfW_{n}=\sfS_{n}\ltimes \set{\pm 1}^{n}$ given by
\[
  (s,(x_{1}, x_{2}, \cdots, x_{n}))\mapsto x_{1}x_{2}\cdots x_{n}.
\]
\trivial{In fact, $\brsgn$ is the unique non-trivial quadratic character of
  $\sfW_{n}=\sfS_{n}\ltimes \set{\pm 1}^{n}$ which is trivial on $\sfS_{n}$. }
Then
\[
  \Ind_{\sfS_{n}}^{\sfW_{n}} 1 = \bigoplus_{a+b=n}\Ind_{W_{a}\times W_{b}}^{W_{n}} 1 \boxtimes \brsgn =\bigoplus_{a+b=n} ([a,],[b,]) .
\]
% Note that we have chosen the embedding $W_{t}\subset S_{2t}$ in $W_{2t}$. We
% have
% \[
%   \Ind_{H_{t}}^{W'_{2t}} \hsgn = \sum_{\sigma\in \Irr(S_{t})} (\sigma,\sigma)_{I}.
% \]

\trivial{ % In McGovern's paper, the coherent continuation representation is
  % described as:
  % \[
  %   \sum_{t,s,a,b}\Ind_{W_t\times (W_s\ltimes W(A_1)^s)\times W_a\times W_b}^{W_{t+2s+a+b}} \sgn\otimes (\triv \otimes \sgn)\otimes \triv\otimes \triv
  % \]
  Now \eqref{eq:CC.C} was obtained by the following branching formula:
  \cite[p220 (6)]{Mc}
  \[
    I_n:= \Ind_{(W_s\ltimes W(A_1)^s)}^{W_{2s}}\triv\otimes \sgn = \sum \lambda\times \lambda
  \]
  where $\lambda$ running over all Young diagrams of size $s$. As McGovern
  claimed the proof of the above formula is similar to Barbasch's proof of
  \cite[Lemma~4.1]{BV.W}:
  \[
    \Ind_{W_n}^{S_{2n}} \triv = \sum \sigma \quad \text{where $\sigma$ has even
      rows only}.
  \]

  Sketch of the proof (use branching rule and dimension counting): Note that
  $\dim I_n = \frac{(2p)! 2^{2p}}{p! 2^{2p}} = (2p)!/p! =\sum_\lambda \dim \lambda\times \lambda$
  (For the last equality:
  $\dim \lambda\times \lambda = (2p)! (\dim \lambda)^2/(p!)^2$ where
  $\dim \lambda$ is the dimension of $S_n$ representation determined by
  $\lambda$; But $\sum (\dim \lambda)^2 = p!$). On the other hand,
  $H :=W_s\ltimes W(A_1)^s\cap W_s\times W_s = \Delta W_s \subset W_{2s}$.
  $\triv \otimes \sgn|_H = \sgn$ of $\Delta W_s$ Therefore,
  $\lambda\times \lambda$ appears in $I_n$ by Mackey formula. Now by dimension
  counting, we get the formula. }

\medskip

We now define various Weyl representations case by case. They will be used to
state the formula of coherent continuation representations.
% In the following $\sigma$ is running over all irreducible representations of
% $\sfS_{t}$.
\begin{itemize}
  \item Suppose $\star= B$, $p+q=2m+1$ is odd. Define
        \[
        \begin{split}
          \cC_b^{n} & :=\bigoplus_{\substack{2t+c+d=n}} \Ind_{\sfH_{t} \times \sfW_{c}\times \sfW_{d}}^{\sfW_{n}}
          \hsgn\boxtimes 1\boxtimes 1 \\
          \cC_g^{p,q} &:=\bigoplus_{\substack{0\leq p-(2t+a+2r)\leq 1\\0\leq q - (2t+a+2s)\leq 1}} \Ind_{\sfH_{t} \times \sfS_{a}\times \sfW_s\times \sfW_r}^{\sfW_{n}}
          \hsgn \boxtimes 1 \boxtimes \sgn \boxtimes \sgn \\
          % &\cong \bigoplus_{\substack{0\leq p-(2t+c+d+2r)\leq 1\\0\leq q - (2t+c+d+2s)\leq 1}} \bigoplus_{\sigma} \Ind_{\sfW_{2t} \times \sfW_{c}\times \sfW_{d}\times \sfW_s\times \sfW_r}^{\sfW_{n}}
          % (\sigma,\sigma) \boxtimes 1 \boxtimes \brsgn \boxtimes \sgn \boxtimes \sgn \\
        \end{split}
        \]
        \trivial[]{ Here is a point which could cases confusion: Although the
        real Weyl group is
        $\sfH_{t}\times \sfW_{a}\times \sfW_{s}\times \sfW_{r}$, the cross
        stabilizer is much smaller
        $=\sfH_{t}\times \sfS_{a}\times \sfW_{s}\times \sfW_{r}$! This is dual
        to the fact that, for the split Cartan and real root $e_{i}$,
        $\sgn\circ e_{i}\colon H\rightarrow \bR^{\times}\rightarrow \set{\pm 1}$
        is non-trivial! The good infinitesimal character takes half-integer
        values. So $s_{e_{i}}$ never cross stabilizing a regular character at
        these infinitesimal characters }
  \item Suppose $\star=C^{*}$. Define
        \[
        \begin{split}
          \cC_b^{n} & :=
          \begin{cases}
            % \Res_{\sfW_{n}}^{\sfW'_{n}}
            \Ind_{\sfH_{t}}^{\sfW_{n}} \hsgn &
            \text{if $n=2t$ is even} \\
            0 & \text{otherwise}\\
          \end{cases}\\
          \cC_g^{2p,2q} %& = \bigoplus_{p+q=m} \Cint{\rho}(\Sp(p,q)) \\
          &:=\bigoplus_{\substack{(t+s,t+r)=(p,q)}} \Ind_{\sfH_{t} \times \sfW_s\times \sfW_t}^{\sfW_{p+q}}
          \hsgn \otimes \sgn \otimes \sgn \\
          % & =\bigoplus_{\substack{(t+s,t+r)=(p,q)}} \bigoplus_{\sigma\in \Irr(\sfS_{t})} \Ind_{\sfW_{2t}\times \sfW_s\times \sfW_r}^{\sfW_{p+q}}
          % (\sigma,\sigma)\otimes \sgn \otimes \sgn \\
        \end{split}
        \]
  \item Suppose $\star=C$. Define
        \[
        \begin{split}
          \cC_b^{n} &
          :=\bigoplus_{\substack{2t+a=n}} %\Res_{\sfW_{n}}^{\sfW'_{n}} \left(
          \Ind_{\sfH_{t} \times \sfS_a}^{\sfW_{n}} \hsgn\otimes 1 %\right)
          \\
          \cC_g^{n,n} &:= \bigoplus_{2t+a+c+d=n}\Ind_{\sfH_{t} \times \sfS_{a} \times \sfW_c\times \sfW_d}^{W_{n}} \hsgn \otimes
          \sgn \otimes 1 \otimes 1\\
          % & =\bigoplus_{\substack{t+r+s+c+d=n}} \bigoplus_{\sigma } \Ind_{\sfW_{2t}\times \sfW_s\times \sfW_r\times \sfW_{c}\times \sfW_{d}}^{\sfW_{n_{g}}}
          % (\sigma,\sigma)\otimes \sgn \otimes \bsgn \otimes 1\otimes 1 \\
        \end{split}
        \]
  \item Suppose $\star=\wtC$. Define
        \[
        \begin{split}
          \cC_b^{n} &
          :=\bigoplus_{\substack{2t+c+d=n}} %\Res_{\sfW_{n}}^{\sfW'_{n}} \left(
          \Ind_{\sfH_{t} \times \sfW_c\times \sfW_{d}}^{\sfW_{n}} \hsgn\boxtimes 1 \boxtimes
          1 %\right)
          \\
          \cC_g^{n,n} &:= \bigoplus_{2t+a+a'=n}\Ind_{\sfH_{t} \times \sfS_{a} \times \sfS_{a'}}^{\sfW_{n}} \hsgn \otimes
          \sgn \otimes 1 \\
          % & =\bigoplus_{\substack{t+r+s+c+d=n}} \bigoplus_{\sigma } \Ind_{\sfW_{2t}\times \sfW_s\times \sfW_r\times \sfW_{c}\times \sfW_{d}}^{\sfW_{n_{g}}}
          % (\sigma,\sigma)\otimes \sgn \otimes \bsgn \otimes 1\otimes 1 \\
        \end{split}
        \]
  \item Suppose $\star=D$ and $p+q=2m$ is even. Define
        \[
        \begin{split}
          \cC_b^{n} & := \bigoplus_{\substack{2t+a=n}}
          \Ind_{\sfH_{t}\times \sfS_{a}}^{\sfW_{n}} \hsgn\otimes 1\\
          \cC_g^{p,q} %& = \bigoplus_{p+q=m} \Cint{\rho}(\Sp(p,q)) \\
          & := \bigoplus_{\substack{2t+c+d+2r=p\\2t+c+d+2s=q}}
          % \Res_{\sfW_{m}}^{\sfW'_{m}}\left(
          \Ind_{\sfH_{t} \times \sfW_s\times \sfW_r\times \sfW'_{c}\times \sfW_{d} }^{\sfW_{(p+q)/2}} \hsgn \otimes \bsgn \otimes \bsgn \otimes 1\otimes
          1 %\right)
          \\
        \end{split}
        \]
  \item Suppose $\star=D^{*}$. Define
        \[
        \begin{split}
          \cC_b^{n} & :=
          \begin{cases}
            \Ind_{\sfH_{t}}^{\sfW'_{n}} \hsgn &
            \text{if $n=2t$ is even} \\
            0 & \text{otherwise}\\
          \end{cases}\\
          \cC_g^{n,n} %& = \bigoplus_{p+q=m} \Cint{\rho}(\Sp(p,q)) \\
          &:=\bigoplus_{\substack{2t+a=n}} \Ind_{\sfH_{t} \times \sfS_{a}}^{\sfW'_{n}}
          \hsgn \otimes \sgn \\
        \end{split}
        \]
\end{itemize}

Now assume $\rank_{\bC} \Gc = n$. We identify $\fhh^{*}$ with $\bC^{n}$. Let $Q$
be the root lattice in $\fhh^{*}$ which is
\[
  Q = \begin{cases}
    \bZ^{n} & \text{if  $\star = B$}\\
    % \set{(a_{1},a_{2},\cdots, a_{n})\in \bZ^{n}|\sum_{i=1}^{n}a_{i} \in 2\bZ}
    \Set{(a_{i})\in \bZ^{n}|\sum_{i=1}^{n}a_{i} \text{ is even}}
    & \text{if  $\star \in \set{C,\wtC,C^{*},D,D^{*}}$}\\
  \end{cases}
\]
For $n_{b}, n_{g}\in \bN$ such that $n_{b}+n_{g}=n$, we consider the lattice
\[
  \Lambda_{n_{b},n_{g}} =
  \begin{cases}
    (\underbrace{\half, \cdots, \half}_{n_{b}\text{-terms}}, \underbrace{0, \cdots, 0}_{n_{g}\text{-terms}}) + Q & \text{when
    } \star\in \set{C,C^{*}, D,D^{*}}\\%\subset \fhh^{*}.
    (\underbrace{0, \cdots, 0}_{n_{b}\text{-terms}}, \underbrace{\half, \cdots, \half}_{n_{g}\text{-terms}}) + Q & \text{when
    } \star\in \set{B,\wtC}. %\subset \fhh^{*}.
  \end{cases}
\]
Clearly,
\[
  W_{\Lambda_{\nbb,\ngg}}:= \set{w| w\cdot \Lambda_{\nbb,\ngg} = \Lambda_{\nbb,\ngg}} = W_{b}\times W_{g}
\]
with $W_{b}$ and $W_{g}$ defined by \eqref{eq:Wbg}.


We define
\[
  \Sign(G):= \begin{cases}
    (p,q) & \text{if } G = \SO(p,q)\\
    (n,n) & \text{if } G = \Sp(2n,\bR) \text{ or } \Mp(2n,\bR)\\
    (2p,2q) & \text{if } G = \Sp(p,q) \\
    (n,n) & \text{if } G = \rO^{*}(2n) \\
  \end{cases}
\]

\begin{prop}
  As $W_{\Lambda_{n_{b},n_{g}}}:= W_{b}\times W_{g}$-module, the coherent
  continuation representation $\Coh_{\Lambda_{n_{b},n_{g}}}(G)$ is isomorphic to
  the restriction to $W_{\Lambda_{n_{b},n_{g}}}$ of
  \[
    \begin{cases}
      \cC_{b}^{n_{b}}\boxtimes\cC_{g}^{p,q} & \text{if } \star \in \set{B,C,\wtC,C^{*},D}\\
      \Ind_{\sfW'_{n_{b}}\times \sfW'_{n_{g}}}^{W''}(\cC_{b}^{n_{b}}\boxtimes\cC_{g}^{p,q}) & \text{if } \star D^{*}\\
    \end{cases}
  \]
  with $(p,q) = \Sign(G) - (n_{b},n_{b})$ and
  \[
    W'' := \left(\sfW_{n_{b}}\times \sfW_{n_{g}}\right)\cap \sfW'_{n_{g}+n_{b}} \quad \text{when
      $\star = D^{*}$.}
  \]


\end{prop}
\begin{remark}
  When $\star \in \set{C^{*},D^{*}}$ and $n_{b}$ is odd,
  $\Coh_{\Lambda_{n_{b},n_{g}}}(G)=0$ by the proposition.
\end{remark}
\begin{proof}[Sketch of the proof]
  When $G$ is linear, the set of regular characters can be enumerated using
  \cite{AC}, then the follows follows from \Cref{thm:cohHC}, see also
  \cite{Mc}*{Applications}. When $G$ is the metaplectic group, the enumeration
  of parameters is contained in Renard-Trapa's work \cite{RT1,RT2}.
  % We give a sketch of the argument.
\end{proof}

We define
\[
  G'_{n} := \begin{cases}
    \GL(n,\bR) & \text{when } \star \in \set{B,C,\wtC,D}, \\
    \GL(\half n,\bH) & \text{when } \star \in \set{C^{*},D^{*}}. \\
  \end{cases}
\]


We make the following definitions:

\begin{defn}
  Let $\PBPsb(\ckcOb)$ be the set of all pairs
  $\uptau = (\imath,\cP)\times(\jmath,\cQ)$ where $(\imath,\cP)$ and
  $(\jmath,\cP)$ are painted partitions such that
  \begin{itemize}
    \item $(\imath,\jmath) = \tau_{b}$ (see \eqref{eq:taub});
    \item the image of $\cP$ is contained in
          \[
          \begin{cases}
            \set{\bullet, c,d}  & \text{if } \star\in \set{B,\wtC} \\
            \set{\bullet, d}  & \text{if } \star\in \set{C,D}\\
            \set{\bullet}  & \text{if } \star\in \set{C^{*},D^{*}}\\
          \end{cases}
          \]
    \item the image of $\cQ$ is contained in

          \[
          \begin{cases}
            \set{\bullet, c}  & \text{if } \star\in \set{C,D}\\
            \set{\bullet}  & \text{if } \star\in \set{B,\wtC, C^{*},D^{*}}\\
          \end{cases}
          \]
  \end{itemize}
\end{defn}

To ease the notation, for each bipartition $\tau$, let
\[
  \PBP_{\star}(\tau) := \Set{ \uptau|\uptau \text{ is a painted partition and
    } \star_{\uptau}=\star, (\imath_{\uptau},\jmath_{\uptau}) = \tau}
  % \uptau=(\imath, \cP)\times (\jmath,\cP)\times \alpha|}
\]
and
\[
  \tPBP_{\star}(\ckcOg) := %\bigsqcup_{\tau\in\LC(\ckcOg)}\PBP_{\star}(\tau).
  \bigsqcup_{\wp \subseteq \CPP(\ckcOg)}\PBP_{\star}(\wttau_{\wp})
\]
where $\wttau_{\wp} := (\imath_{\wp},\jmath_{\wp})$. Similarly, define
\[
  \tPBP_{\Gg}(\ast):= \Set{\uptau\in \tPBP_{\star}(\ast )|\Sign(\uptau)= \Sign(\Gg)}  \quad
  \ast  = \ckcOg \text{ or } \tau.
\]



\begin{prop}\label{prop:BP.PP}
  In all the cases,
  \[
    \PBP_{\star,b}(\ckcO_{b}) = \PP_{G'}(\ckcO'_{b}) =\Unip_{G'}(\ckcO'_{b}).
  \]
\end{prop}
\begin{proof}
  Suppose $\star \in \Set{C^{*},D^{*}}$. Then
  \[
    \begin{split}
      \abs{\PBP_{\star,b}(\ckcO_{b})} = \abs{\PP_{G'}(\ckcO'_{b})} = \abs{\Unip_{\ckcO'_{b}}(G'_{n_{b}})} = 1.
    \end{split}
  \]
  Suppose $\star \in \Set{B,C,\wtC,D}$, $\tau_{b} = (\tau_{L,b},\tau_{R,b})$ and
  $\tau'_{b} = \cO'_{b}$ It is easy to see that we have a bijection:
  \[
    \begin{array}{ccc}
      \PBP_{\star,b}(\ckcO_{b}) &  \longrightarrow & \PP_{A^{\bR}}(\ckcO'_{b})\\
      (\tau_{L,b},\cP)\times (\tau_{R,b},\cQ)& \mapsto & (\cOpb,\cP')
    \end{array}
  \]
  where $\cP'$ is defined by the condition that
  \[
    \cP(\bfcc_{j}(\tau_{L,b}),j)=d \Longleftrightarrow \cP'(\bfcc_{j}(\cOpb),j)=d \quad \forall j=1,2,\cdots, \bfrr_{1}(\cOpb).
  \]
  \trivial[]{
    % Let $\tau' = \ckcO'^{t}_{b}$ and $\tau_{b}=(\tau_{L,b}, \tau_{R,b})$. Here
    % $\tau_{L,b}, \tau_{R,b}$.
    Now the claim follows for the fact that the bottom rows in $\uptau_{L}$ can
    be filled by $\bullet/c$ or $d$ and
    \[
      \bfcc_{i}(\tau_{L,b}) = \bfcc_{j}(\tau_{L,b}) \Leftrightarrow \bfcc_{i}(\cOpb) = \bfcc_{j}(\cOpb) \quad \forall i,j\in \bN^{+}.
    \]
  }

\end{proof}

\begin{prop}
  In all the cases, we have
  % \[
  %   [\tau_{b}: \cC_{b}] = \PBP_{\star,b}(\ckcO_{b}) = \PBP_{G'}(\ckcO'_{n_{b}}) = \Unip_{\ckcO'_{b}}(G'_{n_{b}})
  % \]
  % and
  % \[
  %   \sum_{\tau\in \LC_{\ckcO_{g}}} [\tau:\cC_{g}] = \PBP_{G}(\ckcO_{g}).
  % \]
  \[
    \sum_{\sP\in \tA'(\ckcO)} [\tau_{b}\otimes \tau_{\sP}: \Coh_{\Lambda_{n_{b},n_{g}}}(G)] = \abs{\PBP_{\star,b}(\ckcO_{b})}\cdot \abs{\tPBP_{G_{g}}(\ckcO_{g})}.
  \]
\end{prop}
\begin{proof}
  One can compute the formula using the branching rule of Weyl group
  representations using \Cref{lem:WLcell}. We leave the details to the reader
  when $\star\in \set{B,C,C^{*},D,D^{*}}$.

  We now present the computation for $\star = D^{*}$ to demonstrate the ideas
  (this is the most complicate case).

  Recall that $(W_{b},W_{g}) = (\sfW'_{\nbb},\sfW'_{\ngg})$.

  Suppose $\ngg = 0$ first. Then $\tau_{b} = (\cOpb,\cOpb)_{I}$ and
  \[
    \begin{split}
      [\tau_{b}:\cC_{b}^{\nbb}]_{W_{b}} = &
      [\tau_{b}: \Ind_{\sfH_{\frac{\nbb}{2}}}^{\sfW'_{\nbb}}\tsgn]\\
      = & [(\cOpb,\cOpb)_{I}: \bigoplus_{\sigma}(\sigma,\sigma)_{I}]\\
      = & 1.
    \end{split}
  \]

  Now we assume that $\ngg>0$. Let
  \[
    \wttau_{b} := \Ind_{\sfW'_{\nbb}}^{\sfW_{\nbb}} \tau_{b} = (\cOpb,\cOpb) \AND \wttau_{\wp}: = (\imath_{\wp},\jmath_{\wp}) \quad \forall \wp \subseteq \CPP(\ckcOg).
  \]
  Note that $\imath_{\wp}\neq \jmath_{\wp}$ since
  $\bfcc_{1}(\imath_{\wp})\geq \bfcc_{1}(\jmath_{\wp})$. Therefore, one can see
  that
  \begin{equation}\label{eq:W''}
    \Ind_{W_{b}\times W_{g}}^{W''} \tau_{b}\boxtimes \tau_{\wp}
    = (\wttau_{b}\otimes \wttau_{\wp})|_{W''}.
  \end{equation}

  \trivial[]{ When $\nbb=0$, $W'' = W_{g} = \sfW'_{\ngg}$ and so
    $\wttau_{\wp}|_{\sfW'_{\ngg}} = \tau_{\wp}$.

    Now we we assume $\nbb\neq 0$ and $\ngg\neq 0$ (the general case). This
    follows from the following points
    \begin{itemize}
      \item the dimension of the two sides are equal ($W_{b}\times W_{g}$ has
            index $2$ in $W''$).
      \item
            \[
            \begin{split}
              &[\Ind_{W_{b}\times W_{g}}^{W''}\tau_{b}\boxtimes \tau_{\wp}:(\wttau_{b}\otimes \wttau_{\wp})|_{W''}] \\
              =& [\Ind_{\sfW'_{n_{b}}\times \sfW'_{n_{g}}}^{\sfW_{n_{b}}\times \sfW_{n_{g}}}\tau_{b}\boxtimes \tau_{\wp}:\wttau_{b}\otimes \wttau_{\wp}]\\
              =& [\wttau_{b}\boxtimes \wttau_{\wp} \oplus \wttau_{b}\boxtimes (\wttau_{\wp}\otimes \brsgn):\wttau_{b}\otimes \wttau_{\wp}] =1\\
            \end{split}
            \]
            where $\wttau_{\wp}\otimes \brsgn$ has the bipartition obtained by
            switching the left and right side of $\wttau_{\wp}$.
      \item the LHS is irreducible, by
            \[
            \begin{split}
              & [\Ind_{W_{b}\times W_{g}}^{W''}\tau_{b}\boxtimes \tau_{\wp}:
              \Ind_{W_{b}\times W_{g}}^{W''}\tau_{b}\boxtimes \tau_{\wp}]_{W''}\\
              =&  [\tau_{b}\boxtimes \tau_{\wp} : (\Ind_{W_{b}\times W_{g}}^{W''}\tau_{b}\boxtimes \tau_{\wp})|_{W_{b}\times W_{g}}]\\
              =& [\tau_{b}\boxtimes \tau_{\wp} : \tau_{b}\boxtimes \tau_{\wp} + (\cOpb,\cOpb)_{II} \boxtimes \tau_{\wp} ] = 1
            \end{split}
            \]
    \end{itemize}
  }

  Now we have
  \[
    \begin{split}
      & [\tau_{b}\boxtimes \tau_{\wp} :
      \Ind_{\sfW'_{n_{b}}\times \sfW'_{n_{g}}}^{W''} \cC_{b}^{\nbb} \boxtimes \cC_{g}^{\ngg}]_{\sfW'_{n_{b}}\times \sfW'_{n_{g}}}\\
      = & [\Ind_{W_{b}\times W_{g}}^{W''} \tau_{b}\boxtimes \wttau_{\wp} :
      \Ind_{W_{b}\times W_{g}}^{W''} \cC_{b}^{\nbb} \boxtimes \cC_{g}^{\ngg}]_{W''}\\
      = & [(\wttau_{b}\boxtimes \wttau_{\wp})|_{W''}:
      \Ind_{W_{b}\times W_{g}}^{W''} \cC_{b} \boxtimes \cC_{g}]_{W''}\\
      = & [\wttau_{b}\boxtimes \wttau_{\wp}:
      \Ind_{W_{b}\times W_{g}}^{\sfW_{n_{b}}\times \sfW_{n_{g}}} \cC_{b} \boxtimes \cC_{g}]_{\sfW_{n_{g}}\times \sfW_{n_{b}}}\\
      =& \# \PBP_{\star}(\ckcO_{b})\cdot \# \PBP_{\star}(\wttau_{\wp})
    \end{split}
  \]
  The last equality follows from the branching rules of $\sfW_{n}$ where the
  contribution of $\sfH_{t}$ corresponding to ``$\bullet$'',
  the contribution of $\sfS_{a}$ corresponding to the ``$s$'' on the left and
  ``$r$'' on the right.


  % Suppose $n_{b}=0$ then
  % \[
  %   \begin{split}
  %     [\tau_{\wp} : \cC_{g}]_{\sfW'_{n_{g}}} = &
  %     [\wttau_{\wp}|_{W_{g}}:\sum_{2t+a=n_{g}} \Ind_{\sfH_{t}\times \sfS_{a}}^{\sfW'_{n_{g}}}\tsgn\otimes \sgn]_{\sfW'_{n_{g}}}\\
  %     = & [\wttau_{\wp}: \sum_{2t+a=n_{g}} \Ind_{\sfH_{t}\times \sfS_{a}}^{\sfW'_{n_{g}}}\tsgn\otimes \sgn]_{\sfW_{n_{g}}}\\
  %     = & \PBP_{\star}(\ttau_\wp)
  %   \end{split}
  % \]


  \trivial[]{

    Suppose $\star =C^{*}$.

    For the bad parity, $n_{b}=2n'_{b}$ must be even.
    \[
      \begin{split}
        & [\tau_{b}\boxtimes \tau_{\wp} :
        \cC_{b}^{\nbb} \boxtimes \cC_{g}^{\ngg,\ngg}]_{\sfW'_{n_{b}}\times \sfW_{n_{g}}}\\
        = & [\Ind_{\sfW'_{n_{b}}\times \sfW_{n_{g}}}^{\sfW_{n_{b}}\times \sfW_{n_{g}}} \tau_{b}\boxtimes \tau_{\wp} :
        \cC_{b}^{\nbb} \boxtimes \cC_{g}^{\ngg,\ngg}]_{\sfW_{n_{b}}\times \sfW_{n_{g}}}\\
        = & [\wttau_{b}\boxtimes \tau_{\wp}:
        \cC_{b}^{\nbb} \boxtimes \cC_{g}^{\ngg,\ngg}]_{\sfW_{n_{b}}\times \sfW_{n_{g}}}\\
        =& \# \PBP_{\star}(\ckcO_{b})\cdot \# \PBP_{\star}(\tau_{\wp})
      \end{split}
    \]

    The alternative approach using restriction. For the bad parity,
    $n_{b}=2n'_{b}$ must be even.
    \[
      \begin{split}
        \cC_b^{n_{b}} &=
        \Res_{\sfW_{n_{b}}}^{\sfW'_{n_{b}}} \Ind_{\sfH_{n'_{b}}}^{\sfW_{n_{b}}} \hsgn \\
        &= \bigoplus_{\sigma\in \Irr(\sfS_{n'_{b}})} \left((\sigma,\sigma)_{I} \oplus (\sigma,\sigma)_{II}\right).
      \end{split}
    \]
    For the good parity,
    \[
      \begin{split}
        \cC_g^{2p,2q} %& = \bigoplus_{p+q=m} \Cint{\rho}(\Sp(p,q)) \\
        & =\bigoplus_{\substack{(t+s,t+r)=(p,q)}} \bigoplus_{\sigma} \Ind_{\sfW_{2t}\times \sfW_s\times \sfW_r}^{\sfW_{p+q}}
        (\sigma,\sigma)\otimes \sgn \otimes \sgn \\
      \end{split}
    \]
    Now the branching rule implies the irreducible components of
    $\cC_{g}^{2p,2q}$ are given by the dot-diagram attaching two columns on the
    right, which we mark them by $s$ and $r$ respectively. }


  % Suppose $\star=C$
  % \[
  %   \begin{split}
  %     & [\tau_{b}\boxtimes \tau_{\wp} :
  %     \cC_{b}^{\nbb} \boxtimes \cC_{g}^{\ngg,\ngg}]_{\sfW'_{n_{b}}\times \sfW_{n_{g}}}\\
  %     = & [\Ind_{\sfW'_{n_{b}}\times \sfW_{n_{g}}}^{\sfW_{n_{b}}\times \sfW_{n_{g}}} \tau_{b}\boxtimes \tau_{\wp} :
  %     \cC_{b} \boxtimes \cC_{g}]_{\sfW_{n_{b}}\times \sfW_{n_{g}}}\\
  %     = & [\wttau_{b}\boxtimes \tau_{\wp}:
  %     \cC_{b} \boxtimes \cC_{g}]_{\sfW_{n_{b}}\times \sfW_{n_{g}}}\\
  %     =& \# \PBP_{\star}(\ckcO_{b})\cdot \# \PBP_{\star}(\ckcO_{g};\wp)
  %   \end{split}
  % \]
%
  % For the bad parity, $n_{b}=2n'_{b}$ must be even.
  % \[
  %   \begin{split}
  %     \cC_b^{n_{b}} &= \Res_{\sfW_{n_{b}}}^{\sfW'_{n_{b}}} \left( \bigoplus_{2t+a=n_{b}}\Ind_{\sfH_{t}\times \sfS_{a}}^{\sfW_{n_{b}}} \hsgn \otimes 1
  %     \right)\\
  %     &= \bigoplus_{2t+c+d=n_{b}}\bigoplus_{\sigma\in \Irr(\sfS_{t})} \left((\sigma,\sigma)_{I} \oplus (\sigma,\sigma)_{II}\right) \times ([d,],[c,]).
  %   \end{split}
  % \]
  % Note that $\tau_{b}=(\cO'_{b},\cO'_{b})_{I}$. Therefore we only need to
  % consider the case when $c=d$ in the above formula.
  % \[
  %   \begin{split}
  %     [\tau_{b}: \cC_{b}^{n_{b}}] & = [\cO'_{b}:\bigoplus_{\substack{t+d = n'_{b}\\ \sigma}} \sigma \times 1].
  %   \end{split}
  % \]
  % By the counting of unipotent representation of $\GL(n'_{b},\bR)$. We see
  % that $\PBP_{\star,b}(\ckcO_{b})$ is identified with
  % $\PBP_{A^{\bR}}(\ckcO'_{b})$ by send $(\uptau_{L},\uptau_{R})$ to the
  % painted partition $\uptau'$ such that
  % \[
  %   \uptau_{L}(i,j)=d \Leftrightarrow \uptau'(i,j)=d.
  % \]

  % For the good parity, this is clear by the branching rules.


  \trivial{


    Suppose $\star = D$. Suppose that $n_{g}\neq 0$. Since
    $\bfcc_{1}(\imath_{\wp})>\bfcc_{1}(\jmath_{\wp})$, we have
    $\wttau^{s}_{\wp}:=\wttau_{\wp}\otimes \brsgn\ncong\wttau_{\wp}$.
    \[
      \begin{split}
        & [\tau_{b}\boxtimes \tau_{\wp} :
        \cC_{b} \boxtimes \cC_{g}]_{\sfW'_{n_{b}}\times \sfW'_{n_{g}}}\\
        = & [\Ind_{\sfW'_{n_{b}}\times \sfW'_{n_{g}}}^{\sfW_{n_{b}}\times \sfW_{n_{g}}} \tau_{b}\boxtimes \tau_{\wp} :
        \cC_{b} \boxtimes \cC_{g}]_{\sfW_{n_{b}}\times \sfW_{n_{g}}}\\
        = & [\wttau_{b}\boxtimes \wttau_{\wp}\oplus \wttau_{b}\boxtimes \wttau_{\wp}^{s}:
        \cC_{b} \boxtimes \cC_{g}]_{W''}\\
        = & [\wttau_{b}\boxtimes \wttau_{\wp}:
        \cC_{b} \boxtimes \cC_{g}]_{\sfW_{n_{g}}\times \sfW_{n_{b}}}\\
        =& \# \PBP_{\star}(\ckcO_{b})\cdot \# \PBP_{\star}(\ckcO_{g};\wp)
      \end{split}
    \]
    The terms involving $\wttau_{\sP}^{s}$ vanish since every irreducible
    component $(\sigma_{L},\sigma_{R})$ in $\cC_{g}$ satisfies
    $\sigma_{L}\supseteq \sigma_{R}$ but
    $\bfcc_{1}(\imath_{\wp})>\bfcc_{1}(\jmath_{\wp})$.

    Suppose that $n_{g} = 0$.
    \[
      \begin{split}
        & [\tau_{b} : \cC_{b}^{n_{b}}]_{\sfW'_{n_{b}}}\\
        =& [\Ind_{\sfW'_{n_{b}}}^{\sfW_{n_{b}}} \tau_{b} : \bigoplus_{\substack{2t+a=n}}
        \Ind_{\sfH_{t}\times \sfS_{a}}^{\sfW_{n}} \hsgn\otimes 1] \\
        =& [ \wttau_{b} : \bigoplus_{\substack{2t+a=n}}
        \Ind_{\sfH_{t}\times \sfS_{a}}^{\sfW_{n}} \hsgn\otimes 1] \\
      \end{split}
    \]
    In any cases, the counting formula holds. There is place to confuse: Why
    there shouldn't be double the size of special unipotent representations?

    In fact, $\AC_{\bC}(\pi)$ can only be the fixed type, say $\cO_{I}$! Note
    that we fixed an infinitesimal character which has half-integral values.
    This choice implicitly force us to fix real Siegel parabolic when we do
    induction from $\GL$! The non-trivial outer automorphism, say $c$, will
    permute the infinitesimal character to the another one and we then will have
    $\AC_{\bC}({}^{c}\pi)i = \cO_{II}$.


    Using Barbasch's formula of wavefront, we see that the induction
    $\pi_{I}:=\Ind_{\GL}^{\SO}\pi'$ must be irreducible, where $\pi'$ is a
    unipotent representation of $\GL$. This will also implies
    $\Ind_{\GL}^{\rO}\pi'$ is irreducible and restricted to two $\SO$-modules,
    $\pi_{I}$ and $\pi_{II}$.

  }


  \trivial[]{ Suppose $\star = \wtC$. Then
    \[
      \begin{split}
        & [\tau_{b}\boxtimes \tau_{\wp} :
        \cC_{b} \boxtimes \cC_{g}]_{\sfW_{n_{b}}\times \sfW'_{n_{g}}}\\
        = & [ \tau_{b}\boxtimes \Ind_{\sfW'_{n_{g}}}^{\sfW_{n_{g}}}\tau_{\wp} : \cC_{b} \boxtimes \cC_{g}]_{\sfW_{n_{b}}\times \sfW_{n_{g}}}\\
        = &\sum_{\sP\in \tA(\ckcO)} [\tau_{b}\boxtimes \wttau_{\wp}:
        \cC_{b} \boxtimes \cC_{g}]_{\sfW_{n_{b}}\times \sfW_{n_{g}}}\\
        =& \# \PBP_{\star}(\ckcO_{b})\cdot \# \PBP_{\star}(\ckcO_{g})
      \end{split}
    \]

    Suppose $\star = B$. Then
    \[
      \begin{split}
        & \sum_{\sP\in \tA'(\ckcO)}[\tau_{b}\boxtimes \tau_{\wp} :
        \cC_{b} \boxtimes \cC_{g}]_{\sfW_{n_{b}}\times \sfW'_{n_{g}}}\\
        =& \# \PBP_{\star}(\ckcO_{b})\cdot \# \PBP_{\star}(\ckcO_{g};\wp)
      \end{split}
    \]
  }
\end{proof}


\subsection{Reduction to the good parity case}

In this section, $G$ is a classical group or metaplectic group.

\begin{lem}\label{lem:Unip.BP}
  Suppose $n_{g}=0$. Then we have a bijection
  \[
    \begin{array}{rccc}
      \fI_{b}: &\Unip_{G'}(\ckcO'_{b}) & \longrightarrow & \Unip_{G}(\ckcO) \\
      &\pi' & \mapsto & \pi:=\Ind_{P}^{G}\pi'
    \end{array}
  \]
  where $P$ is the standard parabolic subgroup of $G$ whose Levi subgroup is
  isomorphic to $G'$.
  Let $\sO'$ be the real nilpotent orbit in $G'$ such that $\sO'_{\bC}=\cO$.
  and $\sO$ be the real induction of $\sO'$ to $G$.
  Then
  \[
    \AC(\pi) = \sO
  \]
  is multiplicity one.
\end{lem}
\begin{proof}
  Note that $\WF(\pi)=\sO'$.
  It follows from Barabasch's formula on wavefront cycle that the
  associated variety of $\pi:=\Ind_{P}^{G}\pi'$ is $\sO$ with multiplicity one.
  This immediately implies that $\pi$ is irreducible. In fact, if $\pi$ is
  reducible, $\pi$ must contain a irreducible sub-quotient with infinitesimal
  character $\lambda_{\ckcO}$ and GK-dimension $<\half\dim_{\bC}\cO$, this
  is contradict to \Cref{lem:LC.mu}.
  \trivial[]{
    First, it is not obvious to me that the wavefront of $\Ind_{P}^{G}\pi$
    must be contained in $\Ind\WF(\pi)$. But it is clear that the leading term
    must be $\sum_{\sO\text{ open in } \WF(\pi)}\Ind\sO$. So the boundaries has
    less GK-dimension.

    Suppose $\pi_{0}$ is the sub-quotient with less GK-dimension.
    On the other hand, the maximal primitive ideal $\cI_{\ckcO}$  with infinitesimal
    character must contains $\Ann\pi_{0}$. In other words,
    $\AV_{\bC}(\pi_{0})\subseteq \bcO$ which implies GK-dimension of
    $\pi_{0}\geq \half\dim_{\bC}\cO$, a contradiction.
    % Let $\mu = \lamck$.
    % Note that every $W_{[\mu]}$-module in $\Grt_{\mu}(G)$ is in $\Ind_{W_{\mu}}^{W_{[\mu]}}$
  }
  Note that representations in $\Unip_{G'}(\ckcO'_{b})$ have distinct
  cuspidal data/Langlands parameter. This implies that $\Ind_{P}^{G}\pi'$ has distinct cuspidal data/Langlands when $\pi'$ varies.
  Recall that $\star' \in \set{A^{\bR},A^{\bC}, A^{\bH}}$ depends on $G$.
  Therefore, $\fI_{b}$ is injective. The bijection follows from the counting
  inequality below:
  \[
    \abs{\PP_{\star'}(\ckcO'_{b})}=\abs{\Unip_{G'}}(\ckcO'_{b})\leq \abs{\Unip_{G}(\ckcO)}
    \leq \abs{\PBP_{\star,b}(\ckcO_{b})} = \abs{\PP_{\star'}(\ckcO'_{b})}.
  \]
\end{proof}


Case by case,  we set
\[
  (G_{b},G_{g}) =
  \begin{cases}
    (\SO(n_{b},n_{b}+1),\SO(p,q)) & \text{when } \star = B \\
    (\Sp(2n_{b},\bR),\Sp(2n_{g},\bR)) & \text{when } \star = C \\
    (\Sp(n_{b},n_{b}), \Sp(p,q)) & \text{when } \star = C^{*} \\
    (\Mp(2n_{b},\bR),\Mp(2n_{g},\bR)) & \text{when } \star = \wtC \\
    (\rO^{*}(n_{b}), \rO^{*}(n_{g})) & \text{when } \star = D^{*} \\
    (\SO(n_{b},n_{b}), \SO(p,q) )& \text{when } \star = D \\
  \end{cases}
\]
Let $G'$ be the Levi of the Siegel parabolic in $G_{b}$.

\begin{prop}\label{prop:red}
  There is a bijection
  \begin{equation}\label{eq:IND}
      \begin{array}{rccc}
    \fI\colon &   \Unip_{G'}(\ckcO'_{b})\times \Unip_{G_{g}}(\ckcO_{g})&         \longrightarrow &\Unip_{G}(\ckcO) \\
     &   (\pi',\pi_{0}) & \mapsto & \pi'\rtimes \pi_{0}.
      \end{array}
    \end{equation}
  \end{prop}

% We would like to reduce the problem to consider the bad and good parts
% separately.
First we translate the problem to
regular infinitesimal character using the following lemma:
 The method was already carefully explained in \cite{Mat} and see
\cite{GI}*{Section~3} for a comprehensive account of the problem.


\def\fhhaso{(\fhh^a_1)^*}
\def\fhhast{(\fhh^a_2)^*}
\newcommand{\ff}{f}
\newcommand{\ffcoh}{\varphi}

\begin{lem}[{c. f. \cite{GI}*{Lemma~3.3}}]\label{lem:ff.irr}
  Suppose that
  \begin{enumerate}[label=(\roman*),series=KLff]
    \item \label{it:KLff.1} $G_{1}$ and $G_{2}$ are two real reductive groups in
          the Harish-Chandra class
          \item there is an isomorphism
          \[ \ff\colon \fhhaso\rightarrow \fhhast
          \]
          of the dual of the abstract Cartans of $G_{1}$ and $G_{2}$;
    \item $\lambda_{1} \in \fhhaso$ and $\lambda_{2} = \ff(\lambda_{1})$ are
          regular dominant elements;
          % \item $\lambda_{1}$ and $\lambda_{2}$ are regular;
    \item \label{it:KLff.4} $\ff$ induces a bijection between
          $R^{+}_{[\lambda_{1}]}$ and $R^{+}_{[\lambda_{2}]}$, so that we can
          identify $W_{[\lambda_{1}]}$ with $W_{[\lambda_{2}]}$ via $\ff$;
  \end{enumerate}
  Let $\Coh_{1}$ be a $W_{[\lambda_{1}]}$ submodule of
  $\Coh_{[\lambda_{1}]}(G_{1})$ such that $\ev{\lambda_{1}}(\Coh_{1})$ is
  spanned by irreducible $G_{1}$-modules. Suppose
  \begin{equation}\label{eq:coh.ff}
    \ffcoh\colon \Coh_{1}\longrightarrow \Coh_{[\lambda_{2}]}(G_{2})
  \end{equation}
  is an injection between $W_{[\lambda_{1}]}=W_{[\lambda_{2}]}$-modules such
  that
  \begin{enumerate}[KLff]
    \item \label{it:KLff.5} $\ffcoh(\Theta)(\lambda_{2})$ is irreducible if $\Theta\in \Coh_{1}$
          and $\Theta(\lambda_{1})$ is irreducible.
  \end{enumerate}

  Then for any $\mu_{1}\in [\lambda_{1}]$, the evaluation at $\mu_{1}$ induces
  an injection
  \[
    \ffcoh_{\mu_{1}} \colon \ev{\mu_{1}}(\Coh_{1}) \longrightarrow \ev{\ff(\mu_{1})}(\Coh_{[\lambda_{2}]}(G_{2})).
  \]
  such that $\ffcoh_{\mu_{1}}(\pi)$ is irreducible if $\pi$ is irreducible.
\end{lem}
\begin{proof}
  The injectivity of $\ff_{\mu_{1}}$ is clear from \Cref{lem:coh.count} and the
  injectivity of \eqref{eq:coh.ff}. \trivial[]{Note that $W_{[\lambda_{1}]}$ is a
    finite group!}

  We now prove the second claim. It easy to reduce to the case when $\mu_{1}$ is
  $R^{+}_{[\lambda_{1}]}$ dominant, c.f. \cite{GI}*{Lemma~3.3}.
  Let $\Theta\in \Coh_{1}$ such that
  $\Theta(\mu_{1})=\pi$ and $\Theta(\lambda_{1})$ is irreducible (the existence
  of $\Theta$ is an abstract property of the coherent continuation). By our
  assumption $\ffcoh(\Theta)(\lambda_{1})$ is irreducible. Therefore
  $\ffcoh_{\mu_{1}}(\pi):=\ffcoh(\Theta)(\ff(\mu_{1}))$ must be either
  irreducible or zero (one of the abstract property of coherent continuation).
  Since it is non zero by the injectivity  of $\ffcoh_{\mu_{1}}$, the lemma
  follows.
\end{proof}

In this paper, we use the Kazhdan-Lusztig-Vogan theory to obtain the injection
\eqref{eq:coh.ff} and then reduce the problem to bad and good parities
separately.
Note that the method does not really relies on the Vogan duality and nor
require the whole coherent continuation module are isomorphism.

In additional to \ref{it:KLff.1}-\ref{it:KLff.5} in
\Cref{lem:ff.irr}, we made the following assumptions (c.~f. \cite{GI}*{\S 3E}):
\begin{enumerate}[KLff]
  % \item $G_{1}$ and $G_{2}$ are two real reductive groups ;
  % \item there is an isomorphism $\ff\colon \fhha_{1}\rightarrow \fhha_{2}$
  % between abstract Cartan subalgebras $\fhha_{1}$ and $\fhha_{2}$ of $G_{1}$
  % and $G_{2}$ respectively; ;
  % \item $\lambda_{1}\in \fhhaso$ and $\lambda_2\in \fhhast$ are fixed regular
  % elements such that $\lambda_{1} = \lambda_{2}\circ \ff$;
  % \item $\ff$ induces an isomorphism
  % $\ff\colon R_{\lambda_{1}}^{+}\rightarrow R_{\lambda_{2}}^{+}$, and the
  % associated integral Weyl groups
  % $\ff\colon W_{[\lambda_{1}]}\rightarrow W_{[\lambda_{2}]}$;
  \item there is an injection
        \[
        \ff \colon B\rightarrow \cP_{\lambda_{2}}(G_{2})
        \]
        where $B\subseteq \cP_{\lambda_{1}}(G_{1})$ is a union of blocks of
        $G$-conjugate classes of regular characters with infinitesimal character
        $\lambda_{1}$.
  \item for $\gamma_{1}\in B$ and $\gamma_{2} = \ffcoh(\gamma_{1})\in \cP_{\lambda_{2}}(G_{2})$ the
        following conditions are satisfied
        \begin{enumerate}[label=(\alph*)]
          \item $\ff \circ \theta_{\gamma_{1}} = \theta_{\gamma_{2}}\circ \ff$
                where $\theta_{\gamma_{i}}$ are the Cartan involution induced on
                the corresponding abstract root systems. \trivial[]{ This
                condition implies that, the notion of compact/complex/real and
                $\alpha\in R^{+}(\gamma), \theta(\alpha)\notin R^{+}(\gamma)$
                are preserved by $\ff$. In particular, the integral length
                function $l^{I}$ (defined up to a shifting) of $G_{1}$ and
                $G_{2}$ can be uniformly identified. }
              % \item Suppose $\alpha_{1}$ is noncompact type I (resp.
              % noncompact/real type I/II) if and only if
              % $\alpha_{2}:= \ff(\alpha_{1})$ is noncompact type I (resp. type
              % II).
          \item For simple roots in $R^{+}_{\lambda_{i}}$, the notions of
                noncompact/real type I/II are preserved by $\ff$: $\alpha_{1}$
                is noncompact type I if and only if
                $\alpha_{2}:= \ff(\alpha_{1})$ is noncompact type I, etc;
          \item The cross actions are compatible:
                \[
                \ff(w\cross [\gamma_{1}]) = \ff(w)\cross [\ff(\gamma_{1})] \qquad \forall \gamma_{1}\in B_{1}, w\in W_{[\lambda_{1}]}.
                \]
          \item The Cayley transforms \cite{V4} are compatible:
                \[
                \ff( c^{\alpha_{1}} (\gamma_{1})) = c^{\ff(\alpha_{1})}(\ff(\gamma_{1})) \qquad \forall \gamma_{1}\in B_{1}, \alpha_{1} \text{is
                noncompact imaginary}
                \]
                and
                \[
                  \ff( c_{\alpha_{1}} (\gamma_{1})) = c_{\ff(\alpha_{1})}(\ff(\gamma_{1}))
                \]
                for all $\gamma_{1}\in B_{1}, \alpha_{1}$ is
                real and satisfies parity condition.
          % \item The $\tau$ invariants are compatible:
          %       \[
          %       \ff(\tau(\gamma_{1})) = \tau(\ff(\gamma_{1}))\qquad \forall \gamma_{1}\in B_{1}.
          %       \]
          %       \trivial[]{ This condition seems not been used below, but used
          %       in Gan-Ichino's proof! For linear group the matching of
          %       $\tau$-invariant is automatic since it is explicitly determined
          %       by the types of simple roots. For metaplectic groups, it needs
          %       extra information to determine the $\tau$-invariant, see
          %       \cite{RT1}*{Lemma~6.28}. It is not clear to me that the
          %       $\tau$-invariant must match.

          %       On the other hand, the proof of \Cref{lem:ff.irr} seems implies
          %       that $\tau$-invariants match automatically! }
        \end{enumerate}
\end{enumerate}

%For a fixed block $B$ a regular infinitesimal character $\lambda_{1}$,
Under the above assumptions, let $\Grt_{B}$ be the span of
$\set{\barpi_{\gamma}|\gamma\in B}$ in the
Grothendieck group, and $\Coh_{1} := \ev{\lambda}^{-1}(\Grt_{B})$.
% What we really need is the validity of the following
% conditions:
% \begin{itemize}
%   \item there is an injection of $W_{[\lambda_{1}]}=W_{[\lambda_{2}]}$-module
%   \begin{equation}\label{eq:coh.ff}
%     \ff\colon \Coh_{B_{1}}\rightarrow \Coh_{B_{2}}
%   \end{equation}
%   such that, at our fixed regular infinitesimal character $\lambda_{1}$,
%   $\Theta(\lambda_{1})$ is irreducible implies $\ff(\Theta)(\ff(\lambda_{1}))$
%   is irreducible.
% \end{itemize}
Then $\ffcoh$ is given by sending $\Theta_{\gamma_{1}}$ to
$\Theta_{\ff(\gamma_{1})}$. In fact, the above map on coherent continuation
representations can be lifted to maps between Hecke-modules and the validity of
Kazhdan-Lusztig-Vogan conjecture implies the preservation of irreducibility of
$\ffcoh$ at $\lambda_{1}$.


% \begin{lem}[{c. f. \cite{GI}*{Lemma~3.3}}]\label{lem:ff.irr}
%   Suppose $\mu_{1}\in [\lambda_{1}]$ is dominant and $\Theta\in \Coh_{B_{1}}$.
%   The evaluation at $\mu_{1}$ induces an injection
%   \[
%   \ff_{\mu_{1}}  \ev{\mu_{1}}(\Coh_{B_{1}}) \longrightarrow  \ev{\mu_{2}}(\Coh_{B_{2}}).
%   \]
%   Moreover, $\ff_{\mu_{1}}(\pi)$ is irreducible if $\pi$ i irreducible.
% \end{lem}
% \begin{proof}
%   The injectivity of $\ff_{\mu_{1}}$ is clear from \Cref{lem:coh.count} and the
%   injectivity of \Cref{eq:coh.ff}. \trivial[]{Note that $W_{[\lambda_{1}]}$ is a
%     finite group!}

%   We now prove the second claim. Let $\Theta\in \Coh_{B_{1}}$ such that
%   $\Theta(\mu_{1})=\pi$ and $\Theta(\lambda_{1})$ is irreducible (the existence
%   of $\Theta$ is an abstract property of the coherent continuation). By our
%   assumption $\ff(\Theta)(\lambda_{1})$ is irreducible. Therefore
%   $\ff(\Theta)(\ff(\mu_{1}))$ must be irreducible since it is non zero by the
%   first claim.
% \end{proof}


We set $G_{1} := G_{b}\times G_{g}$ and $G_{2}:=G$.
There is a natural map
\[
\ff\colon G_{1} = G_{b}\times G_{g}\longrightarrow G = G_{2}.
\]
The map is given by \cite{GI}*{\S 3G} when $\star = B$ and given by
\cite{RT2}*{\S 5} when $\star = \wtC$. In the other cases, they are natural
embeddings. The maps between abstract Cartans
$\ff\colon \fhhaso\rightarrow \fhhast$ are given by putting the coordinates of
$G_{b}$ before $G_{g}$.

Let $\lambda \in\fhhast $ be a regular dominant element in $[\lamck]$.
Then
$\lambda_{1}:= \ff^{-1}(\lambda)$ is regular dominant for $G_{1}$.
Let $B := \cP_{\lambda_{1}}(G_{1})$ in all the case.
The map $\ff\colon B\rightarrow \cP_{\lambda}(G_{2})$ is given by the natural maps between real Cartans.
This is clear when $\ff$ is an embedding.
See \cite{GI}*{\S 3} when $G$ is a special orthogonal group and
\cite{RT2}*{\S 5} when $G$ is a metaplectic group.

\begin{proof}[{Proof of \Cref{prop:red}: the injectivity of $\fI$}]
  Let % $\mu_{2} = \lamck$
      % and
  $\mu_{1} = \ff^{-1}(\lamck)$.
  Since coherent continuation is compatible with induction
  \cite{Vg}*{Proposition~7.4.1}, we have the
  following commutative diagram
  \[
    \begin{tikzcd}[column sep={4cm,between origins}]
      &  \Coh_{[\lambda'_{b}]}(G'_{b})\otimes \Coh_{[\lambda_{g}]}(G_{g}) \ar[dl,"\Ind_{P_{b}}^{G_{b}}\otimes \id"']\ar[dr,"\Ind_{P}^{G}"]&\\
      \Coh_{[\lambda_{b}]}(G_{b})\otimes \Coh_{[\lambda_{g}]}(G_{g}) \ar[d,"\ev{\mu_{1}}"'] \ar[rr,"\ffcoh"]& & \Coh_{[\lamck]}(G) \ar[d,"\ev{\ckcO}"]\\
      \Grt_{\mu_{1}}(G_{b}\times G_{g}) \ar[rr,"\ffcoh_{\mu_{1}}"]& &
      \Grt_{\lamck}\\
    \end{tikzcd}
  \]
  Here $P_{b}$ is the parabolic subgroup of $G_{b}$ whose Levi subgroup is
  isomorphic to $G'_{b}$ and $P$ is the parabolic of $G$ whose Levi subgroup is
  isomorphic to $G'_{b}\times G_{g}$. The horizontal line $\ffcoh$ is an
  injection by the above setting, which implies $\ffcoh_{\mu_{1}}$ is injective
  by \Cref{lem:ff.irr}. Note that $\ffcoh_{\mu_{1}}$ sends
  $\Ind_{P_{b}}(\pi')\otimes \pi_{0}$ to $\Ind_{P}^{G} (\pi'\otimes \pi_{0})$.
  \trivial{ Let
    $\Theta'\otimes \Theta_0\in \Coh_{[\lambda'_{b}]}(G'_{b})\otimes \Coh_{[\lambda_{g}]}(G_{g})$
    be coherent family such that
    $\Theta'\otimes \Theta_0(\lamck) = \pi'\otimes \pi_0$. Now
    $\Ind_{P_{b}}(\pi')\otimes \pi_{0} = (\Ind_{P_{b}}\otimes \id) (\Theta'\otimes \Theta_{0})(\mu_{1})$
    and
    $\Ind_{P}(\pi'\otimes \pi_{0}) = \ffcoh(\Ind_{P_{b}}\otimes \id (\Theta'\otimes \Theta_{0}))(\lamck) = (\Ind_{P}^{G}) (\Theta'\otimes \Theta_{0})(\lamck)$
  }
  Now the proposition follows from \Cref{lem:Unip.BP}.
\end{proof}


\begin{lem}\label{lem:BGcount}
  We have
  \[
    \abs{\Unip_{G}(\ckcO)} =
    \abs{\Unip_{G_{b}}(\ckcO_{b})}\cdot
    \abs{\Unip_{G_{g}}(\ckcO_{g})}
  \]
\end{lem}
\begin{proof}
  It suffice to consider the case where $\nbb$ and $\ngg$ are both non-zero.
  Fix a regular dominant infinitesimal character $\lambda\in [\lamck]$ and write
  $(\lambda_{b},\lambda_{g}):=\ff^{-1}(\lambda)$. It requires a some more
  precise information about the blocks/cell of $G$. Suppose $\star \neq D^{*}$.
  Then
  \[
    \ffcoh \colon \cP_{\lambda_{b}}(G_{b})\times \cP_{\lambda_{g}}(G_{g}) \longrightarrow \cP_{\lambda}(G)
  \]
  is a bijection. In particular each Harish-Chandra cell in $\cP_{\lambda}(G)$
  is the product of a cell in $\cP_{\lambda_{b}}(G_{b})$ and a cell in
  $\cP_{\lambda_{g}}(G_{g})$. By \Cref{lem:AV.HC}, this implies $\ffcoh$
  restricted to an isomorphism
  \[
    \Coh_{[\lambda_{b}],\bcO_{b}}(G_{g})\otimes \Coh_{[\lambda_{g}],\bcO_{g}}(G_{b}) \xrightarrow{\ \ \ \ffcoh\ \ \ } \Coh_{[\lambda],\bcO}(G).
  \]
  and the lemma follows by the evaluation to $\lamck$.

  Now consider the case where $\star = D^{*}$. In this case, $\cP_{\lambda}(G)$
  has $2$ blocks. Meanwhile $\cP_{\lambda_{b}}(G_{b})$ and
  $\cP_{\lambda_{g}}(G_{g})$ only have $1$ block. Let
  $B_{1} := \ff(\cP_{\lambda_{b}}(G_{b})\times \cP_{\lambda_{g}}(G_{g}))$ and
  $B_{2}$ be the other block. For any $\sfW'_{n}$-module $\tau$, let $\tau^{s}$
  denote the twist of $\tau$ by any non-trivial element in $\sfW_{n}/\sfW'_{n}$.

  Then \[
    \begin{split}
      \Coh_{B_{1}} &\cong \cC_{b}^{\nbb}\otimes \cC_{g}^{\ngg,\ngg} = \Coh_{[\lambda_{b}],\bcO_{b}}(G_{g})\otimes \Coh_{[\lambda_{g}],\bcO_{g}}(G_{b}),\\
      \Coh_{B_{2}} &\cong (\cC_{b}^{\nbb})^{s}\otimes (\cC_{g}^{\ngg,\ngg})^{s}
    \end{split}
  \]
   and
  \[
    \Coh_{[\lambda]}(G) = \Coh_{B_{1}}\oplus \Coh_{B_{2}}.
  \]
  Clearly $\Coh_{[\lambda], \bcO}$ is compatible with the above decomposition
  and we have
  \[
    \Coh_{[\lambda_{b}],\bcO_{b}}(G_{g})\otimes \Coh_{[\lambda_{g}],\bcO_{g}}(G_{b}) \xrightarrow{\ \ \ \ffcoh\ \ \ } \Coh_{[\lambda],\bcO}(G)\cap \Coh_{B_{1}}.
  \]

  Observe that $[\tau_{b}:(\cC_{b}^{\nbb})^{s}]=0$. Therefore
  \[
    \begin{split}
      \abs{\Unip_{G}(\ckcO)} & = \sum_{\tau_{b}\boxtimes \tau_{g}\in \LC_{\ckcO}} [\tau_{b}\otimes \tau_{g}:\Coh_{[\lambda],\bcO}(G)\cap \Coh_{B_1}]\\
    & \ \ + \sum_{\tau_{b}\boxtimes \tau_{g}\in \LC_{\ckcO}}[\tau_{b}\otimes \tau_{g}:\Coh_{[\lambda],\bcO}(G)\cap \Coh_{B_2}] \\
  &= \sum_{\tau_{b}\boxtimes \tau_{g}\in \LC_{\ckcO}} [\tau_{b}\otimes \tau_{g}:
  \Coh_{[\lambda_{b}],\bcO_{b}}(G_{g})\otimes \Coh_{[\lambda_{g}],\bcO_{g}}(G_{b})]\\
  &= \abs{\Unip_{G_{b}}(\ckcO_{b})}\cdot \abs{\Unip_{G_{g}}(\ckcO_{g})}
\end{split}
\]
\end{proof}

\begin{proof}[{Proof of \Cref{prop:red}: the bijectivity of $\fI$}]
Now the bijectivity of $\fI$ follows by the counting in  \Cref{lem:Unip.BP} and \Cref{lem:BGcount}.
\end{proof}


\end{document}


%%% Local Variables:
%%% mode: latex
%%% TeX-master: "counting_main"
%%% End:
