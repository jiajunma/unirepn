\message{ !name(sunip_new.tex)}\documentclass[12pt,a4paper]{amsart}
\usepackage[margin=2.5cm,marginpar=2cm]{geometry}


\usepackage[bookmarksopen,bookmarksdepth=3]{hyperref}
\usepackage[nameinlink]{cleveref}

\usepackage{array}
%% FONTS
\usepackage{amssymb}
\usepackage{amsmath}
\usepackage{mathrsfs}
\usepackage{mathbbol,mathabx}
\usepackage{amsthm}
\usepackage{graphicx}
\usepackage{braket}
\usepackage{mathtools}

\usepackage{amsrefs}

\usepackage[all,cmtip]{xy}
\usepackage{rotating}
\usepackage{leftidx}
%\usepackage{arydshln}

% circled number
\usepackage{pifont}
\makeatletter
\newcommand*{\circnuma}[1]{%
  \ifnum#1<1 %
    \@ctrerr
  \else
    \ifnum#1>20 %
      \@ctrerr
    \else
      \mbox{\ding{\numexpr 171+(#1)\relax}}%
     \fi
  \fi
}
\makeatother


\DeclareSymbolFont{bbold}{U}{bbold}{m}{n}
\DeclareSymbolFontAlphabet{\mathbbold}{bbold}


%\usepackage[dvipdfx,rgb,table]{xcolor}
\usepackage[rgb,table,dvipsnames]{xcolor}
%\usepackage{color}
%\usepackage{mathrsfs}

\setcounter{tocdepth}{1}
\setcounter{secnumdepth}{3}

%\usepackage[abbrev,shortalphabetic]{amsrefs}


\usepackage{imakeidx}
\def\idxemph#1{\emph{#1}\index{#1}}
\makeindex


\usepackage[normalem]{ulem}


\usepackage[centertableaux]{ytableau}

%\usepackage[mathlines,pagewise]{lineno}
%\linenumbers

\usepackage{enumitem}
%% Enumitem
\newlist{enumC}{enumerate}{1} % Conditions in Lemma/Theorem/Prop
\setlist[enumC,1]{label=(\alph*),wide,ref=(\alph*)}
\crefname{enumCi}{condition}{conditions}
\Crefname{enumCi}{Condition}{Conditions}
\newlist{enumT}{enumerate}{3} % "Theorem"=conclusions in Lemma/Theorem/Prop
\setlist[enumT]{label=(\roman*),wide}
\setlist[enumT,1]{label=(\roman*),wide}
\setlist[enumT,2]{label=(\alph*),ref ={(\roman{enumTi}.\alph*)},left=2em}
\setlist[enumT,3]{label*=.(\arabic*), ref ={(\roman{enumTi}.\alph{enumTii}.\alph*)}}
\crefname{enumTi}{}{}
\Crefname{enumTi}{Item}{Items}
\crefname{enumTii}{}{}
\Crefname{enumTii}{Item}{Items}
\crefname{enumTiii}{}{}
\Crefname{enumTiii}{Item}{Items}
\newlist{enumPF}{enumerate}{3}
%\setlist[enumPF]{label=(\alph*),wide}
\setlist[enumPF,1]{label=(\roman*),wide}
\setlist[enumPF,2]{label=(\alph*),left=2em}
\setlist[enumPF,3]{label=\arabic*).,left=1em}
\newlist{enumS}{enumerate}{3} % Statement outside Lemma/Theorem/Prop
\setlist[enumS]{label=\roman*)}
\setlist[enumS,1]{label=\roman*)}
\setlist[enumS,2]{label=\alph*)}
\setlist[enumS,3]{label=\arabic*.}
\newlist{enumI}{enumerate}{3} % items
\setlist[enumI,1]{label=\roman*),leftmargin=*}
\setlist[enumI,2]{label=\alph*), leftmargin=*}
\setlist[enumI,3]{label=\arabic*), leftmargin=*}
\newlist{enumIL}{enumerate*}{1} %inline enum
\setlist*[enumIL]{label=\roman*)}
\newlist{enumR}{enumerate}{1} % remarks
\setlist[enumR]{label=\arabic*.,wide,labelwidth=!, labelindent=0pt}
\crefname{enumRi}{remark}{remarks}


\newlist{enuma}{enumerate}{1} % Statement in Lemma/Theorem/Prop
\setlist[enuma]{label=(\alph*),nosep,leftmargin=*}

%\definecolor{srcol}{RGB}{255,255,51}

%\definecolor{srcol}{RGB}{255,255,51}
\colorlet{srcol}{black!15}

\crefname{equation}{}{}
\Crefname{equation}{Equation}{Equations}
\Crefname{lem}{Lemma}{Lemma}
\Crefname{thm}{Theorem}{Theorem}

\newlist{des}{enumerate}{1}
\setlist[des]{font=\upshape\sffamily\bfseries, label={}}
%\setlist[des]{before={\renewcommand\makelabel[1]{\sffamily \bfseries ##1 }}}

% editing macros.
%\blendcolors{!80!black}
\long\def\okay#1{\ifcsname highlightokay\endcsname
{\color{red} #1}
\else
{#1}
\fi
}
\long\def\editc#1{{\color{red} #1}}
\long\def\mjj#1{{{\color{blue}#1}}}
\long\def\mjjr#1{{\color{red} (#1)}}
\long\def\mjjd#1#2{{\color{blue} #1 \sout{#2}}}
\def\mjjb{\color{blue}}
\def\mjje{\color{black}}
\def\mjjcb{\color{green!50!black}}
\def\mjjce{\color{black}}

\long\def\sun#1{{{\color{cyan}#1}}}
\long\def\sund#1#2{{\color{cyan}#1  \sout{#2}}}
\long\def\mv#1{{{\color{red} {\bf move to a proper place:} #1}}}
\long\def\delete#1{}

%\reversemarginpar
\newcommand{\lokec}[1]{\marginpar{\color{blue}\tiny #1 \mbox{--loke}}}
\newcommand{\mjjc}[1]{\marginpar{\color{green}\tiny #1 \mbox{--ma}}}


%\def\showtrivial{\relax}

\newcommand{\trivial}[2][]{\if\relax\detokenize{#1}\relax
  {%\hfill\break
   % \begin{minipage}{\textwidth}
      \color{orange} \vspace{0em} $[$  #2 $]$
  %\end{minipage}
  %\break
      \color{black}
  }
  \else
\ifx#1h
\ifcsname showtrivial\endcsname
{%\hfill\break
 % \begin{minipage}{\textwidth}
    \color{orange} \vspace{0em}  $[$ #2 $]$
%\end{minipage}
%\break
    \color{black}
}
\fi
\else {\red Wrong argument!} \fi
\fi
}

\newcommand{\byhide}[2][]{\if\relax\detokenize{#1}\relax
{\color{orange} \vspace{0em} Plan to delete:  #2}
\else
\ifx#1h\relax\fi
\fi
}



\newcommand{\Rank}{\mathrm{rk}}
\newcommand{\cqq}{\mathscr{D}}
\newcommand{\rsym}{\mathrm{sym}}
\newcommand{\rskew}{\mathrm{skew}}
\newcommand{\fraksp}{\mathfrak{sp}}
\newcommand{\frakso}{\mathfrak{so}}
\newcommand{\frakm}{\mathfrak{m}}
\newcommand{\frakp}{\mathfrak{p}}
\newcommand{\pr}{\mathrm{pr}}
\newcommand{\rhopst}{\rho'^*}
\newcommand{\Rad}{\mathrm{Rad}}
\newcommand{\Res}{\mathrm{Res}}
\newcommand{\Hol}{\mathrm{Hol}}
\newcommand{\AC}{\mathrm{AC}}
%\newcommand{\AS}{\mathrm{AS}}
\newcommand{\WF}{\mathrm{WF}}
\newcommand{\AV}{\mathrm{AV}}
\newcommand{\AVC}{\mathrm{AV}_\bC}
\newcommand{\VC}{\mathrm{V}_\bC}
\newcommand{\bfv}{\mathbf{v}}
\newcommand{\depth}{\mathrm{depth}}
\newcommand{\wtM}{\widetilde{M}}
\newcommand{\wtMone}{{\widetilde{M}^{(1,1)}}}

\newcommand{\nullpp}{N(\fpp'^*)}
\newcommand{\nullp}{N(\fpp^*)}
%\newcommand{\Aut}{\mathrm{Aut}}
%\usepackage{mnsymbol}


\def\YD{{\mathsf{YD}}}
\def\SYD{{\mathsf{SYD}}}
\def\MK{\mathsf{MK}}
\def\MYD{{\mathsf{MYD}}}
\def\AND{\quad\text{and}\quad}
\def\deti{{\det}_{\sqrt{-1}}}

\def\AC{\mathrm{AC}}
\def\wAC{\mathrm{AC}^{\mathrm{weak}}}


\def\KM{{\mathcal{K_{\mathsf{M}}}}}

\newcommand{\bfonenp}{\mathbf{1}^{-,+}}
\newcommand{\bfonepn}{\mathbf{1}^{+,-}}
\newcommand{\bfone}{\mathbf{1}}
\newcommand{\piSigma}{\pi_\Sigma}
\newcommand{\piSigmap}{\pi'_\Sigma}


\newcommand{\sfVprime}{\mathsf{V}^\prime}
\newcommand{\sfVdprime}{\mathsf{V}^{\prime \prime}}
\newcommand{\gminusone}{\mathfrak{g}_{-\frac{1}{m}}}

\newcommand{\eva}{\mathrm{eva}}

% \newcommand\iso{\xrightarrow{
%    \,\smash{\raisebox{-0.65ex}{\ensuremath{\scriptstyle\sim}}}\,}}

\def\Ueven{{U_{\rm{even}}}}
\def\Uodd{{U_{\rm{odd}}}}
\def\ttau{\tilde{\tau}}
\def\Wcp{W}
\def\Kur{{K^{\mathrm{u}}}}

\def\Im{\operatorname{Im}}


\providecommand{\bcN}{{\overline{\cN}}}



\makeatletter

\def\gen#1{\left\langle
    #1
      \right\rangle}
\makeatother

\makeatletter
\def\inn#1#2{\left\langle
      \def\ta{#1}\def\tb{#2}
      \ifx\ta\@empty{\;} \else {\ta}\fi ,
      \ifx\tb\@empty{\;} \else {\tb}\fi
      \right\rangle}
\def\binn#1#2{\left\lAngle
      \def\ta{#1}\def\tb{#2}
      \ifx\ta\@empty{\;} \else {\ta}\fi ,
      \ifx\tb\@empty{\;} \else {\tb}\fi
      \right\rAngle}
\makeatother

\makeatletter
\def\binn#1#2{\overline{\inn{#1}{#2}}}
\makeatother


\def\innwi#1#2{\inn{#1}{#2}_{W_i}}
\def\innw#1#2{\inn{#1}{#2}_{\bfW}}
\def\innv#1#2{\inn{#1}{#2}_{\bfV}}
\def\innbfv#1#2{\inn{#1}{#2}_{\bfV}}
\def\innvi#1#2{\inn{#1}{#2}_{V_i}}
\def\innvp#1#2{\inn{#1}{#2}_{\bfV'}}
\def\innp#1#2{\inn{#1}{#2}'}

% choose one of then
\def\simrightarrow{\iso}
\def\surj{\twoheadrightarrow}
%\def\simrightarrow{\xrightarrow{\sim}}

\newcommand\iso{\xrightarrow{
   \,\smash{\raisebox{-0.65ex}{\ensuremath{\scriptstyle\sim}}}\,}}

\newcommand\riso{\xleftarrow{
   \,\smash{\raisebox{-0.65ex}{\ensuremath{\scriptstyle\sim}}}\,}}









\usepackage{xparse}
\def\usecsname#1{\csname #1\endcsname}
\def\useLetter#1{#1}
\def\usedbletter#1{#1#1}

% \def\useCSf#1{\csname f#1\endcsname}

\ExplSyntaxOn

\def\mydefcirc#1#2#3{\expandafter\def\csname
  circ#3{#1}\endcsname{{}^\circ {#2{#1}}}}
\def\mydefvec#1#2#3{\expandafter\def\csname
  vec#3{#1}\endcsname{\vec{#2{#1}}}}
\def\mydefdot#1#2#3{\expandafter\def\csname
  dot#3{#1}\endcsname{\dot{#2{#1}}}}

\def\mydefacute#1#2#3{\expandafter\def\csname a#3{#1}\endcsname{\acute{#2{#1}}}}
\def\mydefbr#1#2#3{\expandafter\def\csname br#3{#1}\endcsname{\breve{#2{#1}}}}
\def\mydefbar#1#2#3{\expandafter\def\csname bar#3{#1}\endcsname{\bar{#2{#1}}}}
\def\mydefhat#1#2#3{\expandafter\def\csname hat#3{#1}\endcsname{\hat{#2{#1}}}}
\def\mydefwh#1#2#3{\expandafter\def\csname wh#3{#1}\endcsname{\widehat{#2{#1}}}}
\def\mydeft#1#2#3{\expandafter\def\csname t#3{#1}\endcsname{\tilde{#2{#1}}}}
\def\mydefu#1#2#3{\expandafter\def\csname u#3{#1}\endcsname{\underline{#2{#1}}}}
\def\mydefr#1#2#3{\expandafter\def\csname r#3{#1}\endcsname{\mathrm{#2{#1}}}}
\def\mydefb#1#2#3{\expandafter\def\csname b#3{#1}\endcsname{\mathbb{#2{#1}}}}
\def\mydefwt#1#2#3{\expandafter\def\csname wt#3{#1}\endcsname{\widetilde{#2{#1}}}}
%\def\mydeff#1#2#3{\expandafter\def\csname f#3{#1}\endcsname{\mathfrak{#2{#1}}}}
\def\mydefbf#1#2#3{\expandafter\def\csname bf#3{#1}\endcsname{\mathbf{#2{#1}}}}
\def\mydefc#1#2#3{\expandafter\def\csname c#3{#1}\endcsname{\mathcal{#2{#1}}}}
\def\mydefsf#1#2#3{\expandafter\def\csname sf#3{#1}\endcsname{\mathsf{#2{#1}}}}
\def\mydefs#1#2#3{\expandafter\def\csname s#3{#1}\endcsname{\mathscr{#2{#1}}}}
\def\mydefcks#1#2#3{\expandafter\def\csname cks#3{#1}\endcsname{{\check{
        \csname s#2{#1}\endcsname}}}}
\def\mydefckc#1#2#3{\expandafter\def\csname ckc#3{#1}\endcsname{{\check{
      \csname c#2{#1}\endcsname}}}}
\def\mydefck#1#2#3{\expandafter\def\csname ck#3{#1}\endcsname{{\check{#2{#1}}}}}

\cs_new:Npn \mydeff #1#2#3 {\cs_new:cpn {f#3{#1}} {\mathfrak{#2{#1}}}}

\cs_new:Npn \doGreek #1
{
  \clist_map_inline:nn {alpha,beta,gamma,Gamma,delta,Delta,epsilon,varepsilon,zeta,eta,theta,vartheta,Theta,iota,kappa,lambda,Lambda,mu,nu,xi,Xi,pi,Pi,rho,sigma,varsigma,Sigma,tau,upsilon,Upsilon,phi,varphi,Phi,chi,psi,Psi,omega,Omega,tG} {#1{##1}{\usecsname}{\useLetter}}
}

\cs_new:Npn \doSymbols #1
{
  \clist_map_inline:nn {otimes,boxtimes} {#1{##1}{\usecsname}{\useLetter}}
}

\cs_new:Npn \doAtZ #1
{
  \clist_map_inline:nn {A,B,C,D,E,F,G,H,I,J,K,L,M,N,O,P,Q,R,S,T,U,V,W,X,Y,Z} {#1{##1}{\useLetter}{\useLetter}}
}

\cs_new:Npn \doatz #1
{
  \clist_map_inline:nn {a,b,c,d,e,f,g,h,i,j,k,l,m,n,o,p,q,r,s,t,u,v,w,x,y,z} {#1{##1}{\useLetter}{\usedbletter}}
}

\cs_new:Npn \doallAtZ
{
\clist_map_inline:nn {mydefsf,mydeft,mydefu,mydefwh,mydefhat,mydefr,mydefwt,mydeff,mydefb,mydefbf,mydefc,mydefs,mydefck,mydefcks,mydefckc,mydefbar,mydefvec,mydefcirc,mydefdot,mydefbr,mydefacute} {\doAtZ{\csname ##1\endcsname}}
}

\cs_new:Npn \doallatz
{
\clist_map_inline:nn {mydefsf,mydeft,mydefu,mydefwh,mydefhat,mydefr,mydefwt,mydeff,mydefb,mydefbf,mydefc,mydefs,mydefck,mydefbar,mydefvec,mydefdot,mydefbr,mydefacute} {\doatz{\csname ##1\endcsname}}
}

\cs_new:Npn \doallGreek
{
\clist_map_inline:nn {mydefck,mydefwt,mydeft,mydefwh,mydefbar,mydefu,mydefvec,mydefcirc,mydefdot,mydefbr,mydefacute} {\doGreek{\csname ##1\endcsname}}
}

\cs_new:Npn \doallSymbols
{
\clist_map_inline:nn {mydefck,mydefwt,mydeft,mydefwh,mydefbar,mydefu,mydefvec,mydefcirc,mydefdot} {\doSymbols{\csname ##1\endcsname}}
}



\cs_new:Npn \doGroups #1
{
  \clist_map_inline:nn {GL,Sp,rO,rU,fgl,fsp,foo,fuu,fkk,fuu,ufkk,uK} {#1{##1}{\usecsname}{\useLetter}}
}

\cs_new:Npn \doallGroups
{
\clist_map_inline:nn {mydeft,mydefu,mydefwh,mydefhat,mydefwt,mydefck,mydefbar} {\doGroups{\csname ##1\endcsname}}
}


\cs_new:Npn \decsyms #1
{
\clist_map_inline:nn {#1} {\expandafter\DeclareMathOperator\csname ##1\endcsname{##1}}
}

\decsyms{Mp,id,SL,Sp,SU,SO,GO,GSO,GU,GSp,PGL,Pic,Lie,Mat,Ker,Hom,Ext,Ind,reg,res,inv,Isom,Det,Tr,Norm,Sym,Span,Stab,Spec,PGSp,PSL,tr,Ad,Br,Ch,Cent,End,Aut,Dvi,Frob,Gal,GL,Gr,DO,ur,vol,ab,Nil,Supp,rank,Sign}

\def\abs#1{\left|{#1}\right|}
\def\norm#1{{\left\|{#1}\right\|}}


% \NewDocumentCommand\inn{m m}{
% \left\langle
% \IfValueTF{#1}{#1}{000}
% ,
% \IfValueTF{#2}{#2}{000}
% \right\rangle
% }
\NewDocumentCommand\cent{o m }{
  \IfValueTF{#1}{
    \mathop{Z}_{#1}{(#2)}}
  {\mathop{Z}{(#2)}}
}


\def\fsl{\mathfrak{sl}}
\def\fsp{\mathfrak{sp}}


%\def\cent#1#2{{\mathrm{Z}_{#1}({#2})}}


\doallAtZ
\doallatz
\doallGreek
\doallGroups
\doallSymbols
\ExplSyntaxOff


% \usepackage{geometry,amsthm,graphics,tabularx,amssymb,shapepar}
% \usepackage{amscd}
% \usepackage{mathrsfs}


\usepackage{diagbox}
% Update the information and uncomment if AMS is not the copyright
% holder.
%\copyrightinfo{2006}{American Mathematical Society}
%\usepackage{nicematrix}
\usepackage{arydshln}
\usepackage[mode=buildnew]{standalone}% requires -shell-escape

\usepackage{tikz,etoolbox}
\usetikzlibrary{matrix,arrows,positioning,backgrounds}
\usetikzlibrary{decorations.pathmorphing,decorations.pathreplacing}
\usetikzlibrary{cd}
% \usetikzlibrary{external}
%   \tikzexternalize
% \usetikzlibrary{cd}

%  \AtBeginEnvironment{tikzcd}{\tikzexternaldisable}
%  \AtEndEnvironment{tikzcd}{\tikzexternalenable}

%  \usetikzlibrary{matrix,arrows,positioning,backgrounds}
%  \usetikzlibrary{decorations.pathmorphing,decorations.pathreplacing}

% % externalization not wog properly
% % \usetikzlibrary{external}
% \tikzexternalize[prefix=figures/]
% % % activate the following such that you can check the macro expansion in
% % % *-figure0.md5 manually
% %\tikzset{external/up to date check=diff}
% \usepackage{environ}

% \def\temp{&} \catcode`&=\active \let&=\temp

% \newcommand{\mytikzcdcontext}[2]{
%   \begin{tikzpicture}[baseline=(maintikzcdnode.base)]
%     \node (maintikzcdnode) [inner sep=0, outer sep=0] {\begin{tikzcd}[#2]
%         #1
%     \end{tikzcd}};
%   \end{tikzpicture}}

% \NewEnviron{mytikzcd}[1][]{%
% % In the following, we need \BODY to expanded before \mytikzcdcontext
% % such that the md5 function gets the tikzcd content with \BODY expanded.
% % Howerver, expand it only once, because the \tikz-macros aren't
% % defined at this point yet. The same thing holds for the arguments to
% % the tikzcd-environment.
% \def\myargs{#1}%
% \edef\mydiagram{\noexpand\mytikzcdcontext{\expandonce\BODY}{\expandonce\myargs}}%
% \mydiagram%
% }

\usepackage{upgreek}

\usepackage{listings}
\lstset{
    basicstyle=\ttfamily\tiny,
    keywordstyle=\color{black},
    commentstyle=\color{white}, % white comments
    stringstyle=\ttfamily, % typewriter type for strings
    showstringspaces=false,
    breaklines=true,
    emph={Output},emphstyle=\color{blue},
}

\newcommand{\BA}{{\mathbb{A}}}
%\newcommand{\BB}{{\mathbb {B}}}
\newcommand{\BC}{{\mathbb {C}}}
\newcommand{\BD}{{\mathbb {D}}}
\newcommand{\BE}{{\mathbb {E}}}
\newcommand{\BF}{{\mathbb {F}}}
\newcommand{\BG}{{\mathbb {G}}}
\newcommand{\BH}{{\mathbb {H}}}
\newcommand{\BI}{{\mathbb {I}}}
\newcommand{\BJ}{{\mathbb {J}}}
\newcommand{\BK}{{\mathbb {U}}}
\newcommand{\BL}{{\mathbb {L}}}
\newcommand{\BM}{{\mathbb {M}}}
\newcommand{\BN}{{\mathbb {N}}}
\newcommand{\BO}{{\mathbb {O}}}
\newcommand{\BP}{{\mathbb {P}}}
\newcommand{\BQ}{{\mathbb {Q}}}
\newcommand{\BR}{{\mathbb {R}}}
\newcommand{\BS}{{\mathbb {S}}}
\newcommand{\BT}{{\mathbb {T}}}
\newcommand{\BU}{{\mathbb {U}}}
\newcommand{\BV}{{\mathbb {V}}}
\newcommand{\BW}{{\mathbb {W}}}
\newcommand{\BX}{{\mathbb {X}}}
\newcommand{\BY}{{\mathbb {Y}}}
\newcommand{\BZ}{{\mathbb {Z}}}
\newcommand{\Bk}{{\mathbf {k}}}

\newcommand{\CA}{{\mathcal {A}}}
\newcommand{\CB}{{\mathcal {B}}}
\newcommand{\CC}{{\mathcal {C}}}

\newcommand{\CE}{{\mathcal {E}}}
\newcommand{\CF}{{\mathcal {F}}}
\newcommand{\CG}{{\mathcal {G}}}
\newcommand{\CH}{{\mathcal {H}}}
\newcommand{\CI}{{\mathcal {I}}}
\newcommand{\CJ}{{\mathcal {J}}}
\newcommand{\CK}{{\mathcal {K}}}
\newcommand{\CL}{{\mathcal {L}}}
\newcommand{\CM}{{\mathcal {M}}}
\newcommand{\CN}{{\mathcal {N}}}
\newcommand{\CO}{{\mathcal {O}}}
\newcommand{\CP}{{\mathcal {P}}}
\newcommand{\CQ}{{\mathcal {Q}}}
\newcommand{\CR}{{\mathcal {R}}}
\newcommand{\CS}{{\mathcal {S}}}
\newcommand{\CT}{{\mathcal {T}}}
\newcommand{\CU}{{\mathcal {U}}}
\newcommand{\CV}{{\mathcal {V}}}
\newcommand{\CW}{{\mathcal {W}}}
\newcommand{\CX}{{\mathcal {X}}}
\newcommand{\CY}{{\mathcal {Y}}}
\newcommand{\CZ}{{\mathcal {Z}}}


\newcommand{\RA}{{\mathrm {A}}}
\newcommand{\RB}{{\mathrm {B}}}
\newcommand{\RC}{{\mathrm {C}}}
\newcommand{\RD}{{\mathrm {D}}}
\newcommand{\RE}{{\mathrm {E}}}
\newcommand{\RF}{{\mathrm {F}}}
\newcommand{\RG}{{\mathrm {G}}}
\newcommand{\RH}{{\mathrm {H}}}
\newcommand{\RI}{{\mathrm {I}}}
\newcommand{\RJ}{{\mathrm {J}}}
\newcommand{\RK}{{\mathrm {K}}}
\newcommand{\RL}{{\mathrm {L}}}
\newcommand{\RM}{{\mathrm {M}}}
\newcommand{\RN}{{\mathrm {N}}}
\newcommand{\RO}{{\mathrm {O}}}
\newcommand{\RP}{{\mathrm {P}}}
\newcommand{\RQ}{{\mathrm {Q}}}
%\newcommand{\RR}{{\mathrm {R}}}
\newcommand{\RS}{{\mathrm {S}}}
\newcommand{\RT}{{\mathrm {T}}}
\newcommand{\RU}{{\mathrm {U}}}
\newcommand{\RV}{{\mathrm {V}}}
\newcommand{\RW}{{\mathrm {W}}}
\newcommand{\RX}{{\mathrm {X}}}
\newcommand{\RY}{{\mathrm {Y}}}
\newcommand{\RZ}{{\mathrm {Z}}}

\DeclareMathOperator{\absNorm}{\mathfrak{N}}
\DeclareMathOperator{\Ann}{Ann}
\DeclareMathOperator{\LAnn}{L-Ann}
\DeclareMathOperator{\RAnn}{R-Ann}
\DeclareMathOperator{\ind}{ind}
%\DeclareMathOperator{\Ind}{Ind}



\def\ckbfG{\check{\bfG}}

\newcommand{\cod}{{\mathrm{cod}}}
\newcommand{\cont}{{\mathrm{cont}}}
\newcommand{\cl}{{\mathrm{cl}}}
\newcommand{\cusp}{{\mathrm{cusp}}}

\newcommand{\disc}{{\mathrm{disc}}}



\newcommand{\Gm}{{\mathbb{G}_m}}



\newcommand{\I}{{\mathrm{I}}}

\newcommand{\Jac}{{\mathrm{Jac}}}
\newcommand{\PM}{{\mathrm{PM}}}


\newcommand{\new}{{\mathrm{new}}}
\newcommand{\NS}{{\mathrm{NS}}}
\newcommand{\N}{{\mathrm{N}}}

\newcommand{\ord}{{\mathrm{ord}}}

%\newcommand{\rank}{{\mathrm{rank}}}

\newcommand{\rk}{{\mathrm{k}}}
\newcommand{\rr}{{\mathrm{r}}}
\newcommand{\rh}{{\mathrm{h}}}

\newcommand{\Sel}{{\mathrm{Sel}}}
\newcommand{\Sim}{{\mathrm{Sim}}}

\newcommand{\wt}{\widetilde}
\newcommand{\wh}{\widehat}
\newcommand{\pp}{\frac{\partial\bar\partial}{\pi i}}
\newcommand{\pair}[1]{\langle {#1} \rangle}
\newcommand{\wpair}[1]{\left\{{#1}\right\}}
\newcommand{\intn}[1]{\left( {#1} \right)}
\newcommand{\sfrac}[2]{\left( \frac {#1}{#2}\right)}
\newcommand{\ds}{\displaystyle}
\newcommand{\ov}{\overline}
\newcommand{\incl}{\hookrightarrow}
\newcommand{\lra}{\longrightarrow}
\newcommand{\imp}{\Longrightarrow}
%\newcommand{\lto}{\longmapsto}
\newcommand{\bs}{\backslash}

\newcommand{\cover}[1]{\widetilde{#1}}

\renewcommand{\vsp}{{\vspace{0.2in}}}

\newcommand{\Norma}{\operatorname{N}}
\newcommand{\Ima}{\operatorname{Im}}
\newcommand{\con}{\textit{C}}
\newcommand{\gr}{\operatorname{gr}}
\newcommand{\ad}{\operatorname{ad}}
\newcommand{\der}{\operatorname{der}}
\newcommand{\dif}{\operatorname{d}\!}
\newcommand{\pro}{\operatorname{pro}}
\newcommand{\Ev}{\operatorname{Ev}}
% \renewcommand{\span}{\operatorname{span}} \span is an innernal command.
%\newcommand{\degree}{\operatorname{deg}}
\newcommand{\Invf}{\operatorname{Invf}}
\newcommand{\Inv}{\operatorname{Inv}}
\newcommand{\slt}{\operatorname{SL}_2(\mathbb{R})}
%\newcommand{\temp}{\operatorname{temp}}
%\newcommand{\otop}{\operatorname{top}}
%\renewcommand{\small}{\operatorname{small}}
\newcommand{\HC}{\operatorname{HC}}
\newcommand{\lef}{\operatorname{left}}
\newcommand{\righ}{\operatorname{right}}
\newcommand{\Diff}{\operatorname{DO}}
\newcommand{\diag}{\operatorname{diag}}
\newcommand{\sh}{\varsigma}
\newcommand{\sch}{\operatorname{sch}}
%\newcommand{\oleft}{\operatorname{left}}
%\newcommand{\oright}{\operatorname{right}}
\newcommand{\open}{\operatorname{open}}
\newcommand{\sgn}{\operatorname{sgn}}
\newcommand{\triv}{\operatorname{triv}}
\newcommand{\Sh}{\operatorname{Sh}}
\newcommand{\oN}{\operatorname{N}}

\newcommand{\oc}{\operatorname{c}}
\newcommand{\od}{\operatorname{d}}
\newcommand{\os}{\operatorname{s}}
\newcommand{\ol}{\operatorname{l}}
\newcommand{\oL}{\operatorname{L}}
\newcommand{\oJ}{\operatorname{J}}
\newcommand{\oH}{\operatorname{H}}
\newcommand{\oO}{\operatorname{O}}
\newcommand{\oS}{\operatorname{S}}
\newcommand{\oR}{\operatorname{R}}
\newcommand{\oT}{\operatorname{T}}
%\newcommand{\rU}{\operatorname{U}}
\newcommand{\oZ}{\operatorname{Z}}
\newcommand{\oD}{\textit{D}}
\newcommand{\oW}{\textit{W}}
\newcommand{\oE}{\operatorname{E}}
\newcommand{\oP}{\operatorname{P}}
\newcommand{\PD}{\operatorname{PD}}
\newcommand{\oU}{\operatorname{U}}

\newcommand{\gC}{{\mathfrak g}_{\C}}
%\renewcommand{\sl}{\mathfrak s \mathfrak l}
\newcommand{\gl}{\mathfrak g \mathfrak l}


\newcommand{\re}{\mathrm e}

\renewcommand{\rk}{\mathrm k}

\newcommand{\g}{\mathfrak g}
\newcommand{\h}{\mathfrak h}
\newcommand{\p}{\mathfrak p}
\newcommand{\Z}{\mathbb{Z}}
\DeclareDocumentCommand{\C}{}{\mathbb{C}}
\newcommand{\R}{\mathbb R}
\newcommand{\Q}{\mathbb Q}
\renewcommand{\H}{\mathbb{H}}
%\newcommand{\N}{\mathbb{N}}
\newcommand{\K}{\mathbb{K}}
%\renewcommand{\S}{\mathbf S}
\newcommand{\M}{\mathbf{M}}
\newcommand{\A}{\mathbb{A}}
\newcommand{\B}{\mathbf{B}}
%\renewcommand{\G}{\mathbf{G}}
\newcommand{\V}{\mathbf{V}}
\newcommand{\W}{\mathbf{W}}
\newcommand{\F}{\mathbf{F}}
\newcommand{\E}{\mathbf{E}}
%\newcommand{\J}{\mathbf{J}}
\renewcommand{\H}{\mathbf{H}}
\newcommand{\X}{\mathbf{X}}
\newcommand{\Y}{\mathbf{Y}}
%\newcommand{\RR}{\mathcal R}
\newcommand{\FF}{\mathcal F}
%\newcommand{\BB}{\mathcal B}
\newcommand{\HH}{\mathcal H}
%\newcommand{\UU}{\mathcal U}
%\newcommand{\MM}{\mathcal M}
%\newcommand{\CC}{\mathcal C}
%\newcommand{\DD}{\mathcal D}
%\def\eDD{\mathrm{d}^{e}}
%\def\eDD{\bigtriangledown}
\def\eDD{\overline{\nabla}}
\def\eDDo{\overline{\nabla}_1}
%\def\eDD{\mathrm{d}}
\def\DD{\nabla}
\def\DDc{\boldsymbol{\nabla}}
\def\gDD{\nabla^{\mathrm{gen}}}
\def\gDDc{\boldsymbol{\nabla}^{\mathrm{gen}}}
%\newcommand{\OO}{\mathcal O}
%\newcommand{\ZZ}{\mathcal Z}
\newcommand{\ve}{{\vee}}
\newcommand{\aut}{\mathcal A}
\newcommand{\ii}{\mathbf{i}}
\newcommand{\jj}{\mathbf{j}}
\newcommand{\kk}{\mathbf{k}}

\newcommand{\la}{\langle}
\newcommand{\ra}{\rangle}
\newcommand{\bp}{\bigskip}
\newcommand{\be}{\begin {equation}}
\newcommand{\ee}{\end {equation}}

\newcommand{\LRleq}{\stackrel{LR}{\leq}}

\numberwithin{equation}{section}


\def\flushl#1{\ifmmode\makebox[0pt][l]{${#1}$}\else\makebox[0pt][l]{#1}\fi}
\def\flushr#1{\ifmmode\makebox[0pt][r]{${#1}$}\else\makebox[0pt][r]{#1}\fi}
\def\flushmr#1{\makebox[0pt][r]{${#1}$}}


%\theoremstyle{Theorem}
% \newtheorem*{thmM}{Main Theorem}
% \crefformat{thmM}{main theorem}
% \Crefformat{thmM}{Main Theorem}
\newtheorem*{thm*}{Theorem}
\newtheorem{thm}{Theorem}[section]
\newtheorem{thml}[thm]{Theorem}
\newtheorem{lem}[thm]{Lemma}
\newtheorem{obs}[thm]{Observation}
\newtheorem{lemt}[thm]{Lemma}
\newtheorem*{lem*}{Lemma}
\newtheorem{whyp}[thm]{Working Hypothesis}
\newtheorem{prop}[thm]{Proposition}
\newtheorem{prpt}[thm]{Proposition}
\newtheorem{prpl}[thm]{Proposition}
\newtheorem{cor}[thm]{Corollary}
%\newtheorem*{prop*}{Proposition}
\newtheorem{claim}[thm]{Claim}
\newtheorem*{claim*}{Claim}
%\theoremstyle{definition}
\newtheorem{defn}[thm]{Definition}
\newtheorem{dfnl}[thm]{Definition}
\newtheorem*{IndH}{Induction Hypothesis}

\newtheorem*{eg*}{Example}
\newtheorem{eg}[thm]{Example}

\theoremstyle{remark}
\newtheorem*{remark}{Remark}
\newtheorem*{remarks}{Remarks}
\newtheorem*{Example}{Example}

\def\cpc{\sigma}
\def\ccJ{\epsilon\dotepsilon}
\def\ccL{c_L}

\def\wtbfK{\widetilde{\bfK}}
%\def\abfV{\acute{\bfV}}
\def\AbfV{\acute{\bfV}}
%\def\afgg{\acute{\fgg}}
%\def\abfG{\acute{\bfG}}
\def\abfV{\bfV'}
\def\afgg{\fgg'}
\def\abfG{\bfG'}

\def\half{{\tfrac{1}{2}}}
\def\ihalf{{\tfrac{\mathbf i}{2}}}
\def\slt{\fsl_2(\bC)}
\def\sltr{\fsl_2(\bR)}

% \def\Jslt{{J_{\fslt}}}
% \def\Lslt{{L_{\fslt}}}
\def\slee{{
\begin{pmatrix}
 0 & 1\\
 0 & 0
\end{pmatrix}
}}
\def\slff{{
\begin{pmatrix}
 0 & 0\\
 1 & 0
\end{pmatrix}
}}\def\slhh{{
\begin{pmatrix}
 1 & 0\\
 0 & -1
\end{pmatrix}
}}
\def\sleei{{
\begin{pmatrix}
 0 & i\\
 0 & 0
\end{pmatrix}
}}
\def\slxx{{\begin{pmatrix}
-\ihalf & \half\\
\phantom{-}\half & \ihalf
\end{pmatrix}}}
% \def\slxx{{\begin{pmatrix}
% -\sqrt{-1}/2 & 1/2\\
% 1/2 & \sqrt{-1}/2
% \end{pmatrix}}}
\def\slyy{{\begin{pmatrix}
\ihalf & \half\\
\half & -\ihalf
\end{pmatrix}}}
\def\slxxi{{\begin{pmatrix}
+\half & -\ihalf\\
-\ihalf & -\half
\end{pmatrix}}}
\def\slH{{\begin{pmatrix}
   0   & -\mathbf i\\
\mathbf i & 0
\end{pmatrix}}
}

\ExplSyntaxOn
\clist_map_inline:nn {J,L,C,X,Y,H,c,e,f,h,}{
  \expandafter\def\csname #1slt\endcsname{{\mathring{#1}}}}
\ExplSyntaxOff

\def\eeslt{\mathring{\bfee}}

\def\Mop{\fT}

\def\fggJ{\fgg_J}
\def\fggJp{\fgg'_{J'}}

\def\NilGC{\Nil_{\bfG}(\fgg)}
\def\NilGCp{\Nil_{\bfG'}(\fgg')}
\def\Nilgp{\Nil_{\fgg'_{J'}}}
\def\Nilg{\Nil_{\fgg_{J}}}
%\def\NilP'{\Nil_{\fpp'}}
\def\peNil{\Nil^{\mathrm{pe}}}
\def\dpeNil{\Nil^{\mathrm{dpe}}}
\def\nNil{\Nil^{\mathrm n}}
\def\eNil{\Nil^{\mathrm e}}


\NewDocumentCommand{\NilP}{t'}{
\IfBooleanTF{#1}{\Nil_{\fpp'}}{\Nil_\fpp}
}

\def\KS{\cK_{\sfS}}
\def\MM{\bfM}
\def\MMP{M}

\NewDocumentCommand{\KTW}{o g}{
  \IfValueTF{#2}{
    \left.\varsigma_{\IfValueT{#1}{#1}}\right|_{#2}}{
    \varsigma_{\IfValueT{#1}{#1}}}
}
\def\IST{\rho}
\def\tIST{\trho}

\NewDocumentCommand{\CHI}{o g}{
  \IfValueTF{#1}{
    {\chi}_{\left[#1\right]}}{
    \IfValueTF{#2}{
      {\chi}_{\left(#2\right)}}{
      {\chi}}
  }
}
\NewDocumentCommand{\PR}{g}{
  \IfValueTF{#1}{
    \mathop{\pr}_{\left(#1\right)}}{
    \mathop{\pr}}
}
\NewDocumentCommand{\XX}{g}{
  \IfValueTF{#1}{
    {\cX}_{\left(#1\right)}}{
    {\cX}}
}
\NewDocumentCommand{\PP}{g}{
  \IfValueTF{#1}{
    {\fpp}_{\left(#1\right)}}{
    {\fpp}}
}
\NewDocumentCommand{\LL}{g}{
  \IfValueTF{#1}{
    {\bfL}_{\left(#1\right)}}{
    {\bfL}}
}
\NewDocumentCommand{\ZZ}{g}{
  \IfValueTF{#1}{
    {\cZ}_{\left(#1\right)}}{
    {\cZ}}
}

\NewDocumentCommand{\WW}{g}{
  \IfValueTF{#1}{
    {\bfW}_{\left(#1\right)}}{
    {\bfW}}
}




\def\gpi{\wp}
% \NewDocumentCommand\KK{g}{
% \IfValueTF{#1}{K_{(#1)}}{K}}
% \NewDocumentCommand\OO{g}{
% \IfValueTF{#1}{\cO_{(#1)}}{K}}
\NewDocumentCommand\XXo{d()}{
\IfValueTF{#1}{\cX^\circ_{(#1)}}{\cX^\circ}}
\def\bfWo{\bfW^\circ}
\def\bfWoo{\bfW^{\circ \circ}}
\def\bfWg{\bfW^{\mathrm{gen}}}
\def\Xg{\cX^{\mathrm{gen}}}
\def\Xo{\cX^\circ}
\def\Xoo{\cX^{\circ \circ}}
\def\fppo{\fpp^\circ}
\def\fggo{\fgg^\circ}
\NewDocumentCommand\ZZo{g}{
\IfValueTF{#1}{\cZ^\circ_{(#1)}}{\cZ^\circ}}

% \ExplSyntaxOn
% \NewDocumentCommand{\bcO}{t' E{^_}{{}{}}}{
%   \overline{\cO\sb{\use_ii:nn#2}\IfBooleanTF{#1}{^{'\use_i:nn#2}}{^{\use_i:nn#2}}
%   }
% }
% \ExplSyntaxOff

\NewDocumentCommand{\bcO}{t'}{
  \overline{\cO\IfBooleanT{#1}{'}}}

\NewDocumentCommand{\oliftc}{g}{
\IfValueTF{#1}{\boldsymbol{\vartheta} (#1)}{\boldsymbol{\vartheta}}
}
\NewDocumentCommand{\oliftr}{g}{
\IfValueTF{#1}{\vartheta_\bR(#1)}{\vartheta_\bR}
}
\NewDocumentCommand{\olift}{g}{
\IfValueTF{#1}{\vartheta(#1)}{\vartheta}
}
% \NewDocumentCommand{\dliftv}{g}{
% \IfValueTF{#1}{\ckvartheta(#1)}{\ckvartheta}
% }
\def\dliftv{\vartheta}
\NewDocumentCommand{\tlift}{g}{
\IfValueTF{#1}{\wtvartheta(#1)}{\wtvartheta}
}

\def\slift{\cL}

\def\BB{\bB}


\def\thetaO#1{\vartheta\left(#1\right)}

\def\bbThetav{\check{\mathbbold{\Theta}}}
\def\Thetav{\check{\Theta}}
\def\thetav{\check{\theta}}

\DeclareDocumentCommand{\NN}{g}{
\IfValueTF{#1}{\fN(#1)}{\fN}
}
\DeclareDocumentCommand{\RR}{m m}{
\fR({#1},{#2})
}

%\DeclareMathOperator*{\sign}{Sign}

% \NewDocumentCommand{\lsign}{m}{
% {}^l\mathrm{Sign}(#1)
% }

% \NewDocumentCommand{\bsign}{m}{
% {}^b\mathrm{Sign}(#1)
% }
%
\def\tsign{{}^t\mathrm{Sign}}
\def\lsign{{}^l\mathrm{Sign}}
\def\bsign{{}^b\mathrm{Sign}}
\def\ssign{\mathrm{Sign}}
\NewDocumentCommand{\sign}{m}{
  \mathrm{Sign}(#1)
}

\NewDocumentCommand\lnn{t+ t- g}{
  \IfBooleanTF{#1}{{}^l n^+\IfValueT{#3}{(#3)}}{
    \IfBooleanTF{#2}{{}^l n^-\IfValueT{#3}{(#3)}}{}
  }
}


% Fancy bcO, support feature \bcO'^a_b = \overline{\cO'^a_b}
\makeatletter
\def\bcO{\def\O@@{\cO}\@ifnextchar'\@Op\@Onp}
\def\@Opnext{\@ifnextchar^\@Opsp\@Opnsp}
\def\@Op{\afterassignment\@Opnext\let\scratch=}
\def\@Opnsp{\def\O@@{\cO'}\@Otsb}
\def\@Onp{\@ifnextchar^\@Onpsp\@Otsb}
\def\@Opsp^#1{\def\O@@{\cO'^{#1}}\@Otsb}
\def\@Onpsp^#1{\def\O@@{\cO^{#1}}\@Otsb}
\def\@Otsb{\@ifnextchar_\@Osb{\@Ofinalnsb}}
\def\@Osb_#1{\overline{\O@@_{#1}}}
\def\@Ofinalnsb{\overline{\O@@}}

% Fancy \command: \command`#1 will translate to {}^{#1}\bfV, i.e. superscript on the
% lift conner.

\def\defpcmd#1{
  \def\nn@tmp{#1}
  \def\nn@np@tmp{@np@#1}
  \expandafter\let\csname\nn@np@tmp\expandafter\endcsname \csname\nn@tmp\endcsname
  \expandafter\def\csname @pp@#1\endcsname`##1{{}^{##1}{\csname @np@#1\endcsname}}
  \expandafter\def\csname #1\endcsname{\,\@ifnextchar`{\csname
      @pp@#1\endcsname}{\csname @np@#1\endcsname}}
}

% \def\defppcmd#1{
% \expandafter\NewDocumentCommand{\csname #1\endcsname}{##1 }{}
% }



\defpcmd{bfV}
\def\KK{\bfK}\defpcmd{KK}
\defpcmd{bfG}
\def\A{\!A}\defpcmd{A}
\def\K{\!K}\defpcmd{K}
\def\G{G}\defpcmd{G}
\def\J{\!J}\defpcmd{J}
\def\L{\!L}\defpcmd{L}
\def\eps{\epsilon}\defpcmd{eps}
\def\pp{p}\defpcmd{pp}
\defpcmd{wtK}
\makeatother


%\def\KC#1{K_{#1,\bC}} %the complexcification of K.

\NewDocumentCommand\KC{s o e{_}}{
  \IfBooleanTF{#1}{
    \IfNoValueTF{#3}{
      \IfNoValueTF{#2}{K_{\bC}^\circ}{K_{#2,\bC}^\circ}}
    {
      \IfNoValueTF{#2}{(K_{\bC})_{#3}^\circ}{(K_{#2,\bC})_{#3}^\circ}}
  }{
    \IfNoValueTF{#3}{
      \IfNoValueTF{#2}{K_{\bC}}{K_{#2,\bC}}}
    {
      \IfNoValueTF{#2}{(K_{\bC})_{#3}}{(K_{#2,\bC})_{#3}}}
  }
}


\def\fggR{\fgg_\bR}
\def\rmtop{{\mathrm{top}}}
\def\dimo{\dim^\circ}

\NewDocumentCommand\LW{g}{
\IfValueTF{#1}{L_{W_{#1}}}{L_{W}}}
%\def\LW#1{L_{W_{#1}}}
\def\JW#1{J_{W_{#1}}}

\def\floor#1{{\lfloor #1 \rfloor}}

\def\KSP{K}
\def\UU{\rU}
\def\UUC{\rU_\bC}
\def\tUUC{\widetilde{\rU}_\bC}
\def\OmegabfW{\Omega_{\bfW}}


\def\BB{\bB}


\def\thetaO#1{\vartheta\left(#1\right)}

\def\Thetav{\check{\Theta}}
\def\thetav{\check{\theta}}

\def\Thetab{\bar{\Theta}}

\def\cKaod{\cK^{\mathrm{aod}}}

%G_V's or G
%%%%%%%%%%%%%%%%%%%%%%%%%%%
% \def\GVr{G_{\bfV}}
% \def\tGVr{\wtG_{\bfV}}
% \def\GVpr{G_{\bfV'}}
% \def\tGVpr{\wtG_{\bfV'}}
% \def\GVpr{G_{\abfV}}
% \def\tGVar{\wtG_{\abfV}}
% \def\GV{\bfG_{\bfV}}
% \def\GVp{\bfG_{\bfV'}}
% \def\KVr{K_{\bfV}}
% \def\tKVr{\wtK_{\bfV}}
% \def\KV{\bfK_{\bfV}}
% \def\KaV{\bfK_{\acute{V}}}

% \def\KV{\bfK}
% \def\KaV{\acute{\bfK}}
% \def\acO{\acute{\cO}}
% \def\asO{\acute{\sO}}
%%%%%%%%%%%%%%%%%%%%%%%%%%%
%%%%%%%%%%%%%%%%%%%%%%%%%%%


\def\mstar{{\star}}

\def\GVr{G}
\def\tGVr{\wtG}
\def\GVpr{G'}
\def\tGVpr{\widetilde{G'}}
\def\GVar{G'}
\def\tGVar{\wtG'}
\def\GV{\bfG}
\def\GVp{\bfG'}
\def\KVr{K_{\bfV}}
\def\tKVr{\wtK_{\bfV}}
\def\KV{\bfK_{\bfV}}
\def\KaV{\bfK_{\acute{V}}}

\def\KV{\bfK}
\def\KaV{\acute{\bfK}}
\def\acO{{\cO'}}
\def\asO{{\sO'}}

\DeclareMathOperator{\sspan}{span}

%%%%%%%%%%%%%%%%%%%%%%%%%%%%

\def\sp{{\mathrm{sp}}}

\def\bfLz{\bfL_0}
\def\sOpe{\sO^\perp}
\def\sOpeR{\sO^\perp_\bR}
\def\sOR{\sO_\bR}

\def\ZX{\cZ_{X}}
\def\gdliftv{\vartheta}
\def\gdlift{\vartheta^{\mathrm{gen}}}
\def\bcOp{\overline{\cO'}}
\def\bsO{\overline{\sO}}
\def\bsOp{\overline{\sO'}}
\def\bfVpe{\bfV^\perp}
\def\bfEz{\bfE_0}
\def\bfVn{\bfV^-}
\def\bfEzp{\bfE'_0}

\def\totimes{\widehat{\otimes}}
\def\dotbfV{\dot{\bfV}}

\def\aod{\mathrm{aod}}
\def\unip{\mathrm{unip}}


\def\ssP{{\ddot\cP}}
\def\ssD{\ddot{\bfD}}
\def\ssdd{\ddot{\bfdd}}
\def\phik{\phi_{\fkk}}
\def\phikp{\phi_{\fkk'}}
%\def\bbfK{\breve{\bfK}}
\def\bbfK{\wtbfK}
\def\brrho{\breve{\rho}}

\def\whAX{\widehat{A_X}}
\def\mktvvp{\varsigma_{{\bf V},{\bf V}'}}

\def\Piunip{\Pi^{\mathrm{unip}}}
\def\cf{\emph{cf.} }
\def\Groth{\mathrm{Groth}}
\def\Irr{\mathrm{Irr}}

\def\edrc{\mathrm{DRC}^{\mathrm e}}
\def\drc{\mathrm{DRC}}
\def\drcs{\mathrm{DRC}^{s}}
\def\drcns{\mathrm{DRC}^{ns}}
\def\LS{\mathrm{LS}}
\def\LLS{\mathrm{{}^{\ell} LS}}
\def\LSaod{\mathrm{LS^{aod}}}
\def\Unip{\mathrm{Unip}}
\def\lUnip{\mathrm{{}^{\ell}Unip}}
\def\tbfxx{\tilde{\bfxx}}
\def\PBPe{\mathrm{PBP}^{\mathrm{ext}}}
\def\PBPes{\mathrm{PBP}^{\mathrm{ext}}_{\star}}
\def\PBPesp{\mathrm{PBP}^{\mathrm{ext}}_{\star'}}
\def\pbp{\mathrm{PBP}}
\def\pbpst{\mathrm{PBP}_{\star}}
\def\pbpssp{\pbp_{\star}^{\mathrm{ps}}}
\def\pbpsns{\pbp_{\star}^{\mathrm{ns}}}
\def\pbpsp{\pbp^{\mathrm{ps}}}
\def\pbpns{\pbp^{\mathrm{ns}}}
\def\DDn{\DD_{\mathrm{naive}}}
\newcommand{\noticed}{noticed }
\newcommand{\ess}{essential }

\def\dsrcd{\set{\bullet,s,r,c,d}}
\def\taupna{{\tau^{\prime}_{\mathrm{naive}}}}
\def\tauna{{\tau_{\mathrm{naive}}}}

% Ytableau tweak
\makeatletter
\pgfkeys{/ytableau/options,
  noframe/.default = false,
  noframe/.is choice,
  noframe/true/.code = {%
    \global\let\vrule@YT=\vrule@none@YT
    \global\let\hrule@YT=\hrule@none@YT
  },
  noframe/false/.code = {%
    \global\let\vrule@YT=\vrule@normal@YT
    \global\let\hrule@YT=\hrule@normal@YT
  },
  noframe/on/.style = {noframe/true},
  noframe/off/.style = {noframe/false},
}

\def\hrule@enon@YT{%
  \hrule width  \dimexpr \boxdim@YT + \fboxrule *2 \relax
  height 0pt
}
\def\vrule@enon@YT{%
  \vrule height \dimexpr  \boxdim@YT + \fboxrule\relax
     width \fboxrule
}

\def\enon{\omit\enon@YT}
\newcommand{\enon@YT}[2][clear]{%
  \def\thisboxcolor@YT{#1}%
  \let\hrule@YT=\hrule@enon@YT
  \let\vrule@YT=\vrule@enon@YT
  \startbox@@YT#2\endbox@YT
  \nullfont
}

\makeatother
%\ytableausetup{noframe=on,smalltableaux}
\ytableausetup{noframe=off,boxsize=1.3em}
\let\ytb=\ytableaushort

\newcommand{\tytb}[1]{{\tiny\ytb{#1}}}

\makeatletter
\newcommand{\dotminus}{\mathbin{\text{\@dotminus}}}

\newcommand{\@dotminus}{%
  \ooalign{\hidewidth\raise1ex\hbox{.}\hidewidth\cr$\m@th-$\cr}%
}
\makeatother


\def\ckcOp{\ckcO^{\prime}}
\def\ckcOpp{\ckcO^{\prime\prime}}

\def\cOp{\cO^{\prime}}
\def\cOpp{\cO^{\prime\prime}}
\def\cLpp{\cL^{\prime\prime}}
\def\cLppp{\cL^{\prime\prime\prime}}
\def\pUpsilon{\Upsilon^+}
\def\nUpsilon{\Upsilon^-}
\def\pcL{\cL^+}
\def\ncL{\cL^-}
\def\pcP{\cP^+}
\def\ncP{\cP^-}
% \def\pcE{\cE^+}
% \def\ncE{\cE^-}
\def\pcC{\cC^+}
\def\ncC{\cC^-}
\def\pcLp{\cL^{\prime+}}
\def\ncLp{\cL^{\prime-}}
\def\pcLpp{\cL^{\prime\prime+}}
\def\ncLpp{\cL^{\prime\prime-}}
\def\pcB{\cB^+}
\def\ncB{\cB^-}
\def\uptaup{\uptau^{\prime}}
\def\uptaupp{\uptau^{\prime\prime}}
\def\uptauppp{\uptau^{\prime\prime\prime}}
\def\bdelta{{\bar{\delta}}}
\def\tcO{\tilde{\cO}}
\def\tcOp{\tcO^{\prime}}
\def\tcOpp{\tcO^{\prime\prime}}
\def\tuptau{\tilde{\uptau}}
\def\tuptaup{\tuptau^{\prime}}
\def\tuptaupp{\tuptau^{\prime\prime}}
\def\tuptauppp{\tuptau^{\prime\prime\prime}}
\def\taup{\tau^{\prime}}
\def\taupp{\tau^{\prime\prime}}
\def\tauppp{\tau^{\prime\prime\prime}}
\def\cpT{\cT^+}
\def\cnT{\cT^-}
\def\cpB{\cB^+}
\def\cnB{\cB^-}
\def\BOX{\mathrm{Box}}
\def\ckDD{{\check\DD}}
\def\deltas{\delta^s}
\def\deltans{\delta^{ns}}

\def\PP{\mathrm{PP}}

\def\uum{{\dotminus}}
\def\uup{\divideontimes}
\def\LEG{\mathrm{Leg}}
\def\PBP{\mathrm{PBP}}
\def\BODY{\mathrm{Body}}
\def\eee{\emptyset}


\def\umm{{=}}
\def\upp{{\ast}}
\def\upp{
  {{\setbox0\hbox{$\times$}
      \rlap{\hbox to \wd0{\hss$+$\hss}}\box0
    }}
}


\def\oAC#1{\AC(#1)}
\def\owAC#1{\wAC(#1)}
\def\pAC#1{\Lambda_{+}(\AC(#1))}
\def\nAC#1{\Lambda_{-}(\AC(#1))}
\def\AOD{\mathrm{AOD}}
\def\ttail#1{{#1}_{\bftt}}

\usepackage{subfiles}



\title[]{Special unipotent representations : orthogonal and symplectic groups}

\author [D. Barbasch] {Dan M. Barbasch}
\address{the Department of Mathematics\\
  310 Malott Hall, Cornell University, Ithaca, New York 14853 }
\email{dmb14@cornell.edu}

\author [J.-J. Ma] {Jia-jun Ma}
\address{School of Mathematical Sciences\\
  Shanghai Jiao Tong University\\
  800 Dongchuan Road, Shanghai, 200240, China} \email{hoxide@sjtu.edu.cn}


\author [B. Sun] {Binyong Sun}
% MCM, HCMS, HLM, CEMS, UCAS,
%\address{Academy of Mathematics and Systems Science\\
%  Chinese Academy of Sciences\\
 % Beijing, 100190, China} \email{sun@math.ac.cn}

\address{Institute for Advanced Study in Mathematics\\
  Zhejiang University\\
  Hangzhou, 310058, China} \email{sunbinyong@zju.edu.cn}


\author [C.-B. Zhu] {Chen-Bo Zhu}
\address{Department of Mathematics\\
  National University of Singapore\\
  10 Lower Kent Ridge Road, Singapore 119076} \email{matzhucb@nus.edu.sg}



\subjclass[2000]{22E45, 22E46} \keywords{orbit method, unitary dual, unipotent
  representation, classical group, theta lifting, moment map}

\begin{document}

\message{ !name(sunip_new.tex) !offset(1943) }
\section{Nilpotent orbits: combinatorics}

\def\sqii{\sqrt{-1}}
\def\St#1{\mathrm{St}_{#1}}
\def\VV#1{{}^{#1}V}
\def\SLT{\varphi}
\def\SLTK{\varphi_{\sfss}}
\def\GC{G_{\bC}}


In this section, we describes our combinatorical parameterization of nilpotent orbits and
the local systems in details.

\subsection{Young diagrams, signed Young diagrams and admissible orbit data}

The set of $\Nil(\fgg)$ nilpotent $\GC$-orbits in $\fgg$ are parameterized
by Young diagrams, see \cite[Section~5.1]{CM}. The nilpotent obit $\GC\cdot X$ generated by a nilpotent
element $X$ in $\fgg$
is attached to the Young diagram $\cO$ such that  such that
\[
\bfcc_{i}(\cO) = \dim (\Ker(X^{i+1})/\Ker(X^{i})), \quad \textrm{for all  }\,
 i\in \bN^{+}
\]
By abuse of notaition, we identify $\cO$ with the orbit $\GC\cdot X$.


Let $\slt$ be the complex Lie algebra consisting of $2\times 2$ complex matrices
of trace zero,
\[
  \Xslt := \begin{pmatrix}1/2 & \phantom{-} \sqii/2 \\ \mathbf \sqii/2 & -1/2 \end{pmatrix},
  \quad \text{and} \quad
\eslt := \begin{pmatrix}0 & 1\\ 0 & 0 \end{pmatrix}.
\]
By the Jacobson-Morozov theorem,  there is a
Lie algebra homomorphism
\[
\SLT \colon \slt \rightarrow \fgg \subset \gl(V)
\]
 such that $\SLT(\Xslt)=X$.

 Let $\St{i}=\Sym^{i-1}(\bC^{2})$ denote the realization of the irreducible
 $\slt$-module $\SLT_{i}$ on the $i-1$-symmetric power of the standard
 representation $\bC^{2}$.

As an $\slt$-module via $\SLT$, we have
\begin{equation}\label{eq:Vl.1}
V = \bigoplus_{i\in \bN^{+}} V_{i} \otimes_\bC \St{i},  \qquad (k:=\depth(\sO)\geq -1)
\end{equation}
and
\[
\VV{i} := \Hom_{\slt}( \St{i},V)
\]
is the multiplicity space and
$\dim \VV{i} = \bfcc_{i}(\cO) - \bfcc_{i-1}(\cO)$.
We view %the Young diagram
$\cO$ as the function
\[
  \cO\colon \bfN^{+} \rightarrow \bN \text{ such that } \cO(i) = \dim \VV{i}.
\]

% Let $\SLT_{i}\colon  \slt\rightarrow \mathfrak{gl}_{n}(\bC)$ be the representation
%  of $\slt$ on $\St{i}$.

 We equip $\St{i}$ with a ``standard'' classical space structure
 $(\St{i},\inn{}{}_{\St{i}},J_{\St{i}},L_{\St{i}})$.% with classical symbol $(\star_{i},p_{i},q_{i})$.
 Here
 \begin{itemize}
   \item the classical signature of $\St{i}$ is $(B,(i+1)/2,(i-1)/2)$ if $i$ is odd and
         $(C,i/2,i/2)$ if $i$ is even,
   \item $J_{\St{i}}$ is the complex conjugation on $\St{i}=\Sym^{i-1}(\bC^{2})$,
   \item $L_{\St{i}} = (-1)^{\floor{\frac{i-1}{2}}}\, \Sym^{i-1}({\tiny\begin{pmatrix}\phantom{-}0 & 1 \\-1 & 0 \end{pmatrix}})$,
   \item $\inn{}{}_{\St{i}}$ is uniquely determined by $J_{\St{i}}$ and $L_{\St{i}}$ up to
         a positive scalar (We fix a form $\inn{}{}_{\St{i}}$ for each $i$).
 \end{itemize}

 {\color{red} $L_{\St{i}}$ should be the Lie group element action.}


 We define a equivalent relation on the set $\set{B,C,\wtC,D,C^{*},D^{*}}$ with
 the relation $B\sim D$ and $C\sim \wtC$.

Let $(V,\inn{}{} J,L)$ be a classical group with signature $\sfss$.
Let $ \sO\in \Nil(\fpp_{\sfss})$.
By \cite{Se} (also see \cite[Section~6]{Vo89}), there is a Lie-algebra
homomorphism $\SLT \colon \slt \rightarrow \fgg \subset \gl(V)$ such that
$\SLT(\Xslt)\in \sO$ and compatible with the classical space structure on $V$.
More precisely, the multiplicity space $\VV{i}$ have a unique classical space
structure
$(\VV{i}, \inn{}{}_{\VV{i}},J_{i},L_{i})$ with classical signature
$\sfss_{i}=(\star_{i},p_{i},q_{i})$
such that $\star_{i}\neq \wtC$ and restricted on the $\slt$-isotypic component $\VV{i}\otimes \St{i}$,
\begin{itemize}
  \item $\inn{}{}$ is given by $\inn{}{}_{\VV{i}}\otimes \inn{}{}_{\St{i}}$,
  \item $J$ is given by $J_{i}\otimes J_{\St{i}}$,
  \item $L$ is given by $L_{i}\otimes L_{\St{i}}$,
  \item the symbol $\star_{i}$ satisfies the following condition
        \begin{equation} \label{eq:ssrel}
          \begin{cases}
            \star_{i} \sim\star_{\sfss} & \text{if $i$ is odd}\\
            \star_{i} \sim \star'_{\sfss} & \text{if $i$ is even}
          \end{cases}
          % $\star_{\sfss_{i}} \sim\star_{\sfss}$ if $i$ is odd and
        % $\star_{\sfss_{i}} \sim \star'_{\sfss}$ if $i$ is even  ($\star'_{\sfss}$
        % is the Howe dual of $\star_{\sfss}$).
      \end{equation}
\end{itemize}
Here $\star'_{\sfss}$ is the Howe dual of $\star_{\sfss}$, and
 $\sim$ is the equivalent relation on $\set{B,C,\wtC,C^{*},D,D^{*}}$ such that
 $B\sim D$ and $C\sim \wtC$.
Then the classical signature $\sfss_{i}$ are independent of the choice of $\SLT$
and uniquely determined the orbit $\sO$.

\medskip


\def\CCSS{\overline{\mathsf{CS}}}
Let
\[
\CCSS := \set{(\star,p,q)| \star\neq \wtC \text{ and }(\star,p,q) \text{ is a classical signature}}.
\]
We identify $\sO$ with the function
\[
  \sO\colon \bN^{+}\rightarrow \CCSS, \qquad i \mapsto \sfss_{i}.
\]

A important feature of a compatible $\SLT$ map is that the Kostant-Sekiguchi
correspondence  is given by %classifies the set $\Nil(\fgg_{\sfss})$ of
%real nilpotent orbits, and
\[
  G_{\sfss}\cdot \SLT(\eslt) \leftrightarrow  \KC[\sfss]\cdot \SLT(\Xslt).
\]
%give the Kostant-Sekiguchi correspondence.




Let
$\sfmm \colon \CCSS\rightarrow \bN$ be the map given by
$(\star,p,q)\mapsto p+q$. Now the  complexification
$\GC\cdot \sO$ of a rational nilpotent orbit is given by $\cO := \sfmm\circ \sO$.
Define
\[
  \begin{split}
  \SYD_{\star}(\cO) &:=\Set{\sO\colon \bN^{+}\rightarrow \CCSS| \begin{minipage}{9em}
      %$\star_{\sO( i )}$ equals $\star$ if $i$ is odd,\\% \\
      %$\star_{\sO( i )}$ equals $\star'$ otherwise, \\
      $\star_{\sO(i)}$ satisfies \eqref{eq:ssrel},\\
      $\sfmm\circ \sO = \cO$
      % % $\star_{\sO( i )} $
      % %$\star'$ if $i$ is even, \\
      % $p_{\sO(i)}=q_{\sO(i)}=0$ except for finite many $i$.
      \end{minipage}
    }\\
    \SYD_{\star} &:= \bigsqcup_{\cO\in \Nil_{\star}} \SYD_{\star}(\cO).
  \end{split}
\]
Now $\SYD_{\star}$ is naturally identified with the set
$\bigsqcup_{\sfss}\Nil(\fgg_{\sfss})$ (the set of real nilpotent orbits)
and $\bigsqcup_{\sfss}\Nil(\fpp_{\sfss})$.
This is nothing but the signed Young diagram classification of rational
nilpotent orbits, see \cite{CM}.


\medskip

For a classical signature $\sfss$, we define the complex group
\[
  \KC[[\sfss]] := \GC^{L_{\sfss}}.
\]

 Suppose $\sO\in \Nil(\bfpp_{\sfss})$, $\SLT$ is
compatible with $\sO$ and $X = \SLT(\Xslt)\in \sO$. Let
\[
  \KC[[\sfss]]_{X}:=\Stab_{\KC[[\sfss]]}(X)
\]
be the stabilizer of $X$ in
$\KC[[\sfss]]$.
% centralizer of $\Im(\phi)$ in $\KC[\sfss]$.
Using the signed Young diagram classification, we have the following well known
fact of the isotropic subgroup.
\begin{lem}\label{lem:KX1}
 Then $\KC[[\sfss]]_{X}= R_X\ltimes U_X$,
  where $U_X$ is the unipotent radical of $\KC[[\sfss]]_{X}$ and $R_{X}$ is a reductive
  group canonically identified with
 \[
   \prod_{i\in \bN^{+}} \KC[[\sfss_{\sO(i)}]].
 \]
\end{lem}
As a consequence, $R_{X}$ is a products of complex general linear group and
orthogonal groups
\[
  \prod_{\substack{\sO(i)=(\star_{i},p_{i},q_{i})\\
      \star_{i}\in \set{B,D}, p_{i}+q_{i}>0
    }} \rO_{p_{i}}(\bC)\times \rO_{q_{i}}(\bC).
\]

\medskip

\def\DCO#1{\mathsf{LB}_{#1}(\sO)}

Fix a one-dimensional character $\chi$ of the connected component $\KC[X]^{\circ}$ of $\KC[X]$,
let
\[
\DCO{\chi} := \set{\text{line bundle $\cE$ on $\sO$ such that $\KC[X]^{\circ}$
    acts on the fiber $\cE_{x}$ by $\chi$}}.
\]
By the structure of $R_{X}$, a line bundle $\cE\in DCO{\chi}$ is completely
determined by
the restruction of the $\KC[X]$-module $\cE_{X}$ on the
the orthogonal factors of $R_{X}$.

Let
\[
  \MK = \CCSS \cup \set{(\star,r,s)|\star\in \set{B,D}, r,s\in \bZ}.
\]
We identify $\cE \in \DCO{\chi}$ with the map
\[
  \cE\colon \bN^{+}\rightarrow  \MK
\]
such that,
for $\sO(i)=(\star_{i},p_{i} q_{i})$,
\begin{itemize}
  \item $\cE(i) := \sO(i)$ if $\star_{i} \notin \set{B,D}$;
  \item $\cE(i) := (\star_{i},(-1)^{\epsilon^{+}}p_{i},(-1)^{\epsilon^{-}}q_{i})$
        if $\star_{i} \in \set{B,D}$, and  the factor
        $\rO_{p_{i}}(\bC)\times \rO_{q_{i}}(\bC)$ acts on $\cE_{X}$ by
        ${\det}^{\epsilon^{+}}\otimes {\det}^{\epsilon^{-}}$.
\end{itemize}



Let
 $\sF\colon \MK\rightarrow \CCSS$ be the map given by
$(\star,r,s)\mapsto (\star,\abs{r},\abs{s})$ and
\[
  \MYD(\sO) := \Set{\cE\colon \bN^{+}\rightarrow \MK|\sF\circ \cE=\sO }.
\]
Thanks to the simple structure of $R_{X}$,  the above recipe allows us to identify $\MYD(\sO)$  with the set $\DCO{\chi}$ when
the later set is non-empty.

Let
\[
  \MYD_{\star}(\cO) := \bigcup_{\sO\in \Nil_{\star}(\cO)}\MYD(\sO)
  \AND
  \MYD_{\star} := \bigcup_{\cO\in \Nil_{\star}}\MYD_{\star}(\cO).
  %\MYD := \bigsqcup_{\star\in\set{B,C,D,C^{*},D^{*}}} \MYD_{\star}
\]
 In particular,we identify $\AOD_{\sfss}(\cO)$ with a subset of
 $\MYD_{\star}(\cO)$, see \Cref{defaod}.



\begin{eg}
 Let $1^{r,s}_{\star}$ denote the element in $\MYD_{\star}$ such that
 \[
   1^{r,s}_{\star}(i) := \begin{cases}
     (\star,r,s) & \text{if } i=1\\
     (\star_{i},0,0) & \text{if $i>0$}\\
     %(\star',0,0) & \text{if $i>0$ and $i$ is even}
   \end{cases}
 \]
 Here the symbol $\star_{i}$ is uniquely determined by \eqref{eq:ssrel}.
\end{eg}

We let $\bZ[\SYD_{\star}]$ and $\bZ[\MYD_{\star}]$
 denote the space of free abelian group generated by signed Young diagrams and
marked Young diagrams respectively.

\subsection{Operations on diagrams}
In this section, we define some operations on the signed Young diagrams and
marked Young diagram.

Suppose $\star\in \set{B,D}$.
We define a $\bZ/2\bZ\times \bZ/2\bZ$ action on $\MYD_{\star}$.
Let $(\epsilon^{+},\epsilon^{-})\in \bZ/2\bZ\times \bZ/2\bZ$ , $\cE\in \MYD_{\star}$ such that $(\star_{i},p_{i},q_{i}):=\cE(i)$,
we define $\cE \otimes (\epsilon^{+},\epsilon^{-})\in \MYD_{\star}$ by
\[
(\cE \otimes 1^{\epsilon^{+},\epsilon^{-}})(i) := \begin{cases}
  (\star_{i}, (-1)^{\frac{\epsilon^{+}(i+1)+\epsilon^{-}(i-1)}{2}}p_{i},(-1)^{\frac{\epsilon^{+}(i-1)+\epsilon^{-}(i+1)}{2}}q_{i})
  & \text{if $i$  is odd,}\\
  (\star_{i}, p_{i},q_{i}) & \text{otherwise.}
\end{cases}
\]


For each $\cE\in \AOD_{\star}$, its twist
$\cE\otimes  (1^{{-,+}})^{\epsilon^{+}}\otimes (1^{+,1})^{\epsilon^{-}}$
is a line bundle  in $\AOD_{\star}$.
As elements in $\MYD_{\star}$, we have
\[
  \cE\otimes  (1^{{-,+}})^{\epsilon^{+}}\otimes (1^{+,1})^{\epsilon^{-}} = \cE \otimes (\epsilon^{+},\epsilon^{-}).
\]

\medskip

Suppose $\star\in \set{C,\wtC,D^{*}}$.

For $\cE\in \MYD_{\star}$ such that $(\star_{i},p_{i},q_{i}):=\cE(i)$,
we define $\maltese\cE\in \MYD_{\star}$ by
\[
(\maltese\cE)(i) := \begin{cases}
  (\star_{i}, -p_{i},-q_{i})& \text{if $i\equiv 2 \pmod{4}$ and $\star_{i}\in \set{B,D}$,}\\
  (\star_{i}, p_{i},q_{i}) & \text{otherwise.}
\end{cases}
\]

Clearly, the a operation $\maltese$ defines an involution on
$\MYD_{\star}$.
%Suppose $\sfss = (\star,n,n)$ is a classical signature.
For each $\cE\in \DCO{\chi}$, its twist  $\cE\otimes \deti$
is a line bundle  in $\DCO{\chi'}$ where  $\chi'$ is a certain character. % $\chi'$.
As elements in $\MYD_{\star}$, we have
\[
  \cE\otimes \deti = \maltese (\cE).
\]

We define the argumentation of a marked Young diagram.
For each $\cE\in \MYD_{\star}$ and $\sfss_{0} = (\star'_{0},p,q)\in \CCSS$ such that
$\star'\sim \star'_{0}$,
let $\cE\cdot \sfss_{0}$ be the marked Young diagram in $\MYD_{\star'}$ given by
\[
  (\cE\cdot \sfss_{0})(i) :=
  \begin{cases}
    \sfss_{0} & \text{if } i=1,\\
    \cE(i-1) & \text{if } i>1.
  \end{cases}
\]

We will use the same notation to denote the  natural extension of above
operations to $\bZ[\MYD_{\star}]$.
% and we will use the same notation

We define the partial order $\succeq$ on $\bZ\times \bZ$ by
\[
  (a,b) \succeq (c,d) \Leftrightarrow
  \text{either } a\geq c\geq 0, b\geq d\geq 0  \text{ or }
    a\leq c\leq 0, b\leq d\leq 0.
\]
For a $\cE\in \MYD$ and $(p_{0},q_{0})\in \bZ\times \bZ$, we write
\[
  \cE\succeq (p_{0},q_{0}) \Leftrightarrow
  (p_{1},q_{1})\succeq (p_{0},q_{0})
  \text{ where }\cE(1) = (\star_{1},p_{1},q_{1}).
\]

Suppose $\cE\succeq (p_{0},q_{0})$, we define the truncation of $\cE$ by
\[
  \Lambda_{(p_{0},q_{0})}(\cE) = \begin{cases}
    (\star_{1},p_{1}-p_{0},q_{1}-q_{0}) & \text{if } i=1,\\
    \cE(i) & \text{if } i>1.
  \end{cases}
\]
We extend to the truncation operation $\Lambda_{p_{0},q_{0}}$ to
$\bZ[\MYD_{\star}]$
by define $\Lambda_{p_{0},q_{0}}(\cE) := 0$ when $\cE \nsucceq (p_{0},q_{0})$.

The augmentation and truncation of signed Young diagram are defined by the same
formula.

% For $\cL = \sum_{i=1}^{k}\cL_{i}\in \KM$ with $\cL_{i}\in \MYD$, we let
% \[
%   \ssign(\cL)= \Set{\ssign(\cL_{i})|i=1,\cdots,k}\text{ and }
%   \lsign(\cL)= \Set{\lsign(\cL_{i})|i=1,\cdots,k}
% \]

For any $\sO\in \SYD_{\star}$, let $\Sign(\sO) = (p,q)$ if $\sO$ corresponds
a nilpotent orbit in $\Nil_{\sfss}(\fpp)$ with $\sfss=(\star,p,q)$.
When $\sO(i)=(\sfss_{i},p_{i},q_{i})$, the signature is
\[
  \Sign(\sO) = \sum_{i} (\floor{\frac{i+1}{2}}p_{i}+\floor{\frac{i}{2}}q_{i}, \floor{\frac{i}{2}}p_{i}+\floor{\frac{i+1}{2}}q_{i})
\]

\subsection{Theta lifts of local systems}

\subsubsection{Lift from type C to D}

For each signature $(p,q)\in \bN\times \bN$ such that $p+q$ is even, we define a map
\[
  \vartheta_{CD,(p,q)} \colon \KM(C)\rightarrow \KM(D)
\]
as the following:
Let $\cL\in \MYD(C)$. Let $(p_{1},q_{1})=\lsign(\cL)$ and
\[
(p_{0}, q_{0})  = (p,q) -\ssign(\cL)-(q_{1},p_{1}).
\]
We define
\[
  \vartheta_{CD,(p,q)}(\cL) =
  \begin{cases}
    (\dagger \maltese^{\frac{p-q}{2}}(\cL))\cdot \dagger_{p_{0}, q_{0}} &
    \text{if } p_{0}\geq 0 \text{ and } q_{0} \geq 0 \\
    0 & \text{otherwise}
  \end{cases}
\]


\subsubsection{Lift from type D to type C}

For each non-zero integer $n\in \bN$, we define a map
\[
  \vartheta_{DC,n} \colon \KM(D)\rightarrow \KM(C)
\]
as the following:
Let $\cL\in \MYD(D)$. Let $(p_{1},q_{1})=\lsign(\cL)$ and
$(p,q) = \ssign(\cL)$.
Let
\[
  n_{0} = n - (p+q)/2 - (p_{1}+q_{1})/2
\]
We define
\[
  \vartheta_{DC,n}(\cL) =
  \begin{cases}
    \maltese^{\frac{p-q}{2}}((\dagger \cL)\cdot \dagger_{n_{0}, n_{0}}) &
    \text{if } n_{0}\in \bZ_{\geq 0}\\
    \maltese^{\frac{p-q}{2}}(\dagger \cL^{+}  + \dagger \cL^{-})& \text{if } n_{0}=-1\\
    0&\text{otherwise}
  \end{cases}
\]
We remark that
$\maltese^{\frac{p-q}{2}}((\dagger \cL)\cdot \dagger_{n_{0}, n_{0}})= (\maltese^{\frac{p-q}{2}}(\dagger\cL))\cdot \dagger_{n_{0},n_{0}}$.

\trivial{
  We take a splitting $\rO(p,q)\times \Sp(n,\bR)$ in to the big metaplectic
  group $\Mp$ such that $\rO(p,\bC)\times\rO(q,\bC)$ acts on the Fock model linearly.
  The maximal compact $K_{\Sp(n,\bR)}= \rU(2n)$ acts on the Fock model by the
  character $\zeta=\det^{(p-q)/2}$.
  For the component of the rational nilpotent orbit $\Sp(2n,\bR)$,
  the factor of the component group is $\Sp$ if the corresponding row has odd
  length.
  Otherwise the component group is $\rO(p_{2k})\times\rO(q_{2k})$ where
  $(p_{2k},q_{2k})$ is the signature corresponding to the $2k$-rows.
  $\zeta|_{\rO(p_{2k})}= \det^{(p-q)k/2}_{\rO(p_{2k})}$ which is nontrivial if
  and only if $p-q\equiv 2\pmod{4}$ and $2k\equiv 2\pmod{4}$.
  This gives the formula above.
}


{
  \color{red}
  In the proof later, we don't care above the marks for rows with length longer
  than 1 in the most of the case. So we will omit $\maltese$ sometimes.
  When there is no confusion, we will simply write $\vartheta$ for
  $\vartheta_{CD,(p,q)}$ and $\vartheta_{CD,(p,q)}$.
}


\subsubsection{Lift from type M to B}

For each signature $(p,q)\in \bN\times \bN$ such that $p+q$ is odd, we define a map
\[
  \vartheta_{MB,(p,q)} \colon \KM(M)\rightarrow \KM(B)
\]
as the following:
Let $\cL\in \MYD(M)$. Let $(p_{1},q_{1})=\lsign(\cL)$ and
\[
(p_{0}, q_{0})  = (p,q) -\ssign(\cL)-(q_{1},p_{1}).
\]
We define
\[
  \vartheta_{MB,(p,q)}(\cL) =
  \begin{cases}
    (\dagger \maltese^{\frac{p-q+1}{2}}(\cL))\cdot \dagger_{p_{0}, q_{0}} &
    \text{if } p_{0}\geq 0 \text{ and } q_{0} \geq 0\\
    0 & \text{otherwise}
  \end{cases}
\]


\subsubsection{Lift from type B to type M}

For each non-zero integer $n\in \bN$, we define a map
\[
  \vartheta_{BM,n} \colon \KM(B)\rightarrow \KM(M)
\]
as the following:
Let $\cL\in \MYD(D)$. Let $(p_{1},q_{1})=\lsign(\cL)$ and
$(p,q) = \ssign(\cL)$.
Let
\[
  n_{0} = n - (p+q)/2 - (p_{1}+q_{1})/2
\]
We define
\[
  \vartheta_{BM,n}(\cL) =
  \begin{cases}
    \maltese^{\frac{p-q-1}{2}}((\dagger \cL)\cdot \dagger_{n_{0}, n_{0}}) &
    \text{if } n_{0}\in \bZ_{\geq 0}\\
    \maltese^{\frac{p-q-1}{2}}(\dagger \cL^{+}  + \dagger \cL^{-})& \text{if } n_{0}=-1\\
    0&\text{otherwise}
  \end{cases}
\]
We remark that
$\maltese^{\frac{p-q-1}{2}}((\dagger \cL)\cdot \dagger_{n_{0}, n_{0}})= (\maltese^{\frac{p-q-1}{2}}(\dagger\cL))\cdot \dagger_{n_{0},n_{0}}$.

\trivial{
  For odd orthogonal-metaplectic group case, we still take the splitting such
  that $\rO(p,q)$ acts linearly.

  The maximal compact $\wtK_{\Sp(n,\bR)}= \widetilde{\rU(2n)}$ acts on the Fock model by the
  character $\zeta=\det^{(p-q)/2}$.

  Then
  $\det^{(p-q)/2}|_{\rO(p_{2k})\times \rO(q_{2k})}=\det^{\frac{(p-q)k}{2}}_{\rO(p_{2k})}\boxtimes
  \det^{\frac{(p-q)k}{2}}_{\rO(q_{2k})}. $
  Note that $p-q$ is an odd number.
  When $k$ is even the character on $\widetilde{\rO(p_{2k})}/\widetilde{\rO(q_{2k})}$ are
  $\bfone$ or $\det$. When $k$ is odd the character would be $\det^{\half}$ or
  $\det^{\half+1}$.

  We assume the default character on $2k$-rows is $\det^{\frac{k}{2}}$, and
  we use $+/-$ to mark the row. Otherwise, we use $\upp/\umm$ to mark the rows.
}

\subsection{Examples and remarks}

\subsubsection{Example}
\begin{eg}
  Let
  $\cT :=\dagger\dagger (\dagger_{2,1}) + \dagger\dagger(\dagger_{{1,2}})$ and $\cP= \ddagger_{1,3}$.
  Then
  \[
    \cT\cdot \cP = \tytb{\uup\uum\uup,\uup\uum\uup,\uum\uup\uum,+,=,=,=}
    \cup
    \tytb{\uup\uum\uup,\uum\uup\uum,\uum\uup\uum,+,=,=,=}.
  \]
\end{eg}


% For each local system $\cL$ of type B/D, define $\pcL$ (resp. $\ncL$) be the
% part of $\cL$ obtained by deleting a ``$+$''-symbol (resp. ``$-$''=symbol) among 1-rows.
% $\pcL:=0$ (resp. $\ncL:=0$) if there is no ``$+$''-symbol (resp.
% ``$-$''-symbol).

% For an local system $\cL$, $\cL\succeq\cC$ means there is a
% factorization of $\cL = \cB\cdot \cC$;
% $\cL\supset \cC$ means there is an irreducible component $\cL_{1}$ of $\cL$ such
% that $\cL_{1}\succeq \cC$.


% \subsubsection{}

%\subsection{Definition of $\eDD$}
\subsubsection{Signs of the local systems obtained by iterated lifting}


Suppose $\cL\in \KM$ is obtained by iterated theta lifting and character
twisting in \Cref{sec:tchar.DB}.
Then the set $\lsign(\cL)$ has restricted possibilities.
First, $\ssign(\cL)$ is always a singleton.

When $\cO$ is of type B/D, $\lsign(\cL)$ is a singleton $\set{(p_{1},q_{1})}$
where
\[
  (p_{1},q_{1})= \ssign(\cL) - (n_{0},n_{0}) \quad \text{and} \quad 2n_{0} = \abs{\eDD(\cO)}.
\]

When $\cO$ is of type C/M,
$\lsign( \cL ) =\set{ (p_{1},q_{1})}$ is a singleton in the most of the case.
The $\lsign(\cL)$
may have two elements if $\cL$ is obtained by good generalized lifting, and it has the form
$\lsign( \cL ) = \set{(p_{1}+1,q_{1}),(p_{1},q_{1}+1)}$.

% Another important invariant is $\bsign( \cL )$ which indicates the signature of
% the local system of length one rows in $\cL$.
\subsection{Dual pairs, descent and lift of nilpotent orbits}\label{sec:descent}

\begin{defn}[Dual pair]\label{def:DP}
\begin{enumT}
\item A \emph{complex dual pair} is a pair consisting of an $\epsilon $-symmetric
  bilinear space and an $\epsilon '$-symmetric bilinear space, where
  $\epsilon ,\epsilon '\in \set{\pm1}$ with $\epsilon \epsilon '=-1$.
\item  A \emph{rational dual pair} is a pair consisting of an $(\epsilon,\dotepsilon)$-space
  and an $(\epsilon',\dotepsilon')$-space, where
  $\epsilon, \epsilon', \dotepsilon, \dotepsilon' \in \set{\pm1}$ with
  $\epsilon\epsilon'= \dotepsilon\dotepsilon'=-1$.
\end{enumT}
\end{defn}


\subsubsection{Lifts and descents of nilpotent orbits}\label{sec:LD}

Let
\begin{equation*}
\bfWo := \set{T \in \bfW | T \text{ is a surjective map from $\bfV$ onto $\bfV'$}}.
\end{equation*}
Clearly $\bfWo\neq \emptyset$ only if $\dim \bfV\geq \dim \bfV'$.


Suppose $\bfee\in \cO \in \Nil_{\bfG}(\fgg)$ and
$\bfee'\in \cO' \in \Nil_{\bfG'}(\fgg')$.  We call $\bfee'$ (resp. $\cO'$) a
descent of $\bfee$ (resp. $\cO$), if there exits  $T\in \bfWo$ such that
$$
\MM(T) = \bfee\quad\textrm{and}\quad \MM'(T) = \bfee'.
$$
Put $$\Xo := \bfW^\circ \cap \cX,$$
 and write $\bfK:=\bfK_{\bfV}$ and $\bfK':=\bfK_{\bfV'}$. Suppose $X\in \sO \in \Nil_{\bfK}(\fpp)$ and
$X'\in \sO' \in \Nil_{\bfK'}(\fpp')$.  We call $X'$ (resp. $\sO'$) a
descent of $X$ (resp. $\sO$), if there exits
$T\in \Xo$ such that
$$
\MMP(T) = X\quad\textrm{and}\quad \MMP'(T) = X'.
$$
In all cases, we will say that $T$ realizes the descent, and $\cO$ (resp. $\sO$) is the lift of $\cO'$ (resp. $\sO'$).
In the notation of \Cref{defdo}, we then have
\[
\DDc(\bfdd_{\cO}) = \bfdd_{\cO'}\quad\textrm{ and }\quad \DD(\ssdd_{\sO}) = \ssdd_{\sO'}.
\]
Hence the notion of descent (for nilpotent orbits) defined here agrees with  that of \Cref{defdo} and \eqref{dedd} (for Young diagrams).
We will thus write
\[
\cO' = \DDc(\cO)=\DDc_{\bfV, \bfV'}(\cO)\quad \textrm{and}\quad  \sO' = \DD(\sO)=\DD_{\bfV, \bfV'}(\sO).
\]
We record a key property on descent and lift:
\begin{equation}\label{eq:def.LsO22}
\MM(\MM'^{-1}(\bcOp)) = \bcO \quad \text{and} \quad
\MMP(\MMP'^{-1}(\bsOp)) = \bsO,
\end{equation}
where ``$\;\overline{\phantom{m}}\;$'' means taking Zariski closure. This is checked by
using explicit formulas in \cite{KP,DKPC} (for complex dual pairs) and
\cite[Lemma~14]{Ohta} (for rational dual pairs).


In fact by \cite[Theorem 1.1]{DKPC}, the notion of lift can be extended to an arbitrary complex dual pair and  an arbitrary complex nilpotent orbit: for any $\cO'\in \Nil_{\bfG'}(\fgg')$, $\MM(\MM'^{-1}(\bcO'))$ equals to the closure of a
 unique nilpotent orbit $\cO \in \Nil_{\bfG}(\fgg)$. We call $\cO$ the \emph{theta lift} of $\cO'$, written as
 \begin{equation}
 \label{def:LC}
  \cO=\oliftc_{\bfV',\bfV}(\cO').
 \end{equation}
%We will also say $\sO'$ is in the domain of theta lift of nilpotent orbits with respect to the rational dual pair $(\bfV',\bfV)$ and similarly write
%\begin{equation}\label{eq:def.LsO}
  %\sO = \olift_{\bfV',\bfV}(\sO').
%\end{equation}

%\begin{remark}
%The maps $\vartheta_{\bfV',\bfV}$ and $\vartheta_{\bfV,\bfV'}$ are not the mutually inverse maps. In fact, for most of
%$\cO'\in \Nil_{\bfG'}(\fgg')$, $\vartheta_{\bfV',\bfV}(\vartheta_{\bfV,\bfV'}(\cO'))\neq \cO'$.
%\end{remark}

\subsubsection{Generalized descent of nilpotent orbits}
\label{def:GD}
Let
\[
\bfWg := \set{T \in \bfW| \text{the image of $T$ is a
      non-degenerate subspace of $\bfV'$}}
\]
and
\[
  \Xg:= \bfWg\cap \cX.
\]
Suppose $X\in \sO \in \Nil_{\bfK}(\fpp)$ and
$X'\in \sO' \in \Nil_{\bfK'}(\fpp')$.  We call $X'$ (resp. $\sO'$) a
generalized descent of $X$ (resp. $\sO$), if there exits
$T\in \Xg$ such that
$$
\MMP(T) = X\quad\textrm{and}\quad \MMP'(T) = X'.
$$
As before, we say that $T$ realizes the generalized descent.

 It is easy to see that for each  nilpotent orbit $\sO\in \Nil_{\bfK}(\fpp)$,
 the following three assertions are equivalent (\emph{cf.} \cite[Table~4]{DKP2}).
 \begin{itemize}
 \item
   The orbit $\sO$ has a generalized descent.
   \item
  The orbit  $\sO$ is contained in the image of the moment map $M$.
  \item
  Write $\ssdd_\sO = [d_0, d_1, \cdots, d_k]\in \ssP$, then
  $$\sign{\bfV'}  \succeq \sum_{i=1}^{k} d_i.$$
  \end{itemize}
  When this is the case,   $\sO$ has a unique generalized
descent $\sO'\in \Nil_{\bfK'}(\fpp')$, and
  \begin{equation}\label{eq:GD}
\ssdd_{\sO'} = [d_{1}+s, d_{2}, \cdots, d_k], \quad \textrm{where } \, s:= \sign{\bfV'} - \sum_{i=1}^{k} d_i.
\end{equation}
We write $\sO' = \gDD_{\bfV,\bfV'}(\sO).$ On the other hand, different nilpotent
orbits may map to a same nilpotent orbit under $\gDD_{\bfV,\bfV'}$.
\trivial[h]{
Using  \cite[Table~4]{DKP2} or the observation that $M^{-1}(X)\cap \Xg$ is the
unique closed
$\bfK'$-orbit in $M^{-1}(X)$ for an element $X\in \sO$, we see that
$\sO'$ is the minimal
$\bfK$-orbit in $M'(M^{-1}(\sO))$ and \cref{eq:GD.min} holds.
}
\trivial[h]{
To show the equation \Cref{eq:GD}, it suffices to consider the descent case,
i.e. when $s = (0,0)$.

Fix a $T\in \cX^\circ \subset \Hom(\bfV,\bfV')$ realizing the descent.
Fix a $L$ invariant decomposition $\bfV = \Ker(X) \oplus \bfY$. $T|_\bfY$ is an
isomorphism.
It suffices to check that $\sign{\bfY} = \sign{\bfV'}$.
This is clear since $\dotepsilon L' T L^{-1}  = \bfii\, T$.  (If $v\in
\bfY^{L, \pm 1}$, then $L' T v = \pm L' T  L^{-1} v = \pm \bfii\, Tv$,
i.e. $v\in \bfV'^{L', \pm \bfii}$. Similarly, If $v\in
\bfY^{L, \pm \bfii}$, then $L' T v = \mp\bfii L' T  L^{-1} v = \mp\bfii \bfii\,
Tv = \pm Tv$,
i.e. $v\in \bfV'^{L', \pm 1}$.
}

Analogously, suppose $\bfee\in \cO\in \Nil_{\bfG}(\fgg)$ and
$\bfee'\in \cO' \in \Nil_{\bfG'}(\fgg')$.  We call $\bfee'$ (resp. $\cO'$) a generalized descent of $\bfee$ (resp.
$\cO$), if there exits an element  $T\in \bfWg$ such that
\[
\MM(T) = \bfee \quad\textrm{and}\quad \MM'(T) = \bfee'.
\]
When this is the case, $\cO'$ is determined by $\cO$ and we write $\cO' = \gDDc_{\bfV,\bfV'}(\cO)$.



  \begin{lem}\label{gendec}
  Assume that $\sO\in \Nil_{\bfK}(\fpp)$ has a generalized descent $\sO'\in \Nil_{\bfK'}(\fpp')$. Let $X\in \sO$ and $T\in \Xg$ such that $M(T)=X$. Then $\bfK'\cdot T$ is the unique closed $\bfK'$-orbit in $M^{-1}(X)$. Moreover,     \begin{equation}\label{eq:GD.min}
\sO' =
M'(M^{-1}(\sO))\cap \CO'=M'(M^{-1}(\sO))\cap \overline{\CO'},
\end{equation}
where $\CO':=\bfG'\cdot \sO'$, which is the generalized descent of $\CO:=\bfG\cdot \sO$.
  \end{lem}
  \begin{proof}
 It is elementary to check that $\bfK'\cdot T$ is Zariski closed in $\cX$.  Then the first assertion follows by using the isomorphism \eqref{clinv}. Note that $\sO'$ is the only $\bfK'$-orbit in
 $\overline{\CO'}\cap \fpp'$ whose Zariski closure contains $\sO'$.
  Thus the first assertion implies the second one.
    \end{proof}





\medskip

In this article, we will need to consider the following special types of nilpotent orbits.

\begin{defn}\label{def:GD.good}
  A nilpotent oribit $\cO \in \Nil_{\bfG}(\fgg)$ with $\rdd_\cO = [c_0, c_{1} \cdots, c_k]
  \in \cP_\epsilon$
  is said to be good for generalized descent if $k\geq 1$ and $c_0 = c_{1}$.  A nilpotent
  orbit $\sO \in \Nil_{\bfK}(\fpp)$ is said to be  \emph{good} for generalized descent if
  the  nilpotent orbit  $\bfG \cdot \sO\in \Nil_{\bfG}(\fgg)$ is  good for generalized descent.
\end{defn}

The following lemma exhibits a certain maximality property of nilpotent orbits which are good for generalized descent.
\begin{lem}\label{lem:GDS.set}
 If $\cO\in \Nil_{\bfG}(\fgg) $ is good for generalized descent, and
  $\cO' = \gDDc_{\bfV,\bfV'}(\cO)$, then
  $\MM(\MM'^{-1}(\bcOp)) = \bcO$. Consequently, if $\sO\in \Nil_{\bfK}(\fpp)$ is good for generalized descent and
  $\sO' = \gDD_{\bfV,\bfV'}(\sO)$, then $\sO$ is an open $\bfK$-orbit in
  $\MMP(\MMP'^{-1}(\bsOp))$.
\end{lem}

%\begin{remark}
%Good geometric properties of good orbits (\Cref{lem:GDS.sh}) will play a crucial role in \Cref{prop:GDS.AC}.
%\end{remark}

\begin{proof}
This is easy to check using the explicit description of
$\MM(\MM'^{-1}(\bcO'))$ in \cite[Theorem~5.2 and 5.6]{DKPC}. % \cite[]{DKP2}).
\end{proof}


%\subsubsection{Extending to $\slt$-modules}
% We now extend the notion of descent to $\slt$ homomorphisms, \Cref{sec:KX}.

% \begin{defn}
%   \begin{enumT}
%   \item \label{it:dec.slt1} Let $(\bfV, \bfV')$ be a rational dual pair.  Let
%     $\gamma =\set{\bfee,\bfhh,\bfff}$ be an $\slt$-triple in $\fgg$ and let
%     $\gamma '=\set{\bfee',\bfhh',\bfff'} $ be an $\slt$-triple in $\fgg'$.
%     % let $\phi\colon \slt\rightarrow \fgg$ be the corresponding Lie algebra
%     % morphism.
%     % let $\phi'\colon \slt\rightarrow \fgg'$ be the corresponding Lie algebra
%     % morphism.
%     We say the $\gamma '$ is a descent of $\gamma $ if
%     \begin{enumC}
%     \item $\bfee'$ is a descent of $\bfee$, and
%     \item there exists
%       \[
%         T\in \bfW_{1} := \set{T\in \bfW| (\bfhh,\bfhh')\cdot T = T}.
%       \]
%       realizing the descent.
%     \end{enumC}
%   \item \label{it:dec.slt2} Let $(\bfV, \bfV')$ be a rational dual pair.
%     \begin{itemize}
%     \item Let $\phi \colon \slt\rightarrow \fgg$ be a $J$-compatible morphism,
%       and $\phi ' \colon \slt\rightarrow \fgg'$ be a $J'$-compatible morphism.
%       We say $\phi '$ is a descent of $\phi $ if
%       \begin{enumC}
%       \item \label{it:dec.slt3} $\phi'(\eslt)$ is a descent of $\phi(\eslt)$,
%         and
%       \item there exists an element $T \in W_1=:W\cap \bfW_1$ realizing the
%         descent.
%       \end{enumC}
%     \item Let $\phi \colon \slt\rightarrow \fgg$ be a $L$-compatible morphism,
%       and $\phi ' \colon \slt\rightarrow \fgg'$ be a $L'$-compatible morphism.
%       We say $\phi '$ is a descent of $\phi $ if
%       \begin{enumC}
%       \item \label{it:dec.slt4} $\phi'(\Xslt)$ is a descent of $\phi(\Xslt)$,
%         and
%       \item there exists an element $T \in \cX_1=:\cX \cap \bfW_1$ realizing the
%         descent.
%       \end{enumC}
%     \end{itemize}
%   \item In all cases, we will say that $T$ realizes the descent, and we write
%     $\phi' = \DD(\phi)$.
%   \end{enumT}
% \end{defn}

% \begin{remark}
% %\begin{enumR}
% %Note that the pair $\set{\bfee,\bfhh}$ already determine an $\slt$-triple. On the other
% %hand, an arbitrary pair $(\bfee,\bfhh)$ where $\bfee$ is nilpotent, $\bfhh$ is semisimple and
% %  $[\bfhh,\bfee]=2\bfee$ may not come from an $\slt$-triple in general.
% Suppose $\bfee'$ is a descent of $\bfee$ realized by $T\in \bfWo$. Given an $\slt$-triple $\set{\bfee,\bfhh,\bfff}$, we define $\bfhh'\in \fgg'$ by
%   $\bfhh'(T(v)) := T(\bfhh(v)) + T(v)$ for all $v\in \bfV$, so that $(\bfhh,\bfhh')\cdot T = T$. One checks that pair
%   $(\bfee',\bfhh')$  extends to an $\slt$-triple $\set{\bfee',\bfhh',\bfff'}$ and thus descent of a nilpotent orbit can always be extended to descent of the corresponding $\slt$-triple\footnote{There is no clean formula for $\bfff'$ but its existence is clear by examining the decomposition in \Cref{eq:Vl.1}. See discussions in \cite{DKP2}.}.
%   Similar statements hold for rational descents.
% \end{remark}

% The arguments in \cite[Section~8]{DKP2} (for stable range case) extends naturely to the descent of nilpotent
% orbit case. Hence we have the compatibility of descent and Kostant-Sekiguchi correspondence:
% \begin{lem}
%   Suppose $\sOr'\in \Nil_{G'}(\fgg'_{\R})$, $\sOr\in \Nil_{G}(\fgg_{\R})$, $\KS{\sOr'}=:\sO'\in \Nil_{\bfK'}(\fpp')$ and
%   $\KS{\sOr}=:\sO\in \Nil_{\bfK}(\fpp)$.
%   Then $\DD {\sOr}  = \sOr'$ if and only if $\DD {\sO} = \sO'$, i.e.
%   \[
%     \DD{\KS{\sOr}} = \KS{\DD{\sOr}}.
%   \]
%   \qed
% \end{lem}


\trivial[h]{An alternative proof: Let $\phi \colon \slt\rightarrow \fgg$ be a $(J,L)$-compatible morphism realizing the Kostant-Sekiguchi correspondence of $\cO$ with $\KS{\cO}$.
Let $\phi'$ be a $J$-compatible descent of $\phi$, realized by an element $T\in W_1$. By conjugating with an element of $G'$ if necessary, one may assume that $\phi'$ is $(J',L')$-compatible.
From the $\slt$-morphisms, we define the Cayley elements
\[
C := \phi(\cslt)\in \bfG, \text{ and } C' := \phi'(\cslt)\in \bfG'.
\]
Let $T' = C'TC^{-1}$. Since $(C'TC^{-1})^\mstar = C T^\mstar C'^{-1}$ we see
that it realize the descent from $\phi(\Xslt)$ to $\phi'(\Xslt)$.
Now it suffices to verify that $T'\in \cX$.
Using the block decomposition \cref{eq:Vl.1}, the problem reduces to case where
$\phi(\eslt)$ is in the principle
nilpotent orbit in $\rO(n+1,n)$ or $\Sp(2n)$. This could be down by a explicit calculation, see formulas in \cite[Section~6]{DKP2}.
}

\subsubsection{Map between isotropy groups}\label{sec:alpha}
%\subsubsection{Correspondence between nilpotent orbits}


% We now review some the geometry of moment maps.

% \begin{defn}
% For any closed set $S'\subset \fpp'$, define
% \[
% \olift{S'} := M(M'^{-1}(S')).
% \]

% Suppose $\cO'$ is a nilpotent $K$-orbit in $\fpp'$, define
% \[
% \olift{\cO'} := \set{\cO | \cO \text{ is open in } \olift{\bcO'}}.
% \]

% Define $\oliftc$ and $\oliftr$ for the complex and real moment maps similarly.
% \end{defn}
% \begin{remark}
% By \cite{DKPC}, $\oliftc(\cO'_\bC)$ is always a singleton for any complex
% nilpotent orbit $\cO'_\bC$.
% On the other hand, by \cite{DKP2}, $\olift(\bcO')$ has several open orbits in
% general.
% \end{remark}




% By \Cref{lem:LM.E}, it is important to understand the set
% $\olift{\bcO'}$.







% \begin{defn}
% Define the partial order on $\bN^2$ by: $(n^+_1,_1)\succeq (\nn+_2,\nn-_2)$ iff
% $\nn+_1\geq \nn+_2$ and $\nn-_1\geq \nn-_2$.
% For $\epsilon$-Hermitian space $V$, define $\sign{V} := (\nn+{V},\nn-{V})$ as
% the following table

% \centerline{
% \begin{tabular}{c|c}
%   \hline
%   $G_\bR$ & $\sign{V}$ \\
%   \hline
%   $\rO(p,q)$ & $(p,q)$\\
%   $\Sp(2n,\bR)$ & $(n,n)$\\
%   $\rU(p,q)$ & $(p,q)$ \\
%   $\rO^*(2n)$ & $(n,n)$\\
%   $\Sp(p,q)$ & $(2p,2q)$\\
%   \hline
% \end{tabular}
% }
% The set $\NilP$ of nilpotent $K$-orbits in $\fpp$ is parameterized by signed Young
% diagram.
% For $\cO' \in \NilP$, let $[c_k,\cdots, c_0]$ be the list of length columns of the
% underlying diagram.
% Let \[
% \lsign{\cO} := (\lnn+{\cO},\lnn-{\cO})
% \]
% be the sign of the most left column
% of $\cO$.
% \end{defn}





% In our paper, we make could describe the lift of orbits explicitly.
% \begin{lem}[]
%   Suppose $\sign{V}\succ \sign{V'}$.
%   Then
%   \[
%     \RR{\fpp'}{\Xo} = \set{\cO'\in \NilP'|d(V,V') \succeq \lsign{\cO'}},
%   \]
%   where $d(V,V'):= \sign{V}-\sign{V'}$.
%   Let $\cO'\in \RR{\fpp'}{\Xo}$, then $\olift{\cO'}$ is the signed Young diagram
%   obtained by attaching $d(V,V')$ on the left of $\cO'$.

%   In particular, $\oliftc{\cO'_\bC} = (\olift{\cO'})_\bC$ corresponding to the
%   Young diagram obtained by attaching a column of $\dim V - \dim V'$ boxes on
%   the left of $\cO'_\bC$.
% \end{lem}
% \begin{proof}
%   The lemma follows immediately by \cite[Theorem~5.2 and 5.6]{DKPC} and
%   \cite[Table~4]{DKP2} (see also \cite{Ohta4} for the unitary group case.).
% \end{proof}

%For $T\in \cX$, $X\in \fpp$ and $X' \in \fpp'$, we denote corresponding isotropy subgroups by
%\[
%  \bfS_T :=
%  \Stab_{\bfK\times \bfK'}(T), \quad  \bfK_X:= \Stab_{\bfK}(X)\quad \text{and}\quad \bfK'_{X'}
%  := \Stab_{\bfK'}(X') \quad \text{respectively}.
%\]

Suppose $\sO\in \Nil_{\bfK}(\fpp)$ admits a descent $\sO'\in
  \Nil_{\bfK'}(\fpp')$. % is the descent of $\sO\in \Nil_{\bfK}(\fpp)$.
  According to
  \cite[Proposition 11.1]{KP} and \cite[Lemmas~13 and 14]{Ohta},
  $\MMP^{-1}(\sO)$
  is a single
  $\bfK\times \bfK'$-orbit contained in $\Xo$ and $\MMP'(\MMP^{-1}(\sO)) =
  \sO'$. Moreover $\bfK'$ acts on $\MMP^{-1}(\sO)$ freely.

  Fix $T\in \MMP^{-1}(\sO)$ which realizes the descent from $X := \MMP(T)\in \sO$
  to $X' := \MMP'(T)\in \sO'$. Denote the respective isotropy subgroups by
  \[
    \bfS_T :=
    \Stab_{\bfK\times \bfK'}(T), \quad  \bfK_X:= \Stab_{\bfK}(X)\quad \text{and}\quad \bfK'_{X'}
    := \Stab_{\bfK'}(X').
  \]
  Then there is a unique homomorphism
  \begin{equation}
    \label{eq:alpha}
    \alpha\colon \bfK_X \mapsto \bfK'_{X'}
  \end{equation}
  such that $\bfS_T$ is the graph of $\alpha$:
  \[
    \bfS_T = \set{(k,\alpha(k))\in \bfK_X\times \bfK'_{X'}|k\in \bfK_X}.
  \]
  % \begin{proof} \cref{it:lemM.0} amounts to the first fundamental theorem of classical invariant theory \cite{Weyl}. \Cref{it:lemM.1} and \cref{it:lemM.2} are in. \cref{it:lemM.3} is an immediate consequence of \cref{it:lemM.2}.
% \end{proof}
% \begin{lem}[{\cite{Weyl}, \cite[Proposition 11.1]{KP}, \cite[Lemmas 13 and 14]{Ohta}}]\label{lem:O1}\label{lem:DS.set}
% Suppose $(\bfV,\bfV')$ is a rational dual pair such that $\dim(\bfV)\geq \dim (\bfV')$.
% The followings hold true.
% \begin{enumT}
% \item \label{it:lemM.0} $\MMP'\colon \cX \rightarrow \fpp'$ is the affine quotient of
%   $\cX$ by the $\bfK$ action. The image of $\MMP$ is reduced and $\MMP\colon \cX \rightarrow
%   M(\cX)$ is the affine quotient of $\cX$ by the $\bfK'$ action.
% \item \label{it:lemM.1} $\MMP' \colon \Xo \rightarrow \fpp'$ is smooth and $\MMP \colon
%   \Xo \rightarrow \MMP(\Xo)$ is a locally trivial fibration with typical fiber
%   $\bfK'$.
% \end{enumT}
% Suppose further $\sO'\in
%   \Nil_{\bfK'}(\fpp')$ is the descent of $\sO\in \Nil_{\bfK}(\fpp)$. The following hold true.
% \begin{enumT}[resume]
% \item \label{it:lemM.2} % $\bfY:=\MMP^{-1}(\sO)$
%   $\MMP^{-1}(\sO)$
%   is a single
%   $\bfK\times \bfK'$-orbit  contained in $\Xo$ and $\MMP'(\MMP^{-1}(\sO)) = \sO'$. Moreover, as the scheme theoretical
%   pre-image of $\sO$, $\MMP^{-1}(\sO)$  is reduced and smooth.
% \item \label{it:lemM.3} Fix $T\in \MMP^{-1}(\sO)$ and let $X := \MMP(T)\in \sO$
%   and $X' := \MMP'(T)\in \sO'$.  Then there is a (unique) homomorphism
% \begin{equation}
% \label{eq:alpha}
% \alpha\colon \bfK_X \mapsto \bfK'_{X'}
% \end{equation}
% such that $\bfS_T$ is the graph of $\alpha$:
% \[
% \bfS_T = \set{(k,\alpha(k))\in \bfK_X\times \bfK'_{X'}|k\in \bfK_X}.
% \]
% \qedhere
% \end{enumT}
% \end{lem}
% \begin{proof} \cref{it:lemM.0} amounts to the first fundamental theorem of classical invariant theory \cite{Weyl}. \Cref{it:lemM.1} and \cref{it:lemM.2} are in \cite[Proposition 11.1]{KP} and \cite[Lemmas 13 and 14]{Ohta}. \cref{it:lemM.3} is an immediate consequence of \cref{it:lemM.2}.
% \end{proof}
The homomorphism $\alpha$ is uniquely determined by the requirement that
\[
\alpha(k)(Tv) = T(kv)\quad \textrm{ for all }v\in \bfV, \, k\in \bfK_X.
\]


\medskip

We recall the notation in \Cref{sec:KX} where $\phik$ is an $L$-compatible
$\slt$-triple attached to $X$. Let $\Hslt := \slH$. Then there is a unique
$L'$-compatible $\slt$-triple $\phikp$ attached to $X'$ such that (see \cite[Section~5.2]{GZ})
$$\phikp(\Hslt)
\circ T - T \circ \phik(\Hslt) = T.$$
As an $\slt$-module via $\phikp$,
\[
\bfV' = \bigoplus_{l\geq 0}^{k-1} \bfV`l' \otimes \bC^{l+1}.
\]
We adopt notations in \Cref{sec:KX} to $X' \in
\sO'$, via $\phikp$. In particular, we have $\bfK_X = \bfR_X \ltimes \bfU_X$ and $\bfK'_{X'} =\bfR_{X'}
\ltimes \bfU_{X'}$. Here $\bfU_X$ and $\bfU'_{X'}$ are the unipotent radicals, and $\bfR_X =
\prod_{l=0}^k {\KK`l}$ and $\bfR_{X'} = \prod_{l=0}^{k-1}{\KK`l'}$ are Levi factors
 of $\bfK_X$ and $\bfK'_{X'}$, respectively.
%which are compatible with the $\slt$-module structures on $\bfV$ and $\bfV'$ respectively.


 For each irreducible $\slt$-module $\bC^{l+1}$ fix a nonzero vector
 $v_l\in (\bC^{l+1})^{\Xslt}$.
 % \Ker(\rdd\phi_l(\Xslt))$
      For each $l\geq 0$, the map
 $\nu \mapsto \nu(v_l)$ identifies $\bfV`l = \Hom_{\slt}(\bC^{l+1},\bfV)$
 (resp. $\bfV`l'$) with a subspace $\bfV`l_0$ of $\bfV$ (resp. $\bfV`l'_0$ of
 $\bfV'$). For $1\leq l\leq k$,  $T^\mstar$ induces a vector space isomorphism\footnote{In fact, it
 is a similitude between the two formed spaces and satisfis $L\circ \tau_{l}=\bfii \,\tau_l\circ L'$.}
\[
\xymatrix@C=5em{
\tau_l\colon \bfV`{l-1}' = \bfV`{l-1}'_0 \ar[r]^<>(.5){v \mapsto T^\mstar(v)}
& \bfV`{l}_0 = \bfV`{l}.
}
\]
This results in an isomorphism (which is independent of the choices of $v_l$ and $v_{l-1}$)
\begin{equation}\label{eq:alpha_l}
    \alpha_l :  \KK`{l}   \xrightarrow{\ \ \cong \ \ }  {\KK`{l-1}'} ,\qquad h_l\mapsto (\tau_l)^{-1} \circ h_l \circ\tau_l.
\end{equation}


\medskip

%The following lemma gives a more precise description of the map $\alpha$.

\begin{lem}\label{lem:alpha.e} The homomorphism $\alpha$ maps $\bfR_X$ into $\bfR_{X'}$ and maps $\bfU_X$ into $\bfU_{X'}$. Moreover, the map
  $\alpha|_{\bfR_X}$ is given by
\[
\xymatrix@R=0em{
\flushmr{\alpha|_{\bfR_X}\colon}\prod_{l=0}^{k} {\KK`l} \ar[r]& \prod_{l=0}^{k-1} {\KK`l'},\\
(h_0, h_1, \cdots h_k) \ar@{|->}[r]& (\alpha_1(h_1),\cdots,
\alpha_k(h_k)), \quad \quad h_l\in \KK`l,
}
\]
where $\alpha _l$ ($1\leq l\leq k$) is given in \eqref{eq:alpha_l}.
\end{lem}
\begin{proof}
  Note that $\alpha$ is a surjection since $M^{-1}(X)$ is a $\bfK'$-orbit (see
  \Cref{lem:DS.sh}~\cref{it:DS.G3} or \cite[Lemma~13]{Ohta}).  So $\alpha(\bfU_X)$ is a unipotent normal
  subgroup in $\bfK'_{X'}$ which must be contained in $\bfU_{X'}$. Note that
  (see \cite[Lemma~3.4.4]{CM})
  \[ \bfR_X = \Stab_{\bfK}(\phi_\fkk(\Xslt)) \cap
    \Stab_{\bfK}(\phi_{\fkk}(\Hslt)).
\] By the defintion of $\alpha$ and
$\phi_{\fkk'}$,  we see
\[
\alpha(\bfR_X) \subset \Stab_{\bfK'}(\phi_{\fkk'}(\Xslt)) \cap
\Stab_{\bfK'}(\phi_{\fkk'}(\Hslt)) = \bfR_{X'}.
\]
The rest follows from the discussions before the lemma.
\end{proof}

Let $A$, $A_X$ and $A'_{X'}$ be the component groups of $G$, $\bfK_X$ and $\bfK'_{X'}$
respectively. We will identify $A$ with the component group of $\bfK$ and let
$\chi|_{A_X}$ denote the pullback of $\chi \in \whA$ via the natural map $A_X
\rightarrow A$.
The homomorphism $\alpha\colon \bfK_X \mapsto \bfK'_{X'}$ induces a homomorphism $\alpha\colon A_X \mapsto A'_{X'}$, which further yields a homomorphism
\[
\widehat{A'_{X'}}\rightarrow \widehat{A_{X}}, \quad \rho'\mapsto \rho'\circ\alpha.
\]

\begin{lem}\label{lem:char.surj}
The following map is surjective:
%\begin{equation}
\[
\xymatrix@R=0em{
\widehat{A} \times \widehat{A'_{X'}} \ar@{->>}[r]&  \widehat{A_X},\\
(\chi,\rho') \ar@{|->}[r] & \chi|_{A_X} \cdot (\rho'\circ\alpha).}
%\end{equation}
\]
\end{lem}

\begin{proof} This follows easily from \Cref{lem:alpha.e} and \Cref{lem:char.res}.
\end{proof}

\subsection{Lifting of equivariant vector bundles and admissible orbit
  data}\label{sec:LVB}
Recall from the Introduction the complexification $\wtbfK$ of $\widetilde K$, which is  $\bfK$ except when $G$ is a  real symplectic group.
For a nilpotent $\bfK$-orbit $\sO\in \Nil_{\bfK}(\fpp)$, % is also viewed as an
% $\wtbfK$-orbit.
let $\cK_{\sO}^{\mathbb p}(\wtbfK)$ denote the Grothendieck group of $\mathbb p$-genuine $\wtbfK$-equivariant coherent sheaves on
$\sO$. This is a free abelian group with a free basis consisting of isomorphism classes of  irreducible $\mathbb p$-genuine $\wtbfK$-equivariant algebraic vector
bundles on $\sO$.  Taking the isotropy representation at a point $X\in \sO$ yields an identification
\begin{equation}\label{idenkr}
  \cK_{\sO}^{\mathbb p}(\wtbfK)=\cR^{\mathbb p}(\wtbfK_{X}),
\end{equation}
where the right hand side denotes
  the Grothendieck group  of the category of $\mathbb p$-genuine
  algebraic representations of the stabilizer group $\wtbfK_X$.


  For $\cO\in \Nil_{\bfG}(\fgg)$, let $\cK^{\mathbb p}_{\cO}(\wt{\bfK})$ denote the
  Grothendieck group
  % (with integral coefficients) of the category
  of $\mathbb p$-genuine $\wt{\bfK}$-equivariant coherent sheaves on $\cO\cap \fpp$. Since
  $\cO\cap \fpp$ is the finite union of its $\bfK$-orbits, we have
  \begin{equation}\label{eq:dec.KO}
    \cK_{\cO}^{\mathbb p}(\wt{\bfK})=\bigoplus_{\sO\textrm{ is a $\bfK$-orbit in $\cO\cap \fpp$}} \cK^{\mathbb p}_{\sO}(\wt{\bfK}).
  \end{equation}

  There is a natural partial order $\succeq$ on $\cK_{\cO}^{\mathbb p}(\wt{\bfK})$ and
  $\cK^{\mathbb p}_{\sO}(\wt{\bfK})$: we say $c_1\succeq c_2$ (or $c_2\preceq c_1$) if $c_1-c_2$ is represented by a
  % non-trivial
  $\mathbb p$-genuine $\wtbfK$-equivariant coherent sheaf. The same notation obviously applies  to other Grothendieck groups.



  % \subsubsection{Lift of algebraic vector bundles I: the descent
  % case}\label{sec:lift.AC}
  \subsubsection{An algebraic character}%\label{sec:lift.AC}
  Attached to a rational dual pair $(\bfV,\bfV')$, there is a distinguished character
  $\mktvvp$ of
  $\wtbfK\times\wtbfK'$ arising from the oscillator
  representation, which we shall describe.

  When $G$ is a real symplectic group, let $\cX_{\bfV}$ denote the $\mathbf i$-eigenspace of
  $L$. Then $\wtbfK$ is identified with \[
    \set{(g,c)\in \GL(\cX_{\bfV})\times
      \bC^\times|\det(g) = c^2}.\]
  It has a %genuine
  character
  $ (g,c)\mapsto c$, which is denoted by $\det_{\cX_{\bfV}}^{\half}$. % $\varsigma_{\bfV}$


  When $G$ is a quaternionic orthogonal group, still let
  $\cX_V$ denote the $\mathbf i$-eigenspace of $L$. Then $\wtbfK = \GL(\cX_{\bfV})$. Let $\det_{\cX_{\bfV}}$ denote its determinant character.

  Write $\sign{\bfV'} = (n'^+,n'^-)$. Then the character $\mktvvp|_{\wtbfK}$ is
  given by the following formula: %\footnote{If we denote $\varsigma_\cX$
  % (resp. trivial character) by $\det_{\cX}^{-\half}$ (resp. $\det_{\cX}^0$),
  % then
  % $\mktvvp|_{\wtbfK}$ always equals $\det_{\cX}^{-\frac{n^+-n^-}{2}}$.}
  \[%\begin{equation} \label{ktwvv1}
    \mktvvp|_{\wtbfK}:=\begin{cases}
      \left(\det_{\cX_{\bfV}}^{\half}\right)^{n'^+-n'^-} ,
      %\varsigma_{\bfV}^{n'^+-n'^-} ,
      & \text{if  $G$ is a real symplectic group};\\
      \det_{\cX_{\bfV}}^{\frac{n'^+-n'^-}{2}},&  \text{if  $G$ is a quaternionic orthogonal group}; \medskip \\
      \textrm{the trivial character, } & \text{otherwise}.\\
    \end{cases}
  \]%\end{equation}
  % The role of $\bfV$ and $\bfV'$ is symmetric, and
  The character $\mktvvp|_{\wtbfK'}$ is given by a similar formula
  with $n'^+-n'^-$ replaced by $n^--n^+$, where $(n^+, n^-)=\sign{\bfV}$.

  \trivial[h]{
    The key property of $\mktvvp$ is that $(\mktvvp)^{-2}$ restricted on the
    conncected component of $\wtK\times \wtK'$ is the determinant
    of the $\wtbfK\times \wtbfK'$ action on $\cX$.

    We calculate $\det|_\cX$. When $\dotepsilon = 1$, $\cX  = \Hom(\bfV_1,
    \bfV'^{L',+\bfii}) \oplus \Hom(\bfV_{-1},\bfV'^{L', -\bfii})$ and
    $\det|_{\cX}(g_1,g_{-1}) = (\det g_1)^{-n'^+}(\det g_2)^{-n'^-}$. (Here by
    right we should take $\mktvvp|_\bfK$ be the character of half
    determinant when $G = \rO(p,q)$.
    However, we always fix the splitting $\rO(p,q)\rightarrow
    \widetilde{\rO(p,q)}$ so that the $\rO(p)\times \rO(q)$-action
    on the minimal $K$-type of the Weil
    representation is trivial.

    When $\dotepsilon = -1$, $\cX = \Hom(\bfV_{+\bfii}, \bfV'_0)\oplus
    \Hom(\bfV_{-\bfii}, \bfV'_{-1})$. Now $\det|_{\cX}(g) = \det(g)^{-n'^+ +
      n'^-}$.

    Similarly, when $\dotepsilon' = -1$, $\det|_{\cX}(g') = \det(g)^{n^+ -
      n'^-}$.
    These yields the claim.
  }

  \subsubsection{Lift of algebraic vector bundles}\label{sec:lift.AC}

  In the rest of this section, we assume that $\mathbb p$ is the parity of $\dim \bfV$ if $\epsilon=1$, and  the parity of $\dim \bfV'$ if $\epsilon'=1$.  Then $\mktvvp|_{\wtbfK}$ and
  $\mktvvp|_{\wtbfK'}$ are $\mathbb p$-genuine.

  Suppose $T\in \Xo$ realizes the descent from  $X = \MMP(T)\in \sO\in \Nil_{\bfK}(\fpp)$
  to
  $X' = \MMP'(T)\in \sO'\in \Nil_{\bfK'}(\fpp')$. Let  $\alpha\colon
  {\bfK}_X\rightarrow {\bfK'}_{X'}$ be the homomorphism as in \cref{eq:alpha}.

  Let $\rho'$ be a $\mathbb p$-genuine algebraic representation  of $\wtbfK'_{X'}$. Then  the representation $\mktvvp|_{\wt{\bfK'}_{X'}}\otimes \rho'$ of $\wtbfK'_{X'}$ descends  to a representation of $\bfK'_{X'}$.  Define
  \begin{equation}\label{defn:tlift.rho}
    \dliftv_{T}(\rho'):= \mktvvp|_{\wt{\bfK}_{X}} \otimes (\mktvvp|_{\wt{\bfK'}_{X'}}\otimes \rho')\circ \alpha,
  \end{equation}
  which  is a $\mathbb p$-genuine algebraic representation of $\wt{\bfK}_{X}$.

  Clearly $\dliftv_T$ induces a homomorphism from
  $\cR^{\mathbb p}(\wtbfK'_{X'})$ to $\cR^{\mathbb p}(\wtbfK_X)$.
  In view of \eqref{idenkr}, we thus have a homomorphism
  \begin{equation}\label{defn:DS.ch}
    \xymatrix{
      \dliftv_{\sO',\sO}\colon \cK^{\mathbb p}_{\sO'}(\wt{\bfK}') \ar[r]&
      \cK^{\mathbb p}_{\sO}(\wt{\bfK}).
    }
  \end{equation}
  This is independent of the choice of $T$.



  Suppose $\cO\in\Nil_{\bfG}(\fgg)$ and $\cO'=\DD(\CO)\in \Nil_{\bfG'}(\fgg')$. Using decomposition \cref{eq:dec.KO}, we define a homomorphism
  \begin{equation}\label{eq:DS.chc}
    \xymatrix{
      \dliftv_{\cO',\cO} := \displaystyle\sum_{\substack{\sO\subset \cO\cap \fpp\\ \sO' =
          \DD(\sO) \subset \fpp'}}\dliftv_{\sO',\sO}\colon \cK^{\mathbb p}_{\cO'}(\wt{\bfK}') \ar[r]&
      \cK^{\mathbb p}_{\cO}(\wt{\bfK})
    }
  \end{equation}
  where the summation
  is over all pairs  $(\sO, \sO')$ such that $\sO'\subset \fpp'$ is the descent of
  $\sO\subset \cO\cap \fpp$.

  % \begin{remark}
  %   The explicit form of $\KTW|_{\wtK'_{X'}}$ is not important for us. We
  %   refer the interested reader to \Cref{sec:KTW} for the explicit formula.
  % \end{remark}

  % Using \Cref{defn:tlift.rho}, we have the following estimate of associated
  % character by


  % Likewise let $\CK_{\KV}(\sO)$ denote the Grothendieck group of the category
  % of $\KV$-equivariant algebraic vector bundles on $\sO$.  Then
  % \begin{equation}\label{decomk}
  %   \CK_{\widetilde \KV}(\sO)=\left\{ \begin{array}{ll}
  %       \CK_{ \KV}(\sO)\oplus \CK_{\widetilde \KV}^{\mathrm{gen}}(\sO), & \textrm{if } (\epsilon, \dot \epsilon)=(-1,-1);\medskip\\
  %       \CK_{ \KV}(\sO),&  \textrm{otherwise, } \\
  %     \end{array}
  %   \right.
  % \end{equation}
  % where $\CK^{\mathrm{gen}}_{\widetilde \KV}(\sO)$ denotes the Grothendieck
  % group of the category of genuine $\widetilde \KV$-equivariant algebraic
  % vector bundles on $\sO$.  Here and as usual, ``genuine" means that the
  % non-trivial element in the kernel of the covering map
  % $\widetilde \KV\rightarrow \KV$ acts through the scalar multiplication by
  % $-1$.


  % \subsubsection{Lift of algebraic vector bundles II: the generalized descent
  % case}
  \medskip
  Now suppose $\sO'=\gDD_{\bfV,\bfV'}(\sO) \in \Nil_{\bfK'}(\fpp')$ is the
  generalized decent of $\sO\in \Nil_{\bfK}(\fpp)$. From the discussion in
  \Cref{def:GD}, there is an $(\epsilon',\dotepsilon')$-space decomposition
  $\bfV' = \bfV'_1\oplus \bfV'_2$ and an element
  \[
  T\in
  \CX_1^\circ:=\{w\in\Hom(\bfV,\bfV'_1)\mid w\textrm{ is surjective}\}\cap \CX\subseteq \CX^{\mathrm{gen}}
  \]
   such that
  $\sO'_1:= \bfK'_1 \cdot X' \in \Nil_{\bfK'_1}(\fpp'_1)$ is the descent of
  $\sO$. Here $X' := M'(T)\in \sO'$,
  $\bfK'_i:= \bfG_{\bfV'_i}^{L'}$ is a
  subgroup of $\bfK'$ for $i=1,2$, and $\fpp'_1 := \fpp_{\bfV'_1}$.


  Let $X := M(T)\in \sO$ and
  \begin{equation}\label{eq:def.alpha1}
    \alpha_1\colon \bfK_X \rightarrow \bfK'_{1,X'}
  \end{equation}
  be
  the (surjective) homomorphism defined in
  \cref{eq:alpha} with respect to the descent from $\sO$ to $\sO'_1$.
  Then the stabilizer $\bfS_T := \Stab_{\bfK\times
    \bfK'}(T)$ of $T$ is given by
  \begin{equation}\label{labst}
    \bfS_T = \Set{(k,\alpha_1(k)k'_2)\in \bfK\times \bfK' |k\in \bfK \text{ and } k'_2\in \bfK'_2}.
  \end{equation}
  For a $\mathbb p$-genuine algebraic representation $\rho'$ of $\wtbfK'_{X'}$, define a
  representation
  \begin{equation}\label{defn:glift.rho}
    \gdlift_{T}(\rho'):= \mktvvp|_{\wt{\bfK}_{X}} \otimes \left((\mktvvp|_{\wt{\bfK}'_{X'}}\otimes
      \rho')^{\bfK'_2}\right)\circ \alpha_1.
  \end{equation}
  Clearly, \cref{defn:glift.rho} is a generalization of \cref{defn:tlift.rho}, as ${\bfK'_2}$ is the trivial group in the latter case.
  As in the descent case, $\gdlift_{T}$ induces a homomorphism
  \[
    \dliftv_{\sO',\sO}\colon \cK^{\mathbb p}_{\wt{\bfK}'}(\sO')
    \rightarrow \cK^{\mathbb p}_{\wt{\bfK}}(\sO).
  \]
  Furthermore, \cref{eq:DS.chc} is extended to the generalized descent case: for every pair $(\cO,\cO')\in \Nil_{\bfG}(\fgg)\times  \Nil_{\bfG'}(\fgg')$ with $\cO' =\gDD_{\bfV,\bfV'}(\cO)$, we define a homomorhism
  \[
    \dliftv_{\cO',\cO} := \sum_{\substack{\sO \subset \cO\cap \fpp,\\
        \sO' = \gDD_{\bfV,\bfV'}(\sO)\subset \fpp'}}\dliftv_{\sO',\sO}\colon \cK^{\mathbb p}_{\wt{\bfK}'}(\cO')
    \rightarrow \cK^{\mathbb p}_{\wt{\bfK}}(\cO).
  \]




  \subsubsection{Lift of admissible orbit data}\label{sec:aod}
  % \subsubsection{Admissible orbit data}

    % The symbol ``$\widetilde{\phantom A}$" will be used in similar situations
  % without further explanation.

  Now let $\sO$ be a $\bfK$-orbit in $\CO\cap \p$, where
  $\cO\in \Nil_{\bfG}^{\mathbb p}(\fgg)$. Let $X\in \sO$.   The component group  $A_X:=\bfK_X/\bfK_X^\circ$  is an
  elementary abelian $2$-group by \Cref{sec:KX}.  Let $\fkk_X$ be the Lie algebra of $\bfK_X$. Denote by $\bbfK_X\rightarrow \bfK_X$ the covering map induced by the covering $\bbfK\rightarrow \bfK$.



  We make the following definition.

\begin{defn}[{\cite[Definition~7.13]{Vo89}}]\label{def:admD}
  Let $\gamma_X$ denote the one-dimensional $\bfK_X$-module
  $\bigwedge^{\rmtop} \fkk_X$ and let $\rdd\gamma_X$ be its
  differential. \footnote{$\rdd\gamma_X$ is the same as $\rdd\gamma_\fkk$ in
    \cite[Theorem~7.11]{Vo89} since $\fkk$ is reductive.}  An irreducible
  representation $\rho$ of $\bbfK_X$ is called \emph{admissible} if
  \begin{enumT}
  \item its differential $\rdd\rho$ is isomorphic to a multiple of
    $\half \rdd\gamma_X$, equivalently,
    \[
      \rho(\exp(x)) = \gamma_X(\exp(x/2))\cdot \id, \quad\textrm{for all }x\in
      \fkk_X\text{, and}
    \]
  \item it is $\bpp$-genuine.
  \end{enumT}

  Let $\Phi_X$ denote the set of all isomorphism classes of admissible
  irreducible representations of $\bbfK_X$.
  % Note that $\Phi_X$ may be empty.
  % \begin{defn}[{\cite[Definition~7.13]{Vo89}}]
  %   If $\Phi_X\neq \emptyset$, we say that the orbit $\sO:=\bfK\cdot X$ is
  %   \emph{admissible}.
\end{defn}

\Cref{def:admD} is obviously consistent with \Cref{defaod}, since a
representation $\rho\in \Phi_X$ determines an admissible orbit datum
$\cE\in \cKaod_\sO(\wtbfK)$, where $\cE$ is a $\bbfK$-equivariant algebraic
vector bundle on $\sO$ whose isotropy representation $\cE_X$ at $X$ is
isomorphic to $\rho$.  We therefore have an identification
\begin{equation}\label{idenip}
  \cKaod_\sO(\wtbfK)=\Phi_X.
\end{equation}
\trivial[h]{
  % \begin{obs}
  Suppose $\sO$ is an admissible orbit and $X\in \sO$.
  When $\bfG$ is an orthogonal group or symplectic group then
  \begin{enumT}
  \item $\Phi_X$ is a set of one-dimensional representations of $\wtK_X$ and
  \item
  \end{enumT}
  When $\bfG$ is a general linear group. Then $\Phi_X$ is a singleton, since $K_X$
  is always connected.
  % \end{obs}

  % \begin{obs}\label{obs:admchar}
  Suppose $\rho_0 \in \Phi_X$ is a character. Then $\Phi_X = \set{\chi \otimes \rho_0|\chi\in
    \whAX}$. Furthermore, if $A_X$
  is abelian, then $\Phi_X$ consists of
  characters and the tensor product action of the character
  group $\whAX$ on $\Phi_X$ is free and transitive, i.e. $\Phi_X$ is an $\whAX$
  torsor.
  % \end{obs}
  % \begin{proof}
  This follows by Mackey theory. Suppose $\rho\in \Phi_X$, then
  $\Hom_{\nu \bbfK_X^\circ}(\rho, \rho_0)\neq 0$.
  Therefore, $\rho$ is a sub-representation of
  \[
    \Ind_{\nu \bbfK_X^\circ}^{\bbfK_X}(\rho_0|_{\nu \bbfK_X^\circ}) = \Ind_{\nu \bbfK_X^\circ}^{\bbfK_X}
    \bfone \otimes \rho_0 = \Ind_{\bfK_X^\circ}^{\bfK_X} \bfone \otimes \rho_0.
  \]
  Hence $\rho \cong \chi \otimes \rho_0$ for some
  character  $\chi\in \widehat{A_X}$.
  The rest is clear.
  % \end{proof}
}
From the structure of $\bfK_X$ in \Cref{lem:KX1} and \Cref{lem:char.res}, it is
easy to see  that $\Phi_X$ consists of one-dimensional representations.
\medskip

\begin{lem}\label{lem:Kaod}
  The tensor product yields a simply transitive action of the character group
  $\widehat{A_X}$ on the set $\Phi_X$.
\end{lem}
\begin{proof}
  It is clear that we only need to show that $\Phi_X$ is nonempty. Consider a
  rational dual pair $(\bfV,\bfV')$ such that
  $\sO' = \DD(\sO) \in \Nil_{\bfK'}(\fpp')$ is the descent of
  $\sO\in \Nil_{\bfK}(\fpp)$. Suppose that $T\in \Xo$ realizes the descent from
  $X$ to $X'\in \sO'$. The proof of \cite[Proposition~6.1]{LM} shows that if
  $\rho'\in \Phi_{X'}$ is admissible, then $\dliftv_T(\rho')$ is
  admissible.
  Therefore, we may do reductions and eventually reduce the problem
  to the case when $\bfV$ is the zero space.  It is clear that $\Phi_X$ is a
  singleton in this case.
  \trivial[h]{
    Note that, the role of $V$ and $V'$ are switched in that proof.
    The short exact sequence for the proof of \cite[(23)]{LM} still
    holds since the moment map $M$ is smooth at
    $T$ (see the property of principle straum in \cite[page 217]{PV} or
    \Cref{sec:Sdes}).
  }
\end{proof}

\begin{remark}
By \Cref{lem:Kaod}, the set $\cKaod_\sO(\wtbfK)$ of admissible orbit data over
$\sO$ has $2^r$ elements, where $r$ is the number of orthogonal groups appearing in the decomposition
of $\bfR_X$ in \Cref{lem:KX1}.
\end{remark}

As in the proof of Lemma \ref{lem:Kaod},
the homomorphism \eqref{defn:DS.ch} restricts to a map
\begin{equation}\label{eq:l.adm}
  \dliftv_{\sO',\sO}\colon \cKaod_{\sO'}(\wtbfK') \rightarrow \cKaod_{\sO}(\wtbfK).
\end{equation}

\medskip

\begin{lem}\label{lem:admchar.surj}
  Let $A$ be the component group of $G$ which is identified with the component
  group of $\bfK$.  Then the following map is
  surjective:\footnote{%The component group $A$ of $G$ is
    % identified with the component group of $\bfK$.
    Under the identification \eqref{idenip}, the tensor product of a character
    $\chi\in \whA$ on $\cKaod_{\sO}(\wtbfK)$ is identified with the tensor
    product of $\chi|_{A_X}$ with the isotropy representation.}
  \[
    \xymatrix@C=4em@R=0em{
      \whA \times \cKaod_{\sO'}(\wtbfK')\ar[r] & \cKaod_{\sO}(\wtbfK),\\
      (\chi,\cE') \ar@{|->}[r] & \chi \otimes \dliftv_{\sO',\sO}(\cE').  }
  \]
  In particular, $\cKaod_{\sO}(\wtbfK)$ is a singleton if $G$ is a quaternionic
  group.
\end{lem}
\begin{proof}
  This follows from \Cref{lem:char.surj} and \Cref{lem:Kaod}.
\end{proof}




\trivial[h]{
  \begin{proof}
    \begin{enumPF}
    \item follows from the fact that \begin{enumIL}
      \item $\gamma_X$ is always trivial since $R_X$ is semisimple and
      \item $\bfK_X$ is connected.
      \end{enumIL}
    \item Recall that $\bfK_X = R_X \ltimes U_X$, where
      $R_X=\prod_{l} {^{l}{\bfK}}$. Note that $\gamma_X|_{U_X}$ is trivial and
      \[
        \gamma_X |_{R_X} = \bigboxtimes_{l} {\det}_{^{l}{\bfK}}^{n_l}
      \]
      where $\det_{^{l}{\bfK}}$ is the restriction of the determinant character
      of $\GL(\bfV`l)$ on $^{l}{\bfK}$, and $n_l$ is a certain non-negative
      integer.

      % Let $R_{-}=\prod_{l \text{ odd}} {^{l}{\bfK}}$ and
      % $R_+ = \prod_{l \text{ even}}{^{l}{\bfK}}$.
      Suppose $G$ is a symplectic group, then $^{l}{\bfK}$ is a product of
      orthogonal groups for $l$ odd, and a general linear group for $l$ even.
      The preimage $\breve{^{l}{\bfK}}$ of $^{l}{\bfK}$ is thus a genuine double
      covering for $l$ even (see \Cref{eq:def.brK}).

      If the parity $\bpp$ is even, then $\Phi_X$ is non-empty if and only if
      $n_l$ is even for all $l$ even. In that case, one may choose
      \begin{equation}\label{eq:rho0.sp}
        \rho_0 =
        \bigboxtimes_{l\text{ odd}} \bfone \boxtimes \bigboxtimes_{l\text{
            even}}{\det}_{^{l}{\bfK}}^{n_l/2}
      \end{equation}
      and apply %\cref{obs:admchar}.

      If the parity $\bpp$ is odd, then $\Phi_X$ is non-empty if and only if
      $n_l$ is odd for all $l$ even. One can choose $\rho_0$ as in
      \eqref{eq:rho0.sp} by replacing $\det_{^{l}{\bfK}}^{n_l/2}$ with the
      genuine character
      $\det_{\breve{^{l}{\bfK}}}^{n_l/2}\colon (g,c) \mapsto c^{n_l}$ of
      $\breve{^{l}{\bfK}}$.

      When $G$ is an orthogonal group, the argument is similar. We leave it to
      the reader.
    \end{enumPF}
  \end{proof}
}







\message{ !name(sunip_new.tex) !offset(7722) }

\end{document}



%%% Local Variables:
%%% coding: utf-8
%%% mode: latex
%%% TeX-engine: xetex
%%% ispell-local-dictionary: "en_US"
%%% End:



