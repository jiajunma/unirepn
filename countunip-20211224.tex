% !TeX program = xelatex
\documentclass[12pt,a4paper]{amsart}
\usepackage[margin=2.5cm,marginpar=2cm]{geometry}

\usepackage[bookmarksopen,bookmarksdepth=2,hidelinks,colorlinks=false]{hyperref}
\usepackage[nameinlink]{cleveref}

% \usepackage[color]{showkeys}
% \makeatletter
%   \SK@def\Cref#1{\SK@\SK@@ref{#1}\SK@Cref{#1}}%
% \makeatother
%% FONTS

\usepackage{amssymb}
%\usepackage{amsmath}
\usepackage{mathrsfs}
\usepackage{mathtools}
%\usepackage{amsrefs}
%\usepackage{mathbbol,mathabx}
\usepackage{amsthm}
\usepackage{graphicx}
\usepackage{braket}
%\usepackage[pointedenum]{paralist}
%\usepackage{paralist}
\usepackage{amscd}

\usepackage[alphabetic]{amsrefs}

\usepackage[all,cmtip]{xy}
\usepackage{rotating}
\usepackage{leftidx}
%\usepackage{arydshln}

%\DeclareSymbolFont{bbold}{U}{bbold}{m}{n}
%\DeclareSymbolFontAlphabet{\mathbbold}{bbold}


%\usepackage[dvipdfx,rgb,table]{xcolor}
\usepackage[rgb,table]{xcolor}
%\usepackage{mathrsfs}

\setcounter{tocdepth}{1}
\setcounter{secnumdepth}{2}

%\usepackage[abbrev,shortalphabetic]{amsrefs}


\usepackage[normalem]{ulem}

% circled number
\usepackage{pifont}
\makeatletter
\newcommand*{\circnuma}[1]{%
  \ifnum#1<1 %
    \@ctrerr
  \else
    \ifnum#1>20 %
      \@ctrerr
    \else
      \mbox{\ding{\numexpr 171+(#1)\relax}}%
     \fi
  \fi
}
\makeatother

\usepackage[centertableaux]{ytableau}


% Ytableau tweak
\makeatletter
\pgfkeys{/ytableau/options,
  noframe/.default = false,
  noframe/.is choice,
  noframe/true/.code = {%
    \global\let\vrule@YT=\vrule@none@YT
    \global\let\hrule@YT=\hrule@none@YT
  },
  noframe/false/.code = {%
    \global\let\vrule@YT=\vrule@normal@YT
    \global\let\hrule@YT=\hrule@normal@YT
  },
  noframe/on/.style = {noframe/true},
  noframe/off/.style = {noframe/false},
}

\def\hrule@enon@YT{%
  \hrule width  \dimexpr \boxdim@YT + \fboxrule *2 \relax
  height 0pt
}
\def\vrule@enon@YT{%
  \vrule height \dimexpr  \boxdim@YT + \fboxrule\relax
     width \fboxrule
}

\def\enon{\omit\enon@YT}
\newcommand{\enon@YT}[2][clear]{%
  \def\thisboxcolor@YT{#1}%
  \let\hrule@YT=\hrule@enon@YT
  \let\vrule@YT=\vrule@enon@YT
  \startbox@@YT#2\endbox@YT
  \nullfont
}

\makeatother
%\ytableausetup{noframe=on,smalltableaux}
\ytableausetup{noframe=off,boxsize=1.3em}
\let\ytb=\ytableaushort

\newcommand{\tytb}[1]{{\tiny\ytb{#1}}}


%\usepackage[mathlines,pagewise]{lineno}
%\linenumbers

\usepackage{enumitem}
%% Enumitem
\newlist{enumC}{enumerate}{1} % Conditions in Lemma/Theorem/Prop
\setlist[enumC,1]{label=(\alph*),wide,ref=(\alph*)}
\crefname{enumCi}{condition}{conditions}
\Crefname{enumCi}{Condition}{Conditions}
\newlist{enumT}{enumerate}{3} % "Theorem"=conclusions in Lemma/Theorem/Prop
\setlist[enumT]{label=(\roman*),wide}
\setlist[enumT,1]{label=(\roman*),wide}
\setlist[enumT,2]{label=(\alph*),ref ={(\roman{enumTi}.\alph*)}}
\setlist[enumT,3]{label=(\arabic*), ref ={(\roman{enumTi}.\alph{enumTii}.\alph*)}}
\crefname{enumTi}{}{}
\Crefname{enumTi}{Item}{Items}
\crefname{enumTii}{}{}
\Crefname{enumTii}{Item}{Items}
\crefname{enumTiii}{}{}
\Crefname{enumTiii}{Item}{Items}
\newlist{enumPF}{enumerate}{3}
\setlist[enumPF]{label=(\alph*),wide}
\setlist[enumPF,1]{label=(\roman*),wide}
\setlist[enumPF,2]{label=(\alph*)}
\setlist[enumPF,3]{label=\arabic*).}
\newlist{enumS}{enumerate}{3} % Statement outside Lemma/Theorem/Prop
\setlist[enumS]{label=\roman*)}
\setlist[enumS,1]{label=\roman*)}
\setlist[enumS,2]{label=\alph*)}
\setlist[enumS,3]{label=\arabic*.}
\newlist{enumI}{enumerate}{3} % items
\setlist[enumI,1]{label=\roman*),leftmargin=*}
\setlist[enumI,2]{label=\alph*), leftmargin=*}
\setlist[enumI,3]{label=\arabic*), leftmargin=*}
\newlist{enumIL}{enumerate*}{1} % inline enum
\setlist*[enumIL]{label=\roman*)}
\newlist{enumR}{enumerate}{1} % remarks
\setlist[enumR]{label=\arabic*.,wide,labelwidth=!, labelindent=0pt}
\crefname{enumRi}{remark}{remarks}

\crefname{equation}{}{}
\Crefname{equation}{Equation}{Equations}
\Crefname{lem}{Lemma}{Lemma}
\Crefname{thm}{Theorem}{Theorem}

\newlist{des}{description}{1}
\setlist[des]{font=\sffamily\bfseries}

% editing macros.
\blendcolors{!80!black}
\long\def\okay#1{\ifcsname highlightokay\endcsname
{\color{red} #1}
\else
{#1}
\fi
}
\long\def\editc#1{{\color{red} #1}}
\long\def\mjj#1{{{\color{blue}#1}}}
\long\def\mjjr#1{{\color{red} (#1)}}
\long\def\mjjd#1#2{{\color{blue} #1 \sout{#2}}}
\def\mjjb{\color{blue}}
\def\mjje{\color{black}}
\def\mjjcb{\color{green!50!black}}
\def\mjjce{\color{black}}

\long\def\sun#1{{{\color{cyan}#1}}}
\long\def\sund#1#2{{\color{cyan}#1  \sout{#2}}}
\long\def\mv#1{{{\color{red} {\bf move to a proper place:} #1}}}
\long\def\delete#1{}

%\reversemarginpar
\newcommand{\lokec}[1]{\marginpar{\color{blue}\tiny #1 \mbox{--loke}}}
\newcommand{\mjjc}[1]{\marginpar{\color{green}\tiny #1 \mbox{--ma}}}

\newcommand{\trivial}[2][]{\if\relax\detokenize{#1}\relax
  {%\hfill\break
   % \begin{minipage}{\textwidth}
      \color{orange} \vspace{0em} $[$  #2 $]$
  %\end{minipage}
  %\break
      \color{black}
  }
  \else
\ifx#1h
\ifcsname showtrivial\endcsname
{%\hfill\break
 % \begin{minipage}{\textwidth}
    \color{orange} \vspace{0em}  $[$ #2 $]$
%\end{minipage}
%\break
    \color{black}
}
\fi
\else {\red Wrong argument!} \fi
\fi
}

\newcommand{\byhide}[2][]{\if\relax\detokenize{#1}\relax
{\color{orange} \vspace{0em} Plan to delete:  #2}
\else
\ifx#1h\relax\fi
\fi
}



\newcommand{\Rank}{\mathrm{rk}}
\newcommand{\cqq}{\mathscr{D}}
\newcommand{\rsym}{\mathrm{sym}}
\newcommand{\rskew}{\mathrm{skew}}
\newcommand{\fraksp}{\mathfrak{sp}}
\newcommand{\frakso}{\mathfrak{so}}
\newcommand{\frakm}{\mathfrak{m}}
\newcommand{\frakp}{\mathfrak{p}}
\newcommand{\pr}{\mathrm{pr}}
\newcommand{\rhopst}{\rho'^*}
\newcommand{\Rad}{\mathrm{Rad}}
\newcommand{\Res}{\mathrm{Res}}
\newcommand{\Hol}{\mathrm{Hol}}
\newcommand{\AC}{\mathrm{AC}}
%\newcommand{\AS}{\mathrm{AS}}
\newcommand{\WF}{\mathrm{WF}}
\newcommand{\AV}{\mathrm{AV}}
\newcommand{\AVC}{\mathrm{AV}_\bC}
\newcommand{\VC}{\mathrm{V}_\bC}
\newcommand{\bfv}{\mathbf{v}}
\newcommand{\depth}{\mathrm{depth}}
\newcommand{\wtM}{\widetilde{M}}
\newcommand{\wtMone}{{\widetilde{M}^{(1,1)}}}

\newcommand{\nullpp}{N(\fpp'^*)}
\newcommand{\nullp}{N(\fpp^*)}
%\newcommand{\Aut}{\mathrm{Aut}}

\def\mstar{{\medstar}}


\newcommand{\bfone}{\mathbf{1}}
\newcommand{\piSigma}{\pi_\Sigma}
\newcommand{\piSigmap}{\pi'_\Sigma}


\newcommand{\sfVprime}{\mathsf{V}^\prime}
\newcommand{\sfVdprime}{\mathsf{V}^{\prime \prime}}
\newcommand{\gminusone}{\mathfrak{g}_{-\frac{1}{m}}}

\newcommand{\eva}{\mathrm{eva}}

% \newcommand\iso{\xrightarrow{
%    \,\smash{\raisebox{-0.65ex}{\ensuremath{\scriptstyle\sim}}}\,}}

\def\Ueven{{U_{\rm{even}}}}
\def\Uodd{{U_{\rm{odd}}}}
\def\ttau{\tilde{\tau}}
\def\Wcp{W}
\def\Kur{{K^{\mathrm{u}}}}

\def\Im{\operatorname{Im}}

\providecommand{\bcN}{{\overline{\cN}}}



\makeatletter

\def\gen#1{\left\langle
    #1
      \right\rangle}
\makeatother

\makeatletter
\def\inn#1#2{\left\langle
      \def\ta{#1}\def\tb{#2}
      \ifx\ta\@empty{\;} \else {\ta}\fi ,
      \ifx\tb\@empty{\;} \else {\tb}\fi
      \right\rangle}
\def\binn#1#2{\left\lAngle
      \def\ta{#1}\def\tb{#2}
      \ifx\ta\@empty{\;} \else {\ta}\fi ,
      \ifx\tb\@empty{\;} \else {\tb}\fi
      \right\rAngle}
\makeatother

\makeatletter
\def\binn#1#2{\overline{\inn{#1}{#2}}}
\makeatother


\def\innwi#1#2{\inn{#1}{#2}_{W_i}}
\def\innw#1#2{\inn{#1}{#2}_{\bfW}}
\def\innv#1#2{\inn{#1}{#2}_{\bfV}}
\def\innbfv#1#2{\inn{#1}{#2}_{\bfV}}
\def\innvi#1#2{\inn{#1}{#2}_{V_i}}
\def\innvp#1#2{\inn{#1}{#2}_{\bfV'}}
\def\innp#1#2{\inn{#1}{#2}'}

% choose one of then
\def\simrightarrow{\iso}
\def\surj{\twoheadrightarrow}
%\def\simrightarrow{\xrightarrow{\sim}}

\newcommand\iso{\xrightarrow{
   \,\smash{\raisebox{-0.65ex}{\ensuremath{\scriptstyle\sim}}}\,}}

\newcommand\riso{\xleftarrow{
   \,\smash{\raisebox{-0.65ex}{\ensuremath{\scriptstyle\sim}}}\,}}









\usepackage{xparse}
\def\usecsname#1{\csname #1\endcsname}
\def\useLetter#1{#1}
\def\usedbletter#1{#1#1}

% \def\useCSf#1{\csname f#1\endcsname}

\ExplSyntaxOn

\def\mydefcirc#1#2#3{\expandafter\def\csname
  circ#3{#1}\endcsname{{}^\circ {#2{#1}}}}
\def\mydefvec#1#2#3{\expandafter\def\csname
  vec#3{#1}\endcsname{\vec{#2{#1}}}}
\def\mydefdot#1#2#3{\expandafter\def\csname
  dot#3{#1}\endcsname{\dot{#2{#1}}}}

\def\mydefacute#1#2#3{\expandafter\def\csname a#3{#1}\endcsname{\acute{#2{#1}}}}
\def\mydefbr#1#2#3{\expandafter\def\csname br#3{#1}\endcsname{\breve{#2{#1}}}}
\def\mydefbar#1#2#3{\expandafter\def\csname bar#3{#1}\endcsname{\bar{#2{#1}}}}
\def\mydefhat#1#2#3{\expandafter\def\csname hat#3{#1}\endcsname{\hat{#2{#1}}}}
\def\mydefwh#1#2#3{\expandafter\def\csname wh#3{#1}\endcsname{\widehat{#2{#1}}}}
\def\mydeft#1#2#3{\expandafter\def\csname t#3{#1}\endcsname{\tilde{#2{#1}}}}
\def\mydefu#1#2#3{\expandafter\def\csname u#3{#1}\endcsname{\underline{#2{#1}}}}
\def\mydefr#1#2#3{\expandafter\def\csname r#3{#1}\endcsname{\mathrm{#2{#1}}}}
\def\mydefb#1#2#3{\expandafter\def\csname b#3{#1}\endcsname{\mathbb{#2{#1}}}}
\def\mydefwt#1#2#3{\expandafter\def\csname wt#3{#1}\endcsname{\widetilde{#2{#1}}}}
%\def\mydeff#1#2#3{\expandafter\def\csname f#3{#1}\endcsname{\mathfrak{#2{#1}}}}
\def\mydefbf#1#2#3{\expandafter\def\csname bf#3{#1}\endcsname{\mathbf{#2{#1}}}}
\def\mydefc#1#2#3{\expandafter\def\csname c#3{#1}\endcsname{\mathcal{#2{#1}}}}
\def\mydefsf#1#2#3{\expandafter\def\csname sf#3{#1}\endcsname{\mathsf{#2{#1}}}}
\def\mydefs#1#2#3{\expandafter\def\csname s#3{#1}\endcsname{\mathscr{#2{#1}}}}
\def\mydefcks#1#2#3{\expandafter\def\csname cks#3{#1}\endcsname{{\check{
        \csname s#2{#1}\endcsname}}}}
\def\mydefckc#1#2#3{\expandafter\def\csname ckc#3{#1}\endcsname{{\check{
      \csname c#2{#1}\endcsname}}}}
\def\mydefck#1#2#3{\expandafter\def\csname ck#3{#1}\endcsname{{\check{#2{#1}}}}}

\cs_new:Npn \mydeff #1#2#3 {\cs_new:cpn {f#3{#1}} {\mathfrak{#2{#1}}}}

\cs_new:Npn \doGreek #1
{
  \clist_map_inline:nn {alpha,beta,gamma,Gamma,delta,Delta,epsilon,varepsilon,zeta,eta,theta,vartheta,Theta,iota,kappa,lambda,Lambda,mu,nu,xi,Xi,pi,Pi,rho,sigma,varsigma,Sigma,tau,upsilon,Upsilon,phi,varphi,Phi,chi,psi,Psi,omega,Omega,tG} {#1{##1}{\usecsname}{\useLetter}}
}

\cs_new:Npn \doSymbols #1
{
  \clist_map_inline:nn {otimes,boxtimes} {#1{##1}{\usecsname}{\useLetter}}
}

\cs_new:Npn \doAtZ #1
{
  \clist_map_inline:nn {A,B,C,D,E,F,G,H,I,J,K,L,M,N,O,P,Q,R,S,T,U,V,W,X,Y,Z} {#1{##1}{\useLetter}{\useLetter}}
}

\cs_new:Npn \doatz #1
{
  \clist_map_inline:nn {a,b,c,d,e,f,g,h,i,j,k,l,m,n,o,p,q,r,s,t,u,v,w,x,y,z} {#1{##1}{\useLetter}{\usedbletter}}
}

\cs_new:Npn \doallAtZ
{
\clist_map_inline:nn {mydefsf,mydeft,mydefu,mydefwh,mydefhat,mydefr,mydefwt,mydeff,mydefb,mydefbf,mydefc,mydefs,mydefck,mydefcks,mydefckc,mydefbar,mydefvec,mydefcirc,mydefdot,mydefbr,mydefacute} {\doAtZ{\csname ##1\endcsname}}
}

\cs_new:Npn \doallatz
{
\clist_map_inline:nn {mydefsf,mydeft,mydefu,mydefwh,mydefhat,mydefr,mydefwt,mydeff,mydefb,mydefbf,mydefc,mydefs,mydefck,mydefbar,mydefvec,mydefdot,mydefbr,mydefacute} {\doatz{\csname ##1\endcsname}}
}

\cs_new:Npn \doallGreek
{
\clist_map_inline:nn {mydefck,mydefwt,mydeft,mydefwh,mydefbar,mydefu,mydefvec,mydefcirc,mydefdot,mydefbr,mydefacute} {\doGreek{\csname ##1\endcsname}}
}

\cs_new:Npn \doallSymbols
{
\clist_map_inline:nn {mydefck,mydefwt,mydeft,mydefwh,mydefbar,mydefu,mydefvec,mydefcirc,mydefdot} {\doSymbols{\csname ##1\endcsname}}
}



\cs_new:Npn \doGroups #1
{
  \clist_map_inline:nn {GL,Sp,rO,rU,fgl,fsp,foo,fuu,fkk,fuu,ufkk,uK} {#1{##1}{\usecsname}{\useLetter}}
}

\cs_new:Npn \doallGroups
{
\clist_map_inline:nn {mydeft,mydefu,mydefwh,mydefhat,mydefwt,mydefck,mydefbar} {\doGroups{\csname ##1\endcsname}}
}


\cs_new:Npn \decsyms #1
{
\clist_map_inline:nn {#1} {\expandafter\DeclareMathOperator\csname ##1\endcsname{##1}}
}

\decsyms{Mp,id,SL,Sp,SU,SO,GO,GSO,GU,GSp,PGL,Pic,Lie,Mat,Ker,Hom,Ext,Ind,reg,res,inv,Isom,Det,Tr,Norm,Sym,Span,Stab,Spec,PGSp,PSL,tr,Ad,Br,Ch,Cent,End,Aut,Dvi,Frob,Gal,GL,Gr,DO,ur,vol,ab,Nil,Supp,rank,Sign}

\def\abs#1{\left|{#1}\right|}
\def\norm#1{{\left\|{#1}\right\|}}


% \NewDocumentCommand\inn{m m}{
% \left\langle
% \IfValueTF{#1}{#1}{000}
% ,
% \IfValueTF{#2}{#2}{000}
% \right\rangle
% }
\NewDocumentCommand\cent{o m }{
  \IfValueTF{#1}{
    \mathop{Z}_{#1}{(#2)}}
  {\mathop{Z}{(#2)}}
}


\def\fsl{\mathfrak{sl}}
\def\fsp{\mathfrak{sp}}


%\def\cent#1#2{{\mathrm{Z}_{#1}({#2})}}


\doallAtZ
\doallatz
\doallGreek
\doallGroups
\doallSymbols
\ExplSyntaxOff


% \usepackage{geometry,amsthm,graphics,tabularx,amssymb,shapepar}
% \usepackage{amscd}
% \usepackage{mathrsfs}


\usepackage{diagbox}
% Update the information and uncomment if AMS is not the copyright
% holder.
%\copyrightinfo{2006}{American Mathematical Society}
%\usepackage{nicematrix}
\usepackage{arydshln}

\usepackage{tikz}
\usetikzlibrary{matrix,arrows,positioning,cd,backgrounds}
\usetikzlibrary{decorations.pathmorphing,decorations.pathreplacing}

\usepackage{upgreek}

\usepackage{listings}
\lstset{
    basicstyle=\ttfamily\tiny,
    keywordstyle=\color{black},
    commentstyle=\color{white}, % white comments
    stringstyle=\ttfamily, % typewriter type for strings
    showstringspaces=false,
    breaklines=true,
    emph={Output},emphstyle=\color{blue},
}

\newcommand{\BA}{{\mathbb{A}}}
%\newcommand{\BB}{{\mathbb {B}}}
\newcommand{\BC}{{\mathbb {C}}}
\newcommand{\BD}{{\mathbb {D}}}
\newcommand{\BE}{{\mathbb {E}}}
\newcommand{\BF}{{\mathbb {F}}}
\newcommand{\BG}{{\mathbb {G}}}
\newcommand{\BH}{{\mathbb {H}}}
\newcommand{\BI}{{\mathbb {I}}}
\newcommand{\BJ}{{\mathbb {J}}}
\newcommand{\BK}{{\mathbb {U}}}
\newcommand{\BL}{{\mathbb {L}}}
\newcommand{\BM}{{\mathbb {M}}}
\newcommand{\BN}{{\mathbb {N}}}
\newcommand{\BO}{{\mathbb {O}}}
\newcommand{\BP}{{\mathbb {P}}}
\newcommand{\BQ}{{\mathbb {Q}}}
\newcommand{\BR}{{\mathbb {R}}}
\newcommand{\BS}{{\mathbb {S}}}
\newcommand{\BT}{{\mathbb {T}}}
\newcommand{\BU}{{\mathbb {U}}}
\newcommand{\BV}{{\mathbb {V}}}
\newcommand{\BW}{{\mathbb {W}}}
\newcommand{\BX}{{\mathbb {X}}}
\newcommand{\BY}{{\mathbb {Y}}}
\newcommand{\BZ}{{\mathbb {Z}}}
\newcommand{\Bk}{{\mathbf {k}}}

\newcommand{\CA}{{\mathcal {A}}}
\newcommand{\CB}{{\mathcal {B}}}
\newcommand{\CC}{{\mathcal {C}}}

\newcommand{\CE}{{\mathcal {E}}}
\newcommand{\CF}{{\mathcal {F}}}
\newcommand{\CG}{{\mathcal {G}}}
\newcommand{\CH}{{\mathcal {H}}}
\newcommand{\CI}{{\mathcal {I}}}
\newcommand{\CJ}{{\mathcal {J}}}
\newcommand{\CK}{{\mathcal {K}}}
\newcommand{\CL}{{\mathcal {L}}}
\newcommand{\CM}{{\mathcal {M}}}
\newcommand{\CN}{{\mathcal {N}}}
\newcommand{\CO}{{\mathcal {O}}}
\newcommand{\CP}{{\mathcal {P}}}
\newcommand{\CQ}{{\mathcal {Q}}}
\newcommand{\CR}{{\mathcal {R}}}
\newcommand{\CS}{{\mathcal {S}}}
\newcommand{\CT}{{\mathcal {T}}}
\newcommand{\CU}{{\mathcal {U}}}
\newcommand{\CV}{{\mathcal {V}}}
\newcommand{\CW}{{\mathcal {W}}}
\newcommand{\CX}{{\mathcal {X}}}
\newcommand{\CY}{{\mathcal {Y}}}
\newcommand{\CZ}{{\mathcal {Z}}}


\newcommand{\RA}{{\mathrm {A}}}
\newcommand{\RB}{{\mathrm {B}}}
\newcommand{\RC}{{\mathrm {C}}}
\newcommand{\RD}{{\mathrm {D}}}
\newcommand{\RE}{{\mathrm {E}}}
\newcommand{\RF}{{\mathrm {F}}}
\newcommand{\RG}{{\mathrm {G}}}
\newcommand{\RH}{{\mathrm {H}}}
\newcommand{\RI}{{\mathrm {I}}}
\newcommand{\RJ}{{\mathrm {J}}}
\newcommand{\RK}{{\mathrm {K}}}
\newcommand{\RL}{{\mathrm {L}}}
\newcommand{\RM}{{\mathrm {M}}}
\newcommand{\RN}{{\mathrm {N}}}
\newcommand{\RO}{{\mathrm {O}}}
\newcommand{\RP}{{\mathrm {P}}}
\newcommand{\RQ}{{\mathrm {Q}}}
%\newcommand{\RR}{{\mathrm {R}}}
\newcommand{\RS}{{\mathrm {S}}}
\newcommand{\RT}{{\mathrm {T}}}
\newcommand{\RU}{{\mathrm {U}}}
\newcommand{\RV}{{\mathrm {V}}}
\newcommand{\RW}{{\mathrm {W}}}
\newcommand{\RX}{{\mathrm {X}}}
\newcommand{\RY}{{\mathrm {Y}}}
\newcommand{\RZ}{{\mathrm {Z}}}

\DeclareMathOperator{\absNorm}{\mathfrak{N}}
\DeclareMathOperator{\Ann}{Ann}
\DeclareMathOperator{\LAnn}{L-Ann}
\DeclareMathOperator{\RAnn}{R-Ann}
\DeclareMathOperator{\ind}{ind}
%\DeclareMathOperator{\Ind}{Ind}



\newcommand{\cod}{{\mathrm{cod}}}
\newcommand{\cont}{{\mathrm{cont}}}
\newcommand{\cl}{{\mathrm{cl}}}
\newcommand{\cusp}{{\mathrm{cusp}}}

\newcommand{\disc}{{\mathrm{disc}}}
\renewcommand{\div}{{\mathrm{div}}}



\newcommand{\Gm}{{\mathbb{G}_m}}



\newcommand{\I}{{\mathrm{I}}}

\newcommand{\Jac}{{\mathrm{Jac}}}
\newcommand{\PM}{{\mathrm{PM}}}


\newcommand{\new}{{\mathrm{new}}}
\newcommand{\NS}{{\mathrm{NS}}}
\newcommand{\N}{{\mathrm{N}}}

\newcommand{\ord}{{\mathrm{ord}}}

%\newcommand{\rank}{{\mathrm{rank}}}

\newcommand{\rk}{{\mathrm{k}}}
\newcommand{\rr}{{\mathrm{r}}}
\newcommand{\rh}{{\mathrm{h}}}

\newcommand{\Sel}{{\mathrm{Sel}}}
\newcommand{\Sim}{{\mathrm{Sim}}}

\newcommand{\wt}{\widetilde}
\newcommand{\wh}{\widehat}
\newcommand{\pp}{\frac{\partial\bar\partial}{\pi i}}
\newcommand{\pair}[1]{\langle {#1} \rangle}
\newcommand{\wpair}[1]{\left\{{#1}\right\}}
\newcommand{\intn}[1]{\left( {#1} \right)}
\newcommand{\sfrac}[2]{\left( \frac {#1}{#2}\right)}
\newcommand{\ds}{\displaystyle}
\newcommand{\ov}{\overline}
\newcommand{\incl}{\hookrightarrow}
\newcommand{\lra}{\longrightarrow}
\newcommand{\imp}{\Longrightarrow}
%\newcommand{\lto}{\longmapsto}
\newcommand{\bs}{\backslash}

\newcommand{\cover}[1]{\widetilde{#1}}

\renewcommand{\vsp}{{\vspace{0.2in}}}

\newcommand{\Norma}{\operatorname{N}}
\newcommand{\Ima}{\operatorname{Im}}
\newcommand{\con}{\textit{C}}
\newcommand{\gr}{\operatorname{gr}}
\newcommand{\ad}{\operatorname{ad}}
\newcommand{\der}{\operatorname{der}}
\newcommand{\dif}{\operatorname{d}\!}
\newcommand{\pro}{\operatorname{pro}}
\newcommand{\Ev}{\operatorname{Ev}}
% \renewcommand{\span}{\operatorname{span}} \span is an innernal command.
%\newcommand{\degree}{\operatorname{deg}}
\newcommand{\Invf}{\operatorname{Invf}}
\newcommand{\Inv}{\operatorname{Inv}}
\newcommand{\slt}{\operatorname{SL}_2(\mathbb{R})}
%\newcommand{\temp}{\operatorname{temp}}
%\newcommand{\otop}{\operatorname{top}}
\renewcommand{\small}{\operatorname{small}}
\newcommand{\HC}{\operatorname{HC}}
\newcommand{\lef}{\operatorname{left}}
\newcommand{\righ}{\operatorname{right}}
\newcommand{\Diff}{\operatorname{DO}}
\newcommand{\diag}{\operatorname{diag}}
\newcommand{\sh}{\varsigma}
\newcommand{\sch}{\operatorname{sch}}
%\newcommand{\oleft}{\operatorname{left}}
%\newcommand{\oright}{\operatorname{right}}
\newcommand{\open}{\operatorname{open}}
\newcommand{\sgn}{\operatorname{sgn}}
\newcommand{\triv}{\operatorname{triv}}
\newcommand{\Sh}{\operatorname{Sh}}
\newcommand{\oN}{\operatorname{N}}

\newcommand{\oc}{\operatorname{c}}
\newcommand{\od}{\operatorname{d}}
\newcommand{\os}{\operatorname{s}}
\newcommand{\ol}{\operatorname{l}}
\newcommand{\oL}{\operatorname{L}}
\newcommand{\oJ}{\operatorname{J}}
\newcommand{\oH}{\operatorname{H}}
\newcommand{\oO}{\operatorname{O}}
\newcommand{\oS}{\operatorname{S}}
\newcommand{\oR}{\operatorname{R}}
\newcommand{\oT}{\operatorname{T}}
%\newcommand{\rU}{\operatorname{U}}
\newcommand{\oZ}{\operatorname{Z}}
\newcommand{\oD}{\textit{D}}
\newcommand{\oW}{\textit{W}}
\newcommand{\oE}{\operatorname{E}}
\newcommand{\oP}{\operatorname{P}}
\newcommand{\PD}{\operatorname{PD}}
\newcommand{\oU}{\operatorname{U}}

\newcommand{\g}{\mathfrak g}
\newcommand{\gC}{{\mathfrak g}_{\C}}
\renewcommand{\k}{\mathfrak k}
\newcommand{\h}{\mathfrak h}
\newcommand{\p}{\mathfrak p}
%\newcommand{\q}{\mathfrak q}
\renewcommand{\a}{\mathfrak a}
\renewcommand{\b}{\mathfrak b}
\renewcommand{\c}{\mathfrak c}
\newcommand{\n}{\mathfrak n}
\renewcommand{\u}{\mathfrak u}
%\renewcommand{\v}{\mathfrak v}
\newcommand{\e}{\mathfrak e}
\newcommand{\f}{\mathfrak f}
\renewcommand{\l}{\mathfrak l}
\renewcommand{\t}{\mathfrak t}
\newcommand{\s}{\mathfrak s}
\renewcommand{\r}{\mathfrak r}
\renewcommand{\o}{\mathfrak o}
\newcommand{\m}{\mathfrak m}
\newcommand{\z}{\mathfrak z}
%\renewcommand{\sl}{\mathfrak s \mathfrak l}
\newcommand{\gl}{\mathfrak g \mathfrak l}


\newcommand{\re}{\mathrm e}

\renewcommand{\rk}{\mathrm k}

\newcommand{\Z}{\mathbb{Z}}
\DeclareDocumentCommand{\C}{}{\mathbb{C}}
\newcommand{\R}{\mathbb R}
\newcommand{\Q}{\mathbb Q}
\renewcommand{\H}{\mathbb{H}}
%\newcommand{\N}{\mathbb{N}}
\newcommand{\K}{\mathbb{K}}
%\renewcommand{\S}{\mathbf S}
\newcommand{\M}{\mathbf{M}}
\newcommand{\A}{\mathbb{A}}
\newcommand{\B}{\mathbf{B}}
%\renewcommand{\G}{\mathbf{G}}
\newcommand{\V}{\mathbf{V}}
\newcommand{\W}{\mathbf{W}}
\newcommand{\F}{\mathbf{F}}
\newcommand{\E}{\mathbf{E}}
%\newcommand{\J}{\mathbf{J}}
\renewcommand{\H}{\mathbf{H}}
\newcommand{\X}{\mathbf{X}}
\newcommand{\Y}{\mathbf{Y}}
%\newcommand{\RR}{\mathcal R}
\newcommand{\FF}{\mathcal F}
%\newcommand{\BB}{\mathcal B}
\newcommand{\HH}{\mathcal H}
%\newcommand{\UU}{\mathcal U}
%\newcommand{\MM}{\mathcal M}
%\newcommand{\CC}{\mathcal C}
%\newcommand{\DD}{\mathcal D}
\def\eDD{\mathrm{d}^{e}}
\def\DD{\nabla}
\def\DDD{{\check\nabla}}
\def\DDc{\boldsymbol{\nabla}}
\def\gDD{\nabla^{\mathrm{gen}}}
\def\gDDc{\boldsymbol{\nabla}^{\mathrm{gen}}}
%\newcommand{\OO}{\mathcal O}
%\newcommand{\ZZ}{\mathcal Z}
\newcommand{\ve}{{\vee}}
\newcommand{\aut}{\mathcal A}
\newcommand{\ii}{\mathbf{i}}
\newcommand{\jj}{\mathbf{j}}
\newcommand{\kk}{\mathbf{k}}

\newcommand{\la}{\langle}
\newcommand{\ra}{\rangle}
\newcommand{\bp}{\bigskip}
\newcommand{\be}{\begin {equation}}
\newcommand{\ee}{\end {equation}}

\newcommand{\LRleq}{\stackrel{LR}{\leq}}

\numberwithin{equation}{section}


\def\flushl#1{\ifmmode\makebox[0pt][l]{${#1}$}\else\makebox[0pt][l]{#1}\fi}
\def\flushr#1{\ifmmode\makebox[0pt][r]{${#1}$}\else\makebox[0pt][r]{#1}\fi}
\def\flushmr#1{\makebox[0pt][r]{${#1}$}}


%\theoremstyle{Theorem}
% \newtheorem*{thmM}{Main Theorem}
% \crefformat{thmM}{main theorem}
% \Crefformat{thmM}{Main Theorem}
\newtheorem*{thm*}{Theorem}
\newtheorem{thm}{Theorem}[section]
\newtheorem{thml}[thm]{Theorem}
\newtheorem{lem}[thm]{Lemma}
\newtheorem{obs}[thm]{Observation}
\newtheorem{lemt}[thm]{Lemma}
\newtheorem*{lem*}{Lemma}
\newtheorem{whyp}[thm]{Working Hypothesis}
\newtheorem{prop}[thm]{Proposition}
\newtheorem{prpt}[thm]{Proposition}
\newtheorem{prpl}[thm]{Proposition}
\newtheorem{cor}[thm]{Corollary}
%\newtheorem*{prop*}{Proposition}
\newtheorem{claim}{Claim}
\newtheorem*{claim*}{Claim}
%\theoremstyle{definition}
\newtheorem{defn}[thm]{Definition}
\newtheorem{dfnl}[thm]{Definition}
\newtheorem*{IndH}{Induction Hypothesis}

\newtheorem*{eg*}{Example}
\newtheorem{eg}[thm]{Example}
\newtheorem{conj}[thm]{Conjecture}

\theoremstyle{remark}
\newtheorem{remark}[thm]{Remark}
\newtheorem{remarks}[thm]{Remarks}


\def\cpc{\sigma}
\def\ccJ{\epsilon\dotepsilon}
\def\ccL{c_L}

\def\wtbfK{\widetilde{\bfK}}
%\def\abfV{\acute{\bfV}}
\def\AbfV{\acute{\bfV}}
%\def\afgg{\acute{\fgg}}
%\def\abfG{\acute{\bfG}}
\def\abfV{\bfV'}
\def\afgg{\fgg'}
\def\abfG{\bfG'}

\def\half{{\tfrac{1}{2}}}
\def\ihalf{{\tfrac{\mathbf i}{2}}}
\def\slt{\fsl_2(\bC)}
\def\sltr{\fsl_2(\bR)}

% \def\Jslt{{J_{\fslt}}}
% \def\Lslt{{L_{\fslt}}}
\def\slee{{
\begin{pmatrix}
 0 & 1\\
 0 & 0
\end{pmatrix}
}}
\def\slff{{
\begin{pmatrix}
 0 & 0\\
 1 & 0
\end{pmatrix}
}}\def\slhh{{
\begin{pmatrix}
 1 & 0\\
 0 & -1
\end{pmatrix}
}}
\def\sleei{{
\begin{pmatrix}
 0 & i\\
 0 & 0
\end{pmatrix}
}}
\def\slxx{{\begin{pmatrix}
-\ihalf & \half\\
\phantom{-}\half & \ihalf
\end{pmatrix}}}
% \def\slxx{{\begin{pmatrix}
% -\sqrt{-1}/2 & 1/2\\
% 1/2 & \sqrt{-1}/2
% \end{pmatrix}}}
\def\slyy{{\begin{pmatrix}
\ihalf & \half\\
\half & -\ihalf
\end{pmatrix}}}
\def\slxxi{{\begin{pmatrix}
+\half & -\ihalf\\
-\ihalf & -\half
\end{pmatrix}}}
\def\slH{{\begin{pmatrix}
   0   & -\mathbf i\\
\mathbf i & 0
\end{pmatrix}}
}

\ExplSyntaxOn
\clist_map_inline:nn {J,L,C,X,Y,H,c,e,f,h,}{
  \expandafter\def\csname #1slt\endcsname{{\mathring{#1}}}}
\ExplSyntaxOff


\def\Mop{\fT}

\def\fggJ{\fgg_J}
\def\fggJp{\fgg'_{J'}}

\def\NilGC{\Nil_{\bfG}(\fgg)}
\def\NilGCp{\Nil_{\bfG'}(\fgg')}
\def\Nilgp{\Nil_{\fgg'_{J'}}}
\def\Nilg{\Nil_{\fgg_{J}}}
%\def\NilP'{\Nil_{\fpp'}}
\def\nNil{\Nil^{\mathrm n}}
\def\eNil{\Nil^{\mathrm e}}


\NewDocumentCommand{\NilP}{t'}{
\IfBooleanTF{#1}{\Nil_{\fpp'}}{\Nil_\fpp}
}

\def\KS{\mathsf{KS}}
\def\MM{\bfM}
\def\MMP{M}

\NewDocumentCommand{\KTW}{o g}{
  \IfValueTF{#2}{
    \left.\varsigma_{\IfValueT{#1}{#1}}\right|_{#2}}{
    \varsigma_{\IfValueT{#1}{#1}}}
}
\def\IST{\rho}
\def\tIST{\trho}

\NewDocumentCommand{\CHI}{o g}{
  \IfValueTF{#1}{
    {\chi}_{\left[#1\right]}}{
    \IfValueTF{#2}{
      {\chi}_{\left(#2\right)}}{
      {\chi}}
  }
}
\NewDocumentCommand{\PR}{g}{
  \IfValueTF{#1}{
    \mathop{\pr}_{\left(#1\right)}}{
    \mathop{\pr}}
}
\NewDocumentCommand{\XX}{g}{
  \IfValueTF{#1}{
    {\cX}_{\left(#1\right)}}{
    {\cX}}
}
\NewDocumentCommand{\PP}{g}{
  \IfValueTF{#1}{
    {\fpp}_{\left(#1\right)}}{
    {\fpp}}
}
\NewDocumentCommand{\LL}{g}{
  \IfValueTF{#1}{
    {\bfL}_{\left(#1\right)}}{
    {\bfL}}
}
\NewDocumentCommand{\ZZ}{g}{
  \IfValueTF{#1}{
    {\cZ}_{\left(#1\right)}}{
    {\cZ}}
}

\NewDocumentCommand{\WW}{g}{
  \IfValueTF{#1}{
    {\bfW}_{\left(#1\right)}}{
    {\bfW}}
}




\def\gpi{\wp}
\NewDocumentCommand\KK{g}{
\IfValueTF{#1}{K_{(#1)}}{K}}
% \NewDocumentCommand\OO{g}{
% \IfValueTF{#1}{\cO_{(#1)}}{K}}
\NewDocumentCommand\XXo{d()}{
\IfValueTF{#1}{\cX^\circ_{(#1)}}{\cX^\circ}}
\def\bfWo{\bfW^\circ}
\def\bfWoo{\bfW^{\circ \circ}}
\def\bfWg{\bfW^{\mathrm{gen}}}
\def\Xg{\cX^{\mathrm{gen}}}
\def\Xo{\cX^\circ}
\def\Xoo{\cX^{\circ \circ}}
\def\fppo{\fpp^\circ}
\def\fggo{\fgg^\circ}
\NewDocumentCommand\ZZo{g}{
\IfValueTF{#1}{\cZ^\circ_{(#1)}}{\cZ^\circ}}

% \ExplSyntaxOn
% \NewDocumentCommand{\bcO}{t' E{^_}{{}{}}}{
%   \overline{\cO\sb{\use_ii:nn#2}\IfBooleanTF{#1}{^{'\use_i:nn#2}}{^{\use_i:nn#2}}
%   }
% }
% \ExplSyntaxOff

\NewDocumentCommand{\bcO}{t'}{
  \overline{\cO\IfBooleanT{#1}{'}}}

\NewDocumentCommand{\oliftc}{g}{
\IfValueTF{#1}{\boldsymbol{\vartheta} (#1)}{\boldsymbol{\vartheta}}
}
\NewDocumentCommand{\oliftr}{g}{
\IfValueTF{#1}{\vartheta_\bR(#1)}{\vartheta_\bR}
}
\NewDocumentCommand{\olift}{g}{
\IfValueTF{#1}{\vartheta(#1)}{\vartheta}
}
% \NewDocumentCommand{\dliftv}{g}{
% \IfValueTF{#1}{\ckvartheta(#1)}{\ckvartheta}
% }
\def\dliftv{\vartheta}
\NewDocumentCommand{\tlift}{g}{
\IfValueTF{#1}{\wtvartheta(#1)}{\wtvartheta}
}

\def\slift{\cL}

\def\BB{\bB}


\def\PhiO#1{\vartheta\left(#1\right)}

\def\bbThetav{\check{\mathbbold{\Phi}}}
\def\Phiv{\check{\Phi}}
\def\Phiv{\check{\Phi}}

\DeclareDocumentCommand{\NN}{g}{
\IfValueTF{#1}{\fN(#1)}{\fN}
}
\DeclareDocumentCommand{\RR}{m m}{
\fR({#1},{#2})
}

%\DeclareMathOperator*{\sign}{Sign}

\NewDocumentCommand{\lsign}{m}{
{}^l\mathrm{Sign}(#1)
}



\NewDocumentCommand\lnn{t+ t- g}{
  \IfBooleanTF{#1}{{}^l n^+\IfValueT{#3}{(#3)}}{
    \IfBooleanTF{#2}{{}^l n^-\IfValueT{#3}{(#3)}}{}
  }
}


% % Fancy bcO, support feature \bcO'^a_{\mathrm b} = \overline{\cO'^a_{\mathrm b}}
% \makeatletter
% \def\bcO{\def\O@@{\cO}\@ifnextchar'\@Op\@Onp}
% \def\@Opnext{\@ifnextchar^\@Opsp\@Opnsp}
% \def\@Op{\afterassignment\@Opnext\let\scratch=}
% \def\@Opnsp{\def\O@@{\cO'}\@Otsb}
% \def\@Onp{\@ifnextchar^\@Onpsp\@Otsb}
% \def\@Opsp^#1{\def\O@@{\cO'^{#1}}\@Otsb}
% \def\@Onpsp^#1{\def\O@@{\cO^{#1}}\@Otsb}
% \def\@Otsb{\@ifnextchar_\@Osb{\@Ofinalnsb}}
% \def\@Osb_#1{\overline{\O@@_{#1}}}
% \def\@Ofinalnsb{\overline{\O@@}}

% Fancy \command: \command`#1 will translate to {}^{#1}\bfV, i.e. superscript on the
% lift conner.

% \def\defpcmd#1{
%   \def\nn@tmp{#1}
%   \def\nn@np@tmp{@np@#1}
%   \expandafter\let\csname\nn@np@tmp\expandafter\endcsname \csname\nn@tmp\endcsname
%   \expandafter\def\csname @pp@#1\endcsname`##1{{}^{##1}{\csname @np@#1\endcsname}}
%   \expandafter\def\csname #1\endcsname{\,\@ifnextchar`{\csname
%       @pp@#1\endcsname}{\csname @np@#1\endcsname}}
% }

% \def\defppcmd#1{
% \expandafter\NewDocumentCommand{\csname #1\endcsname}{##1 }{}
% }



% \defpcmd{bfV}
% \def\KK{\bfK}\defpcmd{KK}
% \defpcmd{bfG}
% \def\A{\!A}\defpcmd{A}
% \def\K{\!K}\defpcmd{K}
% \def\G{G}\defpcmd{G}
% \def\J{\!J}\defpcmd{J}
% \def\L{\!L}\defpcmd{L}
% \def\eps{\epsilon}\defpcmd{eps}
% \def\pp{p}\defpcmd{pp}
% \defpcmd{wtK}
% \makeatother

\def\fggR{\fgg_\bR}
\def\rmtop{{\mathrm{top}}}
\def\dimo{\dim^\circ}
\def\GKdim{\text{GK-}\dim}

\NewDocumentCommand\LW{g}{
\IfValueTF{#1}{L_{W_{#1}}}{L_{W}}}
%\def\LW#1{L_{W_{#1}}}
\def\JW#1{J_{W_{#1}}}

\def\floor#1{{\lfloor #1 \rfloor}}

\def\KSP{K}
\def\UU{\rU}
\def\UUC{\rU_\bC}
\def\tUUC{\widetilde{\rU}_\bC}
\def\OmegabfW{\Omega_{\bfW}}


\def\BB{\bB}


\def\PhiO#1{\vartheta\left(#1\right)}

\def\Phiv{\check{\Phi}}
\def\Phiv{\check{\Phi}}

\def\Phib{\bar{\Phi}}

\def\cKaod{\cK^{\mathrm{aod}}}

\DeclareMathOperator{\sspan}{span}


\def\sp{{\mathrm{sp}}}

\def\bfLz{\bfL_0}
\def\sOpe{\sO^\perp}
\def\sOpeR{\sO^\perp_\bR}
\def\sOR{\sO_\bR}

\def\ZX{\cZ_{X}}
\def\gdliftv{\vartheta}
\def\gdlift{\vartheta^{\mathrm{gen}}}
\def\bcOp{\overline{\cO'}}
\def\bsO{\overline{\sO}}
\def\bsOp{\overline{\sO'}}
\def\bfVpe{\bfV^\perp}
\def\bfEz{\bfE_0}
\def\bfVn{\bfV^-}
\def\bfEzp{\bfE'_0}

\def\totimes{\widehat{\otimes}}
\def\dotbfV{\dot{\bfV}}

\def\aod{\mathrm{aod}}
\def\unip{\mathrm{unip}}
\def\IC{\mathfrak{I}}

\def\PI#1{\Pi_{\cI_{#1}}}
\def\Piunip{\Pi^{\mathrm{unip}}}
\def\cf{\emph{cf.} }
\def\Groth{\mathrm{Groth}}
\def\Irr{\mathrm{Irr}}
\def\Irrsp{\mathrm{Irr}^{\text{sp}}}

\def\edrc{\mathrm{DRC}^{\mathrm e}}
\def\drc{\mathrm{DRC}}
\def\LS{\mathrm{LS}}
\def\Unip{\mathrm{Unip}}


% Ytableau tweak
\makeatletter
\pgfkeys{/ytableau/options,
  noframe/.default = false,
  noframe/.is choice,
  noframe/true/.code = {%
    \global\let\vrule@YT=\vrule@none@YT
    \global\let\hrule@YT=\hrule@none@YT
  },
  noframe/false/.code = {%
    \global\let\vrule@YT=\vrule@normal@YT
    \global\let\hrule@YT=\hrule@normal@YT
  },
  noframe/on/.style = {noframe/true},
  noframe/off/.style = {noframe/false},
}
\makeatother


\def\wAV{\AV^{\mathrm{weak}}}
\def\ckG{\check{G}}
\def\ckGc{\check{G}_{\bC}}
\def\dBV{d_{\mathrm{BV}}}
\def\CP{\mathsf{CP}}
\def\YD{\mathsf{YD}}
\def\SYD{\mathsf{SYD}}
\def\DD{\nabla}

\def\lamck{\lambda_\ckcO}
\def\Lamck{[\lambda_\ckcO]}
\def\lamckb{\lambda_{\ckcO_{\mathrm b}}}
\def\lamckg{\lambda_{\ckcO_{\mathrm g}}}
\def\Wint#1{W_{[#1]}}
\def\CLam{\Coh_{\Lambda}}
\def\Cint#1{\Coh_{[#1]}}
\def\PP{\mathsf{PAP}}
\def\PAP{\mathsf{PAP}}
\def\BOX#1{\mathrm{Box}(#1)}
\DeclareDocumentCommand{\bigtimes}{}{\mathop{\scalebox{1.7}{$\times$}}}
\providecommand\mapsfrom{\scalebox{-1}[1]{$\mapsto$}}

\def\ihh{{i_\fhh}}

\def\Gc{G_\bC}
\def\Gcad{G_\bC^{\text{ad}}}
\def\Gad{\Inn(\fgg)}

\def\hha{{}^a\fhh}
\def\ahh{\hha}
\def\aSR{{}^a\Sigma}
\def\aRp{{}^a\Delta^+}
\def\aX{{}^aX}
\def\aQ{{}^aQ}
\def\aP{{}^aP}
\def\aR{{}^aR}
\def\aRp{{}^aR^+}
\def\asRp{{}^a \Delta^+}
\def\Gfin{\cG(\Gc)}
\def\PiGfin{\Pi_{\mathrm{fin}}( \Gc )}
\def\PiGlfin{\Pi_{\Lambda_0}( \Gc )}
\def\adGfin{\cG_{\mathrm{ad}}(\Gc)}
\def\Ggk{\cG(\fgg,K)}
\def\WT#1{\Delta(#1)}
\def\WG{W(\Gc)}
\def\ch{\mathrm{ch}\,}
\def\Wlam{W_{[\lambda]}}
\def\aLam{a_{\Lambda}}
\def\WLam{W_{\Lam}}
\def\WLamck{W_{[\lambda_{\ckcO}]}}
\def\Wlamck{W_{\lamck}}
\def\Rlam{R_{[\lambda]}}
\def\RLam{R_\Lambda}
\def\RLamp{R_\Lambda^+}
\def\Rplam{R^+(\lambda)}
\def\Glfin{\cG_{\Lambda}(\Gc)}
\def\CL{{\sC}^{\scriptscriptstyle L}}
\def\CR{{\sC}^{\scriptscriptstyle R}}
\def\CLR{{\sC}^{\scriptscriptstyle LR}}
\def\LV{{}^{\scriptscriptstyle L}\sV}
\def\LC{{}^{\scriptscriptstyle L}\sC}
\def\RC{{}^{\scriptscriptstyle R}\sC}
\def\LRC{{}^{\scriptscriptstyle LR}\sC}
\def\ckLC{{}^{\scriptscriptstyle L}\check{\sC}}

\def\LV{{}^{\scriptscriptstyle L}\sV}
\def\ckLV{{}^{\scriptscriptstyle L}\check\sV}
\def\ckLC{{}^{\scriptscriptstyle L}\check\sC}

\def\tLV{{}^{\scriptscriptstyle L}\widetilde{\sV}}
\def\tLC{{}^{\scriptscriptstyle L}\widetilde{\sC}}

\def\brsgn{\breve{\sgn}}
\def\bsgn{\overline{\sgn}}

\def\Wb{W_{b}}
\def\Wg{W_{\mathrm g}}


\def\nbb{n_{\mathrm b}}
\def\ngg{n_{\mathrm g}}
\def\tU{\widetilde{\rU}}

\newcommand{\cross}{\times}
\newcommand{\crossa}{\times^a}

\def\bVL{{\overline{\sV}}^{\scriptscriptstyle L}}
\def\bVR{{\overline{\sV}}^{\scriptscriptstyle R}}
\def\bVLR{{\overline{\sV}}^{\scriptscriptstyle LR}}
\def\VL{{\sV}^{\scriptscriptstyle L}}
\def\VR{{\sV}^{\scriptscriptstyle R}}
\def\VLR{{\sV}^{\scriptscriptstyle LR}}

\def\Con{\sfC}
\def\bCon{\overline{\sfC}}
\def\Re{\mathrm{Re}}
\def\Im{\mathrm{Im}}
\def\AND{\quad \text{and} \quad}
\def\Coh{\mathrm{Coh}}
\def\Cohlm{\Coh_{\Lambda}(\cM)}
\def\ev#1{{\mathrm{ev}_{#1}}}

\def\ppp{\times}
\def\mmm{\slash}


\def\cuprow{{\stackrel{r}{\sqcup}}}
\def\cupcol{{\stackrel{c}{\sqcup}}}

\def\Spr{\mathrm{Springer}}
\def\Prim{\mathrm{Prim}}



\def\imathp{\imath_{\wp}}
\def\jmathp{\jmath_{\wp}}

\def\CQ{\overline{\sfA}}% Lusztig's canonical quotient
\def\CPP{\mathrm{PP}}
\def\CPPs{\mathrm{PP}_{\star}}
%\def\CPP{\mathfrak{P}}
%\def\CPPs{\mathfrak{P}_{\star}}


\def\ceil#1{\lceil #1 \rceil}
\def\symb#1#2{{\left(\substack{{#1}\\{#2}}\right)}}
\def\cboxs#1{\mbox{\scalebox{0.25}{\ytb{\ ,\vdots,\vdots,\ }}}_{#1}}

\def\hsgn{\widetilde{\mathrm{sgn}}}

\def\tPBP{\widetilde{\mathsf{PBP}}}
\def\PBPe{\mathsf{PBP}^{\mathrm{ext}}}
\def\PBPes{\mathsf{PBP}^{\mathrm{ext}}_{\star}}
\def\PBPsb{\mathsf{PBP}_{\star,b}}

\def\bev#1{\overline{\mathrm{ev}}_{#1}}

\def\Prim{\mathrm{Prim}}
% \def\leqL{\stackrel{L}{\leq}}
% \def\leqR{\stackrel{R}{\leq}}
% \def\leqLR{\stackrel{LR}{\leq}}

% \def\leqL{{\leq_L}}
% \def\leqR{{\leq_R}}
% \def\leqLR{{\leq_{LR}}}


\def\Dsp{\cD^{\text{sp}}}
\def\Csp{\sfC^{\text{sp}}}


\def\lneqL{\mathrel{\mathop{<}\limits_{\scriptscriptstyle L}}}
\def\lneqR{\mathrel{\mathop{<}\limits_{\scriptscriptstyle R}}}
\def\lneqLR{\mathrel{\mathop{<}\limits_{\scriptscriptstyle LR}}}

\def\leqL{\mathrel{\mathop{\leq}\limits_{\scriptscriptstyle L}}}
\def\leqR{\mathrel{\mathop{\leq}\limits_{\scriptscriptstyle R}}}
\def\leqLR{\mathrel{\mathop{\leq}\limits_{\scriptscriptstyle LR}}}


\def\approxL{\mathrel{\mathop{\approx}\limits_{\scriptscriptstyle L}}}
\def\approxR{\mathrel{\mathop{\approx}\limits_{\scriptscriptstyle R}}}
\def\approxLR{\mathrel{\mathop{\approx}\limits_{\scriptscriptstyle LR}}}


\def\dphi{\rdd \phi}
\def\CPH{C(H)}
\def\whCPH{\widehat{C(H)}}

\def\Greg{G_{\text{reg}}}
\def\Hnreg{H^-_{\text{reg}}}
\def\Hireg{H_{i,\text{reg}}}
\def\Hnireg{H^-_{i,\text{reg}}}


\def\tsgn{\widetilde{\sgn}}
\def\PBP{\mathsf{PBP}}

\def\ckstar{{\check \star}}
\def\ckfgg{{\check \fgg}}

\def\Inn{\mathrm{Inn}}

\providecommand{\nsubset}{\not\subset}

\def\cuprow{{\,\stackrel{r}{\sqcup}\,}}
\def\cupcol{{\,\stackrel{c}{\sqcup}\,}}

\def\ckcOb{\ckcO_{\mathrm b}}
\def\ckcOpb{\ckcO'_{\mathrm b}}
\def\cOpb{\cO'_{\mathrm b}}
\def\ckcOg{\ckcO_{\mathrm g}}

\def\Gb{G_{\mathrm b}}
\def\Gpb{G'_{\mathrm b}}
\def\Gg{G_{\mathrm g}}




\def\tPBP{\widetilde{\mathsf{PBP}}}
\def\PBPs{\mathsf{PBP}_{\star}}
\def\PBPe{\mathsf{PBP}^{\mathrm{ext}}}
\def\PBPes{\mathsf{PBP}^{\mathrm{ext}}_{\star}}
\def\PBPsb{\mathsf{PBP}_{\star,b}}

\newcommand{\Lam}{{[\lambda]}}

\newcommand{\Rg}{\cR(\fgg)}
\newcommand{\Grt}{\cK}
\newcommand{\nckG}{n_{\ckG}}
%\newcommand{\nb}{n_{\mathrm b}}
%\newcommand{\ng}{n_{\mathrm g}}

\usepackage{xr}
\usepackage{subfiles} % Best loaded last in the preamble



\begin{document}


\title[]{Special unipotent representations of real classical groups: counting}

\author [D. Barbasch] {Dan M. Barbasch}
\address{the Department of Mathematics\\
  310 Malott Hall, Cornell University, Ithaca, New York 14853 }
\email{dmb14@cornell.edu}

\author [J.-J. Ma] {Jia-jun Ma}
\address{School of Mathematical Sciences\\
  Xiamen University\\
  Xiamen, China} \email{hoxide@xmu.edu.cn}

\author [B. Sun] {Binyong Sun}
% MCM, HCMS, HLM, CEMS, UCAS,
\address{Institute for Advanced Study in Mathematics\\
 Zhejiang University\\
  Hangzhou, China} \email{sunbinyong@zju.edu.cn}
  %310058

%\address{Academy of Mathematics and Systems Science\\
  %Chinese Academy of Sciences\\
  %Beijing, 100190, China} \email{sun@math.ac.cn}

\author [C.-B. Zhu] {Chen-Bo Zhu}
\address{Department of Mathematics\\
  National University of Singapore\\
  10 Lower Kent Ridge Road, Singapore 119076} \email{matzhucb@nus.edu.sg}




\subjclass[2010]{22E46, 22E47} \keywords{Special unipotent representation, associated variety, coherent continuation, primitive ideal, cell, classical group}

% \thanks{Supported by NSFC Grant 11222101}

\begin{abstract} Let $G$ be a real reductive group in the Harish-Chandra class. We derive some consequences of theory of coherent continuation representations, primitive ideals and cells to the counting of irreducible representations of $G$ with a given infinitesimal character and a given bound in the complex associated variety. When $G$ is a real classical group (including the real metaplectic group), we give a precise count for the number of special unipotent representations of $G$ attached to $\check \CO$, in the sense of Barbasch and Vogan. Here $\check \CO$ is a nilpotent adjoint orbit in the Langlands dual of $G$ (or the metaplectic dual of $G$ when $G$ is a real metaplectic group).
\end{abstract}




\maketitle



\tableofcontents



\section{Introduction and the main results}\label{sec:intro}

Let $G$ be a real reductive group in the Harish-Chandra class (which may be
linear or non-linear). Write $\fgg$ for the complexified Lie algebra of $G$ and
let $\hha$ denote the universal Cartan subalgebra of $\g$ (also called the abstract Cartan subalgbra in \cite{V4}).
Let $\lambda \in \hha^*$ (a superscript $*$ indicates the dual space). By Harish-Chandra isomorphism, it
determines an algebraic character $\chi_\lambda: \CZ(\g)\rightarrow \C$. Here
$\CZ(\g)$ denotes the center of the universal enveloping algebra
$\mathcal U(\g)$. Denote by $\Irr(G)$ the set of isomorphism classes of
irreducible Casselman-Wallach representations of $G$, and by $\Irr_\lambda(G)$
its subset consisting of the representations with infinitesimal character
$\chi_\lambda$ (or simply $\lambda$). The latter set has finite cardinality.


Let $\Nil(\g^*)$ denote the set of nilpotent elements in $\g^*$. It has only
finitely many orbits under the coadjoint action of the inner automorphism group
$\mathrm{Inn}(\g)$ of $\g$. Let $\sfS$ be an $\mathrm{Inn}(\g)$-stable Zariski
closed subset of $\mathrm{Nil}(\g^*)$. Put
\[
  \Irr_{\lambda,\sfS}(G):=\Set{\pi \in \Irr_{\lambda}(G)| \text{$\AVC(\pi)\subset \sfS$} }.
\]
Here $\AVC(\pi)$ denotes the complex associated variety of $\pi$, namely the
associated variety of the annihilate ideal of $\pi$. It is an
$\mathrm{Inn}(\g)$-stable Zariski closed subset of $\mathrm{Nil}(\g^*)$. An
interesting problem of representation theory is to count the finite set $\Irr_{\lambda,\sfS}(G)$. The coherent continuation
representation (of the integral Weyl group) provides a powerful tool for this problem. The first goal of the paper is to give a systematic treatment of the problem in this general set-up, by building on earlier ideas of several authors including Vogan \cite{Vg}, Joseph \cite{J1,J2},  King \cite{King}, Barbasch-Vogan \cite{BVUni}, Casian \cite{Cas}, as well as Soergel \cite{Soergel}.

\subsection{The coherent continuation representation}\label{sec11}


Write $\mathrm{Rep}(G)$ for the category of Casselman-Wallach representations of $G$, and write $\CK(G)$ for the
Grothendieck group of this category.  Throughout this article we take $\C$ as the coefficient ring to define  Grothendieck groups.
When no confusion is possible, for every object $O$ in an abelian category, we  still use the same symbol to indicate the Grothendieck group element represented by the object $O$.


Write $\mathrm{Rep}_{\lambda, \sfS}(G)$ for the full subcategory of $\mathrm{Rep}(G)$ whose objects are the representations that have
generalized infinitesimal character $\lambda$ and whose complex associated
variety is contained in $\sfS$. Write $\CK_{\lambda,\sfS}(G)$ for the
Grothendieck group of this category. Then
\[
  \sharp (\Irr_{\lambda,\sfS}(G))=\dim \CK_{\lambda,\sfS}(G)\qquad(\sharp\textrm{
    indicates the cardinality of a finite set}).
\]
We also have that
\[
  \CK_\sfS(G)=\bigoplus_{\mu\in W\backslash \hha^*} \CK_{\mu,\sfS}(G)\qquad (W\textrm{
    denotes the Weyl group}),
\]
where $\CK_{\sfS}(G)$ is the Grothendieck group of $\mathrm{Rep}_\sfS(G)$, and latter is the
category of Casselman-Wallach representations of $G$ whose complex associated
variety is contained in $\sfS$.


Let $\Rg$ be the Grothendieck group of the category of finite-dimensional algebraic
representations of $\mathrm{Inn}(\fgg)$. It is
 a commutative $\bC$-algebra under the tensor
product of representations.
Write \[
\Delta\subset Q \quad (\subset \hha^*)
\] for the root system and the root lattice of
$\fgg$, respectively.
%and identified with $\bC[Q/W]$.
By pulling back through the adjoint representation
$G\rightarrow \mathrm{Inn}(\g)$, every algebraic representation of $\mathrm{Inn}(\g)$ is viewed as a representation of $G$.
Under the tensor product of representations, $\CK_\sfS(G)$ is naturally a $\Rg$-module.


Put
\[
\Lam:=\lambda+Q\subset \hha^*,
\]
 and write $W_\Lam$
for its stabilizer in $W$. Then $W_\Lam$ equals the Weyl group of the root
system  (\cite[Section 1.3]{Jan})
\[
\Delta_\Lam:=  \{\alpha \in \Delta\mid \langle \lambda, \alpha^\vee\rangle \in \Z\}\qquad (\alpha^\vee \textrm{ denotes the coroot corresponding to $\alpha$}).
\]
We will often refer to $W_\Lam$ as the integral Weyl group.

\begin{defn}\label{defcoh}
  Let $\CK$ be a $\mathcal R(\g)$-module equipped with a family
  $\{\CK_\mu\}_{\mu\in \Lam}$ of subspaces such that $\CK_{w \cdot \mu}=\CK_\mu$
  for all $w\in W_\Lam$ and $\mu\in \Lam$. A $\CK$-valued coherent family on
  $\Lam$ is a map
  \[
    \Phi: \Lam\rightarrow \CK%, \qquad \mu\mapsto \Phi_\mu
  \]
  satisfying the following two conditions:
  \begin{itemize}
    \item for all $\mu\in \Lam$, $\Phi(\mu)\in \CK_\mu$;
    \item for all finite-dimensional algebraic representations $F$ of $\mathrm{Inn}(\g)$
          and all $\mu\in \Lam$,
          \[
          F \cdot (\Phi(\mu)) = \sum_{\nu} \Phi(\mu+\nu),
          \]
          where $\nu$ runs over all weights of $F$, counted with multiplicities.%  and $F$ is viewed as an element of $\mathcal R(\g)$.
  \end{itemize}
\end{defn}


In the notation of \Cref{defcoh}, let $\Coh_{\Lam}(\CK)$ denote the
vector space of all $\mathcal K$-valued coherent families on $\Lam$. It is a
representation of $W_{\Lam}$ under the action
\[
  (w \cdot \Phi)(\mu) = \Phi(w^{-1}\cdot \mu), \qquad \textrm{for all
  }\ w\in W_\Lam, \ \mu\in \Lam.
\]
This is called a coherent continuation representation. When specifying a coherent continuation representation $\Coh_{\Lam}(\CK)$, we will often explicitly describe $\CK$ as a Grothendieck group, while the $\mathcal R(\g)$-module structure and the $W_\Lam$-invariant family  $\{\CK_\mu\}_{\mu\in \Lam}$ are the ones which are clear from the context.
For example, $\CK_\sfS(G)$ is a $\Rg$-module as described previously, and it is equipped with the family $\{\CK_{\mu, \sfS}(G)\}_{\mu\in \Lam}$ of subspaces. We thus have the coherent continuation representation $\Coh_{\Lam}(\CK_\sfS(G))$ of $W_\Lam$.


%To shorten the notation, put
%\[
%  \Coh_{\Lam,\sfS}(G):=\Coh_{\Lam}(\CK_\sfS(G)).
%\]


\subsection{Counting irreducible representations with a bounded complex
  associated variety} %\label{sec12}

Denote by $W_\lambda$ the stabilizer of
$\lambda$ in $W$. Then $W_\lambda\subset W_\Lam$. Write $1_{W_\lambda}$ for the
trivial representation of $W_{\lambda}$.

Our starting point is the following theorem of Vogan. We will provide a proof due to lack of a convenient reference.
\begin{thm}[Vogan]\label{count1}
  The equality
  \[
    \sharp(\Irr_{\lambda,\sfS}(G)) = [1_{W_{\lambda}}:\Coh_{\Lam}(\CK_\sfS(G))]
  \]
  holds.
  % \[
  %   \dim {\barmu} = \dim (\cohm)_{W_\mu} = [\cohm, 1_{W_\mu}].
  % \]
\end{thm}
Here and henceforth, $[\ : \ ]$ indicates the multiplicity of the first
(irreducible) representation in the second one. Theorem \ref{count1} implies
that
\[
%\begin{equation}\label{countlg}
  \sharp(\Irr_{\lambda,\sfS}(G)) = \sum_{\sigma\in \Irr(W_\Lam)} [1_{W_{\lambda}}: \sigma]\cdot [\sigma: \Coh_{\Lam}(\CK_\sfS(G))].
%\end{equation}
\]
Thus it suffices to understand the multiplicity $ [\sigma: \Coh_{\Lam,\sfS}(G)]$
for every $\sigma\in \Irr(W_\Lam)$.

Let $\sigma\in \Irr(W_\Lam)$. Define the
nilpotent orbit
\[
  \CO_\sigma:=\mathrm{Springer}^{-1}
  (j_{W_\Lam}^W \sigma_0)\subset \mathrm{Nil}(\g^*),
  \]
where $\sigma_0$ denotes the special irreducible representation of $W_\Lam$ that
lies in the same double cell as $\sigma$, $j_{W_\Lam}^W \sigma_0\in \Irr(W)$ denotes
the $j$-induction of $\sigma _0$, and  ``Springer"  indicates the Springer correspondence. See \cite[Chapter 11]{Carter} or Section \ref{secGoldie} for the notion of $j$-induction, and Section \ref{seccell} on special representations and double cells.


Let
\be\label{sfc}
  \Irr_\sfS(W_\Lam):= \Set{\sigma\in \Irr(\WLam)| \cO_{\sigma}\subset \sfS}.
\ee

\begin{thm}\label{count2}
  Suppose that $\sigma\in \Irr(W_\Lam)\setminus \Irr_\sfS(W_\Lam)$. Then
  %$\CO_\sigma\nsubset \sfS$.
  \[
    [\sigma:\Coh_{\Lam}(\CK_\sfS(G))]=0.
  \]
Consequently we have
\begin{equation}\label{leq002}
  \sharp(\Irr_{\lambda,\sfS}(G)) = \sum_{\sigma \in \Irr_\sfS(W_\Lam)} [1_{W_{\lambda}}: \sigma]\cdot [\sigma:\Coh_{\Lam}(\CK_\sfS(G))].
  \end{equation}
  % \[
  %   \dim {\barmu} = \dim (\cohm)_{W_\mu} = [\cohm, 1_{W_\mu}].
  % \]
\end{thm}


Theorem \ref{count2} clearly implies that
\begin{equation}\label{leq2}  \sharp(\Irr_{\lambda,\sfS}(G)) \leq  \sum_{\sigma \in \Irr_\sfS(W_\Lam)} [1_{W_{\lambda}}: \sigma]\cdot [\sigma:\Coh_{\Lam}(\CK(G))].
\end{equation}

Recall the notion of a Harish-Chandra cell representation in $\Coh_{\Lam}(\CK(G))$ (which is a subquotient of $\Coh_{\Lam}(\CK(G))$). See \cite{V4}*{Section 14} or Section \ref{seccell}.

%Also recall the notion of Lustig double cells in $ \Irr(W_{[\lambda]}$ (see Section \ref{seccell}).
% For every Harish-Chandra cell $C$  in $\Coh_{\Lam}(\CK(G))$, write $\CV(C)$ for the  Harish-Chandra cell representation attached to $C$, which is a subquotient representation of $\Coh_{\Lam}(\CK(G))$.

 \begin{thm}\label{counteq}
   % Under the notation of \Cref{lem:lcell.BV}, we have
   Assume that for every Harish-Chandra cell representation $V$ in $\Coh_{\Lam}(\CK(G))$, the set $\{\sigma\in \Irr(W_{[\lambda]}) \,|\, [\sigma: V]\neq 0\}$ is contained in a single double cell. Then
  \begin{equation*}%\label{boundc}
    \sharp(\Irr_{\lambda,\sfS}(G)) = \sum_{\sigma \in \Irr_\sfS(W_\Lam)} [1_{W_{\lambda}}: \sigma]\cdot [\sigma:\Coh_{\Lam}(\CK(G))].
  \end{equation*}
    \end{thm}

% the equality always holds. Combining Theorem \ref{count1}, Theorem
% \ref{count2} and \eqref{leq1}, we conclude that
% \begin{equation}\label{leq2}
%   \sharp(\Irr_{\lambda,\sfS}(G)) \leq \sum_{\sigma\in \Irr(W_\Lam), \CO_\sigma\subset \sfS} [1_{W_{\lambda}}: \sigma]\cdot [\sigma: \Coh_{\Lam}(G)].
% \end{equation}





\subsection{Counting irreducible representations annihilated by a maximal primitive ideal}\label{sec13}
Write $I_\lambda$ for the maximal ideal of $\mathcal U(\g)$ with infinitesimal
character $\lambda$. Its associated variety equals the Zariski closure
$\overline{\CO_\lambda}$ of an $\mathrm{Inn}(\g)$-orbit
$\CO_\lambda\subset\mathrm{Nil}(\g^*) $. Note that an irreducible
Casselman-Wallach representation of $G$ lies in
$\Irr_{\lambda,\overline{\CO_\lambda}}(G)$ if and only if it is annihilated by
$I_\lambda$.


Let
\[
  \LC_{\lambda}:= \Set{\sigma\in \Irr(W_\Lam) | \sigma \text{ occurs in $(J_{W_{\lambda}}^{W_{\Lam}} \sgn )\otimes \sgn$}},
\]
called the Lusztig left cell attached to $\lambda$. Here $J_{W_{\lambda}}^{W_{\Lam}} $ indicates the $J$-induction (see \cite[Chapter 12]{Carter}), and $\sgn$
denotes the sign character (of an appropriate Weyl group).

%Let $\LC_{\lambda}\subset \Irr(W_\Lam)$ be the subset consisting of all the
%irreducible representations that occur in
%\[
%  (J_{W_{\lambda}}^{W_{\Lam}} \sgn )\otimes \sgn,
%\]
%where $J_{W_{\lambda}}^{W_{\Lam}} $ indicates the $J$-induction (see \cite[Chapter 12]{Carter}), and $\sgn$
%denotes the sign character (of an appropriate Weyl group).



 \begin{prop}[{\cite{BVUni}*{(5.26), Proposition~5.28}}]\label{lem:lcell.BV0}
  The following equality of sets holds: 
   \[
     \LC_{\lambda} = \Set{\sigma\in \Irr_{\overline{\CO_\lambda}}(W_\Lam)\, |\,   [1_{W_{\lambda}}:\sigma]\neq 0}.
   \]
   Moreover, $[1_{W_{\lambda}}:\sigma]=1$ when
   $\sigma\in \LC_\lambda$.
 \end{prop}


In view of \Cref{lem:lcell.BV0}, Theorems \ref{count2} and \ref{counteq} have the following consequence.
%nonumber  Combining \eqref{leq2}, \eqref{leq111} and  Propositions \ref{lem:lcell.BV0}, we obtain the following inequality.

 \begin{cor}
   \label{cor:bound} The equality
   \[
     \sharp(\Irr_{\lambda,\overline{\CO_\lambda}}(G)) =\sum_{\sigma\in \LC_\lambda} [\sigma: \Coh_{\Lam}(\CK_{\overline{\CO_\lambda}}(G))] 
   \]
   holds. 
   % Under the notation of \Cref{lem:lcell.BV}, we have
   Consequently,
   \begin{equation}\label{boundc}
     \sharp(\Irr_{\lambda,\overline{\CO_\lambda}}(G)) \leq \sum_{\sigma\in \LC_\lambda} [\sigma: \Coh_{\Lam}(\CK(G))].
   \end{equation}
   If for every Harish-Chandra cell representation $V$  in $\Coh_{\Lam}(\CK(G))$, the set $\{\sigma\in \Irr(W_{[\lambda]}\,|\, [\sigma: V]\neq 0\}$ is contained in a single double cell, then the equality holds in
   \eqref{boundc}.

 \end{cor}


 \subsection{Special unipotent representations of real classical groups}
 \label{sec:defunip}

 We are particularly interested in counting special unipotent representations of
 real classical groups.

 Let $\star$ be one of the 10 symbols
 \[
   \textrm{ $A^\R$, $A^\bH$, $A$, $\widetilde A$,  $B$, $D$, $C$, $\wtC$,
     $D^*$, $C^*$. }
 \]
 Suppose that $G$ is a classical Lie group of type $\star$, namely $G$
 respectively equals one of the following Lie groups:
 \[
   \begin{array}{c}
     \GL_n(\R),  \  \GL_n(\bH),\  \oU(p,q),\  \widetilde \oU(p,q), \smallskip\\
     \SO(p,q)\ (p+q\, \textrm{ is odd}),  \  \SO(p,q)\  (p+q\, \textrm{ is even}),\smallskip\\
     \Sp_{2n}(\R),  \   \widetilde \Sp_{2n}(\R), \  \oO^*(2n), \  \Sp(p,q),\qquad (n, p, q\geq 0).
   \end{array}
 \]
 Here $\wtSp_{2n}(\R)$ denotes the metaplectic double cover of the symplectic
 group $\Sp_{2n}(\R)$ that does not split unless $n=0$, and $\tU(p,q)$ is the double cover of $\rU(p,q)$ defined by a square root of the determinant character.

 Define the Langlands dual $\ckG$ of $G$ to be respectively the complex group
 \[
   \begin{array}{c}
     \GL_n(\C),  \  \GL_{2n}(\C),\  \GL_{p+q}(\C), \  \GL_{p+q}(\C),\smallskip\\
     \Sp_{p+q-1}(\C)\ (p+q\, \textrm{ is odd}),  \  \SO_{p+q}(\C)\  (p+q\, \textrm{ is even}),\smallskip\\
     \SO_{2n+1}(\C),  \ \Sp_{2n}(\C), \  \SO_{2n}(\C), \ \textrm{or } \   \SO_{2p+2q+1}(\C).
   \end{array}
 \]
 Write $\check \g$ for the Lie algebra of $\ckG$, and let $\check \CO\subset\mathrm{Nil}(\check \g)$ be a nilpotent $\ckG$-orbit.

  Let $\lambda_{\ckcO}\in \check \g$ be half of the neutral element in any
 $\fsl_{2}$ triple attached to $\ckcO$, as in \cite[Section 5]{BVUni}. It is a semisimple element and is uniquely determined up to conjugation by $\ckG$.
Using the identification
 \be\label{sse}
   \ckG\backslash  \{\textrm{semisimple element in $\check \g$}\}=W\backslash \hha^*,
 \ee
we view $\lambda_{\check \CO}$ as an element of $ W\backslash \hha^*$, and write $I_{\check \CO}:=I_{\star, \check \CO}$ for the maximal ideal of $\mathcal U(\g)$ with infinitesimal character $\lambda_{\check \CO}$. We remark that in the metaplectic case, namely $G=\widetilde \Sp_{2n}(\R)$, both $\hha$ and $\hha^*$ are identified with $\C^n$ in the usual way and hence \eqref{sse} still holds.

 %By using Harish-Chandra isomorphism, we view $\lambda_{\check \CO}$ as a character $\lambda_{\check \CO}: \mathcal Z(\g)\rightarrow \C$.


Following Barbasch-Vogan \cite{BVUni}, define the set of the special unipotent representations of $G$
 attached to $\ckcO$ by
\[
 %\begin{equation}\label{eq:defuni}
   \begin{split}
     \Unip_{\ckcO}(G):=&  \Unip_{\star, \ckcO}(G) \\
     :=& \begin{cases}
       % \{\pi\in \Irr(G)\mid \pi \textrm{ is annihilated by $ I_{\check \CO}$ or $I'_{\check \CO}$}\}, & \text{if } \star \in \set{D, D^\C, D^*};\\
       \{\pi\in \Irr(G)\mid \pi \textrm{ is genuine  and annihilated by } I_{\check \CO}\}, & \text{if } \star\in \{\widetilde A, \widetilde C\};\\
       \{\pi\in \Irr(G)\mid \pi \textrm{ is annihilated by } I_{\check \CO}\}, & otherwise.\\
     \end{cases}
   \end{split}
% \end{equation}
\]
 Here ``genuine" means that the central subgroup $\{\pm 1\}$ of $G$, which is the kernel of the covering homomorphism $\widetilde \oU(p,q)\rightarrow  \oU(p,q)$ or $\widetilde \Sp_{2n}(\R)\rightarrow \Sp_{2n}(\R)$, acts on $\pi$ through the nontrivial character. 
 % of $G$ representation $\pi$ of $\widetilde \oU(p,q)$ or $\widetilde \Sp_{2n}(\R)$ does not descend to $\oU(p,q)$ or $\Sp_{2n}(\R)$, respectively.

The main goal of the paper is to count the set $\Unip_{\check \CO}(G)$. In view of \Cref{cor:bound}, we will explicitly determine both $\LC_\lambda$ and $\Coh_{\Lam}(\CK(G))$ (for $\lambda =\lambda_{\ckcO}$) and will express the sum $\sum_{\sigma\in \LC_\lambda} [\sigma: \Coh_{\Lam}(\CK(G))]$ as the count of certain combinatorial constructs. These combinatorial constructs provide the key linkage with the authors' second paper \cite{BMSZ2}, whose main goal is to construct all the representations in $\Unip_{\check \CO}(G)$.


\subsection{The cases of general linear groups and unitary groups}


  For a Young diagram $\imath$, write
 \[
   \mathbf r_1(\imath)\geq \mathbf r_2(\imath)\geq \mathbf r_3(\imath)\geq \cdots
 \]
 for its row lengths, and similarly, write
 \[
   \mathbf c_1(\imath)\geq \mathbf c_2(\imath)\geq \mathbf c_3(\imath)\geq \cdots
 \]
 for its column lengths. Denote by
 $\abs{\imath}:=\sum_{i=1}^\infty \mathbf r_i(\imath)$ the total size of
 $\imath$.


When no confusion is possible, we still use $\check \CO$ to denote the Young diagram attached to the nilpotent orbit $\check \CO$. Note that the Young diagram determines the nilpotent orbit unless $\check G=\SO_{4n}(\C)$ ($n\geq 1$) and all the row lengths are even.

Let $\bN^+$ denote the set of positive integers. For any Young diagram $\imath$, we introduce the set $\mathrm{Box}(\imath)$ of
boxes of $\imath$ as the following subset of $\bN^+\times \bN^+$:
\[
% \begin{equation}\label{eq:BOX}
  \mathrm{Box}(\imath):=\Set{(i,j)\in\bN^+\times \bN^+| j\leq \bfrr_i(\imath)}.
%\end{equation}
\]

% We will also call a subset of $\bN^+\times \bN^+$ of the form \eqref{eq:BOX} a
% Young diagram.

% We say that a Young diagram $\imath'$ is contained in $\imath$ (and write
% $\imath'\subset \imath$) if
% \[
%   \mathbf r_i(\imath')\leq \mathbf r_i(\imath)\qquad \textrm{for all
% } i=1,2, 3, \cdots.
% \]
% When this is the case, $\mathrm{Box}(\imath')$ is viewed as a subset of
% $\mathrm{Box}(\imath)$ concentrating on the upper-left corner. We say that a
% subset of $\mathrm{Box}(\imath)$ is a Young subdiagram if it equals
% $\mathrm{Box}(\imath')$ for a Young diagram $\imath'\subset \imath$. In this
% case, we call $\imath'$ the Young diagram corresponding to this Young
% subdiagram.

\renewcommand{\CP}{\mathcal{P}} We also introduce five symbols $\bullet$, $s$,
$r$, $c$ and $d$, and make the following definitions.
\begin{defn}
  A painting on a Young diagram $\imath$ is a map
  \[
    \mathcal P: \mathrm{Box}(\imath) \rightarrow \{\bullet, s, r, c, d \}
  \]
  with the following properties:
  \begin{itemize}
    \item $\mathcal P^{-1}(S)$ is the set of boxes of a Young diagram when
          $S=\{\bullet\}, \{\bullet, s \}, \{\bullet, s, r\}$ or
          $\{\bullet, s, r, c \} $;
    \item when $S=\{s\}$ or $ \{r\}$, every row of $\imath$ has at most one box
          in $\CP^{-1}(S)$;
    \item when $S=\{c\}$ or $ \{d \}$, every column of $\imath$ has at most one
          box in $\CP^{-1}(S)$.
  \end{itemize}
A painted Young diagram is a pair $(\imath, \CP)$ consisting of a Young diagram $\imath$ and a painting $\CP$ on $\imath$.
\end{defn}



\begin{defn}\label{defpbp0}
  Suppose that $\star\in \{A^\R, A^\bH, A, \widetilde A\}$. A painting $\CP$ on a Young diagram
  $\imath$ has type $\star$ if
  \begin{itemize}
    \item the image of $\CP$ is contained in
          \[
          \left\{
          \begin{array}{ll}
            \{\bullet, c, d\}, &\hbox{if $\star=A^\R$}; \smallskip\\
            \{\bullet\}, &\hbox{if $\star=A^\bH$}; \smallskip\\
            \{\bullet, s, r\}, &\hbox{if $\star\in \{A, \widetilde A\}$},            \end{array}
        \right.
          \]
    \item if $\star\in \{A^\R, A^\bH\}$, then $\CP^{-1}(\bullet)$ has even number of
          boxes in every column of $\imath$,
    \item if $\star\in \{A, \widetilde A\}$, then $\CP^{-1}(\bullet)$ has even number of boxes in
          every row of $\imath$.
  \end{itemize}
  Denote by $\PAP_\star(\imath)$ the set of paintings on $\imath^{t}$ that has type $\star$, where $\imath^{t}$
  is the transpose of $\imath$.
   \end{defn}

%Note that in the definition of $\PAP_\star(\imath^{t})$, we have incorporated the transpose map in order to reconcile with the Barbasch-Vogan duality.

The middle letter $A$ in $\PAP$ refers to the common $A$ in $\{A^\R, A^\bH, A, \widetilde A\}$.

Special unipotent representations of general linear groups are
well-understood (see \cite{V.GL}*{Page 450}). In particular, we have the following counting result for general linear groups.

\begin{thm}\label{GLcase}
 Suppose that  $\star\in\{A^\R, A^\bH\}$. Then
  \[
    \sharp(\Unip_{\check \CO}(G))=        \sharp(\PAP_\star(\check \CO)).
    \]

\end{thm}
\begin{remark}
  If $\star=A^\R$, then
  \[
    \sharp(\PAP_\star(\check \CO))=\prod_{i\in \bN^+} (1+\textrm{the
      number of rows of length $i$ in $\check \CO$}).
  \]
  If $\star=A^\bH$, then
  \[
    \sharp(\PAP_\star(\check \CO))= \left\{
      \begin{array}{ll}
        1, &\hbox{if all row lengths of $\check \CO$ are even}; \smallskip\\
        0, &\hbox{otherwise}.  \end{array}
    \right.
  \]

\end{remark}

Now suppose that $\imath$ is a Young diagram and $\CP$ is a painting on $\imath$
that has type $A$ or $\widetilde A$. Define the signature of $\CP$ to be the pair
\begin{equation}\label{eq:signature}
    (p_\CP, q_\cP): = \left (\frac{\sharp(\cP^{-1}(\bullet))}{2}+\sharp(\cP^{-1}(r)),\,
    \frac{ \sharp(\cP^{-1}(\bullet))}{2}+\sharp(\cP^{-1}(s))\right).
\end{equation}
\trivial[h]{ The first equation is the true definition of signature. The second
  one is an easy consequence of the definition of $\AC_\cP$. }

\begin{eg}
  Suppose
  that \[ \check \CO=\ytb{\ \ \ \ \ , \ \ \ , \ , \ , \ }\quad \textrm{and}\quad \CP=\ytb{\bullet\bullet\bullet\bullet r,\bullet\bullet , sr,s,r}\in \mathrm{PAP}_{A}(\check \CO) .
  \]
  Then $(p_\CP, q_\cP)=(6,5)$.

\end{eg}

Given two Young diagrams $\imath$ and $\jmath$, write $\imath\cuprow \jmath$ for
the Young diagram whose multiset of nonzero row lengths equals the union of
those of $\imath$ and $\jmath$. Also write $2\imath =\imath\cuprow \imath$.

For unitary groups, we have the following counting result.
\begin{thm}
  Suppose that $\star=A$ or $\widetilde A$ so that $G=\oU(p,q)$ or $\widetilde \oU(p,q)$, respectively. Assume that there is a decomposition
  \[
    \ckcO=\ckcOg \cuprow 2\ckcOpb
  \]
  with the following property: 
  \begin{itemize}
  \item if $\star=A$, then all nonzero row lengths of $\ckcOg$ have the same parity as $p+q$,
  and all nonzero row lengths of $\ckcOpb$ have different parity as $p+q$;
  \item if $\star=\widetilde A$, then all nonzero row lengths of $\ckcOpb$ have the same parity as $p+q$,
  and all nonzero row lengths of $\ckcOg$ have different parity as $p+q$.
  \end{itemize}
  Then
  \[
    \sharp(\Unip_{\ckcO}(G))= \sharp \set{\CP\in \mathrm{PAP}_\star(\ckcOg)|(p_\CP+\abs{\ckcOpb}, q_\CP+\abs{\ckcOpb})=(p,q)}.
  \]
  If there is no such decomposition, then $\sharp(\Unip_{\check \CO}(G))=0$.

\end{thm}

In particular, when $\star=\widetilde A$ and $p+q$ is odd, the set $\Unip_{\check \CO}(\widetilde \oU(p,q))$ is empty. 

\subsection{Orthogonal and symplectic groups: reduction to good parity}

Now we assume that
$\star\in \Set{ B, D, C, \wtC, D^*, C^*}$.
Then there is a unique decomposition
\[
  \ckcO=\ckcOg \cuprow 2\ckcOpb
\]
such that $\ckcOg$ has $\star$-good parity in the sense that all its nonzero row
lengths are
\[
  \left\{
    \begin{array}{ll}
      \textrm{even}, &\hbox{if $\star\in \set{B, \widetilde C}$}; \smallskip\\
      \textrm{odd}, &\hbox{if $\star\in \set{C, D, D^*, C^*}$},
    \end{array}
  \right.
\]
and $\check \CO'_{\mathrm b}$ has $\star$-bad parity in the sense that all its
nonzero row lengths are
\[
  \left\{
    \begin{array}{ll}
      \textrm{odd}, &\hbox{if  $\star\in \set{B, \widetilde C}$}; \smallskip\\
      \textrm{even}, &\hbox{if  $\star\in \set{C, D, D^*, C^*}$}.
    \end{array}
  \right.
\]

For simplicity, put
\[
  l:=\abs{\ckcOpb},
\]
and
\[
  \Gpb := \begin{cases}
    \GL_{l}(\bR), & \text{if } \star \in \set{B,C,D}; \\
       \widetilde{ \GL}_{l}(\bR), & \text{if } \star =\wtC; \\
    \GL_{\frac{l}{2}}(\bH), & \text{if } \star \in \set{C^{*},D^{*}}. \\
  \end{cases}
\]
Here $ \widetilde{ \GL}_{l}(\bR)$ is the double cover of $ \GL_{l}(\bR)$ that fits the following Cartesian diagram of Lie groups:
\begin{equation}\label{wgll}
\begin{CD}
 \widetilde{ \GL}_{l}(\bR)@>>>  \GL_{l}(\bR)\\
  @VVV @VV g\mapsto \textrm{ sign of $\det(g)$} V\\
  \{\pm 1, \pm \sqrt{-1}\} @> x\mapsto x^2 >> \{\pm 1\}. \\
\end{CD}
\end{equation}

Define
 \[
      \Unip_{\ckcOpb}(\widetilde{ \GL}_{l}(\bR)):=
       \{\pi\in \Irr(\widetilde{ \GL}_{l}(\bR))\mid \pi \textrm{ is genuine  and annihilated by } I_{\ckcOpb}:= I_{A, \ckcOpb}\}.
       \]
        Here and as before, ``genuine" means that the representation $\pi$ of
 $\widetilde \GL_{l}(\R)$ does not descend to $\GL_{l}(\R)$. Then we have a bijective map
 \[
    \Unip_{\ckcOpb}(\GL_{l}(\bR))\rightarrow  \Unip_{\ckcOpb}(\widetilde{ \GL}_{l}(\bR)), \quad \pi\mapsto \pi\otimes \tilde \chi_l,
 \]
 where $\tilde \chi_l$ is the character given by the left vertical arrow of \eqref{wgll}.

Note that $G$ has a closed subgroup isomorphic to $\Gpb$ (as Lie groups) if and only if
\[
  \begin{cases}
    p,q\geq l, & \text{if $G = \SO(p,q)$};\\
    p,q\geq \frac{l}{2}, &  \text{if $G = \Sp(p,q)$};\\
    \text{no condition,} & \text{otherwise}.
  \end{cases}
\]
In such cases, $G$ has a Levi subgroup that is identified with $\Gpb\times \Gg$ (or   $(\Gpb\times \Gg)/\{\pm 1\} $ when $\star=\wtC$), where
\[
  \Gg :=
  \begin{cases}
    \SO(p-l,q-l), & \textrm{if $\star\in \set{B,D}$};\\
  %  \SO_{n-2l}(\bC) &\textrm{if $\star\in \set{B^{\bC},D^{\bC}}$},\\
    \rO^{*}(2n-2l), &\textrm{if $\star = D^{*}$};\\
    \Sp_{2n-2l}(\bR), &\textrm{if $\star = C$};\\
    \wtSp_{2n-2l}(\bR), &\textrm{if $\star = \wtC$};\\
  %  \Sp_{2n-2l}(\bC) &\textrm{if $\star \in \set{C^{\bC},\wtC^{\bC}}$},\\
    \Sp(p-\frac{l}{2},q-\frac{l}{2}), &\textrm{if $\star = C^{*}$}.\\
  \end{cases}
\]
% and respectively put
% \[
%   \begin{array}{rl}
%     \Gg:=  & \SO(p-l,q-l)\ \  (\textrm{when $p,q\geq l$}),   \ \     \SO_{n-2l}(\C),  \  \   \Sp_{2n-2l}(\R), \  \ \Sp_{2n-2l}(\C), \smallskip \\
%     %   & \oO^*(2n-2l), \ \  \Sp(p-\frac{l}{2},q-\frac{l}{2}) \ \  (\textrm{when $p,q\geq 2l$}),  \ \   \widetilde \Sp_{2n-2l}(\R) \ \ \textrm{or }  \ \  \Sp_{2n-2l}(\C),
%      \end{array}
%    \]
%    when \[
%      \begin{array}{rl}
%     G=  & \SO(p,q)   \ \     \SO_{n}(\C),  \  \   \Sp_{2n}(\R), \  \ \Sp_{2n}(\C), \smallskip \\
%     %   & \oO^*(2n), \ \  \Sp(p,q),  \ \   \widetilde \Sp_{2n}(\R) \ \ \textrm{or }  \ \  \Sp_{2n}(\C).
%      \end{array}
%    \]

\begin{thm}\label{reduction}
 If  $G$ has a closed subgroup isomorphic to $\Gpb$, then parabolic induction yields
   a bijection
   \[
 % \begin{equation}\label{eq:IND}
    \begin{array}{rccc}
      \fI\colon &   \Unip_{\ckcO'_{\mathrm b}}( G'_{\mathrm b})\times \Unip_{\ckcO_{\mathrm g}}( G_{\mathrm g})&         \longrightarrow &\Unip_{\ckcO }(G) \\
                &   (\pi',\pi_{0}) & \mapsto & \pi'\rtimes \pi_{0}.
    \end{array}
 % \end{equation}
 \]
  Otherwise,
  \[
    \Unip_{\ckcO}(G)=\emptyset.
  \]
\end{thm}


Combining with the counting result for general linear groups (Theorem \ref{GLcase}), we list the consequences of the
above theorem as follows:
\begin{enumerate}[label=(\alph*)]
  \item Assume that $\star\in \{B,D\}$ so that $G=\SO(p,q)$. Then
        \[
        \sharp(\Unip_{\check \CO}(G))=
        \begin{cases}
          \sharp(\Unip_{\check \CO_{\mathrm g}}(G_{\mathrm g}))\times \sharp(\Unip_{\check \CO'_{\mathrm b}}(\GL_l(\R)) ), &\hbox{if $p,q\geq l$}; \smallskip\\
          0, &\hbox{otherwise.}
        \end{cases}
        \]
  \item Assume that $\star=C^*$ so that $G=\Sp(p,q)$. Then
        \[
        \sharp(\Unip_{\check \CO}(G))=
        \begin{cases}
          \sharp(\Unip_{\check \CO_{\mathrm g}}(G_{\mathrm g} )), &\hbox{if $p,q\geq \frac{l}{2}$}; \smallskip\\
          0, &\hbox{otherwise.}
        \end{cases}
        \]

  \item Assume that $\star\in \{C,\widetilde C\}$ so that $G=\Sp_{2n}(\R)$ or
        $\widetilde \Sp_{2n}(\R)$. Then
        \[
        \sharp(\Unip_{\check \CO}(G))= \sharp(\Unip_{\check \CO_{\mathrm g}}(G_{\mathrm g}))\times \sharp(\Unip_{\check \CO'_{\mathrm b}}(\GL_l(\R)) ). \]
  \item Assume that $\star =D^*$ so that $G=\oO^*(2n)$. Then
        \[
          \sharp(\Unip_{\check \CO}(G))= \sharp(\Unip_{\check \CO_{\mathrm g}}(G_{\mathrm g})).
        \]
\end{enumerate}


 \subsection{Orthogonal and symplectic groups: the case of good parity}\label{secorgp0}
 We now assume that $\check \CO$ has $\star$-good parity, namely
 $\check \CO=\check \CO_{\mathrm g}$. By Theorem \ref{reduction}, the counting
 problem in general is reduced to this case.



 \delete{
   \begin{defn}
     A $\star$-pair is a pair $(i,i+1)$ of consecutive positive integers such
     that
     \[
       \left\{
         \begin{array}{ll}
           i\textrm{ is odd}, \quad &\textrm{if $\star\in\{C, \widetilde{C}, C^*, C^\C, \widetilde C^\C\}$};  \\
           i \textrm{ is even}, \quad &\textrm{if $\star\in\{B, D, D^*, B^\C, D^\C\}$}. \\
         \end{array}
       \right.
     \]
     A $\star$-pair $(i,i+1)$ is said to be primitive in $\check \CO$ if
     $\mathbf r_i(\check \CO)-\mathbf r_{i+1}(\check \CO)$ is positive and even.
     Denote $\mathrm{PP}_\star(\check \CO)$ the set of all $\star$-pairs that
     are primitive in $\check \CO$.
   \end{defn}
 }



\begin{defn}\label{defn:PP}
  A $\star$-pair is a pair $(i,i+1)$ of consecutive positive integers such that
  \[
    \left\{
      \begin{array}{ll}
        i\textrm{ is odd}, \quad &\textrm{if $\star\in\{C, \widetilde{C}, C^*\}$};  \\
        i \textrm{ is even}, \quad &\textrm{if $\star\in\{B, D, D^*\}$}. \\
      \end{array}
    \right.
  \]
  A $\star$-pair $(i,i+1)$ is said to be
  \begin{itemize}
    \item vacant in $\check \CO$, if
          $\mathbf r_i(\check \CO)=\mathbf r_{i+1}(\check \CO)=0$;
    \item balanced in $\check \CO$, if
          $\mathbf r_i(\check \CO)=\mathbf r_{i+1}(\check \CO)>0$;
    \item tailed in $\check \CO$, if
          $\mathbf r_i(\check \CO)-\mathbf r_{i+1}(\check \CO)$ is positive and
          odd;
    \item primitive in $\check \CO$, if
          $\mathbf r_i(\check \CO)-\mathbf r_{i+1}(\check \CO)$ is positive and
          even.
  \end{itemize}
  Denote $\CPP_\star(\check \CO)$ the set of all $\star$-pairs that are
  primitive in $\check \CO$.
\end{defn}

%We continue with the counting problem of $\Unip_{\check \CO}(G)$, when $\check \CO$ has $\star$-good parity.

We attach to $\check \CO$ a pair of Young diagrams
\[
  (\imath_{\check \CO}, \jmath_{\check \CO}):=(\imath_\star(\check \CO), \jmath_\star(\check \CO)),
\]
as follows.

\medskip

\noindent {\bf The case when $\star=B$.} In this case,
\[
  \mathbf c_{1}(\jmath_{\check \CO})=\frac{\mathbf r_1(\check \CO)}{2},
\]
and for all $i\geq 1$,
\[
  \left (\mathbf c_{i}(\imath_{\check \CO}), \mathbf c_{i+1}(\jmath_{\check \CO})\right )= \left (\frac{\mathbf r_{2i}(\check \CO)}{2}, \frac{\mathbf r_{2i+1}(\check \CO)}{2}\right ).
\]

\medskip

\noindent {\bf The case when $\star=\widetilde C$.} In
this case, for all $i\geq 1$,
\[
  (\mathbf c_{i}(\imath_{\check \CO}), \mathbf c_{i}(\jmath_{\check \CO}))= \left (\frac{\mathbf r_{2i-1}(\check \CO)}{2}, \frac{\mathbf r_{2i}(\check \CO)}{2}\right).
\]

\medskip

\noindent {\bf The case when $\star=\{ C,C^*\}$.} In this case, for all
$i\geq 1$,
\[
  (\mathbf c_{i}(\jmath_{\check \CO}), \mathbf c_{i}(\imath_{\check \CO}))= \left\{
    \begin{array}{ll}
      (0,  0), &\hbox{if $(2i-1, 2i)$ is vacant  in $\check \CO$};\smallskip\\
      (\frac{\mathbf r_{2i-1}(\check \CO)-1}{2},  0), & \hbox{if $(2i-1, 2i)$ is tailed in $\check \CO$};\smallskip\\
      (\frac{\mathbf r_{2i-1}(\check \CO)-1}{2},  \frac{\mathbf r_{2i}(\check \CO)+1}{2}), &\hbox{otherwise}.\\
    \end{array}
  \right.
\]
\medskip

\noindent {\bf The case when $\star\in \{D,D^*\}$.} In this case,
\[
  \mathbf c_{1}(\imath_{\check \CO})= \left\{
    \begin{array}{ll}
      0,  &\hbox{if $\mathbf r_1(\check \CO)=0$}; \smallskip\\
      \frac{\mathbf r_1(\check \CO)+1}{2},   &\hbox{if $\mathbf r_1(\check \CO)>0$},\\
    \end{array}
  \right.
\]
and for all $i\geq 1$,
\[
  (\mathbf c_{i}(\jmath_{\check \CO}), \mathbf c_{i+1}(\imath_{\check \CO}))= \left\{
    \begin{array}{ll}
      (0,  0), &\hbox{if $(2i, 2i+1)$ is vacant in $\check \CO$};\smallskip\\
      \left  (\frac{\mathbf r_{2i}(\check \CO)-1}{2},  0\right ), & \hbox{if $(2i, 2i+1)$ is tailed in $\check \CO$};\smallskip\\
      \left  (\frac{\mathbf r_{2i}(\check \CO)-1}{2},  \frac{\mathbf r_{2i+1}(\check \CO)+1}{2}\right ), &\hbox{otherwise}.\\
    \end{array}
  \right.
\]




\begin{eg} Suppose that $\star=C$, and $\check \CO$ is the following Young
  diagram which has $\star$-good parity.
  \begin{equation*}%\label{eq:sp-nsp.C}
    \tytb{\ \ \ \ \  , \ \ \  , \ \ \ , \ \ \  , \ \ \ , \  ,\  }
  \end{equation*}
  Then
  \[
    \CPP_\star(\check \CO)=\{(1,2), (5,6)\}
  \]
  and
  \[
    (\imath_{\check \CO}, \jmath_{\check \CO})= \tytb{\ \ \ ,\ \ } \times \tytb{\ \ \ , \ }.
  \]


\end{eg}



\delete{
  \begin{eg} Suppose that $\star=C$, and $\check \CO$ is the following Young
    diagram which has $\star$-good parity.
    \begin{equation*}%\label{eq:sp-nsp.C}
      \tytb{\ \ \ \ \  , \ \ \  , \ \ \ , \ \ \  , \ \ \ , \  ,\  }
    \end{equation*}
    Then
    \[
      \mathrm{PP}_\star(\check \CO)=\{(1,2), (5,6)\}.
    \]
    and $(\imath_\star(\check \CO, \wp), \jmath_\star(\check \CO,
    \wp))$ %\in \mathrm{BP}_\star(\check \CO)$
    has the following form.

    \begin{equation*}%\label{eq:sp-nsp.C}
      \begin{array}{rclcrcl}
        \wp=\emptyset & : & \tytb{\ \ \ ,\ \  } \times \tytb{\ \ \ , \  }  & \qquad \quad &  \wp=\{(1,2)\}& : & \tytb{\ \ \  , \ \ , \   } \times \tytb{\ \ \  } \medskip \medskip \medskip \\
        \wp=\{(5,6)\} & : & \tytb{\ \ \ ,\ \ \ } \times \tytb{\ \ , \   }  & \qquad \quad &  \wp=\{(1,2), (5,6)\}  & : & \tytb{\ \ \  , \ \ \ ,  \ } \times \tytb{\ \   } \\
      \end{array}
    \end{equation*}

\end{eg}
}

Here and henceforth, when no confusion is possible, we write
$\alpha\times \beta$ for a pair $(\alpha, \beta)$. We will also write
$\alpha\times \beta\times \gamma$ for a triple $(\alpha, \beta, \gamma)$.


We introduce two more symbols $B^+$ and $B^-$, and make the following
definition.
\begin{defn} \label{def:pbp1}
  A painted bipartition is a triple
  $\tau=(\imath, \CP)\times (\jmath, \cQ)\times \alpha$, where $(\imath, \CP)$
  and $ (\jmath, \mathcal Q)$ are painted Young diagrams, and
  $\alpha\in \{B^+,B^-, C,D,\widetilde {C}, C^*, D^*\}$, subject to the
  following conditions:
  \begin{itemize}
          \delete{\item $(\imath, \jmath)\in \mathrm{BP}_\alpha$ if
          $\alpha\notin\{B^+,B^-\}$, and $(\imath, \jmath)\in \mathrm{BP}_{B}$
          if $\alpha\in\{B^+,B^-\}$;}

    \item $\CP^{-1}(\bullet)=\mathcal Q^{-1}(\bullet)$;
    \item the image of $\CP$ is contained in
          \[
          \left\{
          \begin{array}{ll}
            \{\bullet, c\}, &\hbox{if $\alpha=B^+$ or $B^-$}; \smallskip\\
            \{\bullet,  r, c,d\}, &\hbox{if $\alpha=C$}; \smallskip\\
            \{\bullet, s, r, c,d\}, &\hbox{if $\alpha=D$}; \smallskip\\
            \{\bullet, s, c\}, &\hbox{if $\alpha=\widetilde{ C}$}; \smallskip \\
            \{\bullet\}, &\hbox{if $\alpha=C^*$}; \smallskip \\
            \{\bullet, s\}, &\hbox{if $\alpha=D^*$},\\
          \end{array}
          \right.
          \]
    \item the image of $\mathcal Q$ is contained in
          \[
          \left\{
          \begin{array}{ll}
            \{\bullet, s, r, d\}, &\hbox{if $\alpha=B^+$ or $B^-$}; \smallskip\\
            \{\bullet, s\}, &\hbox{if $\alpha=C$}; \smallskip\\
            \{\bullet\}, &\hbox{if $\alpha=D$}; \smallskip\\
            \{\bullet, r, d\}, &\hbox{if $\alpha=\widetilde{ C}$}; \smallskip\\
            \{\bullet, s,r\}, &\hbox{if $\alpha=C^*$}; \smallskip \\
            \{\bullet, r\}, &\hbox{if $\alpha=D^*$}.
          \end{array}
          \right.
          \]

  \end{itemize}
\end{defn}

% \begin{remark}
%   The set of painted bipartition counts the multiplicities of an irreducible
%   representation of $W_{r_{\fgg}}$ occurs in the coherent continuation
%   representation at the infinitesimal character of the trivial representation.
%   For the relationship between painted bipartitions and the coherent
%   continuation representations of Harish-Chandra modules, see \cite{Mc}.
% \end{remark}

For any painted bipartition $\tau$ as in Definition \ref{defpbp0}, we write
\[
  \imath_\tau:=\imath,\ \cP_\tau:=\cP,\ \jmath_\tau:=\jmath,\ \cQ_\tau:=\cQ,\ \alpha_\tau:=\alpha,\ \abs{\tau}:=\abs{\imath}+\abs{\jmath},
\]
and
\[
  \star_\tau:= \left\{
    \begin{array}{ll}
      B, &\hbox{if $\alpha=B^+$ or $B^-$}; \smallskip\\
      \alpha, & \hbox{otherwise}.           \end{array}
  \right.
\]
% Its leading column is then defined to be the first column of $(\jmath, \CQ)$
% when $\star_\tau\in \{B, C,C^*\}$, and the first column of $(\imath, \CP)$
% when $\star_\tau\in \{\widetilde C, D, D^*\}$.

We further define a pair $(p_{\tau}, q_{\tau})$ of natural numbers given by the
following recipe.
\begin{itemize}
  \item If $\star_\tau\in \{B, D, C^*\}$, then $(p_\tau, q_\tau)$ is given by
        counting the various symbols appearing in $(\imath, \CP)$,
        $(\jmath, \cQ)$ and $\{\alpha\}$ :
        \begin{equation*}%\label{ptqt}
          \left\{
            \begin{array}{l}
              p_\tau :=( \# \bullet)+ 2 (\# r) +(\# c )+ (\# d) + (\# B^+);\smallskip\\
              q_\tau :=( \# \bullet)+ 2 (\# s) + (\# c) + (\# d) + (\# B^-).\\
            \end{array}
          \right.
        \end{equation*}
        Here
        \[
        \#\bullet:=\#(\cP^{-1}(\bullet))+\#(\cQ^{-1}(\bullet))
        %\qquad (\textrm{$\#$        indicates the cardinality of a finite set}),
        \]
        and the other terms are similarly defined.
  \item If $\star_\tau\in \{C, \widetilde C, D^*\}$, then
        $p_\tau:=q_\tau:=\abs{\tau}$.
\end{itemize}
\smallskip

We also define a classical group
\begin{equation*}%\label{def:Gt}
  G_\tau:=
  \begin{cases}
    \SO(p_\tau, q_\tau), &\hbox{if $\star_\tau=B$ or $D$}; \smallskip\\
    \Sp_{2\abs{\tau}}(\R), &\hbox{if $\star_\tau=C$}; \smallskip\\
    \widetilde{\Sp}_{2\abs{\tau}}(\R), &\hbox{if $\star_\tau=\widetilde{ C}$}; \smallskip \\
    \Sp(\frac{p_\tau}{2}, \frac{q_\tau}{2}), &\hbox{if $\star_\tau=C^*$}; \smallskip \\
    \oO^*(2\abs{\tau}), &\hbox{if $\star_\tau=D^*$}.\\
  \end{cases}
\end{equation*}


Define
\begin{equation*}%\label{defpbp2}
  \PBP_\star(\check \CO) :=\set{ \uptau\textrm{ is a painted
      bipartition} \mid \star_\uptau = \star, \text{ and
    } (\imath_\tau,\jmath_\tau) = (\imath_{\check \CO}, \jmath_{\check \CO})},
\end{equation*}
and
\begin{equation*} %\label{defpbp3}
    \PBP_{\mathrm g}(\ckcO) :=\set{\uptau\in \PBP_{\star}(\ckcO)| G_{\uptau} = G}.
\end{equation*}

\delete{
  \[
    \begin{array}{rl}
      \mathrm{PBP}_\star(\check \CO):=\{ &
                                           \tau\textrm{ is a painted bipartition}  \mid    \star_\tau = \star,
                                           \text{ and } \\  & (\imath_\tau,\jmath_\tau) = (\imath_{\check \CO}, \jmath_{\check \CO})\}.
    \end{array}
  \]
}


\begin{eg} Suppose that $\star=B$ and
  \[
    \check \CO =\tytb{\ \ \ \ \ \ , \ \ \ \ \ \ , \ \ , \ \ , \ \ }
  \]
  Then
  \[
    \tau:= \tytb{\bullet \bullet ,\bullet , c } \times \tytb{\bullet \bullet d ,\bullet , d }\times B^+\in \mathrm{PBP}_{\star}(\check \CO),
  \]
  and
  \[
    G_\tau=\SO(10,9).
  \]
\end{eg}


We now state our final result on the counting of special unipotent representations.

\begin{thm}\label{countup}
  Assume that $\star\in \{B, C,D,\widetilde {C}, C^*, D^*\}$, and $\check \CO$ has $\star$-good parity. Then
  \[
    \sharp(\Unip_{\ckcO}(G))\leq 2^{\sharp(\CPP_\star(\check \CO))} \cdot \sharp (\PBP_{\mathrm g}(\ckcO)).
  \]
\end{thm}

In \cite{BMSZ2}, the authors have constructed $2^{\sharp(\CPP_\star(\check \CO))} \cdot \sharp (\PBP_{\mathrm g}(\ckcO))$ number of representations in
$\Unip_{\check \CO}(G)$, when $\check \CO$ has $\star$-good parity. See \cite[Theorem 4.1]{BMSZ2}. Thus the equality holds in \Cref{countup}.

\medskip

%\begin{remark}

\subsection{The case of complex classical groups}
Special unipotent representations of complex classical groups are well-understood (\cite{BVUni}, \cite{B89}). We briefly review their counting and constructions in what follows. As the methods of this paper and \cite{BMSZ2} work for complex classical groups, we will present the results in the complex case parallel to those of this paper and \cite{BMSZ2}. For this subsection, we introduce five more symbols $A^\C, B^\C,D^\C, C^\C$, and $\widetilde C^\C$, and let $\star$ be one of them. Let $G$ be a complex classical group of type $\star$, namely $G=\GL_n(\C)$, $\SO_{2n+1}(\C)$, $\SO_{2n}(\C)$, $\Sp_{2n}(\C)$, or $\Sp_{2n}(\C)$ ($n\geq 0$), respectively. The Langlands dual $\check G$ of $G$ is respectively defined to be $\GL_n(\C)$, $\Sp_{2n}(\C)$, $\SO_{2n}(\C)$, $\SO_{2n+1}(\C)$, or $\Sp_{2n}(\C)$. Let $\check \CO$ be a $\check G$-orbit in $\Nil(\check \g)$ where $\check \g$ is the Lie algebra of $\check G$. As in the real case we have a maximal ideal $I_{\check \CO}:=I_{\star, \check \CO}$ of $\CU(\g_0)$, where $\g_0$ is the Lie algebra of $G$ (viewed as a complex Lie group).


 Write $\overline \g_0$ for the complex Lie algebra equipped with a conjugate linear isomorphism $\bar{\phantom a} :\g_0\rightarrow \overline{\g_0}$. The latter induces a  conjugate linear isomorphism $\bar{\phantom a} :\CU(\g_0)\rightarrow \CU( \overline{\g_0})$. Note that $\g_0\times \overline{\g_0}$ equals the complexified Lie algebra $\g$ of $G$.  Define the set of special unipotent representations of $G$
 attached to $\ckcO$ by
 \[
     \Unip_{\ckcO}(G):=  \Unip_{\star, \ckcO}(G)
     :=
       \{\pi\in \Irr(G)\mid \pi \textrm{ is annihilated by } I_{\check \CO}\otimes \CU(\overline{\g_0}) + \CU(\g_0)\otimes \overline{I_{\check \CO}}\, \}.
       \]

 If $\star=A^\C$ so that $G=\GL_n(\C)$, then $\Unip_{\ckcO}(G)$ is a singleton whose unique element is given by the normalized parabolic induction
 $\Ind_{P}^{G} 1_P$, where $P$ is the standard parabolic subgroup whose Levi component equals
 \[
 \GL_{\mathbf r_1(\check \CO)}(\C)\times \GL_{\mathbf r_2(\check \CO)}(\C)\times \dots \times \GL_{\mathbf r_{\mathbf c_1(\check \CO)}(\check \CO)}(\C),
 \]
 and $1_P$ denotes the trivial representation of $P$.
 %See \cite{V.GL}.

Now suppose that $\star\in \{B^\C,D^\C, C^\C, \widetilde C^\C\}$. As in the real case, write
 \[
   \ckcO=\ckcOg \cuprow 2\ckcOpb \quad\textrm{and}\quad l:=\abs{\ckcOpb}
 \]
 so that $G$ has a Levi subgroup  that is identified with $\Gpb\times \Gg$, where $\Gpb=\GL_l(\C)$ and
\[
  \Gg :=
  \begin{cases}
    \SO_{2n-2l+1}(\C), & \textrm{if $\star=B^\C$};\\
    \SO_{2n-2l}(\C), & \textrm{if $\star=D^\C$};\\
    \Sp_{2n-2l}(\C), &\textrm{if $\star \in\{ C^\C, \widetilde C^\C \}$}.
      \end{cases}
\]

Define the set $\CPP_\star(\ckcOg)$ as in the real case. Then by the work of
Barbasch-Vogan \cite[Corollary 5.29]{BVUni} (integral case) and Moeglin-Renard
\cite{MR.C} (general case), we have
   \[
    \sharp(\Unip_{\check \CO}(G))=\sharp(\Unip_{\ckcOg}(\Gg))=2^{\sharp(\CPP_\star(\ckcOg))}.
  \]
   As in the real case, every representation in $\Unip_{\check \CO}(G)$ is
   obtained through irreducible parabolic induction via those of
   $\Unip_{\ckcO'_{\mathrm b}}( G'_{\mathrm b})\times \Unip_{\ckcO_{\mathrm g}}( G_{\mathrm g})$
   (see Theorem \ref{reduction}), and every representation in
   $\Unip_{\ckcOg}(\Gg)$ is obtained through iterated theta lifting (see
   \cite[Theorem 3.5.1]{B17}, \cite{Mo17} and \cite{BMSZ2}).


\vskip.25in

Here are some words on the contents and the organization of this article. In Section 2, we develop some generalities on the coherent continuation representation, which lead to the proofs of Theorems \ref{count1}, \ref{count2} and \ref{counteq}. The generalities include coherent continuation representations for highest weight modules, primitive ideals and Goldie rank polynomials, as well as cell representations in the coherent continuation setting. As mentioned earlier, we build on previous works of several authors.
In Section 3, we give an explicit formula for the coherent continuation representation $\Coh_{\Lam}(\CK(G))$, which is an unpublished result of Barbasch and Vogan. Sections 4 to 7 are devoted to the main concern of the article, which is to give a precise count of special unipotent representations of all real classical groups, using results of Sections 2 and 3. We begin with the general linear groups, followed by the unitary groups, and then to real classical groups of type $\mathrm{BCD}$. All answers are given in terms of combinatorial constructs described earlier in this section. It is worthwhile to note, while the algebraic theory developed in Sections 2 and 3 yields ultimately an upper bound of the count, we are unable to demonstrate the precise count using the algebraic theory alone, due to a certain technical issue on the relationship of a Harish-Chandra cell and a Lusztig double cell, which we have formerly stated as Conjecture \ref{conjcell}. In the case at hand, namely for the real classical groups, we rely on the analytic theory of theta lifting to construct the right number of special unipotent representations (\cite{BMSZ2}), thus arriving at the precise count. It will be clearly desirable to demonstrate the precise count, without recourse to the analytic theory.



\section{Generalities on the coherent continuation representation}
\label{sec:pfGeneral}
 We retain the notation of Sections \ref{sec11}-\ref{sec13}. The main purpose of this section is to prove Theorems \ref{count1}, \ref{count2} and \ref{counteq}.

\subsection{Basic properties of the representation $\Coh_{[\lambda]}(\CK(G))$}

We define a basal vector space to be a complex vector space $V$ equipped with a basis $\mathcal B\subset V$, and call elements of $\mathcal B$ the basal elements in $V$. A subspace of a basal space $V$ is called a basal subspace if it is spanned by a set of basal elements of $V$. For example, if $\CK$ is the Grothendieck group of an abelian category in which all objects have finite length, then $\CK$ is a basal vector space with the irreducible objects as the basis. In particular, $\CK(G)$ is a basal vector space.


Recall that an element $\nu\in \hha^*$ is said to be regular if
\[
    \la \nu, \alpha^\vee\ra\neq 0 \qquad\textrm{for all $\alpha\in \Delta$},
  \]
  and is said to be dominant if
\[
    \la \nu, \alpha^\vee\ra\notin -\bN^+ \qquad\textrm{for all $\alpha\in \Delta^+$}.
  \]
Here $ \Delta^+\subset \Delta$ denotes the set of positive roots.


Write $\mathrm{Rep}_\lambda(G)$ for the category of Casselman-Wallach representations of $G$ of generalized infinitesimal character $\lambda$, and write   $\CK_{\lambda}(G)$ for its  Grothedieck group. Then   $\CK_{\lambda}(G)$  is a basal vector space with the basis $\Irr_\lambda(G)\subset \CK_{\lambda}(G)$.
By evaluating at an element  $\nu\in \Lam$, we get  a linear map
   \[
    \mathrm{ev}_{\nu } \, :\,  \Coh_{[\lambda]}(\CK(G)) \longrightarrow \Grt_{\nu}(G).
  \]



 \begin{lem}\label{lem21}
Let  $\nu\in \Lam$. The map  $\mathrm{ev}_{\nu}$ is surjective, and it is bijective when $\nu$ is regular.
     \end{lem}
\begin{proof}
The surjectivity is due to Schmid and Zuckerman, see  \cite{Vg}*{Theorem~7.2.7}. The injectivity (for $\nu$ regular)  is due to Schmid, see \cite{Vg}*{Proposition~7.2.23}.
\end{proof}




\begin{lem}\label{lemirr}
There is a unique basis $\mathcal B_\Lam(G)$ of  $\Coh_{[\lambda]}(\CK(G))$ such that
\[
  \Irr_\nu(G)\subset  \mathrm{ev}_{\nu }(\mathcal B_\Lam(G))\subset \Irr_\nu(G)\sqcup \{0\},
\]
 for any dominant element $\nu\in \Lam$.
          \end{lem}
\begin{proof}
In view  of Lemma \ref{lem21}, this is implied by  \cite{Vg}*{Corollary~7.3.23}.
\end{proof}

By Lemma \ref{lemirr}, $\Coh_{[\lambda]}(\CK(G))$ is a basal vector space, with the basis $\mathcal B_\Lam(G)$. By Lemma \ref{lem21}, for any element $\nu\in \Lam$ that is regular and dominant, the evaluation map yields a bijection
\[
\mathrm{ev}_{\nu}: \mathcal B_\Lam(G)\xrightarrow{\sim} \Irr_\nu(G).
\]

Recall from the introductory section the root system
 \[
  \Delta_{[\lambda]} := \Set{\alpha\in \Delta\mid  \inn{\lambda}{\ckalpha}\in \bZ}\subset \hha^*.
 \]
 Write $\Delta_{[\lambda]}^+$ for the set of positive roots in $\Delta_{[\lambda]} $. For each $\alpha\in \Delta$, write $s_\alpha\in W$ for the reflection attached to $\alpha$.

 \begin{lem}\label{lemirr4}  (\cite{V4}*{Corollary~7.3.23})
Let $\Phi$ be a basal element in $\Coh_{[\lambda]}(\CK(G))$, and let $\nu$ be a dominant element in $\Lam$. Then  $\Phi(\nu)=0$ if and only if
         \[
         s_\alpha \cdot \Phi =-\Phi  \quad \textrm{for some simple root $\alpha$ of $\Delta_{\Lam}^+$ such that $\la \nu, \alpha^\vee\ra=0$}.
         \]
             \end{lem}

\begin{lem}\label{lemirr6} (\cite{V4}*{Corollary~7.3.23})
Let $\Phi_1, \Phi_2$ be two basal elements in $\Coh_{[\lambda]}(\CK(G))$ such that $\Phi_1(\nu)=\Phi_2(\nu)\neq 0$ for some dominant element $\nu\in \Lam$. Then $\Phi_1=\Phi_2$.
\end{lem}


\begin{lem}\label{lemirr2}
Let $\Phi$ be a basal element in $\Coh_{[\lambda]}(\CK(G))$.  Let $\nu_1, \nu_2\in \Lam$ be two dominant elements. If both $\Phi(\nu_1)$ and $\Phi(\nu_2)$ are nonzero and so are irreducible representations of $G$. Then
          \[
          \AV_\C(\Phi(\nu_1)) = \AV_\C(\Phi(\nu_2)).
          \]
\end{lem}
\begin{proof}
This is implied by
          \cite{Vg}*{Part (a) of Proposition~7.2.22 and Part (b) of Proposition~7.3.10}.
\end{proof}

In the setting of Lemma \ref{lemirr2}, we define the complex associated variety of $\Phi$ to be $ \AV_\C(\Phi):= \AV_\C(\Phi(\nu))$, where $\nu$ is a dominant element in $\Lam$ with $\Phi(\nu)$ nonzero.

Recall that $\sfS\subset \Nil(\g^*)$ is an $\Inn(\g)$-stable Zariski closed set.

\begin{lem}\label{lemirr11}
Let $\Phi$ be a basal element in $\Coh_{[\lambda]}(\CK(G))$. Then  $\Phi \in \Coh_{[\lambda]}(\CK_\sfS(G))$ if and only if $\AV_\C(\Phi)\subset \sfS$.
         \end{lem}
\begin{proof}
The ``$\,$only if$\,$" part is trivial. The ``$\,$if$\,$" part is implied by \cite{Vg}*{Part (a) of Proposition~7.2.22} and
%the formula in the third line of  \cite{Vg}*{page 472}. Note that the latter  formula  is a special case of
\cite{Vg}*{Part (b) of Proposition~7.2.22}.
%and Vogan pointed out that \cite{Vg}*{Proposition~7.2.22} is due to Zuckerman (\cite{Zu}).
\end{proof}

Lemma \ref{lemirr11} implies that the set
\be\label{basiss}
   \mathcal B_{\Lam,\sfS}(G):=\{\Phi\in \mathcal B_\Lam(G)\,|\,\AV_\C(\Phi)\subset \sfS\}
  \  \textrm{
    is a basis of $ \Coh_{[\lambda]}(\CK_\sfS(G))$.}
    \ee
     Thus $ \Coh_{[\lambda]}(\CK_\sfS(G))$ is a basal subrepresentation of $ \Coh_{[\lambda]}(\CK(G))$, namely a subrepresentation as well as a basal subspace.

\begin{lem}\label{sur111}
For all $\nu\in\Lam$, the evaluation map  (at $\nu$)
  \[
    \mathrm{ev}_{\nu , \sfS} \, :\, \Coh_{[\lambda]}(\CK_\sfS(G)) \longrightarrow \Grt_{\nu, \sfS}(G)
  \]
 is surjective.
\end{lem}
\begin{proof}
By using the action of $W_\Lam$, we assume without loss of generality that $\nu$ is dominant. Then the lemma follows by Lemmas \ref{lemirr} and \ref{lemirr11}.
\end{proof}



\delete{
\subsection{Proof of \Cref{count1}}
  \Cref{count1} is an immediate consequence of the following proposition.

\begin{prop}\label{prop:ev}
  For all $\nu\in [\lambda]$, let
  \[
    \left(\Coh_{[\lambda]}(\CK_\sfS(G))\right)_{W_{\nu}}:= \frac{\Coh_{[\lambda]}(\CK_\sfS(G))}
    {\Span\set{\Phi- w\cdot \Phi \mid \Phi\in \Coh_{[\lambda]}(\CK_\sfS(G)), \, w\in W_{\nu}}}
  \]
  be the maximal $W_{\nu}$-invariant quotient of $\Coh_{[\lambda]}(\CK_\sfS(G)$.

  Then the evaluation map $\ev{\nu,\sfS}$ at $\nu$
  descents to an isomorphism
  \[
    \overline{\ev{\nu , \sfS}} \, \colon \left(\Coh_{[\lambda]}(\CK_\sfS(G))\right)_{W_{\nu}} \longrightarrow \Grt_{\nu, \sfS}(G).
  \]
\end{prop}
\begin{proof}
Without loss of generality we assume that $\nu$ is dominant.

By \Cref{sur111}, there is a surjective evaluation  map
  \[
    \ev{\nu, \sfS} \, :\, \Coh_{[\lambda]}(\CK_\sfS(G)) \longrightarrow \Grt_{\nu, \sfS}(G).
  \]
  This map is $W_\lambda$-invariant. Thus we have that
  \begin{eqnarray*}
      \ker \ev{\nu, \sfS}& = & \Span \Set{\Phi\in \mathcal B_{\Lam,\sfS}(G) | \Phi (\nu)=0}\quad \textrm{(by Lemmas \ref{lemirr} and \ref{lemirr6}})\\
                         &  \subseteq & \Span \Set{\Phi - s_{\alpha}\cdot\Phi |
                                        \begin{array}{l} \Phi\in  \mathcal B_{\Lam,\sfS}(G),\\
                                          \alpha  \text{ is a simple root of
                                          $\Delta_{\Lam}^+$ such that
                                          }\la \lambda, \alpha^\vee\ra=0
                                        \end{array}}\\
                         &&\hspace{5em}    \qquad  \textrm{(by Lemma \ref{lemirr4})} \\
           & \subseteq &\Span\Set{\Phi- w\cdot \Phi \mid \Phi\in \Coh_{[\lambda]}(\CK_\sfS(G)), \, w\in W_{\lambda}} \\
    &  \subseteq &  \ker \ev{\lambda, \sfS}. \\
        \end{eqnarray*}
Therefore
\[
 \ker \ev{\lambda, \sfS}=\Span\Set{\Phi- w \cdot\Phi \mid \Phi\in  \Coh_{[\lambda]}(\CK_\sfS(G)), \, w\in W_{\lambda}}.
\]
\end{proof}
}
% This completes the proof of Theorem \ref{count1}.


% \begin{proof}[Proof of \Cref{count1}]
%   This is an immediate consequence of \Cref{prop:ev}.
% \end{proof}

\subsection{Proof of \Cref{count1}}
  \Cref{count1} is an immediate consequence of \Cref{sur111} and the following proposition.

\begin{prop}\label{prop:ev}
  Suppose  $\cS$ is a basal $W_{\Lam}$-submodule of $\Coh_{\Lam}(\cK(G))$.
  For each $\nu\in [\lambda]$, the evaluation map $\ev{\nu}$ at $\nu$
  descents to an isomorphism
  \[
    \overline{\ev{\nu}|_{\cS}} \, \colon \cS_{W_{\nu}} \longrightarrow \ev{\nu}(\cS),
  \]
  where
  \[
    \cS_{W_{\nu}}:=  \cS /\Span\Set{\Phi- w\cdot \Phi | \Phi\in \cS, \, w\in W_{\nu}}
  \]
  is the maximal $W_{\nu}$-invariant quotient of $\cS$.
\end{prop}
\begin{proof}
  \def\BS{\cB_{\cS}} Without loss of generality we assume that $\nu$ is
  dominant. Let $\BS$ be the basis of $\cS$.
  % By \Cref{sur111}, there is a surjective evaluation map
  % \[
  %   \ev{\nu, \sfS} \, :\, \Coh_{[\lambda]}(\CK_\sfS(G)) \longrightarrow \Grt_{\nu, \sfS}(G).
  % \]
  % This map is $W_\lambda$-invariant.
  We have
  \begin{eqnarray*}
    \ker (\ev{\nu}|_{\cS})& = & \Span \Set{\Phi\in \BS| \Phi (\nu)=0}\quad \textrm{(by \Cref{lemirr} and \Cref{lemirr6}})\\
                          &  \subseteq & \Span \Set{\Phi - s_{\alpha}\cdot\Phi |
                                         \begin{array}{l} \Phi\in  \BS,\\
                                           \alpha  \text{ is a simple root of
                                           $\Delta_{\Lam}^+$}\\  \text{ such that
                                           }\la \lambda, \alpha^\vee\ra=0
                                         \end{array}}\\
                          &&\hspace{9.5em}     \textrm{(by Lemma \ref{lemirr4})} \\
                          & \subseteq &\Span\Set{\Phi- w\cdot \Phi | \Phi\in \cS, \, w\in W_{\nu}} \\
                          &  \subseteq &  \ker (\ev{\lambda}|_{\cS}). \hspace{5em} \text{(by $(\Phi-w\cdot \Phi)(\nu) = \Phi(\nu)-\Phi(w^{-1}\cdot \nu) = 0$)}\\
  \end{eqnarray*}

  Therefore
  \[
    \ker (\ev{\lambda}|_{\cS})=\Span\Set{\Phi- w \cdot\Phi \mid \Phi\in\cS, \, w\in W_{\lambda}}.
  \]
  and the proposition follows.
\end{proof}


\subsection{Highest weight modules and coherent continuation representations}

\newcommand{\Rep}{\mathrm{Rep}}
Let $\b$ be a  Borel subalgebra of $\g$. Let $\Rep(\g,\b)$ denote the category of finitely generated $\g$-modules that are unions of finite-dimensional $\b$-submodules, and  let $\Rep_\sfS(\g,\b)$ denote its full subcategory of the modules whose complex associated variety is contained in $\sfS$.
Write $\CK(\g,\b)$ and $\CK_\sfS(\g,\b)$ respectively for the Grothendieck groups of $\Rep(\g,\b)$  and $\Rep_\sfS(\g,\b)$, and form the   coherent continuation representations $\Coh_{\Lam}(\CK(\g,\b))$ and $\Coh_{\Lam}(\CK_\sfS(\g,\b))$.


    Let $H$ be a Cartan subgroup of $G$ such that its complexified Lie algebra $\h$ is contained in $\b$. Recall that a $(\g, H)$-module is defined to be a $\g$-module $V$ together with a locally-finite representation of $H$ on it such that
     \begin{itemize}
     \item
        $h\cdot (X\cdot (h^{-1}\cdot u))=(\Ad_h(X))\cdot u$, for all $h\in H, X\in \CU(\g), u\in V$ ($\Ad$ stands for the Adjoint representation);
        \item the differential of the representation of $H$ and the restriction of the representation of $\g$ yields the same representation of $\h$ on $V$.
     \end{itemize}

Let $\Rep(\g,H,\b)$ denote the category of finitely generated $(\g, H)$-modules that  are unions of finite-dimensional $\b$-submodules.
We define the subcategory $\Rep_\sfS(\g,H,\b)$ of $\Rep(\g,H,\b)$,  the Grothendieck groups $\CK(\g,H, \b)$ and $\CK_\sfS(\g,H, \b)$, and the coherent continuation representations $\Coh_{\Lam}(\CK(\g,H, \b))$ and  $\Coh_{\Lam}(\CK_\sfS(\g,H, \b))$, as before.

 \begin{prop}\label{lem0022}
The representation $\Coh_{\Lam}(\CK_\sfS(\g,H, \b))$ of $W_{[\lambda]}$ is isomorphic to a subrepresentation of $(\Coh_{\Lam}(\CK_\sfS(\g, \b)))^k$, for some $k\in \BN$.
     \end{prop}
\begin{proof}


Write $\Inn_H$ for the Zariski closure of the image of $H$ under  the adjoint representation $G\rightarrow \Inn(\g)$, which is an algebraic torus. Write $Q_H$ for the group of algebraic characters of $\Inn_H$ (which is isomorphic to  the root lattice). By pulling-back through the homomorphism $H\rightarrow \Inn_H$, we view $Q_H$ as a set of characters on $H$.
The tensor product $\beta\otimes \gamma\in \Irr(H)$ is defined for every $\beta\in Q_H$ and $\gamma\in \Irr(H)$. This yields a free action of $Q_H$ on the set $\Irr(H)$.

For each $Q_H$-orbit $\Gamma\subset \Irr(H)$, write $\Rep_{\sfS,\Gamma}(\g,H,\b)$ for the full subcategory of $\Rep_\sfS(\g,H,\b)$ whose objects are  the modules $V$ such that every irreducible subquotient of $V|_H$ ($V$ viewed as a representation of $H$) belongs to $\Gamma$. Write $\CK_{\sfS,\Gamma}(\g,H,\b)$ for the  Grothendieck group of the category $\Rep_{\sfS,\Gamma}(\g,H,\b)$.
Then  we have a decomposition
\[
\CK_\sfS(\g,H,\b)=\bigoplus_{\Gamma\in Q_H\backslash \Irr(H)} \CK_{\sfS,\Gamma}(\g,H,\b),
\]
of $\mathcal R(\g)$-modules, and
\[
\Coh_{\Lam}(\CK_\sfS(\g,H, \b))=\bigoplus_{i=1}^k  \Coh_{\Lam}(\CK_{ \sfS,\Gamma_i}(\g,H,\b)),
\]
for a finite number of orbits $\Gamma_1, \Gamma_2, \cdots, \Gamma_k\in Q_H\backslash \Irr(H)$ ($k\in \bN$). Thus it remains to show that $ \Coh_{\Lam}(\CK_{ \sfS,\Gamma}(\g,H,\b))$ is isomorphic to a subrepresentation of  $\Coh_{\Lam}(\CK_\sfS(\g, \b))$.



For each $\gamma\in \Gamma$, put
\[
  M(\gamma):=\CU(\g)\otimes_{\CU(\b)} \gamma,
\]
which is a module in $\Rep_\Gamma(\g,H,\b)$, where the $\CU(\g)$-action is given by the left multiplication, and the $H$-action is given by
\[
 h\cdot (X\otimes u):=\Ad_h(X)\otimes h \cdot u, \qquad h\in H, \, X\in \CU(\g),\, u\in \gamma.
\]

Note that $\{ M(\gamma)\}_{\gamma\in \Gamma}$ is a basis of the space % (reference?)
\[
\CK_{\Gamma}(\g,H,\b):=\CK_{\Nil(\g^*),\Gamma}(\g,H,\b).
\]
Thus the forgetful functor
\[
   \Rep_{\Gamma}(\g,H,\b):=\Rep_{\Nil(\g^*),\Gamma}(\g,H,\b)\rightarrow  \Rep(\g,\b)
\]
induces an injective linear map
\[
    \CK_{\Gamma}(\g,H,\b)\rightarrow  \CK(\g,\b).
\]
This map is a $\mathcal R(\g)$-module homomorphism, and induces an injective
$\mathcal R(\g)$-module homomorphism
\[
    \CK_{\Gamma,\sfS}(\g,H,\b)\rightarrow  \CK_\sfS(\g,\b).
\]
The above homomorphism induces an embedding
\[
    \Coh_{\Lam}(\CK_{ \sfS,\Gamma}(\g,H,\b))\rightarrow  \Coh_{\Lam}(\CK_\sfS(\g,\b)),
\]
and the proposition follows.
\end{proof}




\subsection{A result of Casian}

 Similar to the subspace $\Grt_{\lambda, \sfS}(G)\subset \Grt(G)$, we define the subspace $\Grt_{\lambda, \sfS}(\g,H,\b)\subset \Grt(\fgg,H,\b)$ in the obvious way.
Let $\{H_1, H_2, \cdots, H_r\}$ ($r\in \bN^+$) be  a set of representatives of the
conjugacy classes of Cartan subgroups of $G$. For each $i=1,2,\cdots, r$, fix a Borel subalgebra $\b_i$ of $\g$ that contains the complexified Lie algebra of $H_i$.

 \begin{prop}\label{cor:HC.embed}
  % [\cite{Cas}*{Theorem~3.1}]
 There is an injective $\mathcal R(\g)$-module homomorphism
 \[
\gamma_{\mathrm g}: \Grt(G)\rightarrow  \bigoplus_{i=1}^{r} \Grt(\fgg,H_{i},\b_{i})
 \]
 such that
 \be\label{gammag}
   \gamma_{\mathrm g}(\Grt_{\lambda, \sfS}(G))\subset  \bigoplus_{i=1}^{r} \Grt_{\lambda, \sfS}(\fgg,H_{i},\b_{i})
 \ee
 for any $\lambda\in \hha^*$ and any $\Inn(\g)$-stable Zariski closed subset $\sfS$ of $\Nil(\g^*)$.

 \end{prop}
\begin{proof}
This follows from the work of Casian (\cite{Cas}). See also {\cite{Mc}}. Since the proposition is not explicitly stated in \cite{Cas}, we briefly recall the argument of Casian for the convenience of the reader.



 Let $\n_i$ denote the nilpotent radical of $[\g,\g]\cap \b_i$ ($i=1,2, \cdots,r$).
 For every $q\in \Z$, let $\gamma_{\n_i}^q$ denote the $q$-th right derived functor of the following left exact functor from the category of $\g$-modules to itself:
 \[
   V\rightarrow \{u\in V\,|\, \n_i^k \cdot v=0\textrm{ for some $k\in \bN^+$}\}.
 \]

Fix a Cartan involution $\theta$ of $G$ and write $K$ for its fixed point group (which is a maximal compact subgroup of $G$).  Without loss of generality we assume that all $H_i$'s are $\theta$-stable.

For every Casselman-Wallach representation $V$ of $G$, write $V_{[K]}$ for the space of $K$-finite vectors in $V$, which is a $(\g,K)$-module of finite length. Then   $\gamma_{\n_i}^q(V_{[K]})$ is naturally a representation in $\Rep(\g, H_i, \b_i)$ (\cite[Corollary 4.9]{Cas}).

We define a linear map
 \[
\gamma_{\mathrm g}: \Grt(G)\rightarrow  \bigoplus_{i=1}^{r} \Grt(\fgg,H_{i},\b_{i})
 \]
 given by
 \[
   \gamma_{\mathrm g}(V)= \left\{\sum_{q\in \Z} (-1)^{q} \gamma^{q}_{\n_i}(V_{[K]})\right\}_{i=1,2, \cdots, r}
 \]
for every Casselman-Wallach representation $V$ of $G$. The Osborne conjecture (see \cite[Theorem 3.1]{Cas}) and \cite[Corollary 4.9]{Cas}) implies that the map $\gamma_{\mathrm g}$ is injective.

Proposition 4.11 of \cite{Cas} implies that the functor $\gamma_{\n_i}^q$ commutes with tensor product with the finite-dimensional representations. Thus $\gamma_{\mathrm g}$ is a $\mathcal R(\g)$-homomorphism. Finally, \cite[Corollary 4.15]{Cas} implies that $\gamma_{\mathrm g}$ satisfies the property in \eqref{gammag}.
\end{proof}




Proposition \ref{cor:HC.embed} implies that the representation $\Coh_{\Lam}(\CK_\sfS(G))$ of $W_{[\lambda]}$ is isomorphic to a subrepresentation of
$\bigoplus_{i=1}^r \Coh_{\Lam}(\CK_{\sfS}(\g,H_i,\b_i))$. Together with Proposition \ref{lem0022}, this implies the following result.

 \begin{prop}\label{lem0033}
The representation $\Coh_{\Lam}(\CK_\sfS(G))$  of $W_{[\lambda]}$ is isomorphic to a subrepresentation of $(\Coh_{\Lam}(\CK_\sfS(\g,\b)))^k$, for some $k\in \BN$.
     \end{prop}




\subsection{Blocks and coherent continuation representations}% and coherent continuation representations}

Fix a Cartan subalgebra $\h$ of $\g$ that is contained in $\b$. Then $\h$ is identified with $\hha$ (since $\b$ has been fixed) and we view $\lambda$ as an element of $\h^*$.
Define the Verma module
\[
  \mathrm M(\lambda):=\mathrm M(\g,\b,\lambda):=\CU(\g)\otimes_{\CU(\b)} \C_{\lambda-\rho},
\]
where $\rho\in \h^*$ is the half sum of the weights of $\b$, $\C_{\lambda-\rho}$ is the one-dimensional $\h$-module corresponds to the character $\lambda-\rho\in \h^*$, and every $\h$-module is viewed an a $\b$-module as usual.
Write
$\oL(\lambda)=\oL(\g,\b,\lambda)$ for the unique irreducible quotient of $ \mathrm M(\lambda)$.

Let $\mathrm{Block}_{\lambda}(\g, \b)$ denote the full subcategory of $\mathrm{Rep}(\g, \b)$ consisting of the modules with generalized infinitesimal character  $\lambda$ whose weights are contained in $\lambda-\rho+Q$, which is called a block in  $\mathrm{Rep}(\g, \b)$. Let $\CK(\mathrm{Block}_{\lambda}(\g, \b))$ denote the Grothendieck group of this category. Then both \[
\textrm{ $\{\oL(\g,\b, w \lambda)\}_{w\in W_{[\lambda]}/W_\lambda}\ \ $ and $\ \  \{\mathrm M(\g,\b, w \lambda)\}_{w\in W_{[\lambda]}/W_\lambda}$}
\]
 are bases of   $\CK(\mathrm{Block}_{\lambda}(\g, \b))$.

The following result is a theorem of Soergel, in a weak form that we require.

\begin{thm}\label{soer}\cite[Section 2.5, Theorem 11]{Soergel}
Let $\g_i$ be a  reductive complex  Lie algebras, with a Borel subalgebra $\b_i$ and a Cartan subalgebra $\h_i\subset \b_i$ $(i=1,2)$. Let $W_i\subset \GL(\h_i)$ be the Weyl group of $\g_i$ and let $\lambda_i\in \h_i^*$ be dominant.  Write $Q_i\subset \h_i^*$ for the root lattice of $\g_i$. Put $[\lambda_i]:=\lambda_i+Q_i$, and let
$
 W_{\lambda_i}\subset W_{[\lambda_i]}
$
denote the stabilizers of $\lambda_i$ and $ [\lambda_i]$ in $W_i$, respectively. Suppose that  there is a  group
isomorphism $\varphi: W_{[\lambda_1]}\rightarrow W_{[\lambda_2]}$ that  takes the set of simple reflections in $W_{[\lambda_1]}$ onto the set of simple reflections in $W_{[\lambda_2]}$, and takes $W_{\lambda_1}$  onto  $ W_{\lambda_2}$. Then there is a linear isomorphism   $\CK(\mathrm{Block}_{\lambda_1}(\g_1, \b_1))\rightarrow \CK(\mathrm{Block}_{\lambda_2}(\g_2, \b_2))$ that sends $\mathrm L(\g_1,\b_1,w_1\lambda_1)$ to $\mathrm L(\g_2,\b_2,\varphi(w_1)\lambda_2)$ and sends $\mathrm M(\g_1,\b_1,w_1\lambda_1)$ to $\mathrm M(\g_2,\b_2,\varphi(w_1)\lambda_2)$, for all $w_1\in W_{[\lambda_1]}$.

\end{thm}



% \subsection{Coherent continuation representations}

 Recall that $Q\subset \h^*$ is the root lattice.
For each $Q$-coset  $\Lambda\subset \h^*$,
 let $\mathrm{Rep}_{\Lambda}(\g, \b)$ denote the full subcategory of $\mathrm{Rep}(\g, \b)$ consisting of the modules whose weights are contained in $\Lambda-\rho$.
 Write $\CK_{ \Lambda}(\g, \b)$ for the Grothendieck group of this subcategory. Then
 \[
   \CK(\g, \b)=\bigoplus_{\Lambda\in Q\backslash \h^*} \CK_{ \Lambda}(\g, \b)
 \]
 and hence
\[
 \Coh_{\Lam}( \CK(\g,\b))=\bigoplus_{\Lambda\in Q\backslash \h^*}\Coh_{\Lam}(\CK_{ \Lambda}(\g, \b) ).
\]
Note that
\[
  \Coh_{\Lam}(\CK_{ \Lambda}(\g, \b))\neq \{0\} \ \ \textrm{ only if } \ \ \Lambda=w \cdot \Lam \textrm{ for some $w\in W$}.
\]
Thus
\be\label{decLam}
 \Coh_{\Lam}( \CK(\g,\b))=\bigoplus_{w\in W/W_\Lam}\Coh_{\Lam}(\CK_{ w \Lam}(\g, \b) ).
\ee


Set
\[
  W_\Lam':=\{w\in W\,|\,  w\Delta_\Lam^+ = \Delta_{w\Lam}^+\}.
\]
Then the group multiplications yield a bijective map
\[
  W_\Lam'\times W_\Lam\rightarrow W.
\]


Let $w'\in W_\Lam'$. %In the rest of this subsection, suppose that $w'\in  W_\Lambda'$ and $w_0\in  W_\Lambda$.
By evaluating at an element $\nu\in \Lam$, we get a linear map
\be\label{isoev}
 \mathrm{ev}_\nu:   \Coh_{\Lam}( \CK_{w'\Lam}(\g,\b))\rightarrow \CK(\mathrm{Block}_{w'\nu}(\g, \b)).
\ee

The following lemma is an analogue of Lemma \ref{lem21} for highest weight modules.

\begin{lem}\label{surv}
The evaluating map \eqref{isoev} is surjective for all $\nu\in \Lam$, and  is bijective when $\nu$ is regular.
\end{lem}
\begin{proof}
The proof of Lemma \ref{lem21} works for this case. See also \cite[Theorem 7.7]{Mil}.
\end{proof}




For every $w_0\in W_\Lam$, define a map
\[
\begin{array}{rcl}
  \Psi_{w'w_0}:=  \Psi_{\g,\b,\Lam, w'w_0}: \Lam&\rightarrow  &\CK_{w' \Lam}(\g,\b), \\
   \nu&\mapsto& \mathrm M(\g,\b, w' w_0 \nu).
   \end{array}
\]
Then $ \Psi_{w'w_0}\in  \Coh_{\Lam}( \CK_{w'\Lam}(\g,\b))$, and Lemma  \ref{surv}  implies  that
\be\label{basisv}
\textrm{$\{ \Psi_{w'w_0}\}_{w_0\in W_\Lam}$ is a basis of $ \Coh_{\Lam}( \CK_{w'\Lam}(\g,\b))$.}
\ee
Theorem \ref{soer}  and Lemma  \ref{surv} imply that there is a unique element
\[
\overline \Psi_{w'w_0}:=\overline \Psi_{\g,\b, \Lam, w'w_0}\in  \Coh_{\Lam}( \CK_{ w'\Lam}(\g,\b))
\]
 such that
\[
\overline \Psi_{w'w_0}(\nu)=  \oL(\g,\b, w' w_0 \nu) \quad \textrm{ for  every $\nu\in \Lam$ that is regular and dominant.}
\]
Then
\begin{equation*}%\label{basisi}
\textrm{$\{\overline \Psi_{w'w_0}\}_{w_0\in W_\Lam}$ is also a basis of $ \Coh_{\Lam}( \CK_{w'\Lam}(\g,\b))$.}
\end{equation*}
We view $ \Coh_{\Lam}( \CK_{w'\Lam}(\g,\b))$ as a basal space with this basis.

Extend the representation of $W_\Lam$ on  $\Coh_{\Lam}( \CK_{w'\Lam}(\g,\b))$ to the group
\[
W_\Lam\times W_\Lam\supset \{1\}\times W_\Lam=W_\Lam
\]
 such that
\[
  (w_1, w_2)\cdot \Psi_{w'w_0}= \Psi_{w' w_1 w_0 w_2^{-1}}\quad \textrm{ for all } w_0, w_1, w_2 \in W_\Lam.
\]
Write $\Coh^{LR}_{\Lam}( \CK_{w'\Lam}(\g,\b))$ for the basal space  $\Coh_{\Lam}( \CK_{w'\Lam}(\g,\b))$ equipped with the above representation of $W_\Lam\times W_\Lam$.
%Write $\widetilde \Coh_{\Lam}( \CK_{w'\Lam}(\g,\b))$ for the basal space  $\Coh_{\Lam}( \CK_{w'\Lam}(\g,\b))$  equipped with the above representation of $W_\Lam\times W_\Lam$.



\subsection{Primitive ideals and Goldie rank polynomials}\label{secGoldie}
Write
 $\oJ(\lambda)=\oJ(\g,\b,\lambda)$ for the annihilator ideal of $\oL(\lambda)$.
Write $\h=\h_{\mathrm s}\oplus \c$, where $\h_{\mathrm s}=\h\cap [\g,\g]$ and $\c$ is the center of $\g$. Then $\h^*=\h_\mathrm s^*\oplus \c^*$.

Let $w\in W$. % and $\Lambda\in \h^*/Q$. %, where $w'\in W_\Lam'$ and $w_0\in W_\Lam$.
Then there is a unique
polynomial function $\tilde p_{\g,\Lam,w}$ on $\h^*$ such that
\begin{itemize}
\item it is $\c^*$-invariant (under the translations);
\item for all $\nu\in \Lam$ that is regular and dominant,
\[
 \tilde p_{\g,\Lam,w}(\nu)=\textrm{Goldie rank of } \CU(\g)/\oJ(w \cdot\nu).
\]
(See \cite[Section 5.12]{J1} and \cite[Section 2.10]{J.av}.)
\end{itemize}

 Write $\check \Delta:=\{\alpha^\vee\,|\, \alpha\in \Delta\}\subset \h$ for the set of coroots. Recall the set
\[
  \Delta_\Lam:=\{\alpha\in \Delta \,|\, \la \lambda,\alpha^\vee\ra \in \Z\}
\]
and put
\[
  \check \Delta_\Lam:=\{\alpha^\vee  \,|\, \alpha \in  \Delta_\Lam \}.
\]



Let $\check \g$ denote the Langlands dual of $\g$ so that $\h^*$ is identified with a Cartan subalgebra of $\check \g$ and $\check \Delta$ is identified with the root system of $\check \g$.  Let $\check \g_\Lam$ denote
 the Lie subalgebra of $\check \g$ containing $\h^*$ whose root system equals $\check \Delta_\Lam$. Let $\g_\Lam$ denote the Langlands dual of $\check \g_\Lam$ so that
$\h$ is identified with a Cartan subalgebra of $\check \g_\Lam$ and $\Delta_\Lam$ is identified with the root system of $\g_\Lam$. Then the Weyl group of $\check \g_\Lam$ is identified with $W_\Lam$. Write $Q_\Lam\subset \h^*$ for the root lattice for $\g_\Lam$. %subgroup generated by $\Delta_\Lam$.



\begin{lem} \label{grp1}
If $w\in W_\Lam$, then the polynomial function
$\tilde p_{\g,\Lam,w}$ equals a nonzero scalar multiple of $\tilde p_{\g_\Lam,Q_\Lam,w}$.
\end{lem}

\begin{proof}
Recall the Jantzen matrix $\{a_{\g,\Lam}(w_1, w_2)\in \Z \}_{w_1, w_2\in W_\Lam}$ that is determined by the equality
\[
  \oL(w_1\nu)=\sum_{w_2\in W_\Lam}  a_{\g,\Lam}(w_1, w_2) \cdot  \mathrm M(w_2 \nu)   \qquad (\textrm{as elements of $\CK(\g,\b)$}), \quad w_1\in W_\Lam.
\]
for some (and all) $\nu\in \Lam$ that is dominant and regular (see \cite[Section 2.15]{Jan}).  By Theorem \ref{soer} (Soegel's theorem), we have that
\be\label{soergel1}
  a_{\g,\Lam}(w_1, w_2)=a_{\g_\Lam,Q_\Lam}(w_1, w_2), \qquad w_1, w_2\in W_\Lam.
\ee

Suppose that $w\in W_\Lam$. Define a polynomial function $p_{\g, \Lam, w}$ on $\h^*\times \h$ by
\be\label{polynomial}
  p_{\g, \Lam, w}(\nu, x):= \sum_{w'\in W_\Lam} a_{\g,\Lam}(w, w')\cdot \la w' \nu, x\ra^m,\qquad \nu\in \h^*, \ x\in \h,
\ee
where $m$ is the smallest non-negative integer that makes the right-hand side of \eqref{polynomial} a nonzero polynomial function.
Then
\[
    p_{\g, \Lam, w}(\nu, x)= \tilde p_{\g,\Lam,w}(\nu)\cdot \tilde p'_{\g,\Lam,w}(x), \qquad \nu\in \h^*, \ x\in \h,
\]
for a unique polynomial function $\tilde p'_{\g,\Lam,w}$ on $\h$ (see \cite{King} and \cite[Section 5.1]{J.hw}).

Applying the above argument to $\g_\Lam$, we have that
\[
    p_{\g_\Lam, Q_\Lam, w}(\nu, x)= \tilde p_{\g_\Lam,Q_\Lam,w}(\nu)\cdot \tilde p'_{\g_\Lam,Q_\Lam,w}(x), \qquad \nu\in \h^*, \ x\in \h.
\]
By \eqref{soergel1}, we have that $ p_{\g, \Lam, w}= p_{\g_\Lam, Q_\Lam, w}$, and therefore the lemma follows.
\end{proof}



Write $w=w'w_0$, where $w'\in W_\Lam'$ and $w_0\in W_\Lam$.


\begin{lem} \label{grp2}
The  equality
\[
\tilde p_{\g,\Lam,w}={w'}^{-1}\cdot \tilde p_{\g, w'\Lam,w'w_0{w'}^{-1}}
\]
holds.

\end{lem}
\begin{proof}
Suppose that $\nu\in \Lam$ is regular and dominant. Then $w'\nu\in w'\Lam$ is also regular and dominant. Thus we have that
\begin{eqnarray*}
   \tilde p_{\g,\Lam,w}(\nu)&=&\textrm{Goldie rank of } \CU(\g)/\oJ(w\nu)\\
   &=&\textrm{Goldie rank of } \CU(\g)/\oJ((w'w_0w'^{-1})(w'\nu))\\
   &=&  \tilde p_{\g,w'\Lam,w'w_0w'^{-1}}(w'\nu)\\
    &=& ( w'^{-1}\cdot \tilde p_{\g,w'\Lam,w'w_0 w'^{-1}})(\nu).
\end{eqnarray*}
This implies the lemma.
\end{proof}


\begin{prop} \label{grp3}
The polynomial function
$
\tilde p_{\g,\Lam,w}$ is a nonzero scalar multiple of $\tilde p_{\g_\Lam,Q_\Lam,w_0}$.
\end{prop}
\begin{proof} We first note that
\[
  w'^{-1}\cdot \tilde p_{\g_{w'\Lam},Q_{w'\Lam},w'w_0w'^{-1}}= \tilde p_{\g_\Lam,Q_\Lam,w_0}.
\]
From Lemma \ref{grp1}, $\tilde p_{\g, w'\Lam,w'w_0{w'}^{-1}} $ is a nonzero scalar multiple of $\tilde p_{\g_{w'\Lam},Q_{w'\Lam},w'w_0{w'}^{-1}}$. Thus
\[
  w'^{-1}.\tilde p_{\g, w'\Lam,w'w_0{w'}^{-1}}= \textrm{a nonzero scalar multiple of }\tilde p_{\g_\Lam,Q_\Lam,w_0},
\]
Therefore  the proposition follows, in view of Lemma \ref{grp2}.
 \end{proof}


 For every $\sigma\in \Irr(W)$, its fake degree is defined to be
  \[
 a(\sigma):=\min\{a\in \BN\,|\, \sigma \textrm{ occurs in the $a$-th symmetric power $\oS^a(\h)$}\}.
 \]
This is well-defined since every  $\sigma\in \Irr(W)$ occurs in the symmetric power $\oS(\h)=\bigoplus_{a\in \BN}\oS^a(\h)$. The representation $\sigma$ is said to be univalent if it occurs in $\oS^{a(\sigma)}(\h_{\mathrm s})$ with multiplicity one. Note that $\h_{\mathrm s}=\h/\c=\h/\h^W$, where $\h^W$ denotes the $W$-fixed vectors in $\h$.

For every univalent irreducible representation $\sigma_0$ of $W_\Lam$, the $W$-subrepresentation of $\oS^{a(\sigma_0)}(\h_{\mathrm s})$ generated by
\[
  \C\otimes \sigma_0\subset \oS^0(\h_{\mathrm s}^{W_\Lam})\otimes \oS^{a(\sigma_0)}(\h_{\mathrm s}\cap [\g_\Lam, \g_\Lam])\subset  \oS^{a(\sigma_0)}(\h_{\mathrm s})
\]
is irreducible and univalent, with the same fake degree as that of $\sigma_0$ (This result is due to Macdonald, Lusztig, and Spaltenstein. See \cite[Chapter 11]{Carter}). This irreducible representation of $W$ is called the $j$-induction of $\sigma_0$, to be denoted by $j_{W_\Lam}^W(\sigma_0)$.


Write $\sigma_{\g,\Lam,w}$ for the $W$-subrepresentation of $\oS(\h)=\C[\h^*]$ (the space of polynomial functions) generated by $\tilde p_{\g,\Lam,w}$.
Similarly, we also have a $W_\Lam$-subrepresentation  $\sigma_{\g_\Lam,Q_\Lam,w_0}$ of $\C[\h^*]$  generated by   $\tilde p_{\g_\Lam,Q_\Lam,w_0}$. It is well-known that $\sigma_{\g_\Lam,Q_\Lam,w_0}$ is a univalent irreducible representation of $W_\Lam$, and $\tilde p_{\g_\Lam,Q_\Lam,w_0}$ is a homogeneous polynomial function whose degree equals the fake degree $a(\sigma_{\g_\Lam,Q_\Lam,w_0})$ (see \cite[Theorem 5.4]{J2}). By Proposition  \ref{grp3},
\be\label{jind0}
\sigma_{\g,\Lam,w}
=  j_{W_\Lam}^W(\sigma_{\g_\Lam,Q_\Lam,w_0}).
\ee



For every $w\in W$, it is known that $\sigma_{\g,\Lam,w}$ is irreducible and Springer (see \cite{Ho} and \cite[Section 2.10]{J.av}).  Here an irreducible representation of $W$ is said to be Springer if it corresponds to a trivial local system of a nilpotent orbit in $\g^*$, under the Springer correspondence. Write $\CO_{\g, \Lam,w}\subset \g^*$ for the corresponding nilpotent orbit:
\begin{equation*} %\label{nilatg}
\CO_{\g, \Lam,w} :=\mathrm{Springer}^{-1}(\sigma_{\g,\Lam,w}).
\end{equation*}

\begin{lem}\label{associatedv} (\cite[Theorem 3.9]{J.av})
For every $w\in W$ and every $\nu \in \Lam$ that is regular and dominant, the associated variety of $\oJ(w\nu)$ equals the Zariski closure of $\CO_{\g, \Lam,w}\subset \g^*$.
\end{lem}



\subsection{Cell representations in the coherent continuation setting}\label{seccell}

\begin{defn}
Let $E$ be a  finite group. A basal representation $V$ of $E$ is basal vector space carrying a representation of $E$. A basal subrepresentation of a basal representation $V$ is a subrepresentation of $V$ that is simultaneously a basal subspace.
\end{defn}

For examples, $ \Coh_{\Lam}( \CK(G))$ and $ \Coh_{\Lam}( \CK_{w'\Lam}(\g,\b))$ (recall that $w'\in W_\Lam'$) are basal representations of $W_\Lam$, and $ \Coh^{LR}_{\Lam}( \CK_{w'\Lam}(\g,\b))$ is a basal representations of $W_\Lam\times W_\Lam$. Theorem \ref{soer} implies that the basal representations  $  \Coh_{\Lam}( \CK_{w'\Lam}(\g,\b))$ and  $\Coh^{LR}_{\Lam}( \CK_{w'\Lam}(\g,\b))$ only depend on the set of simple reflections in $W_\Lam$ (see also  see \cite{BV2}*{Corollary~2.3}). In particular, they do not depend on $w'\in W_\Lam'$.



Let $V$ be a basal representation of a finite group $E$, with basal elements $\CB\subset V$. For each subset $\mathcal C\subset \CB$, write $\la \CC\ra$ for the smallest basal subrepresentation of $V$ containing $\CC$. For each $\phi\in \CB$, write $\la \phi\ra:=\la \{\phi\}\ra$ for simplicity. On the set $\CB$, define a preorder $\leq$ by
\[
  \phi_1\leq \phi_2  \quad \textrm{ if and only if } \quad \la \phi_1\ra \subset \la \phi_2 \ra,
\]
and define an equivalence relation $\approx$ by
\[
  \phi_1 \approx \phi_2  \quad \textrm{ if and only if } \quad \la \phi_1 \ra =\la \phi_2 \ra \qquad (\phi_1, \phi_2\in \CB).
\]
An equivalence class of the relation $\approx$ on the set $\CB$ is called a cell in $V$.

We say that a subset $\CC$ of $\CB$ is order closed if for any $\phi\in \CB$, we have $\phi\in \CC$ whenever $\phi\leq \phi'$ for some $\phi'\in \CC$.
This is equivalent to saying that $\la \CC\ra$ is spanned by $\CC$. In general, write $\overline \CC$ for the smallest order closed subset of $\CB$ containing $\CC$.

When $\CC$ is a cell in $V$, define the cell representation attached to $\CC$ by
\[
  V(\CC):=\la \overline \CC \ra/ \la \overline \CC\setminus \CC\ra.
\]
Note that the set $ \overline \CC\setminus \CC$ is order closed, and $\{\phi+ \la \overline \CC\setminus \CC\ra\}_{ \phi\in \CC}$ is a basis of $V(\CC)$.
%More generally, we have the following elementary lemma whose proof is omitted.

%\begin{lem}\label{unioncell}
%Let $\CC$ be a union of cells in $V$. Assume that for all $\phi_1,\phi_2\in \CC$, $\phi_1\leq \phi_2$ implies that $\phi_1 \approx \phi_2$. Then  $ \overline \CC\setminus \CC$ is also order closed, and $\{\phi+ \la \overline \CC\setminus \CC\ra\}_{ \phi\in \CC}$ is a basis of $V(\CC)$.
%\end{lem}



We are particularly interested in the basal representation $\Coh^{LR}_{\Lam}( \CK_{ w'\Lam}(\g,\b))$ of $W_\Lam\times W_\Lam$. Its set of basal elements is
\[
  \CB:= \{ \overline \Psi_{\g,\b, \Lam, w'w_0}\mid w_0\in W_\Lam\}.
\]
We will write
 $ \la \CC\ra_{LR}:=\la \CC \ra $, the smallest basal subrepresentation of $\Coh^{LR}_{\Lam}( \CK_{ w'\Lam}(\g,\b))$ containing $\CC$, where $\CC$ is a subset of $\CB$.

%For each subset $\CC$ of $\CB$, respectively write
%$\la \CC\ra_{L}$ and $ \la \CC\ra_{LR}$ for the  smallest basal subrepresentations of $ \Coh_{\Lam}( \CK_{w'.\Lam}(\g,\b))$ and $\widetilde \Coh_{\Lam}( \CK_{ w'.\Lam}(\g,\b))$ containing $\CC$.



Note that every irreducible representation of $W_\Lam$ is self-dual. Hence \eqref{basisv} implies that
\be\label{decocoh}
   \Coh^{LR}_{\Lam}( \CK_{w'\Lam}(\g,\b))\cong \bigoplus_{\sigma\in \Irr(W_\Lam)} \sigma\otimes \sigma
\ee
as representations of $W_\Lam\times W_\Lam$. By a double cell in  $\Irr(W_\Lam)$, we mean a set of the form
\[
  \{\sigma\in \Irr(W_\Lam)\mid \sigma\otimes \sigma\,  \textrm{ occurs in the cell representation $ \Coh^{LR}_{\Lam}( \CK_{w'\Lam}(\g,\b))(\mathcal C)$}\},
\]
where
 $\CC$ is a cell in  $ \Coh^{LR}_{\Lam}( \CK_{w'\Lam}(\g,\b))$. The isomorphism \eqref{decocoh} implies  that $\Irr(W_\Lam)$ is  disjoint union of
all its double cells. As noted previously, Theorem \ref{soer} implies that the notion of double cells in   $\Irr(W_\Lam)$ only depends on  the set of simple reflections in $W_\Lam$.

The following conjecture is widely anticipated, although to our knowledge no proof has appeared in the literature. See \cite[page 1055]{V4}.

\begin{conj}\label{conjcell}
For every cell $\CC$ in  $ \Coh_{\Lam}( \CK(G))$, the set
\[
 \{ \sigma\in \Irr(W_\Lam)\mid \sigma\textrm{ occurs in the cell representation $\Coh_{\Lam}( \CK(G))(\CC)$}\}
\]
is contained in a single double cell in $\Irr(W_\Lam)$.
\end{conj}


\begin{remark}
We call a double cell in  $\Irr(W_\Lam)$ a Lusztig double cell. We also call a cell in  $\Coh_{\Lam}( \CK(G))$ a Harish-Chandra cell, and the associated cell representation a Harish-Chandra cell representation.
\end{remark}

\begin{remark}
We note McGovern's observation (\cite[Page 213]{Mc}) which amounts to the assertion in \Cref{conjcell}. It appears to us that the argument is inadequate as presented.
\end{remark}

Recall that an irreducible representation $\sigma_0\in \Irr(W_\Lam)$ is special (in the sense of Lusztig) if and only if it is isomorphic to $\sigma_{\g_\Lam,Q_\Lam,w_0}$ for some $w_0\in W_\Lam$ (see \cite[Theorem 1.1]{BV2}). Let $ \Irr^{\mathrm{sp}}(W_\Lam)\subset  \Irr(W_\Lam)$ denote the subset of special representations.


\begin{lem}\label{doublecell0}
For every $\sigma_0\in  \Irr^{\mathrm{sp}}(W_\Lam)$, the set
\[
  \{ \overline \Psi_{\g,\b, \Lam, w'w_0}\in \Coh^{LR}_{\Lam}( \CK_{ w'\Lam}(\g,\b))\mid \sigma_{\g_\Lam, Q_\Lam, w_0}\cong \sigma_0 \}
\]
is a cell in $\Coh^{LR}_{\Lam}( \CK_{ w'\Lam}(\g,\b))$, and $\sigma_0$ is the unique special representation in the double cell in $\Irr(W_\Lam)$ attached to this cell.
\end{lem}
\begin{proof}
See \cite[Theorem 2.6 and Corollary 2.16]{BV2} and  \cite[Corollary 14.11]{V4}. Note that by Theorem \ref{soer}, we may assume without loss of generality that $\lambda$ is integral so that $\g=\g_\Lam$.
\end{proof}

For every cell $\CC$ in $\Coh^{LR}_{\Lam}( \CK_{ w'\Lam}(\g,\b))$, write
$\CO_\CC:= \CO_{\g, \Lam, w' w_0}$, where $w_0$ is an element of $W_\Lam$ such that    $\overline \Psi_{\g,\b, \Lam, w'w_0}\in \CC$. By Lemma \ref{doublecell0}, this is independent of the choice of $w_0$.



\begin{lem}\label{lemlr0}
Let $\CC_1$ and $\CC_2$ be two cells in  $\Coh^{LR}_{\Lam}( \CK_{ w'\Lam}(\g,\b))$ such that  $\la \CC_1\ra_{LR}\subset \la \CC_2\ra_{LR}$. Then
  \[
    \overline{\CO_{\CC_1}}\subset   \overline{\CO_{\CC_2}},
  \]
  where $\overline{\phantom A}$ indicates the Zariski closure. \end{lem}
  \begin{proof}
  For every basal element $\phi$ in  $\Coh^{LR}_{\Lam}( \CK_{ w'\Lam}(\g,\b))$, write $\la \phi\ra_L$ for the smallest $W_\Lam\times \{1\}$-stable basal subspace of $\Coh^{LR}_{\Lam}( \CK_{ w'\Lam}(\g,\b))$ containing $\phi$, and likewise write $\la \phi\ra_R$ for the smallest $\{1\}\times W_\Lam$-stable basal subspace of $\Coh^{LR}_{\Lam}( \CK_{ w'\Lam}(\g,\b))$ containing $\phi$.

  By the argument in \cite[Section 3]{BVUni}, it suffices to prove the following statement:
   if there exists $\phi_1\in \CC_1$ and $\phi_2\in \CC_2$ such that $\la \phi_1\ra_L\subset \la \phi_2\ra_L$ or $\la \phi_1\ra_R\subset \la \phi_2\ra_R$, then $ \overline{\CO_{\CC_1}}\subset   \overline{\CO_{\CC_2}}$.

 Write $\phi_i= \overline \Psi_{\g,\b, \Lam, w'w_i}$, where $w_i\in W_\Lam$ ($i=1,2$). Let $\nu$ be a regular dominant element in $\Lam$.
 We first assume that  $\la \phi_1\ra_R\subset \la \phi_2\ra_R$. Note that the analogue of Lemma \ref{lemirr11} in the setting of highest weight modules still holds (with the same proof). Thus
Lemma \ref{associatedv} implies that
\[
  \phi_2\in  \Coh_{\Lam}( \CK_{\overline{\CO_{\CC_2}},  w'\Lam}(\g,\b)),
\]
which further implies that
\[
 \phi_1\in  \la \phi_1\ra_R\subset \la \phi_2\ra_R\subset  \Coh_{\Lam}( \CK_{\overline{\CO_{\CC_2}},  w'\Lam}(\g,\b)).
\]
Thus $\oL(\g,\b, w'w_1\nu)\in \CK_{\overline{\CO_{\CC_2}},  w'\Lam}(\g,\b)$, and Lemma  \ref{associatedv} implies that $ \overline{\CO_{\CC_1}}\subset   \overline{\CO_{\CC_2}}$.

Now we assume that $\la \phi_1\ra_L\subset \la \phi_2\ra_L$. Put
$\phi_i':= \overline \Psi_{\g,\b, \Lam, w'w_i^{-1}}$ ($i=1,2$). By \cite[Lemma 5.2]{Lu}, we know that $\phi_i'\in \CC_i$, and by the argument in \cite[Section 3]{BVUni},
$\la \phi'_1\ra_R\subset \la \phi'_2\ra_R$. Therefore  $ \overline{\CO_{\CC_1}}\subset   \overline{\CO_{\CC_2}}$ by the earlier argument. This finishes the proof of the lemma.
  \end{proof}

\subsection{Proof of Theorem \ref{count2} via $\Coh_{\Lam}(\CK_\sfS(\g,\b))$}

Similar to  \eqref{decLam}, we have a decomposition (as $W_\Lam$-modules)
\be\label{decLam2}
 \Coh_{\Lam}( \CK_\sfS(\g,\b))=\bigoplus_{w'\in W/W_\Lam}\Coh_{\Lam}(\CK_{ \sfS, w'\Lam}(\g, \b) ),
\ee
where $\CK_{ \sfS, w'\Lam}(\g, \b)$ is the Grothendieck group of $\Rep_{ \sfS, w'\Lam}(\g, \b)$, and the latter is the full subcategory of
$\Rep(\g, \b)$ whose objects are modules that belong to both $\Rep_{ \sfS}(\g, \b)$ and $\Rep_{w'\Lam}(\g, \b)$.


Using Lemma \ref{associatedv}, the same argument as  the proof of \eqref{basiss} shows that  $\Coh_{\Lam}(\CK_{ \sfS, w'\Lam}(\g, \b) )$ is a basal subrepresentation of $\Coh_{\Lam}(\CK_{ w'\Lam}(\g, \b) )$ spanned by
\[
 \{ \overline \Psi_{\g,\b, \Lam, w'w_0}\mid w_0\in W_\Lam, \, \CO_{\g, \Lam, w'w_0}\subset \sfS \}.
\]
Put
\[
   \Irr_\sfS^{\mathrm{sp}}(W_\Lam):=\{\sigma_0\in  \Irr^{\mathrm{sp}}(W_\Lam)\mid \mathrm{Springer}^{-1}(j_{W_\Lam}^W \sigma_0)\subset \sfS\},
\]
and \eqref{jind0} implies that $\Coh_{\Lam}(\CK_{ \sfS, w'\Lam}(\g, \b) )$ is  spanned by
\be\label{setbasal}
 \{ \overline \Psi_{\g,\b, \Lam, w'w_0}\mid w_0\in W_\Lam, \,  \sigma_{\g_\Lam, Q_\Lam, w_0}\in   \Irr_\sfS^{\mathrm{sp}}(W_\Lam) \}.
\ee

Recall the set $\Irr_\sfS(W_\Lam)$ defined in \eqref{sfc}, which equals
\[
    \{\sigma \in \Irr(W_\Lam)\mid \textrm{$\sigma$ lies in the same double cell with an element of $ \Irr_\sfS^{\mathrm{sp}}(W_\Lam)$}  \}.
\]

Lemma \ref{lemlr0} implies that $  \Coh_{\Lam}( \CK_{\sfS, w'\Lam}(\g,\b))$ is a subrepresentation of $\Coh^{LR}_{\Lam}( \CK_{w'\Lam}(\g,\b))$.
\begin{prop}\label{cohbbs}
As a representation of $W_\Lam\times W_\Lam$, we have
\[
  \Coh^{LR}_{\Lam}( \CK_{\sfS, w'\Lam}(\g,\b))\cong \bigoplus_{\sigma\in \Irr_\sfS(W_\Lam)} \sigma\otimes \sigma.
\]

\end{prop}
\begin{proof}
Recall that
\[
  \CB:= \{ \overline \Psi_{\g,\b, \Lam, w'w_0}\mid w_0\in W_\Lam\}.
\]
Denote by $\CB_0$ the set \eqref{setbasal}, which is a basis of $\Coh_{\Lam}( \CK_{\sfS, w'\Lam}(\g,\b))$. By Lemma \ref{doublecell0}, it is a union of cells in  $\Coh^{LR}_{\Lam}( \CK_{\sfS, w'\Lam}(\g,\b))$. Choose a filtration
\[
  \CB_0\supset \mathcal B_1\supset \cdots \supset \mathcal B_k=\emptyset \qquad (k\in \BN)
\]
of $\CB_0$
such that
\begin{itemize}
\item
$\mathcal C_i:=\mathcal B_i\setminus \mathcal B_{i+1}$ is a cell in $\Coh^{LR}_{\Lam}( \CK_{ w'\Lam}(\g,\b))$ for all $i=0,1, \dots, k-1$, and
\item
for all $i_1, i_2\in \{0,1, \dots, k-1\}$,   $\la \CC_{i_1}\ra_{LR} \subset \la \CC_{i_2}\ra_{LR}$ implies that $i_1\geq i_2$.
\end{itemize}
Then it is elementary to see that  $\la \CB_i\ra_{LR}$ is spanned by $\CB_i$ ($i=0, 1,2, \cdots ,k$). Let $\sigma_i\in \Irr^{\mathrm{sp}}(W_\Lam)$ denote the unique special representation such that $\sigma_i\otimes \sigma_i$ occurs in $\Coh_{\Lam}( \CK_{w'\Lam}(\g,\b))(\CC_i)$. Then Lemmas \ref{associatedv} and \ref{doublecell0} imply that
\[
\{\sigma_0, \sigma_1, \dots, \sigma_{k-1}\}= \Irr_\sfS^{\mathrm{sp}}(W_\Lam).
\]

Now we have that
\begin{eqnarray*}
   &&\Coh^{LR}_{\Lam}( \CK_{\sfS, w'\Lam}(\g,\b))\\
   &\cong & \bigoplus_{i=0}^{k-1}  \la \CB_i\ra_{LR}/\la \CB_{i+1}\ra_{LR}\\
    &\cong & \bigoplus_{i=0}^{k-1}   \Coh^{LR}_{\Lam}( \CK_{\sfS, w'.\Lam}(\g,\b))(\CC_i)\\
    &\cong &\bigoplus_{\sigma\in \Irr_\sfS(W_\Lam)} \sigma\otimes \sigma.
\end{eqnarray*}
\end{proof}

Finally, Theorem \ref{count2} follows from Proposition \ref{lem0033}, \eqref{decLam2}, and Proposition \ref{cohbbs}.



\subsection{Proof of Theorem \ref{counteq}}  Recall that for every basal element $\Phi\in \mathcal B_\Lam(G)$,
\[
   \mathrm{AV}_\C(\Phi)= \mathrm{AV}_\C(\Phi(\nu))\quad \textrm{ for all $\nu\in \Lam$ that is regular and dominant}.
\]


\begin{lem}\label{hcgodie}
For every Harish-Chandra cell $\CC$ in $\Coh_{[\lambda]}(\CK(G))$, there is a special representation $\sigma_\CC\in \Irr^{\mathrm{sp}}(W_\Lam)$ such that $\sigma_\CC$ occurs in $\Coh_{[\lambda]}(\CK(G))(\CC)$ and
\[
  \mathrm{AV}_\C(\Phi)=\mathrm{Springer}^{-1}(j_{W_\Lam}^W \sigma_\CC)
\]
for every $\Phi\in \CC$.
\end{lem}
\begin{proof}
This is proved by King (\cite{King}). See \cite[Corollary 14.11]{V4}.
\end{proof}

\begin{lem}\label{cellsfs}
Assume that for every Harish-Chandra cell $\CC$ in $\Coh_{\Lam}(\CK(G))$, the set
\[
\{\sigma\in \Irr(W_{[\lambda]})\,|\, \textrm{$\sigma$ occurs in $\Coh_{\Lam}(\CK(G))(\CC)$}\}
\]
is contained in a single double cell in $\Irr(W_\Lam)$. Then for every $\sigma\in \Irr_\sfS(W_\Lam)$,
\be\label{eqsigma}
[ \sigma: \Coh_{\Lam}(\CK_\sfS(G))]=[\sigma:\Coh_{\Lam}(\CK(G))].
\ee
\end{lem}
\begin{proof}
Note that
\[
  \Coh_{\Lam}(\CK(G))\\
   \cong \bigoplus_{\CC \textrm{ is a cell in $\Coh_{\Lam}(\CK(G))$}} \Coh_{\Lam}(\CK(G))(\CC),
\]
and
 Lemma \ref{hcgodie} implies that
 \[
   \Coh_{\Lam}(\CK_\sfS(G))\cong  \bigoplus_{\CC \textrm{ is a cell in $\Coh_{\Lam}(\CK(G))$, $\sigma_\CC\in \Irr^{\mathrm{sp}}_\sfS(W_\Lam)$}} \Coh_{\Lam}(\CK(G))(\CC).
 \]
Then Lemma \ref{hcgodie}  and the assumption of the lemma imply \eqref{eqsigma}. \end{proof}

Under the assumption of Lemma \ref{cellsfs}, we have that
\begin{eqnarray*}
   && \sharp(\Irr_{\lambda,\sfS}(G)) \\
   &=&  \sum_{\sigma \in \Irr_\sfS(W_\Lam)} [1_{W_{\lambda}}: \sigma]\cdot [\sigma:\Coh_{\Lam}(\CK_\sfS(G))]   \quad (\textrm{by \eqref{leq002}})\\
    &= & \sum_{\sigma \in \Irr_\sfS(W_\Lam)} [1_{W_{\lambda}}: \sigma]\cdot [\sigma:\Coh_{\Lam}(\CK(G))]   \quad (\textrm{by Lemma \ref{cellsfs}}).\\
\end{eqnarray*}
This completes the proof of Theorem \ref{counteq}.

\section{An explicit formula of $\Coh_{\Lam}(\CK(G))$ \`a la Barbasch and Vogan}
%\label{secExplicit}

 Let $H$ be a  Cartan subgroup of $G$ with complexified Lie algebra $\h$. Write $\t$ for the complexified Lie algebra of the unique maximal compact subgroup of $H$. As before, denote by $\Delta_\h\subset \h^*$ the root system of $\g$. A root $\alpha\in \Delta_\h$ is called imaginary if $\alpha^\vee\in \t$. An imaginary root $\alpha\in \Delta_\h$ is said to be compact if the root spaces $\g_\alpha$ and $\g_{-\alpha}$ are contained in  the  complexified Lie algebra of a common compact subgroup of $G$.



Note that every Casselman-Wallach representation of $H$ is finite dimensional, and for every $\Gamma\in \Irr(H)$, there is a unique element $\mathrm d \Gamma\in \h^*$ such that the differential of $\Gamma$ is isomorphic to a direct sum of  one-dimensional representations of $\h$ attached to $\mathrm d  \Gamma$.

For every Borel subalgebra $\b$ of $\g$ containing $\h$, write
\[
  \xi_\b: \hha\rightarrow \h
\]
for the linear isomorphism attached to $\b$, whose transpose inverse is still denoted by $ \xi_\b: \hha^*\rightarrow \h^*$. Recall that $ \Delta^+\subset \hha^*$ denotes the set of positive roots. For every element $w$ in the abstract Weyl group $W$, put
\[
  \delta(w, \b):=\frac{1}{2}\cdot \sum_{\alpha \textrm{ is an imaginary root in $\xi_\b w \Delta^+$ }} \alpha- \sum_{\beta \textrm{ is a compact imaginary root in $\xi_\b w \Delta^+$ }}\beta\in \h^*.
\]

Denote by $Q_\h\subset \h^*$ the root lattice of $\g$. For every $\nu\in \h^*$, put $[\nu]:=\nu+Q_\h$.
Write $\widetilde \Irr_\Lam(H)$ for the set of all triples $\gamma =(\b, \Gamma, \nu)$ with the following properties:
\begin{itemize}
  \item $\b$ is a Borel subalgebra of $\g$ containing $\h$,  $\Gamma\in \Irr(H)$, and $ \nu \in \h^*$;
  \item $\mathrm d  \Gamma- \nu=\delta(1, \b)$;
  \item $\xi_\b(\Lam)=[\nu]$.
  \end{itemize}

Define the real Weyl group
\[
W_H:=(\textrm{the normalizer of $\h$ in $G$})/H.
\]
As usual, $Q_\h$ is identified with a group of characters on $H$. Define an action of the group $W_H\ltimes Q_\h$ on the set $ \widetilde \Irr_\Lam(H)$ by requiring that
\[
  \beta \cdot(\b, \Gamma, \nu):=(\b,\beta\otimes  \Gamma,  \beta+\nu)
\]
and
\[
  g \cdot(\b, \Gamma, \nu):=(g\cdot \b, g\cdot \Gamma,  g\cdot\nu)
\]
for all $\beta\in Q_\h$, $g\in W_H$ and $(\b, \gamma, \nu)\in \widetilde \Irr_\Lam(H)$.

On the other hand, we have the cross action of $W_\Lam$ on the set $ \widetilde \Irr_\Lam(H)$ (see \cite[Definition 8.3.1]{Vg}):
\begin{eqnarray*}
  && w\cdot (\b, \Gamma, \nu)\\
  &:=&((\xi_\b w^{-1} \xi_\b^{-1})\cdot \b,\, ((\xi_\b w^{-1} \xi_\b^{-1})\cdot \nu-\nu)\otimes (\delta(w^{-1}, \b)-\delta(1,\b))\otimes \Gamma,  \, (\xi_\b w^{-1} \xi_\b^{-1})\cdot \nu),
\end{eqnarray*}
for all $w\in W_\Lam$ and $(\b, \Gamma, \nu)\in \widetilde \Irr_\Lam(H)$. It is routine to check that these two actions are well-defined. Moreover,
the actions of $W_\Lam$ and $W_H$ commute with each other, and
\[
 w\cdot(\beta\cdot(w^{-1}\cdot(\b, \Gamma, \nu)))=((\xi_\b w\xi_\b^{-1})\cdot \beta)\cdot(\b, \Gamma, \nu)
\]
for all $w\in W_\Lam$, $\beta\in Q_\h$ and $(\b, \Gamma, \nu)\in \widetilde \Irr_\Lam(H)$. Thus the cross action descends to an action of $W_\Lam$ on the quotient set
\[
 \overline \Irr_\Lam(H):= (W_H\ltimes Q_\h)\backslash ( \widetilde \Irr_\Lam(H)).
\]


Since $G$ is in the Harish-Chandra's class, the real Weyl group $W_H$ is identified with a subgroup of the Weyl group $W_\h$ of $\g$ with respect to $\h$.
Write $\t_{\mathrm{im}}$ for the subspace of $\t$ spanned by the set $\{\alpha^\vee\mid \alpha \textrm{ is an imaginary root in $\Delta_\h$}\}$. Then $W_H$ stabilizers $\t_{\mathrm{im}}$, and we define a character
\[
  \mathrm{sgn}_\mathrm{im}: W_H\rightarrow \C^\times, \quad w\mapsto \textrm{(the determinant of the map $w: \t_{\mathrm{im}}\rightarrow \t_{\mathrm{im}}$)}.
\]
This is a quadratic character.

Let $\overline \gamma\in  \overline \Irr_\Lam(H)$. Write $W_{\overline \gamma}$ for the stabilizer of $\overline \gamma$ in $W_\Lam$.
Pick a Borel subalgebra $\b$ of $\g$ containing $\h$ that extends to a triple $(\b, \gamma, \nu)\in \overline \gamma$, which is unique up to the action of $W_H$.   It is clear that $\xi_\b w \xi_\b^{-1}\in W_H$ for all $w\in W_{\overline \gamma}$. We define a quadratic character
\[
  \mathrm{sgn}_{\overline \gamma}: W_{\overline \gamma}\rightarrow \C^\times, \qquad w\mapsto \mathrm{sgn}_\mathrm{im}(\xi_\b w \xi_\b^{-1}).
\]
This is independent of the choice of $\b$.

The following result is due to Barbasch-Vogan, in a suitably modified form from \cite{BV.W}*{Proposition~2.4}. As its proof follows the same line as that of \cite{BV.W}*{Proposition~2.4}, we will be content to state the precise result.
%based on the results of \cite{Vg}*{Chapter 8}.

\begin{thm}[{\cf \cite{BV.W}*{Proposition~2.4}}]
  \label{thm:cohHC}
 As a representation of $W_\Lam$,
  \[
    \Coh_{[\lambda]}(\CK(G)) \cong \bigoplus_{H} \bigoplus_{\overline \gamma\in \overline \Irr_\Lam(H)}
    \Ind_{W_{\overline \gamma}}^{W_\Lam}  \mathrm{sgn}_{\overline \gamma},
  \]
  where $H$ runs over a representative set of conjugacy classes of Cartan subgroups of $G$.
\end{thm}




\section{Special unipotent representations in type A}\label{sec:GL}

%Let $\star$, $G$, $\check G$, $\g$, $\check \g$, $\check \CO$, $\lambda_{\check \CO}$, $I_{\check \CO}$ and be as in 

In the rest of the paper we will freely use  the notation of Sections \ref{sec:defunip}-\ref{secorgp0}. Recall the element $\lambda_{\check \CO}\in W\backslash \hha^*$. %When no confusion is possible, we 
We pick an arbitrary representative of it and still write the resulting element by $\lambda_{\check \CO}\in \hha^*$. If $\star\in \{\widetilde A, \widetilde C\}$, we let $\CK'(G)$ denote the Grothendieck group of the category of genuine Casselman-Wallach representations of $G$. Otherwise, put $\CK'(G):=\CK(G)$. The notations $ \CK'_{\sfS}(G)$, $\CK'_{\lambda,\sfS}(G)$, etc., are similarly defined without further explanation.

As an obvious variation of Corollary \ref{cor:bound} for $\lambda=\lambda_{\check \CO}$, we have  that 
 \begin{equation}\label{boundc22}
     \sharp(\Unip_{\check \CO}(G)) =\sum_{\sigma\in \LC_{\lambda_{\check \CO}}} [\sigma: \Coh_{[\lambda_{\check \CO}]}(\CK_{\overline{\CO}}'(G))]  \leq \sum_{\sigma\in \LC_{\lambda_{\check \CO}}} [\sigma: \Coh_{[\lambda_{\check \CO}]}(\CK'(G))],
   \end{equation}
   and the equality holds if the analogous condition in Corollary \ref{cor:bound} holds for $\Coh_{\Lam}(\CK'(G))$.
Here $\overline{\CO}$ is the associated variety of the maximal ideal $I_{\check \CO}$, and $\CO$ is the unique open $G_\C$-orbit in  $\overline{\CO}$. 



\trivial[h]{In this section, suppose that $\star\in \{A^\R, A^\bH\}$ and $\ckG = \GL_{n}(\bC)$  ($n\geq 0$). Thus
\[
  G= \begin{cases}
    \GL_{n}(\bR)  & \text{if } \star=A^\R; \\
    \GL_{\frac{n}{2}}(\bH) & \text{if } \star =A^\bH. \\
  \end{cases}
\]
Note that $n$ has to be even in the latter case.
}

\subsection{Some Weyl group representations}


Let $\sfS_n$ denote the permutation group of the set $\{1,2, \dots, n\}$, which is identified with the Weyl group $W$ of $\g=\g\l_n(\C)$. Let $\YD_{n}$ be the set of Young diagrams of total size $n$.
We identify $\YD_{n}$ with the set $\overline{\Nil}(\g):= \GL_{n}(\bC)\backslash \Nil(\g\l_n(\C))$ of complex nilpotent orbits and
also with the set  $\Irr(\sfS_{n})$ via the Springer
correspondence (see \cite{Carter}*{11.4}).
More specifically, for $ \cO \in \overline \Nil( \g)=\YD_{n}$, the Springer
correspondence is given by Macdonald's construction for $\sfS_{n}$ via $j$-induction:
\[
  \Spr( \cO) = j_{\prod_{j}\sfS_{\bfcc_{j}(\cO)}}^{\sfS_{n}} \sgn.
\]
Recall that $\bfcc_{j}(\cO)$ is the $j$-th column length of $\cO$, and $\sgn$ denotes the sign character.

\trivial[h]{
  %\begin{minipage}{\textwidth}
  Here is the diagram to remember
  \[
    \begin{tikzcd}[ampersand replacement=\&]
     \Nil(\GL_{n}) \ar[rr,"\Spr"]\& \& \Irr(\sfS_{n})\\
      \&\YD_{n} \ar[lu,"\text{Jordan canonical forms}"] \ar[ru,"\text{Macdonald's $j$-induction}"']\&
    \end{tikzcd}
  \]
  % \end{minipage}
}

%For $\ckcO\in \Nil(\GL_{n}(\bC))$, we set % its Barbasch-Vogan dual is
%\[
 % \cO:=\dBV(\ckcO) = \ckcO^{t} \in \Nil(\GL_{n}(\bC)).
%\]


Put $\sfW_n := \sfS_n \ltimes \set{\pm 1}^n$, which is identified with  the Weyl group of type $B_n$
or $C_n$. As always, $\sgn$ denotes the sign character (of an appropriate Weyl group). Denote by $\epsilon$ the unique non-trivial character of $\sfW_n$ which is trivial on $\sfS_n$, which is given by
\[
  (s,(x_{1}, x_{2}, \cdots, x_{n}))\mapsto x_{1}x_{2}\cdots x_{n}.
\]


The group $\sfW_n$ is naturally embedded in $\sfS_{2n}$ via the homomorphism determined by
\[
(i, i+1)\in \sfS_n\mapsto (2 i-1, 2i+1)(2i, 2i+2), \qquad (1\leq i\leq n-1),
\]
and 
        \[
       (1,\cdots,1, \underbrace{-1}_{j\text{-th
        term}}, 1, \cdots, 1)\in \set{\pm 1}^n  \mapsto (2j-1, 2j), \qquad (1\leq j\leq n).
        \]
Here $(i, i+1)$, $(2 i-1, 2i+1)$, etc., indicate the involutions in the permutation groups.  Note that $\epsilon $ is also the restriction of $\sgn$ of $\sfS_{2n}$ to $\sfW_n$.
Recall the following branching formula (see \cite{BV.W}*{Lemma~4.1~(b)}):
\begin{equation}\label{IndFor}
  \Ind_{\sfW_{n}}^{\sfS_{2n}} \epsilon = \bigoplus_{\substack{\sigma\in \YD_{2n}\\
      \bfcc_{i}(\sigma) \text{ is even for all }i\in \bN^+}} \sigma.
\end{equation}



Suppose that $\star\in  \{A^\R, A^\BH, A, \widetilde A\}$ for the moment. We decompose $\ckcO$ into even and odd parts and write %$\ckcO\in \Nil(\ckGc)$ and decompose
\[
  \ckcO = \ckcO_{e} \cuprow \ckcO_{o}
\]
such that all nonzero row lengths of the Young diagrams $\ckcO_e$ and   $\ckcO_o$ are respectively even and odd.
%The infinitesimal character $\lamck$ attached to $\ckcO$ is integral if and only if $\ckcO=\ckcO_{e}$ or $\ckcO_{o}$.
%We identify the Weyl group $W$ of $\ckG=\GL_{n}(\bC)$ with the symmetric group $\sfS_{n}$.
We list
some easily verifiable facts:
\begin{equation}\label{eq:WA}
  \begin{split}
    W_{[\lamck]} & = \sfS_{\abs{\ckcO_{e}}}\times \sfS_{\abs{\ckcO_{o}}}, \\
    W_{\lamck} & = \prod_{i\in \bN^{+}}\sfS_{\bfcc_{i}(\ckcO_{e})}\times \prod_{i\in \bN^{+}}\sfS_{\bfcc_{i}(\ckcO_{o})},\\
    \LC_{\lamck} & = \set{\tau_{\lamck}}, \quad \textrm{where }  \tau_{\lamck} := (j_{W_{\lamck}}^{W_{[\lamck]}}\sgn )\otimes \sgn =  \ckcO_{e}^{t}\boxtimes \ckcO_{o}^{t}.
    %, \AND \\
    %\LC_{\lamck} & = \LRC_{\lamck}= \set{\tau_{\lamck}}, \AND \\
   % \wttau_{\lamck} & := j_{W_{[\lamck]}}^{W}\tau_{\lamck} = \ckcO^{t}.
  \end{split}
\end{equation}
Here and as before, a superscript ``$t$" indicates the transpose of a Young diagram. 



% Fix $\ckcO\in \Nil(\ckG) = \YD_{n}$ and set
% \[
%   \cO :=\dBV(\ckcO) = \ckcO^{t} \AND \wttau := \Spr^{-1}(\cO).
% \]
% Here $\wttau$ has the partition $\cO$.

% In all the cases, $\ckG = \GL_{n}(\bC)$ and $n$ is even if
% $G = \GL_{\frac{n}{2}}(\bH)$.
% The nilpotent orbit $\ckcO$ is parameterized by Young diagrams.


% \subsection{Definition of special unipotent representation }

% In this section, we let $G = \GL_{n}(\bC)$ and $\fgg:=\Lie(G) = \fgl_{n}(\bC)$.
% Then $\fggC:= \fgg \otimes_{\bR}\bC$ is naturally identified with
% $\fgg\times \fgg$ and $\cU(\fggC)=\cU(\fgg)\otimes \cU(\fgg)$.


% We make the following definition
% \begin{defn}
% For $\ckcO\in \Nil(\GL_{n}(\bC))$, let $\cIck$ be the maximal primitive ideal in
% $\cU(\fgg)$ with infinitesimal character $\lamck$.
% We call an irreducible admissible representation of $G$ is a special unipotent
% representation attached to $\ckcO$
% \end{defn}

% % For $\ckcO\in \Nil(\GL_{n}(\bC))$, let
% % \[
% % \lamckC := (\lambck,\lamck)\in \fhh^{*}\times \fhh^{*}.
% % \]
% % Via Harish-Chandra homomorphism, $\lamckC$ determine an infinitesimal character
% % of $\cZ(\fggC)$.

% % see \cite{BVUni}*{Introduction}.

% Following \cite{BVUni}*{Defintion~1.17}, we make the following definition of
% unipotent representation:
% \begin{defn}
% \end{defn}

% Since $\Lie(G)\otimes_{\bR}\bC$ is naturally

\trivial[h]{\subsection{Special unipotent representations of $G = \GL_n(\bC)$}
\def\lamckC{\lambda^\bC_{\ckcO}}
\def\fggR{\fgg_{\bR}}
\def\fggC{\fgg_{\bC}}
\def\cIck{\cI_{\ckcO}}
\def\WC{W^{\bC}}

The classification of special unipotent representations of $\GL_n(\bC)$ with integral infinitesimal character is a special case of the results of \cite{BVUni}. The classification in general is also well known to the experts. %, see \cite{V.GL}.

\begin{thm} Let $\star =A^{\bC}$ so that $G =\GL_n(\bC)$. Suppose $\ckcO\in \YD_{n} = \Nil(\check \g)$ and $\ckcO$ has $k$ rows.
Let
\[
 \pi_{\ckcO} := 1_{\bfrr_1(\ckcO)}\times  1_{\bfrr_2(\ckcO)}\times \cdots
 \times 1_{\bfrr_k(\ckcO)}
\]
be the normalized parabolic induction where $1_{c}$ denotes the trivial
representations of $\GL_c(\bC)$.
Then
\[
  \Unip_{\ckcO}(G) = \Set{\pi_{\ckcO}}.
\]
\end{thm}
\begin{proof}
  The irreducibility of $\pi_{\ckcO}$ is well known, see
  \cite{V.GL}. It is clearly an element in $\Unip_{\ckcO}(G)$ by the construction.

  %The computation of complex groups doubles that of real groups.
  We let $\WC$ denote the Weyl group of the complexification of $G$. As in Section \ref{sec:intro}, let $\lamckC:=(\lamck,\overline{\lamck})$, which is an element in the dual $\ahh^{*}$ of the universal Cartan subalgebra.
  Recall the notation in \cref{eq:WA}.
  We have the following facts:
  \[
    \begin{split}
      \WC  &\cong  \sfS_{n}\times \sfS_{n}\\
      \WC_{[\lamckC]} &= W_{[\lamck]}\times W_{[\lamck]}\\
      \LC_{\lamckC} &= \set{\tau_{\lamck}\boxtimes \tau_{\lamck}}\\
    \end{split}
  \]
  As a representation of $\WC_{[\lamckC]}$, $\Coh_{[\lamckC]}(\CK(G))$ is isomorphic
  to the regular representation of $ \bC[W_{[\lamck]}]$ (under the left and
  right translation), see \cite{BVUni}*{(3.15)}.
  Therefore,
  \[
    \sharp \Unip_{\ckcO}(G)\leq [\tau_{\lamck}\boxtimes \tau_{\lamck}: \bC[W_{[\lamck]}]] = 1
  \]
  by \Cref{cor:bound}.
  This finishes the proof of the theorem.
\end{proof}

}
% In this section, we let $G = \GL_{n}(\bC)$ and

% Then $\fggC:= \fgg \otimes_{\bR}\bC$ is naturally identified with
% $\fgg\times \fgg$ and $\cU(\fggC)=\cU(\fgg)\otimes \cU(\fgg)$.

% When $\lamck$

\trivial[h]{
Since the Lusztig canonical quotient of $\ckcO$ is trivial, the set of unipotent representations
of $G = \GL_n(\bC)$ one-one corresponds to nilpotent orbits in
$\Nil(\ckGc) = \YD_{n}$ if $\lamck$ is integral by
\cite{BVUni}.


\begin{remark}
The Vogan duality gives a duality between Harish-Chandra cells.
In this case, Harish-Chandra cell is the double cell
of Lusztig.
Now we have a duality
\[
 \pi_\ckcO \leftrightarrow \pi_{\ckcO^t}.
\]
\end{remark}

We record the following easy lemma which is a baby case of our results on other
classical groups.
\begin{lem}
  Let $\ckcO' := \DDD(\ckcO)$ be the partition obtained by deleting the first
  column of $\ckcO$. Let $\Phi_{\GL_{a}(\bC),\GL_{b}(\bC)}$ (resp.
  $\Phi_{\GL_{a}(\bC),\GL_{b}(\bC)}$) be the theta lift (resp. big theta lift)
  from $\GL_a(\bC)$ to $\GL_b(\bC)$. Then we have
  \[
    \pi_{\ckcO} = \Phi_{\GL_{\abs{\ckcO'}}(\bC),\GL_{\abs{\ckcO}}(\bC)} (\pi_{\ckcO'})= \Phi_{\GL_{\abs{\ckcO'}}(\bC),\GL_{\abs{\ckcO}}(\bC)} (\pi_{\ckcO'}).
  \]
\end{lem}

One should be able to deduce the local system attached to $\pi_{\ckcO}$ using
the inductive formula.
}

\subsection{Special unipotent representations of $\GL_n(\bR)$}



Special unipotent representations of general linear groups $\GL_{n}(F)$, where $F= \bR$ or $\bH$, are well-known and comprise
representations of the form
\[
  \Ind_{P}^{\GL_{n}(F)}\chi,
\]
where $P$ is a parabolic subgroup of $\GL_{n}(F)$,
and $\chi$ is a character of $P$ trivial on the connected component of $P$. See \cite{V.GL}*{Page 450}. We will review their classifications in the framework of this article.


% We assume that $G  = \GL_{n}(\bC), \GL_{n}(\bR), \GL_{\half n}(\bH)$ and $\star = A,A^{\bC}, A^{\bH}$ respectively.

%



% We list some easy facts below:
% \[
%   \begin{split}
%     W &= \sfS_{n} \\
%     W_{[\lamck]} & = \sfS_{\abs{\ckcO_{e}}}\times \sfS_{\abs{\ckcO_{o}}} \\
%     W_{\lamck} & = \prod_{i\in \bN^{+}}\sfS_{\bfcc_{i}(\ckcO_{e})}\times \prod_{i\in \bN^{+}}\sfS_{\bfcc_{i}(\ckcO_{o})}\\\
%     \tau_{\lamck} & := (j_{W_{\lamck}}^{W_{[\lamck]}}\sgn )\otimes \sgn =  \ckcO_{e}^{t}\boxtimes \ckcO_{o}^{t}\\
%     \LC_{\lamck} & = \LRC_{\lamck}= \set{\tau_{\lamck}}, \AND \\
%     \wttau_{\lamck} & = j_{W_{[\lamck]}}^{W}\tau_{\lamck} = \Spr^{-1}(\cO).
%   \end{split}
% \]


% Now let $\ckcO\in \Nil(\ckGc)$ and decompose
% \[
%   \ckcO = \ckcO_{e} \cuprow \ckcO_{o}
% \]
% where $\ckcO_e$ and $\ckcO_o$ are partitions consist of even and odd rows
% respectively.

% Every irreducible representation in $\Irr(W)$ is special. For the
% infinitesimal character $\lamck$,
% \[

% \]

% In all the cases, let $\DDD$ denote the dual descent of Young diagrams.
% Suppose $\ckcO$ is a Young diagram, it delete the longest row in $\ckcO$


% Let $\YD$ be the set of Young diagrams viewed as a finite multiset of positive
% integers. The set of nilpotent orbits in $\GL_n(\bC)$ is identified with Young
% diagram of $n$ boxes.


% Let $n_e = \abs{\ckcO_e}, n_o =
% \abs{\ckcO_o}$. % and $\lambda_\ckcO = \half \ckhh$.
% By the formula of $a$-function, one can easily see that The cell in
% $W(\lamck)$ consists of the unique representation
% $J_{W_{\lamck}}^{\Wint{\lamck}} (1)$. Now the $W$-cell
% $(J_{W_{\lamck}}^{W_{[\lamck]}} \sgn)\otimes \sgn$ consists a single
% representation
% \[
%   \tau_{\ckcO} = \ckcO_{e}^{t}\boxtimes \ckcO_{o}^{t}.
% \]
% The representation $j_{W_{\lamck}}^{S_{n}} \tau_{\ckcO}$ corresponds to the
% orbit $\cO= \ckcO^t $ under the Springer correspondence.
\trivial[h]{ WLOG, we assume $\ckcO = \ckcO_o$.

  Let $\sigma\in \widehat{S_n}$. We identify $\sigma$ with a Young diagram. Let
  $c_i = \bfcc_i(\sigma)$. Then $\sigma = J^{S_n}_{W'} \epsilon_{W'}$ where
  $W' = \prod S_{c_i}$ (see Carter's book). This implies Lusztig's a-function
  takes value
  \[
    a(\sigma) = \sum_i c_i(c_i-1) /2
  \]
  Comparing the above with the dimension formula of nilpotent Orbits
  \cite{CM}*{Collary~6.1.4}, we get (for the formula, see Bai ZQ-Xie Xun's paper
  on GK dimension of $SU(p,q)$)
  \[
    \half \dim(\sigma) = \dim(L(\lambda)) = n(n-1)/2 - a(\sigma).
  \]
  Here $\dim(\sigma)$ is the dimension of nilpotent orbit attached to the Young
  diagram of $\sigma$ (it is the Springer correspondence, regular orbit maps to
  trivial representation, note that $a(\triv)=0$), $L(\lambda)$ is any highest
  weight module in the cell of $\sigma$.


  Return to our question, let $S' = \prod_i S_{\bfcc_i(\ckcO)}$. We want to find
  the component $\sigma_0$ in $\Ind_{S'}^{S_n} 1$ whose $a(\sigma_0)$ is
  maximal, i.e. the Young diagram of $\sigma_0$ is minimal.

  By the branching rule, $\sigma \subset \Ind_{S'}^{S_n} 1$ is given by adding
  rows of length $\bfcc_i(\ckcO)$ repeatedly (Each time add at most one box in
  each column). Now it is clear that $\sigma_0 = \ckcO^t$ is desired.

  This agrees with the Barbasch-Vogan duality $\dBV$ given by
  \[
    \ckcO \xrightarrow{\Spr}\ckcO \xrightarrow{\otimes \sgn} \ckcO^t \xrightarrow{\Spr} \ckcO^t.
  \]
}

In this subsection, we assume  $\star=A^\R$ so that $G = \GL_{n}(\bR)$.
Recall from \Cref{defpbp0} the set $\PP_{A^\R}(\ckcO)$, which is the set of paintings on $\ckcO^{t}$ that has type $A^\R$.
%We define the set of painted partitions of type $A$ as the following:
%\begin{equation*}%\label{eq:PP.AR}
%  \PP_{A}(\ckcO) = \Set{\uptau:=(\tau, \cP)|
%    \begin{array}{l}
%      \text{$\tau = \ckcO^{t}$}\\
%      \text{$\Im(\cP)\subseteq \set{\bullet,c,d}$}\\
%      \text{$\#\set{i|\cP(i,j)=\bullet}$ is even for all $j\in \bN^{+}$}
%    \end{array}
%  }.
%\end{equation*}
It is easy to see that
\begin{equation}\label{eq:PPA.count}
  \#(\PP_{A^\R}(\ckcO)) = \prod_{r\in \bN^{+}} (\#\set{i\in \bN^{+}| \bfrr_{i}(\ckcO)=r}+1).
\end{equation}

\begin{prop} \label{lem:GL.count}
  Suppose $\star=A^\R$ so that $G = \GL_{n}(\bR)$. Let
  \[
    \cC_l := \bigoplus_{\substack{t,c,d\in \bN \\2t+c+d=l}} \Ind_{\sfW_t\times \sfS_c\times \sfS_d}^{\sfS_{l}} \epsilon \otimes 1\otimes 1.
  \]
  Then, as $W_{[\lamck]}$-modules,
  \[
    \Cint{\lamck}(\CK(G)) \cong \cC_{\abs{\ckcO_{e}}}\otimes \cC_{\abs{\ckcO_{o}}}.
  \]
  Furthermore,
  \begin{equation*}%\label{eq:A.count}
    [\tau_{\lamck}: \Cint{\lamck}(\CK(G))] = \# (\PP_{A^\R}(\ckcO_{e}))\times
    \# (\PP_{A^\R}(\ckcO_{o})) = \# (\PP_{A^\R}(\ckcO)).
  \end{equation*}
  \qed
\end{prop}
\begin{proof}
  The assertion on the coherent continuation representation follows from Theorem \ref{thm:cohHC}.
  The multiplicity formula follows from \eqref{IndFor} and Pieri's rule, in view of the description of the left cell representation $\tau_{\lamck}$ in \eqref{eq:WA}.  The last equality
  follows from \eqref{eq:PPA.count}.
\end{proof}
% The $\Wint{\lamck}$-module $\Cint{\lamck}$ is given by the following formula:
% According to Vogan duality, we can obtain the above formula by tensoring
% $\sgn$ on the formula of the unitary groups in \cite{BV.W}*{Section~4}.

\trivial[h]{
By branching rules of the symmetric groups, $\Unip_{\ckcO}(G)$ can be
parameterized by painted partition.

For $\uptau:=(\tau,\cP)\in \PP_{A}(\ckcO)$, we write $\cP_{\uptau}:= \cP$.

  The typical diagram of all columns with even length $2c$ are
  \[
    \ytb{\bullet\cdots\bullet\bullet\cdots\bullet,\vdots\vdots\vdots\vdots\vdots\vdots, \bullet\cdots\bullet c\cdots c, \bullet\cdots\bullet d\cdots d }
  \]

  The typical diagram of all columns with odd length $2c+1$ are
  \[
    \ytb{\bullet\cdots\bullet\bullet\cdots\bullet,\vdots\vdots\vdots\vdots\vdots\vdots, \bullet\cdots\bullet \bullet\cdots\bullet , c\cdots c d\cdots d }
  \]
}

Let $\sgn_a\colon \GL_a(\bR)\rightarrow \set{\pm 1}$ be the character given by sign of determinant
and $1_a$ be the trivial character of $\GL_a(\bR)$.
For
$\uptau\in \PP_{A^\R}(\ckcO)$, we attache the representation
\begin{equation}\label{eq:u.GLR}
  \pi_\uptau :=
  \bigtimes_{j} \underbrace{1_j \times \cdots \times 1_j}_{c_j\text{-terms}}\times
  \underbrace{\sgn_j \times \cdots \times {\sgn_j} }_{d_j\text{-terms}}.
\end{equation}
where
\begin{itemize}
  \item $j$ runs over all nonzero column lengths in $\ckcO^t$,
  \item $d_j$ is the number of columns of length $j$ ending with the symbol
        ``d'',
  \item $c_j$ is the number of columns of length $j$ ending with the symbol
        ``$\bullet$'' or ``$c$'', and
  \item ``$\times$'' denotes the parabolic induction.
\end{itemize}
Then $\pi_{\uptau}$ is irreducible and belongs to $\Unip_{\ckcO}(G)$ (see  \cite[Theorem 3.8]{V.GL} and \cite[Example~27.5]{ABV}). 

\begin{thm}[\cf \cite{ABV}*{Example~27.5}] Let $\star =A^\R$ so that $G =\GL_n(\bR)$. Then the map
  \[
    \begin{array}{ccc}
      \PP_{A^\R}(\ckcO) & \longrightarrow & \Unip_{\ckcO}(G),\\
      \uptau & \mapsto & \pi_{\uptau}
    \end{array}
  \]
  is bijective. 
\end{thm}
\begin{proof}
  The injectivity is proved in  \cite[Theorem 3.8]{V.GL}.
Then   the map is bijective by \Cref{cor:bound} and \Cref{lem:GL.count}.
\end{proof}

\subsection{Special Unipotent representations of $G=\GL_{n}(\bH)$} In this subsection, we assume $\star=A^{\bH}$ so that $G = \GL_{n}(\bH)$. Then 
$\ckG = \GL_{2n}(\bC)$. 

\begin{thm}
  Let
  \[
  \cC_{l}: = \begin{cases}
    \Ind_{\sfW_{\frac{l}{2}}}^{\sfS_{l}}\epsilon,
    & \text{if $l$ is even;}  \\
      0, & \text{ otherwise. } \\
    \end{cases}
  \]
  Then, as $W_{[\lamck]}$-modules,
  \[
    \Cint{\lamck}(\CK(G))  \cong
    \cC_{\abs{\ckcO_{e}}}\otimes \cC_{\abs{\ckcO_{o}}}.
    % \begin{cases}
    %   \Ind_{\sfW_{\frac{n}{2}}}^{\sfS_{n}}\epsilon & \text{ if
    %   } \ckcO = \ckcO_{e}\\
    %   0 & \text{ otherwise
    %   } \\
    % \end{cases}
  \]
Moreover, 
 \[
\Unip_{\ckcO}(G)  = \begin{cases}
   \set{\pi_{\ckcO}},
    & \text{ if $\ckcO = \ckcO_{e}$;}  \\
      \emptyset, & \text{ otherwise, } \\
    \end{cases}
  \]
where
  \[
    %\pi_{\ckcO} := \bigtimes_i 1_{\bfrr_i(\ckcO)/2}.
    \pi_{\ckcO} := 1_{\bfrr_1(\ckcO)/2}\times 1_{\bfrr_2(\ckcO)/2} \times \cdots
   \times  1_{\bfrr_k(\ckcO)/2},
  \]
  $k$ is the number of nonempty rows of $\ckcO$
  and $1_{a}$ denotes the trivial representation of $\GL_{a}(\bH)$.
% which is a element in $\Unip_{A^{\bH}}(\ckcO)$

% In this case the coherent continuation representation is given by
% \[
%   \Cint{\lamck}(G) = \Ind_{W_m}^{\sfS_{2n}}\epsilon
% \]
% and $\Unip_\ckcO(G)$ is a singleton. %We use partition $\tau:= \ckcO^t$ to parameter special unipotent representations of $\GL_{m}(\bH)$.
\end{thm}
\begin{proof} The assertion on the coherent continuation representation follows from Theorem \ref{thm:cohHC}. For the irreducibility of $\pi_{\ckcO}$, see \cite[Theorem 3.8]{V.GL}. As in the case of $\GL_n(\R)$, we know that $\pi_{\ckcO}\in \Unip_{\ckcO}(G)$ (\cf \cite[Lemma 8.3]{BVUni}). 
In view of the description of left cell representation $\tau_{\lamck}$ in \eqref{eq:WA}, it follows from \eqref{IndFor} that 
\[
[\tau_{\lamck}:\Cint{\lamck}(\CK(G))]  = \begin{cases}
   1,
    & \text{ if $\ckcO = \ckcO_{e}$;}  \\
      0, & \text{ otherwise.} \\
    \end{cases}
  \]
This proves the lemma. % The rest is clear.
%From \eqref{eq:WA}, the even (resp. odd) part of $\tau_{\lamck}$ has even (resp. odd) columns. By \eqref{IndFor}, $\cC_{\abs{\ckcO_{o}}}$, when nonzero, contains representations %indexed by those with even columns. We therefore conclude that
\end{proof}



\subsection{Special unipotent representations of $\rU(p,q)$}

In this subsection, we suppose  $\star \in \set{A, \wtA}$ so that 
\[
  G =
  \begin{cases}
    \rU(p,q),  & \text{if }\star = A;\\
    \tU(p,q),  & \text{if }\star = \wtA.
\end{cases}
\]

%When it is convenient, we also use the symbol $U$ to represent either $A^{*}$ or $\wtA^{*}$.

\trivial[h]{Put $\ckG:=\GL_{p+q}(\C)$, $\check \g:=\g\l_{p+q}(\C)$, and let $\ckcO\subset \mathrm{Nil}(\check \g)$ be a  nilpotent $\ckG$-orbit.  As before, we have an ideal $I_{\ckcO}$ of $\oU(\g)$, where $\g$ is the complexified Lie algebra of $G$. Let
\[
  \Unip_{\ckcO}(G) :=
  \begin{cases}
   \Set{\pi\in \Irr(G)|  \pi \textrm{ is annihilated by } I_{\ckcO}}, & \text{if $\star = A^{*}$}; \\
   \Set{\pi\in \Irr(G)|  \pi \textrm{ is annihilated by } I_{\ckcO}, \text{ and $\pi$ is
       geninue}}, & \text{if $\star = \wtA^{*}$}. \\
  \end{cases}
\]
We will count the number of elements in $\Unip_{\ckcO}(G)$.
}

We call an integer to have the $\star$-good parity, if it has the same parity as
\[
  \begin{cases}
    \text{the parity of $\abs{\ckcO}$} &  \text{if $\star = A^{*}$}, \\
    \text{the parity of $\abs{\ckcO}+1$} &  \text{if $\star = \wtA^{*}$}. \\
  \end{cases}
\]
The other parity is called the $\star$-bad parity.
We make the following decomposition:
\[
\ckcO = \ckcO_{\mathrm g}\cuprow \ckcO_{\mathrm b}
\]
where every nonzero row length in $\ckcO_{\mathrm g}$ (resp. $\ckcO_{\mathrm b}$) has the $\star$-good (resp. $\star$-bad) parity.

Let $(N_{\mathrm g},N_{\mathrm b}):= (\abs{\ckcO_{\mathrm g}},\abs{\ckcO_{\mathrm b}})$. Recall from  \eqref{eq:WA} that
\begin{equation}\label{eq:LC.A}
  \begin{split}
    \WLamck &= \sfS_{N_{\mathrm g}}\times \sfS_{N_{\mathrm b}},\\
        \LC_{\lamck} & = \set{\tau_{\lamck}}, \quad \textrm{where }  \tau_{\lamck}= \ckcO_{e}^{t}\otimes \ckcO_{o}^{t}.
%    \tau_{\lamck}  & = \ckcO_{\mathrm g}^{t}\otimes \ckcO_{\mathrm b}^{t}.\\
  \end{split}
\end{equation}

%We define $\Coh_{\Lamck}(\CK(G))$ to be the coherent continuation representation for the category of genuine
%representations of $G$ with generalized infinitesimal character $\lambda_{\ckcO}$, if $\star = \wtA^{*}$.

\begin{lem}\label{lem:ccrU}
  Let
\[
  \begin{split}
    \cC_{p,q} &:= \bigoplus_{\substack{t,s,r\in \bN\\t+r=p, t+s = q}}
    \Ind_{\sfW_{t}\times \sfS_s\times \sfS_r}^{\sfS_{p+q}}
 1\otimes \sgn \otimes \sgn ,\\
 \cC_{\mathrm b} &:= \begin{cases}
  \Ind_{\sfW_{\frac{N_{\mathrm b}}{2}}}^{\sfS_{N_{\mathrm b}}} 1, & \ \text{if $N_{\mathrm b}$ is even};\\
  0, & \ \text{otherwise}.
 \end{cases}
  \end{split}
\]
Then, as $\WLamck =\sfS_{N_{\mathrm g}}\times \sfS_{N_{\mathrm b}}$-modules,
\[
  \Coh_{\Lamck}(\CK'(G)) \cong \begin{cases}
    \cC_{p-\half N_{\mathrm b},q-\half N_{\mathrm b}}\otimes \cC_{\mathrm b}, &\  \text{if $N_{\mathrm b}$ is even and
      $p,q \geq \half N_{\mathrm b}$};\\
    0, & \ \text{otherwise}.
  \end{cases}
\]
\end{lem}
\begin{proof}
  For $G=\tU(p,q)$, tensoring with the (genuine) $\det^{1/2}$-character yields a
  $W_{\lambda_{\check \CO}}$-module isomorphism
  $\Coh_{[\lamck-(\half, \cdots, \half)]}(\CK(\rU(p,q))) \longrightarrow \Coh_{\Lamck}(\CK'(G))$.
  So the assertion for $\tU(p,q)$ is easily reduced to that of $\rU(p,q)$, which follows from Theorem \ref{thm:cohHC} by a routine
  calculation.
\end{proof}

% \trivial[]{
%   Let
%   $\lambda := (\lambda_{1},\cdots,\lambda_{n})=(\underbrace{1/2, \cdots, 1/2}_{a},\underbrace{0,\cdots,0}_{\mathrm b})$
%   how to compute  $\Coh_{\Lam}(\wtU(p,q))$ of the genuine $\wtU(p,q)$ at the
%   infinitesimal character coset $[\lambda]$?

%   Twist the genuine $\wtU(p,q)$ representations with $\det^{1/2}$ yields
%   an isomorphism between $\Coh_{\Lam}(\wtU(p,q))$ with
%   $\Coh_{\Lam+\half}(U(p,q))$.
% }

% If $p,q \geq \half \nbb$, we write
% \[
% \Gg :=\begin{cases}

% \end{cases}
% \]


\begin{prop} \label{propu0}
The set $\Unip_{\ckcO}(G)$ is empty  unless           \begin{itemize}
            \item $N_{\mathrm b}$ is even and $p,q \geq \half N_{\mathrm b}$; and
            \item each nonzero row length occurs in $\ckcOb$ with even multiplicity.
                 %       \item $p+q$ is even when $\star=\widetilde A$ so that $G =\widetilde \oU(p,q)$.
          \end{itemize}
\end{prop}

\begin{proof} 
%Recall from \eqref{eq:LC.A} that $ \LC_{\lamck}  = \set{ \tau_{\lamck} }$, where $\tau_{\lamck} := \ckcO_{\mathrm g}^{t}\otimes \ckcO_{\mathrm b}^{t}$.
  Suppose that  $\Unip_{\ckcO}(G)$ is nonempty. By \eqref{boundc22}, we have that
  \[
  [\tau_{\lamck}:\Cint{\lamck}(\CK'(G))]\ne 0.
  \]
  Then  by \Cref{lem:ccrU} and \eqref{IndFor}, the two conditions of the proposition are satisfied.  
  \end{proof}

Assume that the two conditions in Proposition \ref{propu0} are satisfied. Then $G$ has a Levi subgroup that is identified with $G'_{\mathrm b}\times G_{\mathrm g}$, where 
$G'_{\mathrm b}:=\GL_{\frac{N_{\mathrm b}}{2}}(\C)$ and 
\[
  G_{\mathrm g} =
  \begin{cases}
    \rU(p-\frac{N_{\mathrm b}}{2},q-\frac{N_{\mathrm b}}{2}),  & \text{if }\star = A;\\
    \tU(p-\frac{N_{\mathrm b}}{2},q-\frac{N_{\mathrm b}}{2}),  & \text{if }\star = \wtA.
\end{cases}
\]
Write $\check \CO_{\mathrm b}=2 \ckcOpb$ and let $\pi_{\ckcOpb}$ denote the
unique element in $\Unip_{\ckcOpb}(\GL_{\half N_{\mathrm b}}(\bC))$. Then for
every $\pi_0\in \Unip_{\ckcO_{\mathrm g}}(\Gg)$, the normalized parabolic
induction $\pi_{\ckcOpb}\rtimes \pi_{0}$ is irreducible by
\cite{Mat96}*{Theorem~3.2.2} and is an element of $\Unip_{\ckcO}(G)$ (\cf
\cite{MR.U}*{Theorem~5.3}).

\begin{prop}
  Assume that the two conditions in \Cref{propu0} are satisfied.
  Then
  \begin{equation}\label{unitarred1}
    \sharp(\Unip_{\ckcO}(G)) =
    \sharp(\Unip_{\ckcO_{\mathrm g}}(\Gg)),
  \end{equation}
  and
  the map
  \[
    \begin{array}{ccc}
      \Unip_{\ckcO_{\mathrm g}}(\Gg)&\longrightarrow &\Unip_{\ckcO}(G),\\
      \pi_{0}& \mapsto & \pi_{\ckcOpb}\rtimes \pi_{0} \\
    \end{array}
  \]
  is bijective. %Consequently, % Then
  % the map
  % \[
  %   \begin{array}{ccc}
  %     \Unip_{\ckcO_{\mathrm g}}(\Gg)&\longrightarrow &\Unip_{\ckcO}(G),\\
  %     \pi_{0}& \mapsto & \pi_{\ckcOpb}\rtimes \pi_{0} \\
  %   \end{array}
  % \]
  % is bijective. Consequently, \be\label{unitarred1} \sharp(\Unip_{\ckcO}(G)) =
  % \sharp(\Unip_{\ckcO_{\mathrm g}}(\Gg)). \ee
\end{prop}

\begin{proof}
  We observe that structure of cells in the basal representation
  $\Coh_{[\lambda_{\check \CO}]}(\CK'(G))$ (see \cite[Theorem 5]{Bo}) implies
  the equality in \eqref{boundc22}. Hence
  \[
    \sharp(\Unip_{\check \CO}(G)) =[\tau_{\lamck}: \Coh_{[\lambda_{\check \CO}]}(\CK'(G))].
  \]
  Similarly,
  \[
    \sharp(\Unip_{\check \CO_\mathrm g}(G_\mathrm g)) = [\tau_{\lamck}: \Coh_{[\lambda_{\check \CO_\mathrm g}]}(\CK'(G_\mathrm g))].
  \]
  Note that $[\ckcO_{\mathrm b}^{t}:\cC_{\mathrm b}]=1$, in view of
  \eqref{IndFor}. The above three equalities clearly imply \eqref{unitarred1}.
 
  For the last assertion, note that the given map is an injection by considering
  their {\color{red}Langlands parameters } %(\cf \cite{MR.U}*{Theorem~5.3})
  and so a bijection by the counting assertion \eqref{unitarred1}.
\end{proof}

  % \begin{enumT}
  %   \item The set $\Unip_{\ckcO_b}(\rU(p,q))\neq \emptyset$ if and only if $p=q$
  %   and each row lenght in $\ckcO$ has even multiplicity.
  %   \item Suppose $\Unip_{\ckcO_b}(\rU(p,p))\neq \emptyset$, let $\ckcO'$ be the
  %   Young diagram such that $\bfrr_i(\ckcO') = \bfrr_{2i}(\ckcO_b)$ and $\pi'$
  %   be the unique special uinpotent representation in
  %   $\Unip_{\ckcO'}(\GL_{p}(\bC))$. Then the unique element in
  %   $\Unip_{\ckcO_b}(\rU(p,p))$ is given by
  %   \[
  %     \pi := \Ind_{P}^{\rU(p,p)} \pi'
  %   \]
  %   where $P$ is a parabolic subgroup in $\rU(p,p)$ with Levi factor equals to
  %   $\GL_p(\bC)$.
  %   \item In general, when $\Unip_{\ckcO_b}(\rU(p,p))\neq \emptyset$, we have a
  %   natural bijection
  %   \[
  %     \begin{array}{rcl}
  %       \Unip_{\ckcO_{\mathrm g}}(\rU(n_1,n_2)) &\longrightarrow& \Unip_{\ckcO}(\rU(n_1+p,n_2+p))\\
  %       \pi_0 & \mapsto & \Ind_P^{\rU(n_1+p,n_2+p)} \pi'\otimes \pi_0
  %     \end{array}
  %   \]
  %   where $P$ is a parabolic subgroup with Levi factor
  %   $\GL_p(\bC)\times \rU(n_1,n_2)$.
  % \end{enumT}

From the above proposition, the problem of construction of all special unipotent representations attached to $\ckcO$ is reduced to the case when $\ckcO = \ckcO_{\mathrm g}$.
Now assume $\ckcO = \ckcO_{\mathrm g}$.
% and so $\Cint{\ckcO}(G)$ corresponds to the blocks of the infinitesimal
% character of the trivial representation.
In this good parity setting, we will state a counting result on
$\Unip_{\ckcO}(G)$. The elements in $\Unip(\ckcO)(G)$ can be constructed by
cohomological induction explicitly and they are irreducible and unitary due to
\cite{Mat96,Tr.U}, see also \cite{Tr.H}*{Section~2} and \cite{MR.U}*{Section~4}.
\trivial[h]{
  See \cite{MR.U}*{First several paragraphs of Section~4}.
}
We refer the reader to \cite{BMSZ2} for the construction of all elements of $\Unip_{\mathrm g}(\ckcO)$ by the method of theta lifting.

Recall from \Cref{defpbp0} the set $\PP_{\star}(\ckcO)$, which is the set of paintings on $\ckcO^{t}$ that has type $\star$.
%We define the set of painted partitions of type $U$ as the following:
%\begin{equation*}%\label{eq:PP.U}
%  \PP_{U}(\ckcO) = \Set{\uptau:=(\tau, \cP)|
%    \begin{array}{l}
%      \text{$\tau = \ckcO^{t}$}\\
%      \text{$\Im(\cP)\subseteq \set{\bullet,s,r}$}\\
%      \text{$\#\set{j|\cP(i,j)=\bullet}$ is even for all $i\in \bN^{+}$}
%    \end{array}
%  }.
%\end{equation*}
For $\cP \in \PP_{\star}(\ckcO)$, we have defined its signature $(p_\CP, q_\cP)$ in \eqref{eq:signature}. Let $\overline \Nil_{G}(\cO)$ denote the set of $G$-orbits in $(\sqrt{-1}\g_0^*)\cap \CO$, where $\g_0$ denotes the Lie algebra of $G$ which equals $\u(p,q)$.  
%For a painted partition $\cP$ of type $U$, define the signature of $\CP$ to be the pair
%\[
%    (p_\CP, q_\cP): = \left (\frac{\sharp(\cP^{-1}(\bullet))}{2}+\sharp(\cP^{-1}(r)),\,
%    \frac{ \sharp(\cP^{-1}(\bullet))}{2}+\sharp(\cP^{-1}(s))\right).
%\]

\begin{thm}[\cf {Barbasch-Vogan\cite{BV.W} and Trapa\cite{Tr.H}*{Theorem~2.1}}]
  Suppose that $\ckcO = \ckcO_{\mathrm g}$. Then
  \begin{equation} \label{sharpuni0} \sharp(\Unip_{\ckcO}(G))= \sharp \set{\CP\in \mathrm{PAP}_{A^*}(\ckcO)|(p_\CP, q_\CP)=(p,q)}=\sharp(\overline \Nil_{G}(\cO)).
  \end{equation}
  Moreover, for every $\pi\in \Unip_{\ckcO}(G)$, its wavefront set $\WF(\pi)$ is
  the closure in $\sqrt{-1}\g_0^*$ of a unique orbit
  $\sO_\pi\in \overline \Nil_{G}(\cO)$, and the map
  \begin{equation}\label{sharpuni1}
    \begin{array}{rcl}
      \Unip_{\ckcO}(G) &\longrightarrow& \overline \Nil_{G}(\cO), \\%\Nil_{\mathrm g}(\cO),\\
      \pi & \mapsto & \sO_\pi.
    \end{array}
  \end{equation}
  is bijective.
\end{thm}

\begin{proof} The first equality  in \eqref{sharpuni0} follows from \Cref{lem:ccrU},
  \eqref{IndFor} and Pieri's rule.
  The second equality in \eqref{sharpuni0} follows directly from
  the bijectivity of \eqref{sharpuni1}.
  \trivial[]{
  {delete the following sentence: }
  The second equality in \eqref{sharpuni0}
  follows from (an easy) direct computation of numbers of elements both-sides.
  }
  % We now prove the second equality,  there is a one-one
  % correspondence of Harish-Chandra cells in $\Cint{\lambda_\ckcO}(\CK'(G))$ with
  % the orbits in $\overline \Nil_{G}(\cO)$ by \cite{BV.W}*{Theorem~4.2} and
  % \cite[Theorem 5]{Bo} (we use \cite{SV}*{Theorem~1.4} to restate the result in
  % terms of real nilpotent ).
  %
 To prove the bijectivity of \eqref{sharpuni1}, we use the one-one
  correspondence of Harish-Chandra cells in $\Cint{\lambda_\ckcO}(\CK'(G))$ with
  the orbits in $\overline \Nil_{G}(\cO)$ by \cite{BV.W}*{Theorem~4.2} and
  \cite[Theorem 5]{Bo} (we use \cite{SV}*{Theorem~1.4} to rephrase the result in
  terms of real nilpotent orbits).
  The Harish-Chandra cell representations are all irreducible and explicitly
  discribed in terms of the Springer correspondence by \cite{Bo}*{Lemma~4}.
  By the proof of \Cref{cohbbs}, we conclude that
  there is exactly one unipotent representation attached to each
  relevent cell and therefore establish the bijection.
  % The second assertion follows from the main result of
  % \cite{SV}*{Theorem~1.4}, \cite[Theorem 5]{Bo}, and \eqref{sharpuni0}.
\end{proof}

\trivial[h]{
From the branching rule, the cell is parameterized by painted partition
\[
\PP{}(\rU):=\set{\uptau\in \PP{}| \begin{array}{l}\Im (\uptau) \subseteq  \set{\bullet, s,r}\\
  \text{``$\bullet$'' occurs even times in each row}
\end{array}
  }.
\]

The bijection $\PP{}(\rU)\rightarrow \SYD, \uptau\mapsto \sO$ is given by the following recipe:
The shape of $\sO$ is the same as that of $\uptau$.
$\sO$ is the unique (up to row switching) signed Young diagram such that
\[
  \sO(i,\bfrr_i(\uptau)) := \begin{cases}
    +,  & \text{when }\uptau(i,\bfrr_i(\uptau))=r;\\
    -,  & \text{otherwise, i.e. }\uptau(i,\bfrr_i(\uptau))\in \set{\bullet,s}.
  \end{cases}
\]

\begin{eg}
  \[
 \ytb{\bullet\bullet\bullet\bullet r,\bullet\bullet , sr,s,r}
 \quad
 \mapsto\quad
 \ytb{+-+-+,+-, -+,-,+}
  \]
\end{eg}
}

\begin{remark}
  Note that the parabolic induction of a real nilpotent orbit may be reducible.
  When $\ckcO_{\mathrm b}\neq \emptyset$, a special unipotent representation may thus have
  a reducible associated variety. Nevertheless, the map
  $\Unip_{\ckcO}(G) \ni \pi \mapsto \WF(\pi)$ remains to be injective in the case of unitary groups. 
\end{remark}

\delete{
We are still in the setting that $\ckcO = \ckcO_{\mathrm g}$. We will sketch the construction of all elements in $\Unip_{\ckcO}(G)$ by iterated theta lifting, as follows.
For each $\uptau =(\tau, \CP)\in \PP_{U}(\ckcO)$ with $(p_\CP, q_\CP)=(p,q)$, we attach a real nilpotent orbit $\sO$ or equivalently a signed Young diagram of signature $(p,q)$
by the following recipe: The shape of $\sO$ is the same as that of $\tau$, and $\sO$ is the unique signed Young diagram (up to row switching) such that
\[
  \sO(i,\bfrr_i(\uptau)) := \begin{cases}
    +,  & \text{when }\uptau(i,\bfrr_i(\uptau))=r;\\
    -,  & \text{otherwise, i.e. }\uptau(i,\bfrr_i(\uptau))\in \set{\bullet,s}.
  \end{cases}
\]
Let $\DD(\sO)$ be the signed Young diagram obtained by deleting the first column of $\sO$.
Suppose $\sO$ has $k$-columns. Inductively we have a sequence of unitary groups
$\rU(p_i,q_i)$ with $(p_0,q_0)=(p,q)$, and $(p_i,q_i) = \Sign(\DD^i(\sO))$ for $i=1, \cdots, k$. Then
\begin{equation}\label{eq:u.U}
%   \pi_\tau = \Phi^{\rU(p_0,q_0)}_{\rU(p_1,q_1)} \Phi^{\rU(p_1,q_1)}_{\rU(p_2,q_2)}\cdots
% \Phi^{\rU(p_{k-1},q_{k-1})}_{\rU(p_k,q_k)}(1)
  \pi_{\uptau}= \Theta_{G_{1},G_{0}} \Theta_{G_{2},G_{1}}\cdots \Theta_{G_{k},G_{k-1}} (1),
\end{equation}
where
\begin{itemize}
  \item
  $G_{0}=G$, and
  \item for $1\leq i\leq k$, $G_{i} = \rU(p_{i},q_{i})$ if $p_{i-1}+q_{i-1}$ is even and
 $G_{i} = \tU(p_{i},q_{i})$ otherwise,
 \item $\Theta_{G_{i},G_{i-1}}$ is the theta lifting from $G_{i}$ to $G_{i-1}$, and
 \item $1$ is the trivial representation of $G_{k}$.
\end{itemize}
%Now
%\begin{equation}\label{eq:AV.U}
%  \sO \mapsto \AV(\pi_{\sO})
%\end{equation}
%give the bijection between the set of singed Young diagrams of shape $\cO$ and
%the set of real forms of $\cO$.
}

\trivial[h]{
  Duality between unitary group and real general linear group.

Suppose $\ckcO = \ckcO_{\mathrm g}$. Form the duality between cells of $\rU(p,q)$ and
$\GL(n,\bR)$. We have an ad-hoc (bijective) duality between unipotent
representations:
\[
  \begin{array}{rcl}
 \dBV\colon \Unip_{\ckcO}(\rU)& \rightarrow &\Unip_{\ckcO^t}(\GL(\bR)) \\
 \pi_\uptau &\mapsto& \pi_{\dBV(\uptau)} \\
  \end{array}
\]

Here $\ckcO^t = \dBV(\ckcO)$ and $\dBV(\uptau)$ is the pained bipartition
obtained by transposeing $\uptau$ and replace $s$ and $r$ by $c$ and $d$
respectively. See \eqref{eq:u.U} and \eqref{eq:u.GLR} for the definition of
special unipotent representations on the two sides.
}

%\subfile{counting_A}



% \subfile{counting_abs}

% \subfile{counting_A}


\section{Counting of special unipotent representations in type BCD}

In this section, we consider the cases when %$\ckstar \in \set{B,C,D}$, i.e
$\star \in \set{B,\wtC, C,D,C^{*}, D^{*}}$.

% We identify $\fhh^{*}$ with $\bZ^{n}$ where $n = \rank(\Gc)$
% and let $\rho$ be the half sum of all positive roots.

Recall the notion of $\star$-good parity (or the good parity for short):
\[
  \text{the good parity} =
\begin{cases}
 %\text{odd} & \text{when } \ckstar\in \set{B,D}\\
 %\text{even} & \text{when } \ckstar = C\\
 \text{odd}, & \text{when } \star \in \set{C,C^{*},D,D^{*}};\\
 \text{even}, & \text{when } \star \in \set{B,\wtC}.\\
\end{cases}
\]
Given a nilpotent $\ckG$-orbit $\ckcO$ in $\mathrm{Nil}(\check \g)$, we have the decomposition
\[
  \ckcO = \ckcO_{\mathrm g}\cuprow \ckcO_{\mathrm b},
\]
where every row length in $\ckcO_{\mathrm g}$ (resp. $\ckcO_{\mathrm b}$) has the good (resp. bad) parity. Note that bad parity rows occur with even multiplicity.
%We have $\ckcG_{\lamck} = \ckcG_{\mathrm g}\otimes \ckcG_{\mathrm b}$.


Recall from \Cref{sec:GL} the Weyl group of type $B_n$ (or $C_n$) realized as $\sfW_n := \sfS_n \ltimes \set{\pm 1}^n$.
Let $\sfW'_n$ be the Weyl group of type $D_{n}$, realized as the kernel of the quadratic character $\epsilon$ of $\sfW_n$. Note that implicitly we have chosen a
  preferred embedding of $\sfS_{n}$ into $\sfW'_{n}$ via the root system of type $D_{n}$.


Let $n_{\mathrm g}$ and $n_{\mathrm b}$ be defined by
%be the rank of ${\check \g}_{\mathrm g}$ and ${\check \g'}_{\mathrm b}$, respectively:
  \begin{equation*}%\label{eq:ngnb}
    (n_{\mathrm g},n_{\mathrm b}) =
    \begin{cases}
      (\half\abs{\ckcO_{\mathrm g}}, \half \abs{\ckcO_{\mathrm b}}) & \text{when
      } \star \in \set{B,\wtC,D,D^{*}},\\
      (\half(\abs{\ckcO_{\mathrm g}}-1), \half \abs{\ckcO_{\mathrm b}}) & \text{when
      } \star \in \set{C,C^{*}}.\\
          \end{cases}
  \end{equation*}
 Then the integral Weyl group $\WLamck$ is a direct product:
    \[
    W_{[\lamck]} =\Wg\times \Wb,
  \]
  where
  \begin{equation}\label{eq:Wbg}
    \begin{split}
    \Wg & := \begin{cases}
      \sfW_{n_{\mathrm g}}  & \text{when } \star \in \set{B,C, C^{*} }, \\
      \sfW'_{n_{\mathrm g}} & \text{when } \star \in \set{\wtC,D,D^{*}},\end{cases}\\
      \Wb & := \begin{cases}
      \sfW_{n_{\mathrm b}}  & \text{when } \star \in \set{B, \wtC}, \\
      \sfW'_{n_{\mathrm b}} & \text{when } \star \in \set{C,C^{*},D,D^{*}}.
      \end{cases}
    \end{split}
  \end{equation}

As usual, we identify $\Irr(\sfW_{n})$ with the set of bipartitions $\tau =(\tau_{L},\tau_{R})$ of total size $n$.
Given the chosen imbedding of $\sfS_{n}$ into $\sfW'_{n}$, we also let $(\tau_{L},\tau_{R})_{I}$ denote the unique irreducible character of $\sfW'_{n}$ given by
  \begin{itemize}
    \item the restriction of
    the irreducible character of $\sfW_{n}$ attached to $(\tau_{L},\tau_{R})$ if
    $\tau_{L}\neq \tau_{R}$, and
    \item
    the character
    $\Ind_{\sfS_{n}}^{\sfW'_{n}} \tau_{L}$ if $\tau_{L}=\tau_{R}$.
  \end{itemize}
  Note that as $\sfW'_{n}$ characters, we always have
  \[
    (\tau_{L},\tau_{R})_{I}=(\tau_{R},\tau_{L})_{I}.
  \]



  \subsection{The left cells}
  \label{sec:LCBCD}
  In this subsection, we describe the Lusztig left cell $\LC_{\lambda_{\ckcO}}$
  attached to $\lambda_{\ckcO}$, for each of the cases when $\star \in \set{B,C,\wtC,C^{*},D,D^{*}}$. We make some definitions first.

  Define the irreducible $W_{\mathrm b}$ representation $\tau_{\mathrm b}$ attached to $\ckcO_{\mathrm b}$ by the following formula:
  \begin{equation}\label{eq:taub}
    \begin{split}
      \tau_{\mathrm b} & := (\tau_{L,b},\tau_{R,b})\\
      &:= \begin{cases}
        \Big(\big(\half(\bfrr_{2}(\ckcO_{\mathrm b})+1), \half(\bfrr_{4}(\ckcO_{\mathrm b})+1), \cdots, \half(\bfrr_{2c}(\ckcO_{\mathrm b})+1)\big),\\
        \hspace{1em}\big(\half(\bfrr_{2}(\ckcO_{\mathrm b})-1), \half(\bfrr_{4}(\ckcO_{\mathrm b})-1), \cdots, \half(\bfrr_{2c}(\ckcO_{\mathrm b})-1) \big)\Big)
        & \text{if } \star \in \set{B,\wtC},\\
        \Big( \big(\half\bfrr_{2}(\ckcO_{\mathrm b}), \half\bfrr_{4}(\ckcO_{\mathrm b}),\cdots, \half\bfrr_{2c}(\ckcO_{\mathrm b})\big), \\
        \hspace{1em} \big(\half\bfrr_{2}(\ckcO_{\mathrm b}), \half\bfrr_{4}(\ckcO_{\mathrm b}),\cdots, \half\bfrr_{2c}(\ckcO_{\mathrm b}) \big)\Big)_{I}
        & \text{if } \star \in \set{C,C^{*}, D,D^{*}},\\
      \end{cases}
    \end{split}
  \end{equation}
  where $2c = \bfcc_{1}(\ckcO_{\mathrm b})$.

  As in \Cref{defn:PP}, set
  \[
    \CPPs(\ckcO_{\mathrm g}) =
    \begin{cases}
      \set{(2i-1,2i)| \bfrr_{2i-1}(\ckcO_{\mathrm g})- \bfrr_{2i}(\ckcO_{\mathrm g})\geq
        2, %\text{and}
        i\in \bN^{+}} & \text{if $\star\in \Set{C,\wtC,C^{*}}$},\\
      \set{(2i,2i+1)| \bfrr_{2i}(\ckcO_{\mathrm g})- \bfrr_{2i+1}(\ckcO_{\mathrm g})\geq
        2, %\text{and }
        i\in \bN^{+}} & \text{if $\star\in \Set{B,D,D^{*}}$}.
    \end{cases}
  \]
  Let
  \[
    \wtA(\ckcO) := \bZ_{2}[\CPP(\ckcO_{\mathrm g})]
  \] be the power set of $\CPPs(\ckcO_{\mathrm g})$.

  For each $\wp\in \wtA(\ckcO)$, we define an element $\tau_{\wp}$ in
  $\Irr(\Wg)$: 
  \begin{equation}\label{eq:tauwp}
    \tau_{\wp} :=
    \begin{cases}
      (\imathp,\jmathp) & \text{when } \star \in \set{B,C, C^{*} }, \\
      (\imathp,\jmathp)_{I} & \text{when } \star \in \set{\wtC,D,D^{*}},
    \end{cases}
  \end{equation}
  and $(\imathp, \jmathp)$ is given by the following recipe:
  \begin{itemize}
    \item Suppose $\star\in \set{C,C^{*}}$ and let
          $l=\min\set{i| \bfrr_{2i}(\ckcO_{\mathrm g})=0}$. Then
          \[
          (\bfcc_{l}(\imathp), \bfcc_{l}(\jmathp)) := (0,\half(\bfrr_{2l+1}(\ckcO_{\mathrm g})-1))
          \]
          and, for all $1\leq i< l$,
          \[
          (\bfcc_{i}(\imathp), \bfcc_{i}(\jmathp)):=
          \begin{cases}
            (\half (\bfrr_{2i}(\ckcO_{\mathrm g})+1), \half (\bfrr_{2i-1}(\ckcO_{\mathrm g})-1))
            & \text{if } (2i-1,2i)\notin \wp,\\
            (\half (\bfrr_{2i-1}(\ckcO_{\mathrm g})+1),\half (\bfrr_{2i}(\ckcO_{\mathrm g})-1)) & \text{otherwise.}
          \end{cases}
          \]
    \item Suppose $\star\in \set{D,D^{*}}$ and let
          $l=\min\set{i| \bfrr_{2i+1}(\ckcO_{\mathrm g})=0}$. Then
          \[
          \begin{split}
            \bfcc_{1}(\imathp) &:=
            \half(\bfrr_{1}(\ckcO_{\mathrm g})+1)\\
            (\bfcc_{l+1}(\imathp), \bfcc_{l}(\jmathp)) &:= (0,\half(\bfrr_{2l}(\ckcO_{\mathrm g})-1))
          \end{split}
          \]
          and, for all $1\leq i<l$,
          \[
          (\bfcc_{i+1}(\imathp), \bfcc_{i}(\jmathp)):=
          \begin{cases}
            \left(\half (\bfrr_{2i+1}(\ckcO_{\mathrm g})+1), \half (\bfrr_{2i}(\ckcO_{\mathrm g})-1)\right)
            & \text{if } (2i,2i+1)\notin \wp,\\
            (\half (\bfrr_{2i}(\ckcO_{\mathrm g})+1),\half (\bfrr_{2i+1}(\ckcO_{\mathrm g})-1)) & \text{otherwise.}
          \end{cases}
          \]
          % \[
          %   (\bfcc_{i}(\imathp), \bfcc_{i}(\jmathp)):=
          %   \begin{cases}
          %     (\half (\bfrr_{2i-1}(\ckcO_{\mathrm g})+1),\half (\bfrr_{2i}(\ckcO_{\mathrm g})-1)) &\text{if } (2i-1,2i)\in \wp, \\
          %     (\half (\bfrr_{2i}(\ckcO_{\mathrm g})+1), \half (\bfrr_{2i-1}(\ckcO_{\mathrm g})-1))
          %     & \text{if } (2i-1,2i)\notin \wp\\
          %     & \text{ and }\bfrr_{2i}(\ckcO_{\mathrm g})\neq 0,
          %     \\
          %     (0,0)
          %     & \text{if } \bfrr_{2i-1}(\ckcO_{\mathrm g})=0,\\
          %     (0, \half (\bfrr_{2i-1}(\ckcO_{\mathrm g})-1)) & \text{otherwise}
          %   \end{cases}
          % \]

          % \[
          %   (\bfcc_{l+1}(\imathp), \bfcc_{l+1}(\jmathp)) := (0,\half(\bfrr_{2l+1}(\ckcO_{\mathrm g})-1))
          % \]
          % and for all $1\leq i\leq l$
    \item Suppose $\star=B$. Then
          \[
          \bfcc_{1}(\jmathp) := \half\bfrr_{1}(\ckcO_{\mathrm g})
          \]
          and for all $i\geq 1$,
          \[
          (\bfcc_{i}(\imathp), \bfcc_{i+1}(\jmathp)):=
          \begin{cases}
            (\half \bfrr_{2i}(\ckcO_{\mathrm g}), \half \bfrr_{2i+1}(\ckcO_{\mathrm g}))
            & \text{if } (2i,2i+1)\notin \wp,\\
            (\half \bfrr_{2i+1}(\ckcO_{\mathrm g}),\half \bfrr_{2i}(\ckcO_{\mathrm g})) & \text{otherwise.}
          \end{cases}
          \]
    \item Suppose $\star = \wtC$. Then for all $i\geq 1$,
          \[
          (\bfcc_{i}(\imathp), \bfcc_{i}(\jmathp)):=
          \begin{cases}
            (\half \bfrr_{2i-1}(\ckcO_{\mathrm g}), \half \bfrr_{2i}(\ckcO_{\mathrm g}))
            & \text{if } (2i-1,2i)\notin \wp,\\
            (\half \bfrr_{2i}(\ckcO_{\mathrm g}),\half \bfrr_{2i-1}(\ckcO_{\mathrm g})) & \text{otherwise.}
          \end{cases}
          \]
          In this case, it is clear that $\tau_{\wp} = \tau_{\wp^{c}}$, where $\wp^{c}$ is the complement of $\wp$ in $\CPPs(\ckcO_{\mathrm g})$.
  \end{itemize}

  %For $\wp\subset\CPPs(\ckcO_{\mathrm g})$, let
  % $\star \in \Set{\wtC,D,D^{*}}$. ### This is not correct!
  %$\star = \wtC$.

  We now define
  \begin{equation*}%\label{def:barA}
  \barA(\ckcO)=
  \begin{cases}
  \wtA(\ckcO)/\{\wp\sim\wp^{c}\} & \text{when } \star =\wtC,\\
  \wtA(\ckcO) & \text{otherwise.}
  \end{cases}
    % \begin{cases}
    %   \wtA(\ckcO)/\wp\sim\wp^{c} & \text{when } \star \in \set{\wtC,D,D^{*}}.\\
    %   \wtA(\ckcO) & \text{when } \star \in \set{B,C,C^{*}},\\
    % \end{cases}
  \end{equation*}
  Here $\wtA(\ckcO)/\{\wp\sim\wp^{c}\}$ denotes the quotient of $\wtA(\ckcO)$ by
  identifying $\wp$ with its complement $\wp^{c}$. Note that by its very definition, we have $\wtA(\ckcO)=\wtA(\ckcO_{\mathrm g})$ and
  $\barA(\ckcO)=\barA(\ckcO_{\mathrm g})$.

\begin{remark} When $\star\neq \wtC$, $\barA(\ckcO)$ gives another description of Lusztig's canonical
  quotient attached to $\ckcO$. The set $\CPP_{\star}(\ckcO)$ appears implicitly in
\cite{So}*{Section~5}.

  \trivial[h]{ This can be seen from the following
    lemma, c.f. \cite{BVUni}*{Proposition~5.28}. }
\end{remark}

  To simplify the notation, we write $\LC_{\ckcO}:= \LC_{\lambda_{\ckcO}}$. Recall that $\LC_{\ckcO}$ is the multiset of irreducible constituents of
  \[
    \LV_{\ckcO}:= \left(J_{\Wlamck}^{\WLamck} \sgn\right) \otimes \sgn.
  \]

  \begin{lem}[c.f. Barbasch-Vogan {\cite{BVUni}*{Proposition~5.28}}]
    \label{lem:Lcell}
    In all cases, $\LC_{\ckcO}$ is multiplicity free, and we have the
    following bijections:
    \[
      \begin{array}{lccccccc}
        \barA(\ckcO)&=&\barA(\ckcO_{\mathrm g}) & \longrightarrow & \LC_{\ckcO_{\mathrm g}}
        & \longrightarrow & \LC_{\ckcO}\\
                    &  &\wp & \mapsto & \tau_{\wp} &
                                                     \mapsto & \tau_{\wp}\otimes \tau_{\mathrm b}.
      \end{array}
    \]
    Moreover,
    \[
      \tau_{\ckcO}:=\tau_{\emptyset}\otimes \tau_{\mathrm b}
    \] is the unique special representation in $\LC_{\ckcO}$ and
    \begin{equation}\label{eq:dBV.W}
      \Spr ^{-1}(j_{\WLamck}^{W}(\tau_{\ckcO})) = \dBV(\ckcO) = \dBV(\ckcO_{\mathrm g}) \cupcol \ckcO_{\mathrm b}^{t}.
    \end{equation}
    \trivial[h]{ The last equality could be checked using Sommer's formula on
      Springer correspondence directly: double columns $(2c+1,2c+1)$ corresponds
      to
      $ B_{c=\alpha_{2i-1}}\times D_{c+1=\alpha_{2i}+1}=D_{c+1=\alpha_{2i-1}+1}\times C_{c=\alpha_{2i}}$
      factor in type $B,\wtC$. double columns $(2c,2c)$ corresponds to factor
      $D_{c=\beta_{2i-1}}\times C_{c=\beta_{2i}}=D_{c=\beta_{2i-1}}\times B_{c=\beta_{2i}}$
      in type $C,C^{*},D,D^{*}$. } Here $\dBV$ is the metaplectic dual if
    $\star=\wtC$ and is the Barbasch-Vogan dual otherwise.
  \end{lem}
  \begin{proof}
    For $\ckcO_{\mathrm g}$, the lemma is given in \cite{BVUni}*{Proposition~5.28}. In general, the lemma follows from an induction on number of columns of
    $\ckcO_{\mathrm g}$ using Lusztig's formula of $J$-induction in \cite{Lu}*{\S 4.4-4.6}. The equality
    \eqref{eq:dBV.W} for linear groups is in \cite{BVUni}*{Proposition~A2}.

    \trivial[h]{ {\bf Suppose $\star=C$.}

      In this case, bad parity is even and each row length occur with even
      multiplicity. Suppose
      $\ckcO_{\mathrm b} = (C_{1}, C_{1}, C_{2},C_{2}, \cdots, C_{k'},C_{k'})$ with
      $c_{1}=2k$ and $k' = \bfrr_{1}(\ckcO_{\mathrm b})$.
      \[
        W_{\lamckb} = S_{C_{1}}\times S_{C_{2}}\times \cdots S_{C_{k'}}.
      \]
      The symbol of trivial representation of trivial group of type D is
      \[
        \binom{0,1, \cdots, k-1}{0,1, \cdots, k-1}.
      \]
      Now it is easy to see that (use the similar computation as below)
      \[
        J_{W_{\lamckb}}^{W_{\mathrm b}}\sgn = ((\half C_{1}, \half C_{2},\cdots, \half C_{k'}),(\half C_{1}, \half C_{2},\cdots, \half C_{k'})).
      \]


      For the good parity part. Let
      $r'_{i} = \floor{\half(\bfrr_{i}(\ckcO_{\mathrm g})-\bfrr_{i+1}(\ckcO_{\mathrm g}))}$.
      Suppose $\ckcO_{\mathrm g}$ has $2l+1$ columns (superscripts denote the
      multiplicity)
      \[
        \ckcO_{\mathrm g} = ((2l+1)^{2r'_{2l+1}+1}, 2l^{2r'_{2l}}, (2l-1)^{2r'_{2l-1}}, \cdots, 2^{2r'_{2}}, 1^{2r'_{1}} )
      \]
      and
      % $\ckcO_{\mathrm g} = (2c_{1}+1, C_{2}, C_{2},C_{3},C_{3},\cdots, C_{k'},C_{k'})$
      % with $2c_{1}+1=2l+1$ and $2k'+1 = \bfrr_{1}(\ckcO_{\mathrm g})$.
      \[
        W_{\lamckg} = W_{l}\times \underbrace{S_{2l+1}\times \cdots \times S_{2l+1}}_{2r'_{2l+1}\text{-terms}} \times \prod_{i<2l+1} \underbrace{S_{i}\times \cdots\times S_{i}}_{r'_{i}\text{-terms}}
      \]

      The symbol of sign representation of $W_{l}$ is
      \[
        \binom{0,1,2, \cdots, l}{1,2, \cdots, l}.
      \]
      The induction begins with the longest columns to the shorter columns

      Induce to include all $2l+1$-length columns yields
      \[
        \binom{r'_{2l+1}+0,r'_{2l+1}+1,r'_{2l+1}+2, \cdots, r'_{2l+1}+l}{ r'_{2l+1}+1,r'_{2l+1}+2, \cdots, r'_{2l+1}+l}.
      \]
      Now move the shorter columns, we see that when even columns
      $(2i)^{2r'_{2i}}$ occurs, it adds $(i)^{r'_{2i}}$ columns on the both
      sides of the bipartition; when odd columns $(2i+1)^{r'_{2i+1}}$ occur, the
      bifurcation happens: one can
      \begin{itemize}
        \item attach columns $(i+1)^{r'_{2i+1}}$ on the left and columns
              $(i)^{r'_{2i+1}}$ on the right, which corresponds to
              $(2i+1,2i+2)\neq \wp$, or
        \item attach columns $(i)^{r'_{2i+1}}$ on the left and columns
              $(i+1)^{r'_{2i+1}}$ on the right, which corresponds to
              $(2i+1,2i+2)\in \wp$,
      \end{itemize}

      Therefore,
      \[
        \begin{array}{ccc}
          J_{W_{\lamckg}}^{W_{\mathrm g}} \sgn
          &\leftrightarrow&  \bF_{2}(\CPP(\ckcO_{\mathrm g}))\\
          (\cktau_{L},\cktau_{R}) =:\cktau_{\wp}&\leftrightarrow & \wp
        \end{array}
      \]
      where
      \[
        \bfrr_{l+1}(\cktau_{L}) = r'_{2l+1} = \half (\bfrr_{2l+1}(\ckcO_{\mathrm g})-1)
      \]
      and, if $(2i-1,2i)\notin \wp$,
      \[
        \begin{split}
          \bfrr_{i}(\cktau_{L}) & = \sum_{l\geq 2i-1} r'_{l}
          = \half(\bfrr_{2i-1}(\ckcO)-1)\\
          \bfrr_{i}(\cktau_{R}) & = 1 + \sum_{l\geq 2i} r'_{l} = \half(\bfrr_{2i}(\ckcO)+1)
        \end{split}
      \]
      if $(2i-1,2i)\in \wp$,
      \[
        \begin{split}
          \bfrr_{i}(\cktau_{L}) & = \sum_{l\geq 2i} r'_{l}
          = \half(\bfrr_{2i}(\ckcO)-1)\\
          \bfrr_{i}(\cktau_{R}) & = 1 + \sum_{l\geq 2i-1} r'_{l} = \half(\bfrr_{2i-1}(\ckcO)+1)
        \end{split}
      \]

      % \[
      %   \begin{split}
      %     \bfrr_{l+1}(\cktau_{L}) & = r'_{2l+1} =
      %     \half (\bfrr_{2l+1}(\ckcO_{\mathrm g})-1)\\
      %     (\bfrr_{i}(\cktau_{L}), \bfrr_{i}(\cktau_{R})) & =
      %     \begin{cases}
      %       (\half(\bfrr_{2i-1}(\ckcO_{\mathrm g})-1), \half(\bfrr_{2i}(\ckcO_{\mathrm g})+1)) & (2i-1,2i)\notin \wp\\
      %       (\half(\bfrr_{2i}(\ckcO_{\mathrm g})-1), \half(\bfrr_{2i-1}(\ckcO_{\mathrm g})+1)) & (2i-1,2i)\in \wp
      %     \end{cases}
      %   \end{split}
      % \]

      Since $\tau_{\wp} = \cktau_{\wp}\otimes \sgn$, we get the claim.

      We adopt the convention that
      \[
        \sfS_{\cO} := \prod_{i\in \bN^{+}}\sfS_{\bfcc_{i}(\cO)}
      \]
      so that $j_{\sfS_{\cO}}^{\sfS_{\abs{\cO}}}\sgn = \cO$ for each partition
      $\cO$.

      Now consider the orbit under the Springer correspondence.

      Let
      $\ckcO'_{\mathrm b}: = [\bfrr_{2}(\ckcO_{\mathrm b}), \bfrr_{4}(\ckcO_{\mathrm b}),\cdots, \bfrr_{2k}(\ckcO_{\mathrm b})]$,
      $\cO'_{\mathrm b}:=(\ckcO'_{\mathrm b})^{t}$ and $\cO_{\mathrm b}:=\cO'_{\mathrm b}\cupcol \cO'_{\mathrm b}$.
      Clearly, $\ckcO_{\mathrm b} = \ckcO'_{\mathrm b}\cuprow \ckcO'_{\mathrm b}$. Note that
      $\tau_{\mathrm b} = j_{S_{\cO'_{\mathrm b}}}^{W'_{\mathrm b}} \sgn$ (by the formula of fake degree
      see Lusztig or Carter's book). So, by induction by stage of $j$-induction,
      we have
      \[
        \wttau_{\cO}:= j_{W'_{\mathrm b}\times W_{\mathrm g}}^{W_{n}} (\tau_{\mathrm b}\otimes \tau_{\emptyset}) = j_{S_{\cO'_{\mathrm b}}\times W_{\mathrm g}}^{W_{n}} \sgn\otimes \tau_{\wp}.
      \]
      By Barbasch-Vogan, $\cO_{\mathrm g}:=\Spr(\tau_{\emptyset}) = d_{BV}(\ckcO_{\mathrm g})$,
      which is well know how to calculate. (In fact, one can deduce the result
      by our computation. )

      Since the Springer correspondence commutes with parabolic induction, we
      get
      $\Spr(\wttau) = \Ind_{\GL_{\cO'_{\mathrm b}}\times \Sp(2g)}^{\Sp(2n)} 0\times \cO_{\mathrm g} = \cO_{\mathrm b}\cupcol \cO_{\mathrm g}$.


      \medskip

      {\bf Suppose $\star=D$.}

      The bad parity part is the same as that of the case when $\star = C$.

      Now consider the good parity part.
      \[
        \ckcO_{\mathrm g} = ((2l)^{2r'_{2l}+1}, (2l-1)^{2r'_{2l-1}}, (2l-2)^{2r'_{2l-2}}, \cdots, 2^{2r'_{2}}, 1^{2r'_{1}} )
      \]
      and
      \[
        W_{\lamckg} = W'_{l}\times \underbrace{S_{2l}\times \cdots \times S_{2l}}_{2r'_{2l}\text{-terms}} \times \prod_{i<2l} \underbrace{S_{i}\times \cdots\times S_{i}}_{r'_{i}\text{-terms}}
      \]

      The symbol of sign representation of $W'_{l}$ is
      \[
        \binom{0,1, \cdots, l-1}{1,2, \cdots, l\phantom{-1}}.
      \]
      (Here we always made the choice of the top and bottom row to compatible
      with the type $C$ case. )

      Induce to include all $2l$-length columns yields
      \[
        \binom{r'_{2l}+0,r'_{2l}+1, \cdots, r'_{2l}+l-1}{ r'_{2l}+1,r'_{2l}+2, \cdots, r'_{2l}+l\phantom{-1}}.
      \]
      Now move the shorter columns. When odd columns $(2i+1)^{2r'_{2i+1}}$
      occurs, it adds $(i)^{r'_{2i+1}}$ columns on the left and
      $(i+1)^{r'_{2i+1}}$ on the right. When even columns $(2i)^{r'_{2i}}$
      occur, the bifurcation happens: one can
      \begin{itemize}
        \item attach columns $(i)^{r'_{2i}}$ on the left and columns
              $(i)^{r'_{2i}}$ on the right, which corresponds to
              $(2i,2i+1)\neq \wp$, or
        \item attach columns $(i-1)^{r'_{2i}}$ on the left and columns
              $(i+1)^{r'_{2i}}$ on the right, which corresponds to
              $(2i,2i+ 1)\in \wp$,
      \end{itemize}

      Therefore,
      \[
        \begin{array}{ccc}
          \bF_{2}(\CPP(\ckcO_{\mathrm g}))&\longrightarrow
          & J_{W_{\lamckg}}^{W_{\mathrm g}} \sgn \\
          \wp&\mapsto&    (\cktau_{L},\cktau_{R}) =:\cktau_{\wp}
        \end{array}
      \]
      where
      \[
        \bfrr_{l}(\cktau_{L}) = r'_{2l} = \half (\bfrr_{2l}(\ckcO_{\mathrm g})-1)
      \]
      \[
        \bfrr_{1}(\cktau_{R}) = 1+ \sum_{i} r'_{i} = \half (\bfrr_{1}(\ckcO_{\mathrm g})+1)
      \]
      and, if $(2i,2i+1)\notin \wp$,
      \[
        \begin{split}
          \bfrr_{i}(\cktau_{L}) & = \sum_{l\geq 2i} r'_{l}
          = \half(\bfrr_{2i}(\ckcO)-1)\\
          \bfrr_{i+1}(\cktau_{R}) & = 1 + \sum_{l\geq 2i+1} r'_{l} = \half(\bfrr_{2i+1}(\ckcO)+1)
        \end{split}
      \]
      if $(2i,2i+1)\in \wp$,
      \[
        \begin{split}
          \bfrr_{i}(\cktau_{L}) & = \sum_{l\geq 2i+1} r'_{l}
          = \half(\bfrr_{2i+1}(\ckcO)-1)\\
          \bfrr_{i}(\cktau_{R}) & = 1 + \sum_{l\geq 2i} r'_{l} = \half(\bfrr_{2i}(\ckcO)+1)
        \end{split}
      \]

      Also note that $\cktau_{\wp}=\cktau_{\wp^{c}}$. The rest parts are the
      same as that of type $C$.

      {\bf Suppose $\star=B$. }

      In this case, bad parity is odd and every odd row occurs with even
      times.

      We can write
      $r'_{i} := \floor{\half(\bfrr_{i}(\ckcO_{\mathrm b})-\bfrr_{i-1}(\ckcO_{\mathrm b}))}$
      \[
        \ckcO_{\mathrm b} % = [2r_{1}+1, 2r_{1}+1, \cdots, 2r_{k}+1,2r_{k}+1]
        % = (2c_{0},2c_{1},2c_{1}, \cdots, 2c_{l}, 2c_{l}).
        = ((2l)^{2r'_{2l}+1}, (2l-1)^{2r'_{2l-1}},\cdots, 1^{2r'_{1}})
      \]
            %             with $k = c_{0}$ and $l = r_{1}$.
      Then
      \[
        W_{\lamckb} = W_{l} \times \underbrace{S_{2l}\times \cdots \times S_{2l}}_{2r'_{2l}\text{-terms}} \times \prod_{i<2l} \underbrace{S_{i}\times \cdots\times S_{i}}_{r'_{i}\text{-terms}}
      \]
      (Note that in the product, $r'_{i}=0$ if $i$ is odd.) The computation of
      $\cksigma_{\mathrm b} = J_{W_{\lamckb}}^{W_{\mathrm b}} \sgn$ is similar to that of the
      good parity for type $C$ with no bifurcating, one deduce that
      $J$-induction and $j$-induction gives the same result.
      \[
        \begin{split}
          \cksigma_{\mathrm b} &=
          \binom{0, 1+r_{l}, 2+r_{l-1}\cdots, l+r_{1}}{1+r_{l},2+r_{l-1}, \cdots, l+r_{1}}\\
          & = ([r_{1},r_{2},\cdots, r_{l}],[r_{1}+1,r_{2}+1,\cdots,r_{l}+1])\\
        \end{split}
      \]
      with $r_{i} = \half\bfrr_{2i-1}(\ckcO_{\mathrm b}) = \half\bfrr_{2i}(\ckcO_{\mathrm b})$.
      Now
      \[
        \sigma_{\mathrm b} = ((r_{1}+1,r_{2}+1,\cdots,r_{l}+1), (r_{1},r_{2},\cdots, r_{l})) = j_{S_{\cO'_{\mathrm b}}}^{W_{\mathrm b}}\sgn
      \]
      where
      $\cO'_{\mathrm b}=(\bfrr_{2}(\ckcO_{\mathrm b}),\bfrr_{4}(\ckcO_{\mathrm b}),\cdots, \bfrr_{2l}(\ckcO_{\mathrm b}))$.
      Under the Springer correspondence of type $B$, it corresponds to
      $\Ind_{\GL_{\mathrm b}}^{\SO(2b+1)}\cO'_{\mathrm b} = \cO'_{\mathrm b}\cuprow \cO'_{\mathrm b}\cuprow (1)$.

      % \[
      %   \begin{split}
      %     \cksigma_{\mathrm b} &:= \sigma_{\mathrm b}\otimes \sgn = j_{W_{\lamckb}}^{W_{\mathrm b}} \sgn \\
      %     & = %\dagger_{2c_{l}}\cdots \dagger_{2c_{1}}
      %     \sigma_{\mathrm b}\otimes \sgn = j_{W_{\lamckb}}^{W_{\mathrm b}} \sgn\otimes
      %     \binom{0, 1, \cdots, c_{0}}{1, \cdots, c_{0}}\\
      %     & =
      %     \binom{0, 1+r_{k}, 2+r_{k-1}\cdots, c_{0}+r_{1}}{1+r_{k},2+r_{k-1}, \cdots, c_{0}+r_{1}}\\
      %     & = ([r_{1},r_{2},\cdots, r_{k}],[r_{1}+1,r_{2}+1,\cdots,r_{k}+1])\\
      %     &= ((c_{1},c_{2},\cdots, c_{k}),(c_{0},c_{1}, \cdots, c_{l}))\\
      %   \end{split}
      % \]


      % We take the convention that $\dagger \cO = [r_{i}+1]$. By abuse of
      % notation, let $\dagger_{n} \sigma$ denote the
      % $j_{S_{n} \times W_{\abs{\sigma}}}^{W_{n+\abs{\sigma}}} \sgn\otimes \sigma$.
      % We can write
      % \[
      %   \ckcO_{\mathrm b} = [2r_{1}+1, 2r_{1}+1, \cdots, 2r_{k}+1,2r_{k}+1] = (2c_{0},2c_{1},2c_{1}, \cdots, 2c_{l}, 2c_{l})
      % \]
      % with $k = c_{0}$ and $l = r_{1}$.

      % \[
      %   \begin{split}
      %     W_{\lamckb} &= W_{c_{0}} \times S_{2c_{1}} \times S_{2c_{2}}\times \cdots \times S_{2c_{l}}\\
      %     \cksigma_{\mathrm b} &:= \sigma_{\mathrm b}\otimes \sgn = j_{W_{\lamckb}}^{W_{\mathrm b}} \sgn \\
      %     & = \dagger_{2c_{l}}\cdots \dagger_{2c_{1}}
      %     \binom{0, 1, \cdots, c_{0}}{1, \cdots, c_{0}}\\
      %     & =
      %     \binom{0, 1+r_{k}, 2+r_{k-1}\cdots, c_{0}+r_{1}}{1+r_{k},2+r_{k-1}, \cdots, c_{0}+r_{1}}\\
      %     & = ([r_{1},r_{2},\cdots, r_{k}],[r_{1}+1,r_{2}+1,\cdots,r_{k}+1])\\
      %     &= ((c_{1},c_{2},\cdots, c_{k}),(c_{0},c_{1}, \cdots, c_{l}))\\
      %   \end{split}
      % \]

      % Therefore
      % \[
      %   \begin{split}
      %     \sigma_{\mathrm b} &= \cksigma_{\mathrm b}\otimes \sgn = ((r_{1}+1,r_{2}+1,\cdots,r_{k}+1),(r_{1},r_{2},\cdots, r_{k})) \\
      %     & = j_{S_{2r_{1}+1}\times \cdots S_{2r_{k}+1}}^{W_{\mathrm b}} \sgn\\
      %     & = j_{S_{\mathrm b}}^{W_{\mathrm b}} (2r_{1}+1, 2r_{2}+1, \cdots, 2r_{k}+1)
      %   \end{split}
      % \]
      % which corresponds to the orbit
      % \[
      %   \cO_{\mathrm b} = (2r_{1}+1, 2r_{1}+1,2r_{2}+1, 2r_{2}+1, \cdots,2r_{k}+1, 2r_{k}+1 ) = \ckcO_{\mathrm b}^{t}.
      % \]
      % (Note that $\cO'_{\mathrm b} = (2r_{1}+1,2r_{2}+1, \cdots, 2r_{k}+1)$ which
      % corresponds to $j_{W_{L_{\mathrm b}}}^{S_{\mathrm b}}\sgn$ and
      % $\ind_{L}^{G} \cO'_{\mathrm b} = \cO_{\mathrm b}$. ) This implies the unique special
      % representation is
      % \[
      %   \sigma_{\mathrm b} = (j_{W_{\lamckb}}^{W_{\mathrm b}}\sgn), \quad \text{where
      % } W_{L,b} = \prod_{i=1}^{k} S_{2r_{i}+1}.
      % \]
      % The $J$-induction is calculated by \cite{Lu}*{(4.5.4)}. It is easy to
      % see that in our case $J_{W_{\lamckb}}^{W_{\mathrm b}} \sgn$ consists of the
      % single special representation by induction.


      Now we consider the good parity part, where each row of $\ckcO_{\mathrm g}$ has
      even length.

      Assume $r'_{i} := \half(\bfrr_{i}(\ckcO_{\mathrm g})-\bfrr_{i-1}(\ckcO_{\mathrm g}))$ and
      so
      \[
        \ckcO_{\mathrm g} % = [2r_{1}+1, 2r_{1}+1, \cdots, 2r_{k}+1,2r_{k}+1]
        % = (2c_{0},2c_{1},2c_{1}, \cdots, 2c_{l}, 2c_{l}).
        = ((2l+1)^{2r'_{2l+1}}, (2l)^{2r'_{2l}},\cdots, 1^{2r'_{1}})
      \]
      % Consider
      % \[
      %   \cO_{\mathrm g} = [2r_{1},2r_{2}, \cdots, 2r_{2k-1},2r_{2k}] = (C_{1},C_{1}, C_{2},C_{2},\cdots, C_{l}, C_{l}).
      % \]
      with $l =\min\set{i|\bfrr_{2i+2}(\ckcO_{\mathrm g}) = 0}$.

      Then
      \[
        W_{\lamckg} = \times \prod_{i\leq 2l+1} \underbrace{S_{i}\times \cdots\times S_{i}}_{r'_{i}\text{-terms}}
      \]

      Note that the trivial representation of the trivial group has symbol
      \[
        \binom{0,1, 2, \cdots, l\phantom{-1}}{0,1, \cdots, l-1}.
      \]


      Induce to include all $2l+1$-length columns yields
      \[
        \binom{r'_{2l+1}+0,r'_{2l+1}+1,r'_{2l+1}+2,\cdots, r'_{2l+1}+l\phantom{-1}}{ r'_{2l+1}+0,r'_{2l+1}+1, \cdots, r'_{2l+1}+l-1}.
      \]
      Now move the shorter columns. When odd columns $(2i+1)^{2r'_{2i+1}}$
      occurs, it adds $(i+1)^{r'_{2i+1}}$ columns on the left and
      $(i)^{r'_{2i+1}}$ on the right. When even columns $(2i)^{r'_{2i}}$ occur,
      the bifurcation happens: one can
      \begin{itemize}
        \item attach columns $(i)^{r'_{2i}}$ on the left and columns
              $(i)^{r'_{2i}}$ on the right, which corresponds to
              $(2i,2i+1)\neq \wp$, or
        \item attach columns $(i-1)^{r'_{2i}}$ on the left and columns
              $(i+1)^{r'_{2i}}$ on the right, which corresponds to
              $(2i,2i+1)\in \wp$.
      \end{itemize}


      Therefore,
      \[
        \begin{array}{ccc}
          \bF_{2}(\CPP(\ckcO_{\mathrm g}))&\longrightarrow
          & J_{W_{\lamckg}}^{W_{\mathrm g}} \sgn \\
          \wp&\mapsto&    (\cktau_{L},\cktau_{R}) =:\cktau_{\wp}
        \end{array}
      \]
      where
      % \[
      %   \bfrr_{2l+1}(\cktau_{L}) = r'_{2l+1} = \half \bfrr_{2l+1}(\ckcO_{\mathrm g})
      % \]
      \[
        \bfrr_{1}(\cktau_{L}) = \sum_{i} r'_{i} = \half \bfrr_{1}(\ckcO_{\mathrm g})
      \]
      and, if $(2i,2i+1)\notin \wp$,
      \[
        \begin{split}
          \bfrr_{i+1}(\cktau_{L}) & = \sum_{l\geq 2i+1} r'_{l}
          = \half\bfrr_{2i+1}(\ckcO)\\
          \bfrr_{i}(\cktau_{R}) & = \sum_{l\geq 2i} r'_{l} = \half\bfrr_{2i}(\ckcO)
        \end{split}
      \]
      if $(2i,2i+1)\in \wp$,
      \[
        \begin{split}
          \bfrr_{i+1}(\cktau_{L}) & = \sum_{l\geq 2i} r'_{l}
          = \half\bfrr_{2i}(\ckcO)\\
          \bfrr_{i}(\cktau_{R}) & = \sum_{l\geq 2i+1} r'_{l} = \half\bfrr_{2i+1}(\ckcO)
        \end{split}
      \]

      Some remarks on the BV-dual. The calculation of $\cO_{\mathrm g}$ from
      $\tau_{\emptyset}$ can be reduced to the case of quasi-distinguished
      orbits (other case are deduced from this by parabolic induction,
      corresponds to attach two even columns for the balanced pairs). Compare
      Sommer's description of Springer correspondence with ours, we deduce that
      \[
        \cO_{\mathrm g} = (\bfrr_{1}(\ckcO_{1})+1,\bfrr_{2}(\ckcO_{2})-1,\bfrr_{3}(\ckcO_{3})+1, \cdots, \bfrr_{2l}(\ckcO_{2l})-1,\bfrr_{2l+1}(\ckcO_{2l+1})+1)
      \]
      The rest parts are similar to that of type $D$.
    }


    We give the main steps of the proof for the case when $\star = \wtC$.


    We first consider the good parity part $\ckcO_{\mathrm g}$, where each row has
    even length.

    Set $r'_{i} := \half(\bfrr_{i}(\ckcO_{\mathrm g})-\bfrr_{i-1}(\ckcO_{\mathrm g}))$,
    $l =\min\set{i|\bfrr_{2i+1}(\ckcO_{\mathrm g})=0}$, and write
    \[
      \ckcO_{\mathrm g} % = [2r_{1}+1, 2r_{1}+1, \cdots, 2r_{k}+1,2r_{k}+1]
      % = (2c_{0},2c_{1},2c_{1}, \cdots, 2c_{l}, 2c_{l}).
      = ((2l)^{2r'_{2l}}, (2l-1)^{2r'_{2l-1}},\cdots, 1^{2r'_{1}})
    \]
    where $i^{r'}$ denotes $r'$-copies of length $i$ columns.
    % Consider
    % \[
    %   \cO_{\mathrm g} = [2r_{1},2r_{2}, \cdots, 2r_{2k-1},2r_{2k}] = (C_{1},C_{1}, C_{2},C_{2},\cdots, C_{l}, C_{l}).
    % \]
    The Weyl group $W_{\mathrm g}$ of good parity is $\sfW'_{n_{\mathrm g}}$ with
    $n_{\mathrm g} = \half\abs{\ckcO_{\mathrm g}}$. For $1\leq k\leq l$, let
    \[
      % S_{r,s} = \prod_{i=r}^{s} \underbrace{\sfS_{i}\times \cdots\times \sfS_{i}}_{r'_{i}\text{-terms}}
      \vec{S}_{i} = \underbrace{\sfS_{i}\times \cdots\times \sfS_{i}}_{r'_{i}\text{-times}} \AND n_{k} = \sum_{i=k}^{2l} i\cdot r'_{i}.
      % \AND n_{r,s} = \sum_{i=r}^{s} i\cdot r'_{i}.
    \]

    Then $W_{\lamckg}=\prod_{i=1}^{l} \vec{S}_{i}$ and
    \[
      \begin{split}
        \ckLV_{\ckcO_{\mathrm g}}& :=J_{W_{\lamckg}}^{\Wg}\sgn\\
        & = J_{\vec{S}_{1}\times \sfW'_{n_{2}}}^{\sfW'_{n_{1}}} \Big(\sgn \otimes J_{\vec{S}_{2}\times \sfW'_{n_{3}}}^{\sfW'_{n_{2}}}\Big(\sgn
        \otimes \cdots\big(J_{\vec{S}_{l}} \sgn\big)\cdots \Big)\Big) \\
      \end{split}
    \]
    Applying \cite{Lu}*{(4.6.2)} inductively, we see that the operation
    $J_{\vec{S}_{i}\times \sfW'_{n_{i+1}}}^{\sfW'_{n_{i}}}(\sgn \otimes \underline{\ \ \ })$ yields a multiplicity-free representation and
    doubles (resp. keeps) the number of irreducible constituents if $i$ is odd (resp. even). Informally we will say that a bifurcation occurs when attaching an odd length column.
    Denote by $\ckLC_{\ckcO_{\mathrm g}}$ the multiset of irreducible constituents of $\ckLV_{\ckcO_{\mathrm g}}$.

    First assume $\CPPs(\ckcO_{\mathrm g}) = \emptyset$. Define
    \[
      \tA'(\ckcO):= \bZ_{2}[\emptyset]
    \]
    to be the trivial group.
    Then $\ckLV_{\ckcO_{\mathrm g}}$ is
    irreducible (with the corresponding bipartition marked by the label $I$).  Hence, $\LC_{\ckcO_{\mathrm g}}$
    %$\LC_{\ckcO)}$
    and $\tA'(\ckcO)$ can be obviously identified.


    Now assume $\CPPs(\ckcO_{\mathrm g})\neq \emptyset$. Then the two parts of the
    bipartition of an irreducible constituent after each operation
    $J_{\vec{S}_{i}\times \sfW'_{n_{i+1}}}^{\sfW'_{n_{i}}}(\sgn \otimes \underline{\ \ \ })$ are different. Let
    \[i_{0}:= \min\Set{i| (2i-1,2i)\in \CPPs(\ckcO_{\mathrm g})}.\]
    % $(2i_{0}-1,2i_{0})$ be the element in such that $i_{0}$ is minimal.
    Then we will have a bijection
    \[
      \tA'(\ckcO):= \set{\wp\in \bZ_{2}[\CPPs(\ckcO_{\mathrm g})]|(2i_{0}-1,2i_{0})\notin \wp} \longrightarrow \ckLC_{\ckcO_{\mathrm g}}
    \]
    which records the bifurcation when attaching odd length columns. More precisely, by Lusztig's formula of $J$-induction, a bijection is given by sending
      $\wp$ to $\cktau_{\wp}:= (\cktau_{L},\cktau_{R})$, where
    \[
      (\bfrr_{i}(\cktau_{L}),\bfrr_{i}(\cktau_{R})) := \begin{cases} (\half\bfrr_{2i}(\ckcO_{\mathrm g}),\half\bfrr_{2i-1}(\ckcO_{\mathrm g}))
        & \text{if } (2i-1,2i)\notin \wp ,\\
        (\half\bfrr_{2i-1}(\ckcO_{\mathrm g}),\half\bfrr_{2i}(\ckcO_{\mathrm g}))
        & \text{if } (2i-1,2i)\in \wp .\\
      \end{cases}
    \]
    By interchanging the two parts of a bipartition belonging to $\ckLC_{\ckcO_{\mathrm g}}$, we thus obtain a bijection of $\tA'(\ckcO)$ with $\LC_{\ckcO_{\mathrm g}}$, which sends $\wp$ to $\tau_{\wp}$.


    \trivial[h]{ Note that the trivial representation of the trivial group is
      represented by the symbol
      \[
        \binom{0,1, 2, \cdots, l-1}{0,1,2, \cdots, l-1}_{I}.
      \]
      % Induce to include all $2l$-length columns yields
      % \[
      %   \binom{r'_{2l+1}+0,r'_{2l+1}+1,r'_{2l+1}+2,\cdots, r'_{2l+1}+l\phantom{-1}}{ r'_{2l+1}+0,r'_{2l+1}+1, \cdots, r'_{2l+1}+l-1}.
      % \]

      Now move the shorter columns. When even columns $(2i)^{2r'_{2i}}$
      occurs, it adds $(i)^{r'_{2i}}$ columns on the left and $(i)^{r'_{2i}}$ on
      the right. When odd columns $(2i-1)^{r'_{2i-1}}$ occur, the bifurcation
      happens: one can
      \begin{itemize}
        \item attach columns $(i-1)^{r'_{2i-1}}$ on the left and columns
              $(i)^{r'_{2i-1}}$ on the right, which corresponds to
              $(2i-1,2i)\neq \wp$, or
        \item attach columns $(i)^{r'_{2i-1}}$ on the left and columns
              $(i-1)^{r'_{2i-1}}$ on the right, which corresponds to
              $(2i-1,2i)\in \wp$.
      \end{itemize}
      Note that when we first encounter the longest odd column, we make the
      choice that the size of left part is larger than that of the right part.
      Now If $(2i-1,2i)\notin \wp$,
      \[
        \begin{split}
          \bfrr_{i}(\cktau_{L}) & = \sum_{l\geq 2i} r'_{l}
          = \half\bfrr_{2i}(\ckcO_{\mathrm g})\\
          \bfrr_{i}(\cktau_{R}) & = \sum_{l\geq 2i-1} r'_{l} = \half\bfrr_{2i-1}(\ckcO_{\mathrm g})
        \end{split}
      \]
      if $(2i-1,2i)\in \wp$,
      \[
        \begin{split}
          \bfrr_{i}(\cktau_{L}) & = \sum_{l\geq 2i-1} r'_{l}
          = \half\bfrr_{2i-1}(\ckcO_{\mathrm g})\\
          \bfrr_{i}(\cktau_{R}) & = \sum_{l\geq 2i} r'_{l} = \half\bfrr_{2i}(\ckcO_{\mathrm g})
        \end{split}
      \]
    }


    Now we consider the bad parity part, where each row has
    odd length.
    %For a partition $\cO$, we set
    %\[
    % \sfS_{\cO} := \prod_{i\in \bN^{+}}\sfS_{\bfcc_{i}(\cO)}
    %\]
    %so that $j_{\sfS_{\cO}}^{\sfS_{\abs{\cO}}}\sgn = \cO$.



    Suppose $\ckcO_{\mathrm b}$ is nonempty
    such that
    \[
      \ckcO_{\mathrm b} = (2c_{0},2c_{1}, 2c_{1}, 2c_{2},2c_{2}, \cdots, 2c_{k}, 2c_{k})
    \]
    where $2k+1=\bfrr_{1}(\ckcO_{\mathrm b})$ and $2c_{i} = \bfcc_{2i+1}(\ckcO_{\mathrm b})$.
    Now
    \[
      W_{\lamckb} = \sfW_{c_{0}} \times \sfS_{2c_{1}} \times \sfS_{2c_{2}}\times \cdots \times \sfS_{2c_{k}}
    \]
    and %$J_{W_{\lamckb}}^{W_{\mathrm b}}\sgn$ is irreducible by
    \[
      \begin{split}
        \cktau_{\mathrm b} &:= J_{W_{\lamckb}}^{W_{\mathrm b}} \sgn
        = ((c_{1},c_{2},\cdots, c_{k}),(c_{0},c_{1}, \cdots, c_{l}))\\
        & = \big([\half(\bfrr_{2}(\ckcO)-1),\half(\bfrr_{4}(\ckcO)-1),\cdots, \half(\bfrr_{2c_{0}}(\ckcO)-1)],\\
        & \hspace{2em} [\half(\bfrr_{2}(\ckcO)+1),\half(\bfrr_{4}(\ckcO)+1),\cdots, \half(\bfrr_{2c_{0}}(\ckcO)+1)]\big)
      \end{split}
    \]
    is irreducible by \cite{Lu}*{(4.5.4)}. Tensoring with sign yields the
    formula of $\tau_{\mathrm b}$. Moreover, by the fake degree formula (see
    \cite{Carter}*{p~376}), we have
    \[
      \tau_{\mathrm b} = \cktau_{\mathrm b}\otimes \sgn = j_{\sfS_{\cO'_{\mathrm b}}}^{\sfW_{n_{\mathrm b}}} \sgn,
    \]
    where
    $ \cO'_{\mathrm b} := (\ckcO'_{\mathrm b})^{t}:=(\bfrr_{2}(\ckcO),\bfrr_{4}(\ckcO),\cdots , \bfrr_{2c_{0}}(\ckcO))$, and $\sfS_{\cO'_{\mathrm b}} := \prod_{1\leq i\leq c_{0}}\sfS_{\bfrr_{2i}(\ckcO)}$. The proof for the first part is now complete.

    \medskip \def\ckfll{\check{\fll}}

    Next we give the main steps of the proof for the second part, when $\star = \wtC$.
    %the proof of \eqref{eq:dBV.W}.

    Recall the definition of the metaplectic Barbasch-Vogan dual in
    \cite{BMSZ1}. A key property is that this duality map commutes with parabolic induction: Suppose
    $\ckfll\subset \ckfgg$ is a parabolic subalgebra of $\ckfgg$ and
    $\fll$ is the corresponding parabolic subalgebra in $\fgg$, then
    \begin{equation*}%\label{eq:inddBV}
      \dBV(\ckcO) =  \Ind_{\fll}^{\fgg}(\dBV(\ckcO_{\ckfll}))
    \end{equation*}
    for each nilpotent orbit $\ckcO$ in $\ckfgg$ such that
    $\ckcO_{\ckfll}:=\ckcO\cap \ckfll\neq \emptyset$. This is clear by reducing
    to the type $B$ case, as in \cite{BMSZ1}*{Proposition~3.8}. By removing pairs
    of rows with the same lengths in $\ckcO$, we are reduced to check the equality in \eqref{eq:dBV.W}
    in the case when $\ckcO_{\mathrm b}=\emptyset$ and
    $\bfrr_{2i-1}(\ckcO_{\mathrm g})>\bfrr_{2i}(\ckcO_{\mathrm g})$ for all $i$ such that
    $i\leq \bfcc_{1}(\ckcO_{\mathrm g})$. In this case, both sides of \eqref{eq:dBV.W}
    can be easily computed and are equal to
    \[
      (\bfrr_{1}(\ckcO)-1, \bfrr_{2}(\ckcO)+1, \cdots,\bfrr_{2c-1}(\ckcO)-1,\bfrr_{2c}(\ckcO)+1),
    \]
    where $c = \min\set{i|\bfrr_{2i+1}(\ckcO)=0}$. The general case follows from the aforementioned compatibility with parabolic induction.  \end{proof}



    % Now we compare the metaplectic dual defined in \cite{BMSZ1} with the Weyl
    % group representations.
    \trivial[h]{ Compare Sommer's description of Springer correspondence we
      deduce that the RHS is
      \[
        \cO_{\mathrm g} = (\bfrr_{1}(\ckcO_{1})-1,\bfrr_{2}(\ckcO_{2})+1,\bfrr_{3}(\ckcO_{3})+1, \cdots, \bfrr_{2l-1}(\ckcO_{2l-1})-1,\bfrr_{2l}(\ckcO_{2l})+1)
      \]
      The LHS is calculated by $((((\ckcO^{t})_{D})^{+})^{-})_{C}$. We write
      $R_{i}=\bfrr_{i}(\ckcO)=2r_{i}$. Now under our assumption,
      $R_{2i-1}>R_{2i}$, we have
      \[
        \begin{split}
          ((((\ckcO^{t})_{D})^{+})^{-})_{C} &=
          ((((R_{1},R_{2}, \cdots, R_{2l-1},R_{2l})_{D})^{+})^{-})_{C}\\
          &=((R_{1}-1,R_{2}, \cdots, R_{2l-1},R_{2l},1))_{C}\\
          &=(R_{1}-1,R_{2}+1, \cdots, R_{2l-1}-1,R_{2l}+1)\\
        \end{split}
      \]
      So the proof is done.

    }

    %At last, one can see that
    %$\dBV(\ckcO_{\mathrm b}\cuprow \ckcO_{\mathrm g}) = \ckcO_{\mathrm b}^{t} \cupcol \dBV(\ckcO_{\mathrm g})$
    %using \eqref{eq:inddBV}.


  % \begin{remark}
  %   When $\star=\wtC$, one can see that
  %   $\dBV(\ckcO_{\mathrm b}\cuprow \ckcO_{\mathrm g}) = \ckcO_{\mathrm b}^{t} \cupcol \dBV(\ckcO_{\mathrm g})$
  %   using \eqref{eq:inddBV}. \trivial[]{ This could be checked using the
  %   formula of Springer correspondence directly, see Sommer's. }
  % \end{remark}
%

Recall from \eqref{eq:Wbg} that $\Wg =\sfW'_{n_{\mathrm g}}$, when $\star \in \set{\wtC,D,D^{*}}$. Since the representation theory of $\sfW_{n}$ is more elementary than that of
$\sfW'_{n}$, we prefer to express everything in terms of $\sfW_{n}$. For this reason, we also define
\begin{equation}\label{eq:ttauwp}
\wttau_{\wp} = (\imath_{\wp},\jmath_{\wp}),
\end{equation}
for any $\wp\in \tA(\ckcO)$. See \eqref{eq:tauwp} for the description of $\imath_{\wp}$ and $\jmath_{\wp}$.

For later use, we record the following lemma, which follows immediately from our explicit descriptions of $\tau_{\wp}$ and $\tau_{\mathrm b}$. 

\begin{lem}\label{lem:WLcell}
  Suppose $\star=\wtC$.  Then
  \[
    \Ind_{\sfW'_{n_{\mathrm g}}\times \sfW_{n_{\mathrm b}}}^{\sfW_{n_{\mathrm g}}\times \sfW_{n_{\mathrm b}}} \tau_{\wp}\boxtimes \tau_{\mathrm b} =
    \begin{cases}
       \wttau_{\emptyset}\boxtimes \tau_{\mathrm b} & \text{if } \tA(\ckcO) =\emptyset,\\
      \wttau_{\wp}\boxtimes \tau_{\mathrm b} \oplus \wttau_{\wp^{c}}\boxtimes \tau_{\mathrm b}
      &\text{otherwise}.
    \end{cases}
  \]
Let
\[
  \tLV_{\ckcO}:= \Ind_{\sfW'_{n_{\mathrm g}}\times \sfW_{n_{\mathrm b}}}^{\sfW_{n_{\mathrm g}}\times \sfW_{n_{\mathrm b}}}\LV_{\ckcO}
\]
and $\tLC_{\ckcO}$ be the set of irreducible constituents of the
$\sfW_{n_{\mathrm g}}\times \sfW_{n_{\mathrm b}}$-module $\tLV_{\ckcO}$. Then we have a
bijection:
  \[
      \begin{array}{lccccccc}
        \tA(\ckcO)&=&\tA(\ckcO_{\mathrm g}) & \longrightarrow & \tLC(\ckcO_{\mathrm g})
        & \longrightarrow & \tLC(\ckcO)\\
                  &  &\wp & \mapsto & \wttau_{\wp}
        & \mapsto & \wttau_{\wp}\otimes \tau_{\mathrm b}.
      \end{array}
  \]
  Suppose $\star\in\set{D,D^{*}}$. Then
  \[
    \Ind_{\sfW'_{n_{\mathrm g}}\times \sfW'_{n_{\mathrm b}}}^{\sfW_{n_{\mathrm g}}\times \sfW_{n_{\mathrm b}}} \tau_{\wp}\boxtimes \tau_{\mathrm b} =
    \begin{cases}
      \wttau_{\mathrm b} & \text{if } n_{\mathrm g}=0,\\
      \wttau_{\wp}\boxtimes \wttau_{\mathrm b}  \oplus \wttau_{\wp}^{s}\boxtimes \wttau_{\mathrm b}
      &\text{otherwise}.
    \end{cases}
  \]
  Here $\wttau_{\wp}^{s}:= \wttau_{\wp}\otimes \brsgn \neq \wttau_{\wp}$, and $\wttau_{\mathrm b} = \Ind_{\sfW'_{n_{\mathrm b}}}^{\sfW_{n_{\mathrm b}}}\tau_{\mathrm b}$.

  % Then we have a bijection
  % \[
  %     \begin{array}{lccccccc}
  %       \tA(\ckcO)&=&\tA(\ckcO_{\mathrm g}) & \longrightarrow & \tLC(\ckcO_{\mathrm g})
  %       & \longrightarrow & \tLC(\ckcO)\\
  %                   &  &\wp & \mapsto & \wttau_{\wp} &
  %                                                    \mapsto & \tau_{\mathrm b}\otimes \wttau_{\wp}.
  %     \end{array}
  % \]
\end{lem}


\subsection{Coherent continuation representations}\label{subsec:explicitCoh}


% \subsubsection{Some subgroups of the Weyl groups}

We first recall some subgroups of relevant Weyl groups and related branching rules.


In the following, all symbols $a,b,c,d,n,p,q,r,s,t$ denote natural numbers. We view $\sfW_{t}$ and
$\sfS_{t}$ as reflection groups acting on $\bC^{t}$ as usual. Let
$\sfH_{t} := \sfW_t\ltimes \set{\pm 1}^t$ be the subgroup in $\sfW_{2t}$ such
that
\begin{itemize}
  \item the first factor $\sfW_{t}$ sits in $\sfS_{2t}$ as the centralizer of the
        involution
        \[
        (12)(34)\cdots ((2t-1)(2t)).
        \]
  \item The element $(1,\cdots,1, \underbrace{-1}_{i\text{-th
        term}}, 1, \cdots, 1)\in \set{\pm 1}^{t}$ acts on $\bC^{2t}$ by
        \[
          % (x_{1},\cdots, x_{2i-2}, x_{2i-1}, x_{2i},x_{2i+1},\cdots, x_{2t} )
        (x_{1},x_{2},\cdots, x_{2t} ) \mapsto (x_{1},\cdots, x_{2i-2}, -x_{2i},-x_{2i-1},x_{2i+1},\cdots, x_{2t}).
        \]
\end{itemize}
Note that $\sfH_{t}$ is also a subgroup of $\sfW'_{2t}$. Define the quadratic
character
\[
  \begin{array}{rccc}
    \hsgn := 1\otimes \sgn\colon & \sfH_{t}=  \sfW_{t}\ltimes \set{\pm 1}^{t}& \longrightarrow & \set{\pm 1}\\
                                 & (g,(a_{1},a_{2},\cdots, a_{t})) & \mapsto & a_{1}a_{2}\cdots a_{t}.
  \end{array}
\]
Here are the two formulas which are important for us (\cite{Mc}*{p220 (6)}):
\begin{equation}\label{eq:CC.C}
  \Ind_{\sfH_{t}}^{\sfW_{2t}} \hsgn = \sum_{\sigma\in \Irr(\sfS_{t})} (\sigma,\sigma)
  \AND
  \Ind_{\sfH_{t}}^{\sfW'_{2t}} \hsgn = \sum_{\sigma\in \Irr(\sfS_{t})} (\sigma,\sigma)_{I}.
\end{equation}


Recall the unique non-trivial character $\epsilon$ of $\sfW_{n}$ which is trivial on $\sfS_{n}$. Denote by $\bsgn$ the inflation of the sign representation of $\sfS_{n}$ to
$\sfW_{n}$. Then we have $\epsilon = \sgn \otimes \bsgn$.

Two other formulas important for us are
\begin{equation}\label{eq:indSW}
\begin{split}
\Ind_{\sfS_{n}}^{\sfW_{n}}\sgn = \bigoplus_{a+b=n}\Ind_{\sfW_{a}\times \sfW_{\mathrm b}}^{\sfW_{n}} \bsgn\boxtimes \sgn =\bigoplus_{a+b=n} ((a,),(b,)), \\
\Ind_{\sfS_{n}}^{\sfW_{n}} 1 = \bigoplus_{a+b=n}\Ind_{\sfW_{a}\times \sfW_{\mathrm b}}^{\sfW_{n}} 1 \boxtimes \epsilon =\bigoplus_{a+b=n} ([a,],[b,]).
\end{split}
\end{equation}

% Note that we have chosen the embedding $W_{t}\subset S_{2t}$ in $W_{2t}$. We
% have
% \[
%   \Ind_{H_{t}}^{W'_{2t}} \hsgn = \sum_{\sigma\in \Irr(S_{t})} (\sigma,\sigma)_{I}.
% \]

\trivial[h]{ % In McGovern's paper, the coherent continuation representation is
  % described as:
  % \[
  %   \sum_{t,s,a,b}\Ind_{W_t\times (W_s\ltimes W(A_1)^s)\times W_a\times W_b}^{W_{t+2s+a+b}} \sgn\otimes (\triv \otimes \sgn)\otimes \triv\otimes \triv
  % \]
  Now \eqref{eq:CC.C} is obtained by the following branching formula:
  \cite[p220 (6)]{Mc}
  \[
    I_n:= \Ind_{(W_s\ltimes W(A_1)^s)}^{W_{2s}}\triv\otimes \sgn = \sum \lambda\times \lambda
  \]
  where $\lambda$ running over all Young diagrams of size $s$. As McGovern
  claimed the proof of the above formula is similar to Barbasch's proof of
  \cite[Lemma~4.1]{BV.W}:
  \[
    \Ind_{W_n}^{S_{2n}} \triv = \sum \sigma \quad \text{where $\sigma$ has even
      rows only}.
  \]

  Sketch of the proof (use branching rule and dimension counting): Note that
  $\dim I_n = \frac{(2p)! 2^{2p}}{p! 2^{2p}} = (2p)!/p! =\sum_\lambda \dim \lambda\times \lambda$
  (For the last equality:
  $\dim \lambda\times \lambda = (2p)! (\dim \lambda)^2/(p!)^2$ where
  $\dim \lambda$ is the dimension of $S_n$ representation determined by
  $\lambda$; But $\sum (\dim \lambda)^2 = p!$). On the other hand,
  $H :=W_s\ltimes W(A_1)^s\cap W_s\times W_s = \Delta W_s \subset W_{2s}$.
  $\triv \otimes \sgn|_H = \sgn$ of $\Delta W_s$ Therefore,
  $\lambda\times \lambda$ appears in $I_n$ by Mackey formula. Now by dimension
  counting, we get the formula. }

\medskip

We now define various Weyl group representations case by case, which will be used to
state the formulas of coherent continuation representations.
% In the following $\sigma$ is running over all irreducible representations of
% $\sfS_{t}$.
\begin{itemize}
  \item Suppose $\star= B$, $p+q=2n+1$ is odd. Define
        \[
        \begin{split}
          \cC_{\mathrm b}^{n} & :=\bigoplus_{\substack{2t+c+d=n}} \Ind_{\sfH_{t} \times \sfW_{c}\times \sfW_{d}}^{\sfW_{n}}
          \hsgn\otimes 1\otimes 1, \\
          \cC_{\mathrm g}^{p,q} &:=\bigoplus_{\substack{0\leq p-(2t+a+2r)\leq 1\\0\leq q - (2t+a+2s)\leq 1}} \Ind_{\sfH_{t} \times \sfS_{a}\times \sfW_s\times \sfW_r}^{\sfW_{n}}
          \hsgn \otimes 1 \otimes \sgn \otimes \sgn. \\
          % &\cong \bigoplus_{\substack{0\leq p-(2t+c+d+2r)\leq 1\\0\leq q - (2t+c+d+2s)\leq 1}} \bigoplus_{\sigma} \Ind_{\sfW_{2t} \times \sfW_{c}\times \sfW_{d}\times \sfW_s\times \sfW_r}^{\sfW_{n}}
          % (\sigma,\sigma) \boxtimes 1 \boxtimes \brsgn \boxtimes \sgn \boxtimes \sgn \\
        \end{split}
        \]
        \trivial[h]{ Here is a point which could cause confusion: Although the
        real Weyl group is
        $\sfH_{t}\times \sfW_{a}\times \sfW_{s}\times \sfW_{r}$, the cross
        stabilizer is much smaller
        $=\sfH_{t}\times \sfS_{a}\times \sfW_{s}\times \sfW_{r}$! This is dual
        to the fact that, for the split Cartan and real root $e_{i}$,
        $\sgn\circ e_{i}\colon H\rightarrow \bR^{\times}\rightarrow \set{\pm 1}$
        is non-trivial! The good infinitesimal character takes half-integer
        values. So $s_{e_{i}}$ never cross stabilizing a regular character at
        these infinitesimal characters }
  \item Suppose $\star=C^{*}$. Define
        \[
        \begin{split}
          \cC_{\mathrm b}^{n} & :=
          \begin{cases}
            % \Res_{\sfW_{n}}^{\sfW'_{n}}
            \Ind_{\sfH_{t}}^{\sfW_{n}} \hsgn &
            \text{if $n=2t$ is even}, \\
            0 & \text{otherwise}.\\
          \end{cases}\\
          \cC_{\mathrm g}^{2p,2q} %& = \bigoplus_{p+q=m} \Cint{\rho}(\Sp(p,q)) \\
          &:=\bigoplus_{\substack{(t+s,t+r)=(p,q)}} \Ind_{\sfH_{t} \times \sfW_s\times \sfW_t}^{\sfW_{p+q}}
          \hsgn \otimes \sgn \otimes \sgn.\\
          % & =\bigoplus_{\substack{(t+s,t+r)=(p,q)}} \bigoplus_{\sigma\in \Irr(\sfS_{t})} \Ind_{\sfW_{2t}\times \sfW_s\times \sfW_r}^{\sfW_{p+q}}
          % (\sigma,\sigma)\otimes \sgn \otimes \sgn \\
        \end{split}
        \]
  \item Suppose $\star=C$. Define
        \[
        \begin{split}
          \cC_{\mathrm b}^{n} &
          :=\bigoplus_{\substack{2t+a=n}} %\Res_{\sfW_{n}}^{\sfW'_{n}} \left(
          \Ind_{\sfH_{t} \times \sfS_a}^{\sfW_{n}} \hsgn\otimes 1, %\right)
          \\
          \cC_{\mathrm g}^{n,n} &:= \bigoplus_{2t+a+c+d=n}\Ind_{\sfH_{t} \times \sfS_{a} \times \sfW_c\times \sfW_d}^{W_{n}} \hsgn \otimes
          \sgn \otimes 1 \otimes 1.\\
          % & =\bigoplus_{\substack{t+r+s+c+d=n}} \bigoplus_{\sigma } \Ind_{\sfW_{2t}\times \sfW_s\times \sfW_r\times \sfW_{c}\times \sfW_{d}}^{\sfW_{n_{\mathrm g}}}
          % (\sigma,\sigma)\otimes \sgn \otimes \bsgn \otimes 1\otimes 1 \\
        \end{split}
        \]
  \item Suppose $\star=\wtC$. Define
        \[
        \begin{split}
          \cC_{\mathrm b}^{n} &
          :=\bigoplus_{\substack{2t+c+d=n}} %\Res_{\sfW_{n}}^{\sfW'_{n}} \left(
          \Ind_{\sfH_{t} \times \sfW_c\times \sfW_{d}}^{\sfW_{n}} \hsgn\otimes 1 \otimes
          1, %\right)
          \\
          \cC_{\mathrm g}^{n,n} &:= \bigoplus_{2t+a+a'=n}\Ind_{\sfH_{t} \times \sfS_{a} \times \sfS_{a'}}^{\sfW_{n}} \hsgn \otimes
          \sgn \otimes 1. \\
          % & =\bigoplus_{\substack{t+r+s+c+d=n}} \bigoplus_{\sigma } \Ind_{\sfW_{2t}\times \sfW_s\times \sfW_r\times \sfW_{c}\times \sfW_{d}}^{\sfW_{n_{\mathrm g}}}
          % (\sigma,\sigma)\otimes \sgn \otimes \bsgn \otimes 1\otimes 1 \\
        \end{split}
        \]
  \item Suppose $\star=D$ and $p+q=2n$ is even. Define
        \[
        \begin{split}
          \cC_{\mathrm b}^{n} & := \bigoplus_{\substack{2t+a=n}}
          \Ind_{\sfH_{t}\times \sfS_{a}}^{\sfW_{n}} \hsgn\otimes 1,\\
          \cC_{\mathrm g}^{p,q} %& = \bigoplus_{p+q=m} \Cint{\rho}(\Sp(p,q)) \\
          & := \bigoplus_{\substack{2t+c+d+2r=p\\2t+c+d+2s=q}}
          % \Res_{\sfW_{m}}^{\sfW'_{m}}\left(
          \Ind_{\sfH_{t} \times \sfW_s\times \sfW_r\times \sfW'_{c}\times \sfW_{d} }^{\sfW_{n}} \hsgn \otimes \bsgn \otimes \bsgn \otimes 1\otimes
          1. %\right)
          \\
        \end{split}
        \]
  \item Suppose $\star=D^{*}$. Define
        \[
        \begin{split}
          \cC_{\mathrm b}^{n} & :=
          \begin{cases}
            \Ind_{\sfH_{t}}^{\sfW'_{n}} \hsgn &
            \text{if $n=2t$ is even}, \\
            0 & \text{otherwise}.\\
          \end{cases}\\
          \cC_{\mathrm g}^{n,n} %& = \bigoplus_{p+q=m} \Cint{\rho}(\Sp(p,q)) \\
          &:=\bigoplus_{\substack{2t+a=n}} \Ind_{\sfH_{t} \times \sfS_{a}}^{\sfW'_{n}}
          \hsgn \otimes \sgn. \\
        \end{split}
        \]
\end{itemize}

Denote by $n:=\rank_{\bC} \Gc$. Identify $\fhh^{*}$ with $\bC^{n}$, and let $Q$
be the root lattice in $\fhh^{*}$:
\[
  Q = \begin{cases}
    \bZ^{n} & \text{if  $\star = B$},\\
    % \set{(a_{1},a_{2},\cdots, a_{n})\in \bZ^{n}|\sum_{i=1}^{n}a_{i} \in 2\bZ}
    \Set{(a_{i})\in \bZ^{n}|\sum_{i=1}^{n}a_{i} \text{ is even}}
    & \text{if  $\star \in \set{C,\wtC,C^{*},D,D^{*}}$}.\\
  \end{cases}
\]
For $n_{\mathrm b}, n_{\mathrm g}\in \bN$ such that $n_{\mathrm b}+n_{\mathrm g}=n$, consider the lattice $\Lambda_{n_{\mathrm b},n_{\mathrm g}}=\lambda_{n_{\mathrm b},n_{\mathrm g}}+Q$, where
\[
  \lambda_{n_{\mathrm b},n_{\mathrm g}} =
  \begin{cases}
    (\underbrace{\half, \cdots, \half}_{n_{\mathrm b}\text{-terms}}, \underbrace{0, \cdots, 0}_{n_{\mathrm g}\text{-terms}})  & \text{when
    } \star\in \set{C,C^{*}, D,D^{*}}\\%\subset \fhh^{*}.
    (\underbrace{0, \cdots, 0}_{n_{\mathrm b}\text{-terms}}, \underbrace{\half, \cdots, \half}_{n_{\mathrm g}\text{-terms}})  & \text{when
    } \star\in \set{B,\wtC}. %\subset \fhh^{*}.
  \end{cases}
\]
Clearly the integral Weyl group
\[
  W_{\Lambda_{\nbb,\ngg}}:= \set{w| w\cdot \Lambda_{\nbb,\ngg} = \Lambda_{\nbb,\ngg}} = W_{\mathrm b}\times W_{\mathrm g},
\]
where $W_{\mathrm b}$ and $W_{\mathrm g}$ are defined in \eqref{eq:Wbg}.


We define
\[
  \Sign(G):= \begin{cases}
    (p,q) & \text{if } G = \SO(p,q),\\
    (n,n) & \text{if } G = \Sp(2n,\bR) \text{ or } \widetilde \Sp(2n,\bR),\\
    (2p,2q) & \text{if } G = \Sp(p,q), \\
    (n,n) & \text{if } G = \rO^{*}(2n). \\
  \end{cases}
\]

When $G=\widetilde \Sp(2n,\bR)$, namely $\star = \wtC$, we let $\Coh_{\Lambda_{n_{\mathrm b},n_{\mathrm g}}}(\CK(G))$ be the coherent continuation representation for the category of genuine representations of $G$ with generalized infinitesimal character $\lambda_{n_{\mathrm b},n_{\mathrm g}}$.

\begin{prop}\label{prop:cohBCD}
  The coherent continuation representation $\Coh_{\Lambda_{n_{\mathrm b},n_{\mathrm g}}}(\CK(G))$ is zero unless $(p_{\mathrm g},q_{\mathrm g}):= \Sign(G) - (n_{\mathrm b},n_{\mathrm b})\in \bZ^{\geq 0}\times \bZ^{\geq 0}$. When this is satisfied, $\Coh_{\Lambda_{n_{\mathrm b},n_{\mathrm g}}}(\CK(G))$ is isomorphic, as a $W_{\Lambda_{n_{\mathrm b},n_{\mathrm g}}}:= W_{\mathrm b}\times W_{\mathrm g}$-module, to the restriction to $W_{\Lambda_{n_{\mathrm b},n_{\mathrm g}}}$ of
  \[
    \begin{cases}
      \cC_{\mathrm b}^{n_{\mathrm b}}\boxtimes\cC_{\mathrm g}^{p_{\mathrm g},q_{\mathrm g}} & \text{if } \star \in \set{B,C,\wtC,C^{*},D},\\
      \Ind_{\sfW'_{n_{\mathrm b}}\times \sfW'_{n_{\mathrm g}}}^{\sfW_{n_{\mathrm b},n_{\mathrm g}}'}(\cC_{\mathrm b}^{n_{\mathrm b}}\boxtimes\cC_{\mathrm g}^{p_{\mathrm g},q_{\mathrm g}}) & \text{if } \star = D^{*}, \\
    \end{cases}
  \]
  where
  \[
    \sfW_{n_{\mathrm b},n_{\mathrm g}}' := \left(\sfW_{n_{\mathrm b}}\times \sfW_{n_{\mathrm g}}\right)\cap \sfW'_{n_{\mathrm g}+n_{\mathrm b}} \quad \text{when $\star = D^{*}$.}
    %W'' := \left(\sfW_{n_{\mathrm b}}\times \sfW_{n_{\mathrm g}}\right)\cap \sfW'_{n_{\mathrm g}+n_{\mathrm b}} \quad \text{when $\star = D^{*}$.}
  \]
\end{prop}

\begin{remark}
  When $\star \in \set{C^{*},D^{*}}$ and $n_{\mathrm b}$ is odd, we have in particular $\Coh_{\Lambda_{n_{\mathrm b},n_{\mathrm g}}}(\CK(G))=0$.
\end{remark}

\begin{proof}
%[Sketch of the proof]
  In view of \Cref{thm:cohHC}, we will need to enumerate the set of regular characters \cite[Section 2]{V4}. 
  When $G$ is linear, this may be done using \cite{AC}. When $G$ is the metaplectic group, this is contained in \cite{RT1,RT2}. All assertions will follow from \Cref{thm:cohHC} by direct computations, case by case. See also
  \cite{Mc}*{Applications}.
  % We give a sketch of the argument.
\end{proof}


\begin{defn}
  Let $\PBPs(\ckcOb;\tau_{\mathrm b})$ be the set of all pairs
  $\uptau = (\imath,\cP)\times(\jmath,\cQ)$ where $(\imath,\cP)$ and
  $(\jmath,\cQ)$ are painted partitions such that
  \begin{itemize}
    \item $(\imath,\jmath) = \tau_{\mathrm b}$ (see \eqref{eq:taub});
    \item the image of $\cP$ is contained in
          \[
          \begin{cases}
            \set{\bullet, c,d}  & \text{if } \star\in \set{B,\wtC}, \\
            \set{\bullet, d}  & \text{if } \star\in \set{C,D},\\
            \set{\bullet}  & \text{if } \star\in \set{C^{*},D^{*}};\\
          \end{cases}
          \]
    \item the image of $\cQ$ is contained in

          \[
          \begin{cases}
            \set{\bullet, c}  & \text{if } \star\in \set{C,D},\\
            \set{\bullet}  & \text{if } \star\in \set{B,\wtC, C^{*},D^{*}}.\\
          \end{cases}
          \]
  \end{itemize}
\end{defn}

We define the partition $\ckcOpb$ by $\ckcOb = 2\ckcOpb$. To be more precise, $\bfrr_{i}(\ckcO'_{\mathrm b}):= \bfrr_{2i}(\ckcO_{\mathrm b})$, for all $i\in \bN^{+}$. Let $\cO'_{\mathrm b}:= (\ckcO'_{\mathrm b})^{t}$.
  %\item $\cO_{\mathrm b}:= \cO'_{\mathrm b}\cupcol \cO'_{\mathrm b}$.
 %$\ckcOpb$ is a partition of $\half\abs{\ckcOb}$.

We also define
\begin{equation}\label{Gpb}
  G'_{\mathrm b} := \begin{cases}
    \GL(n_{\mathrm b},\bR) & \text{when } \star \in \set{B,C,\wtC,D}, \\
    \GL(\half n_{\mathrm b},\bH) & \text{when } \star \in \set{C^{*},D^{*}}.\\
  \end{cases}
\end{equation}

\begin{prop}\label{prop:BP.PP} In all cases, we have
\[
    \begin{split}
      \abs{\PBP_{\star}(\ckcO_{\mathrm b};\tau_{\mathrm b})} = \abs{\PP_{G'_{\mathrm b}}(\ckcO'_{\mathrm b})} = \abs{\Unip_{\ckcO'_{\mathrm b}}(G'_{\mathrm b})},
    \end{split}
  \]
where 
\[\PP_{G'_{\mathrm b}}(\ckcO'_{\mathrm b}):= \begin{cases}
    \PP_{A}(\ckcO'_{\mathrm b}) & \text{when } \star \in \set{B,C,\wtC,D}, \\
    \PP_{A^{\bH}}(\ckcO'_{\mathrm b}) & \text{when } \star \in \set{C^{*},D^{*}}.\\
  \end{cases}
  \]
%is the set of painted partitions of type $A$ or $A^{\bH}$ attached to $\ckcO'_{\mathrm b}$.
\end{prop}

\begin{proof} Suppose $\star \in \Set{C^{*},D^{*}}$. Then
\[
    \begin{split}
      \abs{\PBP_{\star}(\ckcO_{\mathrm b};\tau_{\mathrm b})} = \abs{\PP_{A^{\bH}}(\ckcO'_{\mathrm b})} = \abs{\Unip_{\ckcO'_{\mathrm b}}(G'_{\mathrm b})} = 1.
    \end{split}
\]
  Suppose $\star \in \Set{B,C,\wtC,D}$. Write $\tau_{\mathrm b} = (\tau_{L,b},\tau_{R,b})$ and
  $\tau'_{\mathrm b} = \cO'_{\mathrm b}$. It is easy to see that we have a bijection:
  \[
    \begin{array}{ccc}
      \PBP_{\star}(\ckcO_{\mathrm b};\tau _{\mathrm b}) &  \longrightarrow & \PP_{A}(\ckcO'_{\mathrm b})\\
      (\tau_{L,b},\cP)\times (\tau_{R,b},\cQ)& \mapsto & (\cOpb,\cP'),
    \end{array}
  \]
  where $\cP'$ is defined by the condition that
  \[
    \cP(\bfcc_{j}(\tau_{L,b}),j)=d \Longleftrightarrow \cP'(\bfcc_{j}(\cOpb),j)=d, \quad \forall j=1,2,\cdots, \bfrr_{1}(\cOpb).
  \]
  \trivial[h]{
    % Let $\tau' = \ckcO'^{t}_{\mathrm b}$ and $\tau_{\mathrm b}=(\tau_{L,b}, \tau_{R,b})$. Here
    % $\tau_{L,b}, \tau_{R,b}$.
    Now the claim follows for the fact that the bottom rows in $\uptau_{L}$ can
    be filled by $\bullet/c$ or $d$ and
    \[
      \bfcc_{i}(\tau_{L,b}) = \bfcc_{j}(\tau_{L,b}) \Leftrightarrow \bfcc_{i}(\cOpb) = \bfcc_{j}(\cOpb) \quad \forall i,j\in \bN^{+}.
    \]
  }

\end{proof}

We introduce some additional notation. For each bipartition $\tau$, let
\[
  \PBP_{\star}(\tau) := \Set{ \uptau|\uptau \text{ is a painted bipartition and
    } \star_{\uptau}=\star, (\imath_{\uptau},\jmath_{\uptau}) = \tau}
  % \uptau=(\imath, \cP)\times (\jmath,\cP)\times \alpha|}
\]
and
\[
  \tPBP_{\star}(\ckcO_{\mathrm g}) := %\bigsqcup_{\tau\in\LC(\ckcOg)}\PBP_{\star}(\tau).
  \bigsqcup_{\wp \subseteq \CPP(\ckcO_{\mathrm g})}\PBP_{\star}(\wttau_{\wp}),
\]
where $\wttau_{\wp} := (\imath_{\wp},\jmath_{\wp})$ (see \eqref{eq:ttauwp}). Similarly, define
\[
  \tPBP_{G_{\mathrm g}}(\ast):= \Set{\uptau\in \tPBP_{\star}(\ast )|\Sign(\uptau)= \Sign(G_{\mathrm g})}, \quad  \quad
  \ast  = \ckcO_{\mathrm g} \text{ or } \tau.
\]

Recall from \Cref{cor:bound}, we have the inequality
\[
\begin{split}
\sharp (\Unip_{\ckcO}(G))&\leq \sum_{\sigma\in \LC_{\ckcO}} [\sigma: \Coh_{\Lambda_{n_{\mathrm b},n_{\mathrm g}}}(\CK(G))]\\
  & =\sum_{\wp\in \barA(\ckcO)} [\tau_{\mathrm b}\otimes \tau_{\wp}: \Coh_{\Lambda_{n_{\mathrm b},n_{\mathrm g}}}(\CK(G))] \quad (\textrm{by Lemma \ref{lem:Lcell}}).
\end{split}
\]


\begin{prop}\label{prop:countBCD}
  In all the cases, we have
  % \[
  %   [\tau_{\mathrm b}: \cC_{\mathrm b}] = \PBP_{\star,b}(\ckcO_{\mathrm b}) = \PBP_{G'}(\ckcO'_{n_{\mathrm b}}) = \Unip_{\ckcO'_{\mathrm b}}(G'_{n_{\mathrm b}})
  % \]
  % and
  % \[
  %   \sum_{\tau\in \LC_{\ckcO_{\mathrm g}}} [\tau:\cC_{\mathrm g}] = \PBP_{\mathrm g}(\ckcO_{\mathrm g}).
  % \]
  \[
  \sum_{\wp\in \barA(\ckcO)} [\tau_{\mathrm b}\otimes \tau_{\wp}: \Coh_{\Lambda_{n_{\mathrm b},n_{\mathrm g}}}(\CK(G))]
      = \sharp (\PBP_{\star}(\ckcO_{\mathrm b};\tau _{\mathrm b}))\cdot \sharp(\tPBP_{G_{\mathrm g}}(\ckcO_{\mathrm g})).
  \]
 In particular, when $\ckcO = \ckcO_{\mathrm g}$, namely $\ckcO $ has good parity, we have
  \[
  \sharp (\Unip_{\ckcO}(G))\leq
     \sharp ({\tPBP_{\mathrm g}(\ckcO)}).
  \]
 \end{prop}

 \begin{remark} It will be shown in \Cref{prop:PBP} that $\sharp(\tPBP_{\mathrm g}(\ckcO)) = 2^{\sharp(\CPPs(\ckcO))}\cdot \sharp(\PBP_{\mathrm g}(\ckcO))$. Proposition \ref{prop:countBCD} will thus imply Theorem \ref{countup} in the introductory section.
 \end{remark}

\begin{proof} We use the descriptions of $\tau_{\mathrm b}$ and $\tau_{\wp}$ (in \eqref{eq:taub} and \eqref{eq:tauwp}) to compute the multiplicities, using branching rules of Weyl groups in \eqref{eq:CC.C} and \eqref{eq:indSW} as well as \Cref{lem:WLcell}. We skip the details when $\star \in \set{B,\wtC, C,D,C^{*}}$, and present the computation for $\star = D^{*}$, which is the most complicated case.

  Recall that $(W_{\mathrm b},W_{\mathrm g}) = (\sfW'_{\nbb},\sfW'_{\ngg})$, where $n_{\mathrm b}=\half \abs{\ckcO_{\mathrm b}}$ and $n_{\mathrm g}=\half\abs{\ckcO_{\mathrm g}}$.

  First suppose $\ngg = 0$. Then $\tau_{\mathrm b} = (\cOpb,\cOpb)_{I}$ and
  \[
    \begin{split}
      [\tau_{\mathrm b}:\cC_{\mathrm b}^{\nbb}] = &
      [\tau_{\mathrm b}: \Ind_{\sfH_{\frac{\nbb}{2}}}^{\sfW'_{\nbb}}\tsgn]\\
      = & [(\cOpb,\cOpb)_{I}: \bigoplus_{\sigma}(\sigma,\sigma)_{I}]\\
      = & 1.
    \end{split}
  \]

  Now suppose $\ngg>0$. Let
  \[
    \wttau_{\mathrm b} := \Ind_{\sfW'_{\nbb}}^{\sfW_{\nbb}} \tau_{\mathrm b} = (\cOpb,\cOpb) \AND \wttau_{\wp}: = (\imath_{\wp},\jmath_{\wp}) \quad \forall \wp \subseteq \CPP(\ckcOg).
  \]
  Note that $\imath_{\wp}\neq \jmath_{\wp}$ since
  $\bfcc_{1}(\imath_{\wp})> \bfcc_{1}(\jmath_{\wp})$. It is then easy to see that
  \begin{equation*}%\label{eq:W''}
    \Ind_{\sfW'_{\nbb}\times \sfW'_{\ngg}}^{\sfW_{n_{\mathrm b},n_{\mathrm g}}'} \tau_{\mathrm b}\boxtimes \tau_{\wp}
    = (\wttau_{\mathrm b}\otimes \wttau_{\wp})|_{\sfW_{n_{\mathrm b},n_{\mathrm g}}'}.
  \end{equation*}

  \trivial[h]{ When $\nbb=0$, $W'' = W_{\mathrm g} = \sfW'_{\ngg}$ and so
    $\wttau_{\wp}|_{\sfW'_{\ngg}} = \tau_{\wp}$.

    Now we assume $\nbb\neq 0$ and $\ngg\neq 0$ (the general case). This
    follows from the following points
    \begin{itemize}
      \item the dimension of the two sides are equal ($W_{\mathrm b}\times W_{\mathrm g}$ has
            index $2$ in $W''$).
      \item
            \[
            \begin{split}
              &[\Ind_{W_{\mathrm b}\times W_{\mathrm g}}^{W''}\tau_{\mathrm b}\boxtimes \tau_{\wp}:(\wttau_{\mathrm b}\otimes \wttau_{\wp})|_{W''}] \\
              =& [\Ind_{\sfW'_{n_{\mathrm b}}\times \sfW'_{n_{\mathrm g}}}^{\sfW_{n_{\mathrm b}}\times \sfW_{n_{\mathrm g}}}\tau_{\mathrm b}\boxtimes \tau_{\wp}:\wttau_{\mathrm b}\otimes \wttau_{\wp}]\\
              =& [\wttau_{\mathrm b}\boxtimes \wttau_{\wp} \oplus \wttau_{\mathrm b}\boxtimes (\wttau_{\wp}\otimes \brsgn):\wttau_{\mathrm b}\otimes \wttau_{\wp}] =1\\
            \end{split}
            \]
            where $\wttau_{\wp}\otimes \brsgn$ has the bipartition obtained by
            switching the left and right side of $\wttau_{\wp}$.
      \item the LHS is irreducible, by
            \[
            \begin{split}
              & [\Ind_{W_{\mathrm b}\times W_{\mathrm g}}^{W''}\tau_{\mathrm b}\boxtimes \tau_{\wp}:
              \Ind_{W_{\mathrm b}\times W_{\mathrm g}}^{W''}\tau_{\mathrm b}\boxtimes \tau_{\wp}]_{W''}\\
              =&  [\tau_{\mathrm b}\boxtimes \tau_{\wp} : (\Ind_{W_{\mathrm b}\times W_{\mathrm g}}^{W''}\tau_{\mathrm b}\boxtimes \tau_{\wp})|_{W_{\mathrm b}\times W_{\mathrm g}}]\\
              =& [\tau_{\mathrm b}\boxtimes \tau_{\wp} : \tau_{\mathrm b}\boxtimes \tau_{\wp} + (\cOpb,\cOpb)_{II} \boxtimes \tau_{\wp} ] = 1
            \end{split}
            \]
    \end{itemize}
  }

  For ease of notation, write $\sfW'':=\sfW_{n_{\mathrm b},n_{\mathrm g}}'$. We then have
  \[
    \begin{split}
      & [\tau_{\mathrm b}\boxtimes \tau_{\wp} :
      \Ind_{\sfW'_{n_{\mathrm b}}\times \sfW'_{n_{\mathrm g}}}^{\sfW''} \cC_{\mathrm b}^{\nbb} \boxtimes \cC_{\mathrm g}^{p_{\mathrm g},q_{\mathrm g}}]_{\sfW'_{n_{\mathrm b}}\times \sfW'_{n_{\mathrm g}}}\\
      = & [\Ind_{\sfW'_{n_{\mathrm b}}\times \sfW'_{n_{\mathrm g}}}^{\sfW''} \tau_{\mathrm b}\boxtimes \tau_{\wp} :
      \Ind_{\sfW'_{n_{\mathrm b}}\times \sfW'_{n_{\mathrm g}}}^{\sfW''} \cC_{\mathrm b}^{\nbb} \boxtimes \cC_{\mathrm g}^{p_{\mathrm g},q_{\mathrm g}}]_{\sfW''}\\
      = & [(\wttau_{\mathrm b}\boxtimes \wttau_{\wp})|_{\sfW''}:
      \Ind_{\sfW'_{n_{\mathrm b}}\times \sfW'_{n_{\mathrm g}}}^{\sfW''} \cC_{\mathrm b}^{\nbb} \boxtimes \cC_{\mathrm g}^{p_{\mathrm g},q_{\mathrm g}}]_{\sfW''}\\
      = & [\wttau_{\mathrm b}\boxtimes \wttau_{\wp}:
      \Ind_{\sfW'_{n_{\mathrm b}}\times \sfW'_{n_{\mathrm g}}}^{\sfW_{n_{\mathrm b}}\times \sfW_{n_{\mathrm g}}} \cC_{\mathrm b}^{\nbb} \boxtimes \cC_{\mathrm g}^{p_{\mathrm g},q_{\mathrm g}}]_{\sfW_{n_{\mathrm b}}\times \sfW_{n_{\mathrm g}}}\\
       = & [\wttau_{\mathrm b}:\Ind_{\sfW'_{n_{\mathrm b}}}^{\sfW_{n_{\mathrm b}}} \cC_{\mathrm b}^{\nbb}]_{\sfW_{n_{\mathrm b}}}\cdot
           [\wttau_{\wp}:\Ind_{\sfW'_{n_{\mathrm g}}}^{\sfW_{n_{\mathrm g}}} \cC_{\mathrm g}^{p_{\mathrm g},q_{\mathrm g}}]_{\sfW_{n_{\mathrm g}}}\\
      =& \sharp(\PBP_{\star}(\ckcO_{\mathrm b}; \tau _{\mathrm b})))\cdot \sharp(\PBP_{\star}(\wttau_{\wp})).
    \end{split}
  \]
  The last equality follows from the explicit description of $\cC_{\mathrm b}^{\nbb}$ and $\cC_{\mathrm g}^{p_{\mathrm g},q_{\mathrm g}}$ in Section \ref{subsec:explicitCoh}, and
  the branching rules of $\sfW_{n}$ in \eqref{eq:CC.C} and \eqref{eq:indSW}, where the factor $\sfH_{t}$ amounts to painting ``$\bullet$'' on $\imath_{\wp}$ and $\jmath_{\wp}$, the factor $\sfS_{a}$ amounts to painting ``$s$'' on $\imath_{\wp}$ and painting ``$r$'' on $\jmath_{\wp}$, each ``permissible" way of painting (see \Cref{def:pbp1})  
  contributing $1$ to the multiplicity $[\wttau_{\wp}:\Ind_{\sfW'_{n_{\mathrm g}}}^{\sfW_{n_{\mathrm g}}} \cC_{\mathrm g}^{p_{\mathrm g},q_{\mathrm g}}]_{\sfW_{n_{\mathrm g}}}$.
\end{proof}

  % Suppose $n_{\mathrm b}=0$ then
  % \[
  %   \begin{split}
  %     [\tau_{\wp} : \cC_{\mathrm g}]_{\sfW'_{n_{\mathrm g}}} = &
  %     [\wttau_{\wp}|_{W_{\mathrm g}}:\sum_{2t+a=n_{\mathrm g}} \Ind_{\sfH_{t}\times \sfS_{a}}^{\sfW'_{n_{\mathrm g}}}\tsgn\otimes \sgn]_{\sfW'_{n_{\mathrm g}}}\\
  %     = & [\wttau_{\wp}: \sum_{2t+a=n_{\mathrm g}} \Ind_{\sfH_{t}\times \sfS_{a}}^{\sfW'_{n_{\mathrm g}}}\tsgn\otimes \sgn]_{\sfW_{n_{\mathrm g}}}\\
  %     = & \PBP_{\star}(\ttau_\wp)
  %   \end{split}
  % \]


  \trivial[h]{

    Suppose $\star =C^{*}$.

    For the bad parity, $n_{\mathrm b}=2n'_{\mathrm b}$ must be even.
    \[
      \begin{split}
        & [\tau_{\mathrm b}\boxtimes \tau_{\wp} :
        \cC_{\mathrm b}^{\nbb} \boxtimes \cC_{\mathrm g}^{\ngg,\ngg}]_{\sfW'_{n_{\mathrm b}}\times \sfW_{n_{\mathrm g}}}\\
        = & [\Ind_{\sfW'_{n_{\mathrm b}}\times \sfW_{n_{\mathrm g}}}^{\sfW_{n_{\mathrm b}}\times \sfW_{n_{\mathrm g}}} \tau_{\mathrm b}\boxtimes \tau_{\wp} :
        \cC_{\mathrm b}^{\nbb} \boxtimes \cC_{\mathrm g}^{\ngg,\ngg}]_{\sfW_{n_{\mathrm b}}\times \sfW_{n_{\mathrm g}}}\\
        = & [\wttau_{\mathrm b}\boxtimes \tau_{\wp}:
        \cC_{\mathrm b}^{\nbb} \boxtimes \cC_{\mathrm g}^{\ngg,\ngg}]_{\sfW_{n_{\mathrm b}}\times \sfW_{n_{\mathrm g}}}\\
        =& \# \PBP_{\star}(\ckcO_{\mathrm b})\cdot \# \PBP_{\star}(\tau_{\wp})
      \end{split}
    \]

    The alternative approach using restriction. For the bad parity,
    $n_{\mathrm b}=2n'_{\mathrm b}$ must be even.
    \[
      \begin{split}
        \cC_{\mathrm b}^{n_{\mathrm b}} &=
        \Res_{\sfW_{n_{\mathrm b}}}^{\sfW'_{n_{\mathrm b}}} \Ind_{\sfH_{n'_{\mathrm b}}}^{\sfW_{n_{\mathrm b}}} \hsgn \\
        &= \bigoplus_{\sigma\in \Irr(\sfS_{n'_{\mathrm b}})} \left((\sigma,\sigma)_{I} \oplus (\sigma,\sigma)_{II}\right).
      \end{split}
    \]
    For the good parity,
    \[
      \begin{split}
        \cC_{\mathrm g}^{2p,2q} %& = \bigoplus_{p+q=m} \Cint{\rho}(\Sp(p,q)) \\
        & =\bigoplus_{\substack{(t+s,t+r)=(p,q)}} \bigoplus_{\sigma} \Ind_{\sfW_{2t}\times \sfW_s\times \sfW_r}^{\sfW_{p+q}}
        (\sigma,\sigma)\otimes \sgn \otimes \sgn \\
      \end{split}
    \]
    Now the branching rule implies the irreducible components of
    $\cC_{\mathrm g}^{2p,2q}$ are given by the dot-diagram attaching two columns on the
    right, which we mark them by $s$ and $r$ respectively. }


  % Suppose $\star=C$
  % \[
  %   \begin{split}
  %     & [\tau_{\mathrm b}\boxtimes \tau_{\wp} :
  %     \cC_{\mathrm b}^{\nbb} \boxtimes \cC_{\mathrm g}^{\ngg,\ngg}]_{\sfW'_{n_{\mathrm b}}\times \sfW_{n_{\mathrm g}}}\\
  %     = & [\Ind_{\sfW'_{n_{\mathrm b}}\times \sfW_{n_{\mathrm g}}}^{\sfW_{n_{\mathrm b}}\times \sfW_{n_{\mathrm g}}} \tau_{\mathrm b}\boxtimes \tau_{\wp} :
  %     \cC_{\mathrm b} \boxtimes \cC_{\mathrm g}]_{\sfW_{n_{\mathrm b}}\times \sfW_{n_{\mathrm g}}}\\
  %     = & [\wttau_{\mathrm b}\boxtimes \tau_{\wp}:
  %     \cC_{\mathrm b} \boxtimes \cC_{\mathrm g}]_{\sfW_{n_{\mathrm b}}\times \sfW_{n_{\mathrm g}}}\\
  %     =& \# \PBP_{\star}(\ckcO_{\mathrm b})\cdot \# \PBP_{\star}(\ckcO_{\mathrm g};\wp)
  %   \end{split}
  % \]
%
  % For the bad parity, $n_{\mathrm b}=2n'_{\mathrm b}$ must be even.
  % \[
  %   \begin{split}
  %     \cC_b^{n_{\mathrm b}} &= \Res_{\sfW_{n_{\mathrm b}}}^{\sfW'_{n_{\mathrm b}}} \left( \bigoplus_{2t+a=n_{\mathrm b}}\Ind_{\sfH_{t}\times \sfS_{a}}^{\sfW_{n_{\mathrm b}}} \hsgn \otimes 1
  %     \right)\\
  %     &= \bigoplus_{2t+c+d=n_{\mathrm b}}\bigoplus_{\sigma\in \Irr(\sfS_{t})} \left((\sigma,\sigma)_{I} \oplus (\sigma,\sigma)_{II}\right) \times ([d,],[c,]).
  %   \end{split}
  % \]
  % Note that $\tau_{\mathrm b}=(\cO'_{\mathrm b},\cO'_{\mathrm b})_{I}$. Therefore we only need to
  % consider the case when $c=d$ in the above formula.
  % \[
  %   \begin{split}
  %     [\tau_{\mathrm b}: \cC_{\mathrm b}^{n_{\mathrm b}}] & = [\cO'_{\mathrm b}:\bigoplus_{\substack{t+d = n'_{\mathrm b}\\ \sigma}} \sigma \times 1].
  %   \end{split}
  % \]
  % By the counting of unipotent representation of $\GL(n'_{\mathrm b},\bR)$. We see
  % that $\PBP_{\star,b}(\ckcO_{\mathrm b})$ is identified with
  % $\PBP_{A^{\bR}}(\ckcO'_{\mathrm b})$ by send $(\uptau_{L},\uptau_{R})$ to the
  % painted partition $\uptau'$ such that
  % \[
  %   \uptau_{L}(i,j)=d \Leftrightarrow \uptau'(i,j)=d.
  % \]

  % For the good parity, this is clear by the branching rules.


  \trivial[h]{


    Suppose $\star = D$. Suppose that $n_{\mathrm g}\neq 0$. Since
    $\bfcc_{1}(\imath_{\wp})>\bfcc_{1}(\jmath_{\wp})$, we have
    $\wttau^{s}_{\wp}:=\wttau_{\wp}\otimes \brsgn\ncong\wttau_{\wp}$.
    \[
      \begin{split}
        & [\tau_{\mathrm b}\boxtimes \tau_{\wp} :
        \cC_{\mathrm b} \boxtimes \cC_{\mathrm g}]_{\sfW'_{n_{\mathrm b}}\times \sfW'_{n_{\mathrm g}}}\\
        = & [\Ind_{\sfW'_{n_{\mathrm b}}\times \sfW'_{n_{\mathrm g}}}^{\sfW_{n_{\mathrm b}}\times \sfW_{n_{\mathrm g}}} \tau_{\mathrm b}\boxtimes \tau_{\wp} :
        \cC_{\mathrm b} \boxtimes \cC_{\mathrm g}]_{\sfW_{n_{\mathrm b}}\times \sfW_{n_{\mathrm g}}}\\
        = & [\wttau_{\mathrm b}\boxtimes \wttau_{\wp}\oplus \wttau_{\mathrm b}\boxtimes \wttau_{\wp}^{s}:
        \cC_{\mathrm b} \boxtimes \cC_{\mathrm g}]_{W''}\\
        = & [\wttau_{\mathrm b}\boxtimes \wttau_{\wp}:
        \cC_{\mathrm b} \boxtimes \cC_{\mathrm g}]_{\sfW_{n_{\mathrm g}}\times \sfW_{n_{\mathrm b}}}\\
        =& \# \PBP_{\star}(\ckcO_{\mathrm b})\cdot \# \PBP_{\star}(\ckcO_{\mathrm g};\wp)
      \end{split}
    \]
    The terms involving $\wttau_{\sP}^{s}$ vanish since every irreducible
    component $(\sigma_{L},\sigma_{R})$ in $\cC_{\mathrm g}$ satisfies
    $\sigma_{L}\supseteq \sigma_{R}$ but
    $\bfcc_{1}(\imath_{\wp})>\bfcc_{1}(\jmath_{\wp})$.

    Suppose that $n_{\mathrm g} = 0$.
    \[
      \begin{split}
        & [\tau_{\mathrm b} : \cC_{\mathrm b}^{n_{\mathrm b}}]_{\sfW'_{n_{\mathrm b}}}\\
        =& [\Ind_{\sfW'_{n_{\mathrm b}}}^{\sfW_{n_{\mathrm b}}} \tau_{\mathrm b} : \bigoplus_{\substack{2t+a=n}}
        \Ind_{\sfH_{t}\times \sfS_{a}}^{\sfW_{n}} \hsgn\otimes 1] \\
        =& [ \wttau_{\mathrm b} : \bigoplus_{\substack{2t+a=n}}
        \Ind_{\sfH_{t}\times \sfS_{a}}^{\sfW_{n}} \hsgn\otimes 1] \\
      \end{split}
    \]
    In any cases, the counting formula holds. There is place to confuse: Why
    there shouldn't be double the size of special unipotent representations?

    In fact, $\AC_{\bC}(\pi)$ can only be the fixed type, say $\cO_{I}$! Note
    that we fixed an infinitesimal character which has half-integral values.
    This choice implicitly force us to fix real Siegel parabolic when we do
    induction from $\GL$! The non-trivial outer automorphism, say $c$, will
    permute the infinitesimal character to the another one and we then will have
    $\AC_{\bC}({}^{c}\pi)i = \cO_{II}$.


    Using Barbasch's formula of wavefront, we see that the induction
    $\pi_{I}:=\Ind_{\GL}^{\SO}\pi'$ must be irreducible, where $\pi'$ is a
    unipotent representation of $\GL$. This will also implies
    $\Ind_{\GL}^{\rO}\pi'$ is irreducible and restricted to two $\SO$-modules,
    $\pi_{I}$ and $\pi_{II}$.

  }


  \trivial[h]{ Suppose $\star = \wtC$. Then
    \[
      \begin{split}
        & [\tau_{\mathrm b}\boxtimes \tau_{\wp} :
        \cC_{\mathrm b} \boxtimes \cC_{\mathrm g}]_{\sfW_{n_{\mathrm b}}\times \sfW'_{n_{\mathrm g}}}\\
        = & [ \tau_{\mathrm b}\boxtimes \Ind_{\sfW'_{n_{\mathrm g}}}^{\sfW_{n_{\mathrm g}}}\tau_{\wp} : \cC_{\mathrm b} \boxtimes \cC_{\mathrm g}]_{\sfW_{n_{\mathrm b}}\times \sfW_{n_{\mathrm g}}}\\
        = &\sum_{\sP\in \tA(\ckcO)} [\tau_{\mathrm b}\boxtimes \wttau_{\wp}:
        \cC_{\mathrm b} \boxtimes \cC_{\mathrm g}]_{\sfW_{n_{\mathrm b}}\times \sfW_{n_{\mathrm g}}}\\
        =& \# \PBP_{\star}(\ckcO_{\mathrm b})\cdot \# \PBP_{\star}(\ckcO_{\mathrm g})
      \end{split}
    \]

    Suppose $\star = B$. Then
    \[
      \begin{split}
        & \sum_{\sP\in \tA'(\ckcO)}[\tau_{\mathrm b}\boxtimes \tau_{\wp} :
        \cC_{\mathrm b} \boxtimes \cC_{\mathrm g}]_{\sfW_{n_{\mathrm b}}\times \sfW'_{n_{\mathrm g}}}\\
        =& \# \PBP_{\star}(\ckcO_{\mathrm b})\cdot \# \PBP_{\star}(\ckcO_{\mathrm g};\wp)
      \end{split}
    \]
  }



\subsection{Reduction to the good parity case}

In this section, $G$ is a classical Lie group of type $\star \in \set{B,\wtC, C,D,C^{*}, D^{*}}$.

\begin{lem}\label{lem:Unip.BP} We are in the setting of \Cref{prop:BP.PP}. Suppose that $n_{\mathrm g}=0$.
\begin{enumerate}
\item Then we have a bijection
  \[
    \begin{array}{rccc}
      \fI_{\mathrm b}: &\Unip_{\ckcO'_{\mathrm b}}(G'_{\mathrm b}) & \longrightarrow & \Unip_{\ckcO}(G) \\
      &\pi' & \mapsto & \pi:=\Ind_{P}^{G}\pi'
    \end{array}
  \]
  where $P$ is the standard parabolic subgroup of $G$ whose Levi component is
  isomorphic to $G'_{\mathrm b}$.
 \item Denote by $\sO'$ the real nilpotent orbit in $G'_{\mathrm b}$ such that $\sO'_{\bC}=\cO$, and let $\sO$ be the real induction of $\sO'$ to $G$.
  Then the wavefront cycle of $\pi $ is
  \[
    \WF(\pi) = [\sO].
  \]
 \end{enumerate}
\end{lem}
\begin{proof} First note that $\WF(\pi')=[\sO']$.
  It follows from a result of Barbasch \cite[Corollary 5.0.10]{B.Orbit} that the
  wavefront cycle of $\pi:=\Ind_{P}^{G}\pi'$ is $[\sO]$, with multiplicity one.
  We claim that $\pi$ is irreducible. If not, $\pi$ must contain an irreducible subquotient with infinitesimal
  character $\lambda_{\ckcO}$ and GK-dimension $<\half\dim_{\bC}\cO$, by \cite[Korollar 3.6]{BK}. This will
  contradict to the assertion of \Cref{lem:Lcell} on the associated variety of the maximal primitive ideal $\cI_{\ckcO}$.
  \trivial[h]{
    First, it is not obvious to me that the wavefront of $\Ind_{P}^{G}\pi$
    must be contained in $\Ind\WF(\pi)$. But it is clear that the leading term
    must be $\sum_{\sO\text{ open in } \WF(\pi)}\Ind\sO$. So the boundaries has
    less GK-dimension.

    Suppose $\pi_{0}$ is the sub-quotient with less GK-dimension.
    On the other hand, the maximal primitive ideal $\cI_{\ckcO}$  with infinitesimal
    character must contain $\Ann\pi_{0}$. In other words,
    $\AV_{\bC}(\pi_{0})\supseteq \bcO$ which implies GK-dimension of
    $\pi_{0}\geq \half\dim_{\bC}\cO$, a contradiction.
    % Let $\mu = \lamck$.
    % Note that every $W_{[\mu]}$-module in $\Grt_{\mu}(G)$ is in $\Ind_{W_{\mu}}^{W_{[\mu]}}$
  }
  Note that representations in $\Unip_{\ckcO'_{\mathrm b}}(G'_{\mathrm b})$ have distinct
  cuspidal data/Langlands parameters. This implies that $\Ind_{P}^{G}\pi'$ have distinct cuspidal data/Langlands parameters as $\pi'$ varies.
    Therefore, $\fI_{\mathrm b}$ is injective. The bijection follows from the counting
  inequalities below:
  \[
    \abs{\PP_{\star'}(\ckcO'_{\mathrm b})}=\abs{\Unip_{\ckcO'_{\mathrm b}}(G'_{\mathrm b})}\leq \abs{\Unip_{\ckcO}(G)}
    \leq \abs{\PBP_{\star}(\ckcO_{\mathrm b}; \tau_{\mathrm b})} = \abs{\PP_{\star'}(\ckcO'_{\mathrm b})}.
  \]
  Here $\star' \in \set{A,A^{\bH}}$ depending on $G$. The last equality holds because of \Cref{cor:bound} and \Cref{prop:cohBCD}.
\end{proof}


We now set
\[
  (G_{\mathrm b},G_{\mathrm g}) =
  \begin{cases}
    (\SO(n_{\mathrm b},n_{\mathrm b}+1),\SO(p-n_{\mathrm b},q-n_{\mathrm b})) & \text{when } \star = B, \\
    (\Sp(2n_{\mathrm b},\bR),\Sp(2n_{\mathrm g},\bR)) & \text{when } \star = C, \\
    (\Sp(n_{\mathrm b},n_{\mathrm b}), \Sp(p-\frac{n_{\mathrm b}}{2},q-\frac{n_{\mathrm b}}{2})) & \text{when } \star = C^{*}, \\
    (\Mp(2n_{\mathrm b},\bR),\Mp(2n_{\mathrm g},\bR)) & \text{when } \star = \wtC, \\
    (\rO^{*}(n_{\mathrm b}), \rO^{*}(n_{\mathrm g})) & \text{when } \star = D^{*}, \\
    (\SO(n_{\mathrm b},n_{\mathrm b}), \SO(p-n_{\mathrm b},q-n_{\mathrm b}) )& \text{when } \star = D. \\
  \end{cases}
\]
Note that the group $G'_{\mathrm b}$ (defined in \eqref{Gpb}) is the Levi component of a Siegel parabolic subgroup $P_{\mathrm b}$ of $G_{\mathrm b}$.

Let $P$ be the standard parabolic subgroup of $G$ whose Levi component is isomorphic to $G'_{\mathrm b}\times G_{\mathrm g}$.
\begin{thm}\label{thm:red}
  There is a bijection
  \begin{equation*}%\label{eq:IND}
      \begin{array}{rccc}
    \fI\colon &   \Unip_{\ckcO'_{\mathrm b}}(G'_{\mathrm b})\times \Unip_{\ckcO_{\mathrm g}}(G_{\mathrm g})&         \longrightarrow &\Unip_{\ckcO}(G) \\
     &   (\pi',\pi_{0}) & \mapsto & \pi'\rtimes \pi_{0}.
      \end{array}
    \end{equation*}
    Here $\pi'\rtimes \pi_{0}$ refers to the parabolic induced representation from $P$.
  \end{thm}

% We would like to reduce the problem to consider the bad and good parts
% separately.
The theorem will be proved by translating to a regular infinitesimal character using the following lemma.
 The method is carefully explained in \cite{Mat}. See also
\cite{GI}*{Section~3} for a comprehensive account.


\def\fhhaso{(\fhh^a_1)^*}
\def\fhhast{(\fhh^a_2)^*}
\newcommand{\ff}{f}
\newcommand{\ffcoh}{\varphi}

\begin{lem}[{c. f. \cite{GI}*{Lemma~3.3}}]\label{lem:ff.irr} Suppose that
  \begin{enumerate}[label=(\roman*),series=KLff]
    \item \label{it:KLff.1} $G_{1}$ and $G_{2}$ are two real reductive groups in
          the Harish-Chandra class;
    \item there is an isomorphism
          \[ \ff\colon \fhhaso\rightarrow \fhhast
          \]
          of the duals of the abstract Cartan subalgebras of $G_{1}$ and
          $G_{2}$;
    \item $\lambda_{1} \in \fhhaso$ and $\lambda_{2} = \ff(\lambda_{1})$ are
          dominant regular;
          % \item $\lambda_{1}$ and $\lambda_{2}$ are regular;
    \item %\label{it:KLff.4}
          $\ff$ induces a bijection between $\Delta _{[\lambda_{1}]}$ and
          $\Delta _{[\lambda_{2}]}$, through which we identify
          $W_{[\lambda_{1}]}$ with $W_{[\lambda_{2}]}$;
  \end{enumerate}
  Let $\Coh_{1}$ be a $W_{[\lambda_{1}]}$-submodule of
  $\Coh_{[\lambda_{1}]}(\CK(G_{1}))$ such that $\ev{\lambda_{1}}(\Coh_{1})$ is
  spanned by irreducible $G_{1}$-modules. Suppose
  \begin{equation}\label{eq:coh.ff}
    \ffcoh\colon \Coh_{1}\longrightarrow \Coh_{[\lambda_{2}]}(\CK(G_{2}))
  \end{equation}
  is an injection of $W_{[\lambda_{1}]}=W_{[\lambda_{2}]}$-modules such
  that
  \begin{enumerate}[KLff]
    \item \label{it:KLff.5} $\ffcoh(\Phi)(\lambda_{2})$ is irreducible if $\Phi\in \Coh_{1}$
          and $\Phi(\lambda_{1})$ is irreducible.
  \end{enumerate}

  Then for any $\mu_{1}\in [\lambda_{1}]$, the evaluation at $\mu_{1}$ induces
  an injection
  \[
    \ffcoh_{\mu_{1}} \colon \ev{\mu_{1}}(\Coh_{1}) \longrightarrow \ev{\ff(\mu_{1})}(\Coh_{[\lambda_{2}]}(\CK(G_{2}))). 
  \]
 Furthermore $\ffcoh_{\mu_{1}}(\pi)$ is irreducible if $\pi \in \ev{\mu_{1}}(\Coh_{1})$ is irreducible.
\end{lem}
\begin{proof}
  The first claim on the injectivity of $\ffcoh_{\mu_{1}}$ is clear from \Cref{prop:ev} and the
  injectivity of \eqref{eq:coh.ff}. \trivial[h]{Note that $W_{[\lambda_{1}]}$ is a
    finite group!}

  We now prove the second claim. We may easily reduce it to the case when $\mu_{1}$ is
  $\Delta ^{+}_{[\lambda_{1}]}$ dominant, c.f. \cite{GI}*{Lemma~3.3}.
  Let $\pi \in \ev{\mu_{1}}(\Coh_{1})$. We may find $\Phi\in \Coh_{1}$ such that
  $\Phi(\mu_{1})=\pi$ and $\Phi(\lambda_{1})$ is irreducible (this is a general property of the coherent continuation family). By our
  assumption $\ffcoh(\Phi)(\lambda_{2})$ is irreducible. Therefore
  $\ffcoh_{\mu_{1}}(\pi):=\ffcoh(\Phi)(\ff(\mu_{1}))$ must be either
  irreducible or zero (another general property of the coherent continuation family).
  By the injectivity  of $\ffcoh_{\mu_{1}}$, it is non-zero and so the lemma
  follows.
\end{proof}

%In this paper, we will use the Kazhdan-Lusztig-Vogan theory to obtain the injection
%\eqref{eq:coh.ff} and then reduce the problem to bad and good parities separately.
%Note that the method does not really rely on the Vogan duality and nor
%require the whole coherent continuation modules to be isomorphic.

For a real reductive group $G$ in the Harish-Chandra class, let $\cP_{\lambda}(G)$ denote the set of $K$-conjugate classes of regular characters with infinitesimal character
        $\lambda$, where $K$ is a maximal compact subgroup determined by a fixed Cartan involution. Attached to any regular character $\gamma$, there is a standard representation $\pi _{\gamma}$ and an irreducible representation $\barpi_{\gamma}$. See \cite{Vg}.
        
In order to ensure condition (v) in \Cref{lem:ff.irr}, we will make some additional assumptions (c.~f. \cite{GI}*{\S 3E}).  



%In addition to \ref{it:KLff.1}-\ref{it:KLff.5} in
%\Cref{lem:ff.irr}, we make the following assumptions (c.~f. \cite{GI}*{\S 3E}):
\begin{enumerate}[KLff]
  % \item $G_{1}$ and $G_{2}$ are two real reductive groups ;
  % \item there is an isomorphism $\ff\colon \fhha_{1}\rightarrow \fhha_{2}$
  % between abstract Cartan subalgebras $\fhha_{1}$ and $\fhha_{2}$ of $G_{1}$
  % and $G_{2}$ respectively; ;
  % \item $\lambda_{1}\in \fhhaso$ and $\lambda_2\in \fhhast$ are fixed regular
  % elements such that $\lambda_{1} = \lambda_{2}\circ \ff$;
  % \item $\ff$ induces an isomorphism
  % $\ff\colon R_{\lambda_{1}}^{+}\rightarrow R_{\lambda_{2}}^{+}$, and the
  % associated integral Weyl groups
  % $\ff\colon W_{[\lambda_{1}]}\rightarrow W_{[\lambda_{2}]}$;
  \item there is an injection
        \[
        \ff \colon B\rightarrow \cP_{\lambda_{2}}(G_{2}),
        \]
        where $B\subseteq \cP_{\lambda_{1}}(G_{1})$ is a union of blocks.
  \item for $\gamma_{1}\in B$ and $\gamma_{2} = \ff(\gamma_{1})\in \cP_{\lambda_{2}}(G_{2})$, the conditions (vi), (vii), (viii), (ix) and (x) of \cite{GI}*{\S 3E}) are satisfied. 
  \end{enumerate}
  
  \delete{
  the following conditions are satisfied:
        \begin{enumerate}[label=(\alph*)]
          \item $\ff \circ \theta _{\gamma_{1}} = \theta _{\gamma_{2}}\circ \ff$
                where $\theta _{\gamma_{i}}$ is the Cartan involution on
                the corresponding abstract root system. \trivial[h]{ This
                condition implies that, the notion of compact/complex/real and
                $\alpha\in R^{+}(\gamma), \Phi(\alpha)\notin R^{+}(\gamma)$
                are preserved by $\ff$. In particular, the integral length
                function $l^{I}$ (defined up to a shifting) of $G_{1}$ and
                $G_{2}$ can be uniformly identified. }
              % \item Suppose $\alpha_{1}$ is noncompact type I (resp.
              % noncompact/real type I/II) if and only if
              % $\alpha_{2}:= \ff(\alpha_{1})$ is noncompact type I (resp. type
              % II).
          \item For simple roots in $\Delta ^{+}_{[\lambda_{i}]}$, the notions of
                noncompact/real type I/II \cite{V4} are preserved by $\ff$: $\alpha_{1}$
                is noncompact if and only if
                $\alpha_{2}:= \ff(\alpha_{1})$ is noncompact, and so forth;
          \item The cross actions are compatible under $f$:
                \[
                \ff(w\cross [\gamma_{1}]) = \ff(w)\cross [\ff(\gamma_{1})], \qquad \forall \gamma_{1}\in B, w\in W_{[\lambda_{1}]}.
                \]
          \item The Cayley transforms \cite{V4} are compatible under $f$:
                \[
                \ff( c^{\alpha_{1}} (\gamma_{1})) = c^{\ff(\alpha_{1})}(\ff(\gamma_{1})), \qquad \forall \gamma_{1}\in B, \alpha_{1} \text{ is
                noncompact imaginary}
                \]
                and
                \[
                  \ff( c_{\alpha_{1}} (\gamma_{1})) = c_{\ff(\alpha_{1})}(\ff(\gamma_{1})),
                \]
                for all $\gamma_{1}\in B, \alpha_{1}$ is
                real and satisfies the parity condition \cite[Definition 8.3.11]{V4}.
          % \item The $\tau$ invariants are compatible:
          %       \[
          %       \ff(\tau(\gamma_{1})) = \tau(\ff(\gamma_{1}))\qquad \forall \gamma_{1}\in B_{1}.
          %       \]
          %       \trivial[]{ This condition seems not been used below, but used
          %       in Gan-Ichino's proof! For linear group the matching of
          %       $\tau$-invariant is automatic since it is explicitly determined
          %       by the types of simple roots. For metaplectic groups, it needs
          %       extra information to determine the $\tau$-invariant, see
          %       \cite{RT1}*{Lemma~6.28}. It is not clear to me that the
          %       $\tau$-invariant must match.

          %       On the other hand, the proof of \Cref{lem:ff.irr} seems implies
          %       that $\tau$-invariants match automatically! }
        \end{enumerate}
}

%For a fixed block $B$ a regular infinitesimal character $\lambda_{1}$,
Under the above assumptions, let $\Grt_{\mathrm b}$ be the span of
$\set{\barpi_{\gamma}|\gamma\in B}$ in $\CK_{\lambda_1}(G_1)$, and $\Coh_{1} := \ev{\lambda_1}^{-1}(\Grt_{B})$.
% What we really need is the validity of the following
% conditions:
% \begin{itemize}
%   \item there is an injection of $W_{[\lambda_{1}]}=W_{[\lambda_{2}]}$-module
%   \begin{equation}\label{eq:coh.ff}
%     \ff\colon \Coh_{B_{1}}\rightarrow \Coh_{B_{2}}
%   \end{equation}
%   such that, at our fixed regular infinitesimal character $\lambda_{1}$,
%   $\Phi(\lambda_{1})$ is irreducible implies $\ff(\Phi)(\ff(\lambda_{1}))$
%   is irreducible.
% \end{itemize}
Define $\ffcoh$ by sending $\Phi_{\pi_{\gamma_{1}}}$ to
$\Phi_{\pi_{\gamma_{2}}}$, where $\Phi_{\pi_{\gamma_{1}}}$ denotes the unique coherent family such that $\Phi_{\pi_{\gamma_{1}}}(\lambda _1)=\pi_{\gamma_{1}}$, likewise for $\Phi_{\pi_{\gamma_{2}}}$. 
Given our assumptions $(vi)$ and $(vii)$, argument of \cite{GI}*{\S 3E}) implies that %the above map on coherent continuation representations
$\ffcoh$ can be lifted to a map of Hecke algebra modules and the Kazhdan-Lusztig-Vogan algorithm (\cite{V3,RT1}) implies the preservation of irreducibility of
$\ffcoh$ at $\lambda_{1}$, i.e., condition $(v)$ is satisfied.


% \begin{lem}[{c. f. \cite{GI}*{Lemma~3.3}}]\label{lem:ff.irr}
%   Suppose $\mu_{1}\in [\lambda_{1}]$ is dominant and $\Phi\in \Coh_{B_{1}}$.
%   The evaluation at $\mu_{1}$ induces an injection
%   \[
%   \ff_{\mu_{1}}  \ev{\mu_{1}}(\Coh_{B_{1}}) \longrightarrow  \ev{\mu_{2}}(\Coh_{B_{2}}).
%   \]
%   Moreover, $\ff_{\mu_{1}}(\pi)$ is irreducible if $\pi$ i irreducible.
% \end{lem}
% \begin{proof}
%   The injectivity of $\ff_{\mu_{1}}$ is clear from \Cref{lem:coh.count} and the
%   injectivity of \Cref{eq:coh.ff}. \trivial[]{Note that $W_{[\lambda_{1}]}$ is a
%     finite group!}

%   We now prove the second claim. Let $\Phi\in \Coh_{B_{1}}$ such that
%   $\Phi(\mu_{1})=\pi$ and $\Phi(\lambda_{1})$ is irreducible (the existence
%   of $\Phi$ is an abstract property of the coherent continuation). By our
%   assumption $\ff(\Phi)(\lambda_{1})$ is irreducible. Therefore
%   $\ff(\Phi)(\ff(\mu_{1}))$ must be irreducible since it is non zero by the
%   first claim.
% \end{proof}


We set $G_{1} := G_{\mathrm b}\times G_{\mathrm g}$ and $G_{2}:=G$.
There is a natural map
\[
\ff\colon G_{1} = G_{\mathrm b}\times G_{\mathrm g}\longrightarrow G = G_{2}.
\]
The map is given in \cite{GI}*{\S 3G} when $\star = B$ and in
\cite{RT2}*{\S 5} when $\star = \wtC$. In all other cases, it is the natural
embedding. The map between the duals of abstract Cartan subalgebras
$\ff\colon \fhhaso\rightarrow \fhhast$ is given by putting the coordinates of
$G_{\mathrm b}$ before those of $G_{\mathrm g}$.

Let $\lambda _2\in\fhhast $ be a dominant regular element in $[\lamck]$.
Then
$\lambda_{1}:= \ff^{-1}(\lambda_2)$ is dominant regular for $G_{1}$.
Let $B := \cP_{\lambda_{1}}(G_{1})$ in all cases.
The map $\ff\colon B\rightarrow \cP_{\lambda}(G_{2})$ is induced by the natural map between real Cartan subalgebras.
This is clear when $\ff$ is an embedding.
See \cite{GI}*{\S 3} when $G$ is a special orthogonal group and
\cite{RT2}*{\S 5} when $G$ is a metaplectic group.

\begin{lem}\label{lem:BGcount}
  We have
  \[
    \abs{\Unip_{\ckcO}(G)} =
    \abs{\Unip_{\ckcO_{\mathrm b}}(G_{\mathrm b})}\cdot
    \abs{\Unip_{\ckcO_{\mathrm g}}(G_{\mathrm g})}.
  \]
\end{lem}

\begin{proof}
  It suffices to consider the case where $\nbb$ and $\ngg$ are both non-zero.
  Fix a dominant regular infinitesimal character $\lambda\in [\lamck]$ and write
  $(\lambda_{\mathrm b},\lambda_{\mathrm g}):=\ff^{-1}(\lambda)$. We will require some more
  precise information about the blocks/cells of $G$. Suppose $\star \neq D^{*}$.
  Then by direct computation, 
  \[
    \ff \colon \cP_{\lambda_{\mathrm b}}(G_{\mathrm b})\times \cP_{\lambda_{\mathrm g}}(G_{\mathrm g}) \longrightarrow \cP_{\lambda}(G)
  \]
  is a bijection. In particular each Harish-Chandra cell in $\cP_{\lambda}(G)$
  is the product of a cell in $\cP_{\lambda_{\mathrm b}}(G_{\mathrm b})$ and a cell in
  $\cP_{\lambda_{\mathrm g}}(G_{\mathrm g})$. By \Cref{counteq} and \Cref{lem:lcell.BV0}, this implies $\ffcoh$
  restricts to an isomorphism
  \[
    \Coh_{[\lambda_{\mathrm b}],\bcO_{\mathrm b}}(\CK(G_{\mathrm b}))\otimes \Coh_{[\lambda_{\mathrm g}],\bcO_{\mathrm g}}(\CK(G_{\mathrm g})) \xrightarrow{\ \ \ \ffcoh\ \ \ } \Coh_{[\lambda],\bcO}(\CK(G)).
  \]
  and the lemma follows by the evaluation at $\lamck$.

  Now consider the case where $\star = D^{*}$. In this case, $\cP_{\lambda}(G)$
  has $2$ blocks. Both $\cP_{\lambda_{\mathrm b}}(G_{\mathrm b})$ and
  $\cP_{\lambda_{\mathrm g}}(G_{\mathrm g})$ have only have $1$ block each. Let
  $B_{1} := \ff(\cP_{\lambda_{\mathrm b}}(G_{\mathrm b})\times \cP_{\lambda_{\mathrm g}}(G_{\mathrm g}))$ and
  $B_{2}$ be the other block. For any $\sfW'_{n}$-module $\tau$, let $\tau^{s}$
  denote the twist of $\tau$ by any non-trivial element in $\sfW_{n}/\sfW'_{n}$.

  Then \[
    \begin{split}
      \Coh_{B_{1}} &\cong \cC_{\mathrm b}^{\nbb}\otimes \cC_{\mathrm g}^{p_{\mathrm g},q_{\mathrm g}} = \Coh_{[\lambda_{\mathrm b}],\bcO_{\mathrm b}}(\CK(G_{\mathrm g}))\otimes \Coh_{[\lambda_{\mathrm g}],\bcO_{\mathrm g}}(\CK(G_{\mathrm b})),\\
      \Coh_{B_{2}} &\cong (\cC_{\mathrm b}^{\nbb})^{s}\otimes (\cC_{\mathrm g}^{p_{\mathrm g},q_{\mathrm g}})^{s},
    \end{split}
  \]
   and
  \[
    \Coh_{[\lambda]}(\CK(G)) = \Coh_{B_{1}}\oplus \Coh_{B_{2}}.
  \]
  Clearly $\Coh_{[\lambda], \bcO}$ is compatible with the above decomposition
  and we have
  \[
    \Coh_{[\lambda_{\mathrm b}],\bcO_{\mathrm b}}(\CK(G_{\mathrm g}))\otimes \Coh_{[\lambda_{\mathrm g}],\bcO_{\mathrm g}}(\CK(G_{\mathrm b})) \xrightarrow{\ \ \ \ffcoh\ \ \ } \Coh_{[\lambda],\bcO}(\CK(G))\cap \Coh_{B_{1}}.
  \]

  Observe that $[\tau_{\mathrm b}:(\cC_{\mathrm b}^{\nbb})^{s}]=0$. Therefore by \Cref{cor:bound}, we have
  \[
    \begin{split}
      \abs{\Unip_{\ckcO}(G)} & = \sum_{\tau_{\mathrm b}\boxtimes \tau_{\mathrm g}\in \LC_{\ckcO}} [\tau_{\mathrm b}\otimes \tau_{\mathrm g}:\Coh_{[\lambda],\bcO}(\CK(G))\cap \Coh_{B_1}]\\
    & \ \ + \sum_{\tau_{\mathrm b}\boxtimes \tau_{\mathrm g}\in \LC_{\ckcO}}[\tau_{\mathrm b}\otimes \tau_{\mathrm g}:\Coh_{[\lambda],\bcO}(\CK(G))\cap \Coh_{B_2}] \\
  &= \sum_{\tau_{\mathrm b}\boxtimes \tau_{\mathrm g}\in \LC_{\ckcO}} [\tau_{\mathrm b}\otimes \tau_{\mathrm g}:
  \Coh_{[\lambda_{\mathrm b}],\bcO_{\mathrm b}}(\CK(G_{\mathrm b}))\otimes \Coh_{[\lambda_{\mathrm g}],\bcO_{\mathrm g}}(\CK(G_{\mathrm g}))]\\
  &= \abs{\Unip_{\ckcO_{\mathrm b}}(G_{\mathrm b})}\cdot \abs{\Unip_{\ckcO_{\mathrm g}}(G_{\mathrm g})}.
\end{split}
\]
\end{proof}


\begin{proof}[Proof of \Cref{thm:red}]: We first prove the injectivity of $\fI$. Let % $\mu_{2} = \lamck$
      % and
  $\mu_{1} = \ff^{-1}(\lamck)$.
  Since coherent continuation is compatible with induction
  \cite{Vg}*{Proposition~7.4.1}, we have the
  following commutative diagram
  \[
    \begin{tikzcd}[column sep={4cm,between origins}]
      &  \Coh_{[\lambda'_{\mathrm b}]}(\CK(G'_{\mathrm b}))\otimes \Coh_{[\lambda_{\mathrm g}]}(\CK(G_{\mathrm g})) \ar[dl,"\Ind_{P_{\mathrm b}}^{G_{\mathrm b}}\otimes \id"']\ar[dr,"\Ind_{P}^{G}"]&\\
      \Coh_{[\lambda_{\mathrm b}]}(\CK(G_{\mathrm b}))\otimes \Coh_{[\lambda_{\mathrm g}]}(\CK(G_{\mathrm g})) \ar[d,"\ev{\mu_{1}}"'] \ar[rr,"\ffcoh"]& & \Coh_{[\lamck]}(\CK(G)) \ar[d,"\ev{\ckcO}"]\\
      \Grt_{\mu_{1}}(G_{\mathrm b}\times G_{\mathrm g}) \ar[rr,"\ffcoh_{\mu_{1}}"]& &
      \Grt_{\lamck}(G).\\
    \end{tikzcd}
  \]
  %Here $P_{\mathrm b}$ is the parabolic subgroup of $G_{\mathrm b}$ whose Levi component is isomorphic to $G'_{\mathrm b}$ and
  %$P$ is the parabolic of $G$ whose Levi component is isomorphic to $G'_{\mathrm b}\times G_{\mathrm g}$.
  The horizontal line $\ffcoh$ is clearly an
  injection, which implies $\ffcoh_{\mu_{1}}$ is an injection
  by \Cref{lem:ff.irr}. Note that $\ffcoh_{\mu_{1}}$ sends
  $\Ind_{P_{\mathrm b}}^{G_{\mathrm b}}(\pi')\otimes \pi_{0}$ to $\Ind_{P}^{G} (\pi'\otimes \pi_{0})$.
  \trivial[h]{ Let
    $\Phi'\otimes \Phi_0\in \Coh_{[\lambda'_{\mathrm b}]}(G'_{\mathrm b})\otimes \Coh_{[\lambda_{\mathrm g}]}(G_{\mathrm g})$
    be coherent family such that
    $\Phi'\otimes \Phi_0(\lamck) = \pi'\otimes \pi_0$. Now
    $\Ind_{P_{\mathrm b}}(\pi')\otimes \pi_{0} = (\Ind_{P_{\mathrm b}}\otimes \id) (\Phi'\otimes \Phi_{0})(\mu_{1})$
    and
    $\Ind_{P}(\pi'\otimes \pi_{0}) = \ffcoh(\Ind_{P_{\mathrm b}}\otimes \id (\Phi'\otimes \Phi_{0}))(\lamck) = (\Ind_{P}^{G}) (\Phi'\otimes \Phi_{0})(\lamck)$
  }
  The injectivity of $\fI$ now follows from \Cref{lem:Unip.BP}.

The counting result of \Cref{lem:BGcount} finally implies the bijectivity of $\fI$.
\end{proof}


%\subfile{counting_cl}



\section{Combinatorics of painted bipartitions}

Let $\star$, $G$, and $\check \CO$ be as before. In this section, we assume that $\star \in \Set{B,C,\wtC,C^{*},D,D^{*}}$, and $\ckcO = \ckcOg$, namely $\ckcO $ has $\star$-good parity.

Recall the set of primitive $\star$-pairs $\CPPs(\ckcO)$ in $\ckcO$. For a subset $\wp$ of $\CPPs(\ckcO)$, we have defined a bipartition $\tau_{\wp}=(\imath_{\wp},\jmath_{\wp})$ in \Cref{sec:LCBCD}.

The purpose of this section is to prove the   following combinatorial result. 

\begin{prop} \label{prop:PBP} When $\star\in \set{C^{*}, D^{*}}$,
  \[
    \PBP_{\star}(\tau_{\wp}) = \emptyset, \quad \text{if } \wp \neq \emptyset.
  \]
  When $\star\in \set{B,C,\wtC,D}$,
  \[
    \sharp(\PBP_{\star}(\tau_{\wp})) = \sharp(\PBP_{\star}(\tau_{\emptyset})), \quad \forall \wp \subseteq \CPPs(\ckcO).
  \]
 Consequently we always have
  \[
    \sharp(\tPBP_{\star}(\ckcO)) = 2^{\sharp(\CPPs(\ckcO))}\cdot \sharp(\PBP_{\star}(\ckcO)), \,\, \text{ and } \,\, \sharp(\tPBP_{\mathrm g}(\ckcO)) = 2^{\sharp(\CPPs(\ckcO))}\cdot \sharp(\PBP_{\mathrm g}(\ckcO)).
  \]
\end{prop}

% \subsection{The case of quaternionic groups}

We shall deal with the two quaternionic cases first, which are simple. When $\star \in \set{B, C, \wtC, D}$, the proof of the main statement of the above proposition involves an elaborate reduction argument (by removing elements from $\wp$ one-by-one), and will be handled separately in \Cref{lem:down} below.
\begin{proof}%[Proof of {\Cref{prop:PBP}} the quaternionic case]

    \smallskip

  First consider the case when $\star = C^{*}$. Suppose that
  $\wp \neq \emptyset$. Then we have
  \begin{equation}\label{eq:res.C*}
    \bfcc_{i}(\imath_{\wp}) = \half(\bfrr_{2i-1}(\ckcO)+1)>
    \half(\bfrr_{2i}(\ckcO)-1) = \bfcc_{i}(\jmath_{\wp}),
    \quad \forall \,\, (2i-1, 2i)\in \wp,
  \end{equation}
  Let $\uptau = (\imath_{\wp}, \cP)\times (\jmath_{\wp},\cQ)\times \star$ be an element in $\PBP_{\star}(\tau_{\wp})$. By the requirements of a painted bipartition, we have
  \[
    \bfcc_{i}(\imath_{\wp}) = \sharp\set{j| \cP(i,j)=\bullet} = \sharp\set{j| \cQ(i,j)=\bullet} \leq \bfcc_{i}(\jmath_{\wp}), \quad \forall \,\, i=1,2,3,\cdots,
  \]
  which contradicts \eqref{eq:res.C*}. Hence, $\PBP_{\star}(\tau_{\wp})= \emptyset$.

  \smallskip

  Now consider the case when $\star = D^{*}$. Suppose that $\wp \neq \emptyset$.
  Then we have
  \begin{equation}\label{eq:res.D*}
    \bfcc_{i+1}(\imath_{\wp}) = \half(\bfrr_{2i}(\ckcO)+1)>
    \half(\bfrr_{2i+1}(\ckcO)-1) = \bfcc_{i}(\jmath_{\wp}),
    \quad \forall \,\, (2i, 2i+1)\in \wp.
  \end{equation}
  Let $\uptau = (\imath_{\wp}, \cP)\times (\jmath_{\wp},\cQ)\times \star$ be an element in $\PBP_{\star}(\tau_{\wp})$. By the requirements of a painted bipartition, we have
  \[
    \bfcc_{i+1}(\imath_{\wp}) \leq \sharp\set{j| \cP(i,j)=\bullet} =\sharp\set{j| \cQ(i,j)=\bullet} \leq \bfcc_{i}(\jmath_{\wp}), \quad \forall \,\, i = 1,2,3, \cdots,
  \]
  which contradicts \eqref{eq:res.D*}. Hence, $\PBP_{\star}(\tau_{\wp})= \emptyset$.

This completes the proof for the quaternionic cases.
\end{proof}

The rest of this section is devoted to the proof of the following


\def\PPm{\wp_{\downarrow}}
\def\uptaum{\uptau_{\downarrow}}

% Suppose $\star = \wtC$, $\wp\neq \emptyset$, and
% $t:=\min{t|(2t-1,2t)\in \wp}$. Let $\PPm:=\wp - \set{(2t-1,2t)}$. Let
% $\PPm:=\wp - \set{(2t-1,2t)}$.

\begin{lem}\label{lem:down}
  Suppose $\star \in \set{B, C, \wtC, D}$ and
  $\wp$ is a non-empty subset of $\CPPs(\ckcO)$.
  Let
  \[
    t:=
    \begin{cases}
      \min\set{i|(2i-1,2i)\in \wp} & \text{when $\star \in \set{C,\wtC}$}\\
      \min\set{i|(2i,2i+1)\in \wp} & \text{when $\star \in \set{B,D}$}\\
    \end{cases}
  \]
  and
  \[
    \PPm:=
    \begin{cases}
      \wp - \set{(2t-1,2t)}  & \text{when $\star \in \set{C,\wtC}$}\\
      \wp -  \set{(2t,2t+1)} & \text{when $\star \in \set{B,D}$}\\
    \end{cases}
  \]
  Then
  \[
    \sharp(\PBP_{\star}(\tau_{\PPm})) = \sharp(\PBP_{\star}(\tau_{\wp})).
  \]
\end{lem}
\begin{proof}
  We prove the equality by defining a bijection
  \[
    T_{\PPm,\wp}\colon \PBP_{\star}(\tau_{\PPm}) \rightarrow \PBP_{\star}(\tau_{\wp})\quad \uptaum \mapsto \uptau
  \]
  %and its inverse $T_{\wp,\PPm}$
  explicitly case by case.
  In the following, $\uptau = (\imath_{\wp},\cP_{\uptau})\times (\jmath_{\wp},\cQ_{\uptau})$
  will always denote an element in $\PBP_{\star}(\tau_{\wp})$ and
  $\uptaum = (\imath_{\PPm},\cP_{\uptaum})\times (\jmath_{\PPm},\cQ_{\uptaum})$
  an element in $\PBP_{\star}(\tau_{\PPm})$.

  \medskip

  % \[
  %   T_{\wp,\PPm}\colon \PBP_{\star}(\tau_{\wp}) \rightarrow \PBP_{\star}(\tau_{\PPm}).
  % \]


  %We start with the simplest case.

  \smallskip

  Case $\star = \wtC$:
  %Let $(b_{1},b_{2}) = (\frac{\bfrr_{2t-1}(\ckcO)}{2},\frac{\bfrr_{2t}(\ckcO)}{2})$.
  We have
  \[
    \begin{split}
      (\bfcc_{t}(\imath_{\PPm}), \bfcc_{t}(\jmath_{\PPm}))
      &= (\bfcc_{t}(\jmath_{\wp}), \bfcc_{t}(\imath_{\wp})),\AND\\
      % (\bfcc_{t}(\imath_{\PPm}), \bfcc_{t}(\jmath_{\PPm}))
      % &= (b_{1},b_{2})= (\bfcc_{t}(\jmath_{\wp}), \bfcc_{t}(\imath_{\wp})),\AND\\
      (\bfcc_{i}(\imath_{\PPm}), \bfcc_{i}(\jmath_{\PPm}) )
      & = (\bfcc_{i}(\imath_{\wp}), \bfcc_{i}(\jmath_{\wp}) ) \quad \text{for $i\neq t$}.
    \end{split}
  \]

  For $\uptaum\in\PBPs(\tau_{\PPm})$, define $\uptau=:T_{\PPm,\wp}(\uptaum)$ by the following formula:
  \[
    \begin{split}
      \text{$\forall (i,j)\in \BOX{\imath_{\wp}}$,} \quad   \cP_{\uptau}(i,j) &=  \cP_{\uptaum}(i,j),\\
      \text{$\forall (i,j)\in \BOX{\jmath_{\wp}}$,} \quad \cQ_{\uptau}(i,j) &= \begin{cases}
        r& \text{if $j=t$ and  $\cP_{\uptaum}(i,j)=s$,}\\
        d& \text{if $j=t$ and  $\cP_{\uptaum}(i,j)=c$,}\\
        \cQ_{\uptaum}(i,j) &\text{otherwise.}
      \end{cases}
    \end{split}
  \]
 We easily check that the above formula defines a valid
  painted bipartition $\uptau$ and construct the inverse map $T_{\wp,\PPm}$ by reversing the process.
  This finishes the proof for the case when $\star=\wtC$. \medskip

  \trivial[h]{The inverse map
  \[
    T_{\wp,\PPm}\colon \PBP_{\star}(\tau_{\wp}) \rightarrow \PBP_{\star}(\tau_{\PPm}).
  \]
  is given by the following formula:
  \[
    \begin{split}
      \text{$\forall (i,j)\in \BOX{\imath_{\PPm}}$,} \quad \cP_{\uptaum}(i,j) &= \begin{cases}
        s& \text{if $j=t$ and  $\cQ_{\uptau}(i,j)=r$,}\\
        c& \text{if $j=t$ and  $\cQ_{\uptau}(i,j)=d$,}\\
        \cP_{\uptau}(i,j) &\text{otherwise.}
      \end{cases}\\
      \text{$\forall (i,j)\in \BOX{\jmath_{\PPm}}$,} \quad   \cQ_{\uptaum}(i,j) &=  \cQ_{\uptau}(i,j).\\
    \end{split}
  \]
  }
  % We leave it to the reader to check that the above formula does define a valid
  % painted bipartition $\uptaum$. Retain the above notation, it is easy to check that the
  % inverse map
  % \[
  %   T_{\PPm,\wp}\colon \PBP_{\star}(\tau_{\PPm}) \rightarrow \PBP_{\star}(\tau_{\wp})\quad \uptaum \mapsto \uptau
  % \]
  % is given by the following formula:
  % \[
  %   \begin{split}
  %     \text{$\forall (i,j)\in \BOX{\imath_{\wp}}$,} \quad   \cP_{\uptau}(i,j) &=  \cP_{\uptaum}(i,j),\\
  %     \text{$\forall (i,j)\in \BOX{\jmath_{\wp}}$,} \quad \cQ_{\uptau}(i,j) &= \begin{cases}
  %       r& \text{if $j=t$ and  $\cP_{\uptaum}(i,j)=s$,}\\
  %       d& \text{if $j=t$ and  $\cP_{\uptaum}(i,j)=c$,}\\
  %       \cQ_{\uptaum}(i,j) &\text{otherwise.}
  %     \end{cases}
  %   \end{split}
  % \]

  \medskip

  Case $\star = C$:
  % Let
  % $(b_{1},b_{2}) = (\frac{\bfrr_{2t-1}(\ckcO)-1}{2},\frac{\bfrr_{2t}(\ckcO)+1}{2})$.
  We have
  \[
    \begin{split}
      (\bfcc_{t}(\imath_{\PPm}), \bfcc_{t}(\jmath_{\PPm})) &=
      (\bfcc_{t}(\jmath_{\wp})+1, \bfcc_{t}(\imath_{\wp})-1) \AND \\
     %  &= (b_{2},b_{1}),  \\
     % &= (b_{1}-1,b_{2}+1),\AND\\
      % (\bfcc_{t}(\imath_{\PPm}), \bfcc_{t}(\jmath_{\PPm})) &= (b_{2},b_{1}),  \\
      % (\bfcc_{t}(\imath_{\wp}), \bfcc_{t}(\jmath_{\wp})) &= (b_{1}-1,b_{2}+1),\AND\\
      (\bfcc_{i}(\imath_{\PPm}),\bfcc_{i}(\jmath_{\PPm})) &=(\bfcc_{i}(\imath_{\wp}),\bfcc_{i}(\jmath_{\wp}))\quad \text{for $i\neq t$}.
    \end{split}
  \]
  % Let
  % $a = \half(\bfrr_{2t-1}(\ckcO)-\bfrr_{2t}(\ckcO))-1 = \bfcc_{t}(\jmath_{\PPm})-\bfcc_{t}(\imath_{\PPm})$.

  \trivial[h]{
    The idea of the definition of $T_{\PPm,\wp}$ is that we move ``$s$'' appeared
    in the $t$-th column of $\cQ_{\uptaum}$ to the $t$-th column of
    $\cP_{\uptau}$.
  }

  For $\uptaum\in \PBPs(\tau_{\PPm})$, we define $\uptau$ by the following algorithm:
  \begin{description}
    \item[STEP~1] Define a map
          $\cP'\colon \BOX{\imath_{\wp}}\rightarrow \set{\bullet,r,c,d}$ (as a candidate for $\cP_{\uptau}$), by the following rules:
          \begin{enumerate}[label=(\alph*)]
            \item Suppose
            $\cP_{\uptaum}(\bfcc_{t}(\imath_{\PPm}),t)\neq \bullet$.
            \begin{itemize}
              \item If $\bfcc_{t}(\imath_{\PPm})\geq 2$ and
              $\cP_{\uptaum}(\bfcc_{t}(\imath_{\PPm})-1,t) = c$,
              we define
              \[
                \cP'(i,j) := \begin{cases}
                  r ,& \text{if $j=t$ and $\bfcc_{t}(\imath_{\PPm})-1
                    \leq i \leq \bfcc_{t}(\imath_{\wp})-2$},\\
                  c ,& \text{if $(i,j)=(\bfcc_{t}(\imath_{\wp})-1,t)$},\\
                  d ,& \text{if $(i,j)=(\bfcc_{t}(\imath_{\wp}),t)$},\\
                  \cP_{\uptaum}(i,j) ,&\text{otherwise}.
                \end{cases}
              \]
              \item Otherwise, we define
              \[
                \cP'(i,j) := \begin{cases}
                  r ,& \text{if $j=t$ and $\bfcc_{t}(\imath_{\PPm})
                    \leq i \leq \bfcc_{t}(\imath_{\wp})-1$},\\
                  \cP_{\uptaum}(\bfcc_{t}(\imath_{\PPm}),t) ,&
                  \text{if $(i,j)=(\bfcc_{t}(\imath_{\wp}),t)$},\\
                  \cP_{\uptaum}(i,j) ,&\text{otherwise}.
                \end{cases}
              \]
            \end{itemize}
            \item Suppose $\cP_{\uptaum}(\bfcc_{t}(\imath_{\PPm}),t)=\bullet$.
            \begin{itemize}
              \item If $\bfcc_{t+1}(\imath_{\PPm}) = \bfcc_{t}(\imath_{\PPm})$
              and
              $\cP_{\uptaum}(\bfcc_{t}(\imath_{\PPm}),t+1) = r$,
              we define
              \[
                \cP'(i,j) := \begin{cases}
                  r ,& \text{if $j=t$ and $\bfcc_{t}(\imath_{\PPm})\leq i \leq \bfcc_{t}(\imath_{\wp})-1$},\\
                  c ,& \text{if $(i,j)=(\bfcc_{t+1}(\imath_{\PPm}),t+1)$},\\
                  d ,& \text{if $(i,j)=(\bfcc_{t}(\imath_{\wp}),t)$},\\
                  \cP_{\uptaum}(i,j) ,&\text{otherwise}.
                \end{cases}
              \]
              \item Otherwise, we define
              \[
                \cP'(i,j) := \begin{cases}
                  r ,& \text{if $j=t$ and $\bfcc_{t}(\imath_{\PPm})\leq i \leq \bfcc_{t}(\imath_{\wp})-2$},\\
                  c ,& \text{if $(i,j)=(\bfcc_{t}(\imath_{\wp})-1,t)$},\\
                  d ,& \text{if $(i,j)=(\bfcc_{t}(\imath_{\wp}),t)$},\\
                  \cP_{\uptaum}(i,j) ,&\text{otherwise}.
                \end{cases}
              \]
            \end{itemize}
          \end{enumerate}

    \item[STEP 2] From the construction of $\cP'$, there are four possibilities for $\cP'$ to violate the requirements of a painting on $\imath_{\wp}$, which are detailed as follows. We must have $t>1$ and violations occur in positions of $(i,j)\in \BOX{\imath_{\wp}}$ inside the following $2\times 2$ square
         \[
          A :=
          \begin{pmatrix}
            (\bfcc_{t}(\imath_{\wp})-1,t-1) & (\bfcc_{t}(\imath_{\wp})-1,t) \\
            (\bfcc_{t}(\imath_{\wp})\;\phantom{-1}\;,t-1) & (\bfcc_{t}(\imath_{\wp})\;\phantom{-1}\;,t) \\
          \end{pmatrix}
          \]

          %\[
          %A :=
          %\begin{pmatrix}
          %  \cP'(\bfcc_{t}(\imath_{\wp})-1,t-1) & \cP'(\bfcc_{t}(\imath_{\wp})-1,t) \\
          %  \cP'(\bfcc_{t}(\imath_{\wp})\;\phantom{-1}\;,t-1) & \cP'(\bfcc_{t}(\imath_{\wp})\;\phantom{-1}\;,t) \\
          %\end{pmatrix}
          %\]
          \trivial[h]{
          Note that  $\cP'(\bfcc_{t}(\imath_{\wp})-1,t-1)$ is always equal to
          $\bullet$.
          }

    Let $\cP_{\uptau}\colon \BOX{\imath_{\wp}}\rightarrow \set{\bullet,r,c,d}$
          be the painting on $\imath_{\wp}$ defined as follows:
          \begin{itemize}
            \item When $(i,j)\in \BOX{\imath_{\wp}}$ and
            $\set{i-\bfcc_{t}(\imath_{\wp})+1, j-t+1}\nsubseteq \set{0,1}$
            (i.e. $(i,j)$ is not one of the four boxes corresponding to
            $A$), we define
            \[
              \cP_{\uptau}(i,j):= \cP'(i,j).
            \]

            \item For the four boxes corresponding to $A$, we modify the symbols by
            setting
            \begin{equation} \label{eq:modP}
              \begin{split}
              %  &\begin{pmatrix}
              %    \cP_{\uptau}(\bfcc_{t}(\imath_{\wp})-1,t-1) & \cP_{\uptau}(\bfcc_{t}(\imath_{\wp})-1,t) \\
              %   \cP_{\uptau}(\bfcc_{t}(\imath_{\wp})\;\phantom{-1}\;,t-1)
              %    & \cP_{\uptau}(\bfcc_{t}(\imath_{\wp})\;\phantom{-1}\;,t) \\
              % \end{pmatrix}\\
              \cP_{\uptau}|_A  :=&
                \begin{cases}
                  \begin{pmatrix}
                    r & c\\
                    r & d
                  \end{pmatrix}, & \text{if } \cP'|_A =
                  \begin{pmatrix}
                    \bullet & r\\
                    r & r
                  \end{pmatrix},\\[1.5em]
                  \begin{pmatrix}
                    r & c\\
                    c & d
                  \end{pmatrix}, & \text{if } \cP'|_A =
                  \begin{pmatrix}
                    \bullet & r\\
                    c & r
                  \end{pmatrix},\\[1.5em]
                  \begin{pmatrix}
                    r & c\\
                    d & d
                  \end{pmatrix}, & \text{if } \cP'|_A =
                  \begin{pmatrix}
                    \bullet & r\\
                    d & r
                  \end{pmatrix},\\[1.5em]
                  \begin{pmatrix}
                    c & c\\
                    d & d
                  \end{pmatrix}, & \text{if } \cP'|_A =
                  \begin{pmatrix}
                    \bullet & r\\
                    d & c
                  \end{pmatrix}.\\
                \end{cases}
              \end{split}
            \end{equation}
          \end{itemize}

    \item[STEP 3] The painted bipartition $\uptau$ is uniquely determined by
          $\cP_{\uptau}$. More precisely, $\cQ_{\uptau}$ is given by the following
          formula: for $(i,j)\in \BOX{\jmath_{\wp}}$,
          \[
          \cQ_{\uptau}(i,j) :=
          \begin{cases}
            s, & \begin{minipage}{17em}if $\cP'$ is not a valid painting on $\imath_{\wp}$\\
              and $(i,j)= (\bfcc_{t}(\imath_{\wp})-1,t-1)$,
              \end{minipage}\\
            \cQ_{\uptaum}(i,j), & \text{otherwise.}
            \end{cases}
          \]
  \end{description}

 It is not difficult to check that $\uptau$ is a valid painted bipartition
 and to construct the inverse map $T_{\wp,\PPm}$ by reversing the above steps.

 \trivial[h]{
   The inverse map $T_{\wp,\PPm}$ is given by the following algorithm:
   \begin{description}
     \item[STEP 1] We first recover $\cP'$.
           If $t=1$ or $\cP'(\bfcc_{t}(\imath_{\wp})-1,t-1)=\bullet$, then
           $\cP':= \cP_{\wp}$.
           Otherwise,
           $\cP'$ is given by $\cP_{\wp}$ except the $2\times 2$ square in
           \eqref{eq:modP} which is given by reversing the formula cited.
     \item[STEP 2]

           \def\xxn{\cP_{\uptaum}(\bfcc_t(\imath_{\PPm})-1,t)} %x_0
           \def\xxo{\cP_{\uptaum}(\bfcc_t(\imath_{\wp}),t)} %x_1
           \def\xxd{\cP_{\uptaum}(\bfcc_t(\imath_{\wp}),t+1)} %x_2
           \def\yyn{\cP'(\bfcc_t(\imath_{\PPm})-1,t)} %y_0
           \def\yyo{\cP'(\bfcc_t(\imath_{\wp})-1,t)} %y_1
           \def\yyt{\cP'(\bfcc_t(\imath_{\wp}),t)} %y_3
           \def\yyd{\cP'(\bfcc_t(\imath_{\wp}),t+1)} %y_2
           We have the following cases:
           \begin{enumerate}[label=(\alph*)]
             \item Suppose $\yyo=r$.
             \begin{itemize}
               \item If $\bfcc_{t+1}(\imath_{\wp}) = \bfcc_{t}(\imath_{\PPm})$
               and
               \[
                 (\yyd,\yyt) = (c,d),
               \]
               let
               \[
                 (\xxo,\xxd):=(\bullet, r)
               \]
               \item Otherwise, let \[
                 \xxo:=\yyt.
               \]
             \end{itemize}
             \item Suppose $\yyo=c$
             \begin{itemize}
               \item If $\bfcc_{t}(\imath_{\PPm})\geq 2$ and $\xxn=r$,
               then let
               \[
                 (\xxn,\xxo):=(c,d).
               \]
               \item Otherwise, let
               \[
                 \xxo :=\bullet.
               \]
             \end{itemize}
           \end{enumerate}
           For the boxes $(i,j)$ in $\BOX{\imath_{\uptaum}}$ which are not specified
           in the above procedure, set
           \[
           \cP_{\uptaum}(i,j):=\cP'(i,j).
           \]
     \item[STEP 3]
           Now $\cP_{\uptaum}$ uniquely determine the painted bipartition
           $\uptaum$.
   \end{description}
   The construction of the inverse map implies that $T_{\PPm,\wp}$ is a
   bijection.
 }

  \def\ckcOa{\ckcO^{\uparrow}}
  \def\PPa{\wp^{\uparrow}}
  \def\PPam{\wp^{\uparrow}_{\downarrow}}
  \def\uptaua{\uptau^{\uparrow}}
  \def\tauPPa{\tau_{\PPa}}
  \def\tauPPam{\tau_{\PPam}}
  \def\stara{\star^{\uparrow}}

  \medskip

  {Now suppose $\star \in \set{B,D}$.}
  We will prove the lemma by appealing to the corresponding assertion for the case of $\wtC/C$.
  Let
  \[
    \stara := \begin{cases}
      \wtC ,& \text{when $\star=B$},\\
      C ,& \text{when $\star=D$},
    \end{cases}
  \]
  and $\ckcOa$ be the partition defined by
  \[
    \bfrr_{1}(\ckcOa) = \bfrr_{1}(\ckcO)+2, \AND \bfrr_{i+1}(\ckcOa)
    = \bfrr_{i}(\ckcO)\quad \text{for all $i=1,2,3,\cdots,$}.
  \]
  Clearly
  \[
    \CPP_{\stara }(\ckcOa) = \set{(1,2)}\cup \set{(i+1,i+2)|(i,i+1)\in \CPP_{\star}(\ckcO)}.
  \]




  % Let $\PPa$ denote the subset in
  % $\CPP(\ckcOa)$ defined by
  % \[
  %   \PPa:=\set{(i+1,i+2)|(i,i+1)\in \sP}.
  % \]
  % Let $r_{0}:= \half \bfrr_{1}(\ckcO)+1 = \bfcc_{1}(\imath_{\PPa})$. For $x=c$
  % or $s$, define
  % \[
  %   \PBP_{\wtC}^{x}(\tauPPa):= \Set{\uptaua\in \PBP_{\wtC}(\tau_{\PPa})|\cP_{\uptaua}\left(r_{0},1\right)=x}
  % \]
  % where $\tauPPa$ is the bipartition defined with respect to $\ckcOa$.

  % Let $\PPam:=(\PPa)_{\downarrow}$. It is easy to check that
  % \begin{itemize}
  %   \item
  %   $\PBP_{\wtC}(\tauPPa) = \PBP_{\wtC}^{c}(\tauPPa)\sqcup \PBP_{\wtC}^{s}(\tauPPa)$,
  %   \item the map $T_{\PPa,\PPam}$ restricted into a bijection from
  %   $\PBP_{\wtC}^{x}(\tauPPa)$ onto $\PBP_{\wtC}^{x}(\tauPPam)$ (here $x=s$ or
  %   $c$), and
  %   \item the descent map restricted to bijections
  %   \[
  %     \text{
  %         $\PBP_{\wtC}^{s}(\tau_{\PPa})\longrightarrow \PBP_{B^{+}}(\tau_{\sP})$
  %         and
  %         $\PBP_{\wtC}^{c}(\tau_{\PPa})\longrightarrow \PBP_{B^{-}}(\tau_{\sP})$.
  %     }
  %   \]
  % \end{itemize}

  % Now the bijection $T_{\wp,\wpm}$ is defined by making the following diagram
  % commutative
  % \[
  %   \begin{tikzcd}
  %     \PBP_{\wtC}^{s}(\tau_{\PPa}) \ar[r] \ar[d] & \PBP_{B^{+}}^{s}(\tau_{\wp}) \ar[d]\\
  %     \PBP_{\wtC}^{s}(\tau_{\PPam}) \ar[r] & \PBP_{B^{+}}^{s}(\tau_{\PPm}) \\
  %   \end{tikzcd}
  % \]

 % For each subset $\sP\subset \CPP(\ckcO)$,
  Let $\PPa$ denote the subset in
  $\CPP_{\stara }(\ckcOa)$ defined by
  \[
    \PPa:=\set{(i+1,i+2)|(i,i+1)\in \sP}.
  \]
  % Let $r_{0}:= \half \bfrr_{1}(\ckcO)+1 = \bfcc_{1}(\imath_{\PPa})$.
  % For $x=c$
  % or $s$, define
  % \[
  %   \PBP_{\wtC}^{x}(\tauPPa):= \Set{\uptaua\in \PBP_{\wtC}(\tau_{\PPa})|\cP_{\uptaua}\left(r_{0},1\right)=x}
  % \]
  % where $\tauPPa$ is the bipartition defined with respect to $\ckcOa$.

Let $\PPam:=(\PPa)_{\downarrow}$. Recall from \cite[Section 2.3]{BMSZ2} the (naive) descent map $\DD$ of painted bipartitions.
It is easy to check that the descent maps (horizontal arrows) in the following diagram are bijections.
 %(c.f. \cite{BMSZ1}*{Lemma???}). % restricted to bijections
 % \[
 %   \PBP_{\wtC}(\tau_{\PPa})\longrightarrow \PBP_{B}(\tau_{\sP}).
 % \]
  \[
    \begin{tikzcd}
      \PBP_{\wtC}(\tau_{\PPam}) \ar[r,"\DD",two heads,hook] \ar[d,two heads,hook,"T_{\PPam,\PPa}"']
      & \PBP_{B}(\tau_{\PPm}) \ar[d,dashed,"T_{\PPm,\wp}"]\\
      \PBP_{\wtC}(\tau_{\PPa}) \ar[r,"\DD",two heads,hook] & \PBP_{B}(\tau_{\wp}) \\
    \end{tikzcd}
  \]
 Since there is a bijection $T_{\PPam,\PPa}$ in the left vertical arrow by case $\wtC$, we may define a bijection $T_{\PPm,\wp}$ in the right vertical arrow by making the above diagram commutative.
 This completes the proof for the case $B$. The case $D$ is entirely similar.
\end{proof}





\begin{bibdiv}
  \begin{biblist}
% \bib{AB}{article}{
%   title={Genuine representations of the metaplectic group},
%   author={Adams, Jeffrey},
%   author = {Barbasch, Dan},
%   journal={Compositio Mathematica},
%   volume={113},
%   number={01},
%   pages={23--66},
%   year={1998},
% }

% \bib{Ad83}{article}{
%   author = {Adams, J.},
%   title = {Discrete spectrum of the reductive dual pair $(O(p,q),Sp(2m))$ },
%   journal = {Invent. Math.},
%   number = {3},
%  pages = {449--475},
%  volume = {74},
%  year = {1983}
% }

%\bib{Ad07}{article}{
%  author = {Adams, J.},
%  title = {The theta correspondence over R},
%  journal = {Harmonic analysis, group representations, automorphic forms and invariant theory,  Lect. Notes Ser. Inst. Math. Sci. Natl. Univ. Singap., 12},
% pages = {1--39},
% year = {2007}
% publisher={World Sci. Publ.}
%}


\bib{ABV}{book}{
  title={The Langlands classification and irreducible characters for real reductive groups},
  author={Adams, J.},
  author={Barbasch, D.},
  author={Vogan, D. A.},
  series={Progress in Math.},
  volume={104},
  year={1991},
  publisher={Birkhauser}
}

\bib{AC}{article}{
  title={Algorithms for representation theory of
    real reductive groups},
  volume={8},
  DOI={10.1017/S1474748008000352},
  number={2},
  journal={Journal of the Institute of Mathematics of Jussieu},
  publisher={Cambridge University Press},
  author={Adams, Jeffrey},
  author={du Cloux, Fokko},
  year={2009},
  pages={209-259}
}

% \bib{ArPro}{article}{
%   author = {Arthur, J.},
%   title = {On some problems suggested by the trace formula},
%   journal = {Lie group representations, II (College Park, Md.), Lecture Notes in Math. 1041},
%  pages = {1--49},
%  year = {1984}
% }


% \bib{ArUni}{article}{
%   author = {Arthur, J.},
%   title = {Unipotent automorphic representations: conjectures},
%   %booktitle = {Orbites unipotentes et repr\'esentations, II},
%   journal = {Orbites unipotentes et repr\'esentations, II, Ast\'erisque},
%  pages = {13--71},
%  volume = {171-172},
%  year = {1989}
% }

% \bib{AK}{article}{
%   author = {Auslander, L.},
%   author = {Kostant, B.},
%   title = {Polarizations and unitary representations of solvable Lie groups},
%   journal = {Invent. Math.},
%  pages = {255--354},
%  volume = {14},
%  year = {1971}
% }


% \bib{B.Uni}{article}{
%   author = {Barbasch, D.},
%   title = {Unipotent representations for real reductive groups},
%  %booktitle = {Proceedings of ICM, Kyoto 1990},
%  journal = {Proceedings of ICM (1990), Kyoto},
%    % series = {Proc. Sympos. Pure Math.},
%  %   volume = {68},
%      pages = {769--777},
%  publisher = {Springer-Verlag, The Mathematical Society of Japan},
%       year = {2000},
% }



\bib{B.Orbit}{article}{
  author = {Barbasch, D.},
  title = {Orbital integrals of nilpotent orbits},
 %booktitle = {The mathematical legacy of {H}arish-{C}handra ({B}altimore,{MD}, 1998)},
    journal = {The mathematical legacy of {H}arish-{C}handra, Proc. Sympos. Pure Math.},
    %series={The mathematical legacy of {H}arish-{C}handra, Proc. Sympos. Pure Math},
    volume = {68},
     pages = {97--110},
 publisher = {Amer. Math. Soc., Providence, RI},
      year = {2000},
}

\bib{B89}{article}{
  author = {Barbasch, D.},
  title = {The unitary dual for complex classical Lie groups},
  journal = {Invent. Math.},
  volume = {96},
  number = {1},
 pages = {103--176},
 year = {1989}
}


\bib{B10}{article}{
  author = {Barbasch, D.},
  title = {The unitary spherical spectrum for split classical groups},
  journal = {J. Inst. Math. Jussieu},
% number = {9},
 pages = {265--356},
 volume = {9},
 year = {2010}
}



\bib{B17}{article}{
  author = {Barbasch, D.},
  title = {Unipotent representations and the dual pair correspondence},
  journal = {J. Cogdell et al. (eds.), Representation Theory, Number Theory, and Invariant Theory, In Honor of Roger Howe. Progress in Math.},
  %series ={Progress in Math.},
  volume = {323},
  pages = {47--85},
  year = {2017},
}


\bib{Bo}{article}{
   author={Bo\v{z}i\v{c}evi\'{c}, Mladen},
   title={Double cells for unitary groups},
   journal={J. Algebra},
   volume={254},
   date={2002},
   number={1},
   pages={115--124},
   issn={0021-8693},
   review={\MR{1927434}},
   doi={10.1016/S0021-8693(02)00070-4},
}



\bib{BMSZ1}{article}{
      title={On the notion of metaplectic Barbasch-Vogan duality},
      year={2020},
      author={Barbasch, Dan M.},
      author = {Ma, Jia-jun},
      author = {Sun, Binyong},
      author = {Zhu, Chen-Bo},
      eprint={2010.16089},
      archivePrefix={arXiv},
      primaryClass={math.RT}
}

\bib{BMSZ2}{article}{
      title={Special unipotent representations of real classical groups: constructions and unitarity},
      author={Barbasch, Dan M.},
      author = {Ma, Jia-jun},
      author = {Sun, Binyong},
      author = {Zhu, Chen-Bo},
      year={2021},
      eprint={arXiv:1712.05552},
      archivePrefix={arXiv},
      primaryClass={math.RT}
}



\bib{BV1}{article}{
   author={Barbasch, Dan},
   author={Vogan, David},
   title={Primitive ideals and orbital integrals in complex classical
   groups},
   journal={Math. Ann.},
   volume={259},
   date={1982},
   number={2},
   pages={153--199},
   issn={0025-5831},
   review={\MR{656661}},
   doi={10.1007/BF01457308},
}

\bib{BV2}{article}{
   author={Barbasch, Dan},
   author={Vogan, David},
   title={Primitive ideals and orbital integrals in complex exceptional
   groups},
   journal={J. Algebra},
   volume={80},
   date={1983},
   number={2},
   pages={350--382},
   issn={0021-8693},
   review={\MR{691809}},
   doi={10.1016/0021-8693(83)90006-6},
}

\bib{BV.W}{article}{
  author={Barbasch, Dan},
  author={Vogan, David},
  editor={Trombi, P. C.},
  title={Weyl Group Representations and Nilpotent Orbits},
  bookTitle={Representation Theory of Reductive Groups:
    Proceedings of the University of Utah Conference 1982},
  year={1983},
  publisher={Birkh{\"a}user Boston},
  address={Boston, MA},
  pages={21--33},
  %doi={10.1007/978-1-4684-6730-7_2},
}


\bib{BVUni}{article}{
 author = {Barbasch, D.},
 author = {Vogan, D. A.},
 journal = {Annals of Math.},
 number = {1},
 pages = {41--110},
 title = {Unipotent representations of complex semisimple groups},
 volume = {121},
 year = {1985}
}

% \bib{BB}{article}{
%    author={Beilinson, Alexandre},
%    author={Bernstein, Joseph},
%    title={Localisation de $\mathfrak g$-modules},
%    language={French, with English summary},
%    journal={C. R. Acad. Sci. Paris S\'{e}r. I Math.},
%    volume={292},
%    date={1981},
%    number={1},
%    pages={15--18},
%    issn={0249-6291},
%    review={\MR{610137}},
% }

\bib{BK}{article}{
author={Borho, Walter},
author={Kraft, Hanspeter},
title={\"{U}ber die Gelfand-Kirillov-Dimension},
journal={Math. Ann.},
volume={220},
date={1976},
number={1},
pages={1--24},
issn={0025-5831},
review={\MR{412240}},
doi={10.1007/BF01354525},
}


% \bib{Br}{article}{
%   author = {Brylinski, R.},
%   title = {Dixmier algebras for classical complex nilpotent orbits via Kraft-Procesi models. I},
%   journal = {The orbit method in geometry and physics (Marseille, 2000). Progress in Math.}
%   volume = {213},
%   pages = {49--67},
%   year = {2003},
% }

\bib{BK}{article}{
   author={Brylinski, J.-L.},
   author={Kashiwara, M.},
   title={Kazhdan-Lusztig conjecture and holonomic systems},
   journal={Invent. Math.},
   volume={64},
   date={1981},
   number={3},
   pages={387--410},
   issn={0020-9910},
   review={\MR{632980}},
   doi={10.1007/BF01389272},
}

\bib{Carter}{book}{
   author={Carter, Roger W.},
   title={Finite groups of Lie type},
   series={Wiley Classics Library},
   %note={Conjugacy classes and complex characters;
   %Reprint of the 1985 original;
   %A Wiley-Interscience Publication},
   publisher={John Wiley \& Sons, Ltd., Chichester},
   date={1993},
   pages={xii+544},
   isbn={0-471-94109-3},
   %review={\MR{1266626}},
}

\bib{Cas}{article}{
   author={Casian, L. G.},
   title={Primitive ideals and representations},
   journal={J. Algebra},
   volume={101},
   date={1986},
   number={2},
   pages={497--515},
   issn={0021-8693},
   review={\MR{847174}},
   doi={10.1016/0021-8693(86)90208-5},
}

% \bib{Ca89}{article}{
%  author = {Casselman, W.},
%  journal = {Canad. J. Math.},
%  pages = {385--438},
%  title = {Canonical extensions of Harish-Chandra modules to representations of $G$},
%  volume = {41},
%  year = {1989}
% }



% \bib{Cl}{article}{
%   author = {Du Cloux, F.},
%   journal = {Ann. Sci. \'Ecole Norm. Sup.},
%   number = {3},
%   pages = {257--318},
%   title = {Sur les repr\'esentations diff\'erentiables des groupes de Lie alg\'ebriques},
%   url = {http://eudml.org/doc/82297},
%   volume = {24},
%   year = {1991},
% }

\bib{CM}{book}{
  title = {Nilpotent orbits in semisimple Lie algebra: an introduction},
  author = {Collingwood, D. H.},
  author = {McGovern, W. M.},
  year = {1993},
  publisher = {Van Nostrand Reinhold Co.},
}


% \bib{Dieu}{book}{
%    title={La g\'{e}om\'{e}trie des groupes classiques},
%    author={Dieudonn\'{e}, Jean},
%    year={1963},
% 	publisher={Springer},
%  }

% \bib{DKPC}{article}{
% title = {Nilpotent orbits and complex dual pairs},
% journal = {J. Algebra},
% volume = {190},
% number = {2},
% pages = {518 - 539},
% year = {1997},
% author = {Daszkiewicz, A.},
% author = {Kra\'skiewicz, W.},
% author = {Przebinda, T.},
% }

% \bib{DKP2}{article}{
%   author = {Daszkiewicz, A.},
%   author = {Kra\'skiewicz, W.},
%   author = {Przebinda, T.},
%   title = {Dual pairs and Kostant-Sekiguchi correspondence. II. Classification
% 	of nilpotent elements},
%   journal = {Central European J. Math.},
%   year = {2005},
%   volume = {3},
%   pages = {430--474},
% }


\bib{DM}{article}{
  author = {Dixmier, J.},
  author = {Malliavin, P.},
  title = {Factorisations de fonctions et de vecteurs ind\'efiniment diff\'erentiables},
  journal = {Bull. Sci. Math. (2)},
  year = {1978},
  volume = {102},
  pages = {307--330},
}

%\bibitem[DM]{DM}
%J. Dixmier and P. Malliavin, \textit{Factorisations de fonctions et de vecteurs ind\'efiniment diff\'erentiables}, Bull. Sci. Math. (2), 102 (4),  307-330 (1978).



\bib{Du77}{article}{
  author = {Duflo, M.},
  journal = {Annals of Math.},
  number = {1},
  pages = {107-120},
  title = {Sur la Classification des Ideaux Primitifs Dans
    L'algebre Enveloppante d'une Algebre de Lie Semi-Simple},
  volume = {105},
  year = {1977}
}

% \bib{Du82}{article}{
%  author = {Duflo, M.},
%  journal = {Acta Math.},
%   volume = {149},
%  number = {3-4},
%  pages = {153--213},
%  title = {Th\'eorie de Mackey pour les groupes de Lie alg\'ebriques},
%  year = {1982}
% }



% \bib{GZ}{article}{
% author={Gomez, R.},
% author={Zhu, C.-B.},
% title={Local theta lifting of generalized Whittaker models associated to nilpotent orbits},
% journal={Geom. Funct. Anal.},
% year={2014},
% volume={24},
% number={3},
% pages={796--853},
% }

% \bib{EGAIV2}{article}{
%   title = {\'El\'ements de g\'eom\'etrie alg\'brique IV: \'Etude locale des
%     sch\'emas et des morphismes de sch\'emas. II},
%   author = {Grothendieck, A.},
%   author = {Dieudonn\'e, J.},
%   journal  = {Inst. Hautes \'Etudes Sci. Publ. Math.},
%   volume = {24},
%   year = {1965},
% }


% \bib{EGAIV3}{article}{
%   title = {\'El\'ements de g\'eom\'etrie alg\'brique IV: \'Etude locale des
%     sch\'emas et des morphismes de sch\'emas. III},
%   author = {Grothendieck, A.},
%   author = {Dieudonn\'e, J.},
%   journal  = {Inst. Hautes \'Etudes Sci. Publ. Math.},
%   volume = {28},
%   year = {1966},
% }

\bib{GI}{article}{
   author={Gan, Wee Teck},
   author={Ichino, Atsushi},
   title={On the irreducibility of some induced representations of real
   reductive Lie groups},
   journal={Tunis. J. Math.},
   volume={1},
   date={2019},
   number={1},
   pages={73--107},
   issn={2576-7658},
   review={\MR{3907735}},
   doi={10.2140/tunis.2019.1.73},
}

% \bib{HLS}{article}{
%     author = {Harris, M.},
%     author = {Li, J.-S.},
%     author = {Sun, B.},
%      title = {Theta correspondences for close unitary groups},
%  %booktitle = {Arithmetic Geometry and Automorphic Forms},
%     %series = {Adv. Lect. Math. (ALM)},
%     journal = {Arithmetic Geometry and Automorphic Forms, Adv. Lect. Math. (ALM)},
%     volume = {19},
%      pages = {265--307},
%  publisher = {Int. Press, Somerville, MA},
%       year = {2011},
% }

% \bib{HS}{book}{
%  author = {Hartshorne, R.},
%  title = {Algebraic Geometry},
% publisher={Graduate Texts in Mathematics, 52. New York-Heidelberg-Berlin: Springer-Verlag},
% year={1983},
% }

% \bib{He}{article}{
% author={He, H.},
% title={Unipotent representations and quantum induction},
% journal={arXiv:math/0210372},
% year = {2002},
% }



\bib{Ho}{article}{
author={Hotta, R.},
title={On Joseph's construction of Weyl group representations},
journal={Tohoku Math. J.},
volume={36},
%number = {3},
pages={49--74 },
year={1984},
}




% \bib{Howe79}{article}{
%   title={$\Phi$-series and invariant theory},
%   author={Howe, R.},
%   book = {
%     title={Automorphic Forms, Representations and $L$-functions},
%     series={Proc. Sympos. Pure Math},
%     volume={33},
%     year={1979},
%   },
%   pages={275-285},
% }

% \bib{HoweRank}{article}{
% author={Howe, R.},
% title={On a notion of rank for unitary representations of the classical groups},
% journal={Harmonic analysis and group representations, Liguori, Naples},
% pages={223-331},
% year={1982},
% }

% \bib{Howe89}{article}{
% author={Howe, R.},
% title={Transcending classical invariant theory},
% journal={J. Amer. Math. Soc.},
% volume={2},
% pages={535--552},
% year={1989},
% }

% \bib{Howe95}{article}{,
%   author = {Howe, R.},
%   title = {Perspectives on invariant theory: Schur duality, multiplicity-free actions and beyond},
%   journal = {Piatetski-Shapiro, I. et al. (eds.), The Schur lectures (1992). Ramat-Gan: Bar-Ilan University, Isr. Math. Conf. Proc. 8,},
%   year = {1995},
%   pages = {1-182},
% }



% \bib{HL}{article}{
% author={Huang, J.-S.},
% author={Li, J.-S.},
% title={Unipotent representations attached to spherical nilpotent orbits},
% journal={Amer. J. Math.},
% volume={121},
% number = {3},
% pages={497--517},
% year={1999},
% }


% \bib{HZ}{article}{
% author={Huang, J.-S.},
% author={Zhu, C.-B.},
% title={On certain small representations of indefinite orthogonal groups},
% journal={Represent. Theory},
% volume={1},
% pages={190--206},
% year={1997},
% }


% \bib{H}{book}{
%    author={Humphreys, James E.},
%    title={Representations of semisimple Lie algebras in the BGG category
%    $\scr{O}$},
%    series={Graduate Studies in Mathematics},
%    volume={94},
%    publisher={American Mathematical Society, Providence, RI},
%    date={2008},
%    pages={xvi+289},
%    isbn={978-0-8218-4678-0},
%    review={\MR{2428237}},
%    doi={10.1090/gsm/094},
% }



\bib{Jan}{book}{
   author={Jantzen, J. C.},
   title={Moduln mit einem h\"{o}chsten Gewicht},
   series={Lecture notes in Mathematics},
   volume={750},
   publisher={Springer-Verlag, Berlin/Heidelberg/New York},
   date={1979},
  % pages={xvi+289},
   % isbn={978-0-8218-4678-0},
   % review={\MR{2428237}},
   % doi={10.1090/gsm/094},
}




\bib{J1}{article}{
   author={Joseph, A.},
   title={Goldie rank in the enveloping algebra of a semisimple Lie algebra. I},
   journal={J. Algebra},
   volume={65},
   date={1980},
   number={2},
   pages={269--283},
   issn={0021-8693},
   review={\MR{585721}},
   doi={10.1016/0021-8693(80)90217-3},
}

\bib{J2}{article}{
   author={Joseph, A.},
   title={Goldie rank in the enveloping algebra of a semisimple Lie algebra. II},
   journal={J. Algebra},
   volume={65},
   date={1980},
   number={2},
   pages={284--306},
   issn={0021-8693},
   review={\MR{585721}},
   doi={10.1016/0021-8693(80)90217-3},
}

% \bib{J3}{article}{
%    author={Joseph, A.},
%    title={Goldie rank in the enveloping algebra of a semisimple Lie algebra.
%    III},
%    journal={J. Algebra},
%    volume={73},
%    date={1981},
%    number={2},
%    pages={295--326},
%    issn={0021-8693},
%    review={\MR{640039}},
%    doi={10.1016/0021-8693(81)90324-0},
% }

\bib{J.hw}{article}{
   author={Joseph, A.},
   title={On the variety of a highest weight module},
   journal={J. Algebra},
   volume={88},
   date={1984},
   number={1},
   pages={238--278},
   issn={0021-8693},
   review={\MR{741942}},
   doi={10.1016/0021-8693(84)90100-5},
}


\bib{J.av}{article}{
   author={Joseph, A.},
   title={On the associated variety of a primitive ideal},
   journal={J. Algebra},
   volume={93},
   date={1985},
   number={2},
   pages={509--523},
   issn={0021-8693},
   review={\MR{786766}},
   doi={10.1016/0021-8693(85)90172-3},
}

% \bib{J.ann}{article}{
%    author={Joseph, Anthony},
%    title={Annihilators and associated varieties of unitary highest weight
%    modules},
%    journal={Ann. Sci. \'{E}cole Norm. Sup. (4)},
%    volume={25},
%    date={1992},
%    number={1},
%    pages={1--45},
%    issn={0012-9593},
%    review={\MR{1152612}},
% }


% \bib{JLS}{article}{
% author={Jiang, D.},
% author={Liu, B.},
% author={Savin, G.},
% title={Raising nilpotent orbits in wave-front sets},
% journal={Represent. Theory},
% volume={20},
% pages={419--450},
% year={2016},
% }

\bib{King}{article}{
author={King, D. R.},
title={The character polynomial of the annihilator of an irreducible Harish-Chandra module},
journal={Amer. J. Math.},
volume={103},
%issue ={4},
pages={1195--1240},
year={1981},
}


% \bib{Ki62}{article}{
% author={Kirillov, A. A.},
% title={Unitary representations of nilpotent Lie groups},
% journal={Uspehi Mat. Nauk},
% volume={17},
% issue ={4},
% pages={57--110},
% year={1962},
% }

% \bib{Ko70}{article}{
% author={Kostant, B.},
% title={Quantization and unitary representations},
% journal={Lectures in Modern Analysis and Applications III, Lecture Notes in Math.},
% volume={170},
% pages={87--208},
% year={1970},
% }


% \bib{KP}{article}{
% author={Kraft, H.},
% author={Procesi, C.},
% title={On the geometry of conjugacy classes in classical groups},
% journal={Comment. Math. Helv.},
% volume={57},
% pages={539--602},
% year={1982},
% }

% \bib{KR}{article}{
% author={Kudla, S. S.},
% author={Rallis, S.},
% title={Degenerate principal series and invariant distributions},
% journal={Israel J. Math.},
% volume={69},
% pages={25--45},
% year={1990},
% }


% \bib{Ku}{article}{
% author={Kudla, S. S.},
% title={Some extensions of the Siegel-Weil formula},
% journal={In: Gan W., Kudla S., Tschinkel Y. (eds) Eisenstein Series and Applications. Progress in Mathematics, vol 258. Birkh\"auser Boston},
% %volume={69},
% pages={205--237},
% year={2008},
% }





% \bib{LZ1}{article}{
% author={Lee, S. T.},
% author={Zhu, C.-B.},
% title={Degenerate principal series and local theta correspondence II},
% journal={Israel J. Math.},
% volume={100},
% pages={29--59},
% year={1997},
% }

% \bib{LZ2}{article}{
% author={Lee, S. T.},
% author={Zhu, C.-B.},
% title={Degenerate principal series of metaplectic groups and Howe correspondence},
% journal = {D. Prasad at al. (eds.), Automorphic Representations and L-Functions, Tata Institute of Fundamental Research, India,},
% year = {2013},
% pages = {379--408},
% }

% \bib{Li89}{article}{
% author={Li, J.-S.},
% title={Singular unitary representations of classical groups},
% journal={Invent. Math.},
% volume={97},
% number = {2},
% pages={237--255},
% year={1989},
% }

% \bib{LiuAG}{book}{
%   title={Algebraic Geometry and Arithmetic Curves},
%   author = {Liu, Q.},
%   year = {2006},
%   publisher={Oxford University Press},
% }

% \bib{LM}{article}{
%    author = {Loke, H. Y.},
%    author = {Ma, J.},
%     title = {Invariants and $K$-spectrums of local theta lifts},
%     journal = {Compositio Math.},
%     volume = {151},
%     issue = {01},
%     year = {2015},
%     pages ={179--206},
% }

% \bib{DL}{article}{
%    author={Deligne, P.},
%    author={Lusztig, G.},
%    title={Representations of reductive groups over finite fields},
%    journal={Ann. of Math. (2)},
%    volume={103},
%    date={1976},
%    number={1},
%    pages={103--161},
%    issn={0003-486X},
%    review={\MR{393266}},
%    doi={10.2307/1971021},
% }

% \bib{KL}{article}{
%    author={Kazhdan, David},
%    author={Lusztig, George},
%    title={Representations of Coxeter groups and Hecke algebras},
%    journal={Invent. Math.},
%    volume={53},
%    date={1979},
%    number={2},
%    pages={165--184},
%    issn={0020-9910},
%    review={\MR{560412}},
%    doi={10.1007/BF01390031},
% }

\bib{Lu}{book}{
   author={Lusztig, George},
   title={Characters of reductive groups over a finite field},
   series={Annals of Mathematics Studies},
   volume={107},
   publisher={Princeton University Press, Princeton, NJ},
   date={1984},
   pages={xxi+384},
   isbn={0-691-08350-9},
   isbn={0-691-08351-7},
   review={\MR{742472}},
   doi={10.1515/9781400881772},
}


% \bib{Lu.I}{article}{
%    author={Lusztig, G.},
%    title={Intersection cohomology complexes on a reductive group},
%    journal={Invent. Math.},
%    volume={75},
%    date={1984},
%    number={2},
%    pages={205--272},
%    issn={0020-9910},
%    review={\MR{732546}},
%    doi={10.1007/BF01388564},
% }


% \bib{LS}{article}{
%    author = {Lusztig, G.},
%    author = {Spaltenstein, N.},
%     title = {Induced unipotent classes},
%     journal = {J. London Math. Soc.},
%     volume = {19},
%     year = {1979},
%     pages ={41--52},
% }


% \bib{Ma}{article}{
%    author = {Mackey, G. W.},
%     title = {Unitary representations of group extentions},
%     journal = {Acta Math.},
%     volume = {99},
%     year = {1958},
%     pages ={265--311},
% }

\bib{Mat96}{article}{
   author={Matumoto, Hisayosi},
   title={On the representations of ${\rm U}(m,n)$ unitarily induced from
   derived functor modules},
   journal={Compositio Math.},
   volume={100},
   date={1996},
   number={1},
   pages={1--39},
   issn={0010-437X},
   review={\MR{1377407}},
}

\bib{Mat}{article}{
   author={Matumoto, Hisayosi},
   title={On the representations of ${\rm Sp}(p,q)$ and ${\rm SO}^*(2n)$
   unitarily induced from derived functor modules},
   journal={Compos. Math.},
   volume={140},
   date={2004},
   number={4},
   pages={1059--1096},
   issn={0010-437X},
   review={\MR{2059231}},
   doi={10.1112/S0010437X03000629},
}

\bib{Mc}{article}{
   author = {McGovern, W. M},
    title = {Cells of Harish-Chandra modules for real classical groups},
    journal = {Amer. J.  of Math.},
    volume = {120},
    issue = {01},
    year = {1998},
    pages ={211--228},
}


\bib{Mil}{article}{
   author = {Mili\v{c}i\'c, D.},
    title = {Localizations and representation theory of reductive Lie groups},
    journal = {preprint, http://www.math.utah.edu/~milicic/Eprints/book.pdf},
   % volume = {120},
    %issue = {01},
    %year = {1998},
   % pages ={211--228},
}


% \bib{Mo96}{article}{
%  author={M{\oe}glin, C.},
%     title = {Front d'onde des repr\'esentations des groupes classiques $p$-adiques},
%     journal = {Amer. J. Math.},
%     volume = {118},
%     issue = {06},
%     year = {1996},
%     pages ={1313--1346},
% }

\bib{Mo17}{article}{
  author={M{\oe}glin, C.},
  title = {Paquets d'Arthur Sp\'eciaux Unipotents aux Places Archim\'ediennes et Correspondance de Howe},
  journal = {J. Cogdell et al. (eds.), Representation Theory, Number Theory, and Invariant Theory, In Honor of Roger Howe. Progress in Math.}
  %series ={Progress in Math.},
  volume = {323},
  pages = {469--502}
  year = {2017}
}

\bib{MR.C}{article}{
   author={Moeglin, Colette},
   author={Renard, David},
   title={Paquets d'Arthur des groupes classiques complexes},
   language={French, with English and French summaries},
   conference={
      title={Around Langlands correspondences},
   },
   book={
      series={Contemp. Math.},
      volume={691},
      publisher={Amer. Math. Soc., Providence, RI},
   },
   date={2017},
   pages={203--256},
   review={\MR{3666056}},
   doi={10.1090/conm/691/13899},
}

\bib{MR.U}{article}{
   author={M\oe glin, Colette},
   author={Renard, David},
   title={Sur les paquets d'Arthur des groupes unitaires et quelques
   cons\'{e}quences pour les groupes classiques},
   language={French, with English and French summaries},
   journal={Pacific J. Math.},
   volume={299},
   date={2019},
   number={1},
   pages={53--88},
   issn={0030-8730},
   review={\MR{3947270}},
   doi={10.2140/pjm.2019.299.53},
}


% \bib{MVW}{book}{
%   volume={1291},
%   title={Correspondances de Howe sur un corps $p$-adique},
%   author={M{\oe}glin, C.},
%   author={Vign\'eras, M.-F.},
%   author={Waldspurger, J.-L.},
%   series={Lecture Notes in Mathematics},
%   publisher={Springer}
%   ISBN={978-3-540-18699-1},
%   date={1987},
% }

% \bib{NOTYK}{article}{
%    author = {Nishiyama, K.},
%    author = {Ochiai, H.},
%    author = {Taniguchi, K.},
%    author = {Yamashita, H.},
%    author = {Kato, S.},
%     title = {Nilpotent orbits, associated cycles and Whittaker models for highest weight representations},
%     journal = {Ast\'erisque},
%     volume = {273},
%     year = {2001},
%    pages ={1--163},
% }

% \bib{NOZ}{article}{
%   author = {Nishiyama, K.},
%   author = {Ochiai, H.},
%   author = {Zhu, C.-B.},
%   journal = {Trans. Amer. Math. Soc.},
%   title = {Theta lifting of nilpotent orbits for symmetric pairs},
%   volume = {358},
%   year = {2006},
%   pages = {2713--2734},
% }


% \bib{NZ}{article}{
%    author = {Nishiyama, K.},
%    author = {Zhu, C.-B.},
%     title = {Theta lifting of unitary lowest weight modules and their associated cycles},
%     journal = {Duke Math. J.},
%     volume = {125},
%     number= {03},
%     year = {2004},
%    pages ={415--465},
% }



% \bib{Ohta}{article}{
%   author = {Ohta, T.},
%   %doi = {10.2748/tmj/1178227492},
%   journal = {Tohoku Math. J.},
%   number = {2},
%   pages = {161--211},
%   publisher = {Tohoku University, Mathematical Institute},
%   title = {The closures of nilpotent orbits in the classical symmetric
%     pairs and their singularities},
%   volume = {43},
%   year = {1991}
% }

% \bib{Ohta2}{article}{
%   author = {Ohta, T.},
%   journal = {Hiroshima Math. J.},
%   number = {2},
%   pages = {347--360},
%   title = {Induction of nilpotent orbits for real reductive groups and associated varieties of standard representations},
%   volume = {29},
%   year = {1999}
% }

% \bib{Ohta4}{article}{
%   title={Nilpotent orbits of $\mathbb{Z}_4$-graded Lie algebra and geometry of
%     moment maps associated to the dual pair $(\mathrm{U} (p, q), \mathrm{U} (r, s))$},
%   author={Ohta, T.},
%   journal={Publ. RIMS},
%   volume={41},
%   number={3},
%   pages={723--756},
%   year={2005}
% }

% \bib{PT}{article}{
%   title={Some small unipotent representations of indefinite orthogonal groups and the theta correspondence},
%   author={Paul, A.},
%   author={Trapa, P.},
%   journal={University of Aarhus Publ. Series},
%   volume={48},
%   pages={103--125},
%   year={2007}
% }


% \bib{PV}{article}{
%   title={Invariant Theory},
%   author={Popov, V. L.},
%   author={Vinberg, E. B.},
%   book={
%   title={Algebraic Geometry IV: Linear Algebraic Groups, Invariant Theory},
%   series={Encyclopedia of Mathematical Sciences},
%   volume={55},
%   year={1994},
%   publisher={Springer},}
% }




%\bib{PPz}{article}{
%author={Protsak, V.} ,
%author={Przebinda, T.},
%title={On the occurrence of admissible representations in the real Howe
%    correspondence in stable range},
%journal={Manuscr. Math.},
%volume={126},
%number={2},
%pages={135--141},
%year={2008}
%}


% \bib{PrzInf}{article}{
%       author={Przebinda, T.},
%        title={The duality correspondence of infinitesimal characters},
%         date={1996},
%      journal={Colloq. Math.},
%       volume={70},
%        pages={93--102},
% }


% \bib{Pz}{article}{
% author={Przebinda, T.},
% title={Characters, dual pairs, and unitary representations},
% journal={Duke Math. J. },
% volume={69},
% number={3},
% pages={547--592},
% year={1993}
% }

% \bib{Ra}{article}{
% author={Rallis, S.},
% title={On the Howe duality conjecture},
% journal={Compositio Math.},
% volume={51},
% pages={333--399},
% year={1984}
% }

\bib{RT1}{article}{
   author={Renard, David A.},
   author={Trapa, Peter E.},
   title={Irreducible genuine characters of the metaplectic group:
   Kazhdan-Lusztig algorithm and Vogan duality},
   journal={Represent. Theory},
   volume={4},
   date={2000},
   pages={245--295},
   review={\MR{1795754}},
   doi={10.1090/S1088-4165-00-00105-9},
}

\bib{RT2}{article}{
   author={Renard, David A.},
   author={Trapa, Peter E.},
   title={Irreducible characters of the metaplectic group. II.
   Functoriality},
   journal={J. Reine Angew. Math.},
   volume={557},
   date={2003},
   pages={121--158},
   issn={0075-4102},
   review={\MR{1978405}},
   doi={10.1515/crll.2003.028},
}

% \bib{RT3}{article}{
%    author={Renard, David A.},
%    author={Trapa, Peter E.},
%    title={Kazhdan-Lusztig algorithms for nonlinear groups and applications
%    to Kazhdan-Patterson lifting},
%    journal={Amer. J. Math.},
%    volume={127},
%    date={2005},
%    number={5},
%    pages={911--971},
%    issn={0002-9327},
%    review={\MR{2170136}},
% }


% \bib{Sa}{article}{
% author={Sahi, S.},
% title={Explicit Hilbert spaces for certain unipotent representations},
% journal={Invent. Math.},
% volume={110},
% number = {2},
% pages={409--418},
% year={1992}
% }

% \bib{Se}{article}{
% author={Sekiguchi, J.},
% title={Remarks on real nilpotent orbits of a symmetric pair},
% journal={J. Math. Soc. Japan},
% %publisher={The Mathematical Society of Japan},
% year={1987},
% volume={39},
% number={1},
% pages={127--138},
% }

\bib{SV}{article}{
author = {Schmid, W.},
author = {Vilonen, K.},
journal = {Annals of Math.},
number = {3},
pages = {1071--1118},
%publisher = {Princeton University, Mathematics Department, Princeton, NJ; Mathematical Sciences Publishers, Berkeley},
title = {Characteristic cycles and wave front cycles of representations of reductive Lie groups},
volume = {151},
year = {2000},
}


\bib{Soergel}{article}{
   author={Soergel, Wolfgang},
   title={Kategorie $\scr O$, perverse Garben und Moduln \"{u}ber den
   Koinvarianten zur Weylgruppe},
   language={German, with English summary},
   journal={J. Amer. Math. Soc.},
   volume={3},
   date={1990},
   number={2},
   pages={421--445},
   issn={0894-0347},
   review={\MR{1029692}},
   doi={10.2307/1990960},
}


\bib{So}{article}{
author = {Sommers, E.},
title = {Lusztig's canonical quotient and generalized duality},
journal = {J. Algebra},
volume = {243},
number = {2},
pages = {790--812},
year = {2001},
}

% \bib{SS}{book}{
%   author = {Springer, T. A.},
%   author = {Steinberg, R.},
%   title = {Seminar on algebraic groups and related finite groups; Conjugate classes},
%   series = {Lecture Notes in Math.},
%   volume = {131},
% publisher={Springer},
% year={1970},
% }

% \bib{SZ1}{article}{
% title={A general form of Gelfand-Kazhdan criterion},
% author={Sun, B.},
% author={Zhu, C.-B.},
% journal={Manuscripta Math.},
% pages = {185--197},
% volume = {136},
% year={2011}
% }


%\bib{SZ2}{article}{
%  title={Conservation relations for local theta correspondence},
%  author={Sun, B.},
%  author={Zhu, C.-B.},
%  journal={J. Amer. Math. Soc.},
%  pages = {939--983},
%  volume = {28},
%  year={2015}
%}

\bib{Tr.U}{article}{
   author={Trapa, Peter E.},
   title={Annihilators and associated varieties of $A_{\mathfrak q}(\lambda)$
   modules for $\mathrm U(p,q)$},
   journal={Compositio Math.},
   volume={129},
   date={2001},
   number={1},
   pages={1--45},
   issn={0010-437X},
   review={\MR{1856021}},
   doi={10.1023/A:1013115223377},
}

\bib{Tr.H}{article}{
  title={Special unipotent representations and the Howe correspondence},
  author={Trapa, P.},
  year = {2004},
  journal={University of Aarhus Publication Series},
  volume = {47},
  pages= {210--230}
}

% \bib{Wa}{article}{
%    author = {Waldspurger, J.-L.},
%     title = {D\'{e}monstration d'une conjecture de dualit\'{e} de Howe dans le cas $p$-adique, $p \neq 2$ in Festschrift in honor of I. I. Piatetski-Shapiro on the occasion of his sixtieth birthday},
%   journal = {Israel Math. Conf. Proc., 2, Weizmann, Jerusalem},
%  year = {1990},
% pages = {267-324},
% }

\bib{VGK}{article}{
   author={Vogan, David A., Jr.},
   title={Gel\cprime fand-Kirillov dimension for Harish-Chandra modules},
   journal={Invent. Math.},
   volume={48},
   date={1978},
   number={1},
   pages={75--98},
   issn={0020-9910},
   review={\MR{506503}},
   doi={10.1007/BF01390063},
}



\bib{Vg}{book}{
   author={Vogan, David A.},
   title={Representations of real reductive Lie groups},
   series={Progress in Mathematics},
   volume={15},
   publisher={Birkh\"{a}user, Boston, Mass.},
   date={1981},
   pages={xvii+754},
   isbn={3-7643-3037-6},
   review={\MR{632407}},
}

\bib{V3}{article}{
   author={Vogan, D. A. },
   title={Irreducible characters of semisimple Lie groups. III. Proof of Kazhdan-Lusztig conjecture in the integral case},
   journal={Invent. Math.},
   volume={71},
   date={1983},
   number={2},
   pages={381--417},
}

\bib{V4}{article}{
   author={Vogan, D. A. },
   title={Irreducible characters of semisimple Lie groups. IV.
   Character-multiplicity duality},
   journal={Duke Math. J.},
   volume={49},
   date={1982},
   number={4},
   pages={943--1073},
   issn={0012-7094},
   review={\MR{683010}},
}


\bib{V.GL}{article}{
   author={Vogan, D. A.},
   title={The unitary dual of ${\rm GL}(n)$ over an Archimedean field},
   journal={Invent. Math.},
   volume={83},
   date={1986},
   number={3},
   pages={449--505},
   issn={0020-9910},
   review={\MR{827363}},
   doi={10.1007/BF01394418},
}

\bib{VoBook}{book}{
author = {Vogan, D. A. },
  title={Unitary representations of reductive Lie groups},
  year={1987},
  series = {Ann. of Math. Stud.},
 volume={118},
  publisher={Princeton University Press}
}


\bib{Vo89}{article}{
  author = {Vogan, D. A. },
  title = {Associated varieties and unipotent representations},
 %booktitle ={Harmonic analysis on reductive groups, Proc. Conf., Brunswick/ME (USA) 1989,},
  journal = {Harmonic analysis on reductive groups, Proc. Conf., Brunswick/ME
    (USA) 1989, Prog. Math.},
 volume={101},
  publisher = {Birkh\"{a}user, Boston-Basel-Berlin},
  year = {1991},
pages={315--388},
  editor = {W. Barker and P. Sally},
}

% \bib{Vo98}{article}{
%   author = {Vogan, D. A. },
%   title = {The method of coadjoint orbits for real reductive groups},
%  %booktitle ={Representation theory of Lie groups (Park City, UT, 1998)},
%  journal = {Representation theory of Lie groups (Park City, UT, 1998). IAS/Park City Math. Ser.},
%   volume={8},
%   publisher = {Amer. Math. Soc.},
%   year = {2000},
% pages={179--238},
% }




% \bib{Vo00}{article}{
%   author = {Vogan, D. A. },
%   title = {Unitary representations of reductive Lie groups},
%  %booktitle ={Mathematics towards the Third Millennium (Rome, 1999)},
%  journal ={Mathematics towards the Third Millennium (Rome, 1999). Accademia Nazionale dei Lincei, (2000)},
%   %series = {Accademia Nazionale dei Lincei, 2000},
%  %volume={9},
% pages={147--167},
% }


% \bib{Wa1}{book}{
%   title={Real reductive groups I},
%   author={Wallach, N. R.},
%   year={1988},
%   publisher={Academic Press Inc. }
% }

% \bib{Wa2}{book}{
%   title={Real reductive groups II},
%   author={Wallach, N. R.},
%   year={1992},
%   publisher={Academic Press Inc. }
% }


% \bib{Weyl}{book}{
%   title={The classical groups: their invariants and representations},
%   author={Weyl, H.},
%   year={1947},
%   publisher={Princeton University Press}
% }

% \bib{Ya}{article}{
%   title={Degenerate principal series representations for quaternionic unitary groups},
%   author={Yamana, S.},
%   year = {2011},
%   journal={Israel J. Math.},
%   volume = {185},
%   pages= {77--124}
% }


\bib{Zu}{article}{
  title={Tensor products of finite and infinite dimensional representations of semisimple Lie groups},
  author={Zuckerman, G.},
  year = {1977},
  journal={Ann. of Math. 106},
  volume = {106},
  pages= {295--308}
}


% \bib{EGAIV4}{article}{
%   title = {\'El\'ements de g\'eom\'etrie alg\'brique IV 4: \'Etude locale des
%     sch\'emas et des morphismes de sch\'emas},
%   author = {Grothendieck, Alexandre},
%   author = {Dieudonn\'e, Jean},
%   journal  = {Inst. Hautes \'Etudes Sci. Publ. Math.},
%   volume = {32},
%   year = {1967},
%   pages = {5--361}
% }



\end{biblist}
\end{bibdiv}


\end{document}


%%% Local Variables:
%%% coding: utf-8
%%% mode: latex
%%% TeX-engine: tex
%%% ispell-local-dictionary: "en_US"
%%% End:
