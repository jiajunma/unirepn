% !TEX root = counting_main.tex
\documentclass[counting_main.tex]{subfiles}

\begin{document}


\section{Introduction and the main results}

Let $G$ be a real reductive group in the Harish-Chandra class (which may be
linear or non-linear). Write $\fgg$ for the complexified Lie algebra of $G$ and
let $\hha$ denote its universal Cartan subalgebra (also called the abstract Cartan subalgbra in \cite{V4}).
Let $\lambda \in \hha^*$ (a superscript $*$ indicates the dual space). By Harish-Chandra isomorphism, it
determines an algebraic character $\chi_\lambda: \CZ(\g)\rightarrow \C$. Here
$\CZ(\g)$ denotes the center of the universal enveloping algebra
$\mathcal U(\g)$. Denote by $\Irr(G)$ the set of isomorphism classes of
irreducible Casselman-Wallach representations of $G$, and by $\Irr_\lambda(G)$
its subset consisting of the representations with infinitesimal character
$\chi_\lambda$ (or simply $\lambda$). The latter set has finite cardinality.


Let $\Nil(\g^*)$ denote the set of nilpotent elements in $\g^*$. It has only
finitely many orbits under the coadjoint action of the inner automorphism group
$\mathrm{Inn}(\g)$ of $\g$. Let $\sfS$ be an $\mathrm{Inn}(\g)$-stable Zariski
closed subset of $\mathrm{Nil}(\g^*)$. Put
\[
  \Irr_{\lambda,\sfS}(G):=\Set{\pi \in \Irr_{\lambda}(G)| \text{$\AVC(\pi)\subset \sfS$} }.
\]
Here $\AVC(\pi)$ denotes the complex associated variety of $\pi$, namely the
associated variety of the annihilate ideal of $\pi$. It is an
$\mathrm{Inn}(\g)$-stable Zariski closed subset of $\mathrm{Nil}(\g^*)$. An
interesting problem of representation theory is to count the finite set $\Irr_{\lambda,\sfS}(G)$. The coherent continuation
representation (of an appropriate Weyl group) provides a powerful tool for this problem.

\subsection{The coherent continuation representation}


Consider the category of Cassleman-Wallach representations of $G$ that have
generalized infinitesimal character $\lambda$ and whose complex associated
variety is contained in $\sfS$. Write $\CK_{\lambda,\sfS}(G)$ for the
Grothendieck group with $\C$-coefficients of this category. Then
\[
  \sharp (\Irr_{\lambda,\sfS}(G))=\dim \CK_{\lambda,\sfS}(G)\qquad(\sharp\textrm{
    indicates the cardinality of a finite set}).
\]
We also have that
\[
  \CK_\sfS(G)=\bigoplus_{\mu\in W\backslash \h^*} \CK_{\mu,\sfS}(G)\qquad (W\textrm{
    denotes the Weyl group}),
\]
where $\CK_{\sfS}(G)$ is the Grothendieck group, with $\C$-coefficients, of the
category of Casselman-Wallach representations of $G$ whose complex associated
variety is contained in $\sfS$.


Let $\Rg$ be the Grothendieck group with
$\C$-coefficients of the category of finite-dimensional algebraic
representations of $\mathrm{Inn}(\fgg)$. It is
 a commutative $\bC$-algebra under the tensor
product of representations.
Write \[
\Delta\subset Q \quad (\subset \hha^*)
\] for the root system and the root lattice of
$\fgg$, respectively.
%and identified with $\bC[Q/W]$.
By pulling back through the adjoint representation
$G\rightarrow \mathrm{Inn}(\g)$, every algebraic representation of $\mathrm{Inn}(\g)$ is viewed as a representation of $G$.
Under the tensor product of representations, $\CK_\sfS(G)$ is naturally a $\Rg$-module.


Put
\[
\Lam:=\lambda+Q\subset \hha^*,
\]
 and write $W_\Lam$
for its stabilizer in $W$. Then $W_\Lam$ equals the Weyl group of the root
system
\[
  \{\alpha \in \Delta\mid \langle \lambda, \alpha^\vee\rangle \in \Z\}\qquad (\alpha^\vee \textrm{ denotes the coroot corresponding to $\alpha$}).
\]


\begin{defn}\label{defcoh}
  Let $\CK$ be a $\mathcal R(\g)$-module equipped with a family
  $\{\CK_\mu\}_{\mu\in \Lam}$ of subspaces such that $\CK_{w\cdot \mu}=\CK_\mu$
  for all $w\in W_\Lam$ and $\mu\in \Lam$. A $\CK$-valued coherent family on
  $\Lam$ is a map
  \[
    \Theta: \Lam\rightarrow \CK%, \qquad \mu\mapsto \Theta_\mu
  \]
  satisfying the following two conditions:
  \begin{itemize}
    \item for all $\mu\in \Lam$, $\Theta(\mu)\in \CK_\mu$;
    \item for all finite-dimensional algebraic representations $F$ of $\mathrm{Inn}(\g)$
          and all $\mu\in \Lam$,
          \[
          F\cdot (\Theta(\mu)) = \sum_{\nu} \Theta(\mu+\nu),
          \]
          where $\nu$ runs over all weights of $F$, counted with multiplicities,
          and $F$ is viewed as an element of $\mathcal R(\g)$.
  \end{itemize}
\end{defn}


In the notation of \Cref{defcoh}, let $\Coh_{\Lam}(\CK)$ denote the
vector space of all $\mathcal K$-valued coherent families on $\Lam$. It is a
representation of $W_{\Lam}$ under the action
\[
  (w\cdot \Theta)(\mu) = \Theta(w^{-1}\cdot \mu), \qquad \textrm{for all
  }\ w\in W_\Lam, \ \mu\in \Lam.
\]
To shorten the notation, put
\[
  \Coh_{\Lam,\sfS}(G):=\Coh_{\Lam}(\CK_\sfS(G)).
\]
Here the $\Rg$-module $\CK_\sfS(G)$ is equipped with the family $\{\CK_{\mu, \sfS}(G)\}_{\mu\in \Lam}$ of subspaces.

\subsection{Counting irreducible representations with a bounded complex
  associated variety}

Denote by $W_\lambda$ the stabilizer of
$\lambda$ in $W$. Then $W_\lambda\subset W_\Lam$. Write $1_{W_\lambda}$ for the
trivial representation of $W_{\lambda}$.

Our starting point is the following theorem of Vogan. We will provide a proof due to lack of a convenient reference.
\begin{thm}[Vogan]\label{count1}
  The equality
  \[
    \sharp(\Irr_{\lambda,\sfS}(G)) = [1_{W_{\lambda}}:\Coh_{\Lam,\sfS}(G)]
  \]
  holds.
  % \[
  %   \dim {\barmu} = \dim (\cohm)_{W_\mu} = [\cohm, 1_{W_\mu}].
  % \]
\end{thm}
Here and henceforth, $[\ : \ ]$ indicates the multiplicity of the first
(irreducible) representation in the second one. Theorem \ref{count1} implies
that
\begin{equation}\label{countlg}
  \sharp(\Irr_{\lambda,\sfS}(G)) = \sum_{\sigma\in \Irr(W_\Lam)} [1_{W_{\lambda}}: \sigma]\cdot [\sigma: \Coh_{\Lam,\sfS}(G)].
\end{equation}
Thus it suffices to understand the multiplicity $ [\sigma: \Coh_{\Lam,\sfS}(G)]$
for every $\sigma\in \Irr(W_\Lam)$.

Let $\sigma\in \Irr(W_\Lam)$. Define the
nilpotent orbit
\[
  \CO_\sigma:=\mathrm{Springer}(j_{W_\Lam}^W \sigma_0)\subset \mathrm{Nil}(\g^*) \quad (``\mathrm{Springer}"\textrm{
    indicates the Springer correspondence}),
\]
where $\sigma_0$ denote the special irreducible representation of $W_\Lam$ that
lies in the same double cell as $\sigma$, and $j_{W_\Lam}^W \sigma_0$ denotes
the $j$-induction of $\sigma _0$, which is an irreducible representation of $W$. See \cite[Chapter 11]{Carter} for more details on the $j$-induction.


Let
\[
  \sfC_{\sfS}:= \Set{\sigma\in \Irr(\WLam)| \cO_{\sigma}\subseteq \sfS}.
\]

\begin{prop}[\Cref{cor:coh.HC}]\label{count2}
  Suppose that $\sigma\in \Irr(W_\Lam)$ and $\sigma \notin \sfC_{\sfS}$. Then
  %$\CO_\sigma\nsubset \sfS$.
  \[
    [\sigma: \Coh_{\Lam,\sfS}(G)]=0.
  \]

  % \[
  %   \dim {\barmu} = \dim (\cohm)_{W_\mu} = [\cohm, 1_{W_\mu}].
  % \]
\end{prop}


Combining Theorem \ref{count1} and Proposition \ref{count2}, we conclude that
\begin{equation}\label{leq2}
  \sharp(\Irr_{\lambda,\sfS}(G)) = \sum_{\sigma \in \sfC_{\sfS}} [1_{W_{\lambda}}: \sigma]\cdot [\sigma: \Coh_{\Lam,\sfS}(G)].
\end{equation}

Write
\[
  \Coh_{\Lam}(G):= \Coh_{\Lam,\mathrm{Nil}(\g^*)}(G),
\]
which contains $ \Coh_{\Lam,\sfS}(G)$ as a subrepresentation. It is obvious that for all $\sigma\in \Irr(W_\Lam)$,
\begin{equation}\label{leq1}
  [\sigma: \Coh_{\Lam,\sfS}(G)]\leq [\sigma: \Coh_{\Lam}(G)].
\end{equation}
We expect that if $\sigma \in \sfC_{\sfS}$, then
\begin{equation}\label{leq1}
[\sigma: \Coh_{\Lam,\sfS}(G)]= [\sigma: \Coh_{\Lam}(G)].
\end{equation}
See Remark \ref{r46}.

% the equality always holds. Combining Theorem \ref{count1}, Theorem
% \ref{count2} and \eqref{leq1}, we conclude that
% \begin{equation}\label{leq2}
%   \sharp(\Irr_{\lambda,\sfS}(G)) \leq \sum_{\sigma\in \Irr(W_\Lam), \CO_\sigma\subset \sfS} [1_{W_{\lambda}}: \sigma]\cdot [\sigma: \Coh_{\Lam}(G)].
% \end{equation}





\subsection{Counting irreducible representations annihilated by a maximal primitive ideal}
Write $I_\lambda$ for the maximal ideal of $\mathcal U(\g)$ with infinitesimal
character $\lambda$. Its associated variety equals the Zariski closure
$\overline{\CO_\lambda}$ of an $\mathrm{Inn}(\g)$-orbit
$\CO_\lambda\subset\mathrm{Nil}(\g^*) $. Note that an irreducible
Casselman-Wallach representation of $G$ lies in
$\Irr_{\lambda,\overline{\CO_\lambda}}(G)$ if and only if it is annihilated by
$I_\lambda$.


Let
\[
  \LC_{\lambda}:= \Set{\sigma\in \Irr(W_\Lam)| \sigma \text{ occurs in $(J_{W_{\lambda}}^{W_{\Lam}} \sgn )\otimes \sgn$}}.
\]
Here $J_{W_{\lambda}}^{W_{\Lam}} $ indicates the $J$-induction (see \cite[Chapter 12]{Carter}), and $\sgn$
denotes the sign character (of an appropriate Weyl group).

%Let $\LC_{\lambda}\subset \Irr(W_\Lam)$ be the subset consisting of all the
%irreducible representations that occur in
%\[
%  (J_{W_{\lambda}}^{W_{\Lam}} \sgn )\otimes \sgn,
%\]
%where $J_{W_{\lambda}}^{W_{\Lam}} $ indicates the $J$-induction (see \cite[Chapter 12]{Carter}), and $\sgn$
%denotes the sign character (of an appropriate Weyl group).



 \begin{prop}[{\cite{BVUni}*{(5.26), Proposition~5.28}}]\label{lem:lcell.BV0}
   We have the equality
   \[
     \LC_{\lambda} = \Set{\sigma\in \Irr(W_\Lam)\mid \CO_\sigma\subset \overline{\CO_\lambda}, \ [1_{W_{\lambda}}:\sigma]\neq 0}.
   \]
   Moreover, the multiplicity $[1_{W_{\lambda}}:\sigma]$ is one when
   $\sigma\in \LC_\lambda$.
 \end{prop}

 Combining \eqref{countlg}, Propositions \ref{count2} and \ref{lem:lcell.BV0}, we obtain the following inequality. (The equality is also expected).

 \begin{cor}
   % Under the notation of \Cref{lem:lcell.BV}, we have
   The inequality
   \begin{equation}\label{boundc}
     \sharp(\Irr_{\lambda,\overline{\CO_\lambda}}(G)) \leq \sum_{\sigma\in \LC_\lambda} [\sigma: \Coh_{\Lam}(G)]
   \end{equation}
   holds.
 \end{cor}



 \subsection{Special unipotent representations of classical groups}

 We are particularly interested in counting special unipotent representations of
 real classical groups.

 Let $\star$ be one of the 14 symbols
 \[
   \textrm{ $A$, $A^\C$, $A^\bH$, $A^*$, $B$, $D$, $B^\C$, $D^\C$, $C$, $C^\C$,
     $D^*$, $C^*$, $\wtC$, $\wtC^\C$. }
 \]
 Suppose that $G$ is a classical Lie group of type $\star$, namely $G$
 respectively equals one of the following Lie groups:
 \[
   \begin{array}{c}
     \GL_n(\R), \ \GL_n(\C), \  \GL_n(\bH),\  \oU(p,q),\smallskip\\
     \SO(p,q)\ (p+q\, \textrm{ is odd}),  \  \SO(p,q)\  (p+q\, \textrm{ is even}),\smallskip\\
     \SO_n(\C) \ (n\, \textrm{ is odd}),  \
     \SO_n(\C) \ (n\, \textrm{ is even}),\smallskip \\
     \Sp_{2n}(\R), \ \Sp_{2n}(\C), \  \oO^*(2n), \  \Sp(p,q),\   \widetilde \Sp_{2n}(\R), \ \Sp_{2n}(\C) \qquad (n, p, q\geq 0).
   \end{array}
 \]
 Here $\wtSp_{2n}(\R)$ denotes the metaplectic double cover of the symplectic
 group $\Sp_{2n}(\R)$ that does not split unless $n=0$. As usual, the universal
 Cartan subalgebra $\hha$ of the complexified Lie algebra $\g$ of $G$ is
 respectively identified with
 \[
   \begin{array}{c}
     \C^n, \ \C^n\times \C^n, \ \C^{2n},  \ \C^{p+q},\smallskip \\
     \C^{\frac{p+q-1}{2}},\ \C^{\frac{p+q}{2}}, \smallskip \\
     \C^{\frac{p+q-1}{2}}\times \C^{\frac{p+q-1}{2}},\  \C^{\frac{p+q}{2}}\times \C^{\frac{p+q}{2}}, \smallskip \\
     \C^n,\ \C^n\times \C^n, \ \C^n, \, \C^{p+q},   \ \C^n,\ \C^n\times \C^n.
   \end{array}
 \]


 We define the Langlands dual $\ckG$ of $G$ to be respectively the complex group
 \[
   \begin{array}{c}
     \GL_n(\C), \ \GL_n(\C), \  \GL_{2n}(\C),\  \GL_{p+q}(\C),\smallskip\\
     \Sp_{p+q-1}(\C)\ (p+q\, \textrm{ is odd}),  \  \SO_{p+q}(\C)\  (p+q\, \textrm{ is even}),\smallskip\\
     \Sp_{n-1}(\C) \ (n\, \textrm{ is odd}),  \
     \SO_n(\C) \ (n\, \textrm{ is even}),\smallskip \\
     \SO_{2n+1}(\C), \ \SO_{2n+1}(\C), \  \SO_{2n}(\C), \  \SO_{2p+2q+1}(\C),\    \Sp_{2n}(\C), \  \textrm{or } \  \Sp_{2n}(\C).
   \end{array}
 \]
 Write $\nckG$ for the dimension of the standard representation of $\check G$,
 which respectively equals
 \[
   n, n, 2n, p+q, p+q-1, p+q, n-1, n, 2n+1, 2n+1, 2n, 2p+2q+1, 2n, \textrm{ or
   }\ 2n.
 \]

 For a Young diagram $\imath$, write
 \[
   \mathbf r_1(\imath)\geq \mathbf r_2(\imath)\geq \mathbf r_3(\imath)\geq \cdots
 \]
 for its row lengths, and similarly, write
 \[
   \mathbf c_1(\imath)\geq \mathbf c_2(\imath)\geq \mathbf c_3(\imath)\geq \cdots
 \]
 for its column lengths. Denote by
 $\abs{\imath}:=\sum_{i=1}^\infty \mathbf r_i(\imath)$ the total size of
 $\imath$.

 % From now on, we fix a pinning and so an abstract root system of the group
 % $\ckcG$.


 We recall the parameterization of the set $\Nil(\ckG)$ of nilpotent $\ckG$
 orbits in $\ckfgg:=\Lie(\ckG)$ by Young diagrams. Each nilpotent orbit
 $\ckcO\in \Nil(\ckG)$ is identified with its Young diagram, still denoted by $\ckcO$
 by abuse of notation, together with a possible symbol $I/II$ and with the following
 properties:
 \begin{itemize}
   \item $\abs{\ckcO}=n_{\check G}$;
   \item if $\star\in\set{D, D^\C, D^*,C, C^\C, C^* } $, then all even rows
         occur in $\check \CO$ with even multiplicity;
   \item if $\star\in \set{B, \widetilde C, B^\C, \widetilde C^\C }$, then all
         odd rows occur in $\ckcO$ with even multiplicity.
   \item if $\star\in \set{D,D^{\bC},D^{*}}$ and all rows of $\ckcO$ are even
         (called a very even orbit), then one attach the symbol ``$I$'' or
         ``$II$'' to $\ckcO$. We adopt the convention that the orbits with
         symbol $I$ are those which can be induced from the standard parabolic
         subgroups. \trivial{ Here, $\ckG$ is the dual group of $G$. Since we
         fix an abstract Cartan $\fhh$ and therefore a Borel subalgebra of
         $G$, the corresponding Cartan and Borel are also fixed for $\ckG$! }
 \end{itemize}
 % By abuse of notation, we write $\ckcO$ for the partition $\imath_{\ckcO}$ and
 % Note that the outer automorphism of the special orthogonal group exchanges
 % the symbol $I$ and $II$.
 When $\star\in \set{D,D^{\bC},D^{*}}$, due to the symmetric nature of orbits marked with $I$ and $II$
 for the problems which concern us (see \Cref{rmk:OII}), we will not consider the orbits with ``$II$'' and
 \[
   \text{ a partition is implicitly attached to the symbol ``$I$''.}
 \]


 \smallskip

 From now on, let $\ckcO$ be a Young diagram which is identified with
 a nilpotent orbit of $\ckcG$.

 We define an element $\lambda_{\ckcO}:=\lambda_{\star, \ckcO} \in \hha^*$ as in
 what follows. For every $a\in \bN^+$ (the set of positive integers), write
 \[
   \rho(a):=\begin{cases}
     (\frac{a-1}{2}, \frac{a-3}{2}, \cdots, 2,1),  &\textrm{if $a$ is odd;}\smallskip\\
     (\frac{a-1}{2}, \frac{a-3}{2}, \cdots, \frac{3}{2}, \frac{1}{2}), &\textrm{if $a$ is even,}\\
   \end{cases}
 \]
 and
 \[
   \tilde \rho(a):= (\frac{a-1}{2}, \frac{a-3}{2}, \cdots, \frac{3-a}{2},\frac{1-a}{2}).
 \]
 By convention, $\rho(1)$ is the empty sequence. Write
 $a_1\geq a_2\geq \cdots\geq a_s>0$ ($s\geq 0$) for the nonzero row lengths of
 $\check \CO$. Then we define
 \[
   \lambda_{\check \CO}:= \begin{cases}
     (\tilde \rho( a_1), \tilde \rho(a_2),  \cdots, \tilde \rho(a_s)), & \text{if } \star \in \set{A, A^\bH, A^*};\\
     (\tilde \rho( a_1), \tilde \rho(a_2),  \cdots, \tilde \rho(a_s); \tilde \rho( a_1), \tilde \rho(a_2),  \cdots, \tilde \rho(a_s)), & \text{if } \star =A^\C;\\
     (\rho( a_1), \rho(a_2),  \cdots, \rho(a_s), 0^{l_1} ) , & \text{if } \star \in \set{B, C, D, D^*, C^*,\widetilde C};\\
     (\rho( a_1), \rho(a_2),  \cdots, \rho(a_s), 0^{l_1} ;  \rho( a_1), \rho(a_2),  \cdots, \rho(a_s), 0^{l_1} ) , & \text{if } \star \in \set{B^\C, C^\C, D^\C, \widetilde C^\C}.\\
   \end{cases}
 \]
 Here
 \[
   l_1:= \left\lfloor\frac{\textrm{the number of odd rows of the Young diagram
         of $\check \CO$}}{2}\right\rfloor,
 \]
 and $0^{l_1}$ denotes the sequence of $0$'s of length $l_1$.

 Note that $\lambda_{\ckcO}$ is the element of $\hha^*$ corresponding to half of the neutral element in any
 $\fsl_{2}$ triple attached to $\ckcO$, as in \cite[Section 5]{BVUni}. By using Harish-Chandra isomorphism, we view $\lambda_{\check \CO}$ as a
 character $\lambda_{\check \CO}: \mathcal Z(\g)\rightarrow \C$. Write
 \[
   I_{\check \CO}:=I_{\star, \check \CO}
 \] for the maximal ideal of $\cU(\g)$ with infinitesimal character
 $\lambda_{\check \CO}$.

 Finally, define the set of the special unipotent representations of $G$
 attached to $\ckcO$ by
 \[
   \begin{split}
     \Unip_{\ckcO}(G):=&  \Unip_{\star, \ckcO}(G) \\
     :=& \begin{cases}
       % \{\pi\in \Irr(G)\mid \pi \textrm{ is annihilated by $ I_{\check \CO}$ or $I'_{\check \CO}$}\}, & \text{if } \star \in \set{D, D^\C, D^*};\\
       \{\pi\in \Irr(G)\mid \pi \textrm{ is genuine  and annihilated by } I_{\check \CO}\}, & \text{if } \star =\widetilde C;\\
       \{\pi\in \Irr(G)\mid \pi \textrm{ is annihilated by } I_{\check \CO}\}, & otherwise.\\
     \end{cases}
   \end{split}
 \]
 Here ``genuine" means that the representation $\pi$ of
 $\widetilde \Sp_{2n}(\R)$ does not descend to $\Sp_{2n}(\R)$.

 A main goal of the current paper is to parametrize the set $\Unip_{\check \CO}(G)$, which will be used by the authors to construct all the
 representations in this set (\cite{BMSZ2}).
 %(Define the unipotent packet using  the neutral element first? )

\begin{remark}\label{rmk:OII}
  Suppose $\star=\set{D,D^{\bC},D^{*}}$. Let $\cA$ be an outer automorphism of
  $G$, which induces an automorphism on $\fgg$ and $\cU(\fgg)$. Let
  \[
    I'_{\ckcO}:=\cA(I_{\ckcO}).
  \]
  If $\ckcO$ has only even rows (i.e. $\ckcO$ is very even), then $I'_{\ckcO}$
  is the maximal primitive ideal attached to the nilpotent orbit $\ckcO_{II}$.
  Otherwise $I_{\ckcO}=I'_{\ckcO}$.
\end{remark}



\subsection{The cases of general linear groups and unitary groups}

For any Young diagram $\imath$, we introduce the set $\mathrm{Box}(\imath)$ of
boxes of $\imath$ as the following subset of $\bN^+\times \bN^+$:
\begin{equation}\label{eq:BOX}
  \mathrm{Box}(\imath):=\Set{(i,j)\in\bN^+\times \bN^+| j\leq \bfrr_i(\imath)}.
\end{equation}
% We will also call a subset of $\bN^+\times \bN^+$ of the form \eqref{eq:BOX} a
% Young diagram.

% We say that a Young diagram $\imath'$ is contained in $\imath$ (and write
% $\imath'\subset \imath$) if
% \[
%   \mathbf r_i(\imath')\leq \mathbf r_i(\imath)\qquad \textrm{for all
% } i=1,2, 3, \cdots.
% \]
% When this is the case, $\mathrm{Box}(\imath')$ is viewed as a subset of
% $\mathrm{Box}(\imath)$ concentrating on the upper-left corner. We say that a
% subset of $\mathrm{Box}(\imath)$ is a Young subdiagram if it equals
% $\mathrm{Box}(\imath')$ for a Young diagram $\imath'\subset \imath$. In this
% case, we call $\imath'$ the Young diagram corresponding to this Young
% subdiagram.

\renewcommand{\CP}{\mathcal{P}} We also introduce five symbols $\bullet$, $s$,
$r$, $c$ and $d$, and make the following definitions.
\begin{defn}
  A painting on a Young diagram $\imath$ is a map
  \[
    \mathcal P: \mathrm{Box}(\imath) \rightarrow \{\bullet, s, r, c, d \}
  \]
  with the following properties:
  \begin{itemize}
    \item $\mathcal P^{-1}(S)$ is the set of boxes of a Young diagram when
          $S=\{\bullet\}, \{\bullet, s \}, \{\bullet, s, r\}$ or
          $\{\bullet, s, r, c \} $;
    \item when $S=\{s\}$ or $ \{r\}$, every row of $\imath$ has at most one box
          in $\CP^{-1}(S)$;
    \item when $S=\{c\}$ or $ \{d \}$, every column of $\imath$ has at most one
          box in $\CP^{-1}(S)$.
  \end{itemize}
\end{defn}



\begin{defn}\label{defpbp0}
  Suppose that $\star\in \{A, A^\bH, A^*\}$. A painting $\CP$ on a Young diagram
  $\imath$ has type $\star$ if
  \begin{itemize}
    \item the image of $\CP$ is contained in
          \[
          \left\{
          \begin{array}{ll}
            \{\bullet, c, d\}, &\hbox{if $\star=A$}; \smallskip\\
            \{\bullet\}, &\hbox{if $\star=A^\bH$}; \smallskip\\
            \{\bullet, s, r\}, &\hbox{if $\star=A^*$},            \end{array}
        \right.
          \]
    \item if $\star=A$ or $A^\bH$, then $\CP^{-1}(\bullet)$ has even number of
          boxes in every column of $\imath$,
    \item if $\star=A^*$, then $\CP^{-1}(\bullet)$ has even number of boxes in
          every row of $\imath$.
  \end{itemize}
  Denote by $\PAP_\star(\imath^{t})$ the set of paintings on $\imath$ that has type $\star$ where $\imath^{t}$
  is the transpose of $\imath$.
   \end{defn}

Note that in the definition of $\PAP_\star(\imath^{t})$, we have incorporated the transpose map in order to reconcile with the Barbasch-Vogan duality.
The middle letter $A$ in $\PAP$ refers to the common $A$ in $\{A, A^\bH, A^*\}$.

The special unipotent representations of general linear groups are
well-understood (reference?). In particular, we have the following counting result for general linear groups.

\begin{thm}\label{GLcase}
  The equality
  \[
    \sharp(\Unip_{\check \CO}(G))= \left\{
      \begin{array}{ll}
        \sharp(\PAP_\star(\check \CO)), &\hbox{if $\star\in\{A, A^\bH\}$}; \smallskip\\
        1, &\hbox{if $\star=A^\C$}  \end{array}
    \right.
  \]
  holds.

\end{thm}
\begin{remark}
  If $\star=A$, then
  \[
    \sharp(\PAP_\star(\check \CO))=\prod_{i\in \bN^+} (1+\textrm{the
      number of rows of length $i$ in $\check \CO$})
  \]
  If $\star=A^\bH$, then
  \[
    \sharp(\PAP_\star(\check \CO))= \left\{
      \begin{array}{ll}
        1, &\hbox{if all row lengths of $\check \CO$ are even}; \smallskip\\
        0, &\hbox{otherwise}.  \end{array}
    \right.
  \]

\end{remark}

Suppose that $\imath$ is a Young diagram and $\CP$ is a painting on $\imath$
that has type $A^*$.
Let $$\mathrm{AC}_\CP: \mathrm{Box}(\imath)\rightarrow \{+, -\}$$ to be the map
such that
\begin{itemize}
  \item the symbols $+$ and $-$ occur alternatively in each row of $\imath$;
  \item for all $1\leq i\leq \mathbf c_1(\imath) $,
        \[
        \mathrm{AC}_\CP (i,\bfrr_i(\imath)) := \begin{cases}
          +,  & \text{if  }\CP(i,\bfrr_i(\imath))=r;\\
          -, & \text{if } \CP(i,\bfrr_i(\imath))\in \set{\bullet,s}.
        \end{cases}
        \]
\end{itemize}
Define the signature of $\CP$ to be the pair
\[
  \begin{split}
    (p_\CP, q_\cP): &= (\sharp(\cP^{-1}(r))+\half \sharp(\cP^{-1}(\bullet)),\sharp(\cP^{-1}(s))
    + \half \sharp(\cP^{-1}(\bullet)))\\
    &=
    (\sharp( \mathrm{AC}_\CP^{-1}(+)),\, \sharp( \mathrm{AC}_\CP^{-1}(-))).\\
  \end{split}
\]
\trivial[]{ The first equation is the true definition of signature. The second
  one is an easy consequence of the definition of $\AC_\cP$. }

\begin{eg}
  Suppose
  that \[ \check \CO=\ytb{\ \ \ \ \ , \ \ \ , \ , \ , \ }\quad \textrm{and}\quad \CP=\ytb{\bullet\bullet\bullet\bullet r,\bullet\bullet , sr,s,r}\in \mathrm{PAP}_{A^*}(\check \CO) .
  \]
  Then
  \[
    \mathrm{AC}_\CP= \ytb{+-+-+,+-, -+,-,+}\quad \textrm{and} \quad (p_\CP, q_\cP)=(6,5).
  \]
\end{eg}

Given two Young diagrams $\imath$ and $\jmath$, write $\imath\cuprow \jmath$ for
the Young diagram whose multiset of nonzero row lengths equals the union of
those of $\imath$ and $\jmath$. Also write $2\imath =\imath\cuprow \imath$.

For unitary groups, we have the following counting result.
\begin{thm}
  Assume that $\star=A^*$ so that $G=\oU(p,q)$. If there is a decomposition
  \[
    \ckcO=\ckcOg \cuprow 2\ckcOpb
  \]
  such that all nonzero row lengths of $\ckcOg$ have the same parity as $p+q$,
  and all nonzero row lengths of $\ckcOpb$ have different parity as $p+q$, then
  \[
    \sharp(\Unip_{\ckcO}(G))= \sharp \set{\CP\in \mathrm{PAP}_\star(\ckcOg)|(p_\CP+\abs{\ckcOpb}, q_\CP+\abs{\ckcOpb})=(p,q)}
  \]
  If there is no such decomposition, then $\sharp(\Unip_{\check \CO}(G))=0$.

\end{thm}

\subsection{Orthogonal and symplectic groups: reduction to good parity}

Now we assume that
\[
  \star\in \Set{ B, D, B^\C, D^\C, C, C^\C, D^*, C^*, \widetilde C, \widetilde C^\C}.
\]
Then there is a unique decomposition
\[
  \ckcO=\ckcOg \cuprow 2\ckcOpb
\]
such that $\ckcOg$ has $\star$-good parity in the sense that all its nonzero row
lengths are
\[
  \left\{
    \begin{array}{ll}
      \textrm{even}, &\hbox{if $\star\in \set{B, B^\C, \widetilde C, \widetilde C^\C}$}; \smallskip\\
      \textrm{odd}, &\hbox{if $\star\in \set{C, D, C^\C, D^\C, D^*, C^*}$},
    \end{array}
  \right.
\]
and $\check \CO_{\mathrm b}$ has $\star$-bad parity in the sense that all its
nonzero row lengths are
\[
  \left\{
    \begin{array}{ll}
      \textrm{odd}, &\hbox{if  $\star\in \set{B, B^\C, \widetilde C, \widetilde C^\C}$}; \smallskip\\
      \textrm{even}, &\hbox{if  $\star\in \set{C, D, C^\C, D^\C, D^*, C^*}$}.
    \end{array}
  \right.
\]
For simplicity, put
\[
  l:=\abs{\ckcOpb},
\]
and
\[
  \Gpb := \begin{cases}
    \GL_{l}(\bR) & \text{when } \star \in \set{B,C,\wtC,D}, \\
    \GL_{l}(\bC) & \text{when } \star \in \set{B^{\bC},C^{\bC},\wtC^{\bC},D^{\bC}}. \\
    \GL_{\frac{l}{2}}(\bH) & \text{when } \star \in \set{C^{*},D^{*}}. \\
  \end{cases}
\]
Note that $G$ has a subgroup isomorphic to $\Gpb$ if and only if
\[
  \begin{cases}
    p,q\geq l & \text{when $G = \SO(p,q)$},\\
    p,q\geq \frac{l}{2} &  \text{when $G = \Sp(p,q)$},\\
    \text{no conditions} & \text{otherwise}.
  \end{cases}
\]
In such cases, $G$ has a Levi subgroup isomorphic to $\Gpb\times \Gg$ where
\[
  \Gg :=
  \begin{cases}
    \SO(p-l,q-l) & \textrm{when $\star\in \set{B,D}$},\\
    \SO_{n-2l}(\bC) &\textrm{when $\star\in \set{B^{\bC},D^{\bC}}$},\\
    \rO^{*}(2n-2l) &\textrm{when $\star = D^{*}$},\\
    \Sp_{2n-2l}(\bR) &\textrm{when $\star = C$},\\
    \wtSp_{2n-2l}(\bR) &\textrm{when $\star = \wtC$},\\
    \Sp_{2n-2l}(\bC) &\textrm{when $\star \in \set{C^{\bC},\wtC^{\bC}}$},\\
    \Sp(p-\frac{l}{2},q-\frac{l}{2}) &\textrm{when $\star = C^{*}$}.\\
  \end{cases}
\]
% and respectively put
% \[
%   \begin{array}{rl}
%     \Gg:=  & \SO(p-l,q-l)\ \  (\textrm{when $p,q\geq l$}),   \ \     \SO_{n-2l}(\C),  \  \   \Sp_{2n-2l}(\R), \  \ \Sp_{2n-2l}(\C), \smallskip \\
%     %   & \oO^*(2n-2l), \ \  \Sp(p-\frac{l}{2},q-\frac{l}{2}) \ \  (\textrm{when $p,q\geq 2l$}),  \ \   \widetilde \Sp_{2n-2l}(\R) \ \ \textrm{or }  \ \  \Sp_{2n-2l}(\C),
%      \end{array}
%    \]
%    when \[
%      \begin{array}{rl}
%     G=  & \SO(p,q)   \ \     \SO_{n}(\C),  \  \   \Sp_{2n}(\R), \  \ \Sp_{2n}(\C), \smallskip \\
%     %   & \oO^*(2n), \ \  \Sp(p,q),  \ \   \widetilde \Sp_{2n}(\R) \ \ \textrm{or }  \ \  \Sp_{2n}(\C).
%      \end{array}
%    \]

\begin{thm}\label{reduction}
  We retain the notation of this subsection. When $G$ has a subgroup isomorphic to $\Gpb$, there
  is a bijection
  \begin{equation}\label{eq:IND}
    \begin{array}{rccc}
      \fI\colon &   \Unip_{G'_{b}}(\ckcO'_{b})\times \Unip_{G_{g}}(\ckcO_{g})&         \longrightarrow &\Unip_{G}(\ckcO) \\
                &   (\pi',\pi_{0}) & \mapsto & \pi'\rtimes \pi_{0}.
    \end{array}
  \end{equation}
  Otherwise,
  \[
    \Unip_{G}(\ckcO)=\emptyset.
  \]
\end{thm}


Combining with the counting result for general linear groups (Theorem \ref{GLcase}), we list the consequences of the
above theorem as follows:
\begin{enumerate}[label=(\alph*)]
  \item Assume that $\star\in \{B,D\}$ so that $G=\SO(p,q)$. Then
        \[
        \sharp(\Unip_{\check \CO}(G))=
        \begin{cases}
          \sharp(\Unip_{\check \CO_{\mathrm g}}(G_g))\times \sharp(\Unip_{\check \CO_{\mathrm b}}(\GL_l(\R)) ), &\hbox{if $p,q\geq l$}; \smallskip\\
          0, &\hbox{otherwise.}
        \end{cases}
        \]
  \item Assume that $\star=C^*$ so that $G=\Sp(p,q)$. Then
        \[
        \sharp(\Unip_{\check \CO}(G))=
        \begin{cases}
          \sharp(\Unip_{\check \CO_{\mathrm g}}(G_g )), &\hbox{if $p,q\geq \frac{l}{2}$}; \smallskip\\
          0, &\hbox{otherwise.}
        \end{cases}
        \]

  \item Assume that $\star\in \{C,\widetilde C\}$ so that $G=\Sp_{2n}(\R)$ or
        $\widetilde \Sp_{2n}(\R)$. Then
        \[
        \sharp(\Unip_{\check \CO}(G))= \sharp(\Unip_{\check \CO_{\mathrm g}}(G_g))\times \sharp(\Unip_{\check \CO_{\mathrm b}}(\GL_l(\R)) ). \]
  \item Assume that $\star\in \{B^\C, D^\C, C^\C,\widetilde C^\C, D^*\}$. Then
        \[
        \sharp(\Unip_{\check \CO}(G))= \sharp(\Unip_{\check \CO_{\mathrm g}}(G_g)).
        \]
\end{enumerate}


 \subsection{Orthogonal and symplectic groups: the case of good parity}
 Now we further assume that $\check \CO$ has $\star$-good parity, namely
 $\check \CO=\check \CO_{\mathrm g}$. By Theorem \ref{reduction}, the counting
 problem in general is reduced to this case.



 \delete{
   \begin{defn}
     A $\star$-pair is a pair $(i,i+1)$ of consecutive positive integers such
     that
     \[
       \left\{
         \begin{array}{ll}
           i\textrm{ is odd}, \quad &\textrm{if $\star\in\{C, \widetilde{C}, C^*, C^\C, \widetilde C^\C\}$};  \\
           i \textrm{ is even}, \quad &\textrm{if $\star\in\{B, D, D^*, B^\C, D^\C\}$}. \\
         \end{array}
       \right.
     \]
     A $\star$-pair $(i,i+1)$ is said to be primitive in $\check \CO$ if
     $\mathbf r_i(\check \CO)-\mathbf r_{i+1}(\check \CO)$ is positive and even.
     Denote $\mathrm{PP}_\star(\check \CO)$ the set of all $\star$-pairs that
     are primitive in $\check \CO$.
   \end{defn}
 }



\begin{defn}
  A $\star$-pair is a pair $(i,i+1)$ of consecutive positive integers such that
  \[
    \left\{
      \begin{array}{ll}
        i\textrm{ is odd}, \quad &\textrm{if $\star\in\{C, \widetilde{C}, C^*, C^\C, \widetilde C^\C\}$};  \\
        i \textrm{ is even}, \quad &\textrm{if $\star\in\{B, D, D^*, B^\C, D^\C\}$}. \\
      \end{array}
    \right.
  \]
  A $\star$-pair $(i,i+1)$ is said to be
  \begin{itemize}
    \item vacant in $\check \CO$, if
          $\mathbf r_i(\check \CO)=\mathbf r_{i+1}(\check \CO)=0$;
    \item balanced in $\check \CO$, if
          $\mathbf r_i(\check \CO)=\mathbf r_{i+1}(\check \CO)>0$;
    \item tailed in $\check \CO$, if
          $\mathbf r_i(\check \CO)-\mathbf r_{i+1}(\check \CO)$ is positive and
          odd;
    \item primitive in $\check \CO$, if
          $\mathbf r_i(\check \CO)-\mathbf r_{i+1}(\check \CO)$ is positive and
          even.
  \end{itemize}
  Denote $\CPP_\star(\check \CO)$ the set of all $\star$-pairs that are
  primitive in $\check \CO$.
\end{defn}
We remark that, when $\star\neq \wtC^{\bC}$, the set $\CPP_{\star}(\ckcO)$ appears implicitly in
\cite{So}*{Section~5} (for the purpose of describing Lusztig's canonical quotient).


\begin{thm}[Barbasch-Vogan, Barbasch]\label{complex}
  Assume that $\star\in \{B^\C,D^\C, C^\C, \widetilde C^\C\}$. Then
  \[
    \sharp(\Unip_{\check \CO}(G))=2^{\sharp(\CPP_\star(\check \CO))} .
  \]
\end{thm}
\trivial[]{ Here is a tricky point: When $\star = \wtC^{\bC}$, the Lusztig
  canonical quotient of $\ckcO$ is a quotient by $\wp\sim \wp^{c}$ of
  $\bF_{2}[\CPP_{\star}(\ckcO)]$. However, the counting should still be correct!
}

\smallskip

We continue with the counting problem of $\Unip_{\check \CO}(G)$, when $\check \CO$ has $\star$-good parity.

We attach to $\check \CO$ a pair of Young diagrams
\[
  (\imath_{\check \CO}, \jmath_{\check \CO}):=(\imath_\star(\check \CO), \jmath_\star(\check \CO)),
\]
as follows.

\medskip

\noindent {\bf The case when $\star=\{B, B^\C\}$.} In this case,
\[
  \mathbf c_{1}(\jmath_{\check \CO})=\frac{\mathbf r_1(\check \CO)}{2},
\]
and for all $i\geq 1$,
\[
  \left (\mathbf c_{i}(\imath_{\check \CO}), \mathbf c_{i+1}(\jmath_{\check \CO})\right )= \left (\frac{\mathbf r_{2i}(\check \CO)}{2}, \frac{\mathbf r_{2i+1}(\check \CO)}{2}\right ).
\]

\medskip

\noindent {\bf The case when $\star\in \{\widetilde C, \widetilde C^\C\}$.} In
this case, for all $i\geq 1$,
\[
  (\mathbf c_{i}(\imath_{\check \CO}), \mathbf c_{i}(\jmath_{\check \CO}))= \left (\frac{\mathbf r_{2i-1}(\check \CO)}{2}, \frac{\mathbf r_{2i}(\check \CO)}{2}\right).
\]

\medskip

\noindent {\bf The case when $\star\in \{C,C^*, C^\C\}$.} In this case, for all
$i\geq 1$,
\[
  (\mathbf c_{i}(\jmath_{\check \CO}), \mathbf c_{i}(\imath_{\check \CO}))= \left\{
    \begin{array}{ll}
      (0,  0), &\hbox{if $(2i-1, 2i)$ is vacant  in $\check \CO$};\smallskip\\
      (\frac{\mathbf r_{2i-1}(\check \CO)-1}{2},  0), & \hbox{if $(2i-1, 2i)$ is tailed in $\check \CO$};\smallskip\\
      (\frac{\mathbf r_{2i-1}(\check \CO)-1}{2},  \frac{\mathbf r_{2i}(\check \CO)+1}{2}), &\hbox{otherwise}.\\
    \end{array}
  \right.
\]
\medskip

\noindent {\bf The case when $\star\in \{D,D^*, D^\C\}$.} In this case,
\[
  \mathbf c_{1}(\imath_{\check \CO})= \left\{
    \begin{array}{ll}
      0,  &\hbox{if $\mathbf r_1(\check \CO)=0$}; \smallskip\\
      \frac{\mathbf r_1(\check \CO)+1}{2},   &\hbox{if $\mathbf r_1(\check \CO)>0$},\\
    \end{array}
  \right.
\]
and for all $i\geq 1$,
\[
  (\mathbf c_{i}(\jmath_{\check \CO}), \mathbf c_{i+1}(\imath_{\check \CO}))= \left\{
    \begin{array}{ll}
      (0,  0), &\hbox{if $(2i, 2i+1)$ is vacant in $\check \CO$};\smallskip\\
      \left  (\frac{\mathbf r_{2i}(\check \CO)-1}{2},  0\right ), & \hbox{if $(2i, 2i+1)$ is tailed in $\check \CO$};\smallskip\\
      \left  (\frac{\mathbf r_{2i}(\check \CO)-1}{2},  \frac{\mathbf r_{2i+1}(\check \CO)+1}{2}\right ), &\hbox{otherwise}.\\
    \end{array}
  \right.
\]




\begin{eg} Suppose that $\star=C$, and $\check \CO$ is the following Young
  diagram which has $\star$-good parity.
  \begin{equation*}\label{eq:sp-nsp.C}
    \tytb{\ \ \ \ \  , \ \ \  , \ \ \ , \ \ \  , \ \ \ , \  ,\  }
  \end{equation*}
  Then
  \[
    \CPP_\star(\check \CO)=\{(1,2), (5,6)\}
  \]
  and
  \[
    (\imath_{\check \CO}, \jmath_{\check \CO})= \tytb{\ \ \ ,\ \ } \times \tytb{\ \ \ , \ }.
  \]


\end{eg}



\delete{
  \begin{eg} Suppose that $\star=C$, and $\check \CO$ is the following Young
    diagram which has $\star$-good parity.
    \begin{equation*}\label{eq:sp-nsp.C}
      \tytb{\ \ \ \ \  , \ \ \  , \ \ \ , \ \ \  , \ \ \ , \  ,\  }
    \end{equation*}
    Then
    \[
      \mathrm{PP}_\star(\check \CO)=\{(1,2), (5,6)\}.
    \]
    and $(\imath_\star(\check \CO, \wp), \jmath_\star(\check \CO,
    \wp))$ %\in \mathrm{BP}_\star(\check \CO)$
    has the following form.

    \begin{equation*}\label{eq:sp-nsp.C}
      \begin{array}{rclcrcl}
        \wp=\emptyset & : & \tytb{\ \ \ ,\ \  } \times \tytb{\ \ \ , \  }  & \qquad \quad &  \wp=\{(1,2)\}& : & \tytb{\ \ \  , \ \ , \   } \times \tytb{\ \ \  } \medskip \medskip \medskip \\
        \wp=\{(5,6)\} & : & \tytb{\ \ \ ,\ \ \ } \times \tytb{\ \ , \   }  & \qquad \quad &  \wp=\{(1,2), (5,6)\}  & : & \tytb{\ \ \  , \ \ \ ,  \ } \times \tytb{\ \   } \\
      \end{array}
    \end{equation*}

\end{eg}
}

Here and henceforth, when no confusion is possible, we write
$\alpha\times \beta$ for a pair $(\alpha, \beta)$. We will also write
$\alpha\times \beta\times \gamma$ for a triple $(\alpha, \beta, \gamma)$.


We introduce two more symbols $B^+$ and $B^-$, and make the following
definition.
\begin{defn}\label{defpbp0}
  A painted bipartition is a triple
  $\tau=(\imath, \CP)\times (\jmath, \cQ)\times \alpha$, where $(\imath, \CP)$
  and $ (\jmath, \mathcal Q)$ are painted Young diagrams, and
  $\alpha\in \{B^+,B^-, C,D,\widetilde {C}, C^*, D^*\}$, subject to the
  following conditions:
  \begin{itemize}
          \delete{\item $(\imath, \jmath)\in \mathrm{BP}_\alpha$ if
          $\alpha\notin\{B^+,B^-\}$, and $(\imath, \jmath)\in \mathrm{BP}_{B}$
          if $\alpha\in\{B^+,B^-\}$;}

    \item $\CP^{-1}(\bullet)=\mathcal Q^{-1}(\bullet)$;
    \item the image of $\CP$ is contained in
          \[
          \left\{
          \begin{array}{ll}
            \{\bullet, c\}, &\hbox{if $\alpha=B^+$ or $B^-$}; \smallskip\\
            \{\bullet,  r, c,d\}, &\hbox{if $\alpha=C$}; \smallskip\\
            \{\bullet, s, r, c,d\}, &\hbox{if $\alpha=D$}; \smallskip\\
            \{\bullet, s, c\}, &\hbox{if $\alpha=\widetilde{ C}$}; \smallskip \\
            \{\bullet\}, &\hbox{if $\alpha=C^*$}; \smallskip \\
            \{\bullet, s\}, &\hbox{if $\alpha=D^*$},\\
          \end{array}
          \right.
          \]
    \item the image of $\mathcal Q$ is contained in
          \[
          \left\{
          \begin{array}{ll}
            \{\bullet, s, r, d\}, &\hbox{if $\alpha=B^+$ or $B^-$}; \smallskip\\
            \{\bullet, s\}, &\hbox{if $\alpha=C$}; \smallskip\\
            \{\bullet\}, &\hbox{if $\alpha=D$}; \smallskip\\
            \{\bullet, r, d\}, &\hbox{if $\alpha=\widetilde{ C}$}; \smallskip\\
            \{\bullet, s,r\}, &\hbox{if $\alpha=C^*$}; \smallskip \\
            \{\bullet, r\}, &\hbox{if $\alpha=D^*$}.
          \end{array}
          \right.
          \]

  \end{itemize}
\end{defn}

% \begin{remark}
%   The set of painted bipartition counts the multiplicities of an irreducible
%   representation of $W_{r_{\fgg}}$ occurs in the coherent continuation
%   representation at the infinitesimal character of the trivial representation.
%   For the relationship between painted bipartitions and the coherent
%   continuation representations of Harish-Chandra modules, see \cite{Mc}.
% \end{remark}

For any painted bipartition $\tau$ as in Definition \ref{defpbp0}, we write
\[
  \imath_\tau:=\imath,\ \cP_\tau:=\cP,\ \jmath_\tau:=\jmath,\ \cQ_\tau:=\cQ,\ \alpha_\tau:=\alpha,
\]
and
\[
  \star_\tau:= \left\{
    \begin{array}{ll}
      B, &\hbox{if $\alpha=B^+$ or $B^-$}; \smallskip\\
      \alpha, & \hbox{otherwise}.           \end{array}
  \right.
\]
% Its leading column is then defined to be the first column of $(\jmath, \CQ)$
% when $\star_\tau\in \{B, C,C^*\}$, and the first column of $(\imath, \CP)$
% when $\star_\tau\in \{\widetilde C, D, D^*\}$.

We further define a pair $(p_{\tau}, q_{\tau})$ of natural numbers given by the
following recipe.
\begin{itemize}
  \item If $\star_\tau\in \{B, D, C^*\}$, $(p_\tau, q_\tau)$ is given by
        counting the various symbols appearing in $(\imath, \CP)$,
        $(\jmath, \cQ)$ and $\{\alpha\}$ :
        \begin{equation}\label{ptqt}
          \left\{
            \begin{array}{l}
              p_\tau :=( \# \bullet)+ 2 (\# r) +(\# c )+ (\# d) + (\# B^+);\smallskip\\
              q_\tau :=( \# \bullet)+ 2 (\# s) + (\# c) + (\# d) + (\# B^-).\\
            \end{array}
          \right.
        \end{equation}
        Here
        \[
        \#\bullet:=\#(\cP^{-1}(\bullet))+\#(\cQ^{-1}(\bullet))
        %\qquad (\textrm{$\#$        indicates the cardinality of a finite set}),
        \]
        and the other terms are similarly defined.
  \item If $\star_\tau\in \{C, \widetilde C, D^*\}$,
        $p_\tau:=q_\tau:=\abs{\tau}$.
\end{itemize}
\smallskip

We also define a classical group
\begin{equation}\label{def:Gt}
  G_\tau:=
  \begin{cases}
    \SO(p_\tau, q_\tau), &\hbox{if $\star_\tau=B$ or $D$}; \smallskip\\
    \Sp_{2\abs{\tau}}(\R), &\hbox{if $\star_\tau=C$}; \smallskip\\
    \widetilde{\Sp}_{2\abs{\tau}}(\R), &\hbox{if $\star_\tau=\widetilde{ C}$}; \smallskip \\
    \Sp(\frac{p_\tau}{2}, \frac{q_\tau}{2}), &\hbox{if $\star_\tau=C^*$}; \smallskip \\
    \oO^*(2\abs{\tau}), &\hbox{if $\star_\tau=D^*$}.\\
  \end{cases}
\end{equation}

Define
\[
  \begin{split}
    \PBP_\star(\check \CO) &:=\set{ \uptau\textrm{ is a painted
        bipartition} \mid \star_\uptau = \star, \text{ and
      } (\imath_\tau,\jmath_\tau) = (\imath_{\check \CO}, \jmath_{\check \CO})}, \AND\\
    \PBP_{G}(\ckcO) &:=\set{\uptau\in \PBP_{\star}(\ckcO)| G_{\uptau} = G}.
  \end{split}
\]


\delete{
  \[
    \begin{array}{rl}
      \mathrm{PBP}_\star(\check \CO):=\{ &
                                           \tau\textrm{ is a painted bipartition}  \mid    \star_\tau = \star,
                                           \text{ and } \\  & (\imath_\tau,\jmath_\tau) = (\imath_{\check \CO}, \jmath_{\check \CO})\}.
    \end{array}
  \]
}




\begin{eg} Suppose that $\star=B$ and
  \[
    \check \CO =\tytb{\ \ \ \ \ \ , \ \ \ \ \ \ , \ \ , \ \ , \ \ }
  \]
  Then
  \[
    \tau:= \tytb{\bullet \bullet ,\bullet , c } \times \tytb{\bullet \bullet d ,\bullet , d }\times B^+\in \mathrm{PBP}_{\star}(\check \CO),
  \]
  and
  \[
    G_\tau=\SO(10,9).
  \]
\end{eg}


We now state our final result on the counting of special unipotent representations.

\begin{thm}\label{countup}
  Assume that $\star\in \{B, C,D,\widetilde {C}, C^*, D^*\}$, and $\check \CO$ has $\star$-good parity. Then
  \[
    \sharp(\Unip_{\ckcO}(G))\leq 2^{\sharp(\CPP_\star(\check \CO))} \cdot \sharp \PBP_{G}(\ckcO).
  \]
\end{thm}

In \cite{BMSZ2}, we will construct $2^{\sharp(\CPP_\star(\check \CO))} \cdot \sharp \PBP_{G}(\ckcO)$ number of representations in
$\Unip_{\check \CO}(G)$, and thus the equality holds in \Cref{countup}.

\medskip

Here are some words on the contents and the organization of this article. In Section 2, we introduce the coherent continuation representation in a general set-up and prove a basic result of Vogan on counting representations via the translation principle. In Section 3, we review the theory of primitive ideas and double cells and their connection with the coherent continuation representation. The main result is a characterization of the irreducible constituents of the coherent continuation representation in (a variant of) the category $\CO$, in terms of the Springer correspondence. In Section 4, we examine the relationship of coherent families of Harish-Chandra modules with coherent families in category $\CO$ and prove a upper bound on the counting of irreducible representations annihilated by a maximal primitive ideal. We also give a formula for the coherent continuation representation of Harish-Chandra modules, which is an unpublished result of Barbasch and Vogan. Sections 5 to 7 are devoted to the main concern of the article, which is to give a precise count of special unipotent representations of all real classical groups, starting from the general linear groups and then unitary groups, and finally real classical groups of type $\mathrm{BCD}$. The answer is given in terms of some combinatorial constructs described earlier in this section. It is important to note, while the algebraic theory alluded to yields ultimately an upper bound of the count, we are unable to prove the precise count using the algebraic theory alone, due to a certain technical issue on double cells of Harish-Chandra modules. Instead we rely on the analytic theory of theta lifting to construct the right amount of special unipotent representations (\cite{BMSZ2}) and therefore to close the gap, so to speak. It will be clearly desirable to demonstrate the precise count, without recourse to the analytic theory.


\end{document}

%%% Local Variables:
%%% mode: latex
%%% TeX-master: "counting_main"
%%% End:
