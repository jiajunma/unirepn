% !TeX program = xelatex
\documentclass[12pt,a4paper]{amsart}
\usepackage[margin=2.5cm,marginpar=2cm]{geometry}

\usepackage[bookmarksopen,bookmarksdepth=2,hidelinks,colorlinks=false]{hyperref}
\usepackage[nameinlink]{cleveref}

% \usepackage[color]{showkeys}
% \makeatletter
%   \SK@def\Cref#1{\SK@\SK@@ref{#1}\SK@Cref{#1}}%
% \makeatother
%% FONTS

\usepackage{amssymb}
%\usepackage{amsmath}
\usepackage{mathrsfs}
\usepackage{mathtools}
%\usepackage{amsrefs}
%\usepackage{mathbbol,mathabx}
\usepackage{amsthm}
\usepackage{graphicx}
\usepackage{braket}
%\usepackage[pointedenum]{paralist}
%\usepackage{paralist}


\usepackage[alphabetic]{amsrefs}

\usepackage[all,cmtip]{xy}
\usepackage{rotating}
\usepackage{leftidx}
%\usepackage{arydshln}

%\DeclareSymbolFont{bbold}{U}{bbold}{m}{n}
%\DeclareSymbolFontAlphabet{\mathbbold}{bbold}


%\usepackage[dvipdfx,rgb,table]{xcolor}
\usepackage[rgb,table]{xcolor}
%\usepackage{mathrsfs}

\setcounter{tocdepth}{1}
\setcounter{secnumdepth}{2}

%\usepackage[abbrev,shortalphabetic]{amsrefs}


\usepackage[normalem]{ulem}

% circled number
\usepackage{pifont}
\makeatletter
\newcommand*{\circnuma}[1]{%
  \ifnum#1<1 %
    \@ctrerr
  \else
    \ifnum#1>20 %
      \@ctrerr
    \else
      \mbox{\ding{\numexpr 171+(#1)\relax}}%
     \fi
  \fi
}
\makeatother

\usepackage[centertableaux]{ytableau}


% Ytableau tweak
\makeatletter
\pgfkeys{/ytableau/options,
  noframe/.default = false,
  noframe/.is choice,
  noframe/true/.code = {%
    \global\let\vrule@YT=\vrule@none@YT
    \global\let\hrule@YT=\hrule@none@YT
  },
  noframe/false/.code = {%
    \global\let\vrule@YT=\vrule@normal@YT
    \global\let\hrule@YT=\hrule@normal@YT
  },
  noframe/on/.style = {noframe/true},
  noframe/off/.style = {noframe/false},
}

\def\hrule@enon@YT{%
  \hrule width  \dimexpr \boxdim@YT + \fboxrule *2 \relax
  height 0pt
}
\def\vrule@enon@YT{%
  \vrule height \dimexpr  \boxdim@YT + \fboxrule\relax
     width \fboxrule
}

\def\enon{\omit\enon@YT}
\newcommand{\enon@YT}[2][clear]{%
  \def\thisboxcolor@YT{#1}%
  \let\hrule@YT=\hrule@enon@YT
  \let\vrule@YT=\vrule@enon@YT
  \startbox@@YT#2\endbox@YT
  \nullfont
}

\makeatother
%\ytableausetup{noframe=on,smalltableaux}
\ytableausetup{noframe=off,boxsize=1.3em}
\let\ytb=\ytableaushort

\newcommand{\tytb}[1]{{\tiny\ytb{#1}}}


%\usepackage[mathlines,pagewise]{lineno}
%\linenumbers

\usepackage{enumitem}
%% Enumitem
\newlist{enumC}{enumerate}{1} % Conditions in Lemma/Theorem/Prop
\setlist[enumC,1]{label=(\alph*),wide,ref=(\alph*)}
\crefname{enumCi}{condition}{conditions}
\Crefname{enumCi}{Condition}{Conditions}
\newlist{enumT}{enumerate}{3} % "Theorem"=conclusions in Lemma/Theorem/Prop
\setlist[enumT]{label=(\roman*),wide}
\setlist[enumT,1]{label=(\roman*),wide}
\setlist[enumT,2]{label=(\alph*),ref ={(\roman{enumTi}.\alph*)}}
\setlist[enumT,3]{label=(\arabic*), ref ={(\roman{enumTi}.\alph{enumTii}.\alph*)}}
\crefname{enumTi}{}{}
\Crefname{enumTi}{Item}{Items}
\crefname{enumTii}{}{}
\Crefname{enumTii}{Item}{Items}
\crefname{enumTiii}{}{}
\Crefname{enumTiii}{Item}{Items}
\newlist{enumPF}{enumerate}{3}
\setlist[enumPF]{label=(\alph*),wide}
\setlist[enumPF,1]{label=(\roman*),wide}
\setlist[enumPF,2]{label=(\alph*)}
\setlist[enumPF,3]{label=\arabic*).}
\newlist{enumS}{enumerate}{3} % Statement outside Lemma/Theorem/Prop
\setlist[enumS]{label=\roman*)}
\setlist[enumS,1]{label=\roman*)}
\setlist[enumS,2]{label=\alph*)}
\setlist[enumS,3]{label=\arabic*.}
\newlist{enumI}{enumerate}{3} % items
\setlist[enumI,1]{label=\roman*),leftmargin=*}
\setlist[enumI,2]{label=\alph*), leftmargin=*}
\setlist[enumI,3]{label=\arabic*), leftmargin=*}
\newlist{enumIL}{enumerate*}{1} % inline enum
\setlist*[enumIL]{label=\roman*)}
\newlist{enumR}{enumerate}{1} % remarks
\setlist[enumR]{label=\arabic*.,wide,labelwidth=!, labelindent=0pt}
\crefname{enumRi}{remark}{remarks}

\crefname{equation}{}{}
\Crefname{equation}{Equation}{Equations}
\Crefname{lem}{Lemma}{Lemma}
\Crefname{thm}{Theorem}{Theorem}

\newlist{des}{description}{1}
\setlist[des]{font=\sffamily\bfseries}

% editing macros.
\blendcolors{!80!black}
\long\def\okay#1{\ifcsname highlightokay\endcsname
{\color{red} #1}
\else
{#1}
\fi
}
\long\def\editc#1{{\color{red} #1}}
\long\def\mjj#1{{{\color{blue}#1}}}
\long\def\mjjr#1{{\color{red} (#1)}}
\long\def\mjjd#1#2{{\color{blue} #1 \sout{#2}}}
\def\mjjb{\color{blue}}
\def\mjje{\color{black}}
\def\mjjcb{\color{green!50!black}}
\def\mjjce{\color{black}}

\long\def\sun#1{{{\color{cyan}#1}}}
\long\def\sund#1#2{{\color{cyan}#1  \sout{#2}}}
\long\def\mv#1{{{\color{red} {\bf move to a proper place:} #1}}}
\long\def\delete#1{}

%\reversemarginpar
\newcommand{\lokec}[1]{\marginpar{\color{blue}\tiny #1 \mbox{--loke}}}
\newcommand{\mjjc}[1]{\marginpar{\color{green}\tiny #1 \mbox{--ma}}}

\newcommand{\trivial}[2][]{\if\relax\detokenize{#1}\relax
  {%\hfill\break
   % \begin{minipage}{\textwidth}
      \color{orange} \vspace{0em} $[$  #2 $]$
  %\end{minipage}
  %\break
      \color{black}
  }
  \else
\ifx#1h
\ifcsname showtrivial\endcsname
{%\hfill\break
 % \begin{minipage}{\textwidth}
    \color{orange} \vspace{0em}  $[$ #2 $]$
%\end{minipage}
%\break
    \color{black}
}
\fi
\else {\red Wrong argument!} \fi
\fi
}

\newcommand{\byhide}[2][]{\if\relax\detokenize{#1}\relax
{\color{orange} \vspace{0em} Plan to delete:  #2}
\else
\ifx#1h\relax\fi
\fi
}



\newcommand{\Rank}{\mathrm{rk}}
\newcommand{\cqq}{\mathscr{D}}
\newcommand{\rsym}{\mathrm{sym}}
\newcommand{\rskew}{\mathrm{skew}}
\newcommand{\fraksp}{\mathfrak{sp}}
\newcommand{\frakso}{\mathfrak{so}}
\newcommand{\frakm}{\mathfrak{m}}
\newcommand{\frakp}{\mathfrak{p}}
\newcommand{\pr}{\mathrm{pr}}
\newcommand{\rhopst}{\rho'^*}
\newcommand{\Rad}{\mathrm{Rad}}
\newcommand{\Res}{\mathrm{Res}}
\newcommand{\Hol}{\mathrm{Hol}}
\newcommand{\AC}{\mathrm{AC}}
%\newcommand{\AS}{\mathrm{AS}}
\newcommand{\WF}{\mathrm{WF}}
\newcommand{\AV}{\mathrm{AV}}
\newcommand{\AVC}{\mathrm{AV}_\bC}
\newcommand{\VC}{\mathrm{V}_\bC}
\newcommand{\bfv}{\mathbf{v}}
\newcommand{\depth}{\mathrm{depth}}
\newcommand{\wtM}{\widetilde{M}}
\newcommand{\wtMone}{{\widetilde{M}^{(1,1)}}}

\newcommand{\nullpp}{N(\fpp'^*)}
\newcommand{\nullp}{N(\fpp^*)}
%\newcommand{\Aut}{\mathrm{Aut}}

\def\mstar{{\medstar}}


\newcommand{\bfone}{\mathbf{1}}
\newcommand{\piSigma}{\pi_\Sigma}
\newcommand{\piSigmap}{\pi'_\Sigma}


\newcommand{\sfVprime}{\mathsf{V}^\prime}
\newcommand{\sfVdprime}{\mathsf{V}^{\prime \prime}}
\newcommand{\gminusone}{\mathfrak{g}_{-\frac{1}{m}}}

\newcommand{\eva}{\mathrm{eva}}

% \newcommand\iso{\xrightarrow{
%    \,\smash{\raisebox{-0.65ex}{\ensuremath{\scriptstyle\sim}}}\,}}

\def\Ueven{{U_{\rm{even}}}}
\def\Uodd{{U_{\rm{odd}}}}
\def\ttau{\tilde{\tau}}
\def\Wcp{W}
\def\Kur{{K^{\mathrm{u}}}}

\def\Im{\operatorname{Im}}

\providecommand{\bcN}{{\overline{\cN}}}



\makeatletter

\def\gen#1{\left\langle
    #1
      \right\rangle}
\makeatother

\makeatletter
\def\inn#1#2{\left\langle
      \def\ta{#1}\def\tb{#2}
      \ifx\ta\@empty{\;} \else {\ta}\fi ,
      \ifx\tb\@empty{\;} \else {\tb}\fi
      \right\rangle}
\def\binn#1#2{\left\lAngle
      \def\ta{#1}\def\tb{#2}
      \ifx\ta\@empty{\;} \else {\ta}\fi ,
      \ifx\tb\@empty{\;} \else {\tb}\fi
      \right\rAngle}
\makeatother

\makeatletter
\def\binn#1#2{\overline{\inn{#1}{#2}}}
\makeatother


\def\innwi#1#2{\inn{#1}{#2}_{W_i}}
\def\innw#1#2{\inn{#1}{#2}_{\bfW}}
\def\innv#1#2{\inn{#1}{#2}_{\bfV}}
\def\innbfv#1#2{\inn{#1}{#2}_{\bfV}}
\def\innvi#1#2{\inn{#1}{#2}_{V_i}}
\def\innvp#1#2{\inn{#1}{#2}_{\bfV'}}
\def\innp#1#2{\inn{#1}{#2}'}

% choose one of then
\def\simrightarrow{\iso}
\def\surj{\twoheadrightarrow}
%\def\simrightarrow{\xrightarrow{\sim}}

\newcommand\iso{\xrightarrow{
   \,\smash{\raisebox{-0.65ex}{\ensuremath{\scriptstyle\sim}}}\,}}

\newcommand\riso{\xleftarrow{
   \,\smash{\raisebox{-0.65ex}{\ensuremath{\scriptstyle\sim}}}\,}}









\usepackage{xparse}
\def\usecsname#1{\csname #1\endcsname}
\def\useLetter#1{#1}
\def\usedbletter#1{#1#1}

% \def\useCSf#1{\csname f#1\endcsname}

\ExplSyntaxOn

\def\mydefcirc#1#2#3{\expandafter\def\csname
  circ#3{#1}\endcsname{{}^\circ {#2{#1}}}}
\def\mydefvec#1#2#3{\expandafter\def\csname
  vec#3{#1}\endcsname{\vec{#2{#1}}}}
\def\mydefdot#1#2#3{\expandafter\def\csname
  dot#3{#1}\endcsname{\dot{#2{#1}}}}

\def\mydefacute#1#2#3{\expandafter\def\csname a#3{#1}\endcsname{\acute{#2{#1}}}}
\def\mydefbr#1#2#3{\expandafter\def\csname br#3{#1}\endcsname{\breve{#2{#1}}}}
\def\mydefbar#1#2#3{\expandafter\def\csname bar#3{#1}\endcsname{\bar{#2{#1}}}}
\def\mydefhat#1#2#3{\expandafter\def\csname hat#3{#1}\endcsname{\hat{#2{#1}}}}
\def\mydefwh#1#2#3{\expandafter\def\csname wh#3{#1}\endcsname{\widehat{#2{#1}}}}
\def\mydeft#1#2#3{\expandafter\def\csname t#3{#1}\endcsname{\tilde{#2{#1}}}}
\def\mydefu#1#2#3{\expandafter\def\csname u#3{#1}\endcsname{\underline{#2{#1}}}}
\def\mydefr#1#2#3{\expandafter\def\csname r#3{#1}\endcsname{\mathrm{#2{#1}}}}
\def\mydefb#1#2#3{\expandafter\def\csname b#3{#1}\endcsname{\mathbb{#2{#1}}}}
\def\mydefwt#1#2#3{\expandafter\def\csname wt#3{#1}\endcsname{\widetilde{#2{#1}}}}
%\def\mydeff#1#2#3{\expandafter\def\csname f#3{#1}\endcsname{\mathfrak{#2{#1}}}}
\def\mydefbf#1#2#3{\expandafter\def\csname bf#3{#1}\endcsname{\mathbf{#2{#1}}}}
\def\mydefc#1#2#3{\expandafter\def\csname c#3{#1}\endcsname{\mathcal{#2{#1}}}}
\def\mydefsf#1#2#3{\expandafter\def\csname sf#3{#1}\endcsname{\mathsf{#2{#1}}}}
\def\mydefs#1#2#3{\expandafter\def\csname s#3{#1}\endcsname{\mathscr{#2{#1}}}}
\def\mydefcks#1#2#3{\expandafter\def\csname cks#3{#1}\endcsname{{\check{
        \csname s#2{#1}\endcsname}}}}
\def\mydefckc#1#2#3{\expandafter\def\csname ckc#3{#1}\endcsname{{\check{
      \csname c#2{#1}\endcsname}}}}
\def\mydefck#1#2#3{\expandafter\def\csname ck#3{#1}\endcsname{{\check{#2{#1}}}}}

\cs_new:Npn \mydeff #1#2#3 {\cs_new:cpn {f#3{#1}} {\mathfrak{#2{#1}}}}

\cs_new:Npn \doGreek #1
{
  \clist_map_inline:nn {alpha,beta,gamma,Gamma,delta,Delta,epsilon,varepsilon,zeta,eta,theta,vartheta,Theta,iota,kappa,lambda,Lambda,mu,nu,xi,Xi,pi,Pi,rho,sigma,varsigma,Sigma,tau,upsilon,Upsilon,phi,varphi,Phi,chi,psi,Psi,omega,Omega,tG} {#1{##1}{\usecsname}{\useLetter}}
}

\cs_new:Npn \doSymbols #1
{
  \clist_map_inline:nn {otimes,boxtimes} {#1{##1}{\usecsname}{\useLetter}}
}

\cs_new:Npn \doAtZ #1
{
  \clist_map_inline:nn {A,B,C,D,E,F,G,H,I,J,K,L,M,N,O,P,Q,R,S,T,U,V,W,X,Y,Z} {#1{##1}{\useLetter}{\useLetter}}
}

\cs_new:Npn \doatz #1
{
  \clist_map_inline:nn {a,b,c,d,e,f,g,h,i,j,k,l,m,n,o,p,q,r,s,t,u,v,w,x,y,z} {#1{##1}{\useLetter}{\usedbletter}}
}

\cs_new:Npn \doallAtZ
{
\clist_map_inline:nn {mydefsf,mydeft,mydefu,mydefwh,mydefhat,mydefr,mydefwt,mydeff,mydefb,mydefbf,mydefc,mydefs,mydefck,mydefcks,mydefckc,mydefbar,mydefvec,mydefcirc,mydefdot,mydefbr,mydefacute} {\doAtZ{\csname ##1\endcsname}}
}

\cs_new:Npn \doallatz
{
\clist_map_inline:nn {mydefsf,mydeft,mydefu,mydefwh,mydefhat,mydefr,mydefwt,mydeff,mydefb,mydefbf,mydefc,mydefs,mydefck,mydefbar,mydefvec,mydefdot,mydefbr,mydefacute} {\doatz{\csname ##1\endcsname}}
}

\cs_new:Npn \doallGreek
{
\clist_map_inline:nn {mydefck,mydefwt,mydeft,mydefwh,mydefbar,mydefu,mydefvec,mydefcirc,mydefdot,mydefbr,mydefacute} {\doGreek{\csname ##1\endcsname}}
}

\cs_new:Npn \doallSymbols
{
\clist_map_inline:nn {mydefck,mydefwt,mydeft,mydefwh,mydefbar,mydefu,mydefvec,mydefcirc,mydefdot} {\doSymbols{\csname ##1\endcsname}}
}



\cs_new:Npn \doGroups #1
{
  \clist_map_inline:nn {GL,Sp,rO,rU,fgl,fsp,foo,fuu,fkk,fuu,ufkk,uK} {#1{##1}{\usecsname}{\useLetter}}
}

\cs_new:Npn \doallGroups
{
\clist_map_inline:nn {mydeft,mydefu,mydefwh,mydefhat,mydefwt,mydefck,mydefbar} {\doGroups{\csname ##1\endcsname}}
}


\cs_new:Npn \decsyms #1
{
\clist_map_inline:nn {#1} {\expandafter\DeclareMathOperator\csname ##1\endcsname{##1}}
}

\decsyms{Mp,id,SL,Sp,SU,SO,GO,GSO,GU,GSp,PGL,Pic,Lie,Mat,Ker,Hom,Ext,Ind,reg,res,inv,Isom,Det,Tr,Norm,Sym,Span,Stab,Spec,PGSp,PSL,tr,Ad,Br,Ch,Cent,End,Aut,Dvi,Frob,Gal,GL,Gr,DO,ur,vol,ab,Nil,Supp,rank,Sign}

\def\abs#1{\left|{#1}\right|}
\def\norm#1{{\left\|{#1}\right\|}}


% \NewDocumentCommand\inn{m m}{
% \left\langle
% \IfValueTF{#1}{#1}{000}
% ,
% \IfValueTF{#2}{#2}{000}
% \right\rangle
% }
\NewDocumentCommand\cent{o m }{
  \IfValueTF{#1}{
    \mathop{Z}_{#1}{(#2)}}
  {\mathop{Z}{(#2)}}
}


\def\fsl{\mathfrak{sl}}
\def\fsp{\mathfrak{sp}}


%\def\cent#1#2{{\mathrm{Z}_{#1}({#2})}}


\doallAtZ
\doallatz
\doallGreek
\doallGroups
\doallSymbols
\ExplSyntaxOff


% \usepackage{geometry,amsthm,graphics,tabularx,amssymb,shapepar}
% \usepackage{amscd}
% \usepackage{mathrsfs}


\usepackage{diagbox}
% Update the information and uncomment if AMS is not the copyright
% holder.
%\copyrightinfo{2006}{American Mathematical Society}
%\usepackage{nicematrix}
\usepackage{arydshln}

\usepackage{tikz}
\usetikzlibrary{matrix,arrows,positioning,cd,backgrounds}
\usetikzlibrary{decorations.pathmorphing,decorations.pathreplacing}

\usepackage{upgreek}

\usepackage{listings}
\lstset{
    basicstyle=\ttfamily\tiny,
    keywordstyle=\color{black},
    commentstyle=\color{white}, % white comments
    stringstyle=\ttfamily, % typewriter type for strings
    showstringspaces=false,
    breaklines=true,
    emph={Output},emphstyle=\color{blue},
}

\newcommand{\BA}{{\mathbb{A}}}
%\newcommand{\BB}{{\mathbb {B}}}
\newcommand{\BC}{{\mathbb {C}}}
\newcommand{\BD}{{\mathbb {D}}}
\newcommand{\BE}{{\mathbb {E}}}
\newcommand{\BF}{{\mathbb {F}}}
\newcommand{\BG}{{\mathbb {G}}}
\newcommand{\BH}{{\mathbb {H}}}
\newcommand{\BI}{{\mathbb {I}}}
\newcommand{\BJ}{{\mathbb {J}}}
\newcommand{\BK}{{\mathbb {U}}}
\newcommand{\BL}{{\mathbb {L}}}
\newcommand{\BM}{{\mathbb {M}}}
\newcommand{\BN}{{\mathbb {N}}}
\newcommand{\BO}{{\mathbb {O}}}
\newcommand{\BP}{{\mathbb {P}}}
\newcommand{\BQ}{{\mathbb {Q}}}
\newcommand{\BR}{{\mathbb {R}}}
\newcommand{\BS}{{\mathbb {S}}}
\newcommand{\BT}{{\mathbb {T}}}
\newcommand{\BU}{{\mathbb {U}}}
\newcommand{\BV}{{\mathbb {V}}}
\newcommand{\BW}{{\mathbb {W}}}
\newcommand{\BX}{{\mathbb {X}}}
\newcommand{\BY}{{\mathbb {Y}}}
\newcommand{\BZ}{{\mathbb {Z}}}
\newcommand{\Bk}{{\mathbf {k}}}

\newcommand{\CA}{{\mathcal {A}}}
\newcommand{\CB}{{\mathcal {B}}}
\newcommand{\CC}{{\mathcal {C}}}

\newcommand{\CE}{{\mathcal {E}}}
\newcommand{\CF}{{\mathcal {F}}}
\newcommand{\CG}{{\mathcal {G}}}
\newcommand{\CH}{{\mathcal {H}}}
\newcommand{\CI}{{\mathcal {I}}}
\newcommand{\CJ}{{\mathcal {J}}}
\newcommand{\CK}{{\mathcal {K}}}
\newcommand{\CL}{{\mathcal {L}}}
\newcommand{\CM}{{\mathcal {M}}}
\newcommand{\CN}{{\mathcal {N}}}
\newcommand{\CO}{{\mathcal {O}}}
\newcommand{\CP}{{\mathcal {P}}}
\newcommand{\CQ}{{\mathcal {Q}}}
\newcommand{\CR}{{\mathcal {R}}}
\newcommand{\CS}{{\mathcal {S}}}
\newcommand{\CT}{{\mathcal {T}}}
\newcommand{\CU}{{\mathcal {U}}}
\newcommand{\CV}{{\mathcal {V}}}
\newcommand{\CW}{{\mathcal {W}}}
\newcommand{\CX}{{\mathcal {X}}}
\newcommand{\CY}{{\mathcal {Y}}}
\newcommand{\CZ}{{\mathcal {Z}}}


\newcommand{\RA}{{\mathrm {A}}}
\newcommand{\RB}{{\mathrm {B}}}
\newcommand{\RC}{{\mathrm {C}}}
\newcommand{\RD}{{\mathrm {D}}}
\newcommand{\RE}{{\mathrm {E}}}
\newcommand{\RF}{{\mathrm {F}}}
\newcommand{\RG}{{\mathrm {G}}}
\newcommand{\RH}{{\mathrm {H}}}
\newcommand{\RI}{{\mathrm {I}}}
\newcommand{\RJ}{{\mathrm {J}}}
\newcommand{\RK}{{\mathrm {K}}}
\newcommand{\RL}{{\mathrm {L}}}
\newcommand{\RM}{{\mathrm {M}}}
\newcommand{\RN}{{\mathrm {N}}}
\newcommand{\RO}{{\mathrm {O}}}
\newcommand{\RP}{{\mathrm {P}}}
\newcommand{\RQ}{{\mathrm {Q}}}
%\newcommand{\RR}{{\mathrm {R}}}
\newcommand{\RS}{{\mathrm {S}}}
\newcommand{\RT}{{\mathrm {T}}}
\newcommand{\RU}{{\mathrm {U}}}
\newcommand{\RV}{{\mathrm {V}}}
\newcommand{\RW}{{\mathrm {W}}}
\newcommand{\RX}{{\mathrm {X}}}
\newcommand{\RY}{{\mathrm {Y}}}
\newcommand{\RZ}{{\mathrm {Z}}}

\DeclareMathOperator{\absNorm}{\mathfrak{N}}
\DeclareMathOperator{\Ann}{Ann}
\DeclareMathOperator{\LAnn}{L-Ann}
\DeclareMathOperator{\RAnn}{R-Ann}
\DeclareMathOperator{\ind}{ind}
%\DeclareMathOperator{\Ind}{Ind}



\newcommand{\cod}{{\mathrm{cod}}}
\newcommand{\cont}{{\mathrm{cont}}}
\newcommand{\cl}{{\mathrm{cl}}}
\newcommand{\cusp}{{\mathrm{cusp}}}

\newcommand{\disc}{{\mathrm{disc}}}
\renewcommand{\div}{{\mathrm{div}}}



\newcommand{\Gm}{{\mathbb{G}_m}}



\newcommand{\I}{{\mathrm{I}}}

\newcommand{\Jac}{{\mathrm{Jac}}}
\newcommand{\PM}{{\mathrm{PM}}}


\newcommand{\new}{{\mathrm{new}}}
\newcommand{\NS}{{\mathrm{NS}}}
\newcommand{\N}{{\mathrm{N}}}

\newcommand{\ord}{{\mathrm{ord}}}

%\newcommand{\rank}{{\mathrm{rank}}}

\newcommand{\rk}{{\mathrm{k}}}
\newcommand{\rr}{{\mathrm{r}}}
\newcommand{\rh}{{\mathrm{h}}}

\newcommand{\Sel}{{\mathrm{Sel}}}
\newcommand{\Sim}{{\mathrm{Sim}}}

\newcommand{\wt}{\widetilde}
\newcommand{\wh}{\widehat}
\newcommand{\pp}{\frac{\partial\bar\partial}{\pi i}}
\newcommand{\pair}[1]{\langle {#1} \rangle}
\newcommand{\wpair}[1]{\left\{{#1}\right\}}
\newcommand{\intn}[1]{\left( {#1} \right)}
\newcommand{\sfrac}[2]{\left( \frac {#1}{#2}\right)}
\newcommand{\ds}{\displaystyle}
\newcommand{\ov}{\overline}
\newcommand{\incl}{\hookrightarrow}
\newcommand{\lra}{\longrightarrow}
\newcommand{\imp}{\Longrightarrow}
%\newcommand{\lto}{\longmapsto}
\newcommand{\bs}{\backslash}

\newcommand{\cover}[1]{\widetilde{#1}}

\renewcommand{\vsp}{{\vspace{0.2in}}}

\newcommand{\Norma}{\operatorname{N}}
\newcommand{\Ima}{\operatorname{Im}}
\newcommand{\con}{\textit{C}}
\newcommand{\gr}{\operatorname{gr}}
\newcommand{\ad}{\operatorname{ad}}
\newcommand{\der}{\operatorname{der}}
\newcommand{\dif}{\operatorname{d}\!}
\newcommand{\pro}{\operatorname{pro}}
\newcommand{\Ev}{\operatorname{Ev}}
% \renewcommand{\span}{\operatorname{span}} \span is an innernal command.
%\newcommand{\degree}{\operatorname{deg}}
\newcommand{\Invf}{\operatorname{Invf}}
\newcommand{\Inv}{\operatorname{Inv}}
\newcommand{\slt}{\operatorname{SL}_2(\mathbb{R})}
%\newcommand{\temp}{\operatorname{temp}}
%\newcommand{\otop}{\operatorname{top}}
\renewcommand{\small}{\operatorname{small}}
\newcommand{\HC}{\operatorname{HC}}
\newcommand{\lef}{\operatorname{left}}
\newcommand{\righ}{\operatorname{right}}
\newcommand{\Diff}{\operatorname{DO}}
\newcommand{\diag}{\operatorname{diag}}
\newcommand{\sh}{\varsigma}
\newcommand{\sch}{\operatorname{sch}}
%\newcommand{\oleft}{\operatorname{left}}
%\newcommand{\oright}{\operatorname{right}}
\newcommand{\open}{\operatorname{open}}
\newcommand{\sgn}{\operatorname{sgn}}
\newcommand{\triv}{\operatorname{triv}}
\newcommand{\Sh}{\operatorname{Sh}}
\newcommand{\oN}{\operatorname{N}}

\newcommand{\oc}{\operatorname{c}}
\newcommand{\od}{\operatorname{d}}
\newcommand{\os}{\operatorname{s}}
\newcommand{\ol}{\operatorname{l}}
\newcommand{\oL}{\operatorname{L}}
\newcommand{\oJ}{\operatorname{J}}
\newcommand{\oH}{\operatorname{H}}
\newcommand{\oO}{\operatorname{O}}
\newcommand{\oS}{\operatorname{S}}
\newcommand{\oR}{\operatorname{R}}
\newcommand{\oT}{\operatorname{T}}
%\newcommand{\rU}{\operatorname{U}}
\newcommand{\oZ}{\operatorname{Z}}
\newcommand{\oD}{\textit{D}}
\newcommand{\oW}{\textit{W}}
\newcommand{\oE}{\operatorname{E}}
\newcommand{\oP}{\operatorname{P}}
\newcommand{\PD}{\operatorname{PD}}
\newcommand{\oU}{\operatorname{U}}

\newcommand{\g}{\mathfrak g}
\newcommand{\gC}{{\mathfrak g}_{\C}}
\renewcommand{\k}{\mathfrak k}
\newcommand{\h}{\mathfrak h}
\newcommand{\p}{\mathfrak p}
%\newcommand{\q}{\mathfrak q}
\renewcommand{\a}{\mathfrak a}
\renewcommand{\b}{\mathfrak b}
\renewcommand{\c}{\mathfrak c}
\newcommand{\n}{\mathfrak n}
\renewcommand{\u}{\mathfrak u}
%\renewcommand{\v}{\mathfrak v}
\newcommand{\e}{\mathfrak e}
\newcommand{\f}{\mathfrak f}
\renewcommand{\l}{\mathfrak l}
\renewcommand{\t}{\mathfrak t}
\newcommand{\s}{\mathfrak s}
\renewcommand{\r}{\mathfrak r}
\renewcommand{\o}{\mathfrak o}
\newcommand{\m}{\mathfrak m}
\newcommand{\z}{\mathfrak z}
%\renewcommand{\sl}{\mathfrak s \mathfrak l}
\newcommand{\gl}{\mathfrak g \mathfrak l}


\newcommand{\re}{\mathrm e}

\renewcommand{\rk}{\mathrm k}

\newcommand{\Z}{\mathbb{Z}}
\DeclareDocumentCommand{\C}{}{\mathbb{C}}
\newcommand{\R}{\mathbb R}
\newcommand{\Q}{\mathbb Q}
\renewcommand{\H}{\mathbb{H}}
%\newcommand{\N}{\mathbb{N}}
\newcommand{\K}{\mathbb{K}}
%\renewcommand{\S}{\mathbf S}
\newcommand{\M}{\mathbf{M}}
\newcommand{\A}{\mathbb{A}}
\newcommand{\B}{\mathbf{B}}
%\renewcommand{\G}{\mathbf{G}}
\newcommand{\V}{\mathbf{V}}
\newcommand{\W}{\mathbf{W}}
\newcommand{\F}{\mathbf{F}}
\newcommand{\E}{\mathbf{E}}
%\newcommand{\J}{\mathbf{J}}
\renewcommand{\H}{\mathbf{H}}
\newcommand{\X}{\mathbf{X}}
\newcommand{\Y}{\mathbf{Y}}
%\newcommand{\RR}{\mathcal R}
\newcommand{\FF}{\mathcal F}
%\newcommand{\BB}{\mathcal B}
\newcommand{\HH}{\mathcal H}
%\newcommand{\UU}{\mathcal U}
%\newcommand{\MM}{\mathcal M}
%\newcommand{\CC}{\mathcal C}
%\newcommand{\DD}{\mathcal D}
\def\eDD{\mathrm{d}^{e}}
\def\DD{\nabla}
\def\DDD{{\check\nabla}}
\def\DDc{\boldsymbol{\nabla}}
\def\gDD{\nabla^{\mathrm{gen}}}
\def\gDDc{\boldsymbol{\nabla}^{\mathrm{gen}}}
%\newcommand{\OO}{\mathcal O}
%\newcommand{\ZZ}{\mathcal Z}
\newcommand{\ve}{{\vee}}
\newcommand{\aut}{\mathcal A}
\newcommand{\ii}{\mathbf{i}}
\newcommand{\jj}{\mathbf{j}}
\newcommand{\kk}{\mathbf{k}}

\newcommand{\la}{\langle}
\newcommand{\ra}{\rangle}
\newcommand{\bp}{\bigskip}
\newcommand{\be}{\begin {equation}}
\newcommand{\ee}{\end {equation}}

\newcommand{\LRleq}{\stackrel{LR}{\leq}}

\numberwithin{equation}{section}


\def\flushl#1{\ifmmode\makebox[0pt][l]{${#1}$}\else\makebox[0pt][l]{#1}\fi}
\def\flushr#1{\ifmmode\makebox[0pt][r]{${#1}$}\else\makebox[0pt][r]{#1}\fi}
\def\flushmr#1{\makebox[0pt][r]{${#1}$}}


%\theoremstyle{Theorem}
% \newtheorem*{thmM}{Main Theorem}
% \crefformat{thmM}{main theorem}
% \Crefformat{thmM}{Main Theorem}
\newtheorem*{thm*}{Theorem}
\newtheorem{thm}{Theorem}[section]
\newtheorem{thml}[thm]{Theorem}
\newtheorem{lem}[thm]{Lemma}
\newtheorem{obs}[thm]{Observation}
\newtheorem{lemt}[thm]{Lemma}
\newtheorem*{lem*}{Lemma}
\newtheorem{whyp}[thm]{Working Hypothesis}
\newtheorem{prop}[thm]{Proposition}
\newtheorem{prpt}[thm]{Proposition}
\newtheorem{prpl}[thm]{Proposition}
\newtheorem{cor}[thm]{Corollary}
%\newtheorem*{prop*}{Proposition}
\newtheorem{claim}{Claim}
\newtheorem*{claim*}{Claim}
%\theoremstyle{definition}
\newtheorem{defn}[thm]{Definition}
\newtheorem{dfnl}[thm]{Definition}
\newtheorem*{IndH}{Induction Hypothesis}

\newtheorem*{eg*}{Example}
\newtheorem{eg}[thm]{Example}

\theoremstyle{remark}
\newtheorem{remark}[thm]{Remark}
\newtheorem{remarks}[thm]{Remarks}


\def\cpc{\sigma}
\def\ccJ{\epsilon\dotepsilon}
\def\ccL{c_L}

\def\wtbfK{\widetilde{\bfK}}
%\def\abfV{\acute{\bfV}}
\def\AbfV{\acute{\bfV}}
%\def\afgg{\acute{\fgg}}
%\def\abfG{\acute{\bfG}}
\def\abfV{\bfV'}
\def\afgg{\fgg'}
\def\abfG{\bfG'}

\def\half{{\tfrac{1}{2}}}
\def\ihalf{{\tfrac{\mathbf i}{2}}}
\def\slt{\fsl_2(\bC)}
\def\sltr{\fsl_2(\bR)}

% \def\Jslt{{J_{\fslt}}}
% \def\Lslt{{L_{\fslt}}}
\def\slee{{
\begin{pmatrix}
 0 & 1\\
 0 & 0
\end{pmatrix}
}}
\def\slff{{
\begin{pmatrix}
 0 & 0\\
 1 & 0
\end{pmatrix}
}}\def\slhh{{
\begin{pmatrix}
 1 & 0\\
 0 & -1
\end{pmatrix}
}}
\def\sleei{{
\begin{pmatrix}
 0 & i\\
 0 & 0
\end{pmatrix}
}}
\def\slxx{{\begin{pmatrix}
-\ihalf & \half\\
\phantom{-}\half & \ihalf
\end{pmatrix}}}
% \def\slxx{{\begin{pmatrix}
% -\sqrt{-1}/2 & 1/2\\
% 1/2 & \sqrt{-1}/2
% \end{pmatrix}}}
\def\slyy{{\begin{pmatrix}
\ihalf & \half\\
\half & -\ihalf
\end{pmatrix}}}
\def\slxxi{{\begin{pmatrix}
+\half & -\ihalf\\
-\ihalf & -\half
\end{pmatrix}}}
\def\slH{{\begin{pmatrix}
   0   & -\mathbf i\\
\mathbf i & 0
\end{pmatrix}}
}

\ExplSyntaxOn
\clist_map_inline:nn {J,L,C,X,Y,H,c,e,f,h,}{
  \expandafter\def\csname #1slt\endcsname{{\mathring{#1}}}}
\ExplSyntaxOff


\def\Mop{\fT}

\def\fggJ{\fgg_J}
\def\fggJp{\fgg'_{J'}}

\def\NilGC{\Nil_{\bfG}(\fgg)}
\def\NilGCp{\Nil_{\bfG'}(\fgg')}
\def\Nilgp{\Nil_{\fgg'_{J'}}}
\def\Nilg{\Nil_{\fgg_{J}}}
%\def\NilP'{\Nil_{\fpp'}}
\def\nNil{\Nil^{\mathrm n}}
\def\eNil{\Nil^{\mathrm e}}


\NewDocumentCommand{\NilP}{t'}{
\IfBooleanTF{#1}{\Nil_{\fpp'}}{\Nil_\fpp}
}

\def\KS{\mathsf{KS}}
\def\MM{\bfM}
\def\MMP{M}

\NewDocumentCommand{\KTW}{o g}{
  \IfValueTF{#2}{
    \left.\varsigma_{\IfValueT{#1}{#1}}\right|_{#2}}{
    \varsigma_{\IfValueT{#1}{#1}}}
}
\def\IST{\rho}
\def\tIST{\trho}

\NewDocumentCommand{\CHI}{o g}{
  \IfValueTF{#1}{
    {\chi}_{\left[#1\right]}}{
    \IfValueTF{#2}{
      {\chi}_{\left(#2\right)}}{
      {\chi}}
  }
}
\NewDocumentCommand{\PR}{g}{
  \IfValueTF{#1}{
    \mathop{\pr}_{\left(#1\right)}}{
    \mathop{\pr}}
}
\NewDocumentCommand{\XX}{g}{
  \IfValueTF{#1}{
    {\cX}_{\left(#1\right)}}{
    {\cX}}
}
\NewDocumentCommand{\PP}{g}{
  \IfValueTF{#1}{
    {\fpp}_{\left(#1\right)}}{
    {\fpp}}
}
\NewDocumentCommand{\LL}{g}{
  \IfValueTF{#1}{
    {\bfL}_{\left(#1\right)}}{
    {\bfL}}
}
\NewDocumentCommand{\ZZ}{g}{
  \IfValueTF{#1}{
    {\cZ}_{\left(#1\right)}}{
    {\cZ}}
}

\NewDocumentCommand{\WW}{g}{
  \IfValueTF{#1}{
    {\bfW}_{\left(#1\right)}}{
    {\bfW}}
}




\def\gpi{\wp}
\NewDocumentCommand\KK{g}{
\IfValueTF{#1}{K_{(#1)}}{K}}
% \NewDocumentCommand\OO{g}{
% \IfValueTF{#1}{\cO_{(#1)}}{K}}
\NewDocumentCommand\XXo{d()}{
\IfValueTF{#1}{\cX^\circ_{(#1)}}{\cX^\circ}}
\def\bfWo{\bfW^\circ}
\def\bfWoo{\bfW^{\circ \circ}}
\def\bfWg{\bfW^{\mathrm{gen}}}
\def\Xg{\cX^{\mathrm{gen}}}
\def\Xo{\cX^\circ}
\def\Xoo{\cX^{\circ \circ}}
\def\fppo{\fpp^\circ}
\def\fggo{\fgg^\circ}
\NewDocumentCommand\ZZo{g}{
\IfValueTF{#1}{\cZ^\circ_{(#1)}}{\cZ^\circ}}

% \ExplSyntaxOn
% \NewDocumentCommand{\bcO}{t' E{^_}{{}{}}}{
%   \overline{\cO\sb{\use_ii:nn#2}\IfBooleanTF{#1}{^{'\use_i:nn#2}}{^{\use_i:nn#2}}
%   }
% }
% \ExplSyntaxOff

\NewDocumentCommand{\bcO}{t'}{
  \overline{\cO\IfBooleanT{#1}{'}}}

\NewDocumentCommand{\oliftc}{g}{
\IfValueTF{#1}{\boldsymbol{\vartheta} (#1)}{\boldsymbol{\vartheta}}
}
\NewDocumentCommand{\oliftr}{g}{
\IfValueTF{#1}{\vartheta_\bR(#1)}{\vartheta_\bR}
}
\NewDocumentCommand{\olift}{g}{
\IfValueTF{#1}{\vartheta(#1)}{\vartheta}
}
% \NewDocumentCommand{\dliftv}{g}{
% \IfValueTF{#1}{\ckvartheta(#1)}{\ckvartheta}
% }
\def\dliftv{\vartheta}
\NewDocumentCommand{\tlift}{g}{
\IfValueTF{#1}{\wtvartheta(#1)}{\wtvartheta}
}

\def\slift{\cL}

\def\BB{\bB}


\def\PhiO#1{\vartheta\left(#1\right)}

\def\bbThetav{\check{\mathbbold{\Phi}}}
\def\Phiv{\check{\Phi}}
\def\Phiv{\check{\Phi}}

\DeclareDocumentCommand{\NN}{g}{
\IfValueTF{#1}{\fN(#1)}{\fN}
}
\DeclareDocumentCommand{\RR}{m m}{
\fR({#1},{#2})
}

%\DeclareMathOperator*{\sign}{Sign}

\NewDocumentCommand{\lsign}{m}{
{}^l\mathrm{Sign}(#1)
}



\NewDocumentCommand\lnn{t+ t- g}{
  \IfBooleanTF{#1}{{}^l n^+\IfValueT{#3}{(#3)}}{
    \IfBooleanTF{#2}{{}^l n^-\IfValueT{#3}{(#3)}}{}
  }
}


% % Fancy bcO, support feature \bcO'^a_b = \overline{\cO'^a_b}
% \makeatletter
% \def\bcO{\def\O@@{\cO}\@ifnextchar'\@Op\@Onp}
% \def\@Opnext{\@ifnextchar^\@Opsp\@Opnsp}
% \def\@Op{\afterassignment\@Opnext\let\scratch=}
% \def\@Opnsp{\def\O@@{\cO'}\@Otsb}
% \def\@Onp{\@ifnextchar^\@Onpsp\@Otsb}
% \def\@Opsp^#1{\def\O@@{\cO'^{#1}}\@Otsb}
% \def\@Onpsp^#1{\def\O@@{\cO^{#1}}\@Otsb}
% \def\@Otsb{\@ifnextchar_\@Osb{\@Ofinalnsb}}
% \def\@Osb_#1{\overline{\O@@_{#1}}}
% \def\@Ofinalnsb{\overline{\O@@}}

% Fancy \command: \command`#1 will translate to {}^{#1}\bfV, i.e. superscript on the
% lift conner.

% \def\defpcmd#1{
%   \def\nn@tmp{#1}
%   \def\nn@np@tmp{@np@#1}
%   \expandafter\let\csname\nn@np@tmp\expandafter\endcsname \csname\nn@tmp\endcsname
%   \expandafter\def\csname @pp@#1\endcsname`##1{{}^{##1}{\csname @np@#1\endcsname}}
%   \expandafter\def\csname #1\endcsname{\,\@ifnextchar`{\csname
%       @pp@#1\endcsname}{\csname @np@#1\endcsname}}
% }

% \def\defppcmd#1{
% \expandafter\NewDocumentCommand{\csname #1\endcsname}{##1 }{}
% }



% \defpcmd{bfV}
% \def\KK{\bfK}\defpcmd{KK}
% \defpcmd{bfG}
% \def\A{\!A}\defpcmd{A}
% \def\K{\!K}\defpcmd{K}
% \def\G{G}\defpcmd{G}
% \def\J{\!J}\defpcmd{J}
% \def\L{\!L}\defpcmd{L}
% \def\eps{\epsilon}\defpcmd{eps}
% \def\pp{p}\defpcmd{pp}
% \defpcmd{wtK}
% \makeatother

\def\fggR{\fgg_\bR}
\def\rmtop{{\mathrm{top}}}
\def\dimo{\dim^\circ}
\def\GKdim{\text{GK-}\dim}

\NewDocumentCommand\LW{g}{
\IfValueTF{#1}{L_{W_{#1}}}{L_{W}}}
%\def\LW#1{L_{W_{#1}}}
\def\JW#1{J_{W_{#1}}}

\def\floor#1{{\lfloor #1 \rfloor}}

\def\KSP{K}
\def\UU{\rU}
\def\UUC{\rU_\bC}
\def\tUUC{\widetilde{\rU}_\bC}
\def\OmegabfW{\Omega_{\bfW}}


\def\BB{\bB}


\def\PhiO#1{\vartheta\left(#1\right)}

\def\Phiv{\check{\Phi}}
\def\Phiv{\check{\Phi}}

\def\Phib{\bar{\Phi}}

\def\cKaod{\cK^{\mathrm{aod}}}

\DeclareMathOperator{\sspan}{span}


\def\sp{{\mathrm{sp}}}

\def\bfLz{\bfL_0}
\def\sOpe{\sO^\perp}
\def\sOpeR{\sO^\perp_\bR}
\def\sOR{\sO_\bR}

\def\ZX{\cZ_{X}}
\def\gdliftv{\vartheta}
\def\gdlift{\vartheta^{\mathrm{gen}}}
\def\bcOp{\overline{\cO'}}
\def\bsO{\overline{\sO}}
\def\bsOp{\overline{\sO'}}
\def\bfVpe{\bfV^\perp}
\def\bfEz{\bfE_0}
\def\bfVn{\bfV^-}
\def\bfEzp{\bfE'_0}

\def\totimes{\widehat{\otimes}}
\def\dotbfV{\dot{\bfV}}

\def\aod{\mathrm{aod}}
\def\unip{\mathrm{unip}}
\def\IC{\mathfrak{I}}

\def\PI#1{\Pi_{\cI_{#1}}}
\def\Piunip{\Pi^{\mathrm{unip}}}
\def\cf{\emph{cf.} }
\def\Groth{\mathrm{Groth}}
\def\Irr{\mathrm{Irr}}
\def\Irrsp{\mathrm{Irr}^{\text{sp}}}

\def\edrc{\mathrm{DRC}^{\mathrm e}}
\def\drc{\mathrm{DRC}}
\def\LS{\mathrm{LS}}
\def\Unip{\mathrm{Unip}}


% Ytableau tweak
\makeatletter
\pgfkeys{/ytableau/options,
  noframe/.default = false,
  noframe/.is choice,
  noframe/true/.code = {%
    \global\let\vrule@YT=\vrule@none@YT
    \global\let\hrule@YT=\hrule@none@YT
  },
  noframe/false/.code = {%
    \global\let\vrule@YT=\vrule@normal@YT
    \global\let\hrule@YT=\hrule@normal@YT
  },
  noframe/on/.style = {noframe/true},
  noframe/off/.style = {noframe/false},
}
\makeatother


\def\wAV{\AV^{\mathrm{weak}}}
\def\ckG{\check{G}}
\def\ckGc{\check{G}_{\bC}}
\def\dBV{d_{\mathrm{BV}}}
\def\CP{\mathsf{CP}}
\def\YD{\mathsf{YD}}
\def\SYD{\mathsf{SYD}}
\def\DD{\nabla}

\def\lamck{\lambda_\ckcO}
\def\Lamck{[\lambda_\ckcO]}
\def\lamckb{\lambda_{\ckcO_b}}
\def\lamckg{\lambda_{\ckcO_g}}
\def\Wint#1{W_{[#1]}}
\def\CLam{\Coh_{\Lambda}}
\def\Cint#1{\Coh_{[#1]}}
\def\PP{\mathsf{PAP}}
\def\PAP{\mathsf{PAP}}
\def\BOX#1{\mathrm{Box}(#1)}
\DeclareDocumentCommand{\bigtimes}{}{\mathop{\scalebox{1.7}{$\times$}}}
\providecommand\mapsfrom{\scalebox{-1}[1]{$\mapsto$}}

\def\ihh{{i_\fhh}}

\def\Gc{G_\bC}
\def\Gcad{G_\bC^{\text{ad}}}
\def\Gad{\Inn(\fgg)}

\def\hha{{}^a\fhh}
\def\ahh{\hha}
\def\aSR{{}^a\Sigma}
\def\aRp{{}^a\Delta^+}
\def\aX{{}^aX}
\def\aQ{{}^aQ}
\def\aP{{}^aP}
\def\aR{{}^aR}
\def\aRp{{}^aR^+}
\def\asRp{{}^a \Delta^+}
\def\Gfin{\cG(\Gc)}
\def\PiGfin{\Pi_{\mathrm{fin}}( \Gc )}
\def\PiGlfin{\Pi_{\Lambda_0}( \Gc )}
\def\adGfin{\cG_{\mathrm{ad}}(\Gc)}
\def\Ggk{\cG(\fgg,K)}
\def\WT#1{\Delta(#1)}
\def\WG{W(\Gc)}
\def\ch{\mathrm{ch}\,}
\def\Wlam{W_{[\lambda]}}
\def\aLam{a_{\Lambda}}
\def\WLam{W_{\Lam}}
\def\WLamck{W_{[\lambda_{\ckcO}]}}
\def\Wlamck{W_{\lamck}}
\def\Rlam{R_{[\lambda]}}
\def\RLam{R_\Lambda}
\def\RLamp{R_\Lambda^+}
\def\Rplam{R^+(\lambda)}
\def\Glfin{\cG_{\Lambda}(\Gc)}
\def\CL{{\sC}^{\scriptscriptstyle L}}
\def\CR{{\sC}^{\scriptscriptstyle R}}
\def\CLR{{\sC}^{\scriptscriptstyle LR}}
\def\LV{{}^{\scriptscriptstyle L}\sV}
\def\LC{{}^{\scriptscriptstyle L}\sC}
\def\RC{{}^{\scriptscriptstyle R}\sC}
\def\LRC{{}^{\scriptscriptstyle LR}\sC}
\def\ckLC{{}^{\scriptscriptstyle L}\check{\sC}}

\def\LV{{}^{\scriptscriptstyle L}\sV}
\def\ckLV{{}^{\scriptscriptstyle L}\check\sV}

\def\tLV{{}^{\scriptscriptstyle L}\widetilde{\sV}}
\def\tLC{{}^{\scriptscriptstyle L}\widetilde{\sC}}

\def\brsgn{\breve{\sgn}}
\def\bsgn{\overline{\sgn}}

\def\Wb{W_{b}}
\def\Wg{W_{g}}


\def\nbb{n_{\mathrm b}}
\def\ngg{n_{\mathrm g}}
\def\tU{\widetilde{\rU}}

\newcommand{\cross}{\times}
\newcommand{\crossa}{\times^a}

\def\bVL{{\overline{\sV}}^{\scriptscriptstyle L}}
\def\bVR{{\overline{\sV}}^{\scriptscriptstyle R}}
\def\bVLR{{\overline{\sV}}^{\scriptscriptstyle LR}}
\def\VL{{\sV}^{\scriptscriptstyle L}}
\def\VR{{\sV}^{\scriptscriptstyle R}}
\def\VLR{{\sV}^{\scriptscriptstyle LR}}

\def\Con{\sfC}
\def\bCon{\overline{\sfC}}
\def\Re{\mathrm{Re}}
\def\Im{\mathrm{Im}}
\def\AND{\quad \text{and} \quad}
\def\Coh{\mathrm{Coh}}
\def\Cohlm{\Coh_{\Lambda}(\cM)}
\def\ev#1{{\mathrm{ev}_{#1}}}

\def\ppp{\times}
\def\mmm{\slash}


\def\cuprow{{\stackrel{r}{\sqcup}}}
\def\cupcol{{\stackrel{c}{\sqcup}}}

\def\Spr{\mathrm{Springer}}
\def\Prim{\mathrm{Prim}}



\def\imathp{\imath_{\sP}}
\def\jmathp{\jmath_{\sP}}

\def\CQ{\overline{\sfA}}% Lusztig's canonical quotient
\def\CPP{\mathrm{PP}}
\def\CPPs{\mathrm{PP}_{\star}}
%\def\CPP{\mathfrak{P}}
%\def\CPPs{\mathfrak{P}_{\star}}


\def\ceil#1{\lceil #1 \rceil}
\def\symb#1#2{{\left(\substack{{#1}\\{#2}}\right)}}
\def\cboxs#1{\mbox{\scalebox{0.25}{\ytb{\ ,\vdots,\vdots,\ }}}_{#1}}

\def\hsgn{\widetilde{\mathrm{sgn}}}

\def\tPBP{\widetilde{\mathsf{PBP}}}
\def\PBPe{\mathsf{PBP}^{\mathrm{ext}}}
\def\PBPes{\mathsf{PBP}^{\mathrm{ext}}_{\star}}
\def\PBPsb{\mathsf{PBP}_{\star,b}}

\def\bev#1{\overline{\mathrm{ev}}_{#1}}

\def\Prim{\mathrm{Prim}}
% \def\leqL{\stackrel{L}{\leq}}
% \def\leqR{\stackrel{R}{\leq}}
% \def\leqLR{\stackrel{LR}{\leq}}

% \def\leqL{{\leq_L}}
% \def\leqR{{\leq_R}}
% \def\leqLR{{\leq_{LR}}}


\def\Dsp{\cD^{\text{sp}}}
\def\Csp{\sfC^{\text{sp}}}


\def\lneqL{\mathrel{\mathop{<}\limits_{\scriptscriptstyle L}}}
\def\lneqR{\mathrel{\mathop{<}\limits_{\scriptscriptstyle R}}}
\def\lneqLR{\mathrel{\mathop{<}\limits_{\scriptscriptstyle LR}}}

\def\leqL{\mathrel{\mathop{\leq}\limits_{\scriptscriptstyle L}}}
\def\leqR{\mathrel{\mathop{\leq}\limits_{\scriptscriptstyle R}}}
\def\leqLR{\mathrel{\mathop{\leq}\limits_{\scriptscriptstyle LR}}}


\def\approxL{\mathrel{\mathop{\approx}\limits_{\scriptscriptstyle L}}}
\def\approxR{\mathrel{\mathop{\approx}\limits_{\scriptscriptstyle R}}}
\def\approxLR{\mathrel{\mathop{\approx}\limits_{\scriptscriptstyle LR}}}


\def\dphi{\rdd \phi}
\def\CPH{C(H)}
\def\whCPH{\widehat{C(H)}}

\def\Greg{G_{\text{reg}}}
\def\Hnreg{H^-_{\text{reg}}}
\def\Hireg{H_{i,\text{reg}}}
\def\Hnireg{H^-_{i,\text{reg}}}


\def\tsgn{\widetilde{\sgn}}
\def\PBP{\mathsf{PBP}}

\def\ckstar{{\check \star}}
\def\ckfgg{{\check \fgg}}

\def\Inn{\mathrm{Inn}}

\providecommand{\nsubset}{\not\subset}

\def\cuprow{{\,\stackrel{r}{\sqcup}\,}}
\def\cupcol{{\,\stackrel{c}{\sqcup}\,}}

\def\ckcOb{\ckcO_{\mathrm b}}
\def\ckcOpb{\ckcO'_{\mathrm b}}
\def\cOpb{\cO'_{\mathrm b}}
\def\ckcOg{\ckcO_{\mathrm g}}

\def\Gb{G_{\mathrm b}}
\def\Gpb{G'_{\mathrm b}}
\def\Gg{G_{\mathrm g}}




\def\tPBP{\widetilde{\mathsf{PBP}}}
\def\PBPs{\mathsf{PBP}_{\star}}
\def\PBPe{\mathsf{PBP}^{\mathrm{ext}}}
\def\PBPes{\mathsf{PBP}^{\mathrm{ext}}_{\star}}
\def\PBPsb{\mathsf{PBP}_{\star,b}}

\newcommand{\Lam}{{[\lambda]}}

\newcommand{\Rg}{\cR(\fgg)}
\newcommand{\Grt}{\cK}
\newcommand{\nckG}{n_{\ckG}}
%\newcommand{\nb}{n_{\mathrm b}}
%\newcommand{\ng}{n_{\mathrm g}}

\usepackage{xr}
\usepackage{subfiles} % Best loaded last in the preamble



\begin{document}


\title[]{Counting special unipotent representations of real classical groups}

\author [D. Barbasch] {Dan M. Barbasch}
\address{the Department of Mathematics\\
  310 Malott Hall, Cornell University, Ithaca, New York 14853 }
\email{dmb14@cornell.edu}

\author [J.-J. Ma] {Jia-jun Ma}
\address{School of Mathematical Sciences\\
  Xiamen University\\
  Xiamen, China} \email{hoxide@xmu.edu.cn}

\author [B. Sun] {Binyong Sun}
% MCM, HCMS, HLM, CEMS, UCAS,
\address{Academy of Mathematics and Systems Science\\
  Chinese Academy of Sciences\\
  Beijing, 100190, China} \email{sun@math.ac.cn}

\author [C.-B. Zhu] {Chen-Bo Zhu}
\address{Department of Mathematics\\
  National University of Singapore\\
  10 Lower Kent Ridge Road, Singapore 119076} \email{matzhucb@nus.edu.sg}




\subjclass[2000]{22E45, 22E46} \keywords{orbit method, unitary dual, unipotent
  representation, classical group, theta lifting, moment map}

% \thanks{Supported by NSFC Grant 11222101}
\maketitle


\tableofcontents



\section{Introduction and the main results}

Let $G$ be a real reductive group in the Harish-Chandra class (which may be
linear or non-linear). Write $\fgg$ for the complexified Lie algebra of $G$ and
let $\hha$ denote its universal Cartan subalgebra (also called the abstract Cartan subalgbra in \cite{V4}).
Let $\lambda \in \hha^*$ (a superscript $*$ indicates the dual space). By Harish-Chandra isomorphism, it
determines an algebraic character $\chi_\lambda: \CZ(\g)\rightarrow \C$. Here
$\CZ(\g)$ denotes the center of the universal enveloping algebra
$\mathcal U(\g)$. Denote by $\Irr(G)$ the set of isomorphism classes of
irreducible Casselman-Wallach representations of $G$, and by $\Irr_\lambda(G)$
its subset consisting of the representations with infinitesimal character
$\chi_\lambda$ (or simply $\lambda$). The latter set has finite cardinality.


Let $\Nil(\g^*)$ denote the set of nilpotent elements in $\g^*$. It has only
finitely many orbits under the coadjoint action of the inner automorphism group
$\mathrm{Inn}(\g)$ of $\g$. Let $\sfS$ be an $\mathrm{Inn}(\g)$-stable Zariski
closed subset of $\mathrm{Nil}(\g^*)$. Put
\[
  \Irr_{\lambda,\sfS}(G):=\Set{\pi \in \Irr_{\lambda}(G)| \text{$\AVC(\pi)\subset \sfS$} }.
\]
Here $\AVC(\pi)$ denotes the complex associated variety of $\pi$, namely the
associated variety of the annihilate ideal of $\pi$. It is an
$\mathrm{Inn}(\g)$-stable Zariski closed subset of $\mathrm{Nil}(\g^*)$. An
interesting problem of representation theory is to count the finite set $\Irr_{\lambda,\sfS}(G)$. The coherent continuation
representation (of an appropriate Weyl group) provides a powerful tool for this problem.

\subsection{The coherent continuation representation}\label{sec11}


Write $\mathrm{Rep}(G)$ for the category of Casselman-Wallach representations of $G$, and write $\CK(G)$ for the for the
Grothendieck group of this category.  Throughout this article we take $\C$ as the coefficient ring to define  Grothendieck groups.



Write $\mathrm{Rep}_{\lambda, \sfS}(G)$ for the full subcategory of $\mathrm{Rep}(G)$ whose objects are the representations that have
generalized infinitesimal character $\lambda$ and whose complex associated
variety is contained in $\sfS$. Write $\CK_{\lambda,\sfS}(G)$ for the
Grothendieck group of this category. Then
\[
  \sharp (\Irr_{\lambda,\sfS}(G))=\dim \CK_{\lambda,\sfS}(G)\qquad(\sharp\textrm{
    indicates the cardinality of a finite set}).
\]
We also have that
\[
  \CK_\sfS(G)=\bigoplus_{\mu\in W\backslash \h^*} \CK_{\mu,\sfS}(G)\qquad (W\textrm{
    denotes the Weyl group}),
\]
where $\CK_{\sfS}(G)$ is the Grothendieck group of the
category of Casselman-Wallach representations of $G$ whose complex associated
variety is contained in $\sfS$.


Let $\Rg$ be the Grothendieck group of the category of finite-dimensional algebraic
representations of $\mathrm{Inn}(\fgg)$. It is
 a commutative $\bC$-algebra under the tensor
product of representations.
Write \[
\Delta\subset Q \quad (\subset \hha^*)
\] for the root system and the root lattice of
$\fgg$, respectively.
%and identified with $\bC[Q/W]$.
By pulling back through the adjoint representation
$G\rightarrow \mathrm{Inn}(\g)$, every algebraic representation of $\mathrm{Inn}(\g)$ is viewed as a representation of $G$.
Under the tensor product of representations, $\CK_\sfS(G)$ is naturally a $\Rg$-module.


Put
\[
\Lam:=\lambda+Q\subset \hha^*,
\]
 and write $W_\Lam$
for its stabilizer in $W$. Then $W_\Lam$ equals the Weyl group of the root
system
\[
  \{\alpha \in \Delta\mid \langle \lambda, \alpha^\vee\rangle \in \Z\}\qquad (\alpha^\vee \textrm{ denotes the coroot corresponding to $\alpha$}).
\]


\begin{defn}\label{defcoh}
  Let $\CK$ be a $\mathcal R(\g)$-module equipped with a family
  $\{\CK_\mu\}_{\mu\in \Lam}$ of subspaces such that $\CK_{w\cdot \mu}=\CK_\mu$
  for all $w\in W_\Lam$ and $\mu\in \Lam$. A $\CK$-valued coherent family on
  $\Lam$ is a map
  \[
    \Phi: \Lam\rightarrow \CK%, \qquad \mu\mapsto \Phi_\mu
  \]
  satisfying the following two conditions:
  \begin{itemize}
    \item for all $\mu\in \Lam$, $\Phi(\mu)\in \CK_\mu$;
    \item for all finite-dimensional algebraic representations $F$ of $\mathrm{Inn}(\g)$
          and all $\mu\in \Lam$,
          \[
          F\cdot (\Phi(\mu)) = \sum_{\nu} \Phi(\mu+\nu),
          \]
          where $\nu$ runs over all weights of $F$, counted with multiplicities,
          and $F$ is viewed as an element of $\mathcal R(\g)$.
  \end{itemize}
\end{defn}


In the notation of \Cref{defcoh}, let $\Coh_{\Lam}(\CK)$ denote the
vector space of all $\mathcal K$-valued coherent families on $\Lam$. It is a
representation of $W_{\Lam}$ under the action
\[
  (w\cdot \Phi)(\mu) = \Phi(w^{-1}\cdot \mu), \qquad \textrm{for all
  }\ w\in W_\Lam, \ \mu\in \Lam.
\]
To shorten the notation, put
\[
  \Coh_{\Lam,\sfS}(G):=\Coh_{\Lam}(\CK_\sfS(G)).
\]
Here the $\Rg$-module $\CK_\sfS(G)$ is equipped with the family $\{\CK_{\mu, \sfS}(G)\}_{\mu\in \Lam}$ of subspaces.

\subsection{Counting irreducible representations with a bounded complex
  associated variety}\label{sec12}

Denote by $W_\lambda$ the stabilizer of
$\lambda$ in $W$. Then $W_\lambda\subset W_\Lam$. Write $1_{W_\lambda}$ for the
trivial representation of $W_{\lambda}$.

Our starting point is the following theorem of Vogan. We will provide a proof due to lack of a convenient reference.
\begin{thm}[Vogan]\label{count1}
  The equality
  \[
    \sharp(\Irr_{\lambda,\sfS}(G)) = [1_{W_{\lambda}}:\Coh_{\Lam,\sfS}(G)]
  \]
  holds.
  % \[
  %   \dim {\barmu} = \dim (\cohm)_{W_\mu} = [\cohm, 1_{W_\mu}].
  % \]
\end{thm}
Here and henceforth, $[\ : \ ]$ indicates the multiplicity of the first
(irreducible) representation in the second one. Theorem \ref{count1} implies
that
\begin{equation}\label{countlg}
  \sharp(\Irr_{\lambda,\sfS}(G)) = \sum_{\sigma\in \Irr(W_\Lam)} [1_{W_{\lambda}}: \sigma]\cdot [\sigma: \Coh_{\Lam,\sfS}(G)].
\end{equation}
Thus it suffices to understand the multiplicity $ [\sigma: \Coh_{\Lam,\sfS}(G)]$
for every $\sigma\in \Irr(W_\Lam)$.

Let $\sigma\in \Irr(W_\Lam)$. Define the
nilpotent orbit
\[
  \CO_\sigma:=\mathrm{Springer}(j_{W_\Lam}^W \sigma_0)\subset \mathrm{Nil}(\g^*) \quad (``\mathrm{Springer}"\textrm{
    indicates the Springer correspondence}),
\]
where $\sigma_0$ denote the special irreducible representation of $W_\Lam$ that
lies in the same double cell as $\sigma$, and $j_{W_\Lam}^W \sigma_0$ denotes
the $j$-induction of $\sigma _0$, which is an irreducible representation of $W$. See \cite[Chapter 11]{Carter} for more details on the $j$-induction.


Let
\be\label{sfc}
  \sfC_{\sfS}:= \Set{\sigma\in \Irr(\WLam)| \cO_{\sigma}\subset \sfS}.
\ee

\begin{thm}\label{count2}
  Suppose that $\sigma\in \Irr(W_\Lam)$ and $\sigma \notin \sfC_{\sfS}$. Then
  %$\CO_\sigma\nsubset \sfS$.
  \[
    [\sigma: \Coh_{\Lam,\sfS}(G)]=0.
  \]

  % \[
  %   \dim {\barmu} = \dim (\cohm)_{W_\mu} = [\cohm, 1_{W_\mu}].
  % \]
\end{thm}


Combining Theorems \ref{count1} and \ref{count2}, we conclude that
\begin{equation}\label{leq2}
  \sharp(\Irr_{\lambda,\sfS}(G)) = \sum_{\sigma \in \sfC_{\sfS}} [1_{W_{\lambda}}: \sigma]\cdot [\sigma: \Coh_{\Lam,\sfS}(G)].
\end{equation}



% the equality always holds. Combining Theorem \ref{count1}, Theorem
% \ref{count2} and \eqref{leq1}, we conclude that
% \begin{equation}\label{leq2}
%   \sharp(\Irr_{\lambda,\sfS}(G)) \leq \sum_{\sigma\in \Irr(W_\Lam), \CO_\sigma\subset \sfS} [1_{W_{\lambda}}: \sigma]\cdot [\sigma: \Coh_{\Lam}(G)].
% \end{equation}





\subsection{Counting irreducible representations annihilated by a maximal primitive ideal}\label{sec13}
Write $I_\lambda$ for the maximal ideal of $\mathcal U(\g)$ with infinitesimal
character $\lambda$. Its associated variety equals the Zariski closure
$\overline{\CO_\lambda}$ of an $\mathrm{Inn}(\g)$-orbit
$\CO_\lambda\subset\mathrm{Nil}(\g^*) $. Note that an irreducible
Casselman-Wallach representation of $G$ lies in
$\Irr_{\lambda,\overline{\CO_\lambda}}(G)$ if and only if it is annihilated by
$I_\lambda$.


Let
\[
  \LC_{\lambda}:= \Set{\sigma\in \Irr(W_\Lam) | \sigma \text{ occurs in $(J_{W_{\lambda}}^{W_{\Lam}} \sgn )\otimes \sgn$}}.
\]
Here $J_{W_{\lambda}}^{W_{\Lam}} $ indicates the $J$-induction (see \cite[Chapter 12]{Carter}), and $\sgn$
denotes the sign character (of an appropriate Weyl group).

%Let $\LC_{\lambda}\subset \Irr(W_\Lam)$ be the subset consisting of all the
%irreducible representations that occur in
%\[
%  (J_{W_{\lambda}}^{W_{\Lam}} \sgn )\otimes \sgn,
%\]
%where $J_{W_{\lambda}}^{W_{\Lam}} $ indicates the $J$-induction (see \cite[Chapter 12]{Carter}), and $\sgn$
%denotes the sign character (of an appropriate Weyl group).



 \begin{prop}[{\cite{BVUni}*{(5.26), Proposition~5.28}}]\label{lem:lcell.BV0}
  The set
   \[
     \LC_{\lambda} = \Set{\sigma\in \mathsf C_{\overline{\CO_\lambda}} |\,   [1_{W_{\lambda}}:\sigma]\neq 0}.
   \]
   Moreover, $[1_{W_{\lambda}}:\sigma]=1$ when
   $\sigma\in \LC_\lambda$.
 \end{prop}

Write
\[
  \Coh_{\Lam}(G):= \Coh_{\Lam,\mathrm{Nil}(\g^*)}(G).
\]
It contains $ \Coh_{\Lam,\sfS}(G)$ as a subrepresentation, and hence
\begin{equation}\label{leq111}
  [\sigma: \Coh_{\Lam,\sfS}(G)]\leq [\sigma: \Coh_{\Lam}(G)]\quad \textrm{ for all $\sigma\in \Irr(W_\Lam)$.}
\end{equation}

 Combining \eqref{leq2}, \eqref{leq111} and  Propositions \ref{lem:lcell.BV0}, we obtain the following inequality.

 \begin{cor}
   % Under the notation of \Cref{lem:lcell.BV}, we have
   The inequality
   \begin{equation}\label{boundc}
     \sharp(\Irr_{\lambda,\overline{\CO_\lambda}}(G)) \leq \sum_{\sigma\in \LC_\lambda} [\sigma: \Coh_{\Lam}(G)]
   \end{equation}
   holds.
 \end{cor}


Recall the notion of Harish-Chandra cells for $\Coh_{\Lam}(G)$ (which are called LR-cells for Harish-Chandra modules in \cite{V4}*{Section 14}). For every Harish-Chandra cell $C$  in $\Coh_{\Lam}(G)$, write $\CV(C)$ for the  Harish-Chandra cell representation attached to $C$, which is a subquotient representation of $\Coh_{\Lam}(G)$.

 \begin{thm}
   % Under the notation of \Cref{lem:lcell.BV}, we have
   Assume that for every Harish-Chandra cell $C$  in $\Coh_{\Lam}(G)$, the set $\{\sigma\in \Irr(W_{[\lambda]}\,|\, [\sigma: \CV(C)]\neq 0\}$ is contained in a Lustig double cell representation. Then the equality holds in
   \eqref{boundc}.
    \end{thm}



 \subsection{Special unipotent representations of classical groups}

 We are particularly interested in counting special unipotent representations of
 real classical groups.

 Let $\star$ be one of the 14 symbols
 \[
   \textrm{ $A$, $A^\C$, $A^\bH$, $A^*$, $B$, $D$, $B^\C$, $D^\C$, $C$, $C^\C$,
     $D^*$, $C^*$, $\wtC$, $\wtC^\C$. }
 \]
 Suppose that $G$ is a classical Lie group of type $\star$, namely $G$
 respectively equals one of the following Lie groups:
 \[
   \begin{array}{c}
     \GL_n(\R), \ \GL_n(\C), \  \GL_n(\bH),\  \oU(p,q),\smallskip\\
     \SO(p,q)\ (p+q\, \textrm{ is odd}),  \  \SO(p,q)\  (p+q\, \textrm{ is even}),\smallskip\\
     \SO_n(\C) \ (n\, \textrm{ is odd}),  \
     \SO_n(\C) \ (n\, \textrm{ is even}),\smallskip \\
     \Sp_{2n}(\R), \ \Sp_{2n}(\C), \  \oO^*(2n), \  \Sp(p,q),\   \widetilde \Sp_{2n}(\R), \ \Sp_{2n}(\C) \qquad (n, p, q\geq 0).
   \end{array}
 \]
 Here $\wtSp_{2n}(\R)$ denotes the metaplectic double cover of the symplectic
 group $\Sp_{2n}(\R)$ that does not split unless $n=0$.

  We define the Langlands dual $\ckG$ of $G$ to be respectively the complex group
 \[
   \begin{array}{c}
     \GL_n(\C), \ \GL_n(\C), \  \GL_{2n}(\C),\  \GL_{p+q}(\C),\smallskip\\
     \Sp_{p+q-1}(\C)\ (p+q\, \textrm{ is odd}),  \  \SO_{p+q}(\C)\  (p+q\, \textrm{ is even}),\smallskip\\
     \Sp_{n-1}(\C) \ (n\, \textrm{ is odd}),  \
     \SO_n(\C) \ (n\, \textrm{ is even}),\smallskip \\
     \SO_{2n+1}(\C), \ \SO_{2n+1}(\C), \  \SO_{2n}(\C), \  \SO_{2p+2q+1}(\C),\    \Sp_{2n}(\C), \  \textrm{or } \  \Sp_{2n}(\C).
   \end{array}
 \]
 Write $\check \g$ for the Lie algebra of $\ckG$, and let $\check \CO\subset\mathrm{Nil}(\check \g)$ be a nilpotent $\ckG$-orbit.

  Let $\lambda_{\ckcO}\in \check \g$ be half of the neutral element in any
 $\fsl_{2}$ triple attached to $\ckcO$, as in \cite[Section 5]{BVUni}. It is a semisimple element and is uniquely determined up to conjugation by $\ckG$.

  If $G$ is not a complex group, namely $\star\notin
 \{A^\C, B^\C, D^\C, C^\C, \wtC^C\}$, then we have an identification
 \[
   \ckG\backslash  \{\textrm{semisimple element in $\check \g$}\}=W\backslash \hha^*.
 \]
Under this identification, we view $\lambda_{\check \CO}$ as an element of $ W\backslash \hha^*$, and write $I_{\check \CO}:=I_{\star, \check \CO}$ for the maximal ideal of $\mathcal U(\g)$ with infinitesimal character $\lambda_{\check \CO}$.

 %By using Harish-Chandra isomorphism, we view $\lambda_{\check \CO}$ as a character $\lambda_{\check \CO}: \mathcal Z(\g)\rightarrow \C$.

 For every complex Lie group $H$, write $\overline H$ for a complex Lie group with an anti-holomorphism isomorphism $\overline{\phantom a} : H\rightarrow \overline H$. Similar notation applies to complex Lie algebras. If $G$ is a complex group, namely $\star\in
 \{A^\C, B^\C, D^\C, C^\C, \wtC^C\}$, then $\g \times \overline \g$ is a complexification of $\g$, and we have an identification
 \[
    \ckG\backslash  \{\textrm{semisimple element in $\check \g$}\}\times  \overline{\ckG}\backslash  \{\textrm{semisimple element in $\overline{\check \g}$}\}=W\backslash \hha^*.
 \]
 Under this identification, we view $(\lambda_{\check \CO}, \overline{\lambda_{\check \CO}})$ as an element of $ W\backslash \hha^*$, and write $I_{\check \CO}:=I_{\star, \check \CO}$ for the maximal ideal of $\mathcal U(\g)$ with infinitesimal character $(\lambda_{\check \CO}, \overline{\lambda_{\check \CO}})$.


 Finally, define the set of the special unipotent representations of $G$
 attached to $\ckcO$ by
 \[
   \begin{split}
     \Unip_{\ckcO}(G):=&  \Unip_{\star, \ckcO}(G) \\
     :=& \begin{cases}
       % \{\pi\in \Irr(G)\mid \pi \textrm{ is annihilated by $ I_{\check \CO}$ or $I'_{\check \CO}$}\}, & \text{if } \star \in \set{D, D^\C, D^*};\\
       \{\pi\in \Irr(G)\mid \pi \textrm{ is genuine  and annihilated by } I_{\check \CO}\}, & \text{if } \star =\widetilde C;\\
       \{\pi\in \Irr(G)\mid \pi \textrm{ is annihilated by } I_{\check \CO}\}, & otherwise.\\
     \end{cases}
   \end{split}
 \]
 Here ``genuine" means that the representation $\pi$ of
 $\widetilde \Sp_{2n}(\R)$ does not descend to $\Sp_{2n}(\R)$.
A main goal of the current paper is to parametrize the set $\Unip_{\check \CO}(G)$, which will be used by the authors to construct all the
 representations in this set (\cite{BMSZ2}).


\subsection{The cases of general linear groups and unitary groups}


  For a Young diagram $\imath$, write
 \[
   \mathbf r_1(\imath)\geq \mathbf r_2(\imath)\geq \mathbf r_3(\imath)\geq \cdots
 \]
 for its row lengths, and similarly, write
 \[
   \mathbf c_1(\imath)\geq \mathbf c_2(\imath)\geq \mathbf c_3(\imath)\geq \cdots
 \]
 for its column lengths. Denote by
 $\abs{\imath}:=\sum_{i=1}^\infty \mathbf r_i(\imath)$ the total size of
 $\imath$.


When no confusion is possible, we still use $\check \CO$ to denote the Young diagram attached to the nilpotent orbit $\check \CO$. Note that the Young diagram determines the nilpotent orbit unless $\check G=\SO_{4n}(\C)$ ($n\geq 1$) and all the row lengths are even.

Let $\bN^+$ denote the set of positive integers. For any Young diagram $\imath$, we introduce the set $\mathrm{Box}(\imath)$ of
boxes of $\imath$ as the following subset of $\bN^+\times \bN^+$:
\begin{equation}\label{eq:BOX}
  \mathrm{Box}(\imath):=\Set{(i,j)\in\bN^+\times \bN^+| j\leq \bfrr_i(\imath)}.
\end{equation}
% We will also call a subset of $\bN^+\times \bN^+$ of the form \eqref{eq:BOX} a
% Young diagram.

% We say that a Young diagram $\imath'$ is contained in $\imath$ (and write
% $\imath'\subset \imath$) if
% \[
%   \mathbf r_i(\imath')\leq \mathbf r_i(\imath)\qquad \textrm{for all
% } i=1,2, 3, \cdots.
% \]
% When this is the case, $\mathrm{Box}(\imath')$ is viewed as a subset of
% $\mathrm{Box}(\imath)$ concentrating on the upper-left corner. We say that a
% subset of $\mathrm{Box}(\imath)$ is a Young subdiagram if it equals
% $\mathrm{Box}(\imath')$ for a Young diagram $\imath'\subset \imath$. In this
% case, we call $\imath'$ the Young diagram corresponding to this Young
% subdiagram.

\renewcommand{\CP}{\mathcal{P}} We also introduce five symbols $\bullet$, $s$,
$r$, $c$ and $d$, and make the following definitions.
\begin{defn}
  A painting on a Young diagram $\imath$ is a map
  \[
    \mathcal P: \mathrm{Box}(\imath) \rightarrow \{\bullet, s, r, c, d \}
  \]
  with the following properties:
  \begin{itemize}
    \item $\mathcal P^{-1}(S)$ is the set of boxes of a Young diagram when
          $S=\{\bullet\}, \{\bullet, s \}, \{\bullet, s, r\}$ or
          $\{\bullet, s, r, c \} $;
    \item when $S=\{s\}$ or $ \{r\}$, every row of $\imath$ has at most one box
          in $\CP^{-1}(S)$;
    \item when $S=\{c\}$ or $ \{d \}$, every column of $\imath$ has at most one
          box in $\CP^{-1}(S)$.
  \end{itemize}
\end{defn}



\begin{defn}\label{defpbp0}
  Suppose that $\star\in \{A, A^\bH, A^*\}$. A painting $\CP$ on a Young diagram
  $\imath$ has type $\star$ if
  \begin{itemize}
    \item the image of $\CP$ is contained in
          \[
          \left\{
          \begin{array}{ll}
            \{\bullet, c, d\}, &\hbox{if $\star=A$}; \smallskip\\
            \{\bullet\}, &\hbox{if $\star=A^\bH$}; \smallskip\\
            \{\bullet, s, r\}, &\hbox{if $\star=A^*$},            \end{array}
        \right.
          \]
    \item if $\star=A$ or $A^\bH$, then $\CP^{-1}(\bullet)$ has even number of
          boxes in every column of $\imath$,
    \item if $\star=A^*$, then $\CP^{-1}(\bullet)$ has even number of boxes in
          every row of $\imath$.
  \end{itemize}
  Denote by $\PAP_\star(\imath)$ the set of paintings on $\imath^{t}$ that has type $\star$, where $\imath^{t}$
  is the transpose of $\imath$.
   \end{defn}

%Note that in the definition of $\PAP_\star(\imath^{t})$, we have incorporated the transpose map in order to reconcile with the Barbasch-Vogan duality.

The middle letter $A$ in $\PAP$ refers to the common $A$ in $\{A, A^\bH, A^*\}$.

The special unipotent representations of general linear groups are
well-understood (reference?). In particular, we have the following counting result for general linear groups.

\begin{thm}\label{GLcase}
  The equality
  \[
    \sharp(\Unip_{\check \CO}(G))= \left\{
      \begin{array}{ll}
        \sharp(\PAP_\star(\check \CO)), &\hbox{if $\star\in\{A, A^\bH\}$}; \smallskip\\
        1, &\hbox{if $\star=A^\C$}  \end{array}
    \right.
  \]
  holds.

\end{thm}
\begin{remark}
  If $\star=A$, then
  \[
    \sharp(\PAP_\star(\check \CO))=\prod_{i\in \bN^+} (1+\textrm{the
      number of rows of length $i$ in $\check \CO$})
  \]
  If $\star=A^\bH$, then
  \[
    \sharp(\PAP_\star(\check \CO))= \left\{
      \begin{array}{ll}
        1, &\hbox{if all row lengths of $\check \CO$ are even}; \smallskip\\
        0, &\hbox{otherwise}.  \end{array}
    \right.
  \]

\end{remark}

Suppose that $\imath$ is a Young diagram and $\CP$ is a painting on $\imath$
that has type $A^*$. Define the signature of $\CP$ to be the pair
\[
    (p_\CP, q_\cP): = \left (\frac{\sharp(\cP^{-1}(\bullet))}{2}+\sharp(\cP^{-1}(r)),\,
    \frac{ \sharp(\cP^{-1}(\bullet))}{2}+\sharp(\cP^{-1}(s))\right).
\]
\trivial[h]{ The first equation is the true definition of signature. The second
  one is an easy consequence of the definition of $\AC_\cP$. }

\begin{eg}
  Suppose
  that \[ \check \CO=\ytb{\ \ \ \ \ , \ \ \ , \ , \ , \ }\quad \textrm{and}\quad \CP=\ytb{\bullet\bullet\bullet\bullet r,\bullet\bullet , sr,s,r}\in \mathrm{PAP}_{A^*}(\check \CO) .
  \]
  Then $(p_\CP, q_\cP)=(6,5)$.

\end{eg}

Given two Young diagrams $\imath$ and $\jmath$, write $\imath\cuprow \jmath$ for
the Young diagram whose multiset of nonzero row lengths equals the union of
those of $\imath$ and $\jmath$. Also write $2\imath =\imath\cuprow \imath$.

For unitary groups, we have the following counting result.
\begin{thm}
  Assume that $\star=A^*$ so that $G=\oU(p,q)$. If there is a decomposition
  \[
    \ckcO=\ckcOg \cuprow 2\ckcOpb
  \]
  such that all nonzero row lengths of $\ckcOg$ have the same parity as $p+q$,
  and all nonzero row lengths of $\ckcOpb$ have different parity as $p+q$, then
  \[
    \sharp(\Unip_{\ckcO}(G))= \sharp \set{\CP\in \mathrm{PAP}_\star(\ckcOg)|(p_\CP+\abs{\ckcOpb}, q_\CP+\abs{\ckcOpb})=(p,q)}
  \]
  If there is no such decomposition, then $\sharp(\Unip_{\check \CO}(G))=0$.

\end{thm}

\subsection{Orthogonal and symplectic groups: reduction to good parity}

Now we assume that
\[
  \star\in \Set{ B, D, B^\C, D^\C, C, C^\C, D^*, C^*, \widetilde C, \widetilde C^\C}.
\]
Then there is a unique decomposition
\[
  \ckcO=\ckcOg \cuprow 2\ckcOpb
\]
such that $\ckcOg$ has $\star$-good parity in the sense that all its nonzero row
lengths are
\[
  \left\{
    \begin{array}{ll}
      \textrm{even}, &\hbox{if $\star\in \set{B, B^\C, \widetilde C, \widetilde C^\C}$}; \smallskip\\
      \textrm{odd}, &\hbox{if $\star\in \set{C, D, C^\C, D^\C, D^*, C^*}$},
    \end{array}
  \right.
\]
and $\check \CO'_{\mathrm b}$ has $\star$-bad parity in the sense that all its
nonzero row lengths are
\[
  \left\{
    \begin{array}{ll}
      \textrm{odd}, &\hbox{if  $\star\in \set{B, B^\C, \widetilde C, \widetilde C^\C}$}; \smallskip\\
      \textrm{even}, &\hbox{if  $\star\in \set{C, D, C^\C, D^\C, D^*, C^*}$}.
    \end{array}
  \right.
\]
For simplicity, put
\[
  l:=\abs{\ckcOpb},
\]
and
\[
  \Gpb := \begin{cases}
    \GL_{l}(\bR) & \text{when } \star \in \set{B,C,\wtC,D}, \\
    \GL_{l}(\bC) & \text{when } \star \in \set{B^{\bC},C^{\bC},\wtC^{\bC},D^{\bC}}. \\
    \GL_{\frac{l}{2}}(\bH) & \text{when } \star \in \set{C^{*},D^{*}}. \\
  \end{cases}
\]
Note that $G$ has a closed subgroup isomorphic to $\Gpb$ (as Lie groups) if and only if
\[
  \begin{cases}
    p,q\geq l, & \text{when $G = \SO(p,q)$};\\
    p,q\geq \frac{l}{2}, &  \text{when $G = \Sp(p,q)$};\\
    \text{no condition,} & \text{otherwise}.
  \end{cases}
\]
In such cases, $G$ has a Levi subgroup isomorphic to $\Gpb\times \Gg$ where
\[
  \Gg :=
  \begin{cases}
    \SO(p-l,q-l) & \textrm{when $\star\in \set{B,D}$},\\
    \SO_{n-2l}(\bC) &\textrm{when $\star\in \set{B^{\bC},D^{\bC}}$},\\
    \rO^{*}(2n-2l) &\textrm{when $\star = D^{*}$},\\
    \Sp_{2n-2l}(\bR) &\textrm{when $\star = C$},\\
    \wtSp_{2n-2l}(\bR) &\textrm{when $\star = \wtC$},\\
    \Sp_{2n-2l}(\bC) &\textrm{when $\star \in \set{C^{\bC},\wtC^{\bC}}$},\\
    \Sp(p-\frac{l}{2},q-\frac{l}{2}) &\textrm{when $\star = C^{*}$}.\\
  \end{cases}
\]
% and respectively put
% \[
%   \begin{array}{rl}
%     \Gg:=  & \SO(p-l,q-l)\ \  (\textrm{when $p,q\geq l$}),   \ \     \SO_{n-2l}(\C),  \  \   \Sp_{2n-2l}(\R), \  \ \Sp_{2n-2l}(\C), \smallskip \\
%     %   & \oO^*(2n-2l), \ \  \Sp(p-\frac{l}{2},q-\frac{l}{2}) \ \  (\textrm{when $p,q\geq 2l$}),  \ \   \widetilde \Sp_{2n-2l}(\R) \ \ \textrm{or }  \ \  \Sp_{2n-2l}(\C),
%      \end{array}
%    \]
%    when \[
%      \begin{array}{rl}
%     G=  & \SO(p,q)   \ \     \SO_{n}(\C),  \  \   \Sp_{2n}(\R), \  \ \Sp_{2n}(\C), \smallskip \\
%     %   & \oO^*(2n), \ \  \Sp(p,q),  \ \   \widetilde \Sp_{2n}(\R) \ \ \textrm{or }  \ \  \Sp_{2n}(\C).
%      \end{array}
%    \]

\begin{thm}\label{reduction}
 If  $G$ has a closed subgroup isomorphic to $\Gpb$, then parabolic induction yields
   a bijection
  \begin{equation}\label{eq:IND}
    \begin{array}{rccc}
      \fI\colon &   \Unip_{G'_{b}}(\ckcO'_{b})\times \Unip_{G_{g}}(\ckcO_{g})&         \longrightarrow &\Unip_{G}(\ckcO) \\
                &   (\pi',\pi_{0}) & \mapsto & \pi'\rtimes \pi_{0}.
    \end{array}
  \end{equation}
  Otherwise,
  \[
    \Unip_{G}(\ckcO)=\emptyset.
  \]
\end{thm}


Combining with the counting result for general linear groups (Theorem \ref{GLcase}), we list the consequences of the
above theorem as follows:
\begin{enumerate}[label=(\alph*)]
  \item Assume that $\star\in \{B,D\}$ so that $G=\SO(p,q)$. Then
        \[
        \sharp(\Unip_{\check \CO}(G))=
        \begin{cases}
          \sharp(\Unip_{\check \CO_{\mathrm g}}(G_g))\times \sharp(\Unip_{\check \CO'_{\mathrm b}}(\GL_l(\R)) ), &\hbox{if $p,q\geq l$}; \smallskip\\
          0, &\hbox{otherwise.}
        \end{cases}
        \]
  \item Assume that $\star=C^*$ so that $G=\Sp(p,q)$. Then
        \[
        \sharp(\Unip_{\check \CO}(G))=
        \begin{cases}
          \sharp(\Unip_{\check \CO_{\mathrm g}}(G_g )), &\hbox{if $p,q\geq \frac{l}{2}$}; \smallskip\\
          0, &\hbox{otherwise.}
        \end{cases}
        \]

  \item Assume that $\star\in \{C,\widetilde C\}$ so that $G=\Sp_{2n}(\R)$ or
        $\widetilde \Sp_{2n}(\R)$. Then
        \[
        \sharp(\Unip_{\check \CO}(G))= \sharp(\Unip_{\check \CO_{\mathrm g}}(G_g))\times \sharp(\Unip_{\check \CO'_{\mathrm b}}(\GL_l(\R)) ). \]
  \item Assume that $\star\in \{B^\C, D^\C, C^\C,\widetilde C^\C, D^*\}$. Then
        \[
        \sharp(\Unip_{\check \CO}(G))= \sharp(\Unip_{\check \CO_{\mathrm g}}(G_g)).
        \]
\end{enumerate}


 \subsection{Orthogonal and symplectic groups: the case of good parity}
 Now we further assume that $\check \CO$ has $\star$-good parity, namely
 $\check \CO=\check \CO_{\mathrm g}$. By Theorem \ref{reduction}, the counting
 problem in general is reduced to this case.



 \delete{
   \begin{defn}
     A $\star$-pair is a pair $(i,i+1)$ of consecutive positive integers such
     that
     \[
       \left\{
         \begin{array}{ll}
           i\textrm{ is odd}, \quad &\textrm{if $\star\in\{C, \widetilde{C}, C^*, C^\C, \widetilde C^\C\}$};  \\
           i \textrm{ is even}, \quad &\textrm{if $\star\in\{B, D, D^*, B^\C, D^\C\}$}. \\
         \end{array}
       \right.
     \]
     A $\star$-pair $(i,i+1)$ is said to be primitive in $\check \CO$ if
     $\mathbf r_i(\check \CO)-\mathbf r_{i+1}(\check \CO)$ is positive and even.
     Denote $\mathrm{PP}_\star(\check \CO)$ the set of all $\star$-pairs that
     are primitive in $\check \CO$.
   \end{defn}
 }



\begin{defn}
  A $\star$-pair is a pair $(i,i+1)$ of consecutive positive integers such that
  \[
    \left\{
      \begin{array}{ll}
        i\textrm{ is odd}, \quad &\textrm{if $\star\in\{C, \widetilde{C}, C^*, C^\C, \widetilde C^\C\}$};  \\
        i \textrm{ is even}, \quad &\textrm{if $\star\in\{B, D, D^*, B^\C, D^\C\}$}. \\
      \end{array}
    \right.
  \]
  A $\star$-pair $(i,i+1)$ is said to be
  \begin{itemize}
    \item vacant in $\check \CO$, if
          $\mathbf r_i(\check \CO)=\mathbf r_{i+1}(\check \CO)=0$;
    \item balanced in $\check \CO$, if
          $\mathbf r_i(\check \CO)=\mathbf r_{i+1}(\check \CO)>0$;
    \item tailed in $\check \CO$, if
          $\mathbf r_i(\check \CO)-\mathbf r_{i+1}(\check \CO)$ is positive and
          odd;
    \item primitive in $\check \CO$, if
          $\mathbf r_i(\check \CO)-\mathbf r_{i+1}(\check \CO)$ is positive and
          even.
  \end{itemize}
  Denote $\CPP_\star(\check \CO)$ the set of all $\star$-pairs that are
  primitive in $\check \CO$.
\end{defn}
We remark that, when $\star\neq \wtC^{\bC}$, the set $\CPP_{\star}(\ckcO)$ appears implicitly in
\cite{So}*{Section~5} (for the purpose of describing Lusztig's canonical quotient).


\begin{thm}[Barbasch-Vogan, Barbasch]\label{complex}
  Assume that $\star\in \{B^\C,D^\C, C^\C, \widetilde C^\C\}$. Then
  \[
    \sharp(\Unip_{\check \CO}(G))=2^{\sharp(\CPP_\star(\check \CO))} .
  \]
\end{thm}
\trivial[h]{ Here is a tricky point: When $\star = \wtC^{\bC}$, the Lusztig
  canonical quotient of $\ckcO$ is a quotient by $\wp\sim \wp^{c}$ of
  $\bF_{2}[\CPP_{\star}(\ckcO)]$. However, the counting should still be correct!
}

\smallskip

We continue with the counting problem of $\Unip_{\check \CO}(G)$, when $\check \CO$ has $\star$-good parity.

We attach to $\check \CO$ a pair of Young diagrams
\[
  (\imath_{\check \CO}, \jmath_{\check \CO}):=(\imath_\star(\check \CO), \jmath_\star(\check \CO)),
\]
as follows.

\medskip

\noindent {\bf The case when $\star=\{B, B^\C\}$.} In this case,
\[
  \mathbf c_{1}(\jmath_{\check \CO})=\frac{\mathbf r_1(\check \CO)}{2},
\]
and for all $i\geq 1$,
\[
  \left (\mathbf c_{i}(\imath_{\check \CO}), \mathbf c_{i+1}(\jmath_{\check \CO})\right )= \left (\frac{\mathbf r_{2i}(\check \CO)}{2}, \frac{\mathbf r_{2i+1}(\check \CO)}{2}\right ).
\]

\medskip

\noindent {\bf The case when $\star\in \{\widetilde C, \widetilde C^\C\}$.} In
this case, for all $i\geq 1$,
\[
  (\mathbf c_{i}(\imath_{\check \CO}), \mathbf c_{i}(\jmath_{\check \CO}))= \left (\frac{\mathbf r_{2i-1}(\check \CO)}{2}, \frac{\mathbf r_{2i}(\check \CO)}{2}\right).
\]

\medskip

\noindent {\bf The case when $\star\in \{C,C^*, C^\C\}$.} In this case, for all
$i\geq 1$,
\[
  (\mathbf c_{i}(\jmath_{\check \CO}), \mathbf c_{i}(\imath_{\check \CO}))= \left\{
    \begin{array}{ll}
      (0,  0), &\hbox{if $(2i-1, 2i)$ is vacant  in $\check \CO$};\smallskip\\
      (\frac{\mathbf r_{2i-1}(\check \CO)-1}{2},  0), & \hbox{if $(2i-1, 2i)$ is tailed in $\check \CO$};\smallskip\\
      (\frac{\mathbf r_{2i-1}(\check \CO)-1}{2},  \frac{\mathbf r_{2i}(\check \CO)+1}{2}), &\hbox{otherwise}.\\
    \end{array}
  \right.
\]
\medskip

\noindent {\bf The case when $\star\in \{D,D^*, D^\C\}$.} In this case,
\[
  \mathbf c_{1}(\imath_{\check \CO})= \left\{
    \begin{array}{ll}
      0,  &\hbox{if $\mathbf r_1(\check \CO)=0$}; \smallskip\\
      \frac{\mathbf r_1(\check \CO)+1}{2},   &\hbox{if $\mathbf r_1(\check \CO)>0$},\\
    \end{array}
  \right.
\]
and for all $i\geq 1$,
\[
  (\mathbf c_{i}(\jmath_{\check \CO}), \mathbf c_{i+1}(\imath_{\check \CO}))= \left\{
    \begin{array}{ll}
      (0,  0), &\hbox{if $(2i, 2i+1)$ is vacant in $\check \CO$};\smallskip\\
      \left  (\frac{\mathbf r_{2i}(\check \CO)-1}{2},  0\right ), & \hbox{if $(2i, 2i+1)$ is tailed in $\check \CO$};\smallskip\\
      \left  (\frac{\mathbf r_{2i}(\check \CO)-1}{2},  \frac{\mathbf r_{2i+1}(\check \CO)+1}{2}\right ), &\hbox{otherwise}.\\
    \end{array}
  \right.
\]




\begin{eg} Suppose that $\star=C$, and $\check \CO$ is the following Young
  diagram which has $\star$-good parity.
  \begin{equation*}\label{eq:sp-nsp.C}
    \tytb{\ \ \ \ \  , \ \ \  , \ \ \ , \ \ \  , \ \ \ , \  ,\  }
  \end{equation*}
  Then
  \[
    \CPP_\star(\check \CO)=\{(1,2), (5,6)\}
  \]
  and
  \[
    (\imath_{\check \CO}, \jmath_{\check \CO})= \tytb{\ \ \ ,\ \ } \times \tytb{\ \ \ , \ }.
  \]


\end{eg}



\delete{
  \begin{eg} Suppose that $\star=C$, and $\check \CO$ is the following Young
    diagram which has $\star$-good parity.
    \begin{equation*}\label{eq:sp-nsp.C}
      \tytb{\ \ \ \ \  , \ \ \  , \ \ \ , \ \ \  , \ \ \ , \  ,\  }
    \end{equation*}
    Then
    \[
      \mathrm{PP}_\star(\check \CO)=\{(1,2), (5,6)\}.
    \]
    and $(\imath_\star(\check \CO, \wp), \jmath_\star(\check \CO,
    \wp))$ %\in \mathrm{BP}_\star(\check \CO)$
    has the following form.

    \begin{equation*}\label{eq:sp-nsp.C}
      \begin{array}{rclcrcl}
        \wp=\emptyset & : & \tytb{\ \ \ ,\ \  } \times \tytb{\ \ \ , \  }  & \qquad \quad &  \wp=\{(1,2)\}& : & \tytb{\ \ \  , \ \ , \   } \times \tytb{\ \ \  } \medskip \medskip \medskip \\
        \wp=\{(5,6)\} & : & \tytb{\ \ \ ,\ \ \ } \times \tytb{\ \ , \   }  & \qquad \quad &  \wp=\{(1,2), (5,6)\}  & : & \tytb{\ \ \  , \ \ \ ,  \ } \times \tytb{\ \   } \\
      \end{array}
    \end{equation*}

\end{eg}
}

Here and henceforth, when no confusion is possible, we write
$\alpha\times \beta$ for a pair $(\alpha, \beta)$. We will also write
$\alpha\times \beta\times \gamma$ for a triple $(\alpha, \beta, \gamma)$.


We introduce two more symbols $B^+$ and $B^-$, and make the following
definition.
\begin{defn}\label{defpbp0}
  A painted bipartition is a triple
  $\tau=(\imath, \CP)\times (\jmath, \cQ)\times \alpha$, where $(\imath, \CP)$
  and $ (\jmath, \mathcal Q)$ are painted Young diagrams, and
  $\alpha\in \{B^+,B^-, C,D,\widetilde {C}, C^*, D^*\}$, subject to the
  following conditions:
  \begin{itemize}
          \delete{\item $(\imath, \jmath)\in \mathrm{BP}_\alpha$ if
          $\alpha\notin\{B^+,B^-\}$, and $(\imath, \jmath)\in \mathrm{BP}_{B}$
          if $\alpha\in\{B^+,B^-\}$;}

    \item $\CP^{-1}(\bullet)=\mathcal Q^{-1}(\bullet)$;
    \item the image of $\CP$ is contained in
          \[
          \left\{
          \begin{array}{ll}
            \{\bullet, c\}, &\hbox{if $\alpha=B^+$ or $B^-$}; \smallskip\\
            \{\bullet,  r, c,d\}, &\hbox{if $\alpha=C$}; \smallskip\\
            \{\bullet, s, r, c,d\}, &\hbox{if $\alpha=D$}; \smallskip\\
            \{\bullet, s, c\}, &\hbox{if $\alpha=\widetilde{ C}$}; \smallskip \\
            \{\bullet\}, &\hbox{if $\alpha=C^*$}; \smallskip \\
            \{\bullet, s\}, &\hbox{if $\alpha=D^*$},\\
          \end{array}
          \right.
          \]
    \item the image of $\mathcal Q$ is contained in
          \[
          \left\{
          \begin{array}{ll}
            \{\bullet, s, r, d\}, &\hbox{if $\alpha=B^+$ or $B^-$}; \smallskip\\
            \{\bullet, s\}, &\hbox{if $\alpha=C$}; \smallskip\\
            \{\bullet\}, &\hbox{if $\alpha=D$}; \smallskip\\
            \{\bullet, r, d\}, &\hbox{if $\alpha=\widetilde{ C}$}; \smallskip\\
            \{\bullet, s,r\}, &\hbox{if $\alpha=C^*$}; \smallskip \\
            \{\bullet, r\}, &\hbox{if $\alpha=D^*$}.
          \end{array}
          \right.
          \]

  \end{itemize}
\end{defn}

% \begin{remark}
%   The set of painted bipartition counts the multiplicities of an irreducible
%   representation of $W_{r_{\fgg}}$ occurs in the coherent continuation
%   representation at the infinitesimal character of the trivial representation.
%   For the relationship between painted bipartitions and the coherent
%   continuation representations of Harish-Chandra modules, see \cite{Mc}.
% \end{remark}

For any painted bipartition $\tau$ as in Definition \ref{defpbp0}, we write
\[
  \imath_\tau:=\imath,\ \cP_\tau:=\cP,\ \jmath_\tau:=\jmath,\ \cQ_\tau:=\cQ,\ \alpha_\tau:=\alpha,
\]
and
\[
  \star_\tau:= \left\{
    \begin{array}{ll}
      B, &\hbox{if $\alpha=B^+$ or $B^-$}; \smallskip\\
      \alpha, & \hbox{otherwise}.           \end{array}
  \right.
\]
% Its leading column is then defined to be the first column of $(\jmath, \CQ)$
% when $\star_\tau\in \{B, C,C^*\}$, and the first column of $(\imath, \CP)$
% when $\star_\tau\in \{\widetilde C, D, D^*\}$.

We further define a pair $(p_{\tau}, q_{\tau})$ of natural numbers given by the
following recipe.
\begin{itemize}
  \item If $\star_\tau\in \{B, D, C^*\}$, $(p_\tau, q_\tau)$ is given by
        counting the various symbols appearing in $(\imath, \CP)$,
        $(\jmath, \cQ)$ and $\{\alpha\}$ :
        \begin{equation}\label{ptqt}
          \left\{
            \begin{array}{l}
              p_\tau :=( \# \bullet)+ 2 (\# r) +(\# c )+ (\# d) + (\# B^+);\smallskip\\
              q_\tau :=( \# \bullet)+ 2 (\# s) + (\# c) + (\# d) + (\# B^-).\\
            \end{array}
          \right.
        \end{equation}
        Here
        \[
        \#\bullet:=\#(\cP^{-1}(\bullet))+\#(\cQ^{-1}(\bullet))
        %\qquad (\textrm{$\#$        indicates the cardinality of a finite set}),
        \]
        and the other terms are similarly defined.
  \item If $\star_\tau\in \{C, \widetilde C, D^*\}$,
        $p_\tau:=q_\tau:=\abs{\tau}$.
\end{itemize}
\smallskip

We also define a classical group
\begin{equation}\label{def:Gt}
  G_\tau:=
  \begin{cases}
    \SO(p_\tau, q_\tau), &\hbox{if $\star_\tau=B$ or $D$}; \smallskip\\
    \Sp_{2\abs{\tau}}(\R), &\hbox{if $\star_\tau=C$}; \smallskip\\
    \widetilde{\Sp}_{2\abs{\tau}}(\R), &\hbox{if $\star_\tau=\widetilde{ C}$}; \smallskip \\
    \Sp(\frac{p_\tau}{2}, \frac{q_\tau}{2}), &\hbox{if $\star_\tau=C^*$}; \smallskip \\
    \oO^*(2\abs{\tau}), &\hbox{if $\star_\tau=D^*$}.\\
  \end{cases}
\end{equation}

Define
\[
  \begin{split}
    \PBP_\star(\check \CO) &:=\set{ \uptau\textrm{ is a painted
        bipartition} \mid \star_\uptau = \star, \text{ and
      } (\imath_\tau,\jmath_\tau) = (\imath_{\check \CO}, \jmath_{\check \CO})}, \AND\\
    \PBP_{G}(\ckcO) &:=\set{\uptau\in \PBP_{\star}(\ckcO)| G_{\uptau} = G}.
  \end{split}
\]


\delete{
  \[
    \begin{array}{rl}
      \mathrm{PBP}_\star(\check \CO):=\{ &
                                           \tau\textrm{ is a painted bipartition}  \mid    \star_\tau = \star,
                                           \text{ and } \\  & (\imath_\tau,\jmath_\tau) = (\imath_{\check \CO}, \jmath_{\check \CO})\}.
    \end{array}
  \]
}




\begin{eg} Suppose that $\star=B$ and
  \[
    \check \CO =\tytb{\ \ \ \ \ \ , \ \ \ \ \ \ , \ \ , \ \ , \ \ }
  \]
  Then
  \[
    \tau:= \tytb{\bullet \bullet ,\bullet , c } \times \tytb{\bullet \bullet d ,\bullet , d }\times B^+\in \mathrm{PBP}_{\star}(\check \CO),
  \]
  and
  \[
    G_\tau=\SO(10,9).
  \]
\end{eg}


We now state our final result on the counting of special unipotent representations.

\begin{thm}\label{countup}
  Assume that $\star\in \{B, C,D,\widetilde {C}, C^*, D^*\}$, and $\check \CO$ has $\star$-good parity. Then
  \[
    \sharp(\Unip_{\ckcO}(G))\leq 2^{\sharp(\CPP_\star(\check \CO))} \cdot \sharp \PBP_{G}(\ckcO).
  \]
\end{thm}

In \cite{BMSZ2}, we will construct $2^{\sharp(\CPP_\star(\check \CO))} \cdot \sharp \PBP_{G}(\ckcO)$ number of representations in
$\Unip_{\check \CO}(G)$, and thus the equality holds in \Cref{countup}.

\medskip

Here are some words on the contents and the organization of this article. In Section 2, we introduce the coherent continuation representation in a general set-up and prove a basic result of Vogan on counting representations via the translation principle. In Section 3, we review the theory of primitive ideas and double cells and their connection with the coherent continuation representation. The main result is a characterization of the irreducible constituents of the coherent continuation representation in (a variant of) the category $\CO$, in terms of the Springer correspondence. In Section 4, we examine the relationship of coherent families of Harish-Chandra modules with coherent families in category $\CO$ and prove a upper bound on the counting of irreducible representations annihilated by a maximal primitive ideal. We also give a formula for the coherent continuation representation of Harish-Chandra modules, which is an unpublished result of Barbasch and Vogan. Sections 5 to 7 are devoted to the main concern of the article, which is to give a precise count of special unipotent representations of all real classical groups, starting from the general linear groups and then unitary groups, and finally real classical groups of type $\mathrm{BCD}$. The answer is given in terms of some combinatorial constructs described earlier in this section. It is important to note, while the algebraic theory alluded to yields ultimately an upper bound of the count, we are unable to prove the precise count using the algebraic theory alone, due to a certain technical issue on double cells of Harish-Chandra modules. Instead we rely on the analytic theory of theta lifting to construct the right amount of special unipotent representations (\cite{BMSZ2}) and therefore to close the gap, so to speak. It will be clearly desirable to demonstrate the precise count, without recourse to the analytic theory.


\section{Proofs of Theorems \ref{count1} and  \ref{count2}}


\subsection{Proof of Theorem \ref{count1}} We retain the notation of Sections \ref{sec11}-\ref{sec13}. In this subsection, we are aimed to provide a proof of the following result which is due to Vogan:
 \be\label{eqthemc1}
    \sharp(\Irr_{\lambda,\sfS}(G)) = [1_{W_{\lambda}}:\Coh_{[\lambda],\sfS}(G)].
  \ee

Write $\CK_{\lambda}(G):=  \CK_{\lambda,\Nil(\g^*)}(G)$, which is the Grothedieck group, with complex coefficient, of the category of Casselman-Wallach representations of $G$ of generalized infinitesimal character $\lambda$.
By evaluating at $\lambda$, we have  a linear map
   \[
    \mathrm{ev}_{\lambda } \, :\,  \Coh_{[\lambda]}(G) \longrightarrow \Grt_{\lambda}(G).
  \]


Recall that $\lambda$ is said to be regular if
\[
    \la \lambda, \alpha^\vee\ra\neq 0 \qquad\textrm{for all $\alpha\in \Delta$}.
  \]
In general, pick a regular  element $\mu\in [\lambda]$ such that for all $\alpha\in \Delta$,
    \[
    \la \mu, \alpha^\vee\ra\in \BN^+ \Rightarrow  \la \lambda, \alpha^\vee\ra\in \BN.
  \]



 \begin{lem}\label{lem21}
The map  $\mathrm{ev}_{\mu}$ is bijective.
     \end{lem}
\begin{proof}
The surjectivity (without the regular condition) is due to Schmid and Zuckerman, see  \cite{Vg}*{Theorem~7.2.7}. The injectivity  is due to Schmid, see \cite{Vg}*{Proposition~7.2.23}.

\end{proof}




In view of Lemma \ref{lem21}, for every $\pi\in \Irr_{\mu}(G)$, we have a coherent family
\[
\Phi_\pi:=\mathrm{ev}_{\mu}^{-1}(\pi)\in  \Coh_{[\lambda]}(G).
\]

\begin{lem}\label{lemirr}
For every $\pi\in \Irr_{\mu}(G)$,  either  $\Phi_{\pi}(\lambda)=0$ or
         $\Phi_{\pi}(\lambda)\in \Irr_{\lambda}(G)$. Moreover, the map
         \[
           \{\pi\in \Irr_{\mu}(G)\mid \Phi_{\pi}(\lambda)\in \Irr_{\lambda}(G)\}\rightarrow \Irr_\lambda(G), \quad \pi\mapsto \Phi_{\pi}(\lambda)
         \]
         is bijective.
         \end{lem}
\begin{proof}
In view  of Lemma \ref{lem21}, this is implied by  \cite{Vg}*{Corollary~7.3.23}.
\end{proof}


\begin{lem}\label{lemirr11}
For every $\pi\in \Irr_{\mu, \sfS}(G)$,  $\Phi_{\pi}\in \Coh_{[\lambda],\sfS}(G)$.
         \end{lem}
\begin{proof}
This is implied by \cite{Vg}*{Part (a) of Proposition~7.2.22}  and the formula in the third line of  \cite{Vg}*{page 472}. Note that the latter  formula  is a special case of  \cite{Vg}*{Part (b) of Proposition~7.2.22}, and Vogan pointed out that \cite{Vg}*{Proposition~7.2.22} is due to Zuckerman (\cite{Zu}).

\end{proof}


\begin{lem}\label{lemirr2}
For every $\pi\in \Irr_{\mu}(G)$ such that
         $\Phi_{\pi}(\lambda)\in \Irr_{\lambda}(G)$,
          \[
          \AV_\C(\Phi_{\pi}(\lambda)) = \AV_\C(\pi).
          \]
\end{lem}
\begin{proof}
This is implied by
          \cite{Vg}*{Part (a) of Proposition~7.2.22 and Part (b) of Proposition~7.3.10}.
\end{proof}


Lemmas \ref{lem21}-\ref{lemirr2} imply that the evaluation map  (at $\lambda$)
  \[
    \mathrm{ev}_{\lambda , \sfS} \, :\,  \Coh_{[\lambda],\sfS}(G) \longrightarrow \Grt_{\lambda, \sfS}(G)
  \]
 is surjective.

Put
 \[
  \Delta_{[\lambda]} := \Set{\alpha\in \Delta\mid  \inn{\lambda}{\ckalpha}\in \bZ}\subset \hha^*.
 \]
 It is  a root system whose Weyl group equals  $W_{[\lambda]}$ (the stabilizer of $[\lambda]$ in the Weyl group $W$).  See \cite{V4}*{\S 2}. For every $\alpha\in   \Delta_{[\lambda]}$, let $s_\alpha\in W_{[\lambda]}$ denote the reflection associated to $\alpha$.
Note that
 \[
  \Delta_{\mu}^+:= \Set{\alpha\in \Delta\mid  \inn{\mu}{\ckalpha}\in \BN^+}
 \]
 is a positive system of $ \Delta_{[\lambda]}$.

 The following lemme is proved in \cite{Vg}*{Part (c) of Corollary 7.3.23}.
 \begin{lem}\label{lemirr4}
For every $\pi\in \Irr_{\mu}(G)$,    $\Phi_{\pi}(\lambda)=0$ if and only if
         \[
         s_\alpha.\Phi_{\pi} =-\Phi_{\pi}  \quad \textrm{for some simple root $\alpha$ of $\Delta_{\mu}^+$ such that $\la \lambda, \alpha^\vee\ra=0$}.
         \]
             \end{lem}



Finally, we have that
  \begin{eqnarray*}
      \ker \ev{\lambda, \sfS}& = & \Span \{\Phi_{\pi}\mid \pi\in \Irr_{\mu, \sfS}(G), \Phi_{\pi}(\lambda)=0\}\quad \textrm{(by Lemma \ref{lemirr}})\\
    &  \subset & \Span \{\Phi_\pi - s_{\alpha}. \Phi_{\pi} \mid  \pi\in \Irr_{\mu, \sfS}(G), \, \alpha  \text{ is a simple root of $\Delta_{\mu}^+$ such that} \\
      &&\quad \qquad \qquad \qquad \qquad  \la \lambda, \alpha^\vee\ra=0 \}  \qquad  \textrm{(by Lemma \ref{lemirr4})} \\
           & \subset &\Span\{\Phi- w. \Phi \mid \Phi\in \Coh_{[\mu],\sfS}(G), \, w\in W_{\lambda}\} \\
    &  \subset &  \ker \ev{\lambda, \sfS}. \\
        \end{eqnarray*}
Therefore
\[
 \ker \ev{\lambda, \sfS}=\Span\{\Phi- w. \Phi \mid \Phi\in \Coh_{[\mu],\sfS}(G), \, w\in W_{\lambda}\},
\]
and the equality \eqref{eqthemc1} follows since the map $\ev{\lambda, \sfS}$ is surjective and $W_{\lambda}$-invariant.


\subsection{Highest weight modules}

\newcommand{\Rep}{\mathrm{Rep}}
Let $\b$ be a  Borel subalgebra of $\g$. Let $\Rep(\g,\b)$ denote the category of finitely generated $\g$-modules that are unions of finite-dimensional $\b$-submodules, and  let $\Rep_\sfS(\g,\b)$ denote its full subcategory of the modules whose complex associated variety is contained in $\sfS$.
Write $\CK(\g,\b)$ and $\CK_\sfS(\g,\b)$ respectively for the Grothendieck groups, with $\C$-coefficient, of $\Rep(\g,\b)$  and $\Rep_\sfS(\g,\b)$.  %Write $\Irr(\g,\b)$ for the set of isomorphism classes of irreducible modules in this category.
Similar to $\Coh_{\Lam,\sfS}(G)$, we define a representation
\[
\Coh_{\Lam,\sfS}(\g,\b):=\Coh_{\Lam}(\CK_\sfS(\g,\b))
\]
 of $W_{[\lambda]}$. This  representation  is independent of the Borel subalgebra $\b$.


    Let $H$ be a Cartan subgroup of $G$ such that its complexified Lie algebra $\h$ is contained in $\b$. Recall that a $(\g, H)$-module is defined to be a $\g$-module $V$ together with a locally-finite representation of $H$ on it such that
     \begin{itemize}
     \item
        $h.(X.(h^{-1}.u))=(\Ad_h(X)).u$, for all $h\in H, X\in \CU(\g), u\in V$ ($\Ad$ stands for the adjoint representation);
        \item the differential of the representation of $H$ and the restriction of the representation of $\g$ yields the same representation of $\h$ on $V$.
     \end{itemize}

Let $\Rep(\g,H,\b)$ denote the category of finitely generated $(\g, H)$-modules that  are unions of finite-dimensional $\b$-submodules.
Similar to before, we obviously define $\Rep(\g,H,\b)$, $\Rep_\sfS(\g,H,\b)$, $\CK(\g,H, \b)$, $\CK_\sfS(\g,H, \b)$, $\Coh_{\Lam}(\g,H, \b)$ and  $\Coh_{\Lam,\sfS}(\g,H, \b)$.

 \begin{prop}\label{lem0022}
The representation $\Coh_{\Lam,\sfS}(\g,H,\b)$ of $W_{[\lambda]}$ is isomorphic to a subrepresentation of $(\Coh_{\Lam,\sfS}(\g,\b))^k$, for some $k\in \BN$.
     \end{prop}
\begin{proof}


Write $\Inn_H$ for the Zariski closure of the image of the  the adjoint representation $G\rightarrow \Inn(\g)$, which is an algebraic torus. Write $Q_H$ for the group of algebraic characters of $\Inn_H$ (which is isomorphic to  the root lattice). By pulling-back through the homomorphism $H\rightarrow \Inn_H$, we view $Q_H$ as a set of characters on $H$.
The tensor product $\beta\otimes \gamma\in \Irr(H)$ is defined for every $\beta\in Q_H$ and $\gamma\in \Irr(H)$. This yields a free action of $Q_H$ on the set $\Irr(H)$.

For each $Q_H$-orbit $\Gamma\subset \Irr(H)$, write $\Rep_{\sfS,\Gamma}(\g,H,\b)$ for the full subcategory of $\Rep_\sfS(\g,H,\b)$ whose objects are  the modules $V$ such that every irreducible subquotient of $V|_H$ ($V$ viewed as a representation of $H$) belongs to $\Gamma$. Write $\CK_{\sfS,\Gamma}(\g,H,\b)$ for the  Grothendieck group, with $\C$-coefficient, of the category $\Rep_{\sfS,\Gamma}(\g,H,\b)$.
Then  we have a decomposition
\[
\CK_\sfS(\g,H,\b)=\bigoplus_{\Gamma\in Q_H\backslash \Irr(H)} \CK_{\sfS,\Gamma}(\g,H,\b),
\]
of $\mathcal R(\g)$-modules, and
\[
\Coh_{\Lam,\sfS}(\g,H,\b)=\bigoplus_{i=1}^k  \Coh_{\Lam}(\CK_{ \sfS,\Gamma_i}(\g,H,\b)),
\]
for a finite number of orbits $\Gamma_1, \Gamma_2, \cdots, \Gamma_k\in Q_H\backslash \Irr(H)$ ($k\in \bN$). Thus it remains to show that $ \Coh_{\Lam}(\CK_{ \sfS,\Gamma}(\g,H,\b))$ is isomorphic to a subrepresentation of  $\Coh_{\Lam,\sfS}(\g,\b)$.



For each $\gamma\in \Gamma$, put
\[
  M(\gamma):=\CU(\g)\otimes_{\CU(\b)} \gamma,
\]
which is a module in $\Rep_\Gamma(\g,H,\b)$, where the $\CU(\g)$-action is given by the left multiplication, and the $H$-action is give by
\[
 h.(X\otimes u):=\Ad_h(X)\otimes h.u, \qquad h\in H, \, X\in \CU(\g),\, u\in \gamma.
\]

Note that $\{ M(\gamma)\}_{\gamma\in \Gamma}$ is a basis of the space (reference?)
\[
\CK_{\Gamma}(\g,H,\b):=\CK_{\Nil(\g^*),\Gamma}(\g,H,\b).
\]
Thus the forgetful functor
\[
   \Rep_{\Gamma}(\g,H,\b):=\Rep_{\Nil(\g^*),\Gamma}(\g,H,\b)\rightarrow  \Rep(\g,\b)
\]
induces an injective linear map
\[
    \CK_{\Gamma}(\g,H,\b)\rightarrow  \CK(\g,\b).
\]
This map is a $\mathcal R(\g)$-module homomorphism, and induces an injective
$\mathcal R(\g)$-module homomorphism
\[
    \CK_{\Gamma,\sfS}(\g,H,\b)\rightarrow  \CK_\sfS(\g,\b).
\]
The above homomorphism induces an embedding
\[
    \Coh_{\Lam}(\CK_{ \sfS,\Gamma}(\g,H,\b))\rightarrow  \Coh_{\Lam,\sfS}(\g,\b),
\]
and this follows the proposition.

\end{proof}




\subsection{A result of Casian}

 Similar to the subspace $\Grt_{\lambda, \sfS}(G)\subset \Grt(G)$, we obviously define the subspace $\Grt_{\lambda, \sfS}(G)\subset \Grt_{\lambda, \sfS}(\fgg,H,\b)$.
Let $\{H_1, H_2, \cdots, H_r\}$ ($r\in \bN^+$) be  a set of representatives of the
  conjugacy classes of  Cartan subgroups of $G$. For each $i=1,2,\cdots, r$, fix a Borel subalgebra $\b_i$ of $\g$ that contains the complexified Lie algebra of $H_i$.

 \begin{prop}\label{cor:HC.embed}
  % [\cite{Cas}*{Theorem~3.1}]
 There is an injective $\mathcal R(\g)$-module homomorphsm
 \[
\gamma_G: \Grt(G)\rightarrow  \bigoplus_{i=1}^{r} \Grt(\fgg,H_{i},\b_{i})
 \]
 such that
 \be\label{gammag}
   \gamma_G(\Grt_{\lambda, \sfS}(G))\subset  \bigoplus_{i=1}^{r} \Grt_{\lambda, \sfS}(\fgg,H_{i},\b_{i})
 \ee
 for all $\lambda\in \hha^*$ and all $\Inn(\g)$-stable Zariski closed subset $\sfS$ of $\Nil(\g^*)$.

 \end{prop}
\begin{proof}
This follows from the work of Casian (\cite{Cas}). See also {\cite{Mc}}. Since the proposition is not explicitly formulated in \cite{Cas}, we briefly recall the argument of Casian for the convenience of the reader.



 Let $\n_i$ denote the nilpotent radical of $[\g,\g]\cap \b_i$ ($i=1,2, \cdots,r$).
 For every $q\in \Z$, let $\gamma_{\n_i}^q$ denote the $q$-th right derived functor of the following left exact functor from the category of $\g$-modules to itself:
 \[
   V\rightarrow \{u\in V\,|\, \n_i^k v=0\textrm{ for some $k\in \bN^+$}\}.
 \]

Fix a Cartan involution $\theta$ of $G$ and write $K$ for its fixed point group (which is a maximal compact subgroup of $G$).  Without loss of generality we assume that all $H_i$'s are $\theta$-stable.

For every Casselman-Wallach representation $V$ of $G$, write $V_{[K]}$ for the space of $K$-finite vectors in $V$, which is a $(\g,K)$-module of finite length. Then   $\gamma_{\n_i}^q(V_{[K]})$ is naturally a representation in $\Rep(\g, H_i, \b_i)$ (\cite[Corollary 4.9]{Cas}).

We define a linear map
 \[
\gamma_G: \Grt(G)\rightarrow  \bigoplus_{i=1}^{r} \Grt(\fgg,H_{i},\b_{i})
 \]
 such that
 \[
   \gamma_G(V)= \left\{\sum_{q\in \Z} (-1)^{q} \gamma^{q}_{\n_i}(V_{[K]})\right\}_{i=1,2, \cdots, r}
 \]
for every Casselman-Wallach representation $V$ of $G$. The Osborne conjecture (see \cite[Theorem 3.1]{Cas}) and \cite[Corollary 4.9]{Cas} implies that the map $\gamma_G$ is injective.

\cite[Proposition 4.11]{Cas} implies that the functor $\gamma_{\n_i}^q$ commutes with tensor product with the finite-dimensional representations. Thus $\gamma_G$ is a $\mathcal R(\g)$-homomorphism. Finally, \cite[Corollary 4.15]{Cas} implies that \eqref{gammag} holds.


\end{proof}




Proposition \ref{cor:HC.embed} implies that the representation $\Coh_{\Lam,\sfS}(G)$ of $W_{[\lambda]}$ is isomorphic to a subrepresentation of
$\bigoplus_{i=1}^r \Coh_{\Lam,\sfS}(\g,H_i,\b_i)$. Together with Proposition \ref{lem0022}, this implies the following result.

 \begin{prop}\label{lem22}
The representation $\Coh_{\Lam,\sfS}(G)$ of $W_{[\lambda]}$ is isomorphic to a subrepresentation of $(\Coh_{\Lam,\sfS}(\g,\b))^k$, for some $k\in \BN$.
     \end{prop}

\subsection{Associated variety of primitive ideals}
Fix a Cartan subalgebra $\h$ of $\g$ that is contained in $\b$. Then $\h$ is identified with $\hha$ (since $\b$ has been fixed) and we view $\lambda$ as an element of $\h^*$.
Define the Verma module
\[
  \mathrm M(\lambda):=\CU(\g)\otimes_{\CU(\b)} \C_{\lambda-\rho},
\]
where $\C_{\lambda-\rho}$ is the one-dimensional $\h$-module corresponds to the character $\lambda-\rho\in \h^*$, and every $\h$-module is viewed an a $\b$-module as usual.
Write
$\oL(\lambda)$ for the unique irreducible quotient of $ \mathrm M(\lambda)$, and write $\oJ(\lambda)$ for the annihilator ideal of $\oL(\lambda)$.

Write $\h=\h_s\oplus \c$, where $\h_s=\h\cap [\g,\g]$ and $\c$ is the center of $\g$. Then $\h^*=\h_\mathrm s^*\oplus \c^*$. For each $w\in W$ and each $Q$-coset $\Lambda\subset \h^*$, there is a unique
polynomial function $\tilde p_{w,\Lambda}$ on $\h^*$ such that
\begin{itemize}
\item it is $\c^*$-invariant (under the translations),
\item for all $\lambda'\in \Lambda$ that is dominant,
\[
 \tilde p_{w,\Lambda}(\lambda')=\textrm{Goldie rank of } \CU(\g)/\oJ(w.\lambda').
\]

\end{itemize}


\[
  w_i^{-1} \Lambda^+\subset(w_i^{-1}\Lambda)^+.
\]
\[
w_i. \tilde p_{ww_i, w_i^{-1}\Lambda}=\tilde p_{w,\Lambda}
\]


Pick  $w\in W$ such that $w^{-1}.\lambda$ is dominant. By \cite{Jos}, there is a unique polynomial function $p_{w,\lambda}$ on $\h^*$ such that
\[
  p_{w,\lambda}(\lambda')=\textrm{Goldie rank of } \CU(\g)/\oJ(w.\lambda')
\]
for all $\lambda'\in \Lam$ that is dominant.




\subsection{The representation $\Coh_{\Lam,\sfS}(\g,\b)$}
Recall that $Q\subset \hha^*=\h^*$ is the root lattice. For every $Q$-coset $\Lambda\subset \h^*$, write $\Rep_{\sfS,\Lambda}(\g,\b)$ for the full subcategory of $\Rep_\sfS(\g,\b)$ whose objects are  the modules $V$ such that every irreducible subquotient of $V|_\h$ belongs to $\Lambda$. Write $\CK_{\sfS,\Lambda}(\g,\b)$ for the  Grothendieck group of the category $\Rep_{\sfS,\Lambda}(\g,\b)$.  Then
\[
   \CK_{\sfS}(\g,\b)=\bigoplus_{\Lambda\in Q\backslash \h^*} \CK_{\sfS,\Lambda}(\g,\b)
\]
as $\mathcal R(\g)$-modules,
and
\[
\Coh_{\Lam,\sfS}(\g,\b) =\bigoplus_{w\in W/W_\Lam} \Coh_{\Lam}( \CK_{\sfS, [ w.\lambda -\rho]}(\g,\b)),
\]
as representations of $W_\Lam$, where $\rho\in \h^*$ is the half sum of the weights of $\b$. Recall the set
$  \sfC_{\sfS}\subset \Irr(W_\Lam)$ from \eqref{sfc}.


\begin{prop}
For every $w\in W$,
\[
  \Coh_{\Lam}( \CK_{\sfS, [ w.\lambda -\rho]}(\g,\b))\cong \bigoplus_{\sigma\in \sfC_{\sfS}} \sigma^{\dim \sigma}
\]
as representations of $W_\Lam$.
\end{prop}
\begin{proof}
Assume without loss of generality that $\lambda$ is dominant, namely
\[
  \la \lambda, \alpha^\vee\ra\notin -\BN^+\qquad\textrm{for all $\alpha\in \Delta$}.
\]
As before, suppose that $\mu\in \Lam$ is regular and dominant. Note that there is a unique element $w'\in w W_\Lam$ that has minimal length. such that

\end{proof}

Under the $W$ action, the $W_{[\lambda]}$-module
        $\sigma_{w}\subset S^{a(w)}(\fhh)$ generates an irreducible $W$-module
        $\wtsigma_{w}:=j_{W_{[\lambda]}}^{W}\sigma_{w}$. The $W$-module
        $\wtsigma_{w}$ corresponds to the a nilpotent orbit $\cO_{\wtsigma_{w}}$
        with trivial local system under the Springer correspondence. Now the
        complex associated variety
        \[
        \AVC(L(w\lambda)) = \Gad\cdot \AV(L(w\lambda)) =\bcO_{\wtsigma_{w}},
        \]
        see \cite{J.av}*{Section~2.10}.


$\Rep_{\Gamma}(\g,\b)$ for the subcategory of $\Rep_(\g,\b)$


For every $w\in W$,

\subsection{}

Retain the notation in \Cref{thm:gamma.HC}, we write
\[
  \gamma_{\fnn}:= \sum_{q\in \bN} (-1)^{q} \gamma^{q}_{\fnn}\colon \Grt(\fgg,K) \longrightarrow \Grt(\fgg,H,\fnn)
\]

% \[
%   \gamma_{\fnn}:= \sum_{q\in \bN} (-1)^{q} \gamma^{q}_{\fnn}\colon \Coh_{[\lambda]}(\fgg,K) \longrightarrow \Coh_{[\lambda]}(\fgg,H,\fnn)
% \]

\begin{cor}[c.f. {\cite{Mc}}]\label{cor:HC.embed}
  % [\cite{Cas}*{Theorem~3.1}]
  Let $H_{1}, H_{2}, \cdots, H_{s}$ form a set of representatives of the
  conjugacy class of $\Phi$-stable Cartan subgroup of $G$. Fix maximal
  nilpotent Lie subalgebra $\fnn_{i}$ for each $H_{i}$ as in
  \Cref{thm:gamma.HC}. Then
  \[
    \begin{array}{cccc}
      \gamma:=\oplus_{i} \gamma_{\fnn_{i}}: &\Grt(\fgg,K)
      &\longrightarrow & \bigoplus_{i=1}^{s} \Grt(\fgg,H_{i},\fnn_{i})\\
    \end{array}
  \]
  is an embedding of abelian groups.
\end{cor}
\begin{proof}
  Let $\Hireg$ be the set of regular semisimple elements in $H_{i}$. By the
  results of Harish-Chandra, taking the character of the elements induces an
  embedding of $\Grt(\fgg,K)$ into the space of real analytic functions on
  $\bigsqcup_{i} \Hireg$. Since $\Hnireg$ is open in $\Hireg$ and meets all the
  connected components of $\Hireg$, any real analytic function on
  $\bigsqcup_{i} \Hireg$ is determined by its restriction on
  $\bigsqcup_{i} \Hnireg$. Now \eqref{eq:char} implies that the global character
  of any $M\in \Grt(\fgg,K)$ can be computed from the image $\gamma(M)$.
  % $\Phi M$ of an element $M\in \Grt_{\chi}(\fgg,K)$ is completely determined
  % by the formal character $\mathrm{ch}(\gamma(M))$.
\end{proof}


Write $\CK(\g,H,\b)$ for the Grothendieck group, with $\C$-coefficient, of this category. It is naturally a $\CR(\g)$-module. Respectively similar to  $\Rep_\sfS(\g,\b)$, $\CK_\sfS(\g,\b)$ and  $\Coh_\sfS(\g,\b)$, we define $\Rep_\sfS(\g,H,\b)$, $\CK_\sfS(\g,H, \b)$ and  $\Coh_\sfS(\g,H, \b)$.




The following result is a consequence of the work of Casian (\cite{Cas}). In this subsection, we will sketch a proof of it for completeness.
 \begin{prop}\label{lem22}
The representation $\Coh_{\Lam,\sfS}(G)$ of $W_{[\lambda]}$ is isomorphic to a subrepresentation of $(\Coh_{\Lam,\sfS}(\g,\b))^k$, for some $k\in \BN$.
     \end{prop}


     Let $H$ be a Cartan subgroup of $G$ such that its complexified Lie algebra $\h$ is contained in $\b$. Recall that a $(\g, H)$-module is defined to be a $\g$-module $V$ together with a locally-finite representation of $H$ on it such that
     \begin{itemize}
     \item
        $h.(X.(h^{-1}.u))=(\Ad_h(X)).u$, for all $h\in H, X\in \CU(\g), u\in V$ ($\Ad$ stands for the adjoint representation);
        \item the differential of the representation of $H$ and the restriction of the representation of $\g$ yields the same representation of $\h$ on $V$.
     \end{itemize}

Let $\Rep(\g,H,\b)$ denote the category of finitely generated $(\g, H)$-modules that  are unions of finite-dimensional $\b$-submodules.  Write $\CK(\g,H,\b)$ for the Grothendieck group, with $\C$-coefficient, of this category. It is naturally a $\CR(\g)$-module. Respectively similar to  $\Rep_\sfS(\g,\b)$, $\CK_\sfS(\g,\b)$ and  $\Coh_\sfS(\g,\b)$, we define $\Rep_\sfS(\g,H,\b)$, $\CK_\sfS(\g,H, \b)$ and  $\Coh_\sfS(\g,H, \b)$.



, and
whose complex associated variety is contained in $\sfS$.


We have that
\[
\CK(\g,H,\b)=\bigoplus_{\Gamma\in Q_H\backslash \Irr(H)} \CK_{\Gamma}(\g,H,\b),
\]
where $ \CK_{\Gamma}(\g,H,\b)$ is the the  Grothendieck group, with $\C$-coefficient, of the category $\Rep_\Gamma(\g,H,\b)$, and
whose complex associated variety is contained in $\sfS$.


Let $\Gamma$ be an orbit of this



Write $\Irr(\g,H, \b)$ for the set of isomorphism classes of irreducible modules in this category.
As before, we  obviously define  $\Irr_{\lambda}(\g,H, \b)$,  $\Irr_{\lambda,\sfS}(\g,H, \b)$,   $\Coh_{\Lam}(\g,H, \b)$ and $\Coh_{\Lam,\sfS}(\g,H, \b)$.



Write $\Irr(\g,\b)$ for the set of isomorphism classes of irreducible modules in this category.
We  obviously define  $\Irr_{\lambda}(\g,\b)$,  $\Irr_{\lambda,\sfS}(\g,\b)$,   $\Coh_{\Lam}(\g,\b)$ and $\Coh_{\Lam,\sfS}(\g,\b)$,
 respectively similar to  $\Irr_{\lambda}(G)$,  $\Irr_{\lambda,\sfS}(G)$,   $\Coh_{\Lam,\sfS}(G)$ and  $\Coh_{\Lam,\sfS}(G)$. The representation  $\Coh_{\Lam,\sfS}(\g,\b)$ of $W_{[\lambda]}$
is independent of the Borel subalgebra $\b$.


,
 respectively similar to  $\Irr_{\lambda}(G)$,  $\Irr_{\lambda,\sfS}(G)$,   $\Coh_{\Lam,\sfS}(G)$ and  $\Coh_{\Lam,\sfS}(G)$. The representation  $\Coh_{\Lam,\sfS}(\g,\b)$ of $W_{[\lambda]}$
is independent of the Borel subalgebra $\b$.


\begin{proof}
Suppose that $\lambda$ is regular and let $\Phi\in   \Coh_{[\lambda]}(G)$. By using \cite{Vg}*{Proposition~7.2.22, part (a)}  and the formula in the third line of  \cite{Vg}*{page 472}, we may explicitly recover $\Phi$ from $\Phi(\lambda)$ and show that $\Phi\in \Coh_{[\lambda],\sfS}(G)$ whenever $\Phi(\lambda)\in \Grt_{\lambda, \sfS}(G)$.   Together with the second assertion of Lemma \ref{lem21}, this proves the lemma. Note that the formula in the third line of  \cite{Vg}*{page 472} is a special case of  \cite{Vg}*{Proposition~7.2.22, part (b)}, and Vogan pointed out that \cite{Vg}*{Proposition~7.2.22} is due to Zuckerman (\cite{Zu}).

\end{proof}


%  For the proof of \eqref{eqthemc1}, by possibly replacing $\lambda$ by  an element in  $W_{[\lambda]}.\lambda$, we assume without loss of generality that $\lambda\in \hha^*$ is dominant is the sense that
 % \[
  %  \la \lambda, \alpha^\vee\ra\notin -\BN^+\qquad\textrm{for all $\alpha\in \Delta^+\, $ (the set of positive roots)}.
  %\]



 Let $\Phi\in   \Coh_{[\lambda]}(G)$, then $\Phi\in  \Coh_{[\lambda], \sfS}(G)$  if and only if $\Phi(\mu)\in \Grt_{\mu, \sfS}(G)$.

 \begin{lem}
  Let $\Phi\in   \Coh_{[\lambda]}(G)$, then $\Phi\in  \Coh_{[\lambda], \sfS}(G)$  if and only if $\Phi(\mu)\in \Grt_{\mu, \sfS}(G)$.
    \end{lem}

  By evaluating at $\lambda$ and $\mu$, we have  respective linear maps
   \[
    \mathrm{ev}_{\lambda } \, :\,  \Coh_{[\lambda]}(G) \longrightarrow \Grt_{\lambda}(G)
  \]
  and
    \[
    \mathrm{ev}_{\mu } \, :\,  \Coh_{[\lambda]}(G) \longrightarrow \Grt_{\mu}(G).
  \]
  By \cite{Vg}*{Theorem~7.2.7 and Proposition~7.2.27}, the first  map   is surjective, and the second one is an isomorphism.

    \begin{lem}
     The map $\mathrm{ev}_{\lambda } $ is surjective and the map $\mathrm{ev}_{\mu}$ is bijective.
  \end{lem}



  \begin{lem}
     The map $\mathrm{ev}_{\lambda } $ is surjective and the map $\mathrm{ev}_{\mu}$ is bijective.
  \end{lem}
  \begin{proof}
Evaluations at $\lambda$ and $\mu$ respectively yields linear maps
   \[
    \widetilde{\mathrm{ev}}_{\lambda } \, :\,  \Coh_{[\lambda]}(G) \longrightarrow \CK_{\lambda,\Nil(\g^*)}(G)
      \]
  and
    \[
   \widetilde{ \mathrm{ev}}_{\mu } \, :\,  \Coh_{[\lambda]}(G) \longrightarrow \CK_{\mu,\Nil(\g^*)}(G).
  \]
By \cite{Vg}*{Theorem~7.2.7 and Proposition~7.2.27}, the first  map   is surjective, and the second one is an isomorphism.
\end{proof}
 \[
    \mathrm{ev}_{\lambda } \, :\,  \Coh_{[\lambda],\sfS}(G) \longrightarrow \Grt_{\lambda,\sfS}(G)
  \]
  and
    \[
    \mathrm{ev}_{\mu } \, :\,  \Coh_{[\lambda],\sfS}(G) \longrightarrow \Grt_{\mu,\sfS}(G).
  \]


  \begin{lem}
  The kernel of $\mathrm{ev}_\lambda$ equals $\Span\Set{\Phi- w\cdot \Phi |\Phi\in \Coh_{[\mu],\sfS}(G)} $.
  \end{lem}
  \begin{proof} \[
    \begin{split}
      \ker \ev{\mu} = & \Span \Set{\Phi_{\pi}| \pi\in \Irr_{\lambda, \sfS}(G) \text{
          s.t. }
        \tau(\pi)\cap R_{\mu}\neq \emptyset}\\
      \subseteq & \Span \Set{\half(\Phi_\pi - s_{\alpha}\cdot \Phi_{\pi}) | \pi\in \Irr_{\lambda, \sfS}(G) \text{
          and } \alpha \in
        \tau(\pi)\cap R_{\mu}}\\
      & \ \ \ \  \text{(by the definition of $\tau(\pi)$ in \eqref{eq:taupi}.)} \\
      \subseteq &\Span\Set{\Phi- w\cdot \Phi |\Phi\in \Coh_{[\mu],\sfS}(G)} \\
      \subseteq & \ker \ev{\mu}. \\
      &\ \ \ \ \text{(by
        $w\cdot \Phi(\mu) = \Phi(w^{-1}\cdot \mu)=\Phi_\pi(\mu)$)}
    \end{split}
  \]

\end{proof}


\begin{lem}\label{lem:coh.count}
  For each $\mu$ and closed $\Inn(\fgg)$-invariant subset $\sfS$ in the
  nilpotent cone of $\fgg$, we have an isomorphism
  \[
    \bev{\mu}\colon \left(\Coh_{[\mu],\sfS}(G)\right)_{W_{\mu}} \longrightarrow \Grt_{\mu,\sfS}(G).
  \]
  In particular,
  \[
    \sharp(\Irr_{\mu,\sfS}(G)) = [1_{W_{\mu}}:\Coh_{[\mu],\sfS}(G)]
  \]
  % \[
  %   \dim {\barmu} = \dim (\cohm)_{W_\mu} = [\cohm, 1_{W_\mu}].
  % \]
\end{lem}
\begin{proof}
  This is a consequence of the formal properties of the translation functor,
  especially the theory of $\tau$-invariant.

  \def\Parm{\mathrm{Parm}} \def\cof{\Phi}

  {\bf The properties of the coherent family and translation principal:} (We
  refers to \cite{Vg}*{Section~7} for the proofs which also work in our
  (possibly non-linear) setting.)

  \begin{enumerate}[label=(\alph*)]
    \item \label{it:t1} The evaluation map
          \[
          \ev{\mu}\colon \Coh_{[\mu]}(G)\longrightarrow \Grt_{\mu}(G)
          \]
          is surjective for any $\mu \in [\mu]$, see \cite{Vg}*{Theorem~7.2.7}.
  \end{enumerate}
  Without of loss of generality, we may assume that $\mu\in \ahh$ is dominant
  and we fix a regular element $\lambda \in [\mu]$ so that $\mu$ is dominant
  with respect to $R^{+}_{[\lambda]}$.
  \begin{enumerate}[resume*]
    \item \label{it:t2} The evaluation map $\ev{\lambda}$ at $\lambda$ is an
          isomorphism \cite{Vg}*{Proposition~7.2.27}.
  \end{enumerate}
  For each $\pi\in \Grt_{\lambda}(G)$, let
  \[
    \Phi_{\pi}:= (\ev{\lambda})^{-1}(\pi)
  \]
  to be the unique coherent family such that $\Phi_{\pi}(\lambda) = \pi$,
  \begin{enumerate}[resume*]
    \item \label{it:t3} If $\pi\in \Irr_{\lambda}(G)$, then $\Phi_{\pi}(\mu)$
          is either zero or an irreducible $G$-module
          \cite{Vg}*{Proposition~7.3.10, Corollary~7.3.23}.
    \item \label{it:t4} For $\pi\in \Irr_{\lambda}(G)$,
          \[
          \AV(\Phi_{\pi}(\mu)) = \AV(\pi)
          \]
          whenever $\mu$ is dominant and $\Phi_{\pi}(\mu)$ non-zero.
          \trivial[]{ This because
          $\pi = \psi_{\mu}^{\lambda}(\Phi_{\pi}(\mu))$ and
          $\Phi_{\pi}(\mu) = \psi_{\lambda}^{\mu}(\pi)$. Here is the
          translation functor from $\lambda$ to $\mu$ see
          \cite{Vg}*{Definition~4.5.7}. Translation dose not increase the
          associated variety. }
    \item \label{it:t5} If $\pi$ and $\pi'$ are in $\Irr_{\lambda}(G)$ such that
          $\Phi_{\pi}(\mu) = \Phi_{\pi'}(\mu)$ is non-zero, then $\pi=\pi'$
          (see \cite{V4}*{Corollary~7.3.23}).
  \end{enumerate}
  For $\pi\in \Irr_{\lambda}(G)$, define the $\tau$-invariant of $\pi$ to be
  \begin{equation}\label{eq:taupi}
    \tau(\pi) := \Set{\alpha\in R^{+}_{[\lambda]}|
      \begin{array}{l}
        \text{$\alpha$ is simple and }\\
        s_{\alpha}\cdot \Phi_{\pi}(\lambda) = -\Phi_{\pi}(\lambda)
      \end{array}
    }
  \end{equation}
  \begin{enumerate}[resume*]
    \item
          \label{it:t6}
          $\Phi_{\pi}(\mu) =0$ if and only if
          $\tau(\gamma)\cap R_\mu \neq \emptyset$
          \cite{Vg}*{Corollary~7.3.23~(c)}.
  \end{enumerate}

  Now we start to prove the lemma. By \ref{it:t4}, we see that $\ev{\lambda}$
  induces a $\WLam$-module isomorphism
  \[
    \ev{\lambda} \colon \Coh_{[\lambda],\sfS}(G) \xrightarrow{\ \ \cong \ \ } \Grt_{\lambda,\sfS}(G).
  \]

  By \ref{it:t6}, the set %the translation principle,
  \[
    \begin{split}
      & \Set{\Phi_{\pi}(\mu)| \pi\in \Irr_{\lambda, \sfS}(G)
        \text{ s.t. } \Phi_{\pi}(\mu)\neq 0} \\
      = & \Set{\Phi_{\pi}(\mu)| \pi\in \Irr_{\lambda, \sfS}(G) \text{ s.t.
        } \tau(\pi)\cap R_{\mu}= \emptyset}.
    \end{split}
  \]
  forms a basis of $\Grt_{\mu,\sfS}(G)$. \trivial{ The set consists of distinct
    (so linearly independent) irreducible $G$-modules by \ref{it:t3} and
    \ref{it:t5}. They are spanning set by \ref{it:t1}. For the support
    condition, see \ref{it:t4}. The $\tau$-invariant condition is by
    \ref{it:t6}.
    % by \ref{it:t6} and
  } Hence
  \[
    \begin{split}
      \ker \ev{\mu} = & \Span \Set{\Phi_{\pi}| \pi\in \Irr_{\lambda, \sfS}(G) \text{
          s.t. }
        \tau(\pi)\cap R_{\mu}\neq \emptyset}\\
      \subseteq & \Span \Set{\half(\Phi_\pi - s_{\alpha}\cdot \Phi_{\pi}) | \pi\in \Irr_{\lambda, \sfS}(G) \text{
          and } \alpha \in
        \tau(\pi)\cap R_{\mu}}\\
      & \ \ \ \  \text{(by the definition of $\tau(\pi)$ in \eqref{eq:taupi}.)} \\
      \subseteq &\Span\Set{\Phi- w\cdot \Phi |\Phi\in \Coh_{[\mu],\sfS}(G)} \\
      \subseteq & \ker \ev{\mu}. \\
      &\ \ \ \ \text{(by
        $w\cdot \Phi(\mu) = \Phi(w^{-1}\cdot \mu)=\Phi_\pi(\mu)$)}
    \end{split}
  \]
  Since
  $\left(\Coh_{[\mu],\sfS}(G)\right)_{W_{\mu}} =\Coh_{[\mu],\sfS}(G)\slash \Span\set{\Phi- w\cdot \Phi |\Phi\in \Coh_{[\mu],\sfS}(G)} $,
  the lemma follows.
\end{proof}

\begin{remark}
  The above argument can be easily generalized to other situations when
  translation principal holds. For example, we define the space
  $\Coh_{\Lam,\sfS}(\fgg,\fhh,\fnn)$ of coherent families in category $\cO$
  whose associated variety are contained in $\sfS$. Then, for $\mu\in \Lam$,
  $\ev{\mu}$ also induces an isomorphism
  \[
    \left(\Coh_{[\lambda],\sfS}(\fgg,\fhh,\fnn)\right)_{W_{\mu}} \xrightarrow{\ \ \cong \ \ } \Grt_{W\cdot \mu,\sfS}(\fgg,\fhh,\fnn).
  \]
\end{remark}



\section{Counting representations via the translation principle}

% \subsection{Coherent family}

% Let $Q$ be the $\fhh$ root lattice of $\fgg$. Let $\cG$ be the Grothendieck
% group of finite dimensional $\fgg$-modules occur in $S(\fgg)$. Note that $\cG$
% is also an algebra under tensor product.
% % Via highest weight theory, there is a unique irreducib Via highest weight
% % theory, $Q$ is identified with the set of irreducible $\fgg$-modules occur
% % in $S(\fgg)$.
% For each finite dimensional $\fhh$-module $F$, let $\WT{F}$ denote the
% multi-set of $\fhh$-weights in $F$.
In this section, let $\fgg$ be a reductive Lie algebra with a fixed abstract
Cartan subalgebra $\hha$. Recall the definition of coherent families.

Let
\[
  \aSR \subseteq \aR\subsetneq \aQ \subsetneq \ahh^{*}
\]
be the set of simple roots, positive roots and root lattice in $\ahh^{*}$.

Let $W$ be the Weyl group of $\fgg$ generated by simple reflections
$\set{s_{\alpha}|\alpha \in \aSR}$ which acts on $\ahh^{*}$.

Via the highest weight theory, every $W$-orbit $W\cdot \mu$ in $\aQ$ corresponds
with the irreducible finite dimensional $\fgg$-representation $F_{\mu}$ with
extremal weight $\mu$ and $F_{\mu}$ occurs in $\rS(\fgg)$.


% Now the Grothendieck group $\Gfin$ of finite dimensional representation of
% $\Gc$ is identified with $\bZ[\aP/W]$. In fact $\Gfin$ is a $\bZ$-algebra
% under the tensor product and equipped with the involution $F\mapsto F^*$.

% Fix a $W$-invariant sub-lattice $\Lambda_0\subset \aX$ containing $\aQ$.

% Let $\Pi$ $\Glfin$ be the $\star$-invariant subalgebra of $\Gfin$ generated by
% irreducible representations corresponds to $\Lambda_0/W$.


For any $\lambda\in \hha^{*}$, we consider the lattice
\[
  \Lam := \lambda + Q \subset \ahh^{*}
\]
and define
\begin{equation}
  \label{eq:wlam}
  \begin{split}
    R_{[\lambda]} &:= \Set{\alpha\in \aR| \inn{\lambda}{\ckalpha}\in \bZ},\\
    W_{[\lambda]} &:=
    \set{w\in W | w\cdot \lambda  - \lambda \in \aQ}\\
    R_{\lambda} &:= \Set{\alpha\in \aR| \inn{\lambda}{\ckalpha}=0}, \AND\\
    W_{\lambda} &:= \braket{s_\alpha|\alpha\in R_{\lambda}} = \braket{w\in W|w\cdot \lambda = \lambda} \subseteq W.
  \end{split}
\end{equation}
It is known that $R_{[\lambda]}$ is a root system (see \cite{V4}*{\S 2}) and
\[
  \begin{split}
    W_{[\lambda]} &= \Stab_{W}([\lambda]) = \braket{s_\alpha|\alpha\in R_{[\lambda]}} \subseteq W\quad
    \text{and}\\
    W_{ \lambda } &=   \braket{s_\alpha|\alpha \in R_{\lambda}} \subseteq W_{[\lambda]}.\\
  \end{split}
\]
In fact $W_{\lambda}$ is a parabolic subgroup of $W_{[\lambda]}$ by Chevalley's
theorem \cite{Vg}*{Lemma~6.3.28}.

When $\lambda$ is regular, let
\[
  R^{+}_{\Lam} := \Set{\alpha\in R_{\Lam}| \inn{\ckalpha}{\lambda}>0}
\]
be the fixed positive root system in $R_{\Lam}$. \trivial[]{ There is no harm to
  assume in the very beginning that $\lambda$ is regular dominant with respect
  to the $\aSR$, i.e. $\inn{\lambda}{\ckalpha}>0$ for all $\alpha\in \aSR$. }

In the following we define the notion of coherent family based on the lattice
$\Lam$ in a quite general setting.

\begin{defn}
  Suppose that $\cK$ is an $\Rg$-module with action:
  \[
    \begin{array}{ccc}
      \Rg \times \cK & \xrightarrow{\ \ \ \otimes \ \ \ } & \cK\\
      (F,m) & \mapsto & F\otimes m.
    \end{array}
  \]
  In addition, a subgroup $\cK_{\mu}$ of $\cK$ is fixed for each $\mu \in \Lam$
  such that $\cK_{\mu} = \cK_{w\cdot \mu}$ for any $\mu\in [\lambda]$ and
  $w\in \WLam$.
  % for each $W_{[\lambda]}$-orbit
  % $\barmu := W_{\Lambda} \cdot \mu\in \Lambda/W_{\Lambda}$.

  A function $f\colon \Lam \rightarrow \cK$ is called a coherent family based on
  $\Lam$ if it satisfies $f(\mu)\in \cK_\mu$ and
  \[
    F\otimes f(\mu) = \sum_{\nu \in \WT{F}} f(\mu+\nu) \qquad \forall \mu\in \Lam, F\in \Rg.
  \]
  where $\WT{F}$ denotes the multi-set of $\fhh$-weight in $F$. Let
  $\Coh_{\Lambda}(\cK)$ be the abelian group of all coherent families based on
  $\Lam$ and taking value in $\cK$. We can define $\WLam$ action on
  $\Coh_[\lambda](\cK)$ by
  \[
    w\cdot f(\mu) = f(w^{-1}\cdot \mu) \qquad \forall \mu\in \Lam, w\in \WLam.
  \]


  Let $\Coh_{\Lam}(\cK_{\mu})$ denote the vector space of all coherent families
  on $\Lam$ taking value in $\cK_{\mu}$ at $\mu\in \Lam$. It is a representation
  of $\WLam$ under the action
  \[
    (w\cdot \Phi)(\mu) = \Phi(w^{-1}\cdot \mu), \qquad \textrm{for all
    }\ w\in W_\Lam, \ \mu\in \Lam.
  \]
\end{defn}
For any $\mu\in \Lam$, let
\[
  \ev{\mu}\colon \Coh_{\Lam}(\cK_{\mu}) \longrightarrow \cK_{\mu}
\]
be the map of evaluation at $\mu$. The purpose of this section is to understand
the $\ev{\mu}$.

In this paper, we will consider the following cases.
\begin{eg}
  Suppose $\cK=\bC$ and
  \[
    F\otimes m := \dim(F)\cdot m \quad \text{for all } F\in \cG \text{ and
    } m\in \cK.
  \]
  We let $\cK_{\mu} := \cK = \bC$ for every $\mu\in \Lam$. Then the set of
  $W$-harmonic polynomials on $\fhh$ is naturally identified with
  $\Coh_{\Lam}(\cK)$ via the restriction on $[\lambda]$ by Vogan
  \cite{VGK}*{Lemma~4.3}. \trivial{ Note that the polynomials are $W$-harmonic
    not necessary $W_{[\lambda]}$-harmonic. ($W_{[\lambda]}$-invariant
    differential operators are more than $W$-invariant differential operators.)
  }
\end{eg}


\begin{eg}\label{eg:hw}
  Fix a Cartan subalgebra $\fhh$ and the identification
  $\ihh\colon \fhh\rightarrow \ahh $ of $\fhh$ with $\ahh$. The choice of $\ihh$
  determines the positive root system in $\fhh^{*}$ by transport of structure.

  Let $\fbb$ be the Borel subalgebra generated by $\fhh$ and the negative
  $\fhh$-roots in $\fgg$ and $\fnn$ be the nilpotent radical of $\fbb$.

  Let $\Grt(\fgg,\fhh,\fnn)$ be the Grothendieck group of the category $\cO$
  with coefficients in $\bC$, i.e. the category of finitely generated
  $\cU(\fgg)$-modules with semisimple $\fhh$-action and locally finite
  $\fnn$-action. There $\Rg$ acts on $\Grt(\fgg,\fhh,\fnn)$ via the tensor
  product of $\fgg$-modules. For each $W$-orbit $W\cdot \mu\in \fhh^{*}/W$, let
  \[
    \Grt_{W\cdot \mu}(\fgg,\fhh,\fnn)
  \]
  be the subgroup spanned by the $\fgg$-modules with infinitesimal character
  $\chi_{\mu}$.

  Let
  \[
    \Coh_{[\lambda]}(\fgg,\fhh,\fnn) := \Coh_{[\lambda]}(\cK_{\mu}) \quad \text{with} \quad \cK_{\mu} := \Grt_{W\cdot \mu}(\fgg,\fhh,\fnn).
  \]
  be the coherent family of highest weight modules.


  % Verma modules gives a basis of We now review the well understood structure
  % of $\Coh_{[\lambda]}$. is well understood.
  Let $\rho := \sum_{\alpha\in \WT{\fnn}} \alpha$. For $\lambda\in \ahh^{*}$,
  let
  \[
    M(\lambda) := \cU(\fgg)\otimes_{\cU(\fbb)} \bC_{\ihh^{-1}(\lambda)-\rho}
  \]
  be the Verma module with highest/lowest weight $\ihh^{-1}(\lambda)-\rho$ and
  $L(\lambda) $ be the unique irreducible quotient of $M(\lambda)$.

  Each $w\in W$ defines a coherent family %such that
  \[
    M_w(\mu) := M(w\cdot \mu) \quad \forall \mu \in [\lambda].
    % M_w(\mu) := M(w \cdot \mu) \quad \forall \mu \in [\lambda].
  \]
  % where $w_{0}$ is the longest element in $W$.

  The map
  \[
    \begin{array}{ccc}
      \bC[W] & \longrightarrow & \Coh_{[\lambda]}(\fgg,\fhh,\fnn)  \\
      w& \mapsto &M_{w}
    \end{array}
  \]
  is $W_{[\lambda]}$-module isomorphism where $W_{[\lambda]}$ acts on $\bC[W]$
  by right translation.


  We remark that, the setup here is opposite to the usual one. But it is
  consistent with the Springer theory when we study primitive ideals. Under this
  setup, $\AVC(L_{e}(\lambda))$ is the nil-cone.

  One of the crucial property is that each irreducible module can be fitted into
  a coherent family. %in other words.
  More precisely, the evaluation map descents to yields an isomorphism
  $\bev{\mu}$ in the following diagram.
  \begin{equation}\label{eq:bev.catO}
    \begin{tikzcd}
      \Coh_{[\lambda]}(\fgg,\fhh,\fnn)\ar[r,"\Phi\mapsto \Phi(\mu)"] \ar[d]&
      \Grt_{W\cdot \mu}(\fgg,\fhh,\fnn)\\
      \left(\Coh_{[\lambda]}(\fgg,\fhh,\fnn)\right)_{W_{\mu}} \ar[ru,hook,two heads,"\bev{\mu}"']
    \end{tikzcd}
  \end{equation}

  \trivial[]{ The subjectivity is because of Verma modules form a basis of the
    category $\cO$. The LHS of $\bev{\mu}$ has dimension $\sharp {W/W_{\mu}}$,
    the RHS has dimension $\sharp W\cdot \mu$. Now the isomorphism follows by
    dimension counting. }
  % the The space $\Coh_\Lambda(\cG(\fgg,\fhh,\fnn))$ and
  % $\Coh_{\Lambda(\cG)}$defined similarly.

  % Note that the lattice $\Lambda$ is stable under the $\Wlam$ action.
\end{eg}

\begin{eg}\label{eg:Coh.HC}
  Suppose $G$ is a reductive Lie group in the Harish-Chandra class. Let
  $\Grt(G)$ be the Grothendieck group of finite length admissible $G$-modules
  and $\Grt_{W\cdot \mu}(G)$ be the subgroup of $\Grt(\fgg,K)$ generated by the
  set of irreducible $G$-modules with infinitesimal character $\chi_{\mu}$. By
  pulling back via the adjoint action $G\rightarrow \Inn(\fgg)$, we identify
  $\Rg$ with a subgroup in $\Grt(G)$. Let $\Rg$ acts on $\Grt(G)$ via tensor
  product.

  We write
  \[
    \Coh_{[\lambda]}(G) := \Coh_{[\lambda]}(\Grt_{\mu}(G))
  \]
  for the space of coherent family of $G$-modules which taking values in
  $\Grt_{\mu}(G)$ at $\mu\in [\lambda]$.

  \trivial[]{ By \cite{Vg}*{0.4.6}, we can naturally identify $Q$ with the set
    of $H^{s}$-weights consisting the characters occurs in $S(\fgg)$ where
    $H^{s}$ is a maximally split Cartan in $G$. Therefore, the set of
    irreducible $G$-submodules occur in $S(\fgg)$ is also naturally identified
    with $Q/W=\Rg$. We let $\Rg$ acts on $\Grt(G)$ by the tensor product of
    $G$-modules.

    Note that by the assumption that $G$ is in the Harish-Chandra class, each
    irreducible $\fgg$-submodule $F$ embeds in $S(\fgg)$ is automatically
    globalized to a $G$-module. The point is that the globalization is
    independent of the embedding of $F$ in $S(\fgg)$! }

\end{eg}

\begin{eg}
  Fix a $\Inn(\fgg)$-invariant closed subset $\sfS$ in the nilpotent cone of
  $\fgg$. Let $\Grt_{\sfS}(G)$ be the subgroup of $\Grt(G)$ spanned by
  $G$-modules whose complex associated varieties are contained in $\sfS$. For
  each $\mu\in \ahh^{*}$, define
  \[
    \Grt_{\mu,\sfS}(G) := \Grt_{ \mu}(G)\cap \Grt_{\sfS}(G).
  \]
  and write
  \[
    \Coh_{[\lambda],\sfS}(G):= \Coh_{[\lambda]}(\Grt_{\mu,\sfS}(G))
  \]
  which is a subspace of $\Coh_{\Lam}(G)$.

  \trivial[]{ Note that
    \[
      \AVC(\pi\otimes F) = \AVC(\pi).
    \]
    for each finite length $G$-module $\pi$ and finite dimensional $G$-module
    $F$. Therefore the $W_{[\lambda]}$-module
    \[
      \Coh_{[\lambda],\sfS}(G) = (\ev{\mu})^{-1}(\Grt_{\mu,\sfS}(G))
    \]
    for any regular $\mu\in [\lambda]$. }
\end{eg}


% \begin{eg}
%   For each infinitesimal character $\chi$ and a close $G$-invariant set
%   $\cZ\in \cN_{\fgg}$. Let $\Grt_{\,\cZ}(\fgg,K)$ be the Grothendieck group of
%   $(\fgg,K)$-module with infinitesimal character $\chi$ and complex associated
%   variety contained $\cZ$. Similarly, let $\Grt_{\chi,\cZ}()$
% \end{eg}

The following lemma shows that the diagram \cref{eq:bev.catO} still holds for
$\Coh_{[\lambda],\sfS}(G)$ and \Cref{count1} is its direct consequence.

% This is one of the first step towards the counting of the set
% \[
%   \Irr_{\mu,\sfS}(G):=\Set{\pi \in \Irr_{\mu}(G)| \text{$\AVC(\pi)\subset \sfS$} }.
% \]
% where
% \[
%   \Irr_{\mu}(G):=\Set{\pi \in \Irr(G)| \pi \text{ has infinitesimal character
% }\mu}.
% \]


\section{Review of Primitive ideals and Weyl group representations}

\subsection{Associated varieties of a primitive ideals and double cells in
  $\WLam$}
In this section, we review the notion of double cells and its relation with the
associated varieties of primitive ideals, see \cite{BV2,J.av}. We follows
closely to \cite{BV2}. We retain the notation in \Cref{eg:hw}.

Let $\Prim_{W\cdot \lambda}(\fgg)$ be the set of primitive ideals in $\cU(\fgg)$
with infinitesimal character $\lambda$. Let $\lambda \in \fhh^{*}$, each
primitive ideal is the annihilator of a highest weight module by Duflo
\cite{Du77}. In other words, the following map is surjective
\[
  \begin{array}{ccl}
    \WLam &\longrightarrow &  \Prim_{W\cdot \lambda}(\fgg)\\
    w & \mapsto & I(w\cdot \lambda) := \Ann L(w\cdot \lambda).
  \end{array}
\]



% We now recall some results about the blocks in category $\cO$.

By the translation principal, we concentrate the discussion in the regular
infinitesimal character case.

% From now on, we follows the convention in \cite{BV2}. Let $\lambda$ be a
% regular element in $\fhh^{*}$ such that
% $R^{+}_{[\lambda]}\subset - \WT{\fnn}$. \trivial{ Here $\lambda$ is regular
% anti-dominant ($\inn{\lambda}{\ckalpha}\notin \bN$ for each
% $\alpha\in \WT{\fnn}$) with respect to the root system defining highest weight
% modules, but it is dominant with respect to $R^{+}_{[\lambda]}$. }

For each $w\in W_{[\lambda]}$, define
\[
  a(w) := \abs{\WT{\fnn}} - \GKdim(L(w\lambda)).
\]
\trivial{ Suppose $\lambda$ is integral, then Under this definition,
  $a_{w_{0}} = \abs{\WT{\fnn}}$ and $a_{e} =0$. } For each $w$, one can attach a
polynomial $\wtpp_{w}$ such that
$\wtpp_{w}(\mu) = \rank(\cU(\fgg)/\Ann(L(w\mu)))$ when $\mu\in [\lambda]$ is
dominant (i.e. $-\inn{\mu}{\ckalpha}\notin \bN^{+}$ for all
$\alpha\in R^{+}_{[\lambda]}$). $\wtpp_{w}$ is called the Goldie-rank polynomial
attached to the primitive ideal $\Ann(L(w\lambda))$. Fix a dominant regular
element $\delta$ in $\fhh$ (i.e. $\inn{\delta}{\alpha}>0$ for each
$\alpha\in \WT{\fnn}$). Let
\[
  r_{w} = \sum_{y\in W_{[\lambda]}} a_{y,w} (y^{-1}\delta)^{a(w)} \in S(\fhh)
\]
where $a_{y,w}$ is determined by the equation
\[
  L(w\lambda) = \sum_{y\in W_{[\lambda]}} a_{y,w} M(w\lambda)
\]
in $\Grt(\fgg,\fhh,\fnn)$. Then $r_{w}$ is a positive multiple of $\wtpp_{w}$
\cite{J2}*{Section~1.4}.

% Let $w_{0}$ (resp. $w_{[\lambda]}$)be the longest element in $W$ (resp.
% $\Wlam$).
A partial order $\leqL$ can be define on $W_{[\lambda]}$ by the following
condition \cite{BV2}*{Proposition~2.9}
\begin{equation}\label{eq:leqL}
  \begin{split}
    w_{1} \leqL w_{2} & \Leftrightarrow
    I(w_{1}\lambda)\subseteq I(w_{2}\lambda)\\
    & \Leftrightarrow
    % [L(w_{2}^{-1}\lambda), L(w_{1}^{-1}\lambda)\otimes S(\fgg)] \neq 0.
    L(w_{2}^{-1}\lambda) \text{ is a subquotient of
    } L(w_{1}^{-1}\lambda)\otimes S(\fgg).
  \end{split}
\end{equation}

We say $w_{1} \approxL w_{2}$ if and only if $w_{1}\leqL w_{2}\leqL w_{1}$. For
$w\in W_{[\lambda]}$, we call
\[
  \CL_{w} := \Set{ w' \in W_{[\lambda]}| w\approxL w'}
\]
the left cell in $W_{[\lambda]}$ containing $w$.

In summary, we have a bijection
\[
  \begin{array}{ccc}
    W_{[\lambda]}/ \approxL &\longrightarrow & \Prim_{\lambda}(\fgg)\\
    \CL_{w} & \mapsto & \Ann L(w\lambda).
  \end{array}
\]
% here $w$ is a arbitrary element in $\LC$. Moreover, for each left cell
% $\LC_{w}$



The partial order $\leqR$ is defined by
\[
  w_{1}\leqR w_{2} \Leftrightarrow w_{1}^{-1} \leqL w_{2}^{-1}.
\]
The partial order $\leqLR$ is defined to be the minimal partial order containing
$\leqL$ and $\leqR$. The relation $\approxR$, $\approxLR$, right cell $\CR_{ w}$
and double cells $\CLR_{w}$ are defined similarly.

Since the Kazhdan-Lusztig conjecture has been proven (for the integral
infinitesimal character case by \cite{BB,BK} and reduced to the integral
infinitesimal character case by \cite{Soergel} (see \cite{H}*{Section~13.13})),
the definition of the order $\leqL$ is the same as the partial order defined by
Kazhdan-Lusztig \cite{KL} which only depends on the Coxeter group structure of
$W_{[\lambda]}$, see \cite{BV2}*{Corollary~2.3}. \trivial[]{ Note that
  $x\lneqL y$ implies $a(x)<a(y)$ and $x\approxLR y$ implies $a(x)=a(y)$. }

Note that the left cells are exactly the fibers of the map $w\mapsto \wtpp_{w}$.
Take a double cell $\CLR_{w}$ in $W_{[\lambda]}$ and a set of representatives
$\set{w_{1}, w_{2}, \cdots, w_{k}}$ of the left cells in $\CLR_{w}$.
% and decompose it into disjoint union of left-cells
% \[
%   \LRC_{w} = \bigsqcup_{i=1}^{k} \LC_{w_{i}}.
% \]
% Due to Joseph\cite{J2} and Barbasch-Vogan\cite{BV1,BV2}, we have the following
% statements:
%
Due to Barbasch-Vogan\cite{BV1,BV2} and Joseph\cite{J1,J2,J3,J.av}, the
following statements holds:
\begin{itemize}
  \item the set of Goldie rank polynomials
        $\set{\wtpp_{w_{i}}|i = 1,2,\cdots,k}$ form a basis of a special
        representation $\sigma_{w}$ of $W_{[\lambda]}$ realized in
        $S^{a(w)}(\fhh)$;
  \item the multiplicity of $\sigma_{w}$ in $S^{a(w)}(\fhh)$ is one,
  \item $a(w)$ is the minimal degree $m$ such that $\sigma_{w}$ occurs in
        $S^{m}(\fhh)$ which is the fake degree and the generic degree of the
        special representation $\sigma_{w}$. \trivial[]{ When
        $W_{[\lambda]} = W$, this is the definition of the fake degree.
        Otherwise, $\fhh = \fhh_{0}\oplus \fhh^{W_{\lambda}}$ where $\fhh_{0}$
        is the span of coroots of $W_{[\lambda]}$. Then $S(\fhh_{0})$ is embeds
        in $S(\fhh)$. }
  \item the map
        $W_{[\lambda]}/\approxLR \; \ni \CLR_{w}\mapsto \sigma_{w}\in \Irr(W_{[\lambda]})$
        yields a bijection between the set of double cells and the set
        $\Irrsp(W_{[\lambda]})$ of special representations of $W_{[\lambda]}$.
  \item Under the $W$ action, the $W_{[\lambda]}$-module
        $\sigma_{w}\subset S^{a(w)}(\fhh)$ generates an irreducible $W$-module
        $\wtsigma_{w}:=j_{W_{[\lambda]}}^{W}\sigma_{w}$. The $W$-module
        $\wtsigma_{w}$ corresponds to the a nilpotent orbit $\cO_{\wtsigma_{w}}$
        with trivial local system under the Springer correspondence. Now the
        complex associated variety
        \[
        \AVC(L(w\lambda)) = \Gad\cdot \AV(L(w\lambda)) =\bcO_{\wtsigma_{w}},
        \]
        see \cite{J.av}*{Section~2.10}.
\end{itemize}
\trivial[]{ Note that the $j$-induction is not injective in general. For
  example,
  $j_{S_{a}\times S_{b}}^{S_{a+b}} \tau_{a}\otimes \tau_{b} = \tau_{a}\cupcol \tau_{b}$
  where $\tau_{a}$, $\tau_{b}$ are partitions. }

% \subsubsection*{Clan of primitive ideals}
% Under the above notation, we say two
Following Joseph, we say two primitive ideals $I(w\lambda)$ and $I(w'\lambda)$
with $w, w'\in \WLam$ are called in the same \emph{clan} if and only if
$w\approxLR w'$ or equivalently $\sigma_{w}=\sigma_{w'}$.

Obviously, the associated varieties of two primitive ideals in the same clan has
the same associated variety. However the reverse dose not holds in the
non-integral infinitesimal character case in general.

\subsubsection*{Coherent continuation representations}
Now we recall the relationships between cells and the coherent continuation
representations.

For each $w\in W$, we define a coherent family $L_{w}$ by the condition
$L_{w}(\lambda) = L(w\cdot \lambda)$.

For each $\nu\in \fhh^{*}$, let $\cO_{[\nu]}(\fgg,\fhh,\fnn)$ be the subcategory
of the category $\cO(\fgg,\fhh,\fnn)$ consists of $\fgg$-modules whose
$\fhh$-weights are contained in $[\nu]$,
\[
  \Grt_{W\cdot\lambda,[\nu]}(\fgg,\fhh,\fnn) := \Grt_{[\nu]}(\fgg,\fhh,\fnn)\cap \Grt_{W\cdot \lambda}(\fgg,\fhh,\fnn)
\]
which is spanned by a block in the category $\cO$ if
$W\cdot \lambda \cap [\mu+\rho]\neq \emptyset$.

Consider the following subgroup of coherent continuation
\[
  \Coh_{[\lambda]}(\fgg,\fhh,\fnn; [\nu]) = \Set{\Phi\in \Coh_{[\lambda]}(\fgg,\fhh,\fnn)| \Phi(\lambda) \in \Grt_{W\cdot \lambda, [\nu]}(\fgg,\fhh,\fnn) }.
\]
We can identify $\bC[\WLam]$ with $\Coh_{\Lam}(\fgg,\fhh,\fnn,[\lambda-\rho])$
via $w \mapsto M_{w}$.

For each $w\in W_{[\lambda]}$, let
\[
  \begin{split}
    \bVR_{w}& :=\Span\Set{L_{w'}|w\leqR w'} \\
    \VR_{w}&= \bVR _{w}\left/ \sum_{w\lneqL w'} \bVR_{w'} \right.
  \end{split}
\]
By \eqref{eq:leqL}, $\bVR_{w}$ and $\VR_{w}$ are $W_{[\lambda]}$-modules under
right translation/coherent continuation action.

We define left $W_{[\lambda]}$-module $\bVL_{w}$ and $\VL_{w}$ using $\leqL$ and
$W_{[\lambda]}\times W_{[\lambda]}$-module $\bVLR_{w}$ and $\VLR_{w}$ using
$\leqLR$ similarly.

Suppose $\sigma_{1}, \sigma_{2}\in \Irr(W_{[\lambda]})$. We define
\[
  \sigma_{1}\leqLR \sigma_{2} \Leftrightarrow \exists w\in W_{[\lambda]} \text{
    such that }
  \begin{cases}
    \sigma_{1} \otimes \sigma_{1} \text{ occurs in } \VLR_{w}
    \text{ and } \\
    \sigma_{2} \otimes \sigma_{2} \text{ occurs in } \bVLR_{w}.
  \end{cases}
\]
Now $\sigma_{1}\approxLR \sigma_{2}$ if and only if there exists a
$w\in W_{[\lambda]}$ such that $\sigma_{1} \otimes \sigma_{1}$ and
$\sigma_{2} \otimes \sigma_{2}$ both occur in $\VLR_{w}$. Now $\leqLR$ is a well
defined partial order and $\approxLR$ is an equivalent relation on
$\Irr(W_{[\lambda]})$ respectively. We write
$\LRC_{\sigma}\subseteq \Irr(\Wlam)$ for the double cell containing $\sigma$.
\trivial[]{ A priori
  $\sigma_{1}\approxLR \sigma_{2}\Leftrightarrow \sigma_{1}\leqLR \sigma_{2}\leqLR \sigma_{1}$.

  But note that $\bigoplus_{w\in \Wlam/\approxLR} \VLR_{w} \cong \bC[\Wlam]$ and
  $\sigma\otimes \sigma$ has multiplicity one in $\bC[\Wlam]$ which implies the
  claim. }

A left (resp. right cell) in $\Irr(\Wlam)$ is the multiset of the irreducible
constituents in $\VL_{w}$ (resp. left cell) for some $w\in \Wlam$.

The equivalence of Barbasch-Vogan's definition and Lusztig's definition of cells
in $\Irr(\Wlam)$ is a consequence of Kazadan-Lusztig conjecture, see
\cite{BV2}*{remarks after Corollary~2.16}.

The structure of double and left cells are explicitly described in
\cite{Lu}*{Section~4}. In particular, $\sigma_{w}$ is the unique special
representation occurs in the double cell
\[
  \LRC_{w}:= \Set{\sigma| \sigma\otimes \sigma \text{ occurs in
    } \VLR_{w}} \subseteq \Irr(\Wlam).
\]
For this reason, we also write
\[
  \LRC_{\sigma}:=\LRC_{w}
\]
where $\sigma=\sigma_{w}$ is the unique special representation in $\LRC_{w}$.
The generic degree ``a''-function is constant on the double cells and order
preserving: for each $\sigma'\in \LRC_{\sigma}$, the generic degree
$a(\sigma')=a(\sigma)$; $a(\sigma')<a(\sigma'')$ if $\sigma'\lneqLR \sigma''$.

In summary, we have bijections
\[
  \Wlam/\approxLR \longleftrightarrow\Irrsp(\Wlam)\longleftrightarrow \Irr(\Wlam)/\approxLR.
\]
We write $\VLR_{\sigma}$ to be the unique double cell representation containing
$\sigma$ and $\bVLR_{\sigma}$ to be the unique upper cone representation which
is isomorphic to $\bigoplus_{\sigma\leqLR \sigma'}\sigma'\otimes \sigma'$.

Recall that, for each $\sigma\in \Irr(\WLam)$, we define
\[
  \cO_{\sigma}:= \Spr(j_{\WLam}^{W}\sigma_{0}).
\]
where $\sigma_{0}$ is the unique special representation in $\LRC_{\sigma}$.

We have the following well known fact. % from the above discussion.
\begin{lem}\label{lem:LRorbit}
  Suppose $\sigma,\sigma'\in \Irr(\Wlam)$ such that $\sigma\leqLR \sigma'$. Then
  \[
    \bcO_{\sigma} \supseteq \bcO_{\sigma'}
  \]
  and the strict inclusion holds if $\sigma\lneqLR \sigma'$.\qed
\end{lem}
\trivial[]{ Here we give a sketch of the proof. By the definition of cons. There
  exists a chain of elements,
  $w_{1}\leqL w_{2}\leqR w_{3}\cdots \leqL w_{2k} \leqR w_{2k+1} $ such that
  $\sigma \approxLR \sigma_{w_{1}}$ and $\sigma'\approxLR \sigma_{w_{2k+1}}$. Now, for
  $1\leq l\leq k$,
  \begin{itemize}
    \item $\bcO_{\sigma_{w_{2l-1}}}\supseteq \bcO_{\sigma_{w_{2l}}}$ by
          $I(w_{2l-1}\lambda)\subseteq I(w_{2l}\lambda)$ and
    \item $\bcO_{\sigma_{w_{2l}}}\supseteq \bcO_{\sigma_{w_{2l+1}}}$ since
    $L(w^{-1}_{2l+1}\lambda)$ is a subquotient of
     $L(w^{-1}_{2l}\lambda)\otimes S(\fgg)$.
  \end{itemize}
  Since $\cO_{\sigma} = \cO_{\sigma_{w_{1}}}$ and
  $\cO_{\sigma'} = \cO_{w_{2k+1}}$, the inclusion follows.
  The strict inequality is clear by consider the GK-dimension, or the $a$-value. }


  For any $\Gad$-invariant closed set $\sfS$ in the nil-cone of $\fgg$, let
  \begin{equation}\label{eq:C.S}
      \Csp_{\sfS} := \Set{\sigma \in \Irrsp(\WLam)| \Spr(j_{\WLam}^{W}\sigma)\subseteq \sfS}. \\
  \end{equation}
  As  corollary of \Cref{lem:LRorbit}, we have
\[
  \sfC_{\sfS} = \Set{\sigma\in \Irr(\Wlam)|
    \exists \sigma_{0} \in \Csp_{\sfS}
    \text{ such that } \sigma_{0} \leqLR \sigma}.
\]



When study unipotent representations, we need the following lemma by
Barbasch-Vogan \cite{BVUni}.


% Let $a(\sigma)$ be the generic degree of a Weyl group representation $\sigma$.

\begin{lem}[{\cite{BVUni}*{(5.26), Proposition~5.28}}]
  \label{lem:LC.mu}
  \label{lem:lcell.BV}
  Suppose $\mu\in \ahh^{*}$. Let
  \[
    a_{\mu} = \max\set{a(\sigma)| \sigma \in \Irr(W_{[\mu]}) \text{ and
      } [1_{W_{\mu}}: \sigma]\neq 0}.
  \]
  and
  \[
    \LC_{\mu} := \set{\sigma \in \Irr(W_{[\mu]}) | a(\sigma) = a_{\mu} \text{
        and } [1_{W_{\mu}}: \sigma]\neq 0 }.
  \]
  Then
  % \begin{itemize}
            %       \item $\Wlamck$ is a Levi subgroup of $\WLamck$, and
            %       \item
  $\LC_{\mu}$ is a left cell of $W_{[\mu]}$ given by
  \begin{equation}\label{eq:LC.mu}
    \LC_{\mu}=(J_{W_{\mu}}^{W_{[\mu]}} \sgn )\otimes \sgn
  \end{equation}
  which contains a unique special representation
  \[
    \sigma_{\mu}=(j_{W_{\mu}}^{W_{[\mu]}} \sgn )\otimes \sgn.
  \]
  Moreover, $\LC_{\mu}$ is multiplicity free, which is equivalent to
  \[
    [1_{W_{\mu}}:\sigma]=1 \quad \text{for each } \sigma\in \LC_{\mu}.
  \]

  Let
  \begin{equation}\label{eq:O.mu}
    \cO_{\mu} = \Spr(j_{W_{[\mu]}}^{W}\sigma_{\mu}).
  \end{equation}
  Then
  \[
    \bcO_{\mu} = \AV(\cI_{\mu}) \AND \LC_{\mu} = \Set{\sigma\in \sfC_{\bcO_{\mu}}| [1_{W_{\mu}}:\sigma]\neq 0}
  \]
  where $\cI_{\mu}$ is the maximal primitive ideal with infinitesimal character
  $\chi_{\mu}$. \qed
\end{lem}

\trivial{ This is essentially contained in \cite{BVUni}.

  We adapt the notation in \cite{BVUni}: two special representations
  $\sigma \LRleq \sigma'$ if and only if $\cO_{\sigma}\supseteq \cO_{\sigma'}$
  where $\cO_{\sigma}:=\Spr(\sigma)$. The generic degree of $\sigma$ is denoted
  by $a(\sigma)$. Note that the ordering of double cells/special representation
  is the same as the closure relation on special nilpotent orbits, see
  \cite{BVUni}*{Prop 3.23}.

  Note that induction maps left cone representation to a left cone
  representation \cite{BVUni}*{Prop~4.14~(a)}. Therefore
  $\Ind_{W_{\mu}}^{W_{[\mu]}}\sgn$ is a left cone representation.
  $J_{W_{\mu}}^{W_{[\mu]}}\sgn$ is a left cell (since $J$-induction preserves
  left cell \cite{BVUni}*{Prop~4.14~(b)}), it consists of the constituents in
  the induced representation with the minimal generic degree (by the definition
  of $J$-induction), it is also the set of constituents in
  $\Ind_{W_{\mu}}^{W_{[\mu]}}\sgn$ sit in the same $\approxLR$ equivalence class
  (a unique double cell $\cD$).


  Recall that tensoring with sign (or rather twisting $w_{0}$) is an order
  reversing bijection of left cells in $\WLamck$ and induces a $LR$-order
  reversing bijection on $\Irr(\Wlamck)$, see \cite{BV2}*{Prop.~2.25}. Therefore
  $\Ind_{\Wlamck}^{\WLamck} 1 = \left(\Ind_{\Wlamck}^{\WLamck} \sgn\right)\otimes \sgn$
  has a set of constituents which is maximal under the $LR$-order, in particular
  the generic degree takes maximal value on these representations.

  Hence we get the conclusion. }


\subsection{Compare blocks}
Now we compare different blocks. Without of loss of generality, we assume
$\lambda$ is in the anti-dominant cone of $\WT{\fnn}$, i.e
$\inn{\lambda}{\ckalpha}<0$ for all $\alpha\in \WT{\fnn}$.

Let $k=\sharp(W/\WLam)$ and
\[
  % \Set{r_1,\cdots, r_{k}} \Set{r|\text{the length of $r$ with respect to
  % simple roots in $-\WT{\fnn}$ is minimal in $r\Wlam$} }
  \Set{r_1,\cdots, r_{k}} := \Set{r|l(r) \text{ is minimal among elements in
    }r\WLam }
\]
% \[
%   \set{r_{i}\Wlam|i =1, 2, \cdots, k}
% \] be the list of right cosets of $W/\Wlam$.
be the set of distinguished representatives of the right cosets of $\Wlam$ where
$l(r)$ denote the length function with respect to the simple roots in
$-\WT{\fnn}$. In other words, $r_{i}$ is the unique element in the coset
$r_{i}\Wlam$ such that $r_{i}\lambda$ is anti-dominant with respect to
$\WT{\fnn}$, i.e. $R^{+}_{[r_{i}\lambda]}\subseteq -\WT{\fnn}$.
% We choose $r_{i}$ to be the unique element in the coset $r_{i}\Wlam$ such that
% $r_{i}\lambda$ is anti-dominant, i.e.
% $R^{+}_{[r_{i}\lambda]}\subseteq -\WT{\fnn}$.

% Let $S_{[\lambda]}$ be the set of simple roots in $R^{+}_{[\lambda]}$. Apply
% Soergel's theorem \cite{H}*{13.13}, we have Then
% \begin{itemize}
%   \item The map $L(w\lambda) \mapsto L(r_{i} w\lambda)$ with $w$ running over
          %      $w\in W_{[\lambda]}$ induces an equivalence of category from $\cO_{W\cdot \lambda, [\lambda]}(\fgg,\fhh,\fnn)\cap$ to $\cO_{W\cdot\lambda, [r_{i}\lambda]}$ by Soegel's theorem \cite{H}*{13.13}.
          %     \item $w\mapsto r_{i} w r_{i}^{-1}$ induces isomorphism
          %     $W_{[\lambda]}\rightarrow W_{[r_{i}\lambda]}$ and preserves the
          %     cell structures.
          %     \item The following $\Wlam$-module isomorphism
          %           \[
          %           \begin{tikzcd}
          %             & \bC[W_{}]& \\
          %           \end{tikzcd}
          %           \]
          %   \end{itemize}

Now the map $L(w\lambda) \mapsto L(r_{i} w\lambda)$ with $w$ running over
$w\in W_{[\lambda]}$ induces an equivalence of category from
$\cO_{W\cdot \lambda, [\lambda]}(\fgg,\fhh,\fnn)\cap$ to
$\cO_{W\cdot\lambda, [r_{i}\lambda]}$ by Soegel's theorem \cite{H}*{13.13}. In
particular $w\mapsto r_{i} w r_{i}^{-1}$ induces isomorphism
$W_{[\lambda]}\rightarrow W_{[r_{i}\lambda]}$ and preserves the cell structures.
In other words, the following $\Wlam$-module isomorphism
\[
  \begin{tikzcd}
    & \bC[W_{}]\ar[dl, "w\mapsto M_{w}"'] \ar[dr,"w\mapsto M_{r_{i}w}"]& \\
    \Coh_{[\lambda]}(\fgg,\fhh,\fnn;[\lambda-\rho])\ar[rr] & & \Coh_{[\lambda]}(\fgg,\fhh,\fnn;[r_{i}\lambda-\rho])
  \end{tikzcd}
\]
maps cell representations to cell representations.

\trivial{ Let
  $C = \set{x\in \fhh| \inn{x}{\alpha} >0 \ \forall \alpha\in \WT{\fnn}}$ and
  $D_{\lambda} = \set{x\in \fhh| -\inn{x}{\beta} >0 \ \forall \beta\in R^+_{[\lambda]}}$.
  $C$ and $D_{\lambda}$ are fundamental domains of $\fhh$ under $W$ and
  $W_{[\lambda]}$-actions. Clearly, $D_{w\lambda} = w D_{\lambda}$. The
  condition that $R^{+}_{[\mu]}\subset - \WT{\fnn}$ is equivalent to
  $D_{\mu}\supset C$.

  Now it is clear $D_{\lambda}$ is the union of $r_{i}^{-1} C$ when $r_{i}$
  running over the preferred coset representatives of $W/W_{[\lambda]}$. }

Recall the definition of clan of the primitive ideals. By the comparing the
definition of Goldie rank polynomials, we see that $I(r_{i}w)$ and $I(r_{j}w')$
in the same clan if and only if $w\approxLR w'$. In other word, the clan is only
depends on the $W_{[\lambda]}$-type of the double cell containing $L(w\lambda)$
where $w\in W$.

\medskip


The above discussion yields
the following.
\begin{lem}\label{lem:C.S}
  Fix a $\Gad$-invariant subset $\sfS$ in the nilpotent cone of $\fgg$.

  Then, as an $W_{[\lambda]}$-module
  \[
    \begin{split}
      \Coh_{[\lambda],\sfS}(\fgg,\fhh,\fnn) &= \bigoplus_{i=1}^{k}
      \Coh_{[\lambda], \sfS}(\fgg,\fhh,\fnn;[r_{i}\lambda-\rho])\\
      & \cong \bigoplus_{i=1}^{k} \sum_{\sigma\in \Csp_{\sfS}} \VLR_{\sigma}\\
      & \cong \bigoplus_{i=1}^{k} \bigoplus_{\sigma\in \sfC_{\sfS}} (\dim \sigma) \sigma
    \end{split}
  \]
\end{lem}

As a baby case of the counting theorem for special unipotent representation of
real reductive groups, we have the following counting theorem in the category
$\cO$.

We fix an regular element $\lambda\in [\mu]$ such that $\mu$ is dominant with
respect to $R^{+}_{[\lambda]}$. Let $S_{[\lambda]}$ be the set of simple roots
in $R^{+}_{[\lambda]}$. Let $\sfS_{\mu}$ be the subset of simple roots in
$R^{+}_{[\lambda]}$ orthogonal to $\mu$. Observe that $W_{\mu}$ is always a
parabolic subgroup attached to $\sfS_{\mu}$ in $W_{[\lambda]} = W_{[\mu]}$. Let
$\sfD_{\sfS_{\mu}}$ be the set of distinguished right coset representatives of
$W_{[\lambda]}/W_{\mu}$. \trivial[]{ $r\in \sfD_{\sfS_{\mu}}$ is the element
  with minimal lenght in $rW_{\mu}$. Recall that
  $\tau(w) = \set{\alpha\in S_{[\lambda]}|w\alpha\notin R^{+}_{[\lambda]} }$
  Note that $\tau(w)\cap R_{\mu}\neq \emptyset$ is equivalent to require that
  $w \sfS_{\mu}\subseteq R^{+}_{[\lambda]}$, i.e. $w$ is a minimal length
  element. See for example, Carter, Simple groups of Lie type, Theorem~2.5.8. }

\begin{thm}
  Let $\cO$ be an nilpotent orbit in $\fgg$ and $\mu\in \ahh^{*}$. Let
  $\Pi_{W\cdot \mu, \cO}$ be the set of irreducible highest weight modules $\pi$
  such that $\AVC(\pi) = \bcO$. Let
  \begin{equation}\label{eq:DC.O.mu}
    \begin{split}
      \Dsp_{\cO,\mu} &:= \Set{\sigma\in \Irrsp(W_{[\mu]})|\Spr(j_{W_{[\mu]}}^{W}\sigma) = \cO}\quad \text{
        and }\\
      \cD_{\cO,\mu} &= \bigcup_{\sigma\in \Dsp_{\cO,\mu}} \LRC_{\sigma}.
    \end{split}
  \end{equation}
  % $ be the set of special representations such that
  Then
  \[
    \abs{\Pi_{W\cdot\mu,\cO}} = \abs{W/W_{[\mu]}}\cdot \sum_{\sigma\in \cD_{\cO,\mu}} \left(\dim \sigma \cdot [1_{W_{\mu}}:\sigma]\right).
  \]


  Let
  \[
    \CLR_{\sigma,\mu} = \CLR_{\sigma}\cap \sfD_{\sfS_{\mu}}
  \]
  and $\CLR_{\cO,\mu} = \bigcup_{\sigma\in \Dsp_{\cO},\mu} \CLR_{\sigma,\mu}$. Then
  \[
    \Pi_{W\cdot\mu,\cO} = \Set{L(r_{i}w)|w\in \CLR_{\cO,\mu} \text{ and
      } i=1,2,\cdots,k} .
  \]
  Here $\VLR_{\sigma}$ is understood as a submodule of $\bC[\Wlam]$.
\end{thm}



For each $\mu\in \fhh^{*}$, let $\cI_{\mu}$ be the maximal primitive ideal
having infinitesimal character $\mu$. Let $\Pi_{W\cdot \mu}$ be the set of all
irreducible highest weight modules whose annihilator ideal are $\cI_{\mu}$. Then
\[
  \Pi_{W\cdot \mu} = \Pi_{W\cdot \mu, \cO_{\mu}}
\]
where $\cO_{\mu}$ is given by \eqref{eq:O.mu}.


Combine the above theorem with \Cref{lem:lcell.BV}, we have the following
counting theorem.
\begin{thm}
  Retain the notation in
  \Cref{lem:LC.mu}.
  \[ \abs{\Pi_{W\cdot \mu}} = \abs{W/W_{[\mu]}}\cdot \dim \LC_{\mu}.
  \]
  Moreover,
  \[
    \Pi_{W\cdot\mu} = \Set{L(r_{i}w)|w\in \CLR_{\sigma_{\mu}}\cap \sfD_{\sfS_{\mu}} \text{
        and } i=1,2,\cdots,k}
  \]
  \qed
\end{thm}



\subsection{A variation}
In this section, let
$(\fgg, H,\fnn)$ be a triple that
\begin{itemize}
  \item $\fgg$ is a complex reductive Lie algebra,
  \item $H$ is a Lie group such that
        $\fhh_{0}:= \Lie(H)$ is a real form of
       a Cartan subalgebra  $\fhh$ of $\fgg$,
  \item $\fnn$ is a maximal nilpotent subalgebra
        stable under the $\fhh$-action.
\end{itemize}
Let $\cO'(\fgg, H,\fnn)$ be the category of $(\fgg, H)$-module such that
$M\in \cO'(\fgg,H,\fnn)$
if and only if
\begin{itemize}
  \item $M$ is finitely generated as $\cU(\fgg)$-module,
  \item $\fnn$ acts on $M$ locally nilpotently, and
  \item $M$ decomposes in to a direct sum of finite dimension $H$-modules.
\end{itemize}
Let $\Grt(\fgg,H,\fnn)$ be the Grothendieck group of $\cO(\fgg,H,\fnn)$.

We write $H_{0}$ for the connected component of $H$ which is abelian.
Since $H_{0} = \exp(\fhh_{0})$ is central in $H$, for each $\phi\in \Irr(H)$
$\phi|_{H_{0}}$ is a multiple of character.
Hence taking the derivative yields a
well defined map
\[
\rdd \colon \Irr(H)\longrightarrow \fhh^{*}
\]
sending $\phi$ to $\dphi$.


Since $\rdd$ restricted on the lattice
\[
\tQ:=\Set{\phi\in \Irr(H)|\phi \text{
    occurs in } S(\fgg)}
\]
 is a bijection onto the root lattice $Q$ \cite{Vg}*{0.4.6}, we identify the root lattice $Q$ with the $\tQ$ in $\Irr(H)$.


Now assume $\phi\in \Irr(H)$ and let
\[
  [\phi] := \Set{\phi+\alpha| \alpha\in \tQ}
\]
and
$\cO'(\fgg,H,\fnn;[\phi])$ be the subcategory of $\cO'(\fgg,H,\fnn)$
consists of modules whose $H$ irreducible components are contained in $[\phi]$.
Define $\Coh_{[\lambda]}(\fgg,H,\fnn;[\phi])$ to be the space of coherent
families taking value in $\cO'(\fgg,H,\fnn;[\phi])$ and
$\Coh_{[\lambda]}(\fgg,H,\fnn;[\phi])$ to be its subspace whose complex
associated variety is contained in $\sfS$ for a $\Ad(\fgg)$-invariant closed subset $\sfS$ in the nilpotent cone of
$\fgg$.


We have the following lemma.
\begin{lem}
  Let $\phi\in \Irr(H)$ and fix a $\lambda\in \fhh^{*}$ such that
  $[\rdd \phi + \rho ]\cap W\cdot \lambda\neq \emptyset$.
  Then the forgetful functor
  \[
    \cF\colon \cO'(\fgg,H,\fnn)\longrightarrow \cO'(\fgg,\fhh,\fnn)
  \]
  induces a $W_{[\lambda]}$-module isomorphism
  \[
    \cF\colon \Coh_{[\lambda],\sfS}(\fgg,H,\fnn;[\phi])\longrightarrow
    \Coh_{[\lambda],\sfS}(\fgg,\fhh,\fnn;[\rdd\phi])
  \]
\end{lem}
\begin{proof}
  When $\sfS$ is the whole nilpotent cone, the isomorphism is
  given by identifying both sides with $\bC[W_{[\lambda]}]$ via Verma modules
  such that
  \[
  \wtM_{1}(\rdd \phi+\rho):=\cU(\fgg)\otimes_{\cU(\fhh\oplus\fnn),H}\phi
  \mapsto (\dim\phi)\cdot M_{1}(\rdd\phi+\rho):= \cU(\fgg)\otimes_{\cU(\fhh\oplus\fnn)}\rdd\phi.
  \]
  For $\phi'\in \Irr(H)$, $M(\phi'+\rho)$ has a unique irreducible quotient
  $L(\phi')$ by the same argument of the same argument for the highest weight module.   Now we have, $L(\phi'+\rho) = \dim(\phi')\cdot L(\rdd\phi'+\rho)$.
  \trivial{
    These claims should be also much more clear from the D-module point of view.
    The middle extension functor only see the $\cD_{\lambda}$-module structure
    and keeps the $T$-module structure automatically. Here $T$ is the maximal
    compact subgroup of $H$.
  }
  Since the associated variety only depends on the $\cU(\fgg)$-module structure,
  the rest part of the lemma follows.   {\color{red} Check!!}
\end{proof}

\trivial{
  Note that $H_{0}$ is abelian and $H = H_{0}$.
  By the assumption of Harish-Chandra class,
  $H = H_{0}\times H/H_{0}$. Here $H/H_{0}$ is a finite group maybe non-abelian.
}

The above lemma have the following immediate consequence.
\begin{cor}
  Retain the notation in \Cref{lem:C.S}. Then, for $\sigma\in \Irr(W_{[\lambda]})$
  \[
[\sigma:\Coh_{[\lambda],\sfS}(\fgg,H,\fnn)] \neq 0 \Leftrightarrow
  \sigma\in \sfC_{\sfS}.
  \]   \qed
\end{cor}



\section{Counting Harish-Chandra modules with support condition}

In this section, let $G$ be a real reductive group in the Harish-Chandra class.
Here $G$ could be a nonlinear group.
We retain the notation in \Cref{eg:Coh.HC}.
We recall the argument before \cite{Mc}*{Theorem~1}.

We fix a Cartan involution $\Phi$ on $G$ and let $K = G^{\Phi}$
be the maximal compact subgroup of $G$.

% \begin{thm}[Barbasch-Vogan]\label{thm:count}
%   For a complex nilpotent orbit $\cO$,
%   define
%   \[
%     S_{\cO} = \left\{\sigma \in \widehat{W_{[\mu]}}|
%       \Spr(j_{W_{[\mu]}}^{W} \sigma_{s}) = \cO
%     \right\}
%   \]
%   where $\sigma_{s}$ is the special representation in the double cell containing
%   $\sigma$.

%   Let $\Pi_{\cO,\mu}(G)$ denote the set of irreducible admissible $G$-module with
%   complex associated variety $\overline{\cO}$.
%   % Let $W_{\mu}$ be the stabilizer of $\mu$ and $W_{[\mu]}$ be the stabilizer of
%   % the lattice
%   Then
%   \[
%     \# \Pi_{\cO,\mu}(G) =
%     \sum_{\sigma\in S_{\cO}} [\sigma: \mathrm{Coh}_{[\mu]}(G)] \cdot
%     [1_{W_{\mu}}, \sigma|_{W_{\mu}}].
%   \]
%   % Here $[\sgima : \ ]$ denote the multiplicity of $\sigma$
%   % and  $W_{\mu}$ is the stabilizer of $\mu$.
% \end{thm}

\subsection{An embedding of coherent families of Harish-Chandra modules into
  that of category $\cO$}
In this section, we recall a result in \cite{Cas}. Let $\fbb = \fhh\oplus \fnn$
be a Borel subalgebra in $\fgg$ with the nilradical $\fnn$ and $\fhh$ a Cartan
subalgebra in $\fbb$. For a subalgebra $\fuu$ of $\fnn$ and $q\in \bN$, Casian
defined the localization functors $\gamma_{\fuu}^{q}$ on the category of
$\fuu$-module. By \cite{Cas}*{Proposition~4.8}, $\gamma_{\fuu}^{q}$ can be
defined as the right derived functor of the functor $\gamma_{\fuu}^{0}$ which
sends a $\fuu$-module $M$ to
\[
  \gamma_{\fuu}^{0}(M):= \Set{v\in M| \fuu^{k} v = 0 \text{ for some positive
      integer $k$}}.
\]
In particular, the $\fuu$-action on $\gamma_{\fuu}^{q}$ is locally nilpotent.

Suppose $M$ is a $\fgg$-module, we have (see \cite{Cas}*{Proposition~4.14})
\begin{equation}\label{eq:anngamma}
  \Ann M \subseteq \Ann (\gamma_{\fuu}^{q}(M)).
\end{equation}
Moreover, $\gamma_{\fuu}^{q}$ commutes with tensoring finite diemsional
representations of $\fgg$, i.e. for a finite dimensional $\fgg$-module $F$ there
is a natural isomorphism (c.f. \cite{Cas}*{Proposition~4.11})
\begin{equation}\label{eq:Fgamma}
  F\otimes \gamma_{\fuu}^{q}(M)\cong
  \gamma_{\fuu}^{q}(F\otimes M).
\end{equation}
If the $\fgg$-module have finite dimensional $\fnn$-cohomology, then
$\gamma_{\fnn}^{q}(M)$ is in the category $\cO'$. See
\cite{Cas}*{Proposition~4.9}.



Suppose $M$ is a $(\fgg,K)$-module. From the definition of $\gamma_{\fuu}^{q}$,
we can see that $\gamma_{\fuu}^{q}(M)$ is naturally a $(\fgg, K_{L})$-module
where $K_{L}$ denote the normalizer of $\fuu$ in $K$. Let $\fll$ be the
normailzer of $\fuu$ in $\fgg$. Then there is the a spectrum sequence of
$(\fll,K_{L})$-module convergent to $H^{p+q}(\fuu,M)$ (see
\cite{Cas}*{Proposition~4.4}):
\begin{equation}\label{eq:ugamma}
  H^{q}(\fuu,\gamma_{\fuu}^{p}(M)) \Rightarrow H^{p+q}(\fuu,M).
\end{equation}

For our application, we always assume $M$ is a $(\fgg,K)$-module, and take
$\fuu=\fnn$.

Let $H$ be a $\Phi$-stable Cartan subgroup of $G$, $T=H^{\Phi}$ be the
maximal compact subgroup of $H$ and $\fhh:=\Lie(H)_{\bC}$ is the corresponding
Cartan subalgebra in $\fgg$. We can view a finite dimensional $H$-module as a
$(\fhh,T)$-module and vice versa.

The localization functor $\gamma_{\fnn}^{q}$ is compatible with coherent
continuation.
\begin{lem}\label{lem:coh.gamma}
  Assume that each irreducible $(\fgg,K)$-module have finite dimensional
  $\fnn$-cohomology. For each $q\in \bN$, we have the following map between
  $\Wlam$-modules:
  \[
    \begin{array}{ccc}
      \gamma_{\fnn}^{q}\colon \Coh_{[\lambda]}(\fgg,K)
      &\longrightarrow
      & \Coh_{[\lambda]}(\fgg,H,\fnn) \\
      \Phi & \mapsto & \gamma_{\fnn}^{q}\circ \Phi.
    \end{array}
  \]
\end{lem}
\begin{proof}
  This is a consequence of \eqref{eq:anngamma} and \eqref{eq:Fgamma}. \trivial[]{
    \[
      \begin{split}
        F\otimes \gamma_{\fnn}^q\Phi(\mu) & =
        \gamma_{\fnn}^{q}(F\otimes \Phi(\mu))\\
        &= \gamma_{\fnn}^{q}(\sum_{\beta\in \WT{F}} \Phi(\mu+\beta))\\
        &= \sum_{\beta\in \WT{F}}\gamma_{\fnn}^{q}( \Phi(\mu+\beta))\\
      \end{split}
    \]
  }
\end{proof}

We fix a positive system of real roots $\Delta^{+}_{\bR}$ and a Borel subalgebra
$\fbb=\fhh\oplus \fnn$ such that $\Delta^{+}_{\bR}\subset \WT{\fnn}$. We let
$e^{\alpha}\in \Irr(H)$ be the $H$-character on the $\alpha$-root space.

For a finite length $(\fgg,K)$-module $M$, we view its global character
$\Phi_{G}(M)$ as a analytic function defined on the set $\Greg$ of regular
semisimple elements on $G$.


The following theorem is crucial.
\begin{thm}[\cite{Cas}*{Theorem~3.1}]\label{thm:gamma.HC}
  % Let $\bfnn$ be the maximal nilptent Lie subalgebra of $\fgg$ with spanned by
  % roots in $\Delta^{+}$.
  Let $M$ be a finite length $(\fgg,K)$-module. The following statements hold.
  \begin{enumT}
    \item The Lie algebra cohomology $H^{q}(\fnn,M)$ is finite dimensional for
    each $q\in \bN$. In particular, $\gamma_{\fnn}^{q}(M)$ is in the category
    $\cO'(\fgg,H,\fnn)$ for eah $q\in \bN$.
    \item
    % $\Ann M \subseteq \Ann (\gamma_{\fnn}^{q}M)$.
    % \item
    Let
    \[
      \Hnreg:= \Set{h\in H|
        \begin{array}{l}
          \text{$h$ is regular semisimple}\\
          \abs{e^{\alpha}(h)}<1 \text{ for each real root }
          \alpha\in \Delta^{+}_{\bR}
        \end{array}
      }
    \]
    Then
    \begin{equation}\label{eq:char}
      \begin{split}
        \Phi_{G}(M)|_{\Hnreg} &= \frac{\sum_{q\in \bN} (-1)^{q}\Phi_{H} \left(H^{q}(\fnn,M)\right)}
        {\prod_{\alpha\in \WT{\fnn}}(1- e^{\alpha})}\\
        &= \frac{\sum_{p,q\in \bN} (-1)^{p+q}\Phi_{H} \left(H^{q}(\fnn,\gamma_{\fnn}^{p}(M))\right)}
        {\prod_{\alpha\in \WT{\fnn}}(1- e^{\alpha})}\\
      \end{split}
    \end{equation}
  \end{enumT}
  \qed
\end{thm}
\begin{proof}
  The theorem is a recollection of Casian's results in loc. cit. The last
  equality in \eqref{eq:char} follows from \eqref{eq:ugamma}. The last
  expression in \eqref{eq:char} is also a finite sum, since only finite may
  terms of $\gamma_{\fnn}^{p}(M)$ are non-zero and they are in the category
  $\cO'$.
\end{proof}


Retain the notation in \Cref{thm:gamma.HC}, we write
\[
  \gamma_{\fnn}:= \sum_{q\in \bN} (-1)^{q} \gamma^{q}_{\fnn}\colon \Grt(\fgg,K) \longrightarrow \Grt(\fgg,H,\fnn)
\]

% \[
%   \gamma_{\fnn}:= \sum_{q\in \bN} (-1)^{q} \gamma^{q}_{\fnn}\colon \Coh_{[\lambda]}(\fgg,K) \longrightarrow \Coh_{[\lambda]}(\fgg,H,\fnn)
% \]

\begin{cor}[c.f. {\cite{Mc}}]\label{cor:HC.embed}
  % [\cite{Cas}*{Theorem~3.1}]
  Let $H_{1}, H_{2}, \cdots, H_{s}$ form a set of representatives of the
  conjugacy class of $\Phi$-stable Cartan subgroup of $G$. Fix maximal
  nilpotent Lie subalgebra $\fnn_{i}$ for each $H_{i}$ as in
  \Cref{thm:gamma.HC}. Then
  \[
    \begin{array}{cccc}
      \gamma:=\oplus_{i} \gamma_{\fnn_{i}}: &\Grt(\fgg,K)
      &\longrightarrow & \bigoplus_{i=1}^{s} \Grt(\fgg,H_{i},\fnn_{i})\\
    \end{array}
  \]
  is an embedding of abelian groups.
\end{cor}
\begin{proof}
  Let $\Hireg$ be the set of regular semisimple elements in $H_{i}$. By the
  results of Harish-Chandra, taking the character of the elements induces an
  embedding of $\Grt(\fgg,K)$ into the space of real analytic functions on
  $\bigsqcup_{i} \Hireg$. Since $\Hnireg$ is open in $\Hireg$ and meets all the
  connected components of $\Hireg$, any real analytic function on
  $\bigsqcup_{i} \Hireg$ is determined by its restriction on
  $\bigsqcup_{i} \Hnireg$. Now \eqref{eq:char} implies that the global character
  of any $M\in \Grt(\fgg,K)$ can be computed from the image $\gamma(M)$.
  % $\Phi M$ of an element $M\in \Grt_{\chi}(\fgg,K)$ is completely determined
  % by the formal character $\mathrm{ch}(\gamma(M))$.
\end{proof}

\begin{cor}\label{cor:coh.HC}
  Retain the notation in \Cref{cor:HC.embed}.
  Let $\sfS$ be a $\Gcad$-invariant closed subset in the nilpotent cone of
  $\fgg$ and $\lambda\in \fhh^{*}$.
  Then $\gamma$ induces an embedding of $\Wlam$-module.
  \[
    \begin{array}{cccc}
      \gamma:=\oplus_{i} \gamma_{\fnn_{i}}: &\Coh_{[\lambda],\sfS}(\fgg,K)
      &\longrightarrow & \bigoplus_{i=1}^{s} \Coh_{[\lambda],\sfS}(\fgg,H_{i},\fnn_{i})\\
      &\Phi& \mapsto &\gamma\circ \Phi
    \end{array}
  \]
  In particular, $[\sigma:\Coh_{[\lambda],\sfS}(\fgg,K)]\neq 0$ only if
  $\sigma\in \sfC_{\sfS}$.
\end{cor}
\begin{proof}
  The maps is well-defined by \eqref{eq:anngamma}. It is an embedding by
  \Cref{cor:HC.embed}.
\end{proof}

\medskip

Now we get the following theorem on the upper bound of small representations.
\begin{thm}
  Let $\mu\in \fhh^{*}$. Let $\Pi_{W\cdot \mu}(G)$ be the set of irreducible
  $G$-modules annihilated by the maximal primitive ideal $\cI_{\mu}$ of
  infinitesimal character $\mu$. Let $\LC_{\mu}$ be the left cell of
  $\Irr(W_{[\mu]})$ defined in \eqref{eq:LC.mu} and $\cO_{\mu}$ be the nilpotent
  orbit defined by \eqref{eq:O.mu}. Then
  \[
    \begin{split}
      \abs{\Pi_{W\cdot \mu}(G)} &= \sum_{\sigma\in \LC_{\mu}}
      [\sigma:\Coh_{[\mu],\bcO_{\mu}}(G)]\\
      &\leq \sum_{\sigma\in \LC_{\mu}}
      [\sigma:\Coh_{[\mu]}(G)]\\
    \end{split}
  \]
\end{thm}
\begin{proof}
  The equality follows from \Cref{lem:coh.count}, \Cref{cor:coh.HC} and
  \Cref{lem:LC.mu}. The inequality is also clear since
  $\Coh_{[\mu],\bcO_{\mu}}(G)$ is a submodule of $\Coh_{[\mu]}(G)$.
\end{proof}

% Then the localization functor induces a homomorphism
% \[
%   \begin{array}{cccc}
%       \gamma_{\fnn}: &\Grt_{\chi,\cZ}(\fgg,K) &\longrightarrow & \Grt_{\chi,\cZ}(\cO_{\whH})\\
%       & M &\mapsto & \sum_{q}  (-1)^{q} \gamma^{q}_{\fnn} M
%     \end{array}
%   \]
%   Fix an infinitesimal character $\chi$, and a close $G$-invariant set
%   $\cZ\in \cN_{\fgg}$.


% \def\VHC{\sV^{\mathrm{HC}}}
% \begin{cor}
%   Fix an irreducible $(\fgg,K)$-module $\pi$ with infinitesimal character
%   $\chi_{\lambda}$. Let $\VHC(\pi)$ be the Harish-Chandra cell representation
%   containing $\pi$ and $\cD$ be the double cell in $\widehat{W_{[\lambda]}}$
%   containing the special representation $\sigma(\pi)$attached to $\Ann(\pi)$.
%   Then $[\sigma, \VHC(\pi)]\neq 0$ only if $\sigma \in \cD$. Moreover,
%   $\sigma(\pi)$ always occures in $\VHC(\pi)$
% \end{cor}
% \begin{proof}
%   The occurrence of $\sigma(\pi)$ is a result of King.

%   Note that we have an embedding
%   \[
%     \gamma \colon \Grt_{\chi}(\fgg,K)\longrightarrow \bigoplus_{i}\Grt(\fgg,\fbb,\lambda).
%   \]
%   where the left hand sides is identified with a finite copies of
%   $\bC[W_{[\lambda]}]$.

%   Since $\Ann(\pi)\subseteq \Ann (\gamma_{\fnn}^{q}(\pi))$, we conclude that
%   $[\sigma, \VHC(\pi)]\neq 0$ implies that $\sigma(\pi)\leqLR \sigma$.

%   By the Vogan duality, $\cD\otimes \sgn$ is also a Harish-Chandra cell. So we
%   have $\sigma(\pi)\otimes \sgn \leqLR \sigma\otimes \sgn$.

%   Therefore, $\sigma(\pi)\approxLR$
% \end{proof}

\begin{remark}\label{r46}

  Let $\lambda$ be a regular element in $[\mu]$. Assume that for each
  Harish-Chandra cell representation $\sV^{HC}$ at the infinitesimal character
  $\chi_{\lambda}$, the $W_{[\mu]}$-representations occur in $\sV^{HC}$ are all
  contained in a unique double cell. Then one can prove that the inequality in
  the above theorem is sharp (c.f. \cite{BV.W}). However, we do not know such
  kind of claim holds in a general.
\end{remark}


Another interesting consequence of \Cref{cor:coh.HC} is that we can determine
the complex associated variety attached to a Harish-Chandra cell by the
character of the cell representation.

\begin{lem}\label{lem:AV.HC}
  Let $\sC$ be an Harish-Chandra cell at a regular infinitesimal character
  $\lambda$ and $\sV$ be the corresponding cell representation. \trivial[]{ I
    guess the assumption of ``regular'' is unnecessary. } Then there is a unique
  special representation $\sigma\in \Irrsp(W_{[\lambda]})$ such that $\sigma$
  has minimal generic degree in $\sV$ and every irreducible modules $\pi$ in
  $\sC$ has complex associated variety
  \[
    \AVC(\pi) = \Spr(\Ind_{W_{[\lambda]}}^{W}\sigma).
  \]
  Moreover, $\sigma$ is precisely the unique irreducible character
  with minimal fake degree occurring in $\sV$.
\end{lem}
\begin{proof}
  Let $\pi\in \sC$ and the primitive ideal $\Ann(\pi)$ is attached to the
  special representation $\sigma \in \Irr(W_{[\lambda]})$.
  By \Cref{cor:coh.HC}, every irreducible character $\sigma'$ occur in
  $\sV$ satisfies $\sigma \leqLR \sigma'$.
  On the other hand, $\sigma$ do occurs in $\sV$ by Barbasch-Vogan and King's
  result, see \cite{Cas}*{Remark~3.2}.
  The rest part of the lemma is clear.
\end{proof}


\subsection{Coherent continuation representation of Harish-Chandra modules}

In this section, we provides the proof of \Cref{thm:cohHC}, which is a formula of coherent continuation representation due to Barbasch-Vogan.
Note that the formula holds for non-linear groups as well as the non-integral
infinitesimal character!

\medskip

First recall the definition of regular characters \cite{Vg}*{Definition~6.6.1}.

A regular character of $G$ is a tuple a tuple $\gamma := (H,\Gamma,\bargamma)$
such that
\begin{itemize}
  \item $H$ is a $\Phi$-stable Cartan subgroup of $G$,
  \item $\Gamma$ is a continuous character of $H$,
  \item if $\alpha$ is imaginary, then $\inn{\bargamma}{\ckalpha}$ is real and
        non-zero,
  \item the differential of $\Gamma$ is
        \[
        \bargamma+ \rho_{i} - 2\rho_{ic}
        \]
        where %$\rho_{i}$ is the half sum of imaginary roots $\$ such that $\inn{\bargamma}{\ckalpha}$
        \[
        \rho_{i} =\half\sum_{\substack{\alpha \text{
        imaginary}\\ \inn{\bargamma}{\ckalpha}>0}} \alpha \quad\text{and}\quad \rho_{ic} =\half\sum_{\substack{\alpha \text{
        compact imaginary}\\ \inn{\bargamma}{\ckalpha}>0}} \alpha.
        \]
\end{itemize}
We say $\gamma$ is non-singular if $\bargamma$ is regular in $\fhh^*$. The group $K$ acts on the set of regular characters of $G$ we let $[\gamma]$ to denote the $K$-conjugacy class of regular character containing $\gamma$.

Let $\cR(G)$ denote the set of regular characters of $G$ and $\cR_{\lambda}(G)$ be the subset of $\cR(G)$ consists of regular characters with infinitesimal character $\lambda$. Let $\cP(G):=\cR(G)/K$ and $\cP_{\lambda}(G):=\cR_{\lambda}(G)/K$ be the corresponding sets of $K$-conjugacy classes.
% of regular characters and $\cP_\lambda(G)$ be the subset with infinitesimal character $\lambda$. By abuse of notation, we identify $\gamma$ with its conjugacy class.

For each regular character $\gamma$, we attach a standard representation
$\pi_\gamma$ with infinitesimal character $\bargamma$ and let $\barpi_\gamma$ be
the maximal completely reducible submodule of $\pi_\gamma$
\cite{Vg}*{6.5.2,6.5.11,6.6.3}. Then $\barpi_\gamma$ and the image of
$\pi_\gamma$ in the Grothendieck group of $(\fgg,K)$-module only depends on the
$K$-conjugation class of $\gamma$. This justifies the consideration of $\cP(G)$.


Now we assume that $\lambda\in \fhh^*$ is a regular element. Then
$\barpi_\gamma$ is the unique irreducible submodule in $\pi_\gamma$.
% Let $\Pi_{\lambda}(G)$ be the set of irreducible $(\fgg,K)$-modules with
% infinitesimal character $\lambda$, then
The following map is bijective (Langlands classification)
\[
  \begin{array}{ccc}
    \cP_{\lambda}(G) & \longrightarrow & \Irr_{\lambda}(G)\\
    {[\gamma]} & \mapsto & \barpi_{\gamma}
  \end{array}
\]
and $\Set{\pi_\gamma|\gamma\in \cP_{\lambda}(G)}$ forms a basis of
$\Grt_{\lambda}(G)$. As a consequence, we have an isomorphism of vector spaces
\[
  \begin{array}{ccc}
    \bC[\cP_{\lambda}(G)] & \longrightarrow & \Coh_{[\lambda]}(G)\\
    {[\gamma]} & \mapsto & \Phi_{\gamma}
  \end{array}
\]
where $\Phi_{\gamma}$ is the unique coherent family such that
$\Phi_{\gamma}(\lambda) = \pi_{\gamma}$.
In the following, we identify the two sides implicitly.

\medskip

\def\Wa{W^{a}}
\def\WiR{W_{i\bR}}
\def\fhhiR{\fhh_{i\bR}}
\def\WR{W_{\bR}}
\def\WC{W_{\bC}}
\def\lama{\lambda^{a}}
\def\Wlama{W^a_{[\lambda]}}
For any
regular element $\lambda\in \fhh^{*}$, let $\lambda^{a}$ be the dominant element
in the abstract Cartan $\fhh^{a}$ corresponding to $\lambda$ and
\[
  i_{\lambda}\colon W(\fgg,\fhh)\longrightarrow \Wa
\]
be the identification of $W(\fgg,\fhh)$ with the abstract Weyl group $\Wa$.

From now on we fix a regular element $\lambda \in \fhh^{*}$.

Let $\Wlama:= i_{\lambda}(W_{[\lambda]})$ be the abstract integral Weyl group.
We identify $W_{[\lambda]}$ with $\Wlama$ implicitly.

In the following, we
assume $\gamma=(H,\Gamma,\bargamma)\in \cR_{\lambda}(G)$. On one hand, a element
$w$ in the real Weyl group $W(G,H)$ acts on $\gamma$ by conjugation and gives a
new regular character denoted by $w\cdot \gamma$. \trivial{ Note that
  $W(G,H) = W(K_{\bC},H_{\bC}):= N_{K_{\bC}}(H_{\bC})/K_{\bC}\cap H_{\bC}$ is a
  subgroup of $W(\fgg,\fhh)$. }

On the other hand, $W_{[\bargamma]}$ acts on $\gamma$ by cross action
$(w,\gamma)\mapsto w\times gamma$ \cite{Vg}*{8.3.1}. On define the abstract Weyl
group $\Wlama$ acts on $\cR_{\lambda}(G)$ by
\[
  w \crossa \gamma := \dot{w}^{-1}\times \gamma \text{ with
  } \dot{w}\in W_{[\lambda]}, w = i_{\bargamma}(\dot{w}),
\]
see \cite{V4}*{Definition~4.2}.

The cross action of $\Wlama$ descents to an action on $\cP_{\lambda}(G)$.
{\color{red} reference? Maybe \cite{V4}?} Let $\Wa_{[\gamma]}\subset \Wlama$ be the stabilizer of
$[\gamma]$ under cross action. Then
\[
  \Wa_{[\gamma]}:= i_{\bargamma}(W_{[\gamma]})
  \quad \text{ with }\quad
  W_{[\gamma]}:=\Set{w\in W(G,H)| w\times \gamma = w\cdot \gamma}.
\]
The $\Phi$-action on $\fhh$ induces an action on $W(\fgg,\fhh)$. Then we have
the following tower of groups
\[
  W_{[\gamma]} \subseteq W(G,H) \subseteq W(\fgg,\fhh)^{\Phi}
\]
and
\[
  W(\fgg,\fhh)^{\Phi}=(\WC)^{\Phi}\ltimes (\WiR\times \WR)
\]
where $\WC$, $\WiR$ and $\WR$ are Weyl groups of complex, compact and real
roots respectively (see \cite{V4}*{Proposition~3.12} and
\cite{AC}*{(12.1)-(12.5)}).


Define a quadratic character on $W(\fgg,\fhh)^{\Phi}$ by
\[
  \begin{array}{rccc}
    \sgn_{\fhh}\colon  W(\fgg,\fhh)^{\Phi} =
    & (\WC)^{\Phi}\ltimes (\WiR\times \WR)
    & \longrightarrow & \Set{\pm 1}\\
    & (w_{\bC},w_{i\bR},w_{\bR}) & \mapsto & \sgn_{\WiR}(w_{i\bR})
  \end{array}
\]
where $\sgn_{\WiR}$ denote the sign character of the imaginary Weyl group. Let
$\fhhiR^{*}$ be the span of imaginary roots, then
\[
  \sgn_{\fhh}(w) = \det(w|_{\fhhiR^{*}}).
\]
Let $\sgn_{[\gamma]}:=\sgn_{\fhh}|_{W_{[\gamma]}}$ be the restriction of
$\sgn_{\fhh}$ on the cross stabilizer $W_{[\gamma]}$.

The following result is a unpublished result of Barbasch-Vogan based on results
in \cite{Vg}*{Chapter 8}, a same argument works equally well with non-linear groups
in the Harish-Chandra class.

\begin{thm}[{c.f. \cite{BV.W}*{Proposition~2.4}}]
  \label{thm:cohHC}
  Suppose $\Pi_{\lambda}(G) = \bigsqcup_{i=1}^{k} \Wlama\crossa [\gamma_{i}]$
  where $\gamma_{i}=(H_{i},\Gamma_{i},\bargamma_{i})$ are representatives of the
  $\Wlama$-orbits of $\Pi_{\lambda}(G)$. Then
  \[
    \Coh_{[\lambda]}(G) \cong \bigoplus_{i=1}^{k}
    \Ind_{W_{[\gamma_{i}]}}^{\Wlama} \sgn_{[\gamma_{i}]}.
  \]
\end{thm}
\begin{proof}[Sketch of the proof]
  To avoid the confusion, we use $t(w)$ to denote the coherent continuation
  action.
  The action of simple roots in $R^{+}_{\lambda}$ on the basis $\Phi_{\gamma}$
  is given calculated in \cite{Vg}*{Chapter 8} and summarized in
  \cite{V4}*{Definition~14.4}. The calculation reduced to the representation
  theory of $\SL(2,\bR)$.
  For non-linear groups see \cite{RT3}*{Definition~9.4} and note that
  the formula is exactly the same as that of linear groups for integral simple
  roots.
  \trivial[]{
    Let $\cT_{\lambda_{1}}^{\lambda_{2}}$ be the translation functor from
    infinitesimal character $\lambda_{1}$ to $\lambda_{2}$.
    Fix a
    By abstract non-sense, one have \cite{Vg}*{Prop~7.2.22}
    \[
      \Phi_{\pi}(\lambda) + s_{\alpha}\cdot \Phi_{\pi}(\lambda)=
      \cT_{\lambda_{0}}^{\lambda}\cT_{\lambda}^{\lambda_{0}}(\pi)=:
      \phi_{\alpha} \psi_{\alpha} (\pi)
    \]
    where $\lambda_{0}$ is an element such that $\inn{\lambda_{0}}{\alpha}=0$
    and $\inn{\lambda_{0}}{\beta}>0$ for all
    $\alpha\neq \beta \in R^{+}_{[\lambda]}$.
    By lifting to the covering group, we can assume that
    $\lambda-\lambda_{0}$ is a weight of a finite dimensional representation
    of $G$.

    Now the computation reduces to compute the RHS
    $=\phi_{\alpha} \psi_{\alpha} (\pi)$ of the above equality. (Sometimes, we
    need \cite{Vg}*{Prop~8.3.18} for the explicit computation of cross action.)
    This is computed case by case (let $t(w)$ denote the coherent continuation
    action):
    \begin{itemize}
      \item For compact imaginary roots, by Hecht-Schmid's ``A proof of
      Blattner's conjecture'', where RHS $=0$.
      \[
        t(s_{\alpha}) \gamma = - \gamma = - s_{\alpha}\cross \gamma.
      \]
      \item For non-compact imaginary roots, reduce to $\SL(2,\bR)$
      \cite{Vg}*{8.4.5,8.4.6}.
      \[
        t(s_{\alpha}) \gamma = - s_{\alpha}\cross \gamma + R
      \]
      where $R = c^{\alpha}(\gamma)$ or
      $\gamma^{\alpha}_{+}+\gamma^{\alpha}_{-}$ is a combination of
      regular characters on the Cartan subgroup $H^{\alpha}$ (which has
      higher $\bR$-rank, in fact
      $\rank_{\bR}H^{\alpha} = \rank_{\bR} H +1$).
      \item For real roots we can use \cite{Vg}*{8.3.19} which is a consquence
      of the (cohomological induction) construction of the standard
      module. The result says if $w$ acts trivial on $\ftt=\fhh^{\Phi}$,
      then $t(w^{-1})\gamma = w\cross \gamma$. When $\alpha$ is a real
      root, then clearly $s_{\alpha}$ acts on $\ftt$ trivially (since
      $-\alpha(x) = \Phi(\alpha)(x) = \alpha(\Phi(x)) = \alpha(x) = 0
      \ \forall x\in \ftt$)
      and we have
      \[
        t(s_{\alpha}) \gamma = s_{\alpha}\cross \gamma.
      \]
      \item For complex root, use \cite{Vg}*{8.2.7} (whose proof relies on a
      long exact sequence in \cite{Vg}*{7.4.3(a)} which is also formal) we get:
      $\gamma + s_{\alpha}\cross \gamma = \phi_{\alpha} \psi_{\alpha} (\gamma)$
      (here $\gamma-n\alpha = s_{\alpha}\cross \gamma$ by the definition of
      cross action).
      In particular, we have
      \[
        t(s_{\alpha}) \gamma = s_{\alpha}\cross \gamma.
      \]
    \end{itemize}
 }

 Let
 \[
   \cP_{\lambda,r}(G):= \Set{[(H,\Gamma,\bargamma)]|
     \text{real rank of $H$ is $r$}}
 \]
 and $\cP_{\lambda,\geq r} = \bigsqcup_{l\geq r}\cP_{\lambda,l}$. Define
 \[
   \Coh_{[\lambda],\geq r}(G) :=
   \Span\Set{\Phi_{[\gamma]}|[\gamma]\in \cP_{\lambda,\geq r}}
 \]
 From the explicit formula of the coherent continuation actions on the standard
 modules, we have that, for $[\gamma]\in \cP_{\lambda,\geq r}(G)$,
 \[
   t(s_{\alpha})\; [\gamma]
   \equiv
   \begin{cases}
     - s_{\alpha}\crossa [\gamma] & \text{if $\alpha$ is imaginary,}\\
     \phantom{-} s_{\alpha}\crossa [\gamma] & \text{otherwise.}
   \end{cases}
 \]
 Now it is elementary to deduce that, as $\Wlama$-module,
 \[
   \frac{\Coh_{[\lambda],\geq r}(G)}{\Coh_{[\lambda],\geq r+1}(G)}
   \cong \bigoplus_{\Wlama \crossa [\gamma]}
   \Ind_{W_{[\gamma]}}^{\Wlama} \sgn_{[\gamma]}
 \]
 where the summation runs over the cross action orbits
 in $\cP_{\lambda,r}(G)$.
 Since $\Wlama$ is a finite group, we get the theorem by the complete
 reduciblity of $\Coh_{[\lambda]}(G)$.
\end{proof}

We remark that, when $\lambda$ is integral, the set $\cP_{\lambda}(G)$ can be enumerated using \cite{AC} (the
algorithm is implemented in atlas) for linear groups.
Under atlas' parameters, the cross action is also easy to calculate.
For the metaplectic group, the problem was solved by Renard-Trapa \cite{RT1,RT2}.




\subsection{Counting equality by Vogan duality}
In this section, we prove the counting equality for small representations
only assuming the existence of Vogan duality for $G$.

\def\dG{\ckM}
\def\CHC{\sC^{\mathrm{HC}}}

We recall the following key properties of Vogan duality. Fix a regular element
$\lambda$ in $\hha^{*}$, and $B$ is a block in $\cP_{\lambda}(G)$.
Suppose $B$ satisfies the Vogan duality. This means that
there are
\begin{itemize}
  \item
a real reductive group $\dG$ (depends on $\lambda$) in the Harish-Chandra,

\item a block $\ckB$ in $\cP_{\ckrho}(\ckcG)$ where $\ckrho$ is the half sun of
roots of $\ckG$, and
\item
is an bijection %$[\gamma]\in \cP_{\lambda}(G)$
  \[
    \begin{array}{rccc}
     &\cP_{\lambda}(G)&\rightarrow& \cP_{\ckrho}(\dG)\\
      &\gamma & \mapsto& \ckgamma
    \end{array}
  \]
\end{itemize}
such that the above bijection respects Harish-Chandra cells and for $\gamma$
\[
  \LV(\gamma) \cong \LV(\ckgamma)\otimes \sgn.
\]

Now we consider the partition of $\cP_{\lambda}(G)$ into Harish-Chandra cells.
Let
\[
\fC:= \set{\CHC_{\gamma}|\gamma\in \cP_{\lambda}(G)}
\]
be the set of Harish-Chandra cells.
Fix an element $\mu\in [\lambda]$ and define
\[
\begin{split}
  \fC_{\mu}  := &\set{\CHC_{\gamma} \in \fC| [1_{W_{\mu}}:\LV(\gamma)]\neq 0}, \AND\\
  \fC'_{\mu} :=& \fC - \fC_{\mu}.
\end{split}
\]

The following lemma is crucial for us.
Recall the definition of  the special representation $\sigma_{\mu}$.
\begin{lem}
For each cell $\CHC_{\gamma}\in \fC_{\mu}$.
We have the following possible cases
\begin{itemize}
  \item either
        $\sigma_{\mu}$ occurs in $\LV_{\gamma}$ and in which case every irreducible representation $\sigma$ occurs in  $\LV_{\gamma}$
        satisfies $\sigma\approxLR \sigma_{\mu}$, or
  \item $\sigma_{\mu}$ does not occurs in $\LV_{\gamma}$ and every irreducible representation $\sigma$ occurs in  $\LV_{\gamma}$
        satisfies $\sigma\lneqL \sigma_{\mu}$.
\end{itemize}
\end{lem}
\begin{proof}
  Note that $\sigma_{\mu}$ is the maximal irr. repn. under $\leqLR$ order
  satisfies $[1_{W_{\mu}}: \sigma_{\mu}]\neq 0$.

  By Vogan duality, every representation in $\LV_{\ckgamma}$ contains
  $\sgn_{W_{\mu}}$.
  Which implies that, every repn. $\sigma$ in $\CHC_{\gamma}$  satisfies
  \[
   \sigma_{\mu}\otimes \sgn\leqLR \sigma\otimes \sgn
   \Leftrightarrow \sigma \leqLR \sigma_{\mu}
  \]
  Suppose $\sigma_{\mu}$ occurs

\end{proof}


%\subfile{counting_abs}



\section{Unipotent representations of general linear groups}

The classification of unipotent representations of general linear groups is well
known to the experts. We review the result and provide some proves due to the
lack of reference.

% We assume that $G  = \GL_{n}(\bC), \GL_{n}(\bR), \GL_{\half n}(\bH)$ and $\star = A,A^{\bC}, A^{\bH}$ respectively.

In all the cases, $\ckG = \GL_{n}(\bC)$ and $n$ is even if
$G = \GL_{\frac{n}{2}}(\bH)$.
The nilpotent orbit $\ckcO$ is parameterized by Young diagrams.
Let $\YD_{n}$ be the set of Young diagrams with $n$-boxes.

Fix $\ckcO\in \Nil(\ckG) = \YD_{n}$ and set
\[
  \cO :=\dBV(\ckcO) = \ckcO^{t} \AND \wttau := \Spr^{-1}(\cO).
\]
Here $\wttau$ has the partition $\cO$.



\subsection{Special unipotent representations of $G = \GL_n(\bC)$}
When $G = \GL_{n}(\bC)$ the classification is a special case in \cite{BVUni}.

Since the Lusztig canonical quotient of $\ckcO$ is trivial, the set of unipotent representations
of $G = \GL_n(\bC)$ one-one corresponds to nilpotent orbits in $\Nil(\ckGc) = \YD_{n}$ by
\cite{BVUni}.

\begin{thm}
Suppose $\ckcO\in \YD_{n} = \Nil(\ckGc)$ and $\ckcO$ has $k$ columns.
Let
\[
 \pi_{\ckcO} := 1_{\bfrr_1(\ckcO)}\times  1_{\bfrr_2(\ckcO)}\times \cdots
 \times 1_{\bfrr_k(\ckcO)}
\]
be the normalized parabolic induction where $1_{c}$ denote the trivial
representations of $\GL_c(\bC)$.
Then
\[
  \Unip_{A^{\bC}}(\ckcO) = \Set{\pi_{\ckcO}}.
\]
\end{thm}


\begin{remark}
The Vogan duality gives a duality between Harish-Chandra cells.
In this case, Harish-Chandra cells is the double cell
of Lusztig.
Now we have a duality
\[
 \pi_\ckcO \leftrightarrow \pi_{\ckcO^t}.
\]
\end{remark}

We record the following easy lemma which is a baby case of our results on other
classical groups.
\begin{lem}
  Let $\ckcO' := \DDD(\ckcO)$ be the partition obtained by deleting the first
  column of $\ckcO$. Let $\Phi_{\GL_{a}(\bC),\GL_{b}(\bC)}$ (resp.
  $\Phi_{\GL_{a}(\bC),\GL_{b}(\bC)}$) be the theta lift (resp. big theta lift)
  from $\GL_a(\bC)$ to $\GL_b(\bC)$. Then we have
  \[
    \pi_{\ckcO} = \Phi_{\GL_{\abs{\ckcO'}}(\bC),\GL_{\abs{\ckcO}}(\bC)} (\pi_{\ckcO'})= \Phi_{\GL_{\abs{\ckcO'}}(\bC),\GL_{\abs{\ckcO}}(\bC)} (\pi_{\ckcO'}).
  \]
\end{lem}


\subsection{Special unipotent representations of $\GL_n(\bR)$ and
  $\GL_{\frac{n}{2}}(\bR)$}
In this section, we assume $G = \GL_{n}(\bR)$ or $\GL_{\frac{n}{2}}(\bR)$, i.e.
$\star\in \set{A,A^{\bH}}$.


Suppose there is decomposition %$\ckcO\in \Nil(\ckGc)$ and decompose
\[
  \ckcO = \ckcO_{e} \cuprow \ckcO_{o}
\]
where $\ckcO_e$ and $\ckcO_o$ are partitions consist of even and odd rows
respectively.

We list some easy facts below:
\[
  \begin{split}
    W &= \sfS_{n} \\
    W_{[\lamck]} & = \sfS_{\abs{\ckcO_{e}}}\times \sfS_{\abs{\ckcO_{o}}} \\
    W_{\lamck} & = \prod_{i\in \bN^{+}}\sfS_{\bfcc_{i}(\ckcO_{e})}\times \prod_{i\in \bN^{+}}\sfS_{\bfcc_{i}(\ckcO_{o})}\\\
    \tau_{\lamck} & := (j_{W_{\lamck}}^{W_{[\lamck]}}\sgn )\otimes \sgn =  \ckcO_{e}^{t}\boxtimes \ckcO_{o}^{t}\\
    \LC_{\lamck} & = \LRC_{\lamck}= \set{\tau_{\lamck}}, \AND \\
    \wttau_{\lamck} & = j_{W_{[\lamck]}}^{W}\tau_{\lamck} = \Spr^{-1}(\cO).
  \end{split}
\]


% Now let $\ckcO\in \Nil(\ckGc)$ and decompose
% \[
%   \ckcO = \ckcO_{e} \cuprow \ckcO_{o}
% \]
% where $\ckcO_e$ and $\ckcO_o$ are partitions consist of even and odd rows
% respectively.

% Every irreducible representation in $\Irr(W)$ is special. For the
% infinitesimal character $\lamck$,
% \[

% \]

% In all the cases, let $\DDD$ denote the dual descent of Young diagrams.
% Suppose $\ckcO$ is a Young diagram, it delete the longest row in $\ckcO$


% Let $\YD$ be the set of Young diagrams viewed as a finite multiset of positive
% integers. The set of nilpotent orbits in $\GL_n(\bC)$ is identified with Young
% diagram of $n$ boxes.



Let $\sfW_n := \sfS_n \ltimes \set{\pm 1}^n$ denote the Weyl group of type $B_n$
or $C_n$. Let $\sgn$ denote the sign representation of the Weyl group.

The group $\sfW_n$ is naturally embedded in $\sfS_{2n}$. For $\sfW_n$, let
$\epsilon$ denote the unique non-trivial character which is trivial on $\sfS_n$,
which is also the restriction of the $\sgn$ of $S_{2n}$ on $W_n$. The following
branching formula is known (see \cite{BV.W}*{Lemma~4.1~(b)})
\begin{equation}
  \Ind_{\sfW_{n}}^{\sfS_{2n}} = \bigoplus_{\substack{\sigma\in \YD_{2n}\\
      \bfcc_{i}(\sigma) \text{ is even } \forall i\in \bN}} \sigma.
\end{equation}

% Let $n_e = \abs{\ckcO_e}, n_o =
% \abs{\ckcO_o}$. % and $\lambda_\ckcO = \half \ckhh$.
% By the formula of $a$-function, one can easily see that The cell in
% $W(\lamck)$ consists of the unique representation
% $J_{W_{\lamck}}^{\Wint{\lamck}} (1)$. Now the $W$-cell
% $(J_{W_{\lamck}}^{W_{[\lamck]}} \sgn)\otimes \sgn$ consists a single
% representation
% \[
%   \tau_{\ckcO} = \ckcO_{e}^{t}\boxtimes \ckcO_{o}^{t}.
% \]
% The representation $j_{W_{\lamck}}^{S_{n}} \tau_{\ckcO}$ corresponds to the
% orbit $\cO= \ckcO^t $ under the Springer correspondence.
\trivial[h]{ WLOG, we assume $\ckcO = \ckcO_o$.

  Let $\sigma\in \widehat{S_n}$. We identify $\sigma$ with a Young diagram. Let
  $c_i = \bfcc_i(\sigma)$. Then $\sigma = J^{S_n}_{W'} \epsilon_{W'}$ where
  $W' = \prod S_{c_i}$ (see Carter's book). This implies Lusztig's a-function
  takes value
  \[
    a(\sigma) = \sum_i c_i(c_i-1) /2
  \]
  Comparing the above with the dimension formula of nilpotent Orbits
  \cite{CM}*{Collary~6.1.4}, we get (for the formula, see Bai ZQ-Xie Xun's paper
  on GK dimension of $SU(p,q)$)
  \[
    \half \dim(\sigma) = \dim(L(\lambda)) = n(n-1)/2 - a(\sigma).
  \]
  Here $\dim(\sigma)$ is the dimension of nilpotent orbit attached to the Young
  diagram of $\sigma$ (it is the Springer correspondence, regular orbit maps to
  trivial representation, note that $a(\triv)=0$), $L(\lambda)$ is any highest
  weight module in the cell of $\sigma$.


  Return to our question, let $S' = \prod_i S_{\bfcc_i(\ckcO)}$. We want to find
  the component $\sigma_0$ in $\Ind_{S'}^{S_n} 1$ whose $a(\sigma_0)$ is
  maximal, i.e. the Young diagram of $\sigma_0$ is minimal.

  By the branching rule, $\sigma \subset \Ind_{S'}^{S_n} 1$ is given by adding
  rows of length $\bfcc_i(\ckcO)$ repeatly (Each time add at most one box in
  each column). Now it is clear that $\sigma_0 = \ckcO^t$ is desired.

  This agrees with the Barbasch-Vogan duality $\dBV$ given by
  \[
    \ckcO \xrightarrow{\Spr}\ckcO \xrightarrow{\otimes \sgn} \ckcO^t \xrightarrow{\Spr} \ckcO^t.
  \]
}

We define the set of painted partitions of type $A$ as the following:
\begin{equation}\label{eq:PP.AR}
  \PP_{A}(\ckcO) = \Set{\uptau:=(\tau, \cP)|
    \begin{array}{l}
      \text{$\tau = \ckcO^{t}$}\\
      \text{$\Im(\cP)\subseteq \set{\bullet,c,d}$}\\
      \text{$\#\set{i|\cP(i,j)=\bullet}$ is even for all $j\in \bN^{+}$}
    \end{array}
  }.
\end{equation}
It is easy to see that
\begin{equation}\label{eq:PPA.count}
  \#(\PP_{A}(\ckcO)) = \prod_{r\in \bN^{+}} (\#\set{i\in \bN^{+}| \bfrr_{i}(\ckcO)=r}+1)
\end{equation}

\begin{lem} \label{lem:GL.count}
  Suppose $G = \GL_{n}(\bR)$. Let
  \[
    \cC_n := \bigoplus_{\substack{t,c,d\in \bN \\2t+c+d=n}} \Ind_{\sfW_t\times \sfS_c\times \sfS_d}^{\sfS_{n}} \sgn \otimes 1\otimes 1.
  \]
  Then, as $W_{[\lamck]}$-module,
  \[
    \Cint{\lamck} \cong \cC_{\abs{\ckcO_{e}}}\otimes \cC_{\abs{\ckcO_{o}}}.
  \]
  Furthermore,
  \begin{equation}\label{eq:A.count}
    [\tau_{\lamck}: \Cint{\ckcO}] = \# (\PP_{A}(\ckcO_{e}))\times
    \# (\PP_{A}(\ckcO_{o})) = \# (\PP_{A}(\ckcO)).
  \end{equation}
  \qed
\end{lem}
\begin{proof}
  We only explain the last equation and leave the rest to the reader.
  The multiplicity formula follows from the Pieri rule and the last equality
  follows from \eqref{eq:PPA.count}.
\end{proof}
% The $\Wint{\lamck}$-module $\Cint{\lamck}$ is given by the following formula:
% According to Vogan duality, we can obtain the above formula by tensoring
% $\sgn$ on the forumla of the unitary groups in \cite{BV.W}*{Section~4}.

\trivial[]{
By branching rules of the symmetric groups, $\Unip_{\ckcO}(G)$ can be
parameterized by painted partition.

For $\uptau:=(\tau,\cP)\in \PP_{A}(\ckcO)$, we write $\cP_{\uptau}:= \cP$.

  The typical diagram of all columns with even length $2c$ are
  \[
    \ytb{\bullet\cdots\bullet\bullet\cdots\bullet,\vdots\vdots\vdots\vdots\vdots\vdots, \bullet\cdots\bullet c\cdots c, \bullet\cdots\bullet d\cdots d }
  \]

  The typical diagram of all columns with odd length $2c+1$ are
  \[
    \ytb{\bullet\cdots\bullet\bullet\cdots\bullet,\vdots\vdots\vdots\vdots\vdots\vdots, \bullet\cdots\bullet \bullet\cdots\bullet , c\cdots c d\cdots d }
  \]
}

Let $\sgn_a\colon \GL_a(\bR)\rightarrow \set{\pm 1}$ be the sign of determinant
and $1_a$ be the trivial representation of $\GL_a(\bR)$.
For
$\uptau\in \PP_{A}{\ckcO}$, we attache the representation
\begin{equation}\label{eq:u.GLR}
  \pi_\uptau :=
  \bigtimes_{j} \underbrace{1_j \times \cdots \times 1_j}_{c_j\text{-terms}}\times
  \underbrace{\sgn_j \times \cdots \times {\sgn_j} }_{d_j\text{-terms}}.
\end{equation}
where
\begin{itemize}
  \item $j$ running over all column lengths in $\ckcO^t$,
  \item $d_j$ is the number of columns of length $j$ ending with the symbol
        ``d'',
  \item $c_j$ is the number of columns of length $j$ ending with the symbol
        ``$\bullet$'' or ``$c$'', and
  \item ``$\times$'' denote the parabolic induction.
\end{itemize}

\begin{thm}[c.f. Vogan]
  The following map is a bijection
  \[
    \begin{array}{ccc}
      \PP_{A}(\ckcO) & \longrightarrow & \Unip_{A}(\ckcO)\\
      \uptau & \mapsto & \pi_{\uptau}.
    \end{array}
  \]
\end{thm}
\begin{proof}
  The irreducibility and unitarity of $\pi_{\uptau}$ see \cite{V.GL}.
  The map is injective since $\pi_{\uptau_{1}}$ and $\pi_{\uptau_{2}}$ have different
  cuspidal data if  $\uptau_{1}\neq \uptau_{2}$.
  The map is bijective by \Cref{lem:GL.count}.
\end{proof}

\subsection{Special Unipotent representations of $G=\GL_{m}(\bH)$}

Retain the definitions in the previous section. Now suppose $G = \GL_{\frac{n}{2}}(\bH)$.

\begin{lem}
  Let
  \[
  \cC_{n} = \begin{cases}
    \Ind_{\sfW_{\frac{n}{2}}}^{\sfS_{n}}\epsilon
    & \text{if $n$ is even, }  \\
      0 & \text{ otherwise. } \\
    \end{cases}
  \]
  We have
  \[
    \Cint{\lamck}(G)  =
    \cC_{\abs{\ckcO_{g}}}\otimes \cC_{\abs{\ckcO_{b}}}.
    % \begin{cases}
    %   \Ind_{\sfW_{\frac{n}{2}}}^{\sfS_{n}}\epsilon & \text{ if
    %   } \ckcO = \ckcO_{e}\\
    %   0 & \text{ otherwise
    %   } \\
    % \end{cases}
  \]
  In particular, $\Unip_\ckcO(G) = \emptyset$ if $\ckcO \neq \ckcO_e$.

  When $\ckcO = \ckcO_{e}$,
  \[
    \Unip_{A^{\bH}}(G) = \set{\pi_{\ckcO}}
  \]
  where
  \[
    %\pi_{\ckcO} := \bigtimes_i 1_{\bfrr_i(\ckcO)/2}.
    \pi_{\ckcO} := 1_{\bfrr_1(\ckcO)/2}\times 1_{\bfrr_2(\ckcO)/2} \times \cdots
   \times  1_{\bfrr_k(\ckcO)/2},
  \]
  $k = \bfcc_{1}(\ckcO)$
  and $1_{a}$ denote the trivial representation of $\GL_{a}(\bH)$.
% which is a element in $\Unip_{A^{\bH}}(\ckcO)$

% In this case the coherent continuation representation is given by
% \[
%   \Cint{\lamck}(G) = \Ind_{W_m}^{\sfS_{2n}}\epsilon
% \]
% and $\Unip_\ckcO(G)$ is a singleton. %We use partition $\tau:= \ckcO^t$ to parameter special unipotent representations of $\GL_{m}(\bH)$.
\end{lem}
\begin{proof}
  The $G$-module $\pi_{\ckcO}$ is irreducible, unitary and unipotent by Vogan
  \cite{V.GL}.
  The rest parts is clear by $[\tau_{\ckcO}:\Cint{\lamck}(G)] = 0$ if
  $\ckcO \neq \ckcO_{e}$
  and $[\tau_{\ckcO}:\Cint{\lamck}(G)] = 1$ otherwise.
\end{proof}



\section{Special unipotent representations of $\rU(p,q)$}

In this section, let $\star \in \set{A^{*}, \wtA^{*}}$ and
\[
  G :=
  \begin{cases}
    \rU(p,q)  & \text{when }\star = A^{*}\\
    \tU(p,q)  & \text{when }\star = \wtA^{*}
\end{cases}
\]
where $\tU(p,q)$ is the determinant square double cover of $\rU(p,q)$.

For each nilpotent orbit $\ckcO\in \Nil(\ckG)$, let
\[
  \Unip_{G}(\ckcO) :=
  \begin{cases}
   \Set{\pi\in \Irr(G)| \Ann\pi = I_{\ckcO}} & \text{if $\star = A^{*}$}, \\
   \Set{\pi\in \Irr(G)| \Ann\pi = I_{\ckcO} \text{ and $\pi$ is
       geninue}} & \text{if $\star = A^{*}$}. \\
  \end{cases}
\]

In this section we count the number of elements in $\Unip_{G}(\ckcO)$.

We define
\[
  \text{``good parity ''} = \begin{cases}
    \text{the parity of $\abs{\ckcO}$} &  \text{if $\star = A^{*}$}, \\
    \text{the parity of $\abs{\ckcO}+1$} &  \text{if $\star = \wtA^{*}$}, \\
  \end{cases}
\]
and the other parity is called the bad parity.

We make the following decomposition:
\[
\ckcO = \ckcO_{g}\cuprow \ckcO_{b}
\]
where every row in $\ckcO_{g}$ has the good parity
and every row in $\ckcO_{b}$ has the bad parity.

Let $(\ngg,\nbb) = (\abs{\ckcO_g},\abs{\ckcO_b})$.
Clearly
\[
  \begin{split}
    \WLamck &= \sfS_{\ngg}\times \sfS_{\nbb}\\
    \tau_{\lamck}  & = \ckcO_{g}^{t}\boxtimes \ckcO_{b}^{t}\\
  \end{split}
\]

We define $\Coh_{\Lamck}$ to be the coherent continuations of genuine
representations if $\star = \wtA^{*}$.

\begin{lem}
  Let
\[
  \begin{split}
    \cC_{p,q} &= \bigoplus_{\substack{t,s,r\in \bN\\t+r=p, t+s = q}}
    \Ind_{\sfW_{t}\times \sfS_s\times \sfS_r}^{\sfS_{n_g}}
 1\otimes \sgn \otimes \sgn \\
 \cC_{b} &= \begin{cases}
  \Ind_{\sfW_{\frac{\nbb}{2}}}^{\sfS_{\nbb}} 1 & \text{if $\nbb$ is even}\\
  0 & \text{otherwise}.
 \end{cases}
  \end{split}
\]
Then, as the $\WLamck =\sfS_{n_g}\times \sfS_{n_b}$ module,
\[
  \Coh_{\Lamck}(G) = \begin{cases}
    \cC_{p-\half \nbb,q-\half \nbb}\otimes \cC_{b} & \text{if $\nbb$ is even and
      $p,q \geq \half \nbb$}\\
    0 & \text{otherwise}.
  \end{cases}
\]
\end{lem}
\begin{proof}
  For non-linear group, tensoring the genuine $\det^{1/2}$-character yields a
  $\WLam$-module isomorphism
  $\Coh_{[\lamck+(\half, \cdots, \half)]}(\rU(p,q)) \longrightarrow \Coh_{\Lamck}(G)$.
  So the problem reduces to the linear group case, which follows from a routine
  calculation.
\end{proof}

% \trivial[]{
%   Let
%   $\lambda := (\lambda_{1},\cdots,\lambda_{n})=(\underbrace{1/2, \cdots, 1/2}_{a},\underbrace{0,\cdots,0}_{b})$
%   how to compute  $\Coh_{\Lam}(\wtU(p,q))$ of the genuine $\wtU(p,q)$ at the
%   infinitesimal character coset $[\lambda]$?

%   Twist the genuine $\wtU(p,q)$ representations with $\det^{1/2}$ yields
%   an isomorphism between $\Coh_{\Lam}(\wtU(p,q))$ with
%   $\Coh_{\Lam+\half}(U(p,q))$.
% }

% If $p,q \geq \half \nbb$, we write
% \[
% \Gg :=\begin{cases}

% \end{cases}
% \]

Note that the structure of Harish-Chandra cell is determined by \cite{Bo}.
As a consequence, we have the following sharp counting theorem.

\begin{thm}
  Retain the above setting. Then
  \begin{enumerate}[label=(\alph*)]
    \item $\Unip_{G}(\ckcO)= \emptyset$ if any of the following conditions are
          not satisfied
          \begin{itemize}
            \item each row length in $\ckcOb$ occurs in even multiplicity, i.e.
                  $\bfcc_{i}(\ckcOb)$ is even for every $i\in \bN^{+}$,
            \item $p,q \geq \half \nbb$,
            \item and $p+q$ is even when $G = \wtU(p,q)$.
          \end{itemize}
    \item If the above conditions are satisfied,
          \[
          \sharp(\Unip_{G}(\ckcO)) = \sharp(\Unip_{\Gg}(\ckcO_{g}) = \Nil_{G}(\cO).
          \]
          Here $\Gg = \rU(p-\half \nbb,q-\half \nbb)$ or
          $\tU(p-\half \nbb,q-\half \nbb)$ and $\Nil_{G}(\cO)$ denote the set of
          real nilpotent orbit in $G$ whose complexification is $\cO$.
    \item Moreover, write $\ckcOb = \ckcOpb\cuprow \ckcOpb$ and let
          $\pi_{\ckcOpb}$ be the unique element in
          $\Unip_{\GL_{\half\nbb}(\bC)}(\ckcOpb)$. Then we have a bijection
          \[
          \begin{array}{ccc}
            \Unip_{\Gg}(\ckcOg) &\longrightarrow & \Unip_{G}(\ckcO)\\
            \pi_{0}& \mapsto & \pi_{\ckcOpb}\rtimes \pi_{0}. \\
          \end{array}
          \]
  \end{enumerate}
\end{thm}
\begin{proof}
  The lemma follows from the calculation of $[\tau_{\lamck}:\Cint{\lamck}(G)]$.

  When $G = \tU(p,q)$ and $p+q$ is odd, the good parity is even and so
  $\abs{\ckcOb}$ must be odd. There must be a row in $\ckcOb$ occurs in
  odd multiplicity, which implies $\Unip_{G}(\ckcO)=\emptyset$.
\end{proof}

  % \begin{enumT}
  %   \item The set $\Unip_{\ckcO_b}(\rU(p,q))\neq \emptyset$ if and only if $p=q$
  %   and each row lenght in $\ckcO$ has even multiplicity.
  %   \item Suppose $\Unip_{\ckcO_b}(\rU(p,p))\neq \emptyset$, let $\ckcO'$ be the
  %   Young diagram such that $\bfrr_i(\ckcO') = \bfrr_{2i}(\ckcO_b)$ and $\pi'$
  %   be the unique special uinpotent representation in
  %   $\Unip_{\ckcO'}(\GL_{p}(\bC))$. Then the unique element in
  %   $\Unip_{\ckcO_b}(\rU(p,p))$ is given by
  %   \[
  %     \pi := \Ind_{P}^{\rU(p,p)} \pi'
  %   \]
  %   where $P$ is a parabolic subgroup in $\rU(p,p)$ with Levi factor equals to
  %   $\GL_p(\bC)$.
  %   \item In general, when $\Unip_{\ckcO_b}(\rU(p,p))\neq \emptyset$, we have a
  %   natural bijection
  %   \[
  %     \begin{array}{rcl}
  %       \Unip_{\ckcO_g}(\rU(n_1,n_2)) &\longrightarrow& \Unip_{\ckcO}(\rU(n_1+p,n_2+p))\\
  %       \pi_0 & \mapsto & \Ind_P^{\rU(n_1+p,n_2+p)} \pi'\otimes \pi_0
  %     \end{array}
  %   \]
  %   where $P$ is a parabolic subgroup with Levi factor
  %   $\GL_p(\bC)\times \rU(n_1,n_2)$.
  % \end{enumT}

The above theorem ensure us to reduce the problem to the case when $\ckcO = \ckcO_g$.
We give some comments on the construction.
Now assume $\ckcO = \ckcO_g$ and so $\Cint{\ckcO}$ corresponds to the blocks of
the infinitesimal character of the trivial representation.

By \cite{BV.W}*{Theorem~4.2} and \cite{Bo},  Harish-Chandra cells in $\Cint{\ckcO}$ are in one-one
correspondence to real nilpotent orbits in $\cO:=\dBV(\ckcO)=\ckcO^t$.

\trivial{
From the branching rule, the cell is parametered by painted partition
\[
\PP{}(\rU):=\set{\uptau\in \PP{}| \begin{array}{l}\Im (\uptau) \subseteq  \set{\bullet, s,r}\\
  \text{``$\bullet$'' occures even times in each row}
\end{array}
  }.
\]

The bijection $\PP{}(\rU)\rightarrow \SYD, \uptau\mapsto \sO$ is given by the following recipe:
The shape of $\sO$ is the same as that of $\uptau$.
$\sO$ is the unique (upto row switching) signed Young diagram such that
\[
  \sO(i,\bfrr_i(\uptau)) := \begin{cases}
    +,  & \text{when }\uptau(i,\bfrr_i(\uptau))=r;\\
    -,  & \text{otherwise, i.e. }\uptau(i,\bfrr_i(\uptau))\in \set{\bullet,s}.
  \end{cases}
\]

\begin{eg}
  \[
 \ytb{\bullet\bullet\bullet\bullet r,\bullet\bullet , sr,s,r}
 \quad
 \mapsto\quad
 \ytb{+-+-+,+-, -+,-,+}
  \]
\end{eg}
}

Now the following lemma is clear.
\begin{lem}
When $\ckcO=\ckcO_g$, the associated varity of every special unipotent representations in $\Unip_\ckcO(\rU)$
is irreducible. Moreover, the following map  is a bijection.
\[
  \begin{array}{rcl}
  \Unip_{\ckcO}(\rU(p,q)) &\longrightarrow& \Nil_{G}(\ckcO)\\
  \pi_0 & \mapsto & \wAV(\pi_0).
  \end{array}
\]
\qed
\end{lem}
\begin{remark}
  Note that the parabolic induction of an rational nilpotent orbit can be reducible.
  Therefore, when $\ckcO_b\neq \emptyset$, the special unipotent representations can have
  reducible associated variety. Meanwhile, it is easy to see that the map
  $\Unip_{\ckcO}(\rU) \ni \pi \mapsto\wAV(\pi)$ is still injective.
\end{remark}

We will show that every elements in $\Unip_{\ckcO_g}$ can be constructed by iterated theta lifting.
For each $\uptau$, let $\sO$ be the corresponding real nilpotent orbit. Let
$\Sign(\sO)$ be the signature of $\sO$, $\DD(\sO)$ be the signed Young diagram
obtained by deleteing the first column of $\sO$.
Suppose $\sO$ has $k$-columns. Inductively we have a sequence of unitary groups
$\rU(p_i,q_i)$ with $(p_i,q_i) = \Sign(\DD^i(\sO))$ for $i=0, \cdots, k$. Then
\begin{equation}\label{eq:u.U}
%   \pi_\tau = \Phi^{\rU(p_0,q_0)}_{\rU(p_1,q_1)} \Phi^{\rU(p_1,q_1)}_{\rU(p_2,q_2)}\cdots
% \Phi^{\rU(p_{k-1},q_{k-1})}_{\rU(p_k,q_k)}(1)
  \pi_\tau = \Phi_{G_{1},G_{0}} \Phi_{G_{2},G_{1}}\cdots \Phi_{G_{k},G_{k-1}} (1)
\end{equation}
where
\begin{itemize}
  \item
  $G_{0}=G$, and
  \item for $0<i\leq k$, $G_{i} = \rU(p_{i},q_{i})$ if $p_{i-1}+q_{i-1}$ is even and
 $G_{i} = \tU(p_{i},q_{i})$ otherwise, and
  \item
$1$ is the trivial representation of $G_{k}$.
\end{itemize}


\trivial{
  Duality between unitary group and real general linear group.

Suppose $\ckcO = \ckcO_g$. Form the duality between cells of $\rU(p,q)$ and
$\GL(n,\bR)$. We have an ad-hoc (bijective) duality between unipotent
representations:
\[
  \begin{array}{rcl}
 \dBV\colon \Unip_{\ckcO}(\rU)& \rightarrow &\Unip_{\ckcO^t}(\GL(\bR)) \\
 \pi_\uptau &\mapsto& \pi_{\dBV(\uptau)} \\
  \end{array}
\]

Here $\ckcO^t = \dBV(\ckcO)$ and $\dBV(\uptau)$ is the pained bipartition
obtained by transposeing $\uptau$ and replace $s$ and $r$ by $c$ and $d$
respectively. See \eqref{eq:u.U} and \eqref{eq:u.GLR} for the definition of
special unipotent representations on the two sides.
}

%\subfile{counting_A}



% \subfile{counting_abs}

% \subfile{counting_A}


\section{Counting in type BCD}

In this section, we consider the case when %$\ckstar \in \set{B,C,D}$, i.e
$\star \in \set{B,\wtC, C,D,C^{*}, D^{*}}$.

% We identify $\fhh^{*}$ with $\bZ^{n}$ where $n = \rank(\Gc)$
% and let $\rho$ be the half sum of all positive roots.

Recall that
\[
  \text{good parity} =
\begin{cases}
 %\text{odd} & \text{when } \ckstar\in \set{B,D}\\
 %\text{even} & \text{when } \ckstar = C\\
 \text{odd} & \text{when } \star \in \set{C,C^{*},D,D^{*}}\\
 \text{even} & \text{when } \star \in \set{B,\wtC}\\
\end{cases}
\]
For $\ckcO\in \Nil(\ckcG)$, we have the decomposition
\[
  \ckcO = \ckcOb\cuprow \ckcOg \AND \ckcOb = \ckcOpb\cuprow \ckcOpb
\]
where $\ckcOpb$ is a partition of $\half\abs{\ckcOb}$.

%We have $\ckcG_{\lamck} = \ckcG_{b}\otimes \ckcG_{g}$.
Let $n_{b}$ and $n_{g}$ be the rank of $\ckcG_{b}$ and $\ckcG_{g}$
respectively. We have
  \[
    (\nbb,\ngg) =
    \begin{cases}
      (\half \abs{\ckcO_{b}}, \half(\abs{\ckcO_{g}}-1)) & \text{when
      } \star \in \set{C,C^{*}}\\
      (\half \abs{\ckcO_{b}}, \half\abs{\ckcO_{g}}) & \text{when
      } \star \in \set{B,\wtC,D,D^{*}}\\
    \end{cases}
  \]
  and integral Weyl group $\WLamck$is a product of two factors
  \[
    W_{[\lamck]} =\Wb\times \Wg
  \]
  where
  \begin{equation}\label{eq:Wbg}
    \begin{split}
    \Wb & := \begin{cases}
      \sfW_{n_{b}}  & \text{when } \star \in \set{B, \wtC} \\
      \sfW'_{n_{b}} & \text{when } \star \in \set{C,C^{*},D,D^{*}}
      \end{cases}\\
    \Wg & := \begin{cases}
      \sfW_{n_{g}}  & \text{when } \star \in \set{B,C, C^{*} } \\
      \sfW'_{n_{g}} & \text{when } \star \in \set{\wtC,D,D^{*}}
      \end{cases}
    \end{split}
  \end{equation}

  When $\Wb$ or $\Wg$ is a Weyl group of type $D_{n}$, we always have the
  preferred embedding of $\sfS_{n}$ into $\sfW'_{n}$ given by the root system of
  $\ckfgg$. The label $I$ on the irreducible character of $\sfW_{n}$ is refer to
  this particular embedding.

  More precisely, we identify bipartition with $n$ parts with $\Irr(\sfW_{n})$.
  To ease the notations, we let $(\tau_{L},\tau_{R})_{I}$ denote
  the unique irreducible character of $\sfW'_{n}$ given by
  \begin{itemize}
    \item the restriction of
    the irreducible character of $\sfW_{n}$ attached to $(\tau_{L},\tau_{R})$ if
    $\tau_{L}\neq \tau_{R}$, and
    \item
    the character
    $\Ind_{\sfS_{\frac{n}{2}}}^{\sfW_{n}} \tau_{L}$ if $\tau_{L}=\tau_{R}$.
  \end{itemize}
  We remark that we always have
  \[
    (\tau_{L},\tau_{R})_{I}=(\tau_{R},\tau_{L})_{I}
  \]
  as $\sfW'_{n}$-character.


  \subsection{The left cell}
  \label{sec:LCBCD}
  In this subsection, we described the Lusztig left cell attached to
  $\lambda_{\ckcO}$ in each cases, where
  $\star \in \set{B,C,\wtC,C^{*},D,D^{*}}$.


  To state the results, we made some definitions first. Define the irreducible
  $W_{b}$ representation attached to $\ckcO_{b}$ by the following formula
  \begin{equation}\label{eq:taub}
    \begin{split}
      \tau_{b} & := (\tau_{L,b},\tau_{R,b})\\
      &:= \begin{cases}
        \Big(\big(\half(\bfrr_{2}(\ckcO_{b})+1), \half(\bfrr_{4}(\ckcO_{b})+1), \cdots, \half(\bfrr_{2c}(\ckcO_{b})+1)\big),\\
        \hspace{1em}\big(\half(\bfrr_{2}(\ckcO_{b})-1), \half(\bfrr_{4}(\ckcO_{b})-1), \cdots, \half(\bfrr_{2c}(\ckcO_{b})-1) \big)\Big)
        & \text{if } \star \in \set{B,\wtC},\\
        \Big( \big(\half\bfrr_{2}(\ckcO_{b}), \half\bfrr_{4}(\ckcO_{b}),\cdots, \half\bfrr_{2c}(\ckcO_{b})\big), \\
        \hspace{1em} \big(\half\bfrr_{2}(\ckcO_{b}), \half\bfrr_{4}(\ckcO_{b}),\cdots, \half\bfrr_{2c}(\ckcO_{b}) \big)\Big)_{I}
        & \text{if } \star \in \set{C,C^{*}, D,D^{*}},\\
      \end{cases}
    \end{split}
  \end{equation}
  with $2c = \bfcc_{1}(\ckcO_{b})$. Define partitions
  \begin{itemize}
    \item $\ckcO'_{b}$ such that $\bfrr_{i}(\ckcO'_{b}):= \bfrr_{2i}(\ckcO_{b})$
          for all $i\in \bN^{+}$,
    \item $\cO'_{b}:= (\ckcO'_{b})^{t}$, and
    \item $\ckcO_{b}:= \cO'_{b}\cupcol \cO'_{b}$.
  \end{itemize}

  Set
  \[
    \CPPs(\ckcO_{g}) =
    \begin{cases}
      \set{(2i-1,2i)| \bfrr_{2i-1}(\ckcO_{g})- \bfrr_{2i}(\ckcO_{g})\geq
        2, %\text{and}
        i\in \bN^{+}} & \text{if $\star\in \Set{C,\wtC,C^{*}}$}\\
      \set{(2i,2i+1)| \bfrr_{2i}(\ckcO_{g})- \bfrr_{2i+1}(\ckcO_{g})\geq
        2, %\text{and }
        i\in \bN^{+}} & \text{if $\star\in \Set{B,D,D^{*}}$}.
    \end{cases}
  \]
  Let
  \[
    \wtA(\ckcO) := \bF_{2}[\CPP(\ckcO_{g})]
  \] be the power set of $\CPPs(\ckcO_{g})$.

  For each $\wp\in \wtA(\ckcO)$ we define an element $\tau_{\wp}$ in
  $\Irr(\Wg)$. Here
  \[
    \tau_{\wp} :=
    \begin{cases}
      (\imathp,\jmathp) & \text{when } \star \in \set{B,C, C^{*} } \\
      (\imathp,\jmathp)_{I} & \text{when } \star \in \set{\wtC,D,D^{*}}
    \end{cases}
  \]
  and $(\imathp, \jmathp)$ are given by the following formulas:
  \begin{itemize}
    \item Suppose $\star\in \set{C,C^{*}}$ and let
          $l=\min\set{i| \bfrr_{2i}(\ckcO_{g})=0}$. Then
          \[
          (\bfcc_{l}(\imathp), \bfcc_{l}(\jmathp)) := (0,\half(\bfrr_{2l+1}(\ckcO_{g})-1))
          \]
          and, for all $1\leq i< l$
          \[
          (\bfcc_{i}(\imathp), \bfcc_{i}(\jmathp)):=
          \begin{cases}
            (\half (\bfrr_{2i}(\ckcO_{g})+1), \half (\bfrr_{2i-1}(\ckcO_{g})-1))
            & \text{if } (2i-1,2i)\notin \wp,\\
            (\half (\bfrr_{2i-1}(\ckcO_{g})+1),\half (\bfrr_{2i}(\ckcO_{g})-1)) & \text{otherwise.}
          \end{cases}
          \]
    \item Suppose $\star\in \set{D,D^{*}}$ and let
          $l=\min\set{i| \bfrr_{2i+1}(\ckcO_{g})=0}$. Then
          \[
          \begin{split}
            \bfcc_{1}(\imathp) &:=
            \half(\bfrr_{1}(\ckcO_{g})+1)\\
            (\bfcc_{l+1}(\imathp), \bfcc_{l}(\jmathp)) &:= (0,\half(\bfrr_{2l}(\ckcO_{g})-1))
          \end{split}
          \]
          and, for all $1\leq i<l$
          \[
          (\bfcc_{i+1}(\imathp), \bfcc_{i}(\jmathp)):=
          \begin{cases}
            \left(\half (\bfrr_{2i+1}(\ckcO_{g})+1), \half (\bfrr_{2i}(\ckcO_{g})-1)\right)
            & \text{if } (2i,2i+1)\notin \wp,\\
            (\half (\bfrr_{2i}(\ckcO_{g})+1),\half (\bfrr_{2i+1}(\ckcO_{g})-1)) & \text{otherwise.}
          \end{cases}
          \]
          % \[
          %   (\bfcc_{i}(\imathp), \bfcc_{i}(\jmathp)):=
          %   \begin{cases}
          %     (\half (\bfrr_{2i-1}(\ckcO_{g})+1),\half (\bfrr_{2i}(\ckcO_{g})-1)) &\text{if } (2i-1,2i)\in \wp, \\
          %     (\half (\bfrr_{2i}(\ckcO_{g})+1), \half (\bfrr_{2i-1}(\ckcO_{g})-1))
          %     & \text{if } (2i-1,2i)\notin \wp\\
          %     & \text{ and }\bfrr_{2i}(\ckcO_{g})\neq 0,
          %     \\
          %     (0,0)
          %     & \text{if } \bfrr_{2i-1}(\ckcO_{g})=0,\\
          %     (0, \half (\bfrr_{2i-1}(\ckcO_{g})-1)) & \text{otherwise}
          %   \end{cases}
          % \]

          % \[
          %   (\bfcc_{l+1}(\imathp), \bfcc_{l+1}(\jmathp)) := (0,\half(\bfrr_{2l+1}(\ckcO_{g})-1))
          % \]
          % and for all $1\leq i\leq l$
    \item Suppose $\star=B$. Then
          \[
          \bfcc_{1}(\jmathp) := \half\bfrr_{1}(\ckcO_{g})
          \]
          and for all $i\geq 1$
          \[
          (\bfcc_{i}(\imathp), \bfcc_{i+1}(\jmathp)):=
          \begin{cases}
            (\half \bfrr_{2i}(\ckcO_{g}), \half \bfrr_{2i+1}(\ckcO_{g}))
            & \text{if } (2i,2i+1)\notin \wp,\\
            (\half \bfrr_{2i+1}(\ckcO_{g}),\half \bfrr_{2i}(\ckcO_{g})) & \text{otherwise.}
          \end{cases}
          \]
    \item Suppose $\star = \wtC$. Then for all $i\geq 1$
          \[
          (\bfcc_{i}(\imathp), \bfcc_{i}(\jmathp)):=
          \begin{cases}
            (\half \bfrr_{2i-1}(\ckcO_{g}), \half \bfrr_{2i}(\ckcO_{g}))
            & \text{if } (2i-1,2i)\notin \wp,\\
            (\half \bfrr_{2i}(\ckcO_{g}),\half \bfrr_{2i-1}(\ckcO_{g})) & \text{otherwise.}
          \end{cases}
          \]
  \end{itemize}

  For $\wp\subset\CPPs(\ckcO_{g})$, let $\wp^{c}$ be the complement of $\wp$ in
  $\CPPs(\ckcO_{g})$ and we have $\tau_{\wp} = \tau_{\wp^{c}}$ if
  % $\star \in \Set{\wtC,D,D^{*}}$. ### This is not correct!
  $\star = \wtC$.

  We define
  \[
    \barA(\ckcO)=
    \begin{cases}
      \wtA(\ckcO)/\wp\sim\wp^{c} & \text{when } \star =\wtC,\\
      \wtA(\ckcO) & \text{otherwise.}\\
    \end{cases}
    % \begin{cases}
    %   \wtA(\ckcO)/\wp\sim\wp^{c} & \text{when } \star \in \set{\wtC,D,D^{*}}.\\
    %   \wtA(\ckcO) & \text{when } \star \in \set{B,C,C^{*}},\\
    % \end{cases}
  \]
  Here $\wtA(\ckcO)/\wp\sim\wp^{c}$ denote the quotient of $\wtA(\ckcO)$ by
  identifying $\wp$ with its complement $\wp^{c}$.

  When $\star\neq \wtC$, $\barA(\ckcO)$ is nothing but the Lusztig canonical
  quotient attached to $\ckcO$. \trivial{ This can be seem from the following
    lemma, c.f. \cite{BVUni}*{Proposition~5.28}. }

  Note that by definition, we have $\wtA(\ckcO)=\wtA(\ckcO_{g})$ and
  $\barA(\ckcO)=\barA(\ckcO_{g})$.

  Recall that
  \[
    \LV_{\ckcO}:= \left(J_{\Wlamck}^{\WLamck} \sgn\right) \otimes \sgn.
  \]
  and $\LC_{\ckcO}$ is the multiset of irreducible components

  \begin{lem}[c.f. Barbasch-Vogan{\cite{BVUni}*{Proposition~5.28}}]
    \label{lem:Lcell}
    In all the cases, $\LC_{\ckcO}$ is multiplicity free and we have the
    following bijections
    \[
      \begin{array}{lccccccc}
        \barA(\ckcO)&=&\barA(\ckcO_{g}) & \longrightarrow & \LC(\ckcO_{g})
        & \longrightarrow & \LC(\ckcO)\\
                    &  &\wp & \mapsto & \tau_{\wp} &
                                                     \mapsto & \tau_{b}\otimes \tau_{\wp}.
      \end{array}
    \]
    Moreover,
    \[
      \tau_{\ckcO}=\tau_{b}\otimes \tau_{\emptyset}
    \] is the unique special representation in $\LC_{\ckcO}$ and
    \begin{equation}\label{eq:dBV.W}
      \Spr(j_{\WLamck}^{W}(\tau_{\ckcO})) = \dBV(\ckcO) = \cO_{b}\cupcol \dBV(\ckcO_{g}).
    \end{equation}
    \trivial[]{ The last equality could be checked using Sommer's formula on
      Springer correspondence directly: double columns $(2c+1,2c+1)$ corresponds
      to
      $ B_{c=\alpha_{2i-1}}\times D_{c+1=\alpha_{2i}+1}=D_{c+1=\alpha_{2i-1}+1}\times C_{c=\alpha_{2i}}$
      factor in type $B,\wtC$. double columns $(2c,2c)$ corresponds to factor
      $D_{c=\beta_{2i-1}}\times C_{c=\beta_{2i}}=D_{c=\beta_{2i-1}}\times B_{c=\beta_{2i}}$
      in type $C,C^{*},D,D^{*}$. } Here $\dBV$ is the metaplectic dual if
    $\star=\wtC$ and is the Barbasch-Vogan dual otherwise.
  \end{lem}
  \begin{proof}
    For $\ckcO_{g}$, the lemma is given by \cite{BVUni}*{Proposition~5.28}. For
    all the cases, the lemma follow from an induction on number of columns using
    Lusztig's formula of $J$-induction in \cite{Lu}*{\S 4.4-4.6}. The equality
    \eqref{eq:dBV.W} is due to Barbasch-Vogan for linear groups
    \cite{BVUni}*{Proposition~A2}.

    \trivial[h]{ {\bf Suppose $\star=C$.}

      In this case, bad parity is even and each row length occur with even
      multiplicity. Suppose
      $\ckcO_{b} = (C_{1}, C_{1}, C_{2},C_{2}, \cdots, C_{k'},C_{k'})$ with
      $c_{1}=2k$ and $k' = \bfrr_{1}(\ckcO_{b})$.
      \[
        W_{\lamckb} = S_{C_{1}}\times S_{C_{2}}\times \cdots S_{C_{k'}}.
      \]
      The symbol of trivial representation of trivial group of type D is
      \[
        \binom{0,1, \cdots, k-1}{0,1, \cdots, k-1}.
      \]
      Now it is easy to see that (use the similar computation as below)
      \[
        J_{W_{\lamckb}}^{W_{b}}\sgn = ((\half C_{1}, \half C_{2},\cdots, \half C_{k'}),(\half C_{1}, \half C_{2},\cdots, \half C_{k'})).
      \]


      For the good parity part. Let
      $r'_{i} = \floor{\half(\bfrr_{i}(\ckcO_{g})-\bfrr_{i+1}(\ckcO_{g}))}$.
      Suppose $\ckcO_{g}$ has $2l+1$ columns (superscripts denote the
      multiplicity)
      \[
        \ckcO_{g} = ((2l+1)^{2r'_{2l+1}+1}, 2l^{2r'_{2l}}, (2l-1)^{2r'_{2l-1}}, \cdots, 2^{2r'_{2}}, 1^{2r'_{1}} )
      \]
      and
      % $\ckcO_{g} = (2c_{1}+1, C_{2}, C_{2},C_{3},C_{3},\cdots, C_{k'},C_{k'})$
      % with $2c_{1}+1=2l+1$ and $2k'+1 = \bfrr_{1}(\ckcO_{g})$.
      \[
        W_{\lamckg} = W_{l}\times \underbrace{S_{2l+1}\times \cdots \times S_{2l+1}}_{2r'_{2l+1}\text{-terms}} \times \prod_{i<2l+1} \underbrace{S_{i}\times \cdots\times S_{i}}_{r'_{i}\text{-terms}}
      \]

      The symbol of sign representation of $W_{l}$ is
      \[
        \binom{0,1,2, \cdots, l}{1,2, \cdots, l}.
      \]
      The induction begins with the longest columns to the shorter columns

      Induce to include all $2l+1$-length columns yields
      \[
        \binom{r'_{2l+1}+0,r'_{2l+1}+1,r'_{2l+1}+2, \cdots, r'_{2l+1}+l}{ r'_{2l+1}+1,r'_{2l+1}+2, \cdots, r'_{2l+1}+l}.
      \]
      Now move the the shorter columns, we see that when even columns
      $(2i)^{2r'_{2i}}$ occurs, it adds $(i)^{r'_{2i}}$ columns on the both
      sides of the bipartition; when odd columns $(2i+1)^{r'_{2i+1}}$ occur, the
      bifurcation happens: one can
      \begin{itemize}
        \item attach columns $(i+1)^{r'_{2i+1}}$ on the left and columns
              $(i)^{r'_{2i+1}}$ on the right, which corresponds to
              $(2i+1,2i+2)\neq \wp$, or
        \item attach columns $(i)^{r'_{2i+1}}$ on the left and columns
              $(i+1)^{r'_{2i+1}}$ on the right, which corresponds to
              $(2i+1,2i+2)\in \wp$,
      \end{itemize}

      Therefore,
      \[
        \begin{array}{ccc}
          J_{W_{\lamckg}}^{W_{g}} \sgn
          &\leftrightarrow&  \bF_{2}(\CPP(\ckcO_{g}))\\
          (\cktau_{L},\cktau_{R}) =:\cktau_{\wp}&\leftrightarrow & \wp
        \end{array}
      \]
      where
      \[
        \bfrr_{l+1}(\cktau_{L}) = r'_{2l+1} = \half (\bfrr_{2l+1}(\ckcO_{g})-1)
      \]
      and, if $(2i-1,2i)\notin \wp$,
      \[
        \begin{split}
          \bfrr_{i}(\cktau_{L}) & = \sum_{l\geq 2i-1} r'_{l}
          = \half(\bfrr_{2i-1}(\ckcO)-1)\\
          \bfrr_{i}(\cktau_{R}) & = 1 + \sum_{l\geq 2i} r'_{l} = \half(\bfrr_{2i}(\ckcO)+1)
        \end{split}
      \]
      if $(2i-1,2i)\in \wp$,
      \[
        \begin{split}
          \bfrr_{i}(\cktau_{L}) & = \sum_{l\geq 2i} r'_{l}
          = \half(\bfrr_{2i}(\ckcO)-1)\\
          \bfrr_{i}(\cktau_{R}) & = 1 + \sum_{l\geq 2i-1} r'_{l} = \half(\bfrr_{2i-1}(\ckcO)+1)
        \end{split}
      \]

      % \[
      %   \begin{split}
      %     \bfrr_{l+1}(\cktau_{L}) & = r'_{2l+1} =
      %     \half (\bfrr_{2l+1}(\ckcO_{g})-1)\\
      %     (\bfrr_{i}(\cktau_{L}), \bfrr_{i}(\cktau_{R})) & =
      %     \begin{cases}
      %       (\half(\bfrr_{2i-1}(\ckcO_{g})-1), \half(\bfrr_{2i}(\ckcO_{g})+1)) & (2i-1,2i)\notin \wp\\
      %       (\half(\bfrr_{2i}(\ckcO_{g})-1), \half(\bfrr_{2i-1}(\ckcO_{g})+1)) & (2i-1,2i)\in \wp
      %     \end{cases}
      %   \end{split}
      % \]

      Since $\tau_{\wp} = \cktau_{\wp}\otimes \sgn$, we get the claim.

      We adopt the convention that
      \[
        \sfS_{\cO} := \prod_{i\in \bN^{+}}\sfS_{\bfcc_{i}(\cO)}
      \]
      so that $j_{\sfS_{\cO}}^{\sfS_{\abs{\cO}}}\sgn = \cO$ for each partition
      $\cO$.

      Now consider the orbit under the Springer correspondence.

      Let
      $\ckcO'_{b}: = [\bfrr_{2}(\ckcO_{b}), \bfrr_{4}(\ckcO_{b}),\cdots, \bfrr_{2k}(\ckcO_{b})]$,
      $\cO'_{b}:=(\ckcO'_{b})^{t}$ and $\cO_{b}:=\cO'_{b}\cupcol \cO'_{b}$.
      Clearly, $\ckcO_{b} = \ckcO'_{b}\cuprow \ckcO'_{b}$. Note that
      $\tau_{b} = j_{S_{\cO'_{b}}}^{W'_{b}} \sgn$ (by the formula of fake degree
      see Lusztig or Carter's book). So, by induction by stage of $j$-induction,
      we have
      \[
        \wttau_{\cO}:= j_{W'_{b}\times W_{g}}^{W_{n}} (\tau_{b}\otimes \tau_{\emptyset}) = j_{S_{\cO'_{b}}\times W_{g}}^{W_{n}} \sgn\otimes \tau_{\wp}.
      \]
      By Barbasch-Vogan, $\cO_{g}:=\Spr(\tau_{\emptyset}) = d_{BV}(\ckcO_{g})$,
      which is well know how to calculate. (In fact, one can deduce the result
      by our computation. )

      Since the Springer correspondence commutes with parabolic induction, we
      get
      $\Spr(\wttau) = \Ind_{\GL_{\cO'_{b}}\times \Sp(2g)}^{\Sp(2n)} 0\times \cO_{g} = \cO_{b}\cupcol \cO_{g}$.


      \medskip

      {\bf Suppose $\star=D$.}

      The bad parity part is the same as that of the case when $\star = C$.

      Now consider the good parity part.
      \[
        \ckcO_{g} = ((2l)^{2r'_{2l}+1}, (2l-1)^{2r'_{2l-1}}, (2l-2)^{2r'_{2l-2}}, \cdots, 2^{2r'_{2}}, 1^{2r'_{1}} )
      \]
      and
      \[
        W_{\lamckg} = W'_{l}\times \underbrace{S_{2l}\times \cdots \times S_{2l}}_{2r'_{2l}\text{-terms}} \times \prod_{i<2l} \underbrace{S_{i}\times \cdots\times S_{i}}_{r'_{i}\text{-terms}}
      \]

      The symbol of sign representation of $W'_{l}$ is
      \[
        \binom{0,1, \cdots, l-1}{1,2, \cdots, l\phantom{-1}}.
      \]
      (Here we always made the choice of the top and bottom row to compatible
      with the type $C$ case. )

      Induce to include all $2l$-length columns yields
      \[
        \binom{r'_{2l}+0,r'_{2l}+1, \cdots, r'_{2l}+l-1}{ r'_{2l}+1,r'_{2l}+2, \cdots, r'_{2l}+l\phantom{-1}}.
      \]
      Now move the the shorter columns. When odd columns $(2i+1)^{2r'_{2i+1}}$
      occurs, it adds $(i)^{r'_{2i+1}}$ columns on the left and
      $(i+1)^{r'_{2i+1}}$ on the right. When even columns $(2i)^{r'_{2i}}$
      occur, the bifurcation happens: one can
      \begin{itemize}
        \item attach columns $(i)^{r'_{2i}}$ on the left and columns
              $(i)^{r'_{2i}}$ on the right, which corresponds to
              $(2i,2i+1)\neq \wp$, or
        \item attach columns $(i-1)^{r'_{2i}}$ on the left and columns
              $(i+1)^{r'_{2i}}$ on the right, which corresponds to
              $(2i,2i+ 1)\in \wp$,
      \end{itemize}

      Therefore,
      \[
        \begin{array}{ccc}
          \bF_{2}(\CPP(\ckcO_{g}))&\longrightarrow
          & J_{W_{\lamckg}}^{W_{g}} \sgn \\
          \wp&\mapsto&    (\cktau_{L},\cktau_{R}) =:\cktau_{\wp}
        \end{array}
      \]
      where
      \[
        \bfrr_{l}(\cktau_{L}) = r'_{2l} = \half (\bfrr_{2l}(\ckcO_{g})-1)
      \]
      \[
        \bfrr_{1}(\cktau_{R}) = 1+ \sum_{i} r'_{i} = \half (\bfrr_{1}(\ckcO_{g})+1)
      \]
      and, if $(2i,2i+1)\notin \wp$,
      \[
        \begin{split}
          \bfrr_{i}(\cktau_{L}) & = \sum_{l\geq 2i} r'_{l}
          = \half(\bfrr_{2i}(\ckcO)-1)\\
          \bfrr_{i+1}(\cktau_{R}) & = 1 + \sum_{l\geq 2i+1} r'_{l} = \half(\bfrr_{2i+1}(\ckcO)+1)
        \end{split}
      \]
      if $(2i,2i+1)\in \wp$,
      \[
        \begin{split}
          \bfrr_{i}(\cktau_{L}) & = \sum_{l\geq 2i+1} r'_{l}
          = \half(\bfrr_{2i+1}(\ckcO)-1)\\
          \bfrr_{i}(\cktau_{R}) & = 1 + \sum_{l\geq 2i} r'_{l} = \half(\bfrr_{2i}(\ckcO)+1)
        \end{split}
      \]

      Also note that $\cktau_{\wp}=\cktau_{\wp^{c}}$. The rest parts are the
      same as that of type $C$.

      {\bf Suppose $\star=B$. }

      In this case, bad parity is odd and every odd row occurs with with even
      times.

      We can write
      $r'_{i} := \floor{\half(\bfrr_{i}(\ckcO_{b})-\bfrr_{i-1}(\ckcO_{b}))}$
      \[
        \ckcO_{b} % = [2r_{1}+1, 2r_{1}+1, \cdots, 2r_{k}+1,2r_{k}+1]
        % = (2c_{0},2c_{1},2c_{1}, \cdots, 2c_{l}, 2c_{l}).
        = ((2l)^{2r'_{2l}+1}, (2l-1)^{2r'_{2l-1}},\cdots, 1^{2r'_{1}})
      \]
            %             with $k = c_{0}$ and $l = r_{1}$.
      Then
      \[
        W_{\lamckb} = W_{l} \times \underbrace{S_{2l}\times \cdots \times S_{2l}}_{2r'_{2l}\text{-terms}} \times \prod_{i<2l} \underbrace{S_{i}\times \cdots\times S_{i}}_{r'_{i}\text{-terms}}
      \]
      (Note that in the product, $r'_{i}=0$ if $i$ is odd.) The computation of
      $\cksigma_{b} = J_{W_{\lamckb}}^{W_{b}} \sgn$ is similar to that of the
      good parity for type $C$ with no bifurcating, one deduce that
      $J$-induction and $j$-induction gives the same result.
      \[
        \begin{split}
          \cksigma_{b} &=
          \binom{0, 1+r_{l}, 2+r_{l-1}\cdots, l+r_{1}}{1+r_{l},2+r_{l-1}, \cdots, l+r_{1}}\\
          & = ([r_{1},r_{2},\cdots, r_{l}],[r_{1}+1,r_{2}+1,\cdots,r_{l}+1])\\
        \end{split}
      \]
      with $r_{i} = \half\bfrr_{2i-1}(\ckcO_{b}) = \half\bfrr_{2i}(\ckcO_{b})$.
      Now
      \[
        \sigma_{b} = ((r_{1}+1,r_{2}+1,\cdots,r_{l}+1), (r_{1},r_{2},\cdots, r_{l})) = j_{S_{\cO'_{b}}}^{W_{b}}\sgn
      \]
      where
      $\cO'_{b}=(\bfrr_{2}(\ckcO_{b}),\bfrr_{4}(\ckcO_{b}),\cdots, \bfrr_{2l}(\ckcO_{b}))$.
      Under the Springer correspondence of type $B$, it corresponds to
      $\Ind_{\GL_{b}}^{\SO(2b+1)}\cO'_{b} = \cO'_{b}\cuprow \cO'_{b}\cuprow (1)$.

      % \[
      %   \begin{split}
      %     \cksigma_{b} &:= \sigma_{b}\otimes \sgn = j_{W_{\lamckb}}^{W_{b}} \sgn \\
      %     & = %\dagger_{2c_{l}}\cdots \dagger_{2c_{1}}
      %     \sigma_{b}\otimes \sgn = j_{W_{\lamckb}}^{W_{b}} \sgn\otimes
      %     \binom{0, 1, \cdots, c_{0}}{1, \cdots, c_{0}}\\
      %     & =
      %     \binom{0, 1+r_{k}, 2+r_{k-1}\cdots, c_{0}+r_{1}}{1+r_{k},2+r_{k-1}, \cdots, c_{0}+r_{1}}\\
      %     & = ([r_{1},r_{2},\cdots, r_{k}],[r_{1}+1,r_{2}+1,\cdots,r_{k}+1])\\
      %     &= ((c_{1},c_{2},\cdots, c_{k}),(c_{0},c_{1}, \cdots, c_{l}))\\
      %   \end{split}
      % \]


      % We take the convention that $\dagger \cO = [r_{i}+1]$. By abuse of
      % notation, let $\dagger_{n} \sigma$ denote the
      % $j_{S_{n} \times W_{\abs{\sigma}}}^{W_{n+\abs{\sigma}}} \sgn\otimes \sigma$.
      % We can write
      % \[
      %   \ckcO_{b} = [2r_{1}+1, 2r_{1}+1, \cdots, 2r_{k}+1,2r_{k}+1] = (2c_{0},2c_{1},2c_{1}, \cdots, 2c_{l}, 2c_{l})
      % \]
      % with $k = c_{0}$ and $l = r_{1}$.

      % \[
      %   \begin{split}
      %     W_{\lamckb} &= W_{c_{0}} \times S_{2c_{1}} \times S_{2c_{2}}\times \cdots \times S_{2c_{l}}\\
      %     \cksigma_{b} &:= \sigma_{b}\otimes \sgn = j_{W_{\lamckb}}^{W_{b}} \sgn \\
      %     & = \dagger_{2c_{l}}\cdots \dagger_{2c_{1}}
      %     \binom{0, 1, \cdots, c_{0}}{1, \cdots, c_{0}}\\
      %     & =
      %     \binom{0, 1+r_{k}, 2+r_{k-1}\cdots, c_{0}+r_{1}}{1+r_{k},2+r_{k-1}, \cdots, c_{0}+r_{1}}\\
      %     & = ([r_{1},r_{2},\cdots, r_{k}],[r_{1}+1,r_{2}+1,\cdots,r_{k}+1])\\
      %     &= ((c_{1},c_{2},\cdots, c_{k}),(c_{0},c_{1}, \cdots, c_{l}))\\
      %   \end{split}
      % \]

      % Therefore
      % \[
      %   \begin{split}
      %     \sigma_{b} &= \cksigma_{b}\otimes \sgn = ((r_{1}+1,r_{2}+1,\cdots,r_{k}+1),(r_{1},r_{2},\cdots, r_{k})) \\
      %     & = j_{S_{2r_{1}+1}\times \cdots S_{2r_{k}+1}}^{W_{b}} \sgn\\
      %     & = j_{S_{b}}^{W_{b}} (2r_{1}+1, 2r_{2}+1, \cdots, 2r_{k}+1)
      %   \end{split}
      % \]
      % which corresponds to the orbit
      % \[
      %   \cO_{b} = (2r_{1}+1, 2r_{1}+1,2r_{2}+1, 2r_{2}+1, \cdots,2r_{k}+1, 2r_{k}+1 ) = \ckcO_{b}^{t}.
      % \]
      % (Note that $\cO'_{b} = (2r_{1}+1,2r_{2}+1, \cdots, 2r_{k}+1)$ which
      % corresponds to $j_{W_{L_{b}}}^{S_{b}}\sgn$ and
      % $\ind_{L}^{G} \cO'_{b} = \cO_{b}$. ) This implies the unique special
      % representation is
      % \[
      %   \sigma_{b} = (j_{W_{\lamckb}}^{W_{b}}\sgn), \quad \text{where
      % } W_{L,b} = \prod_{i=1}^{k} S_{2r_{i}+1}.
      % \]
      % The $J$-induction is calculated by \cite{Lu}*{(4.5.4)}. It is easy to
      % see that in our case $J_{W_{\lamckb}}^{W_{b}} \sgn$ consists of the
      % single special representation by induction.


      Now we consider the good parity parts, where each row of $\ckcO_{g}$ has
      even length.

      Assume $r'_{i} := \half(\bfrr_{i}(\ckcO_{g})-\bfrr_{i-1}(\ckcO_{g}))$ and
      so
      \[
        \ckcO_{g} % = [2r_{1}+1, 2r_{1}+1, \cdots, 2r_{k}+1,2r_{k}+1]
        % = (2c_{0},2c_{1},2c_{1}, \cdots, 2c_{l}, 2c_{l}).
        = ((2l+1)^{2r'_{2l+1}}, (2l)^{2r'_{2l}},\cdots, 1^{2r'_{1}})
      \]
      % Consider
      % \[
      %   \cO_{g} = [2r_{1},2r_{2}, \cdots, 2r_{2k-1},2r_{2k}] = (C_{1},C_{1}, C_{2},C_{2},\cdots, C_{l}, C_{l}).
      % \]
      with $l =\min\set{i|\bfrr_{2i+2}(\ckcO_{g}) = 0}$.

      Then
      \[
        W_{\lamckg} = \times \prod_{i\leq 2l+1} \underbrace{S_{i}\times \cdots\times S_{i}}_{r'_{i}\text{-terms}}
      \]

      Note that the trivial representation of the trivial group has symbol
      \[
        \binom{0,1, 2, \cdots, l\phantom{-1}}{0,1, \cdots, l-1}.
      \]


      Induce to include all $2l+1$-length columns yields
      \[
        \binom{r'_{2l+1}+0,r'_{2l+1}+1,r'_{2l+1}+2,\cdots, r'_{2l+1}+l\phantom{-1}}{ r'_{2l+1}+0,r'_{2l+1}+1, \cdots, r'_{2l+1}+l-1}.
      \]
      Now move the the shorter columns. When odd columns $(2i+1)^{2r'_{2i+1}}$
      occurs, it adds $(i+1)^{r'_{2i+1}}$ columns on the left and
      $(i)^{r'_{2i+1}}$ on the right. When even columns $(2i)^{r'_{2i}}$ occur,
      the bifurcation happens: one can
      \begin{itemize}
        \item attach columns $(i)^{r'_{2i}}$ on the left and columns
              $(i)^{r'_{2i}}$ on the right, which corresponds to
              $(2i,2i+1)\neq \wp$, or
        \item attach columns $(i-1)^{r'_{2i}}$ on the left and columns
              $(i+1)^{r'_{2i}}$ on the right, which corresponds to
              $(2i,2i+1)\in \wp$.
      \end{itemize}


      Therefore,
      \[
        \begin{array}{ccc}
          \bF_{2}(\CPP(\ckcO_{g}))&\longrightarrow
          & J_{W_{\lamckg}}^{W_{g}} \sgn \\
          \wp&\mapsto&    (\cktau_{L},\cktau_{R}) =:\cktau_{\wp}
        \end{array}
      \]
      where
      % \[
      %   \bfrr_{2l+1}(\cktau_{L}) = r'_{2l+1} = \half \bfrr_{2l+1}(\ckcO_{g})
      % \]
      \[
        \bfrr_{1}(\cktau_{L}) = \sum_{i} r'_{i} = \half \bfrr_{1}(\ckcO_{g})
      \]
      and, if $(2i,2i+1)\notin \wp$,
      \[
        \begin{split}
          \bfrr_{i+1}(\cktau_{L}) & = \sum_{l\geq 2i+1} r'_{l}
          = \half\bfrr_{2i+1}(\ckcO)\\
          \bfrr_{i}(\cktau_{R}) & = \sum_{l\geq 2i} r'_{l} = \half\bfrr_{2i}(\ckcO)
        \end{split}
      \]
      if $(2i,2i+1)\in \wp$,
      \[
        \begin{split}
          \bfrr_{i+1}(\cktau_{L}) & = \sum_{l\geq 2i} r'_{l}
          = \half\bfrr_{2i}(\ckcO)\\
          \bfrr_{i}(\cktau_{R}) & = \sum_{l\geq 2i+1} r'_{l} = \half\bfrr_{2i+1}(\ckcO)
        \end{split}
      \]

      Some remarks on the BV-dual. The calculation of $\cO_{g}$ from
      $\tau_{\emptyset}$ can be reduced to the case of quasi-distinguished
      orbits (other case are deduced from this by parabolic induction,
      corresponds to attach two even columns for the balanced pairs). Compare
      Sommer's description of Springer correspondence with ours, we deduce that
      \[
        \cO_{g} = (\bfrr_{1}(\ckcO_{1})+1,\bfrr_{2}(\ckcO_{2})-1,\bfrr_{3}(\ckcO_{3})+1, \cdots, \bfrr_{2l}(\ckcO_{2l})-1,\bfrr_{2l+1}(\ckcO_{2l+1})+1)
      \]
      The rest parts are similar to that of type $D$.
    }


    We sketch the proof for the case when $\star = \wtC$.


    For a partition $\cO$, we set
    \[
      \sfS_{\cO} := \prod_{i\in \bN^{+}}\sfS_{\bfcc_{i}(\cO)}
    \]
    so that $j_{\sfS_{\cO}}^{\sfS_{\abs{\cO}}}\sgn = \cO$.


    We first consider the good parity (even) part.

    Now we consider the good parity parts, where each row of $\ckcO_{g}$ has
    even length.

    We set $r'_{i} := \half(\bfrr_{i}(\ckcO_{g})-\bfrr_{i-1}(\ckcO_{g}))$,
    $l =\min\set{i|\bfrr_{2i+1}(\ckcO_{g})=0}$, and write
    \[
      \ckcO_{g} % = [2r_{1}+1, 2r_{1}+1, \cdots, 2r_{k}+1,2r_{k}+1]
      % = (2c_{0},2c_{1},2c_{1}, \cdots, 2c_{l}, 2c_{l}).
      = ((2l)^{2r'_{2l}}, (2l-1)^{2r'_{2l-1}},\cdots, 1^{2r'_{1}})
    \]
    where $i^{r'}$ denotes $r'$-copies of length $i$ columns.
    % Consider
    % \[
    %   \cO_{g} = [2r_{1},2r_{2}, \cdots, 2r_{2k-1},2r_{2k}] = (C_{1},C_{1}, C_{2},C_{2},\cdots, C_{l}, C_{l}).
    % \]
    The Weyl group of good parity is $\sfW'_{n_{g}}$ with
    $n_{g} = \half\abs{\ckcO_{g}}$. For $1\leq k\leq l$, let
    \[
      % S_{r,s} = \prod_{i=r}^{s} \underbrace{\sfS_{i}\times \cdots\times \sfS_{i}}_{r'_{i}\text{-terms}}
      \vec{S}_{i} = \underbrace{\sfS_{i}\times \cdots\times \sfS_{i}}_{r'_{i}\text{-terms}} \AND n_{k} = \sum_{i=k}^{l} i\cdot r'_{i}.
      % \AND n_{r,s} = \sum_{i=r}^{s} i\cdot r'_{i}.
    \]

    Then $W_{\lamckg}=\prod_{i=1}^{l} \vec{S}_{i}$ and
    \[
      \begin{split}
        \ckLV_{\ckcO_{g}}& :=J_{W_{\lamckg}}^{\Wg}\sgn\\
        & = J_{\vec{S}_{1}\times \sfW'_{n_{2}}}^{W_{n_{1}}} \Big(\sgn \otimes J_{\vec{S}_{2}\times \sfW'_{n_{3}}}^{\sfW'_{n_{2}}}\Big(\sgn
        \otimes \cdots\big(J_{\vec{S}_{l}} \sgn\big)\cdots \Big)\Big) \\
      \end{split}
    \]
    Applying \cite{Lu}*{(4.6.2)} inductively, we see that the operation
    $J_{\vec{S}_{i}\times \sfW'_{n_{i+1}}}^{\sfW'_{n_{i}}}(\sgn \otimes \underline{\ \ \ })$
    doubles (resp. keeps) the number of irreducible components if $i$ is odd
    (resp. even).

    Suppose $\CPPs(\ckcO_{g}) = \emptyset$.
    We define
    \[
      \tA'(\ckcO):= \bF_{2}[\emptyset]
    \]
    to be the trivial group.
    Then $\ckLV_{\ckcO_{g}}$ is
    irreducible and marked by label $I$.  Hence, $\LC(\ckcO_{g})$,
    $\LC(\ckcO)$ and $\tA'(\ckcO)$ are naturally identified.


    Now assume $\CPPs(\ckcO_{g})\neq \emptyset$. Then the two parts of the
    bipartition of an irreducible component are different. Let
    $i_{0}:= \min\Set{i| (2i-1,2i)\in \CPPs(\ckcO_{g})}$.
    % $(2i_{0}-1,2i_{0})$ be the element in such that $i_{0}$ is minimal.
    Then we have a bijection
    \[
      \tA'(\ckcO):= \set{\wp\in \bF_{2}[\CPPs(\ckcO_{g})]|(2i_{0}-1,2i_{0})\notin \wp} \longrightarrow \ckLV_{\ckcO_{g}}
    \]
    to record the bifurcation when attaching odd length columns. Here we send
    $\wp$ to $\cktau_{\wp}:= (\cktau_{L},\cktau_{R})$ such that
    \[
      (\bfrr_{i}(\cktau_{L}),\bfrr_{i}(\cktau_{R})) := \begin{cases} (\half\bfrr_{2i}(\ckcO_{g}),\half\bfrr_{2i-1}(\ckcO_{g}))
        & \text{if } (2i-1,2i)\notin \wp\\
        (\half\bfrr_{2i-1}(\ckcO_{g}),\half\bfrr_{2i}(\ckcO_{g}))
        & \text{if } (2i-1,2i)\in \wp\\
      \end{cases}
    \]
    We obtain the structure of $\LC_{\ckcO_{g}}$ by tensoring the sign
    representation.

    \trivial[]{ Note that the trivial representation of the trivial group is
      represented by the symbol
      \[
        \binom{0,1, 2, \cdots, l-1}{0,1,2, \cdots, l-1}_{I}.
      \]
      % Induce to include all $2l$-length columns yields
      % \[
      %   \binom{r'_{2l+1}+0,r'_{2l+1}+1,r'_{2l+1}+2,\cdots, r'_{2l+1}+l\phantom{-1}}{ r'_{2l+1}+0,r'_{2l+1}+1, \cdots, r'_{2l+1}+l-1}.
      % \]

      Now move the the shorter columns. When even columns $(2i)^{2r'_{2i}}$
      occurs, it adds $(i)^{r'_{2i}}$ columns on the left and $(i)^{r'_{2i}}$ on
      the right. When odd columns $(2i-1)^{r'_{2i-1}}$ occur, the bifurcation
      happens: one can
      \begin{itemize}
        \item attach columns $(i-1)^{r'_{2i-1}}$ on the left and columns
              $(i)^{r'_{2i-1}}$ on the right, which corresponds to
              $(2i-1,2i)\neq \wp$, or
        \item attach columns $(i)^{r'_{2i-1}}$ on the left and columns
              $(i-1)^{r'_{2i-1}}$ on the right, which corresponds to
              $(2i-1,2i)\in \wp$.
      \end{itemize}
      Note that when we first encounter the longest odd column, we make the
      choice that the size of left part is larger than that of the right part.
      Now If $(2i-1,2i)\notin \wp$,
      \[
        \begin{split}
          \bfrr_{i}(\cktau_{L}) & = \sum_{l\geq 2i} r'_{l}
          = \half\bfrr_{2i}(\ckcO_{g})\\
          \bfrr_{i}(\cktau_{R}) & = \sum_{l\geq 2i-1} r'_{l} = \half\bfrr_{2i-1}(\ckcO_{g})
        \end{split}
      \]
      if $(2i-1,2i)\in \wp$,
      \[
        \begin{split}
          \bfrr_{i}(\cktau_{L}) & = \sum_{l\geq 2i-1} r'_{l}
          = \half\bfrr_{2i-1}(\ckcO_{g})\\
          \bfrr_{i}(\cktau_{R}) & = \sum_{l\geq 2i} r'_{l} = \half\bfrr_{2i}(\ckcO_{g})
        \end{split}
      \]
    }


    Now we consider the bad parity (odd) part. Suppose $\ckcO_{b}$ is nonempty
    such that
    \[
      \ckcO_{b} = (2c_{0},2c_{1}, 2c_{1}, 2c_{2},2c_{2}, \cdots, 2c_{k}, 2c_{k})
    \]
    where $2k+1=\bfrr_{1}(\ckcO_{b})$ and $2c_{i} = \bfcc_{2i+1}(\ckcO_{b})$.
    Now
    \[
      W_{\lamckb} = \sfW_{c_{0}} \times \sfS_{2c_{1}} \times \sfS_{2c_{2}}\times \cdots \times \sfS_{2c_{k}}
    \]
    and %$J_{W_{\lamckb}}^{W_{b}}\sgn$ is irreducible by
    \[
      \begin{split}
        \cktau_{b} &:= J_{W_{\lamckb}}^{W_{b}} \sgn
        = ((c_{1},c_{2},\cdots, c_{k}),(c_{0},c_{1}, \cdots, c_{l}))\\
        & = \big([\half(\bfrr_{2}(\ckcO)-1),\half(\bfrr_{4}(\ckcO)-1),\cdots, \half(\bfrr_{2c_{0}}(\ckcO)-1)],\\
        & \hspace{2em} [\half(\bfrr_{2}(\ckcO)+1),\half(\bfrr_{4}(\ckcO)+1),\cdots, \half(\bfrr_{2c_{0}}(\ckcO)+1)]\big)
      \end{split}
    \]
    is irreducible by \cite{Lu}*{(4.5.4)}. Tensoring with sign yields the
    formula of $\tau_{b}$. Moreover, by the fake degree formula (see
    \cite{Carter}*{p~376}), we have
    \[
      \tau_{b} = \cktau_{b}\otimes \sgn = j_{\sfS_{\cO'_{b}}}^{\sfW_{n_{b}}} \sgn.
    \]
    where
    $ \cO'_{b} := (\ckcO'_{b})^{t}:=(\bfrr_{2}(\ckcO),\bfrr_{4}(\ckcO),\cdots , \bfrr_{2c_{0}}(\ckcO))$.

    \medskip \def\ckfll{\check{\fll}}

    Now we sketch the proof of \eqref{eq:dBV.W}.

    Recall the definition of the metaplectic Barbasch-Vogan dual in
    \cite{BMSZ1}. The duality map commutes with parabolic induction: Suppose
    $\ckfll\subset \check \ckfgg$ is a parabolic subalgebra of $\ckfgg$ and
    $\fll$ is the corresponding parabolic subalgebra in $\fgg$, then
    \begin{equation}\label{eq:inddBV}
      \dBV(\ckcO) =  \Ind_{\fll}^{\fgg}(\dBV(\ckcO_{\ckfll}))
    \end{equation}
    for each nilpotent orbit $\ckcO$ in $\ckfgg$ such that
    $\ckcO_{\ckfll}:=\ckcO\cap \ckfll\neq \emptyset$. This is clear by reducing
    to the type $B$ case, see \cite{BMSZ1}*{Proposition~3.8}. By removing pairs
    of rows with the same lengths in $\ckcO$, we reduced to check the equality
    in the case when $\ckcO_{b}=\emptyset$ and
    $\bfrr_{2i-1}(\ckcO_{g})>\bfrr_{2i}(\ckcO_{g})$ for all $i$ such that
    $i\leq \bfcc_{1}(\ckcO_{g})$. In this case, both sides of \eqref{eq:dBV.W}
    can be easily computed directly, which equals to
    \[
      (\bfrr_{1}(\ckcO)-1, \bfrr_{2}(\ckcO)+1, \cdots,\bfrr_{2c-1}(\ckcO)-1,\bfrr_{2c}(\ckcO)+1)
    \]
    with $c = \min\set{i|\bfrr_{2i+1}(\ckcO)=0}$.

    % Now we compare the metaplectic dual defined in \cite{BMSZ1} with the Weyl
    % group representations.
    \trivial[]{ Compare Sommer's description of Springer correspondence we
      deduce that the RHS is
      \[
        \cO_{g} = (\bfrr_{1}(\ckcO_{1})-1,\bfrr_{2}(\ckcO_{2})+1,\bfrr_{3}(\ckcO_{3})+1, \cdots, \bfrr_{2l-1}(\ckcO_{2l-1})-1,\bfrr_{2l}(\ckcO_{2l})+1)
      \]
      The LHS is calculated by $((((\ckcO^{t})_{D})^{+})^{-})_{C}$. We write
      $R_{i}=\bfrr_{i}(\ckcO)=2r_{i}$. Now under our assumption,
      $R_{2i-1}>R_{2i}$, we have
      \[
        \begin{split}
          ((((\ckcO^{t})_{D})^{+})^{-})_{C} &=
          ((((R_{1},R_{2}, \cdots, R_{2l-1},R_{2l})_{D})^{+})^{-})_{C}\\
          &=((R_{1}-1,R_{2}, \cdots, R_{2l-1},R_{2l},1))_{C}\\
          &=(R_{1}-1,R_{2}+1, \cdots, R_{2l-1}-1,R_{2l}+1)\\
        \end{split}
      \]
      So the proof is done.

    }

    At last, one can see that
    $\dBV(\ckcO_{b}\cuprow \ckcO_{g}) = \ckcO_{b}^{t} \cupcol \dBV(\ckcO_{g})$
    using \eqref{eq:inddBV}.
  \end{proof}

  % \begin{remark}
  %   When $\star=\wtC$, one can see that
  %   $\dBV(\ckcO_{b}\cuprow \ckcO_{g}) = \ckcO_{b}^{t} \cupcol \dBV(\ckcO_{g})$
  %   using \eqref{eq:inddBV}. \trivial[]{ This could be checked using the
  %   formula of Springer correspondence directly, see Sommer's. }
  % \end{remark}
%
Since the representation theory of $\sfW_{n}$ is more elementary than that of
$\sfW'_{n}$, we prefer to induces every thing to $\sfW_{n}$.
We record the following simple lemma for the later use.

\begin{lem}\label{lem:WLcell}
  Suppose $\star=\wtC$. For any $\wp\in \tA(\ckcO)$, let
  $\wttau_{\wp} = (\imath_{\wp},\jmath_{\wp})$. Then
  \[
    \Ind_{\sfW_{n_{b}}\times \sfW'_{n_{g}}}^{\sfW_{n_{b}}\times \sfW_{n_{g}}} \tau_{b}\boxtimes \tau_{\wp} =
    \begin{cases}
      \tau_{b}\boxtimes \wttau_{\emptyset} & \text{if } \tA(\ckcO) =\emptyset,\\
      \tau_{b}\boxtimes \wttau_{\wp} \oplus \tau_{b}\boxtimes \wttau_{\wp^{c}}
      &\text{otherwise}.
    \end{cases}
  \]
Let
\[
  \tLV_{\ckcO}:= \Ind_{W_{b}\times W_{g}}^{\sfW_{n_{b}}\times \sfW_{n_{g}}}\LV_{\ckcO}
\]
and $\tLC_{\ckcO}$ be the set of irreducible components of the
$\sfW_{n_{b}}\times \sfW_{n_{g}}$-module $\tLV_{\ckcO}$. Then we have a
bijection
  \[
      \begin{array}{lccccccc}
        \tA(\ckcO)&=&\tA(\ckcO_{g}) & \longrightarrow & \tLC(\ckcO_{g})
        & \longrightarrow & \tLC(\ckcO)\\
                  &  &\wp & \mapsto & \wttau_{\wp}
        & \mapsto & \tau_{b}\otimes \wttau_{\wp}.
      \end{array}
  \]
  Suppose $\star\in\set{D,D^{*}}$. For any $\wp\in \tA(\ckcO)$, let
  $\wttau_{\wp} = (\imath_{\wp},\jmath_{\wp})$ and $\wttau_{b} = \Ind_{\sfW'_{n_{b}}}^{\sfW_{n_{b}}}\tau_{b}$.
  Then
  \[
    \Ind_{\sfW'_{n_{b}}\times \sfW'_{n_{g}}}^{\sfW_{n_{b}}\times \sfW_{n_{g}}} \tau_{b}\boxtimes \tau_{\wp} =
    \begin{cases}
      \wttau_{b} & \text{if } n_{g}=0\\
      \wttau_{b}\boxtimes \wttau_{\wp} \oplus \wttau_{b}\boxtimes \wttau_{\wp}^{s}
      &\text{otherwise}.
    \end{cases}
  \]
  where $\wttau_{\wp}^{s}:= \wttau_{\wp}\otimes \brsgn \neq \wttau_{\sP}$.

  % Then we have a bijection
  % \[
  %     \begin{array}{lccccccc}
  %       \tA(\ckcO)&=&\tA(\ckcO_{g}) & \longrightarrow & \tLC(\ckcO_{g})
  %       & \longrightarrow & \tLC(\ckcO)\\
  %                   &  &\wp & \mapsto & \wttau_{\wp} &
  %                                                    \mapsto & \tau_{b}\otimes \wttau_{\wp}.
  %     \end{array}
  % \]
\end{lem}
\begin{proof}
  The lemma follows immediately from the above lemma on the explicit
  descriptions of the left cells.
\end{proof}


\subsection{Coherent continuation representations}


% \subsubsection{Some subgroups of the Weyl groups}
Before we start to describe the coherent continuation representations we first
recall some subgroups of the Weyl group and the related branching rule.


In the following $a,b,c,d,n,p,q,r,s,t\in \bN$. We view $\sfW_{2t}$ and
$\sfS_{2t}$ as the reflection group acts on $\bC^{t}$ as usual. Let
$\sfH_{t} := \sfW_t\ltimes \set{\pm 1}^t$ be the subgroup in $\sfW_{2t}$ such
that
\begin{itemize}
  \item the first factor $\sfW_{t}$ sits in $\sfS_{2t}$ commuting with the
        involution
        \[
        (12)(34)\cdots ((2t-1)(2t)).
        \]
  \item The element $(1,\cdots,1, \underbrace{-1}_{i\text{-th
        term}}, 1, \cdots, 1)\in \set{\pm 1}^{t}$ acts on $\bC^{2t}$ by
        \[
          % (x_{1},\cdots, x_{2i-2}, x_{2i-1}, x_{2i},x_{2i+1},\cdots, x_{2t} )
        (x_{1},x_{2},\cdots, x_{2t} ) \mapsto (x_{1},\cdots, x_{2i-2}, -x_{2i},-x_{2i-1},x_{2i+1},\cdots, x_{2t}).
        \]
\end{itemize}
Note that $\sfH_{t}$ is also a subgroup of $\sfW'_{2t}$. Define the quadratic
character
\[
  \begin{array}{rccc}
    \hsgn := 1\otimes \sgn\colon & \sfH_{t}=  \sfW_{t}\ltimes \set{\pm 1}^{t}& \longrightarrow & \set{\pm 1}\\
                                 & (g,(a_{1},a_{2},\cdots, a_{t})) & \mapsto & a_{1}a_{2}\cdots a_{t}.
  \end{array}
\]
The most important formulas are
\begin{equation}\label{eq:CC.C}
  \Ind_{\sfH_{t}}^{\sfW_{2t}} \hsgn = \sum_{\sigma\in \Irr(\sfS_{t})} (\sigma,\sigma)
  \AND
  \Ind_{\sfH_{t}}^{\sfW'_{2t}} \hsgn = \sum_{\sigma\in \Irr(\sfS_{t})} (\sigma,\sigma)_{I},
\end{equation}
see \cite{Mc}*{p220 (6)}.

We denote $\bsgn$ the inflation of the sign representation of $\sfS_{n}$ to
$\sfW_{n}$. Then
\[
  \Ind_{\sfS_{n}}^{\sfW_{n}}\sgn = \bigoplus_{a+b=n}\Ind_{W_{a}\times W_{b}}^{W_{n}} \bsgn\boxtimes \sgn =\bigoplus_{a+b=n} ((a,),(b,)) .
\]
Let $\brsgn := \sgn \otimes \bsgn$ which is the quadratic character of
$\sfW_{n}=\sfS_{n}\ltimes \set{\pm 1}^{n}$ given by
\[
  (s,(x_{1}, x_{2}, \cdots, x_{n}))\mapsto x_{1}x_{2}\cdots x_{n}.
\]
\trivial{In fact, $\brsgn$ is the unique non-trivial quadratic character of
  $\sfW_{n}=\sfS_{n}\ltimes \set{\pm 1}^{n}$ which is trivial on $\sfS_{n}$. }
Then
\[
  \Ind_{\sfS_{n}}^{\sfW_{n}} 1 = \bigoplus_{a+b=n}\Ind_{W_{a}\times W_{b}}^{W_{n}} 1 \boxtimes \brsgn =\bigoplus_{a+b=n} ([a,],[b,]) .
\]
% Note that we have chosen the embedding $W_{t}\subset S_{2t}$ in $W_{2t}$. We
% have
% \[
%   \Ind_{H_{t}}^{W'_{2t}} \hsgn = \sum_{\sigma\in \Irr(S_{t})} (\sigma,\sigma)_{I}.
% \]

\trivial{ % In McGovern's paper, the coherent continuation representation is
  % described as:
  % \[
  %   \sum_{t,s,a,b}\Ind_{W_t\times (W_s\ltimes W(A_1)^s)\times W_a\times W_b}^{W_{t+2s+a+b}} \sgn\otimes (\triv \otimes \sgn)\otimes \triv\otimes \triv
  % \]
  Now \eqref{eq:CC.C} was obtained by the following branching formula:
  \cite[p220 (6)]{Mc}
  \[
    I_n:= \Ind_{(W_s\ltimes W(A_1)^s)}^{W_{2s}}\triv\otimes \sgn = \sum \lambda\times \lambda
  \]
  where $\lambda$ running over all Young diagrams of size $s$. As McGovern
  claimed the proof of the above formula is similar to Barbasch's proof of
  \cite[Lemma~4.1]{BV.W}:
  \[
    \Ind_{W_n}^{S_{2n}} \triv = \sum \sigma \quad \text{where $\sigma$ has even
      rows only}.
  \]

  Sketch of the proof (use branching rule and dimension counting): Note that
  $\dim I_n = \frac{(2p)! 2^{2p}}{p! 2^{2p}} = (2p)!/p! =\sum_\lambda \dim \lambda\times \lambda$
  (For the last equality:
  $\dim \lambda\times \lambda = (2p)! (\dim \lambda)^2/(p!)^2$ where
  $\dim \lambda$ is the dimension of $S_n$ representation determined by
  $\lambda$; But $\sum (\dim \lambda)^2 = p!$). On the other hand,
  $H :=W_s\ltimes W(A_1)^s\cap W_s\times W_s = \Delta W_s \subset W_{2s}$.
  $\triv \otimes \sgn|_H = \sgn$ of $\Delta W_s$ Therefore,
  $\lambda\times \lambda$ appears in $I_n$ by Mackey formula. Now by dimension
  counting, we get the formula. }

\medskip

We now define various Weyl representations case by case. They will be used to
state the formula of coherent continuation representations.
% In the following $\sigma$ is running over all irreducible representations of
% $\sfS_{t}$.
\begin{itemize}
  \item Suppose $\star= B$, $p+q=2m+1$ is odd. Define
        \[
        \begin{split}
          \cC_b^{n} & :=\bigoplus_{\substack{2t+c+d=n}} \Ind_{\sfH_{t} \times \sfW_{c}\times \sfW_{d}}^{\sfW_{n}}
          \hsgn\boxtimes 1\boxtimes 1 \\
          \cC_g^{p,q} &:=\bigoplus_{\substack{0\leq p-(2t+a+2r)\leq 1\\0\leq q - (2t+a+2s)\leq 1}} \Ind_{\sfH_{t} \times \sfS_{a}\times \sfW_s\times \sfW_r}^{\sfW_{n}}
          \hsgn \boxtimes 1 \boxtimes \sgn \boxtimes \sgn \\
          % &\cong \bigoplus_{\substack{0\leq p-(2t+c+d+2r)\leq 1\\0\leq q - (2t+c+d+2s)\leq 1}} \bigoplus_{\sigma} \Ind_{\sfW_{2t} \times \sfW_{c}\times \sfW_{d}\times \sfW_s\times \sfW_r}^{\sfW_{n}}
          % (\sigma,\sigma) \boxtimes 1 \boxtimes \brsgn \boxtimes \sgn \boxtimes \sgn \\
        \end{split}
        \]
        \trivial[]{ Here is a point which could cases confusion: Although the
        real Weyl group is
        $\sfH_{t}\times \sfW_{a}\times \sfW_{s}\times \sfW_{r}$, the cross
        stabilizer is much smaller
        $=\sfH_{t}\times \sfS_{a}\times \sfW_{s}\times \sfW_{r}$! This is dual
        to the fact that, for the split Cartan and real root $e_{i}$,
        $\sgn\circ e_{i}\colon H\rightarrow \bR^{\times}\rightarrow \set{\pm 1}$
        is non-trivial! The good infinitesimal character takes half-integer
        values. So $s_{e_{i}}$ never cross stabilizing a regular character at
        these infinitesimal characters }
  \item Suppose $\star=C^{*}$. Define
        \[
        \begin{split}
          \cC_b^{n} & :=
          \begin{cases}
            % \Res_{\sfW_{n}}^{\sfW'_{n}}
            \Ind_{\sfH_{t}}^{\sfW_{n}} \hsgn &
            \text{if $n=2t$ is even} \\
            0 & \text{otherwise}\\
          \end{cases}\\
          \cC_g^{2p,2q} %& = \bigoplus_{p+q=m} \Cint{\rho}(\Sp(p,q)) \\
          &:=\bigoplus_{\substack{(t+s,t+r)=(p,q)}} \Ind_{\sfH_{t} \times \sfW_s\times \sfW_t}^{\sfW_{p+q}}
          \hsgn \otimes \sgn \otimes \sgn \\
          % & =\bigoplus_{\substack{(t+s,t+r)=(p,q)}} \bigoplus_{\sigma\in \Irr(\sfS_{t})} \Ind_{\sfW_{2t}\times \sfW_s\times \sfW_r}^{\sfW_{p+q}}
          % (\sigma,\sigma)\otimes \sgn \otimes \sgn \\
        \end{split}
        \]
  \item Suppose $\star=C$. Define
        \[
        \begin{split}
          \cC_b^{n} &
          :=\bigoplus_{\substack{2t+a=n}} %\Res_{\sfW_{n}}^{\sfW'_{n}} \left(
          \Ind_{\sfH_{t} \times \sfS_a}^{\sfW_{n}} \hsgn\otimes 1 %\right)
          \\
          \cC_g^{n,n} &:= \bigoplus_{2t+a+c+d=n}\Ind_{\sfH_{t} \times \sfS_{a} \times \sfW_c\times \sfW_d}^{W_{n}} \hsgn \otimes
          \sgn \otimes 1 \otimes 1\\
          % & =\bigoplus_{\substack{t+r+s+c+d=n}} \bigoplus_{\sigma } \Ind_{\sfW_{2t}\times \sfW_s\times \sfW_r\times \sfW_{c}\times \sfW_{d}}^{\sfW_{n_{g}}}
          % (\sigma,\sigma)\otimes \sgn \otimes \bsgn \otimes 1\otimes 1 \\
        \end{split}
        \]
  \item Suppose $\star=\wtC$. Define
        \[
        \begin{split}
          \cC_b^{n} &
          :=\bigoplus_{\substack{2t+c+d=n}} %\Res_{\sfW_{n}}^{\sfW'_{n}} \left(
          \Ind_{\sfH_{t} \times \sfW_c\times \sfW_{d}}^{\sfW_{n}} \hsgn\boxtimes 1 \boxtimes
          1 %\right)
          \\
          \cC_g^{n,n} &:= \bigoplus_{2t+a+a'=n}\Ind_{\sfH_{t} \times \sfS_{a} \times \sfS_{a'}}^{\sfW_{n}} \hsgn \otimes
          \sgn \otimes 1 \\
          % & =\bigoplus_{\substack{t+r+s+c+d=n}} \bigoplus_{\sigma } \Ind_{\sfW_{2t}\times \sfW_s\times \sfW_r\times \sfW_{c}\times \sfW_{d}}^{\sfW_{n_{g}}}
          % (\sigma,\sigma)\otimes \sgn \otimes \bsgn \otimes 1\otimes 1 \\
        \end{split}
        \]
  \item Suppose $\star=D$ and $p+q=2m$ is even. Define
        \[
        \begin{split}
          \cC_b^{n} & := \bigoplus_{\substack{2t+a=n}}
          \Ind_{\sfH_{t}\times \sfS_{a}}^{\sfW_{n}} \hsgn\otimes 1\\
          \cC_g^{p,q} %& = \bigoplus_{p+q=m} \Cint{\rho}(\Sp(p,q)) \\
          & := \bigoplus_{\substack{2t+c+d+2r=p\\2t+c+d+2s=q}}
          % \Res_{\sfW_{m}}^{\sfW'_{m}}\left(
          \Ind_{\sfH_{t} \times \sfW_s\times \sfW_r\times \sfW'_{c}\times \sfW_{d} }^{\sfW_{(p+q)/2}} \hsgn \otimes \bsgn \otimes \bsgn \otimes 1\otimes
          1 %\right)
          \\
        \end{split}
        \]
  \item Suppose $\star=D^{*}$. Define
        \[
        \begin{split}
          \cC_b^{n} & :=
          \begin{cases}
            \Ind_{\sfH_{t}}^{\sfW'_{n}} \hsgn &
            \text{if $n=2t$ is even} \\
            0 & \text{otherwise}\\
          \end{cases}\\
          \cC_g^{n,n} %& = \bigoplus_{p+q=m} \Cint{\rho}(\Sp(p,q)) \\
          &:=\bigoplus_{\substack{2t+a=n}} \Ind_{\sfH_{t} \times \sfS_{a}}^{\sfW'_{n}}
          \hsgn \otimes \sgn \\
        \end{split}
        \]
\end{itemize}

Now assume $\rank_{\bC} \Gc = n$. We identify $\fhh^{*}$ with $\bC^{n}$. Let $Q$
be the root lattice in $\fhh^{*}$ which is
\[
  Q = \begin{cases}
    \bZ^{n} & \text{if  $\star = B$}\\
    % \set{(a_{1},a_{2},\cdots, a_{n})\in \bZ^{n}|\sum_{i=1}^{n}a_{i} \in 2\bZ}
    \Set{(a_{i})\in \bZ^{n}|\sum_{i=1}^{n}a_{i} \text{ is even}}
    & \text{if  $\star \in \set{C,\wtC,C^{*},D,D^{*}}$}\\
  \end{cases}
\]
For $n_{b}, n_{g}\in \bN$ such that $n_{b}+n_{g}=n$, we consider the lattice
\[
  \Lambda_{n_{b},n_{g}} =
  \begin{cases}
    (\underbrace{\half, \cdots, \half}_{n_{b}\text{-terms}}, \underbrace{0, \cdots, 0}_{n_{g}\text{-terms}}) + Q & \text{when
    } \star\in \set{C,C^{*}, D,D^{*}}\\%\subset \fhh^{*}.
    (\underbrace{0, \cdots, 0}_{n_{b}\text{-terms}}, \underbrace{\half, \cdots, \half}_{n_{g}\text{-terms}}) + Q & \text{when
    } \star\in \set{B,\wtC}. %\subset \fhh^{*}.
  \end{cases}
\]
Clearly,
\[
  W_{\Lambda_{\nbb,\ngg}}:= \set{w| w\cdot \Lambda_{\nbb,\ngg} = \Lambda_{\nbb,\ngg}} = W_{b}\times W_{g}
\]
with $W_{b}$ and $W_{g}$ defined by \eqref{eq:Wbg}.


We define
\[
  \Sign(G):= \begin{cases}
    (p,q) & \text{if } G = \SO(p,q)\\
    (n,n) & \text{if } G = \Sp(2n,\bR) \text{ or } \Mp(2n,\bR)\\
    (2p,2q) & \text{if } G = \Sp(p,q) \\
    (n,n) & \text{if } G = \rO^{*}(2n) \\
  \end{cases}
\]

\begin{prop}
  As $W_{\Lambda_{n_{b},n_{g}}}:= W_{b}\times W_{g}$-module, the coherent
  continuation representation $\Coh_{\Lambda_{n_{b},n_{g}}}(G)$ is isomorphic to
  the restriction to $W_{\Lambda_{n_{b},n_{g}}}$ of
  \[
    \begin{cases}
      \cC_{b}^{n_{b}}\boxtimes\cC_{g}^{p,q} & \text{if } \star \in \set{B,C,\wtC,C^{*},D}\\
      \Ind_{\sfW'_{n_{b}}\times \sfW'_{n_{g}}}^{W''}(\cC_{b}^{n_{b}}\boxtimes\cC_{g}^{p,q}) & \text{if } \star D^{*}\\
    \end{cases}
  \]
  with $(p,q) = \Sign(G) - (n_{b},n_{b})$ and
  \[
    W'' := \left(\sfW_{n_{b}}\times \sfW_{n_{g}}\right)\cap \sfW'_{n_{g}+n_{b}} \quad \text{when
      $\star = D^{*}$.}
  \]


\end{prop}
\begin{remark}
  When $\star \in \set{C^{*},D^{*}}$ and $n_{b}$ is odd,
  $\Coh_{\Lambda_{n_{b},n_{g}}}(G)=0$ by the proposition.
\end{remark}
\begin{proof}[Sketch of the proof]
  When $G$ is linear, the set of regular characters can be enumerated using
  \cite{AC}, then the follows follows from \Cref{thm:cohHC}, see also
  \cite{Mc}*{Applications}. When $G$ is the metaplectic group, the enumeration
  of parameters is contained in Renard-Trapa's work \cite{RT1,RT2}.
  % We give a sketch of the argument.
\end{proof}

We define
\[
  G'_{n} := \begin{cases}
    \GL(n,\bR) & \text{when } \star \in \set{B,C,\wtC,D}, \\
    \GL(\half n,\bH) & \text{when } \star \in \set{C^{*},D^{*}}. \\
  \end{cases}
\]


We make the following definitions:

\begin{defn}
  Let $\PBPsb(\ckcOb)$ be the set of all pairs
  $\uptau = (\imath,\cP)\times(\jmath,\cQ)$ where $(\imath,\cP)$ and
  $(\jmath,\cP)$ are painted partitions such that
  \begin{itemize}
    \item $(\imath,\jmath) = \tau_{b}$ (see \eqref{eq:taub});
    \item the image of $\cP$ is contained in
          \[
          \begin{cases}
            \set{\bullet, c,d}  & \text{if } \star\in \set{B,\wtC} \\
            \set{\bullet, d}  & \text{if } \star\in \set{C,D}\\
            \set{\bullet}  & \text{if } \star\in \set{C^{*},D^{*}}\\
          \end{cases}
          \]
    \item the image of $\cQ$ is contained in

          \[
          \begin{cases}
            \set{\bullet, c}  & \text{if } \star\in \set{C,D}\\
            \set{\bullet}  & \text{if } \star\in \set{B,\wtC, C^{*},D^{*}}\\
          \end{cases}
          \]
  \end{itemize}
\end{defn}

To ease the notation, for each bipartition $\tau$, let
\[
  \PBP_{\star}(\tau) := \Set{ \uptau|\uptau \text{ is a painted partition and
    } \star_{\uptau}=\star, (\imath_{\uptau},\jmath_{\uptau}) = \tau}
  % \uptau=(\imath, \cP)\times (\jmath,\cP)\times \alpha|}
\]
and
\[
  \tPBP_{\star}(\ckcOg) := %\bigsqcup_{\tau\in\LC(\ckcOg)}\PBP_{\star}(\tau).
  \bigsqcup_{\wp \subseteq \CPP(\ckcOg)}\PBP_{\star}(\wttau_{\wp})
\]
where $\wttau_{\wp} := (\imath_{\wp},\jmath_{\wp})$. Similarly, define
\[
  \tPBP_{\Gg}(\ast):= \Set{\uptau\in \tPBP_{\star}(\ast )|\Sign(\uptau)= \Sign(\Gg)}  \quad
  \ast  = \ckcOg \text{ or } \tau.
\]



\begin{prop}\label{prop:BP.PP}
  In all the cases,
  \[
    \PBP_{\star,b}(\ckcO_{b}) = \PP_{G'}(\ckcO'_{b}) =\Unip_{G'}(\ckcO'_{b}).
  \]
\end{prop}
\begin{proof}
  Suppose $\star \in \Set{C^{*},D^{*}}$. Then
  \[
    \begin{split}
      \abs{\PBP_{\star,b}(\ckcO_{b})} = \abs{\PP_{G'}(\ckcO'_{b})} = \abs{\Unip_{\ckcO'_{b}}(G'_{n_{b}})} = 1.
    \end{split}
  \]
  Suppose $\star \in \Set{B,C,\wtC,D}$, $\tau_{b} = (\tau_{L,b},\tau_{R,b})$ and
  $\tau'_{b} = \cO'_{b}$ It is easy to see that we have a bijection:
  \[
    \begin{array}{ccc}
      \PBP_{\star,b}(\ckcO_{b}) &  \longrightarrow & \PP_{A^{\bR}}(\ckcO'_{b})\\
      (\tau_{L,b},\cP)\times (\tau_{R,b},\cQ)& \mapsto & (\cOpb,\cP')
    \end{array}
  \]
  where $\cP'$ is defined by the condition that
  \[
    \cP(\bfcc_{j}(\tau_{L,b}),j)=d \Longleftrightarrow \cP'(\bfcc_{j}(\cOpb),j)=d \quad \forall j=1,2,\cdots, \bfrr_{1}(\cOpb).
  \]
  \trivial[]{
    % Let $\tau' = \ckcO'^{t}_{b}$ and $\tau_{b}=(\tau_{L,b}, \tau_{R,b})$. Here
    % $\tau_{L,b}, \tau_{R,b}$.
    Now the claim follows for the fact that the bottom rows in $\uptau_{L}$ can
    be filled by $\bullet/c$ or $d$ and
    \[
      \bfcc_{i}(\tau_{L,b}) = \bfcc_{j}(\tau_{L,b}) \Leftrightarrow \bfcc_{i}(\cOpb) = \bfcc_{j}(\cOpb) \quad \forall i,j\in \bN^{+}.
    \]
  }

\end{proof}

\begin{prop}
  In all the cases, we have
  % \[
  %   [\tau_{b}: \cC_{b}] = \PBP_{\star,b}(\ckcO_{b}) = \PBP_{G'}(\ckcO'_{n_{b}}) = \Unip_{\ckcO'_{b}}(G'_{n_{b}})
  % \]
  % and
  % \[
  %   \sum_{\tau\in \LC_{\ckcO_{g}}} [\tau:\cC_{g}] = \PBP_{G}(\ckcO_{g}).
  % \]
  \[
    \sum_{\sP\in \tA'(\ckcO)} [\tau_{b}\otimes \tau_{\sP}: \Coh_{\Lambda_{n_{b},n_{g}}}(G)] = \abs{\PBP_{\star,b}(\ckcO_{b})}\cdot \abs{\tPBP_{G_{g}}(\ckcO_{g})}.
  \]
\end{prop}
\begin{proof}
  One can compute the formula using the branching rule of Weyl group
  representations using \Cref{lem:WLcell}. We leave the details to the reader
  when $\star\in \set{B,C,C^{*},D,D^{*}}$.

  We now present the computation for $\star = D^{*}$ to demonstrate the ideas
  (this is the most complicate case).

  Recall that $(W_{b},W_{g}) = (\sfW'_{\nbb},\sfW'_{\ngg})$.

  Suppose $\ngg = 0$ first. Then $\tau_{b} = (\cOpb,\cOpb)_{I}$ and
  \[
    \begin{split}
      [\tau_{b}:\cC_{b}^{\nbb}]_{W_{b}} = &
      [\tau_{b}: \Ind_{\sfH_{\frac{\nbb}{2}}}^{\sfW'_{\nbb}}\tsgn]\\
      = & [(\cOpb,\cOpb)_{I}: \bigoplus_{\sigma}(\sigma,\sigma)_{I}]\\
      = & 1.
    \end{split}
  \]

  Now we assume that $\ngg>0$. Let
  \[
    \wttau_{b} := \Ind_{\sfW'_{\nbb}}^{\sfW_{\nbb}} \tau_{b} = (\cOpb,\cOpb) \AND \wttau_{\wp}: = (\imath_{\wp},\jmath_{\wp}) \quad \forall \wp \subseteq \CPP(\ckcOg).
  \]
  Note that $\imath_{\wp}\neq \jmath_{\wp}$ since
  $\bfcc_{1}(\imath_{\wp})\geq \bfcc_{1}(\jmath_{\wp})$. Therefore, one can see
  that
  \begin{equation}\label{eq:W''}
    \Ind_{W_{b}\times W_{g}}^{W''} \tau_{b}\boxtimes \tau_{\wp}
    = (\wttau_{b}\otimes \wttau_{\wp})|_{W''}.
  \end{equation}

  \trivial[]{ When $\nbb=0$, $W'' = W_{g} = \sfW'_{\ngg}$ and so
    $\wttau_{\wp}|_{\sfW'_{\ngg}} = \tau_{\wp}$.

    Now we we assume $\nbb\neq 0$ and $\ngg\neq 0$ (the general case). This
    follows from the following points
    \begin{itemize}
      \item the dimension of the two sides are equal ($W_{b}\times W_{g}$ has
            index $2$ in $W''$).
      \item
            \[
            \begin{split}
              &[\Ind_{W_{b}\times W_{g}}^{W''}\tau_{b}\boxtimes \tau_{\wp}:(\wttau_{b}\otimes \wttau_{\wp})|_{W''}] \\
              =& [\Ind_{\sfW'_{n_{b}}\times \sfW'_{n_{g}}}^{\sfW_{n_{b}}\times \sfW_{n_{g}}}\tau_{b}\boxtimes \tau_{\wp}:\wttau_{b}\otimes \wttau_{\wp}]\\
              =& [\wttau_{b}\boxtimes \wttau_{\wp} \oplus \wttau_{b}\boxtimes (\wttau_{\wp}\otimes \brsgn):\wttau_{b}\otimes \wttau_{\wp}] =1\\
            \end{split}
            \]
            where $\wttau_{\wp}\otimes \brsgn$ has the bipartition obtained by
            switching the left and right side of $\wttau_{\wp}$.
      \item the LHS is irreducible, by
            \[
            \begin{split}
              & [\Ind_{W_{b}\times W_{g}}^{W''}\tau_{b}\boxtimes \tau_{\wp}:
              \Ind_{W_{b}\times W_{g}}^{W''}\tau_{b}\boxtimes \tau_{\wp}]_{W''}\\
              =&  [\tau_{b}\boxtimes \tau_{\wp} : (\Ind_{W_{b}\times W_{g}}^{W''}\tau_{b}\boxtimes \tau_{\wp})|_{W_{b}\times W_{g}}]\\
              =& [\tau_{b}\boxtimes \tau_{\wp} : \tau_{b}\boxtimes \tau_{\wp} + (\cOpb,\cOpb)_{II} \boxtimes \tau_{\wp} ] = 1
            \end{split}
            \]
    \end{itemize}
  }

  Now we have
  \[
    \begin{split}
      & [\tau_{b}\boxtimes \tau_{\wp} :
      \Ind_{\sfW'_{n_{b}}\times \sfW'_{n_{g}}}^{W''} \cC_{b}^{\nbb} \boxtimes \cC_{g}^{\ngg}]_{\sfW'_{n_{b}}\times \sfW'_{n_{g}}}\\
      = & [\Ind_{W_{b}\times W_{g}}^{W''} \tau_{b}\boxtimes \wttau_{\wp} :
      \Ind_{W_{b}\times W_{g}}^{W''} \cC_{b}^{\nbb} \boxtimes \cC_{g}^{\ngg}]_{W''}\\
      = & [(\wttau_{b}\boxtimes \wttau_{\wp})|_{W''}:
      \Ind_{W_{b}\times W_{g}}^{W''} \cC_{b} \boxtimes \cC_{g}]_{W''}\\
      = & [\wttau_{b}\boxtimes \wttau_{\wp}:
      \Ind_{W_{b}\times W_{g}}^{\sfW_{n_{b}}\times \sfW_{n_{g}}} \cC_{b} \boxtimes \cC_{g}]_{\sfW_{n_{g}}\times \sfW_{n_{b}}}\\
      =& \# \PBP_{\star}(\ckcO_{b})\cdot \# \PBP_{\star}(\wttau_{\wp})
    \end{split}
  \]
  The last equality follows from the branching rules of $\sfW_{n}$ where the
  contribution of $\sfH_{t}$ corresponding to ``$\bullet$'',
  the contribution of $\sfS_{a}$ corresponding to the ``$s$'' on the left and
  ``$r$'' on the right.


  % Suppose $n_{b}=0$ then
  % \[
  %   \begin{split}
  %     [\tau_{\wp} : \cC_{g}]_{\sfW'_{n_{g}}} = &
  %     [\wttau_{\wp}|_{W_{g}}:\sum_{2t+a=n_{g}} \Ind_{\sfH_{t}\times \sfS_{a}}^{\sfW'_{n_{g}}}\tsgn\otimes \sgn]_{\sfW'_{n_{g}}}\\
  %     = & [\wttau_{\wp}: \sum_{2t+a=n_{g}} \Ind_{\sfH_{t}\times \sfS_{a}}^{\sfW'_{n_{g}}}\tsgn\otimes \sgn]_{\sfW_{n_{g}}}\\
  %     = & \PBP_{\star}(\ttau_\wp)
  %   \end{split}
  % \]


  \trivial[]{

    Suppose $\star =C^{*}$.

    For the bad parity, $n_{b}=2n'_{b}$ must be even.
    \[
      \begin{split}
        & [\tau_{b}\boxtimes \tau_{\wp} :
        \cC_{b}^{\nbb} \boxtimes \cC_{g}^{\ngg,\ngg}]_{\sfW'_{n_{b}}\times \sfW_{n_{g}}}\\
        = & [\Ind_{\sfW'_{n_{b}}\times \sfW_{n_{g}}}^{\sfW_{n_{b}}\times \sfW_{n_{g}}} \tau_{b}\boxtimes \tau_{\wp} :
        \cC_{b}^{\nbb} \boxtimes \cC_{g}^{\ngg,\ngg}]_{\sfW_{n_{b}}\times \sfW_{n_{g}}}\\
        = & [\wttau_{b}\boxtimes \tau_{\wp}:
        \cC_{b}^{\nbb} \boxtimes \cC_{g}^{\ngg,\ngg}]_{\sfW_{n_{b}}\times \sfW_{n_{g}}}\\
        =& \# \PBP_{\star}(\ckcO_{b})\cdot \# \PBP_{\star}(\tau_{\wp})
      \end{split}
    \]

    The alternative approach using restriction. For the bad parity,
    $n_{b}=2n'_{b}$ must be even.
    \[
      \begin{split}
        \cC_b^{n_{b}} &=
        \Res_{\sfW_{n_{b}}}^{\sfW'_{n_{b}}} \Ind_{\sfH_{n'_{b}}}^{\sfW_{n_{b}}} \hsgn \\
        &= \bigoplus_{\sigma\in \Irr(\sfS_{n'_{b}})} \left((\sigma,\sigma)_{I} \oplus (\sigma,\sigma)_{II}\right).
      \end{split}
    \]
    For the good parity,
    \[
      \begin{split}
        \cC_g^{2p,2q} %& = \bigoplus_{p+q=m} \Cint{\rho}(\Sp(p,q)) \\
        & =\bigoplus_{\substack{(t+s,t+r)=(p,q)}} \bigoplus_{\sigma} \Ind_{\sfW_{2t}\times \sfW_s\times \sfW_r}^{\sfW_{p+q}}
        (\sigma,\sigma)\otimes \sgn \otimes \sgn \\
      \end{split}
    \]
    Now the branching rule implies the irreducible components of
    $\cC_{g}^{2p,2q}$ are given by the dot-diagram attaching two columns on the
    right, which we mark them by $s$ and $r$ respectively. }


  % Suppose $\star=C$
  % \[
  %   \begin{split}
  %     & [\tau_{b}\boxtimes \tau_{\wp} :
  %     \cC_{b}^{\nbb} \boxtimes \cC_{g}^{\ngg,\ngg}]_{\sfW'_{n_{b}}\times \sfW_{n_{g}}}\\
  %     = & [\Ind_{\sfW'_{n_{b}}\times \sfW_{n_{g}}}^{\sfW_{n_{b}}\times \sfW_{n_{g}}} \tau_{b}\boxtimes \tau_{\wp} :
  %     \cC_{b} \boxtimes \cC_{g}]_{\sfW_{n_{b}}\times \sfW_{n_{g}}}\\
  %     = & [\wttau_{b}\boxtimes \tau_{\wp}:
  %     \cC_{b} \boxtimes \cC_{g}]_{\sfW_{n_{b}}\times \sfW_{n_{g}}}\\
  %     =& \# \PBP_{\star}(\ckcO_{b})\cdot \# \PBP_{\star}(\ckcO_{g};\wp)
  %   \end{split}
  % \]
%
  % For the bad parity, $n_{b}=2n'_{b}$ must be even.
  % \[
  %   \begin{split}
  %     \cC_b^{n_{b}} &= \Res_{\sfW_{n_{b}}}^{\sfW'_{n_{b}}} \left( \bigoplus_{2t+a=n_{b}}\Ind_{\sfH_{t}\times \sfS_{a}}^{\sfW_{n_{b}}} \hsgn \otimes 1
  %     \right)\\
  %     &= \bigoplus_{2t+c+d=n_{b}}\bigoplus_{\sigma\in \Irr(\sfS_{t})} \left((\sigma,\sigma)_{I} \oplus (\sigma,\sigma)_{II}\right) \times ([d,],[c,]).
  %   \end{split}
  % \]
  % Note that $\tau_{b}=(\cO'_{b},\cO'_{b})_{I}$. Therefore we only need to
  % consider the case when $c=d$ in the above formula.
  % \[
  %   \begin{split}
  %     [\tau_{b}: \cC_{b}^{n_{b}}] & = [\cO'_{b}:\bigoplus_{\substack{t+d = n'_{b}\\ \sigma}} \sigma \times 1].
  %   \end{split}
  % \]
  % By the counting of unipotent representation of $\GL(n'_{b},\bR)$. We see
  % that $\PBP_{\star,b}(\ckcO_{b})$ is identified with
  % $\PBP_{A^{\bR}}(\ckcO'_{b})$ by send $(\uptau_{L},\uptau_{R})$ to the
  % painted partition $\uptau'$ such that
  % \[
  %   \uptau_{L}(i,j)=d \Leftrightarrow \uptau'(i,j)=d.
  % \]

  % For the good parity, this is clear by the branching rules.


  \trivial{


    Suppose $\star = D$. Suppose that $n_{g}\neq 0$. Since
    $\bfcc_{1}(\imath_{\wp})>\bfcc_{1}(\jmath_{\wp})$, we have
    $\wttau^{s}_{\wp}:=\wttau_{\wp}\otimes \brsgn\ncong\wttau_{\wp}$.
    \[
      \begin{split}
        & [\tau_{b}\boxtimes \tau_{\wp} :
        \cC_{b} \boxtimes \cC_{g}]_{\sfW'_{n_{b}}\times \sfW'_{n_{g}}}\\
        = & [\Ind_{\sfW'_{n_{b}}\times \sfW'_{n_{g}}}^{\sfW_{n_{b}}\times \sfW_{n_{g}}} \tau_{b}\boxtimes \tau_{\wp} :
        \cC_{b} \boxtimes \cC_{g}]_{\sfW_{n_{b}}\times \sfW_{n_{g}}}\\
        = & [\wttau_{b}\boxtimes \wttau_{\wp}\oplus \wttau_{b}\boxtimes \wttau_{\wp}^{s}:
        \cC_{b} \boxtimes \cC_{g}]_{W''}\\
        = & [\wttau_{b}\boxtimes \wttau_{\wp}:
        \cC_{b} \boxtimes \cC_{g}]_{\sfW_{n_{g}}\times \sfW_{n_{b}}}\\
        =& \# \PBP_{\star}(\ckcO_{b})\cdot \# \PBP_{\star}(\ckcO_{g};\wp)
      \end{split}
    \]
    The terms involving $\wttau_{\sP}^{s}$ vanish since every irreducible
    component $(\sigma_{L},\sigma_{R})$ in $\cC_{g}$ satisfies
    $\sigma_{L}\supseteq \sigma_{R}$ but
    $\bfcc_{1}(\imath_{\wp})>\bfcc_{1}(\jmath_{\wp})$.

    Suppose that $n_{g} = 0$.
    \[
      \begin{split}
        & [\tau_{b} : \cC_{b}^{n_{b}}]_{\sfW'_{n_{b}}}\\
        =& [\Ind_{\sfW'_{n_{b}}}^{\sfW_{n_{b}}} \tau_{b} : \bigoplus_{\substack{2t+a=n}}
        \Ind_{\sfH_{t}\times \sfS_{a}}^{\sfW_{n}} \hsgn\otimes 1] \\
        =& [ \wttau_{b} : \bigoplus_{\substack{2t+a=n}}
        \Ind_{\sfH_{t}\times \sfS_{a}}^{\sfW_{n}} \hsgn\otimes 1] \\
      \end{split}
    \]
    In any cases, the counting formula holds. There is place to confuse: Why
    there shouldn't be double the size of special unipotent representations?

    In fact, $\AC_{\bC}(\pi)$ can only be the fixed type, say $\cO_{I}$! Note
    that we fixed an infinitesimal character which has half-integral values.
    This choice implicitly force us to fix real Siegel parabolic when we do
    induction from $\GL$! The non-trivial outer automorphism, say $c$, will
    permute the infinitesimal character to the another one and we then will have
    $\AC_{\bC}({}^{c}\pi)i = \cO_{II}$.


    Using Barbasch's formula of wavefront, we see that the induction
    $\pi_{I}:=\Ind_{\GL}^{\SO}\pi'$ must be irreducible, where $\pi'$ is a
    unipotent representation of $\GL$. This will also implies
    $\Ind_{\GL}^{\rO}\pi'$ is irreducible and restricted to two $\SO$-modules,
    $\pi_{I}$ and $\pi_{II}$.

  }


  \trivial[]{ Suppose $\star = \wtC$. Then
    \[
      \begin{split}
        & [\tau_{b}\boxtimes \tau_{\wp} :
        \cC_{b} \boxtimes \cC_{g}]_{\sfW_{n_{b}}\times \sfW'_{n_{g}}}\\
        = & [ \tau_{b}\boxtimes \Ind_{\sfW'_{n_{g}}}^{\sfW_{n_{g}}}\tau_{\wp} : \cC_{b} \boxtimes \cC_{g}]_{\sfW_{n_{b}}\times \sfW_{n_{g}}}\\
        = &\sum_{\sP\in \tA(\ckcO)} [\tau_{b}\boxtimes \wttau_{\wp}:
        \cC_{b} \boxtimes \cC_{g}]_{\sfW_{n_{b}}\times \sfW_{n_{g}}}\\
        =& \# \PBP_{\star}(\ckcO_{b})\cdot \# \PBP_{\star}(\ckcO_{g})
      \end{split}
    \]

    Suppose $\star = B$. Then
    \[
      \begin{split}
        & \sum_{\sP\in \tA'(\ckcO)}[\tau_{b}\boxtimes \tau_{\wp} :
        \cC_{b} \boxtimes \cC_{g}]_{\sfW_{n_{b}}\times \sfW'_{n_{g}}}\\
        =& \# \PBP_{\star}(\ckcO_{b})\cdot \# \PBP_{\star}(\ckcO_{g};\wp)
      \end{split}
    \]
  }
\end{proof}


\subsection{Reduction to the good parity case}

In this section, $G$ is a classical group or metaplectic group.

\begin{lem}\label{lem:Unip.BP}
  Suppose $n_{g}=0$. Then we have a bijection
  \[
    \begin{array}{rccc}
      \fI_{b}: &\Unip_{G'}(\ckcO'_{b}) & \longrightarrow & \Unip_{G}(\ckcO) \\
      &\pi' & \mapsto & \pi:=\Ind_{P}^{G}\pi'
    \end{array}
  \]
  where $P$ is the standard parabolic subgroup of $G$ whose Levi subgroup is
  isomorphic to $G'$.
  Let $\sO'$ be the real nilpotent orbit in $G'$ such that $\sO'_{\bC}=\cO$.
  and $\sO$ be the real induction of $\sO'$ to $G$.
  Then
  \[
    \AC(\pi) = \sO
  \]
  is multiplicity one.
\end{lem}
\begin{proof}
  Note that $\WF(\pi)=\sO'$.
  It follows from Barabasch's formula on wavefront cycle that the
  associated variety of $\pi:=\Ind_{P}^{G}\pi'$ is $\sO$ with multiplicity one.
  This immediately implies that $\pi$ is irreducible. In fact, if $\pi$ is
  reducible, $\pi$ must contain a irreducible sub-quotient with infinitesimal
  character $\lambda_{\ckcO}$ and GK-dimension $<\half\dim_{\bC}\cO$, this
  is contradict to \Cref{lem:LC.mu}.
  \trivial[]{
    First, it is not obvious to me that the wavefront of $\Ind_{P}^{G}\pi$
    must be contained in $\Ind\WF(\pi)$. But it is clear that the leading term
    must be $\sum_{\sO\text{ open in } \WF(\pi)}\Ind\sO$. So the boundaries has
    less GK-dimension.

    Suppose $\pi_{0}$ is the sub-quotient with less GK-dimension.
    On the other hand, the maximal primitive ideal $\cI_{\ckcO}$  with infinitesimal
    character must contains $\Ann\pi_{0}$. In other words,
    $\AV_{\bC}(\pi_{0})\subseteq \bcO$ which implies GK-dimension of
    $\pi_{0}\geq \half\dim_{\bC}\cO$, a contradiction.
    % Let $\mu = \lamck$.
    % Note that every $W_{[\mu]}$-module in $\Grt_{\mu}(G)$ is in $\Ind_{W_{\mu}}^{W_{[\mu]}}$
  }
  Note that representations in $\Unip_{G'}(\ckcO'_{b})$ have distinct
  cuspidal data/Langlands parameter. This implies that $\Ind_{P}^{G}\pi'$ has distinct cuspidal data/Langlands when $\pi'$ varies.
  Recall that $\star' \in \set{A^{\bR},A^{\bC}, A^{\bH}}$ depends on $G$.
  Therefore, $\fI_{b}$ is injective. The bijection follows from the counting
  inequality below:
  \[
    \abs{\PP_{\star'}(\ckcO'_{b})}=\abs{\Unip_{G'}}(\ckcO'_{b})\leq \abs{\Unip_{G}(\ckcO)}
    \leq \abs{\PBP_{\star,b}(\ckcO_{b})} = \abs{\PP_{\star'}(\ckcO'_{b})}.
  \]
\end{proof}


Case by case,  we set
\[
  (G_{b},G_{g}) =
  \begin{cases}
    (\SO(n_{b},n_{b}+1),\SO(p,q)) & \text{when } \star = B \\
    (\Sp(2n_{b},\bR),\Sp(2n_{g},\bR)) & \text{when } \star = C \\
    (\Sp(n_{b},n_{b}), \Sp(p,q)) & \text{when } \star = C^{*} \\
    (\Mp(2n_{b},\bR),\Mp(2n_{g},\bR)) & \text{when } \star = \wtC \\
    (\rO^{*}(n_{b}), \rO^{*}(n_{g})) & \text{when } \star = D^{*} \\
    (\SO(n_{b},n_{b}), \SO(p,q) )& \text{when } \star = D \\
  \end{cases}
\]
Let $G'$ be the Levi of the Siegel parabolic in $G_{b}$.

\begin{prop}\label{prop:red}
  There is a bijection
  \begin{equation}\label{eq:IND}
      \begin{array}{rccc}
    \fI\colon &   \Unip_{G'}(\ckcO'_{b})\times \Unip_{G_{g}}(\ckcO_{g})&         \longrightarrow &\Unip_{G}(\ckcO) \\
     &   (\pi',\pi_{0}) & \mapsto & \pi'\rtimes \pi_{0}.
      \end{array}
    \end{equation}
  \end{prop}

% We would like to reduce the problem to consider the bad and good parts
% separately.
First we translate the problem to
regular infinitesimal character using the following lemma:
 The method was already carefully explained in \cite{Mat} and see
\cite{GI}*{Section~3} for a comprehensive account of the problem.


\def\fhhaso{(\fhh^a_1)^*}
\def\fhhast{(\fhh^a_2)^*}
\newcommand{\ff}{f}
\newcommand{\ffcoh}{\varphi}

\begin{lem}[{c. f. \cite{GI}*{Lemma~3.3}}]\label{lem:ff.irr}
  Suppose that
  \begin{enumerate}[label=(\roman*),series=KLff]
    \item \label{it:KLff.1} $G_{1}$ and $G_{2}$ are two real reductive groups in
          the Harish-Chandra class
          \item there is an isomorphism
          \[ \ff\colon \fhhaso\rightarrow \fhhast
          \]
          of the dual of the abstract Cartans of $G_{1}$ and $G_{2}$;
    \item $\lambda_{1} \in \fhhaso$ and $\lambda_{2} = \ff(\lambda_{1})$ are
          regular dominant elements;
          % \item $\lambda_{1}$ and $\lambda_{2}$ are regular;
    \item \label{it:KLff.4} $\ff$ induces a bijection between
          $R^{+}_{[\lambda_{1}]}$ and $R^{+}_{[\lambda_{2}]}$, so that we can
          identify $W_{[\lambda_{1}]}$ with $W_{[\lambda_{2}]}$ via $\ff$;
  \end{enumerate}
  Let $\Coh_{1}$ be a $W_{[\lambda_{1}]}$ submodule of
  $\Coh_{[\lambda_{1}]}(G_{1})$ such that $\ev{\lambda_{1}}(\Coh_{1})$ is
  spanned by irreducible $G_{1}$-modules. Suppose
  \begin{equation}\label{eq:coh.ff}
    \ffcoh\colon \Coh_{1}\longrightarrow \Coh_{[\lambda_{2}]}(G_{2})
  \end{equation}
  is an injection between $W_{[\lambda_{1}]}=W_{[\lambda_{2}]}$-modules such
  that
  \begin{enumerate}[KLff]
    \item \label{it:KLff.5} $\ffcoh(\Phi)(\lambda_{2})$ is irreducible if $\Phi\in \Coh_{1}$
          and $\Phi(\lambda_{1})$ is irreducible.
  \end{enumerate}

  Then for any $\mu_{1}\in [\lambda_{1}]$, the evaluation at $\mu_{1}$ induces
  an injection
  \[
    \ffcoh_{\mu_{1}} \colon \ev{\mu_{1}}(\Coh_{1}) \longrightarrow \ev{\ff(\mu_{1})}(\Coh_{[\lambda_{2}]}(G_{2})).
  \]
  such that $\ffcoh_{\mu_{1}}(\pi)$ is irreducible if $\pi$ is irreducible.
\end{lem}
\begin{proof}
  The injectivity of $\ff_{\mu_{1}}$ is clear from \Cref{lem:coh.count} and the
  injectivity of \eqref{eq:coh.ff}. \trivial[]{Note that $W_{[\lambda_{1}]}$ is a
    finite group!}

  We now prove the second claim. It easy to reduce to the case when $\mu_{1}$ is
  $R^{+}_{[\lambda_{1}]}$ dominant, c.f. \cite{GI}*{Lemma~3.3}.
  Let $\Phi\in \Coh_{1}$ such that
  $\Phi(\mu_{1})=\pi$ and $\Phi(\lambda_{1})$ is irreducible (the existence
  of $\Phi$ is an abstract property of the coherent continuation). By our
  assumption $\ffcoh(\Phi)(\lambda_{1})$ is irreducible. Therefore
  $\ffcoh_{\mu_{1}}(\pi):=\ffcoh(\Phi)(\ff(\mu_{1}))$ must be either
  irreducible or zero (one of the abstract property of coherent continuation).
  Since it is non zero by the injectivity  of $\ffcoh_{\mu_{1}}$, the lemma
  follows.
\end{proof}

In this paper, we use the Kazhdan-Lusztig-Vogan theory to obtain the injection
\eqref{eq:coh.ff} and then reduce the problem to bad and good parities
separately.
Note that the method does not really relies on the Vogan duality and nor
require the whole coherent continuation module are isomorphism.

In additional to \ref{it:KLff.1}-\ref{it:KLff.5} in
\Cref{lem:ff.irr}, we made the following assumptions (c.~f. \cite{GI}*{\S 3E}):
\begin{enumerate}[KLff]
  % \item $G_{1}$ and $G_{2}$ are two real reductive groups ;
  % \item there is an isomorphism $\ff\colon \fhha_{1}\rightarrow \fhha_{2}$
  % between abstract Cartan subalgebras $\fhha_{1}$ and $\fhha_{2}$ of $G_{1}$
  % and $G_{2}$ respectively; ;
  % \item $\lambda_{1}\in \fhhaso$ and $\lambda_2\in \fhhast$ are fixed regular
  % elements such that $\lambda_{1} = \lambda_{2}\circ \ff$;
  % \item $\ff$ induces an isomorphism
  % $\ff\colon R_{\lambda_{1}}^{+}\rightarrow R_{\lambda_{2}}^{+}$, and the
  % associated integral Weyl groups
  % $\ff\colon W_{[\lambda_{1}]}\rightarrow W_{[\lambda_{2}]}$;
  \item there is an injection
        \[
        \ff \colon B\rightarrow \cP_{\lambda_{2}}(G_{2})
        \]
        where $B\subseteq \cP_{\lambda_{1}}(G_{1})$ is a union of blocks of
        $G$-conjugate classes of regular characters with infinitesimal character
        $\lambda_{1}$.
  \item for $\gamma_{1}\in B$ and $\gamma_{2} = \ffcoh(\gamma_{1})\in \cP_{\lambda_{2}}(G_{2})$ the
        following conditions are satisfied
        \begin{enumerate}[label=(\alph*)]
          \item $\ff \circ \Phi_{\gamma_{1}} = \Phi_{\gamma_{2}}\circ \ff$
                where $\Phi_{\gamma_{i}}$ are the Cartan involution induced on
                the corresponding abstract root systems. \trivial[]{ This
                condition implies that, the notion of compact/complex/real and
                $\alpha\in R^{+}(\gamma), \Phi(\alpha)\notin R^{+}(\gamma)$
                are preserved by $\ff$. In particular, the integral length
                function $l^{I}$ (defined up to a shifting) of $G_{1}$ and
                $G_{2}$ can be uniformly identified. }
              % \item Suppose $\alpha_{1}$ is noncompact type I (resp.
              % noncompact/real type I/II) if and only if
              % $\alpha_{2}:= \ff(\alpha_{1})$ is noncompact type I (resp. type
              % II).
          \item For simple roots in $R^{+}_{\lambda_{i}}$, the notions of
                noncompact/real type I/II are preserved by $\ff$: $\alpha_{1}$
                is noncompact type I if and only if
                $\alpha_{2}:= \ff(\alpha_{1})$ is noncompact type I, etc;
          \item The cross actions are compatible:
                \[
                \ff(w\cross [\gamma_{1}]) = \ff(w)\cross [\ff(\gamma_{1})] \qquad \forall \gamma_{1}\in B_{1}, w\in W_{[\lambda_{1}]}.
                \]
          \item The Cayley transforms \cite{V4} are compatible:
                \[
                \ff( c^{\alpha_{1}} (\gamma_{1})) = c^{\ff(\alpha_{1})}(\ff(\gamma_{1})) \qquad \forall \gamma_{1}\in B_{1}, \alpha_{1} \text{is
                noncompact imaginary}
                \]
                and
                \[
                  \ff( c_{\alpha_{1}} (\gamma_{1})) = c_{\ff(\alpha_{1})}(\ff(\gamma_{1}))
                \]
                for all $\gamma_{1}\in B_{1}, \alpha_{1}$ is
                real and satisfies parity condition.
          % \item The $\tau$ invariants are compatible:
          %       \[
          %       \ff(\tau(\gamma_{1})) = \tau(\ff(\gamma_{1}))\qquad \forall \gamma_{1}\in B_{1}.
          %       \]
          %       \trivial[]{ This condition seems not been used below, but used
          %       in Gan-Ichino's proof! For linear group the matching of
          %       $\tau$-invariant is automatic since it is explicitly determined
          %       by the types of simple roots. For metaplectic groups, it needs
          %       extra information to determine the $\tau$-invariant, see
          %       \cite{RT1}*{Lemma~6.28}. It is not clear to me that the
          %       $\tau$-invariant must match.

          %       On the other hand, the proof of \Cref{lem:ff.irr} seems implies
          %       that $\tau$-invariants match automatically! }
        \end{enumerate}
\end{enumerate}

%For a fixed block $B$ a regular infinitesimal character $\lambda_{1}$,
Under the above assumptions, let $\Grt_{B}$ be the span of
$\set{\barpi_{\gamma}|\gamma\in B}$ in the
Grothendieck group, and $\Coh_{1} := \ev{\lambda}^{-1}(\Grt_{B})$.
% What we really need is the validity of the following
% conditions:
% \begin{itemize}
%   \item there is an injection of $W_{[\lambda_{1}]}=W_{[\lambda_{2}]}$-module
%   \begin{equation}\label{eq:coh.ff}
%     \ff\colon \Coh_{B_{1}}\rightarrow \Coh_{B_{2}}
%   \end{equation}
%   such that, at our fixed regular infinitesimal character $\lambda_{1}$,
%   $\Phi(\lambda_{1})$ is irreducible implies $\ff(\Phi)(\ff(\lambda_{1}))$
%   is irreducible.
% \end{itemize}
Then $\ffcoh$ is given by sending $\Phi_{\gamma_{1}}$ to
$\Phi_{\ff(\gamma_{1})}$. In fact, the above map on coherent continuation
representations can be lifted to maps between Hecke-modules and the validity of
Kazhdan-Lusztig-Vogan conjecture implies the preservation of irreducibility of
$\ffcoh$ at $\lambda_{1}$.


% \begin{lem}[{c. f. \cite{GI}*{Lemma~3.3}}]\label{lem:ff.irr}
%   Suppose $\mu_{1}\in [\lambda_{1}]$ is dominant and $\Phi\in \Coh_{B_{1}}$.
%   The evaluation at $\mu_{1}$ induces an injection
%   \[
%   \ff_{\mu_{1}}  \ev{\mu_{1}}(\Coh_{B_{1}}) \longrightarrow  \ev{\mu_{2}}(\Coh_{B_{2}}).
%   \]
%   Moreover, $\ff_{\mu_{1}}(\pi)$ is irreducible if $\pi$ i irreducible.
% \end{lem}
% \begin{proof}
%   The injectivity of $\ff_{\mu_{1}}$ is clear from \Cref{lem:coh.count} and the
%   injectivity of \Cref{eq:coh.ff}. \trivial[]{Note that $W_{[\lambda_{1}]}$ is a
%     finite group!}

%   We now prove the second claim. Let $\Phi\in \Coh_{B_{1}}$ such that
%   $\Phi(\mu_{1})=\pi$ and $\Phi(\lambda_{1})$ is irreducible (the existence
%   of $\Phi$ is an abstract property of the coherent continuation). By our
%   assumption $\ff(\Phi)(\lambda_{1})$ is irreducible. Therefore
%   $\ff(\Phi)(\ff(\mu_{1}))$ must be irreducible since it is non zero by the
%   first claim.
% \end{proof}


We set $G_{1} := G_{b}\times G_{g}$ and $G_{2}:=G$.
There is a natural map
\[
\ff\colon G_{1} = G_{b}\times G_{g}\longrightarrow G = G_{2}.
\]
The map is given by \cite{GI}*{\S 3G} when $\star = B$ and given by
\cite{RT2}*{\S 5} when $\star = \wtC$. In the other cases, they are natural
embeddings. The maps between abstract Cartans
$\ff\colon \fhhaso\rightarrow \fhhast$ are given by putting the coordinates of
$G_{b}$ before $G_{g}$.

Let $\lambda \in\fhhast $ be a regular dominant element in $[\lamck]$.
Then
$\lambda_{1}:= \ff^{-1}(\lambda)$ is regular dominant for $G_{1}$.
Let $B := \cP_{\lambda_{1}}(G_{1})$ in all the case.
The map $\ff\colon B\rightarrow \cP_{\lambda}(G_{2})$ is given by the natural maps between real Cartans.
This is clear when $\ff$ is an embedding.
See \cite{GI}*{\S 3} when $G$ is a special orthogonal group and
\cite{RT2}*{\S 5} when $G$ is a metaplectic group.

\begin{proof}[{Proof of \Cref{prop:red}: the injectivity of $\fI$}]
  Let % $\mu_{2} = \lamck$
      % and
  $\mu_{1} = \ff^{-1}(\lamck)$.
  Since coherent continuation is compatible with induction
  \cite{Vg}*{Proposition~7.4.1}, we have the
  following commutative diagram
  \[
    \begin{tikzcd}[column sep={4cm,between origins}]
      &  \Coh_{[\lambda'_{b}]}(G'_{b})\otimes \Coh_{[\lambda_{g}]}(G_{g}) \ar[dl,"\Ind_{P_{b}}^{G_{b}}\otimes \id"']\ar[dr,"\Ind_{P}^{G}"]&\\
      \Coh_{[\lambda_{b}]}(G_{b})\otimes \Coh_{[\lambda_{g}]}(G_{g}) \ar[d,"\ev{\mu_{1}}"'] \ar[rr,"\ffcoh"]& & \Coh_{[\lamck]}(G) \ar[d,"\ev{\ckcO}"]\\
      \Grt_{\mu_{1}}(G_{b}\times G_{g}) \ar[rr,"\ffcoh_{\mu_{1}}"]& &
      \Grt_{\lamck}\\
    \end{tikzcd}
  \]
  Here $P_{b}$ is the parabolic subgroup of $G_{b}$ whose Levi subgroup is
  isomorphic to $G'_{b}$ and $P$ is the parabolic of $G$ whose Levi subgroup is
  isomorphic to $G'_{b}\times G_{g}$. The horizontal line $\ffcoh$ is an
  injection by the above setting, which implies $\ffcoh_{\mu_{1}}$ is injective
  by \Cref{lem:ff.irr}. Note that $\ffcoh_{\mu_{1}}$ sends
  $\Ind_{P_{b}}(\pi')\otimes \pi_{0}$ to $\Ind_{P}^{G} (\pi'\otimes \pi_{0})$.
  \trivial{ Let
    $\Phi'\otimes \Phi_0\in \Coh_{[\lambda'_{b}]}(G'_{b})\otimes \Coh_{[\lambda_{g}]}(G_{g})$
    be coherent family such that
    $\Phi'\otimes \Phi_0(\lamck) = \pi'\otimes \pi_0$. Now
    $\Ind_{P_{b}}(\pi')\otimes \pi_{0} = (\Ind_{P_{b}}\otimes \id) (\Phi'\otimes \Phi_{0})(\mu_{1})$
    and
    $\Ind_{P}(\pi'\otimes \pi_{0}) = \ffcoh(\Ind_{P_{b}}\otimes \id (\Phi'\otimes \Phi_{0}))(\lamck) = (\Ind_{P}^{G}) (\Phi'\otimes \Phi_{0})(\lamck)$
  }
  Now the proposition follows from \Cref{lem:Unip.BP}.
\end{proof}


\begin{lem}\label{lem:BGcount}
  We have
  \[
    \abs{\Unip_{G}(\ckcO)} =
    \abs{\Unip_{G_{b}}(\ckcO_{b})}\cdot
    \abs{\Unip_{G_{g}}(\ckcO_{g})}
  \]
\end{lem}
\begin{proof}
  It suffice to consider the case where $\nbb$ and $\ngg$ are both non-zero.
  Fix a regular dominant infinitesimal character $\lambda\in [\lamck]$ and write
  $(\lambda_{b},\lambda_{g}):=\ff^{-1}(\lambda)$. It requires a some more
  precise information about the blocks/cell of $G$. Suppose $\star \neq D^{*}$.
  Then
  \[
    \ffcoh \colon \cP_{\lambda_{b}}(G_{b})\times \cP_{\lambda_{g}}(G_{g}) \longrightarrow \cP_{\lambda}(G)
  \]
  is a bijection. In particular each Harish-Chandra cell in $\cP_{\lambda}(G)$
  is the product of a cell in $\cP_{\lambda_{b}}(G_{b})$ and a cell in
  $\cP_{\lambda_{g}}(G_{g})$. By \Cref{lem:AV.HC}, this implies $\ffcoh$
  restricted to an isomorphism
  \[
    \Coh_{[\lambda_{b}],\bcO_{b}}(G_{g})\otimes \Coh_{[\lambda_{g}],\bcO_{g}}(G_{b}) \xrightarrow{\ \ \ \ffcoh\ \ \ } \Coh_{[\lambda],\bcO}(G).
  \]
  and the lemma follows by the evaluation to $\lamck$.

  Now consider the case where $\star = D^{*}$. In this case, $\cP_{\lambda}(G)$
  has $2$ blocks. Meanwhile $\cP_{\lambda_{b}}(G_{b})$ and
  $\cP_{\lambda_{g}}(G_{g})$ only have $1$ block. Let
  $B_{1} := \ff(\cP_{\lambda_{b}}(G_{b})\times \cP_{\lambda_{g}}(G_{g}))$ and
  $B_{2}$ be the other block. For any $\sfW'_{n}$-module $\tau$, let $\tau^{s}$
  denote the twist of $\tau$ by any non-trivial element in $\sfW_{n}/\sfW'_{n}$.

  Then \[
    \begin{split}
      \Coh_{B_{1}} &\cong \cC_{b}^{\nbb}\otimes \cC_{g}^{\ngg,\ngg} = \Coh_{[\lambda_{b}],\bcO_{b}}(G_{g})\otimes \Coh_{[\lambda_{g}],\bcO_{g}}(G_{b}),\\
      \Coh_{B_{2}} &\cong (\cC_{b}^{\nbb})^{s}\otimes (\cC_{g}^{\ngg,\ngg})^{s}
    \end{split}
  \]
   and
  \[
    \Coh_{[\lambda]}(G) = \Coh_{B_{1}}\oplus \Coh_{B_{2}}.
  \]
  Clearly $\Coh_{[\lambda], \bcO}$ is compatible with the above decomposition
  and we have
  \[
    \Coh_{[\lambda_{b}],\bcO_{b}}(G_{g})\otimes \Coh_{[\lambda_{g}],\bcO_{g}}(G_{b}) \xrightarrow{\ \ \ \ffcoh\ \ \ } \Coh_{[\lambda],\bcO}(G)\cap \Coh_{B_{1}}.
  \]

  Observe that $[\tau_{b}:(\cC_{b}^{\nbb})^{s}]=0$. Therefore
  \[
    \begin{split}
      \abs{\Unip_{G}(\ckcO)} & = \sum_{\tau_{b}\boxtimes \tau_{g}\in \LC_{\ckcO}} [\tau_{b}\otimes \tau_{g}:\Coh_{[\lambda],\bcO}(G)\cap \Coh_{B_1}]\\
    & \ \ + \sum_{\tau_{b}\boxtimes \tau_{g}\in \LC_{\ckcO}}[\tau_{b}\otimes \tau_{g}:\Coh_{[\lambda],\bcO}(G)\cap \Coh_{B_2}] \\
  &= \sum_{\tau_{b}\boxtimes \tau_{g}\in \LC_{\ckcO}} [\tau_{b}\otimes \tau_{g}:
  \Coh_{[\lambda_{b}],\bcO_{b}}(G_{g})\otimes \Coh_{[\lambda_{g}],\bcO_{g}}(G_{b})]\\
  &= \abs{\Unip_{G_{b}}(\ckcO_{b})}\cdot \abs{\Unip_{G_{g}}(\ckcO_{g})}
\end{split}
\]
\end{proof}

\begin{proof}[{Proof of \Cref{prop:red}: the bijectivity of $\fI$}]
Now the bijectivity of $\fI$ follows by the counting in  \Cref{lem:Unip.BP} and \Cref{lem:BGcount}.
\end{proof}


%\subfile{counting_cl}



\section{Combinatorics of painted bi-partitions}

In this section, we assume $\star \in \Set{B,C,\wtC,C^{*},D,D^{*}}$, and $\ckcO = \ckcOg$, namely $\ckcO $ has $\star$-good parity.

Recall the set of primitive $\star$-pairs $\CPPs(\ckcO)$ in $\ckcO$. For a subset $\wp$ of $\CPPs(\ckcO)$, we have defined a bipartition $\tau_{\wp}=(\imath_{\wp},\jmath_{\wp})$ in \Cref{sec:LCBCD}.

The main result of this section is the following

\begin{prop} \label{prop:PBP} When $\star\in \set{C^{*}, D^{*}}$,
  \[
    \PBP_{\star}(\tau_{\wp}) = \emptyset, \quad \text{if } \wp \neq \emptyset.
  \]
  When $\star\in \set{B,C,\wtC,D}$,
  \[
    \sharp(\PBP_{\star}(\tau_{\wp})) = \sharp(\PBP_{\star}(\tau_{\emptyset})), \quad \forall \wp \subseteq \CPPs(\ckcO).
  \]
 Consequently we always have
  \[
    \sharp(\tPBP_{\star}(\ckcO)) = \sharp(\PBPe_{\star}(\ckcO)).
  \]
\end{prop}

% \subsection{The case of quaternionic groups}

We shall deal with the two quaternionic cases first, which are simple. When $\star \in \set{B, C, \wtC, D}$, the proof of the main statement of the above proposition involves an elaborate reduction argument (by removing elements from $\wp$ one-by-one), and will be handled separately in \Cref{lem:down} below.
\begin{proof}%[Proof of {\Cref{prop:PBP}} the quaternionic case]

    \smallskip

  First consider the case when $\star = C^{*}$. Suppose that
  $\wp \neq \emptyset$. Then we have
  \begin{equation}\label{eq:res.C*}
    \bfcc_{i}(\imath_{\wp}) = \half(\bfrr_{2i-1}(\ckcO)+1)>
    \half(\bfrr_{2i}(\ckcO)-1) = \bfcc_{i}(\jmath_{\wp}),
    \quad \forall \,\, (2i-1, 2i)\in \wp,
  \end{equation}
  Let $\uptau = (\imath_{\wp}, \cP)\times (\jmath_{\wp},\cQ)\times \star$ be an element in $\PBP_{\star}(\tau_{\wp})$. By the requirements of a painted bipartition, we have
  \[
    \bfcc_{i}(\imath_{\wp}) = \sharp\set{j| \cP(i,j)=\bullet} = \sharp\set{j| \cQ(i,j)=\bullet} \leq \bfcc_{i}(\jmath_{\wp}), \quad \forall \,\, i=1,2,3,\cdots,
  \]
  which contradicts \eqref{eq:res.C*}. Hence, $\PBP_{\star}(\tau_{\wp})= \emptyset$.

  \smallskip

  Now consider the case when $\star = D^{*}$. Suppose that $\wp \neq \emptyset$.
  Then we have
  \begin{equation}\label{eq:res.D*}
    \bfcc_{i+1}(\imath_{\wp}) = \half(\bfrr_{2i}(\ckcO)+1)>
    \half(\bfrr_{2i+1}(\ckcO)-1) = \bfcc_{i}(\jmath_{\wp}),
    \quad \forall \,\, (2i, 2i+1)\in \wp.
  \end{equation}
  Let $\uptau = (\imath_{\wp}, \cP)\times (\jmath_{\wp},\cQ)\times \star$ be an element in $\PBP_{\star}(\tau_{\wp})$. By the requirements of a painted bipartition, we have
  \[
    \bfcc_{i+1}(\imath_{\wp}) \leq \sharp\set{j| \cP(i,j)=\bullet} =\sharp\set{j| \cQ(i,j)=\bullet} \leq \bfcc_{i}(\jmath_{\wp}), \quad \forall \,\, i = 1,2,3, \cdots,
  \]
  which contradicts \eqref{eq:res.D*}. Hence, $\PBP_{\star}(\tau_{\wp})= \emptyset$.

This completes the proof for the quaternionic cases.
\end{proof}

The rest of this section is devoted to the proof of the following


\def\PPm{\wp_{\downarrow}}
\def\uptaum{\uptau_{\downarrow}}

% Suppose $\star = \wtC$, $\wp\neq \emptyset$, and
% $t:=\min{t|(2t-1,2t)\in \wp}$. Let $\PPm:=\wp - \set{(2t-1,2t)}$. Let
% $\PPm:=\wp - \set{(2t-1,2t)}$.

\begin{lem}\label{lem:down}
  Suppose $\star \in \set{B, C, \wtC, D}$ and
  $\wp$ is a non-empty subset of $\CPPs(\ckcO)$.
  Let
  \[
    t:=
    \begin{cases}
      \min\set{i|(2i-1,2i)\in \wp} & \text{when $\star \in \set{C,\wtC}$}\\
      \min\set{i|(2i,2i+1)\in \wp} & \text{when $\star \in \set{B,D}$}\\
    \end{cases}
  \]
  and
  \[
    \PPm:=
    \begin{cases}
      \wp - \set{(2t-1,2t)}  & \text{when $\star \in \set{C,\wtC}$}\\
      \wp -  \set{(2t,2t+1)} & \text{when $\star \in \set{B,D}$}\\
    \end{cases}
  \]
  Then
  \[
    \sharp(\PBP_{\star}(\tau_{\PPm})) = \sharp(\PBP_{\star}(\tau_{\wp})).
  \]
\end{lem}
\begin{proof}
  We prove the equality by defining a bijection
  \[
    T_{\PPm,\wp}\colon \PBP_{\star}(\tau_{\PPm}) \rightarrow \PBP_{\star}(\tau_{\wp})\quad \uptaum \mapsto \uptau
  \]
  %and its inverse $T_{\wp,\PPm}$
  explicitly case by case.
  In the following, $\uptau = (\imath_{\wp},\cP_{\uptau})\times (\jmath_{\wp},\cQ_{\uptau})$
  will always denote an element in $\PBP_{\star}(\tau_{\wp})$ and
  $\uptaum = (\imath_{\PPm},\cP_{\uptaum})\times (\jmath_{\PPm},\cQ_{\uptaum})$
  an element in $\PBP_{\star}(\tau_{\PPm})$.

  \medskip

  % \[
  %   T_{\wp,\PPm}\colon \PBP_{\star}(\tau_{\wp}) \rightarrow \PBP_{\star}(\tau_{\PPm}).
  % \]


  %We start with the simplest case.

  \smallskip

  Case $\star = \wtC$:
  %Let $(b_{1},b_{2}) = (\frac{\bfrr_{2t-1}(\ckcO)}{2},\frac{\bfrr_{2t}(\ckcO)}{2})$.
  We have
  \[
    \begin{split}
      (\bfcc_{t}(\imath_{\PPm}), \bfcc_{t}(\jmath_{\PPm}))
      &= (\bfcc_{t}(\jmath_{\wp}), \bfcc_{t}(\imath_{\wp})),\AND\\
      % (\bfcc_{t}(\imath_{\PPm}), \bfcc_{t}(\jmath_{\PPm}))
      % &= (b_{1},b_{2})= (\bfcc_{t}(\jmath_{\wp}), \bfcc_{t}(\imath_{\wp})),\AND\\
      (\bfcc_{i}(\imath_{\PPm}), \bfcc_{i}(\jmath_{\PPm}) )
      & = (\bfcc_{i}(\imath_{\wp}), \bfcc_{i}(\jmath_{\wp}) ) \quad \text{for $i\neq t$}.
    \end{split}
  \]

  For $\uptaum\in\PBPs(\tau_{\PPm})$, define $\uptau=:T_{\PPm,\wp}(\uptaum)$ by the following formula:
  \[
    \begin{split}
      \text{$\forall (i,j)\in \BOX{\imath_{\wp}}$,} \quad   \cP_{\uptau}(i,j) &=  \cP_{\uptaum}(i,j),\\
      \text{$\forall (i,j)\in \BOX{\jmath_{\wp}}$,} \quad \cQ_{\uptau}(i,j) &= \begin{cases}
        r& \text{if $j=t$ and  $\cP_{\uptaum}(i,j)=s$,}\\
        d& \text{if $j=t$ and  $\cP_{\uptaum}(i,j)=c$,}\\
        \cQ_{\uptaum}(i,j) &\text{otherwise.}
      \end{cases}
    \end{split}
  \]
 We easily check that the above formula defines a valid
  painted bipartition $\uptau$ and construct the inverse map $T_{\wp,\PPm}$ by reversing the process.
  This finishes the proof for the case when $\star=\wtC$. \medskip

  \trivial[]{The inverse map
  \[
    T_{\wp,\PPm}\colon \PBP_{\star}(\tau_{\wp}) \rightarrow \PBP_{\star}(\tau_{\PPm}).
  \]
  is given by the following formula:
  \[
    \begin{split}
      \text{$\forall (i,j)\in \BOX{\imath_{\PPm}}$,} \quad \cP_{\uptaum}(i,j) &= \begin{cases}
        s& \text{if $j=t$ and  $\cQ_{\uptau}(i,j)=r$,}\\
        c& \text{if $j=t$ and  $\cQ_{\uptau}(i,j)=d$,}\\
        \cP_{\uptau}(i,j) &\text{otherwise.}
      \end{cases}\\
      \text{$\forall (i,j)\in \BOX{\jmath_{\PPm}}$,} \quad   \cQ_{\uptaum}(i,j) &=  \cQ_{\uptau}(i,j).\\
    \end{split}
  \]
  }
  % We leave it to the reader to check that the above formula does define a valid
  % painted bipartition $\uptaum$. Retain the above notation, it is easy to check that the
  % inverse map
  % \[
  %   T_{\PPm,\wp}\colon \PBP_{\star}(\tau_{\PPm}) \rightarrow \PBP_{\star}(\tau_{\wp})\quad \uptaum \mapsto \uptau
  % \]
  % is given by the following formula:
  % \[
  %   \begin{split}
  %     \text{$\forall (i,j)\in \BOX{\imath_{\wp}}$,} \quad   \cP_{\uptau}(i,j) &=  \cP_{\uptaum}(i,j),\\
  %     \text{$\forall (i,j)\in \BOX{\jmath_{\wp}}$,} \quad \cQ_{\uptau}(i,j) &= \begin{cases}
  %       r& \text{if $j=t$ and  $\cP_{\uptaum}(i,j)=s$,}\\
  %       d& \text{if $j=t$ and  $\cP_{\uptaum}(i,j)=c$,}\\
  %       \cQ_{\uptaum}(i,j) &\text{otherwise.}
  %     \end{cases}
  %   \end{split}
  % \]

  \medskip

  Case $\star = C$:
  % Let
  % $(b_{1},b_{2}) = (\frac{\bfrr_{2t-1}(\ckcO)-1}{2},\frac{\bfrr_{2t}(\ckcO)+1}{2})$.
  We have
  \[
    \begin{split}
      (\bfcc_{t}(\imath_{\PPm}), \bfcc_{t}(\jmath_{\PPm})) &=
      (\bfcc_{t}(\jmath_{\wp})+1, \bfcc_{t}(\imath_{\wp})-1) \AND \\
     %  &= (b_{2},b_{1}),  \\
     % &= (b_{1}-1,b_{2}+1),\AND\\
      % (\bfcc_{t}(\imath_{\PPm}), \bfcc_{t}(\jmath_{\PPm})) &= (b_{2},b_{1}),  \\
      % (\bfcc_{t}(\imath_{\wp}), \bfcc_{t}(\jmath_{\wp})) &= (b_{1}-1,b_{2}+1),\AND\\
      (\bfcc_{i}(\imath_{\PPm}),\bfcc_{i}(\jmath_{\PPm})) &=(\bfcc_{i}(\imath_{\wp}),\bfcc_{i}(\jmath_{\wp}))\quad \text{for $i\neq t$}.
    \end{split}
  \]
  % Let
  % $a = \half(\bfrr_{2t-1}(\ckcO)-\bfrr_{2t}(\ckcO))-1 = \bfcc_{t}(\jmath_{\PPm})-\bfcc_{t}(\imath_{\PPm})$.

  \trivial[]{
    The idea of the definition of $T_{\PPm,\wp}$ is that we move ``$s$'' appeared
    in the $t$-th column of $\cQ_{\uptaum}$ to the $t$-th column of
    $\cP_{\uptau}$.
  }

  For $\uptaum\in \PBPs(\tau_{\PPm})$, we define $\uptau$ by the following algorithm:
  \begin{description}
    \item[STEP~1] Define a map
          $\cP'\colon \BOX{\imath_{\wp}}\rightarrow \set{\bullet,r,c,d}$ (as a candidate for $\cP_{\uptau}$), by the following rules:
          \begin{enumerate}[label=(\alph*)]
            \item Suppose
            $\cP_{\uptaum}(\bfcc_{t}(\imath_{\PPm}),t)\neq \bullet$.
            \begin{itemize}
              \item If $\bfcc_{t}(\imath_{\PPm})\geq 2$ and
              $\cP_{\uptaum}(\bfcc_{t}(\imath_{\PPm})-1,t) = c$,
              we define
              \[
                \cP'(i,j) := \begin{cases}
                  r ,& \text{if $j=t$ and $\bfcc_{t}(\imath_{\PPm})-1
                    \leq i \leq \bfcc_{t}(\imath_{\wp})-2$},\\
                  c ,& \text{if $(i,j)=(\bfcc_{t}(\imath_{\wp})-1,t)$},\\
                  d ,& \text{if $(i,j)=(\bfcc_{t}(\imath_{\wp}),t)$},\\
                  \cP_{\uptaum}(i,j) ,&\text{otherwise}.
                \end{cases}
              \]
              \item Otherwise, we define
              \[
                \cP'(i,j) := \begin{cases}
                  r ,& \text{if $j=t$ and $\bfcc_{t}(\imath_{\PPm})
                    \leq i \leq \bfcc_{t}(\imath_{\wp})-1$},\\
                  \cP_{\uptaum}(\bfcc_{t}(\imath_{\PPm}),t) ,&
                  \text{if $(i,j)=(\bfcc_{t}(\imath_{\wp}),t)$},\\
                  \cP_{\uptaum}(i,j) ,&\text{otherwise}.
                \end{cases}
              \]
            \end{itemize}
            \item Suppose $\cP_{\uptaum}(\bfcc_{t}(\imath_{\PPm}),t)=\bullet$.
            \begin{itemize}
              \item If $\bfcc_{t+1}(\imath_{\PPm}) = \bfcc_{t}(\imath_{\PPm})$
              and
              $\cP_{\uptaum}(\bfcc_{t}(\imath_{\PPm}),t+1) = r$,
              we define
              \[
                \cP'(i,j) := \begin{cases}
                  r ,& \text{if $j=t$ and $\bfcc_{t}(\imath_{\PPm})\leq i \leq \bfcc_{t}(\imath_{\wp})-1$},\\
                  c ,& \text{if $(i,j)=(\bfcc_{t+1}(\imath_{\PPm}),t+1)$},\\
                  d ,& \text{if $(i,j)=(\bfcc_{t}(\imath_{\wp}),t)$},\\
                  \cP_{\uptaum}(i,j) ,&\text{otherwise}.
                \end{cases}
              \]
              \item Otherwise, we define
              \[
                \cP'(i,j) := \begin{cases}
                  r ,& \text{if $j=t$ and $\bfcc_{t}(\imath_{\PPm})\leq i \leq \bfcc_{t}(\imath_{\wp})-2$},\\
                  c ,& \text{if $(i,j)=(\bfcc_{t}(\imath_{\wp})-1,t)$},\\
                  d ,& \text{if $(i,j)=(\bfcc_{t}(\imath_{\wp}),t)$},\\
                  \cP_{\uptaum}(i,j) ,&\text{otherwise}.
                \end{cases}
              \]
            \end{itemize}
          \end{enumerate}

    \item[STEP 2] From the construction of $\cP'$, there are four possibilities for $\cP'$ to violate the requirements of a painting on $\imath_{\wp}$, which are detailed as follows. We must have $t>1$ and violations occur in positions of $(i,j)\in \BOX{\imath_{\wp}}$ inside the following $2\times 2$ square
         \[
          A :=
          \begin{pmatrix}
            (\bfcc_{t}(\imath_{\wp})-1,t-1) & (\bfcc_{t}(\imath_{\wp})-1,t) \\
            (\bfcc_{t}(\imath_{\wp})\;\phantom{-1}\;,t-1) & (\bfcc_{t}(\imath_{\wp})\;\phantom{-1}\;,t) \\
          \end{pmatrix}
          \]

          %\[
          %A :=
          %\begin{pmatrix}
          %  \cP'(\bfcc_{t}(\imath_{\wp})-1,t-1) & \cP'(\bfcc_{t}(\imath_{\wp})-1,t) \\
          %  \cP'(\bfcc_{t}(\imath_{\wp})\;\phantom{-1}\;,t-1) & \cP'(\bfcc_{t}(\imath_{\wp})\;\phantom{-1}\;,t) \\
          %\end{pmatrix}
          %\]
          \trivial[]{
          Note that  $\cP'(\bfcc_{t}(\imath_{\wp})-1,t-1)$ is always equal to
          $\bullet$.
          }

    Let $\cP_{\uptau}\colon \BOX{\imath_{\wp}}\rightarrow \set{\bullet,r,c,d}$
          be the painting on $\imath_{\wp}$ defined as follows:
          \begin{itemize}
            \item When $(i,j)\in \BOX{\imath_{\wp}}$ and
            $\set{\bfcc_{t}(\imath_{\wp})-i,t-1}\cap\set{0,1}=\emptyset$
            (i.e. $(i,j)$ is not one of the four boxes corresponding to
            $A$), we define
            \[
              \cP_{\uptau}(i,j):= \cP'(i,j).
            \]

            \item For the four boxes corresponding to $A$, we modify the symbols by
            setting
            \begin{equation} \label{eq:modP}
              \begin{split}
                &\begin{pmatrix}
                  \cP_{\uptau}(\bfcc_{t}(\imath_{\wp})-1,t-1) & \cP_{\uptau}(\bfcc_{t}(\imath_{\wp})-1,t) \\
                  \cP_{\uptau}(\bfcc_{t}(\imath_{\wp})\;\phantom{-1}\;,t-1)
                  & \cP_{\uptau}(\bfcc_{t}(\imath_{\wp})\;\phantom{-1}\;,t) \\
                \end{pmatrix}\\
                :=&
                \begin{cases}
                  \begin{pmatrix}
                    r & c\\
                    r & d
                  \end{pmatrix}, & \text{if } A =
                  \begin{pmatrix}
                    \bullet & r\\
                    r & r
                  \end{pmatrix},\\[1.5em]
                  \begin{pmatrix}
                    r & c\\
                    c & d
                  \end{pmatrix}, & \text{if } A =
                  \begin{pmatrix}
                    \bullet & r\\
                    c & r
                  \end{pmatrix},\\[1.5em]
                  \begin{pmatrix}
                    r & c\\
                    d & d
                  \end{pmatrix}, & \text{if } A =
                  \begin{pmatrix}
                    \bullet & r\\
                    d & r
                  \end{pmatrix},\\[1.5em]
                  \begin{pmatrix}
                    c & c\\
                    d & d
                  \end{pmatrix}, & \text{if } A =
                  \begin{pmatrix}
                    \bullet & r\\
                    d & c
                  \end{pmatrix}.\\
                \end{cases}
              \end{split}
            \end{equation}
          \end{itemize}

    \item[STEP 3] The painted bipartition $\uptau$ is uniquely determined by
          $\cP_{\uptau}$. More precisely, $\cQ_{\uptau}$ is given by the following
          formula: for $(i,j)\in \BOX{\jmath_{\wp}}$,
          \[
          \cQ_{\uptau}(i,j) :=
          \begin{cases}
            s, & \begin{minipage}{17em}if $\cP'$ is not a valid painting on $\imath_{\wp}$\\
              and $(i,j)= (\bfcc_{t}(\imath_{\wp})-1,t-1)$,
              \end{minipage}\\
            \cQ_{\uptaum}(i,j), & \text{otherwise.}
            \end{cases}
          \]
  \end{description}

 It is not difficult to check that $\uptau$ is a valid painted bipartition
 and to construct the inverse map $T_{\wp,\PPm}$ by reversing the above steps.

 \trivial[]{
   The inverse map $T_{\wp,\PPm}$ is given by the following algorithm:
   \begin{description}
     \item[STEP 1] We first recover $\cP'$.
           If $t=1$ or $\cP'(\bfcc_{t}(\imath_{\wp})-1,t-1)=\bullet$, then
           $\cP':= \cP_{\wp}$.
           Otherwise,
           $\cP'$ is given by $\cP_{\wp}$ except the $2\times 2$ square in
           \eqref{eq:modP} which is given by reversing the formula cited.
     \item[STEP 2]

           \def\xxn{\cP_{\uptaum}(\bfcc_t(\imath_{\PPm})-1,t)} %x_0
           \def\xxo{\cP_{\uptaum}(\bfcc_t(\imath_{\wp}),t)} %x_1
           \def\xxd{\cP_{\uptaum}(\bfcc_t(\imath_{\wp}),t+1)} %x_2
           \def\yyn{\cP'(\bfcc_t(\imath_{\PPm})-1,t)} %y_0
           \def\yyo{\cP'(\bfcc_t(\imath_{\wp})-1,t)} %y_1
           \def\yyt{\cP'(\bfcc_t(\imath_{\wp}),t)} %y_3
           \def\yyd{\cP'(\bfcc_t(\imath_{\wp}),t+1)} %y_2
           We have the following cases:
           \begin{enumerate}[label=(\alph*)]
             \item Suppose $\yyo=r$.
             \begin{itemize}
               \item If $\bfcc_{t+1}(\imath_{\wp}) = \bfcc_{t}(\imath_{\PPm})$
               and
               \[
                 (\yyd,\yyt) = (c,d),
               \]
               let
               \[
                 (\xxo,\xxd):=(\bullet, r)
               \]
               \item Otherwise, let \[
                 \xxo:=\yyt.
               \]
             \end{itemize}
             \item Suppose $\yyo=c$
             \begin{itemize}
               \item If $\bfcc_{t}(\imath_{\PPm})\geq 2$ and $\xxn=r$,
               then let
               \[
                 (\xxn,\xxo):=(c,d).
               \]
               \item Otherwise, let
               \[
                 \xxo :=\bullet.
               \]
             \end{itemize}
           \end{enumerate}
           For the boxes $(i,j)$ in $\BOX{\imath_{\uptaum}}$ which are not specified
           in the above procedure, set
           \[
           \cP_{\uptaum}(i,j):=\cP'(i,j).
           \]
     \item[STEP 3]
           Now $\cP_{\uptaum}$ uniquely determine the painted bipartition
           $\uptaum$.
   \end{description}
   The construction of the inverse map implies that $T_{\PPm,\wp}$ is a
   bijection.
 }

  \def\ckcOa{\ckcO^{\uparrow}}
  \def\PPa{\wp^{\uparrow}}
  \def\PPam{\wp^{\uparrow}_{\downarrow}}
  \def\uptaua{\uptau^{\uparrow}}
  \def\tauPPa{\tau_{\PPa}}
  \def\tauPPam{\tau_{\PPam}}
  \def\stara{\star^{\uparrow}}

  \medskip

  {Now suppose $\star \in \set{B,D}$.}
  We will prove the lemma by appealing to the corresponding assertion for the case of $\wtC/C$.
  Let
  \[
    \stara := \begin{cases}
      \wtC ,& \text{when $\star=B$},\\
      C ,& \text{when $\star=D$},
    \end{cases}
  \]
  and $\ckcOa$ be the partition defined by
  \[
    \bfrr_{1}(\ckcOa) = \bfrr_{1}(\ckcO)+2, \AND \bfrr_{i+1}(\ckcOa)
    = \bfrr_{i}(\ckcO)\quad \text{for all $i=1,2,3,\cdots,$}.
  \]
  Clearly
  \[
    \CPP_{\stara }(\ckcOa) = \set{(1,2)}\cup \set{(i+1,i+2)|(i,i+1)\in \CPP_{\star}(\ckcO)}.
  \]


  % Let $\PPa$ denote the subset in
  % $\CPP(\ckcOa)$ defined by
  % \[
  %   \PPa:=\set{(i+1,i+2)|(i,i+1)\in \sP}.
  % \]
  % Let $r_{0}:= \half \bfrr_{1}(\ckcO)+1 = \bfcc_{1}(\imath_{\PPa})$. For $x=c$
  % or $s$, define
  % \[
  %   \PBP_{\wtC}^{x}(\tauPPa):= \Set{\uptaua\in \PBP_{\wtC}(\tau_{\PPa})|\cP_{\uptaua}\left(r_{0},1\right)=x}
  % \]
  % where $\tauPPa$ is the bipartition defined with respect to $\ckcOa$.

  % Let $\PPam:=(\PPa)_{\downarrow}$. It is easy to check that
  % \begin{itemize}
  %   \item
  %   $\PBP_{\wtC}(\tauPPa) = \PBP_{\wtC}^{c}(\tauPPa)\sqcup \PBP_{\wtC}^{s}(\tauPPa)$,
  %   \item the map $T_{\PPa,\PPam}$ restricted into a bijection from
  %   $\PBP_{\wtC}^{x}(\tauPPa)$ onto $\PBP_{\wtC}^{x}(\tauPPam)$ (here $x=s$ or
  %   $c$), and
  %   \item the descent map restricted to bijections
  %   \[
  %     \text{
  %         $\PBP_{\wtC}^{s}(\tau_{\PPa})\longrightarrow \PBP_{B^{+}}(\tau_{\sP})$
  %         and
  %         $\PBP_{\wtC}^{c}(\tau_{\PPa})\longrightarrow \PBP_{B^{-}}(\tau_{\sP})$.
  %     }
  %   \]
  % \end{itemize}

  % Now the bijection $T_{\wp,\wpm}$ is defined by making the following diagram
  % commutative
  % \[
  %   \begin{tikzcd}
  %     \PBP_{\wtC}^{s}(\tau_{\PPa}) \ar[r] \ar[d] & \PBP_{B^{+}}^{s}(\tau_{\wp}) \ar[d]\\
  %     \PBP_{\wtC}^{s}(\tau_{\PPam}) \ar[r] & \PBP_{B^{+}}^{s}(\tau_{\PPm}) \\
  %   \end{tikzcd}
  % \]

 % For each subset $\sP\subset \CPP(\ckcO)$,
  Let $\PPa$ denote the subset in
  $\CPP_{\stara }(\ckcOa)$ defined by
  \[
    \PPa:=\set{(i+1,i+2)|(i,i+1)\in \sP}.
  \]
  % Let $r_{0}:= \half \bfrr_{1}(\ckcO)+1 = \bfcc_{1}(\imath_{\PPa})$.
  % For $x=c$
  % or $s$, define
  % \[
  %   \PBP_{\wtC}^{x}(\tauPPa):= \Set{\uptaua\in \PBP_{\wtC}(\tau_{\PPa})|\cP_{\uptaua}\left(r_{0},1\right)=x}
  % \]
  % where $\tauPPa$ is the bipartition defined with respect to $\ckcOa$.

Let $\PPam:=(\PPa)_{\downarrow}$. Recall from \cite[Section 2.3]{BMSZ2} the (naive) descent map $\DD$ of painted bipartitions.
It is easy to check that the descent maps (horizontal arrows) in the following diagram are bijections.
 %(c.f. \cite{BMSZ1}*{Lemma???}). % restricted to bijections
 % \[
 %   \PBP_{\wtC}(\tau_{\PPa})\longrightarrow \PBP_{B}(\tau_{\sP}).
 % \]
  \[
    \begin{tikzcd}
      \PBP_{\wtC}(\tau_{\PPam}) \ar[r,"\DD",two heads,hook] \ar[d,two heads,hook,"T_{\PPam,\PPa}"']
      & \PBP_{B}(\tau_{\PPm}) \ar[d,dashed,"T_{\PPm,\wp}"]\\
      \PBP_{\wtC}(\tau_{\PPa}) \ar[r,"\DD",two heads,hook] & \PBP_{B}(\tau_{\wp}) \\
    \end{tikzcd}
  \]
 Since there is a bijection $T_{\PPam,\PPa}$ in the left vertical arrow by case $\wtC$, we may define a bijection $T_{\PPm,\wp}$ in the right vertical arrow by making the above diagram commutative.
 This completes the proof for the case $B$. The case $D$ is entirely similar.
\end{proof}





\begin{bibdiv}
  \begin{biblist}
% \bib{AB}{article}{
%   title={Genuine representations of the metaplectic group},
%   author={Adams, Jeffrey},
%   author = {Barbasch, Dan},
%   journal={Compositio Mathematica},
%   volume={113},
%   number={01},
%   pages={23--66},
%   year={1998},
% }

\bib{Ad83}{article}{
  author = {Adams, J.},
  title = {Discrete spectrum of the reductive dual pair $(O(p,q),Sp(2m))$ },
  journal = {Invent. Math.},
  number = {3},
 pages = {449--475},
 volume = {74},
 year = {1983}
}

%\bib{Ad07}{article}{
%  author = {Adams, J.},
%  title = {The theta correspondence over R},
%  journal = {Harmonic analysis, group representations, automorphic forms and invariant theory,  Lect. Notes Ser. Inst. Math. Sci. Natl. Univ. Singap., 12},
% pages = {1--39},
% year = {2007}
% publisher={World Sci. Publ.}
%}


\bib{ABV}{book}{
  title={The Langlands classification and irreducible characters for real reductive groups},
  author={Adams, J.},
  author={Barbasch, D.},
  author={Vogan, D. A.},
  series={Progress in Math.},
  volume={104},
  year={1991},
  publisher={Birkhauser}
}

\bib{AC}{article}{
  title={Algorithms for representation theory of
    real reductive groups},
  volume={8},
  DOI={10.1017/S1474748008000352},
  number={2},
  journal={Journal of the Institute of Mathematics of Jussieu},
  publisher={Cambridge University Press},
  author={Adams, Jeffrey},
  author={du Cloux, Fokko},
  year={2009},
  pages={209-259}
}

\bib{ArPro}{article}{
  author = {Arthur, J.},
  title = {On some problems suggested by the trace formula},
  journal = {Lie group representations, II (College Park, Md.), Lecture Notes in Math. 1041},
 pages = {1--49},
 year = {1984}
}


\bib{ArUni}{article}{
  author = {Arthur, J.},
  title = {Unipotent automorphic representations: conjectures},
  %booktitle = {Orbites unipotentes et repr\'esentations, II},
  journal = {Orbites unipotentes et repr\'esentations, II, Ast\'erisque},
 pages = {13--71},
 volume = {171-172},
 year = {1989}
}

\bib{AK}{article}{
  author = {Auslander, L.},
  author = {Kostant, B.},
  title = {Polarizations and unitary representations of solvable Lie groups},
  journal = {Invent. Math.},
 pages = {255--354},
 volume = {14},
 year = {1971}
}


\bib{B.Uni}{article}{
  author = {Barbasch, D.},
  title = {Unipotent representations for real reductive groups},
 %booktitle = {Proceedings of ICM, Kyoto 1990},
 journal = {Proceedings of ICM (1990), Kyoto},
   % series = {Proc. Sympos. Pure Math.},
 %   volume = {68},
     pages = {769--777},
 publisher = {Springer-Verlag, The Mathematical Society of Japan},
      year = {2000},
}

\bib{Bo}{article}{
   author={Bo\v{z}i\v{c}evi\'{c}, Mladen},
   title={Double cells for unitary groups},
   journal={J. Algebra},
   volume={254},
   date={2002},
   number={1},
   pages={115--124},
   issn={0021-8693},
   review={\MR{1927434}},
   doi={10.1016/S0021-8693(02)00070-4},
}



\bib{BMSZ1}{article}{
      title={On the notion of metaplectic Barbasch-Vogan duality},
      year={2020},
      author={Barbasch, Dan M.},
      author = {Ma, Jia-jun},
      author = {Sun, Binyong},
      author = {Zhu, Chen-Bo},
      eprint={2010.16089},
      archivePrefix={arXiv},
      primaryClass={math.RT}
}

\bib{BMSZ2}{article}{
      title={Special unipotent representations: orthogonal and symplectic groups},
      author={Barbasch, Dan M.},
      author = {Ma, Jia-jun},
      author = {Sun, Binyong},
      author = {Zhu, Chen-Bo},
      year={2021},
      eprint={1712.05552},
      archivePrefix={arXiv},
      primaryClass={math.RT}
}

\bib{BV1}{article}{
   author={Barbasch, Dan},
   author={Vogan, David},
   title={Primitive ideals and orbital integrals in complex classical
   groups},
   journal={Math. Ann.},
   volume={259},
   date={1982},
   number={2},
   pages={153--199},
   issn={0025-5831},
   review={\MR{656661}},
   doi={10.1007/BF01457308},
}

\bib{BV2}{article}{
   author={Barbasch, Dan},
   author={Vogan, David},
   title={Primitive ideals and orbital integrals in complex exceptional
   groups},
   journal={J. Algebra},
   volume={80},
   date={1983},
   number={2},
   pages={350--382},
   issn={0021-8693},
   review={\MR{691809}},
   doi={10.1016/0021-8693(83)90006-6},
}

\bib{BV.W}{article}{
  author={Barbasch, Dan},
  author={Vogan, David},
  editor={Trombi, P. C.},
  title={Weyl Group Representations and Nilpotent Orbits},
  bookTitle={Representation Theory of Reductive Groups:
    Proceedings of the University of Utah Conference 1982},
  year={1983},
  publisher={Birkh{\"a}user Boston},
  address={Boston, MA},
  pages={21--33},
  %doi={10.1007/978-1-4684-6730-7_2},
}



\bib{B.Orbit}{article}{
  author = {Barbasch, D.},
  title = {Orbital integrals of nilpotent orbits},
 %booktitle = {The mathematical legacy of {H}arish-{C}handra ({B}altimore,{MD}, 1998)},
    journal = {The mathematical legacy of {H}arish-{C}handra, Proc. Sympos. Pure Math.},
    %series={The mathematical legacy of {H}arish-{C}handra, Proc. Sympos. Pure Math},
    volume = {68},
     pages = {97--110},
 publisher = {Amer. Math. Soc., Providence, RI},
      year = {2000},
}



\bib{B10}{article}{
  author = {Barbasch, D.},
  title = {The unitary spherical spectrum for split classical groups},
  journal = {J. Inst. Math. Jussieu},
% number = {9},
 pages = {265--356},
 volume = {9},
 year = {2010}
}



\bib{B17}{article}{
  author = {Barbasch, D.},
  title = {Unipotent representations and the dual pair correspondence},
  journal = {J. Cogdell et al. (eds.), Representation Theory, Number Theory, and Invariant Theory, In Honor of Roger Howe. Progress in Math.},
  %series ={Progress in Math.},
  volume = {323},
  pages = {47--85},
  year = {2017},
}

\bib{BVUni}{article}{
 author = {Barbasch, D.},
 author = {Vogan, D. A.},
 journal = {Annals of Math.},
 number = {1},
 pages = {41--110},
 title = {Unipotent representations of complex semisimple groups},
 volume = {121},
 year = {1985}
}

\bib{BB}{article}{
   author={Beilinson, Alexandre},
   author={Bernstein, Joseph},
   title={Localisation de $\mathfrak g$-modules},
   language={French, with English summary},
   journal={C. R. Acad. Sci. Paris S\'{e}r. I Math.},
   volume={292},
   date={1981},
   number={1},
   pages={15--18},
   issn={0249-6291},
   review={\MR{610137}},
}

\bib{Br}{article}{
  author = {Brylinski, R.},
  title = {Dixmier algebras for classical complex nilpotent orbits via Kraft-Procesi models. I},
  journal = {The orbit method in geometry and physics (Marseille, 2000). Progress in Math.}
  volume = {213},
  pages = {49--67},
  year = {2003},
}

\bib{BK}{article}{
   author={Brylinski, J.-L.},
   author={Kashiwara, M.},
   title={Kazhdan-Lusztig conjecture and holonomic systems},
   journal={Invent. Math.},
   volume={64},
   date={1981},
   number={3},
   pages={387--410},
   issn={0020-9910},
   review={\MR{632980}},
   doi={10.1007/BF01389272},
}

\bib{Carter}{book}{
   author={Carter, Roger W.},
   title={Finite groups of Lie type},
   series={Wiley Classics Library},
   %note={Conjugacy classes and complex characters;
   %Reprint of the 1985 original;
   %A Wiley-Interscience Publication},
   publisher={John Wiley \& Sons, Ltd., Chichester},
   date={1993},
   pages={xii+544},
   isbn={0-471-94109-3},
   %review={\MR{1266626}},
}

\bib{Cas}{article}{
   author={Casian, Luis G.},
   title={Primitive ideals and representations},
   journal={J. Algebra},
   volume={101},
   date={1986},
   number={2},
   pages={497--515},
   issn={0021-8693},
   review={\MR{847174}},
   doi={10.1016/0021-8693(86)90208-5},
}

\bib{Ca89}{article}{
 author = {Casselman, W.},
 journal = {Canad. J. Math.},
 pages = {385--438},
 title = {Canonical extensions of Harish-Chandra modules to representations of $G$},
 volume = {41},
 year = {1989}
}



\bib{Cl}{article}{
  author = {Du Cloux, F.},
  journal = {Ann. Sci. \'Ecole Norm. Sup.},
  number = {3},
  pages = {257--318},
  title = {Sur les repr\'esentations diff\'erentiables des groupes de Lie alg\'ebriques},
  url = {http://eudml.org/doc/82297},
  volume = {24},
  year = {1991},
}

\bib{CM}{book}{
  title = {Nilpotent orbits in semisimple Lie algebra: an introduction},
  author = {Collingwood, D. H.},
  author = {McGovern, W. M.},
  year = {1993},
  publisher = {Van Nostrand Reinhold Co.},
}


% \bib{Dieu}{book}{
%    title={La g\'{e}om\'{e}trie des groupes classiques},
%    author={Dieudonn\'{e}, Jean},
%    year={1963},
% 	publisher={Springer},
%  }

\bib{DKPC}{article}{
title = {Nilpotent orbits and complex dual pairs},
journal = {J. Algebra},
volume = {190},
number = {2},
pages = {518 - 539},
year = {1997},
author = {Daszkiewicz, A.},
author = {Kra\'skiewicz, W.},
author = {Przebinda, T.},
}

\bib{DKP2}{article}{
  author = {Daszkiewicz, A.},
  author = {Kra\'skiewicz, W.},
  author = {Przebinda, T.},
  title = {Dual pairs and Kostant-Sekiguchi correspondence. II. Classification
	of nilpotent elements},
  journal = {Central European J. Math.},
  year = {2005},
  volume = {3},
  pages = {430--474},
}


\bib{DM}{article}{
  author = {Dixmier, J.},
  author = {Malliavin, P.},
  title = {Factorisations de fonctions et de vecteurs ind\'efiniment diff\'erentiables},
  journal = {Bull. Sci. Math. (2)},
  year = {1978},
  volume = {102},
  pages = {307--330},
}

%\bibitem[DM]{DM}
%J. Dixmier and P. Malliavin, \textit{Factorisations de fonctions et de vecteurs ind\'efiniment diff\'erentiables}, Bull. Sci. Math. (2), 102 (4),  307-330 (1978).



\bib{Du77}{article}{
  author = {Duflo, M.},
  journal = {Annals of Math.},
  number = {1},
  pages = {107-120},
  title = {Sur la Classification des Ideaux Primitifs Dans
    L'algebre Enveloppante d'une Algebre de Lie Semi-Simple},
  volume = {105},
  year = {1977}
}

\bib{Du82}{article}{
 author = {Duflo, M.},
 journal = {Acta Math.},
  volume = {149},
 number = {3-4},
 pages = {153--213},
 title = {Th\'eorie de Mackey pour les groupes de Lie alg\'ebriques},
 year = {1982}
}



\bib{GZ}{article}{
author={Gomez, R.},
author={Zhu, C.-B.},
title={Local theta lifting of generalized Whittaker models associated to nilpotent orbits},
journal={Geom. Funct. Anal.},
year={2014},
volume={24},
number={3},
pages={796--853},
}

\bib{EGAIV2}{article}{
  title = {\'El\'ements de g\'eom\'etrie alg\'brique IV: \'Etude locale des
    sch\'emas et des morphismes de sch\'emas. II},
  author = {Grothendieck, A.},
  author = {Dieudonn\'e, J.},
  journal  = {Inst. Hautes \'Etudes Sci. Publ. Math.},
  volume = {24},
  year = {1965},
}


\bib{EGAIV3}{article}{
  title = {\'El\'ements de g\'eom\'etrie alg\'brique IV: \'Etude locale des
    sch\'emas et des morphismes de sch\'emas. III},
  author = {Grothendieck, A.},
  author = {Dieudonn\'e, J.},
  journal  = {Inst. Hautes \'Etudes Sci. Publ. Math.},
  volume = {28},
  year = {1966},
}

\bib{GI}{article}{
   author={Gan, Wee Teck},
   author={Ichino, Atsushi},
   title={On the irreducibility of some induced representations of real
   reductive Lie groups},
   journal={Tunis. J. Math.},
   volume={1},
   date={2019},
   number={1},
   pages={73--107},
   issn={2576-7658},
   review={\MR{3907735}},
   doi={10.2140/tunis.2019.1.73},
}

\bib{HLS}{article}{
    author = {Harris, M.},
    author = {Li, J.-S.},
    author = {Sun, B.},
     title = {Theta correspondences for close unitary groups},
 %booktitle = {Arithmetic Geometry and Automorphic Forms},
    %series = {Adv. Lect. Math. (ALM)},
    journal = {Arithmetic Geometry and Automorphic Forms, Adv. Lect. Math. (ALM)},
    volume = {19},
     pages = {265--307},
 publisher = {Int. Press, Somerville, MA},
      year = {2011},
}

\bib{HS}{book}{
 author = {Hartshorne, R.},
 title = {Algebraic Geometry},
publisher={Graduate Texts in Mathematics, 52. New York-Heidelberg-Berlin: Springer-Verlag},
year={1983},
}

\bib{He}{article}{
author={He, H.},
title={Unipotent representations and quantum induction},
journal={arXiv:math/0210372},
year = {2002},
}

\bib{HL}{article}{
author={Huang, J.-S.},
author={Li, J.-S.},
title={Unipotent representations attached to spherical nilpotent orbits},
journal={Amer. J. Math.},
volume={121},
number = {3},
pages={497--517},
year={1999},
}


\bib{HZ}{article}{
author={Huang, J.-S.},
author={Zhu, C.-B.},
title={On certain small representations of indefinite orthogonal groups},
journal={Represent. Theory},
volume={1},
pages={190--206},
year={1997},
}



\bib{Howe79}{article}{
  title={$\Phi$-series and invariant theory},
  author={Howe, R.},
  book = {
    title={Automorphic Forms, Representations and $L$-functions},
    series={Proc. Sympos. Pure Math},
    volume={33},
    year={1979},
  },
  pages={275-285},
}

\bib{HoweRank}{article}{
author={Howe, R.},
title={On a notion of rank for unitary representations of the classical groups},
journal={Harmonic analysis and group representations, Liguori, Naples},
pages={223-331},
year={1982},
}

\bib{Howe89}{article}{
author={Howe, R.},
title={Transcending classical invariant theory},
journal={J. Amer. Math. Soc.},
volume={2},
pages={535--552},
year={1989},
}

\bib{Howe95}{article}{,
  author = {Howe, R.},
  title = {Perspectives on invariant theory: Schur duality, multiplicity-free actions and beyond},
  journal = {Piatetski-Shapiro, I. et al. (eds.), The Schur lectures (1992). Ramat-Gan: Bar-Ilan University, Isr. Math. Conf. Proc. 8,},
  year = {1995},
  pages = {1-182},
}

\bib{H}{book}{
   author={Humphreys, James E.},
   title={Representations of semisimple Lie algebras in the BGG category
   $\scr{O}$},
   series={Graduate Studies in Mathematics},
   volume={94},
   publisher={American Mathematical Society, Providence, RI},
   date={2008},
   pages={xvi+289},
   isbn={978-0-8218-4678-0},
   review={\MR{2428237}},
   doi={10.1090/gsm/094},
}

\bib{J1}{article}{
   author={Joseph, A.},
   title={Goldie rank in the enveloping algebra of a semisimple Lie algebra.
   I},
   journal={J. Algebra},
   volume={65},
   date={1980},
   number={2},
   pages={269--283},
   issn={0021-8693},
   review={\MR{585721}},
   doi={10.1016/0021-8693(80)90217-3},
}

\bib{J2}{article}{
   author={Joseph, A.},
   title={Goldie rank in the enveloping algebra of a semisimple Lie algebra.
   II},
   journal={J. Algebra},
   volume={65},
   date={1980},
   number={2},
   pages={284--306},
   issn={0021-8693},
   review={\MR{585721}},
   doi={10.1016/0021-8693(80)90217-3},
}

\bib{J3}{article}{
   author={Joseph, A.},
   title={Goldie rank in the enveloping algebra of a semisimple Lie algebra.
   III},
   journal={J. Algebra},
   volume={73},
   date={1981},
   number={2},
   pages={295--326},
   issn={0021-8693},
   review={\MR{640039}},
   doi={10.1016/0021-8693(81)90324-0},
}



\bib{J.av}{article}{
   author={Joseph, Anthony},
   title={On the associated variety of a primitive ideal},
   journal={J. Algebra},
   volume={93},
   date={1985},
   number={2},
   pages={509--523},
   issn={0021-8693},
   review={\MR{786766}},
   doi={10.1016/0021-8693(85)90172-3},
}

\bib{J.ann}{article}{
   author={Joseph, Anthony},
   title={Annihilators and associated varieties of unitary highest weight
   modules},
   journal={Ann. Sci. \'{E}cole Norm. Sup. (4)},
   volume={25},
   date={1992},
   number={1},
   pages={1--45},
   issn={0012-9593},
   review={\MR{1152612}},
}

\bib{J.hw}{article}{
   author={Joseph, Anthony},
   title={On the variety of a highest weight module},
   journal={J. Algebra},
   volume={88},
   date={1984},
   number={1},
   pages={238--278},
   issn={0021-8693},
   review={\MR{741942}},
   doi={10.1016/0021-8693(84)90100-5},
}


\bib{JLS}{article}{
author={Jiang, D.},
author={Liu, B.},
author={Savin, G.},
title={Raising nilpotent orbits in wave-front sets},
journal={Represent. Theory},
volume={20},
pages={419--450},
year={2016},
}

\bib{Ki62}{article}{
author={Kirillov, A. A.},
title={Unitary representations of nilpotent Lie groups},
journal={Uspehi Mat. Nauk},
volume={17},
issue ={4},
pages={57--110},
year={1962},
}


\bib{Ko70}{article}{
author={Kostant, B.},
title={Quantization and unitary representations},
journal={Lectures in Modern Analysis and Applications III, Lecture Notes in Math.},
volume={170},
pages={87--208},
year={1970},
}


\bib{KP}{article}{
author={Kraft, H.},
author={Procesi, C.},
title={On the geometry of conjugacy classes in classical groups},
journal={Comment. Math. Helv.},
volume={57},
pages={539--602},
year={1982},
}

\bib{KR}{article}{
author={Kudla, S. S.},
author={Rallis, S.},
title={Degenerate principal series and invariant distributions},
journal={Israel J. Math.},
volume={69},
pages={25--45},
year={1990},
}


\bib{Ku}{article}{
author={Kudla, S. S.},
title={Some extensions of the Siegel-Weil formula},
journal={In: Gan W., Kudla S., Tschinkel Y. (eds) Eisenstein Series and Applications. Progress in Mathematics, vol 258. Birkh\"auser Boston},
%volume={69},
pages={205--237},
year={2008},
}





\bib{LZ1}{article}{
author={Lee, S. T.},
author={Zhu, C.-B.},
title={Degenerate principal series and local theta correspondence II},
journal={Israel J. Math.},
volume={100},
pages={29--59},
year={1997},
}

\bib{LZ2}{article}{
author={Lee, S. T.},
author={Zhu, C.-B.},
title={Degenerate principal series of metaplectic groups and Howe correspondence},
journal = {D. Prasad at al. (eds.), Automorphic Representations and L-Functions, Tata Institute of Fundamental Research, India,},
year = {2013},
pages = {379--408},
}

\bib{Li89}{article}{
author={Li, J.-S.},
title={Singular unitary representations of classical groups},
journal={Invent. Math.},
volume={97},
number = {2},
pages={237--255},
year={1989},
}

\bib{LiuAG}{book}{
  title={Algebraic Geometry and Arithmetic Curves},
  author = {Liu, Q.},
  year = {2006},
  publisher={Oxford University Press},
}

\bib{LM}{article}{
   author = {Loke, H. Y.},
   author = {Ma, J.},
    title = {Invariants and $K$-spectrums of local theta lifts},
    journal = {Compositio Math.},
    volume = {151},
    issue = {01},
    year = {2015},
    pages ={179--206},
}

\bib{DL}{article}{
   author={Deligne, P.},
   author={Lusztig, G.},
   title={Representations of reductive groups over finite fields},
   journal={Ann. of Math. (2)},
   volume={103},
   date={1976},
   number={1},
   pages={103--161},
   issn={0003-486X},
   review={\MR{393266}},
   doi={10.2307/1971021},
}

\bib{KL}{article}{
   author={Kazhdan, David},
   author={Lusztig, George},
   title={Representations of Coxeter groups and Hecke algebras},
   journal={Invent. Math.},
   volume={53},
   date={1979},
   number={2},
   pages={165--184},
   issn={0020-9910},
   review={\MR{560412}},
   doi={10.1007/BF01390031},
}

\bib{Lu}{book}{
   author={Lusztig, George},
   title={Characters of reductive groups over a finite field},
   series={Annals of Mathematics Studies},
   volume={107},
   publisher={Princeton University Press, Princeton, NJ},
   date={1984},
   pages={xxi+384},
   isbn={0-691-08350-9},
   isbn={0-691-08351-7},
   review={\MR{742472}},
   doi={10.1515/9781400881772},
}


\bib{LS}{article}{
   author = {Lusztig, G.},
   author = {Spaltenstein, N.},
    title = {Induced unipotent classes},
    journal = {j. London Math. Soc.},
    volume = {19},
    year = {1979},
    pages ={41--52},
}

\bib{Lu.I}{article}{
   author={Lusztig, G.},
   title={Intersection cohomology complexes on a reductive group},
   journal={Invent. Math.},
   volume={75},
   date={1984},
   number={2},
   pages={205--272},
   issn={0020-9910},
   review={\MR{732546}},
   doi={10.1007/BF01388564},
}


\bib{Ma}{article}{
   author = {Mackey, G. W.},
    title = {Unitary representations of group extentions},
    journal = {Acta Math.},
    volume = {99},
    year = {1958},
    pages ={265--311},
}

\bib{Mat}{article}{
   author={Matumoto, Hisayosi},
   title={On the representations of ${\rm Sp}(p,q)$ and ${\rm SO}^*(2n)$
   unitarily induced from derived functor modules},
   journal={Compos. Math.},
   volume={140},
   date={2004},
   number={4},
   pages={1059--1096},
   issn={0010-437X},
   review={\MR{2059231}},
   doi={10.1112/S0010437X03000629},
}

\bib{Mc}{article}{
   author = {McGovern, W. M},
    title = {Cells of Harish-Chandra modules for real classical groups},
    journal = {Amer. J.  of Math.},
    volume = {120},
    issue = {01},
    year = {1998},
    pages ={211--228},
}

\bib{Mo96}{article}{
 author={M{\oe}glin, C.},
    title = {Front d'onde des repr\'esentations des groupes classiques $p$-adiques},
    journal = {Amer. J. Math.},
    volume = {118},
    issue = {06},
    year = {1996},
    pages ={1313--1346},
}

\bib{Mo17}{article}{
  author={M{\oe}glin, C.},
  title = {Paquets d'Arthur Sp\'eciaux Unipotents aux Places Archim\'ediennes et Correspondance de Howe},
  journal = {J. Cogdell et al. (eds.), Representation Theory, Number Theory, and Invariant Theory, In Honor of Roger Howe. Progress in Math.}
  %series ={Progress in Math.},
  volume = {323},
  pages = {469--502}
  year = {2017}
}

\bib{MR19}{article}{
   author={M\oe glin, Colette},
   author={Renard, David},
   title={Sur les paquets d'Arthur des groupes unitaires et quelques
   cons\'{e}quences pour les groupes classiques},
   language={French, with English and French summaries},
   journal={Pacific J. Math.},
   volume={299},
   date={2019},
   number={1},
   pages={53--88},
   issn={0030-8730},
   review={\MR{3947270}},
   doi={10.2140/pjm.2019.299.53},
}


\bib{MVW}{book}{
  volume={1291},
  title={Correspondances de Howe sur un corps $p$-adique},
  author={M{\oe}glin, C.},
  author={Vign\'eras, M.-F.},
  author={Waldspurger, J.-L.},
  series={Lecture Notes in Mathematics},
  publisher={Springer}
  ISBN={978-3-540-18699-1},
  date={1987},
}

\bib{NOTYK}{article}{
   author = {Nishiyama, K.},
   author = {Ochiai, H.},
   author = {Taniguchi, K.},
   author = {Yamashita, H.},
   author = {Kato, S.},
    title = {Nilpotent orbits, associated cycles and Whittaker models for highest weight representations},
    journal = {Ast\'erisque},
    volume = {273},
    year = {2001},
   pages ={1--163},
}

\bib{NOZ}{article}{
  author = {Nishiyama, K.},
  author = {Ochiai, H.},
  author = {Zhu, C.-B.},
  journal = {Trans. Amer. Math. Soc.},
  title = {Theta lifting of nilpotent orbits for symmetric pairs},
  volume = {358},
  year = {2006},
  pages = {2713--2734},
}


\bib{NZ}{article}{
   author = {Nishiyama, K.},
   author = {Zhu, C.-B.},
    title = {Theta lifting of unitary lowest weight modules and their associated cycles},
    journal = {Duke Math. J.},
    volume = {125},
    number= {03},
    year = {2004},
   pages ={415--465},
}



\bib{Ohta}{article}{
  author = {Ohta, T.},
  %doi = {10.2748/tmj/1178227492},
  journal = {Tohoku Math. J.},
  number = {2},
  pages = {161--211},
  publisher = {Tohoku University, Mathematical Institute},
  title = {The closures of nilpotent orbits in the classical symmetric
    pairs and their singularities},
  volume = {43},
  year = {1991}
}

\bib{Ohta2}{article}{
  author = {Ohta, T.},
  journal = {Hiroshima Math. J.},
  number = {2},
  pages = {347--360},
  title = {Induction of nilpotent orbits for real reductive groups and associated varieties of standard representations},
  volume = {29},
  year = {1999}
}

\bib{Ohta4}{article}{
  title={Nilpotent orbits of $\mathbb{Z}_4$-graded Lie algebra and geometry of
    moment maps associated to the dual pair $(\mathrm{U} (p, q), \mathrm{U} (r, s))$},
  author={Ohta, T.},
  journal={Publ. RIMS},
  volume={41},
  number={3},
  pages={723--756},
  year={2005}
}

\bib{PT}{article}{
  title={Some small unipotent representations of indefinite orthogonal groups and the theta correspondence},
  author={Paul, A.},
  author={Trapa, P.},
  journal={University of Aarhus Publ. Series},
  volume={48},
  pages={103--125},
  year={2007}
}


\bib{PV}{article}{
  title={Invariant Theory},
  author={Popov, V. L.},
  author={Vinberg, E. B.},
  book={
  title={Algebraic Geometry IV: Linear Algebraic Groups, Invariant Theory},
  series={Encyclopedia of Mathematical Sciences},
  volume={55},
  year={1994},
  publisher={Springer},}
}




%\bib{PPz}{article}{
%author={Protsak, V.} ,
%author={Przebinda, T.},
%title={On the occurrence of admissible representations in the real Howe
%    correspondence in stable range},
%journal={Manuscr. Math.},
%volume={126},
%number={2},
%pages={135--141},
%year={2008}
%}


\bib{PrzInf}{article}{
      author={Przebinda, T.},
       title={The duality correspondence of infinitesimal characters},
        date={1996},
     journal={Colloq. Math.},
      volume={70},
       pages={93--102},
}


\bib{Pz}{article}{
author={Przebinda, T.},
title={Characters, dual pairs, and unitary representations},
journal={Duke Math. J. },
volume={69},
number={3},
pages={547--592},
year={1993}
}

\bib{Ra}{article}{
author={Rallis, S.},
title={On the Howe duality conjecture},
journal={Compositio Math.},
volume={51},
pages={333--399},
year={1984}
}

\bib{RT1}{article}{
   author={Renard, David A.},
   author={Trapa, Peter E.},
   title={Irreducible genuine characters of the metaplectic group:
   Kazhdan-Lusztig algorithm and Vogan duality},
   journal={Represent. Theory},
   volume={4},
   date={2000},
   pages={245--295},
   review={\MR{1795754}},
   doi={10.1090/S1088-4165-00-00105-9},
}

\bib{RT2}{article}{
   author={Renard, David A.},
   author={Trapa, Peter E.},
   title={Irreducible characters of the metaplectic group. II.
   Functoriality},
   journal={J. Reine Angew. Math.},
   volume={557},
   date={2003},
   pages={121--158},
   issn={0075-4102},
   review={\MR{1978405}},
   doi={10.1515/crll.2003.028},
}

\bib{RT3}{article}{
   author={Renard, David A.},
   author={Trapa, Peter E.},
   title={Kazhdan-Lusztig algorithms for nonlinear groups and applications
   to Kazhdan-Patterson lifting},
   journal={Amer. J. Math.},
   volume={127},
   date={2005},
   number={5},
   pages={911--971},
   issn={0002-9327},
   review={\MR{2170136}},
}


\bib{Sa}{article}{
author={Sahi, S.},
title={Explicit Hilbert spaces for certain unipotent representations},
journal={Invent. Math.},
volume={110},
number = {2},
pages={409--418},
year={1992}
}

\bib{Se}{article}{
author={Sekiguchi, J.},
title={Remarks on real nilpotent orbits of a symmetric pair},
journal={J. Math. Soc. Japan},
%publisher={The Mathematical Society of Japan},
year={1987},
volume={39},
number={1},
pages={127--138},
}

\bib{SV}{article}{
  author = {Schmid, W.},
  author = {Vilonen, K.},
  journal = {Annals of Math.},
  number = {3},
  pages = {1071--1118},
  %publisher = {Princeton University, Mathematics Department, Princeton, NJ; Mathematical Sciences Publishers, Berkeley},
  title = {Characteristic cycles and wave front cycles of representations of reductive Lie groups},
  volume = {151},
year = {2000},
}


\bib{Soergel}{article}{
   author={Soergel, Wolfgang},
   title={Kategorie $\scr O$, perverse Garben und Moduln \"{u}ber den
   Koinvarianten zur Weylgruppe},
   language={German, with English summary},
   journal={J. Amer. Math. Soc.},
   volume={3},
   date={1990},
   number={2},
   pages={421--445},
   issn={0894-0347},
   review={\MR{1029692}},
   doi={10.2307/1990960},
}


\bib{So}{article}{
author = {Sommers, E.},
title = {Lusztig's canonical quotient and generalized duality},
journal = {J. Algebra},
volume = {243},
number = {2},
pages = {790--812},
year = {2001},
}

\bib{SS}{book}{
  author = {Springer, T. A.},
  author = {Steinberg, R.},
  title = {Seminar on algebraic groups and related finite groups; Conjugate classes},
  series = {Lecture Notes in Math.},
  volume = {131},
publisher={Springer},
year={1970},
}

\bib{SZ1}{article}{
title={A general form of Gelfand-Kazhdan criterion},
author={Sun, B.},
author={Zhu, C.-B.},
journal={Manuscripta Math.},
pages = {185--197},
volume = {136},
year={2011}
}


%\bib{SZ2}{article}{
%  title={Conservation relations for local theta correspondence},
%  author={Sun, B.},
%  author={Zhu, C.-B.},
%  journal={J. Amer. Math. Soc.},
%  pages = {939--983},
%  volume = {28},
%  year={2015}
%}



\bib{Tr}{article}{
  title={Special unipotent representations and the Howe correspondence},
  author={Trapa, P.},
  year = {2004},
  journal={University of Aarhus Publication Series},
  volume = {47},
  pages= {210--230}
}

% \bib{Wa}{article}{
%    author = {Waldspurger, J.-L.},
%     title = {D\'{e}monstration d'une conjecture de dualit\'{e} de Howe dans le cas $p$-adique, $p \neq 2$ in Festschrift in honor of I. I. Piatetski-Shapiro on the occasion of his sixtieth birthday},
%   journal = {Israel Math. Conf. Proc., 2, Weizmann, Jerusalem},
%  year = {1990},
% pages = {267-324},
% }

\bib{VGK}{article}{
   author={Vogan, David A., Jr.},
   title={Gel\cprime fand-Kirillov dimension for Harish-Chandra modules},
   journal={Invent. Math.},
   volume={48},
   date={1978},
   number={1},
   pages={75--98},
   issn={0020-9910},
   review={\MR{506503}},
   doi={10.1007/BF01390063},
}



\bib{Vg}{book}{
   author={Vogan, David A.},
   title={Representations of real reductive Lie groups},
   series={Progress in Mathematics},
   volume={15},
   publisher={Birkh\"{a}user, Boston, Mass.},
   date={1981},
   pages={xvii+754},
   isbn={3-7643-3037-6},
   review={\MR{632407}},
}

\bib{V4}{article}{
   author={Vogan, D. A. },
   title={Irreducible characters of semisimple Lie groups. IV.
   Character-multiplicity duality},
   journal={Duke Math. J.},
   volume={49},
   date={1982},
   number={4},
   pages={943--1073},
   issn={0012-7094},
   review={\MR{683010}},
}


\bib{V.GL}{article}{
   author={Vogan, David A., Jr.},
   title={The unitary dual of ${\rm GL}(n)$ over an Archimedean field},
   journal={Invent. Math.},
   volume={83},
   date={1986},
   number={3},
   pages={449--505},
   issn={0020-9910},
   review={\MR{827363}},
   doi={10.1007/BF01394418},
}

\bib{VoBook}{book}{
author = {Vogan, D. A. },
  title={Unitary representations of reductive Lie groups},
  year={1987},
  series = {Ann. of Math. Stud.},
 volume={118},
  publisher={Princeton University Press}
}


\bib{Vo89}{article}{
  author = {Vogan, D. A. },
  title = {Associated varieties and unipotent representations},
 %booktitle ={Harmonic analysis on reductive groups, Proc. Conf., Brunswick/ME (USA) 1989,},
  journal = {Harmonic analysis on reductive groups, Proc. Conf., Brunswick/ME
    (USA) 1989, Prog. Math.},
 volume={101},
  publisher = {Birkh\"{a}user, Boston-Basel-Berlin},
  year = {1991},
pages={315--388},
  editor = {W. Barker and P. Sally},
}

\bib{Vo98}{article}{
  author = {Vogan, D. A. },
  title = {The method of coadjoint orbits for real reductive groups},
 %booktitle ={Representation theory of Lie groups (Park City, UT, 1998)},
 journal = {Representation theory of Lie groups (Park City, UT, 1998). IAS/Park City Math. Ser.},
  volume={8},
  publisher = {Amer. Math. Soc.},
  year = {2000},
pages={179--238},
}




\bib{Vo00}{article}{
  author = {Vogan, D. A. },
  title = {Unitary representations of reductive Lie groups},
 %booktitle ={Mathematics towards the Third Millennium (Rome, 1999)},
 journal ={Mathematics towards the Third Millennium (Rome, 1999). Accademia Nazionale dei Lincei, (2000)},
  %series = {Accademia Nazionale dei Lincei, 2000},
 %volume={9},
pages={147--167},
}


\bib{Wa1}{book}{
  title={Real reductive groups I},
  author={Wallach, N. R.},
  year={1988},
  publisher={Academic Press Inc. }
}

\bib{Wa2}{book}{
  title={Real reductive groups II},
  author={Wallach, N. R.},
  year={1992},
  publisher={Academic Press Inc. }
}


\bib{Weyl}{book}{
  title={The classical groups: their invariants and representations},
  author={Weyl, H.},
  year={1947},
  publisher={Princeton University Press}
}

\bib{Ya}{article}{
  title={Degenerate principal series representations for quaternionic unitary groups},
  author={Yamana, S.},
  year = {2011},
  journal={Israel J. Math.},
  volume = {185},
  pages= {77--124}
}


\bib{Zu}{article}{
  title={Tensor products of finite and infinite dimensional representations of semisimple Lie groups},
  author={Zuckerman, G.},
  year = {1977},
  journal={Ann. of Math. 106},
  volume = {106},
  pages= {295--308}
}


% \bib{EGAIV4}{article}{
%   title = {\'El\'ements de g\'eom\'etrie alg\'brique IV 4: \'Etude locale des
%     sch\'emas et des morphismes de sch\'emas},
%   author = {Grothendieck, Alexandre},
%   author = {Dieudonn\'e, Jean},
%   journal  = {Inst. Hautes \'Etudes Sci. Publ. Math.},
%   volume = {32},
%   year = {1967},
%   pages = {5--361}
% }



\end{biblist}
\end{bibdiv}


\end{document}


%%% Local Variables:
%%% coding: utf-8
%%% mode: latex
%%% TeX-engine: tex
%%% ispell-local-dictionary: "en_US"
%%% End:
