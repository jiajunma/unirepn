\documentclass[counting_main.tex]{subfiles}
\begin{document}

\section{Unipotent representations of general linear groups}

The classification of unipotent representations of general linear groups is well
known to the experts. We review the result and provide some proves due to the
lack of reference.

% We assume that $G  = \GL_{n}(\bC), \GL_{n}(\bR), \GL_{\half n}(\bH)$ and $\star = A,A^{\bC}, A^{\bH}$ respectively.

In all the cases, $\ckG = \GL_{n}(\bC)$ and $n$ is even if
$G = \GL_{\frac{n}{2}}(\bH)$.
The nilpotent orbit $\ckcO$ is parameterized by Young diagrams.
Let $\YD_{n}$ be the set of Young diagrams with $n$-boxes.

Fix $\ckcO\in \Nil(\ckG) = \YD_{n}$ and set
\[
  \cO :=\dBV(\ckcO) = \ckcO^{t} \AND \wttau := \Spr^{-1}(\cO).
\]
Here $\wttau$ has the partition $\cO$.



\subsection{Special unipotent representations of $G = \GL_n(\bC)$}
When $G = \GL_{n}(\bC)$ the classification is a special case in \cite{BVUni}.

Since the Lusztig canonical quotient of $\ckcO$ is trivial, the set of unipotent representations
of $G = \GL_n(\bC)$ one-one corresponds to nilpotent orbits in $\Nil(\ckGc) = \YD_{n}$ by
\cite{BVUni}.

\begin{thm}
Suppose $\ckcO\in \YD_{n} = \Nil(\ckGc)$ and $\ckcO$ has $k$ columns.
Let
\[
 \pi_{\ckcO} := 1_{\bfrr_1(\ckcO)}\times  1_{\bfrr_2(\ckcO)}\times \cdots
 \times 1_{\bfrr_k(\ckcO)}
\]
be the normalized parabolic induction where $1_{c}$ denote the trivial
representations of $\GL_c(\bC)$.
Then
\[
  \Unip_{A^{\bC}}(\ckcO) = \Set{\pi_{\ckcO}}.
\]
\end{thm}


\begin{remark}
The Vogan duality gives a duality between Harish-Chandra cells.
In this case, Harish-Chandra cells is the double cell
of Lusztig.
Now we have a duality
\[
 \pi_\ckcO \leftrightarrow \pi_{\ckcO^t}.
\]
\end{remark}

We record the following easy lemma which is a baby case of our results on other
classical groups.
\begin{lem}
  Let $\ckcO' := \DDD(\ckcO)$ be the partition obtained by deleting the first
  column of $\ckcO$. Let $\theta_{\GL_{a}(\bC),\GL_{b}(\bC)}$ (resp.
  $\Theta_{\GL_{a}(\bC),\GL_{b}(\bC)}$) be the theta lift (resp. big theta lift)
  from $\GL_a(\bC)$ to $\GL_b(\bC)$. Then we have
  \[
    \pi_{\ckcO} = \theta_{\GL_{\abs{\ckcO'}}(\bC),\GL_{\abs{\ckcO}}(\bC)} (\pi_{\ckcO'})= \Theta_{\GL_{\abs{\ckcO'}}(\bC),\GL_{\abs{\ckcO}}(\bC)} (\pi_{\ckcO'}).
  \]
\end{lem}


\subsection{Special unipotent representations of $\GL_n(\bR)$ and
  $\GL_{\frac{n}{2}}(\bR)$}
In this section, we assume $G = \GL_{n}(\bR)$ or $\GL_{\frac{n}{2}}(\bR)$, i.e.
$\star\in \set{A,A^{\bH}}$.


Suppose there is decomposition %$\ckcO\in \Nil(\ckGc)$ and decompose
\[
  \ckcO = \ckcO_{e} \cuprow \ckcO_{o}
\]
where $\ckcO_e$ and $\ckcO_o$ are partitions consist of even and odd rows
respectively.

We list some easy facts below:
\[
  \begin{split}
    W &= \sfS_{n} \\
    W_{[\lamck]} & = \sfS_{\abs{\ckcO_{e}}}\times \sfS_{\abs{\ckcO_{o}}} \\
    W_{\lamck} & = \prod_{i\in \bN^{+}}\sfS_{\bfcc_{i}(\ckcO_{e})}\times \prod_{i\in \bN^{+}}\sfS_{\bfcc_{i}(\ckcO_{o})}\\\
    \tau_{\lamck} & := (j_{W_{\lamck}}^{W_{[\lamck]}}\sgn )\otimes \sgn =  \ckcO_{e}^{t}\boxtimes \ckcO_{o}^{t}\\
    \LC_{\lamck} & = \LRC_{\lamck}= \set{\tau_{\lamck}}, \AND \\
    \wttau_{\lamck} & = j_{W_{[\lamck]}}^{W}\tau_{\lamck} = \Spr^{-1}(\cO).
  \end{split}
\]


% Now let $\ckcO\in \Nil(\ckGc)$ and decompose
% \[
%   \ckcO = \ckcO_{e} \cuprow \ckcO_{o}
% \]
% where $\ckcO_e$ and $\ckcO_o$ are partitions consist of even and odd rows
% respectively.

% Every irreducible representation in $\Irr(W)$ is special. For the
% infinitesimal character $\lamck$,
% \[

% \]

% In all the cases, let $\DDD$ denote the dual descent of Young diagrams.
% Suppose $\ckcO$ is a Young diagram, it delete the longest row in $\ckcO$


% Let $\YD$ be the set of Young diagrams viewed as a finite multiset of positive
% integers. The set of nilpotent orbits in $\GL_n(\bC)$ is identified with Young
% diagram of $n$ boxes.



Let $\sfW_n := \sfS_n \ltimes \set{\pm 1}^n$ denote the Weyl group of type $B_n$
or $C_n$. Let $\sgn$ denote the sign representation of the Weyl group.

The group $\sfW_n$ is naturally embedded in $\sfS_{2n}$. For $\sfW_n$, let
$\epsilon$ denote the unique non-trivial character which is trivial on $\sfS_n$,
which is also the restriction of the $\sgn$ of $S_{2n}$ on $W_n$. The following
branching formula is known (see \cite{BV.W}*{Lemma~4.1~(b)})
\begin{equation}
  \Ind_{\sfW_{n}}^{\sfS_{2n}} = \bigoplus_{\substack{\sigma\in \YD_{2n}\\
      \bfcc_{i}(\sigma) \text{ is even } \forall i\in \bN}} \sigma.
\end{equation}

% Let $n_e = \abs{\ckcO_e}, n_o =
% \abs{\ckcO_o}$. % and $\lambda_\ckcO = \half \ckhh$.
% By the formula of $a$-function, one can easily see that The cell in
% $W(\lamck)$ consists of the unique representation
% $J_{W_{\lamck}}^{\Wint{\lamck}} (1)$. Now the $W$-cell
% $(J_{W_{\lamck}}^{W_{[\lamck]}} \sgn)\otimes \sgn$ consists a single
% representation
% \[
%   \tau_{\ckcO} = \ckcO_{e}^{t}\boxtimes \ckcO_{o}^{t}.
% \]
% The representation $j_{W_{\lamck}}^{S_{n}} \tau_{\ckcO}$ corresponds to the
% orbit $\cO= \ckcO^t $ under the Springer correspondence.
\trivial[h]{ WLOG, we assume $\ckcO = \ckcO_o$.

  Let $\sigma\in \widehat{S_n}$. We identify $\sigma$ with a Young diagram. Let
  $c_i = \bfcc_i(\sigma)$. Then $\sigma = J^{S_n}_{W'} \epsilon_{W'}$ where
  $W' = \prod S_{c_i}$ (see Carter's book). This implies Lusztig's a-function
  takes value
  \[
    a(\sigma) = \sum_i c_i(c_i-1) /2
  \]
  Comparing the above with the dimension formula of nilpotent Orbits
  \cite{CM}*{Collary~6.1.4}, we get (for the formula, see Bai ZQ-Xie Xun's paper
  on GK dimension of $SU(p,q)$)
  \[
    \half \dim(\sigma) = \dim(L(\lambda)) = n(n-1)/2 - a(\sigma).
  \]
  Here $\dim(\sigma)$ is the dimension of nilpotent orbit attached to the Young
  diagram of $\sigma$ (it is the Springer correspondence, regular orbit maps to
  trivial representation, note that $a(\triv)=0$), $L(\lambda)$ is any highest
  weight module in the cell of $\sigma$.


  Return to our question, let $S' = \prod_i S_{\bfcc_i(\ckcO)}$. We want to find
  the component $\sigma_0$ in $\Ind_{S'}^{S_n} 1$ whose $a(\sigma_0)$ is
  maximal, i.e. the Young diagram of $\sigma_0$ is minimal.

  By the branching rule, $\sigma \subset \Ind_{S'}^{S_n} 1$ is given by adding
  rows of length $\bfcc_i(\ckcO)$ repeatly (Each time add at most one box in
  each column). Now it is clear that $\sigma_0 = \ckcO^t$ is desired.

  This agrees with the Barbasch-Vogan duality $\dBV$ given by
  \[
    \ckcO \xrightarrow{\Spr}\ckcO \xrightarrow{\otimes \sgn} \ckcO^t \xrightarrow{\Spr} \ckcO^t.
  \]
}

We define the set of painted partitions of type $A$ as the following:
\begin{equation}\label{eq:PP.AR}
  \PP_{A}(\ckcO) = \Set{\uptau:=(\tau, \cP)|
    \begin{array}{l}
      \text{$\tau = \ckcO^{t}$}\\
      \text{$\Im(\cP)\subseteq \set{\bullet,c,d}$}\\
      \text{$\#\set{i|\cP(i,j)=\bullet}$ is even for all $j\in \bN^{+}$}
    \end{array}
  }.
\end{equation}
It is easy to see that
\begin{equation}\label{eq:PPA.count}
  \#(\PP_{A}(\ckcO)) = \prod_{r\in \bN^{+}} (\#\set{i\in \bN^{+}| \bfrr_{i}(\ckcO)=r}+1)
\end{equation}

\begin{lem} \label{lem:GL.count}
  Suppose $G = \GL_{n}(\bR)$. Let
  \[
    \cC_n := \bigoplus_{\substack{s,a,b\in \bN \\2s+a+b=n}} \Ind_{W_s\times S_a\times S_b}^{S_{n}} \sgn \otimes 1\otimes 1.
  \]
  Then, as $W_{[\lamck]}$-module,
  \[
    \Cint{\lamck} \cong \cC_{\abs{\ckcO_{e}}}\otimes \cC_{\abs{\ckcO_{o}}}.
  \]
  Furthermore,
  \begin{equation}\label{eq:A.count}
    [\tau_{\lamck}: \Cint{\ckcO}] = \# (\PP_{A}(\ckcO_{e}))\times
    \# (\PP_{A}(\ckcO_{o})) = \# (\PP_{A}(\ckcO)).
  \end{equation}
  \qed
\end{lem}
\begin{proof}
  We only explain the last equation and leave the rest to the reader.
  The multiplicity formula follows from the Pieri rule and the last equality
  follows from \eqref{eq:PPA.count}.
\end{proof}
% The $\Wint{\lamck}$-module $\Cint{\lamck}$ is given by the following formula:
% According to Vogan duality, we can obtain the above formula by tensoring
% $\sgn$ on the forumla of the unitary groups in \cite{BV.W}*{Section~4}.

\trivial[]{
By branching rules of the symmetric groups, $\Unip_{\ckcO}(G)$ can be
parameterized by painted partition.

For $\uptau:=(\tau,\cP)\in \PP_{A}(\ckcO)$, we write $\cP_{\uptau}:= \cP$.

  The typical diagram of all columns with even length $2c$ are
  \[
    \ytb{\bullet\cdots\bullet\bullet\cdots\bullet,\vdots\vdots\vdots\vdots\vdots\vdots, \bullet\cdots\bullet c\cdots c, \bullet\cdots\bullet d\cdots d }
  \]

  The typical diagram of all columns with odd length $2c+1$ are
  \[
    \ytb{\bullet\cdots\bullet\bullet\cdots\bullet,\vdots\vdots\vdots\vdots\vdots\vdots, \bullet\cdots\bullet \bullet\cdots\bullet , c\cdots c d\cdots d }
  \]
}

Let $\sgn_a\colon \GL_a(\bR)\rightarrow \set{\pm 1}$ be the sign of determinant
and $1_a$ be the trivial representation of $\GL_a(\bR)$.
For
$\uptau\in \PP_{A}{\ckcO}$, we attache the representation
\begin{equation}\label{eq:u.GLR}
  \pi_\uptau :=
  \bigtimes_{j} \underbrace{1_j \times \cdots \times 1_j}_{c_j\text{-terms}}\times
  \underbrace{\sgn_j \times \cdots \times {\sgn_j} }_{d_j\text{-terms}}.
\end{equation}
where
\begin{itemize}
  \item $j$ running over all column lengths in $\ckcO^t$,
  \item $d_j$ is the number of columns of length $j$ ending with the symbol
        ``d'',
  \item $c_j$ is the number of columns of length $j$ ending with the symbol
        ``$\bullet$'' or ``$c$'', and
  \item ``$\times$'' denote the parabolic induction.
\end{itemize}

\begin{thm}[c.f. Vogan]
  The following map is a bijection
  \[
    \begin{array}{ccc}
      \PP_{A}(\ckcO) & \longrightarrow & \Unip_{A}(\ckcO)\\
      \uptau & \mapsto & \pi_{\uptau}.
    \end{array}
  \]
\end{thm}
\begin{proof}
  The irreducibility and unitarity of $\pi_{\uptau}$ see \cite{V.GL}.
  The map is injective since $\pi_{\uptau_{1}}$ and $\pi_{\uptau_{2}}$ have different
  cuspidal data if  $\uptau_{1}\neq \uptau_{2}$.
  The map is bijective by \Cref{lem:GL.count}.
\end{proof}

\subsection{Special Unipotent representations of $G=\GL_{m}(\bH)$}

Retain the definitions in the previous section. Now suppose $G = \GL_{\frac{n}{2}}(\bH)$.

\begin{lem}
  We have
  \[
    \Cint{\lamck}(G)  = \begin{cases}
      \Ind_{\sfW_{\frac{n}{2}}}^{\sfS_{n}}\epsilon & \text{ if
      } \ckcO = \ckcO_{e}\\
      0 & \text{ otherwise
      } \\
    \end{cases}
  \]
  In particular, $\Unip_\ckcO(G) = \emptyset$ if $\ckcO \neq \ckcO_e$.

  When $\ckcO = \ckcO_{e}$,
  \[
    \Unip_{A^{\bH}}(G) = \set{\pi_{\ckcO}}
  \]
  where
  \[
    %\pi_{\ckcO} := \bigtimes_i 1_{\bfrr_i(\ckcO)/2}.
    \pi_{\ckcO} := 1_{\bfrr_1(\ckcO)/2}\times 1_{\bfrr_2(\ckcO)/2} \times \cdots
   \times  1_{\bfrr_k(\ckcO)/2},
  \]
  $k = \bfcc_{1}(\ckcO)$
  and $1_{a}$ denote the trivial representation of $\GL_{a}(\bH)$.
% which is a element in $\Unip_{A^{\bH}}(\ckcO)$

% In this case the coherent continuation representation is given by
% \[
%   \Cint{\lamck}(G) = \Ind_{W_m}^{\sfS_{2n}}\epsilon
% \]
% and $\Unip_\ckcO(G)$ is a singleton. %We use partition $\tau:= \ckcO^t$ to parameter special unipotent representations of $\GL_{m}(\bH)$.
\end{lem}
\begin{proof}
  The $G$-module $\pi_{\ckcO}$ is irreducible, unitary and unipotent by Vogan
  \cite{V.GL}.
  The rest parts is clear by $[\tau_{\ckcO}:\Cint{\lamck}(G)] = 0$ if
  $\ckcO \neq \ckcO_{e}$
  and $[\tau_{\ckcO}:\Cint{\lamck}(G)] = 1$ otherwise.
\end{proof}



\section{Special unipotent representations of $\rU(p,q)$}
\def\nbb{n_{\mathrm b}}
\def\ngg{n_{\mathrm g}}

In this section, we assume $G  = \rU(p,q)$ so that $\star = A^{*}$.
Note that $\ckG = \GL_{n}(\bC)$.

We make the following decomposition:
\[
\ckcO = \ckcO_{g}\cuprow \ckcO_{b}
\]
where every row in $\ckcO_{g}$ has the parity of $\abs{\ckcO}$ (called ``good parity'')
and every row in $\ckcO_{b}$ has the parity different from $\abs{\ckcO}$ (called
``bad parity'').
Let $(\ngg,\nbb) = (\abs{\ckcO_g},\abs{\ckcO_b})$.
Clearly
\[
  \begin{split}
    \WLamck &= \sfS_{\ngg}\times \sfS_{\nbb}\\
    \tau_{\lamck}  & = \ckcO_{g}^{t}\boxtimes \ckcO_{b}^{t}\\
  \end{split}
\]

\begin{lem}
  Let
\end{lem}


Now as the $S_{n_g}\times S_{n_b}$
\[
\bigoplus_{\substack{p,q\in \bN\\p+q=n}} \Cint{\lamck}(\rU(p,q)) = \cC_{g}\otimes \cC_{b}
\]
where
\[
  \begin{split}
 \cC_{g} &= \bigoplus_{\substack{s,a,b\in \bN\\2s+a+b=n_g}} \Ind_{W_{s}\times S_a\times S_b}^{S_{n_g}}
 1\otimes \sgn\otimes \sgn \\
 \cC_{b} &= \begin{cases}
  \Ind_{W_{\frac{n_b}{2}}}^{S_{n_b}} 1 & \text{if $n_b$ is even}\\
  0 & \text{otherwise}.
 \end{cases}
  \end{split}
\]

By the above formula, we have
\begin{lem}
  \begin{enumT}
    \item
The set $\Unip_{\ckcO_b}(\rU(p,q))\neq \emptyset$ if and only if $p=q$ and
each row lenght in $\ckcO$ has even multiplicity.
\item
Suppose $\Unip_{\ckcO_b}(\rU(p,p))\neq \emptyset$, let $\ckcO'$ be the Young diagram
such that $\bfrr_i(\ckcO') = \bfrr_{2i}(\ckcO_b)$ and $\pi'$ be the unique special
uinpotent representation in $\Unip_{\ckcO'}(\GL_{p}(\bC))$.
Then the unique element in $\Unip_{\ckcO_b}(\rU(p,p))$  is given by
\[
  \pi := \Ind_{P}^{\rU(p,p)} \pi'
\]
where $P$ is a parabolic subgroup in $\rU(p,p)$ with Levi factor equals
to $\GL_p(\bC)$.
\item
In general, when $\Unip_{\ckcO_b}(\rU(p,p))\neq \emptyset$, we have a natural bijection
\[
  \begin{array}{rcl}
  \Unip_{\ckcO_g}(\rU(n_1,n_2)) &\longrightarrow& \Unip_{\ckcO}(\rU(n_1+p,n_2+p))\\
  \pi_0 & \mapsto & \Ind_P^{\rU(n_1+p,n_2+p)} \pi'\otimes \pi_0
  \end{array}
\]
where $P$ is a parabolic subgroup with Levi factor $\GL_p(\bC)\times \rU(n_1,n_2)$.
  \end{enumT}
\end{lem}

The above lemma ensure us to reduce the problem to the case when $\ckcO = \ckcO_g$.
Now assume $\ckcO = \ckcO_g$ and so $\Cint{\ckcO}$ corresponds to the blocks of
the infinitesimal character of the trivial representation.

By \cite{BV.W}*{Theorem~4.2},  Harish-Chandra cells in $\Cint{\ckcO}$ are in one-one
correspondence to real nilpotent orbits in $\cO:=\dBV(\ckcO)=\ckcO^t$.

\trivial{
From the branching rule, the cell is parametered by painted partition
\[
\PP{}(\rU):=\set{\uptau\in \PP{}| \begin{array}{l}\Im (\uptau) \subseteq  \set{\bullet, s,r}\\
  \text{``$\bullet$'' occures even times in each row}
\end{array}
  }.
\]

The bijection $\PP{}(\rU)\rightarrow \SYD, \uptau\mapsto \sO$ is given by the following recipe:
The shape of $\sO$ is the same as that of $\uptau$.
$\sO$ is the unique (upto row switching) signed Young diagram such that
\[
  \sO(i,\bfrr_i(\uptau)) := \begin{cases}
    +,  & \text{when }\uptau(i,\bfrr_i(\uptau))=r;\\
    -,  & \text{otherwise, i.e. }\uptau(i,\bfrr_i(\uptau))\in \set{\bullet,s}.
  \end{cases}
\]

\begin{eg}
  \[
 \ytb{\bullet\bullet\bullet\bullet r,\bullet\bullet , sr,s,r}
 \quad
 \mapsto\quad
 \ytb{+-+-+,+-, -+,-,+}
  \]
\end{eg}
}

Now the following lemma is clear.
\begin{lem}
When $\ckcO=\ckcO_g$, the associated varity of every special unipotent representations in $\Unip_\ckcO(\rU)$
is irreducible. Moreover, the following map  is a bijection.
\[
  \begin{array}{rcl}
  \Unip_{\ckcO_g}(\rU(n_1,n_2)) &\longrightarrow& \set{\text{rational forms of $\ckcO^t$}}\\
  \pi_0 & \mapsto & \wAV(\pi_0).
  \end{array}
\]
\qed
\end{lem}
\begin{remark}
  Note that the parabolic induction of an rational nilpotent orbit can be reducible.
  Therefore, when $\ckcO_b\neq \emptyset$, the special unipotent representations can have
  reducible associated variety. Meanwhile, it is easy to see that the map
  $\Unip_{\ckcO}(\rU) \ni \pi \mapsto\wAV(\pi)$ is still injective.
\end{remark}

We will show that every elements in $\Unip_{\ckcO_g}$ can be constructed by iterated theta lifting.
For each $\uptau$, let $\sO$ be the corresponding real nilpotent orbit. Let
$\Sign(\sO)$ be the signature of $\sO$, $\DD(\sO)$ be the signed Young diagram
obtained by deleteing the first column of $\sO$.
Suppose $\sO$ has $k$-columns. Inductively we have a sequence of unitary groups
$\rU(p_i,q_i)$ with $(p_i,q_i) = \Sign(\DD^i(\sO))$ for $i=0, \cdots, k$. Then
\begin{equation}\label{eq:u.U}
  \pi_\tau = \theta^{\rU(p_0,q_0)}_{\rU(p_1,q_1)} \theta^{\rU(p_1,q_1)}_{\rU(p_2,q_2)}\cdots
\theta^{\rU(p_{k-1},q_{k-1})}_{\rU(p_k,q_k)}(1)
\end{equation}
where $1$ is the trivial representation of $\rU(p_k,q_k)$.


Suppose $\ckcO = \ckcO_g$. Form the duality between cells of $\rU(p,q)$ and
$\GL(n,\bR)$. We have an ad-hoc (bijective) duality between unipotent
representations:
\[
  \begin{array}{rcl}
 \dBV\colon \Unip_{\ckcO}(\rU)& \rightarrow &\Unip_{\ckcO^t}(\GL(\bR)) \\
 \pi_\uptau &\mapsto& \pi_{\dBV(\uptau)} \\
  \end{array}
\]

Here $\ckcO^t = \dBV(\ckcO)$ and $\dBV(\uptau)$ is the pained bipartition
obtained by transposeing $\uptau$ and replace $s$ and $r$ by $c$ and $d$
respectively. See \eqref{eq:u.U} and \eqref{eq:u.GLR} for the definition of
special unipotent representations on the two sides.


\end{document}

%%% Local Variables:
%%% mode: latex
%%% TeX-master: "counting_main"
%%% End:
