\documentclass[counting_main.tex]{subfiles}
\begin{document}

\section{Metaplectic Barbasch-Vogan duality and Weyl group representations}

In this section, we exam primitive ideals for $\fgg=\fsp(2n,\bC)$ at half
integral infinitesimal character and explain the metaplectic Barbasch-Vogan dual
in terms of the cells of $\sfW'_{n}$.

Let $\ckcO$ be a metaplectic ``good'' nilpotent orbit in $\fsp(2n,\bC)$, i.e.
$\bfrr_i(\ckcO)$ is even for each $i\in \bN^+$.

%We write $ \bfcc_{2i} (\ckcO)  =: 2 c_i + \epsilon_i =: C_i$ such that $\epsilon_i \in \set{0,1}$.
Then $W_{[\lambda_{\ckcO}]} = W'_n$ and the left cell is given by
\[
(J_{W_\ckcO}^{W'_n} \sgn) \otimes \sgn \quad \text{ with } \quad W_\ckcO = \prod_{i\in \bN^+} S_{\bfcc_{2i}(\ckcO)}.
\]
Here $W_\ckcO$ is an subgroup of $S_n$ and the embedding of $S_n$ in $W'_n$ is fixed.
When $n$ is even,we the symbol of $J_{S_n}^{W'_n} \sgn$ is degenerate and we label it by ``$I$''.

\def\PPtC{\PP{\tC}}
 Let
 \[
  \PPtC(\ckcO) = \set{(2i-1,2i+2)|i\in \bN^+, \bfrr_{2i-1}(\ckcO)\neq \bfrr_{2i}(\ckcO)}.
 \]
  For each $\wp\subset \PPtC(\ckcO)$, let $\tau_\wp := (\imath_\wp,\jmath_\wp)$ be the bipartition
  given by
  \[
    (\bfrr_i(\imath_\wp),\bfrr_i(\jmath_\wp)) =\begin{cases}
      (\frac{\bfrr_{2i-1}(\ckcO)}{2}, \frac{\bfrr_{2i}(\ckcO)}{2}),   & \text{if } (2i-1,2i)\notin \wp, \\
      (\frac{\bfrr_{2i}(\ckcO)}{2}, \frac{\bfrr_{2i-1}(\ckcO)}{2}), & \text{otherwise.}
    \end{cases}
  \]
  If $\tau_\wp$ is degenerate, it represent the element in $\widehat{W'_n}$
  with label $I$.
  Let $\wp^c$ be the complement of $\wp$ in $\PPtC(\ckcO)$.
  Then $\tau_\wp$ and $\tau_{\wp^c}$ represent the same irreducible $W'_n$-module.

  \def\LCC{{}^L\sC'}
Using the induction formula in \cite{L}*{(4.6.6) (4.6.7)}, we have the following lemma.
\begin{lem}
The representation
\[
  \LCC(\ckcO) :=J_{W_\ckcO}^{W'_n} \sgn
\]
is multiplicity free.
The map
\[
  \bZ[\PPtC(\ckcO)]/\sim \longrightarrow \set{\tau\in \widehat{W'_n}| \tau\subset \LCC(\ckcO)}
  \qquad \wp \mapsto \tau_\wp
\]
is a bijection where $\sim$ is the equivalent relation identifying $\wp$ with $\wp^c$.
% irreducible components in $\LCC(\ckcO)$ are in one-one correspond
% to the quotient of  by the equivalent relation $\wp\sim \wp^c$
Moreover, the special representation in $\LCC(\ckcO)$ is $\tau_\emptyset$.
\qed
\end{lem}

In the following, we write
\[
\tau_{\ckcO}:=\tau_\emptyset
\]
for the unique special
representation in $\LCC(\ckcO)$.

\trivial[h]{
 Use the formula, if $\ckcO$ has only two rows $[2r_1,r_2]$.
 Then $W_{\ckcO} = \underbrace{S_2\times \cdots \times S_2}_{r_2} \times
 \underbrace{ S_1 \times \cdots \times S_1}_{r_1-r_2}$ and the claim is clear.

 The corresponding symbol is
 \[
  \binom{r_1}{r_2}
 \]

 We prove by induction on the number of rows.
 Suppose $\ckcO  = [2k+2r_1, 2r+2r_2, \ckcO'] $ where $2k$ is the number of columns in $\ckcO'$.
 If $r_1=r_2 =0$, then
 the symbol of $\ckcO$ are given by
 \[
  \binom{a_1,a_2, \cdots, a_k, 2k+r_1}{b_1, b_2, \cdots, b_k, 2k+r_2 }
  \text{ or }
  \binom{a_1,a_2, \cdots, a_k, 2k+r_2}{b_1, b_2, \cdots, b_k, 2k+r_1 }.
 \]
where
\[
  \binom{a_1,a_2, \cdots, a_k}{b_1, b_2, \cdots, b_k}
\]
is the symbol attached to $\ckcO'$.

This gives the claim.
}


Let $\tau_\ckcO  = (\imath,\jmath)$ such that $\bfrr_i(\imath)\leq \bfrr_i(\jmath)$ for all $i\in \bN^+$.
Then the unique special representation
in $\Ind_{W_\ckcO}^{W'_n} 1$ where the $a$ function take maximal value is given
by the bipartition
\[
  \tau_{\cO}:=(\jmath^t,\imath^t).
\]
Let $\sigma'(\cO)$ denote the $\tau_{\cO}$-isotypic component of $W'_n$-harmonic polynomials in $S(\fhh)$.
Comparing the  fake degree formulas of type $C$ and $D$ (see \cite{Carter}*{Proposition 11.4.3, 11.4.4}), we conclude that
the $W_n$-representation
\begin{equation}\label{eq:sigma.tC}
\sigma(\cO):= \bC[W_n]\cdot \sigma'(\cO)
\end{equation}
is irreducible whose type is also given by the
bipartition $\tau_{\cO}$. The type of $\sigma(\cO)$ is $j_{W'}^{W} \sigma'(\cO)$.

\begin{lem}\label{lem:MD1}
  Suppose $\sigma'$ is a $W'_n$-special representation.% and $\sigma = j_{W'_n}^{W_n}$.
  Let $\ckcO'$ be the partition of type $D$ corresponds to the $W'_n$-special representation
  $\sigma'\otimes \sgn_D$. Then $\sigma:=j_{W'_n}^{W_n} \sigma'$ corresponds to the partition $\ckcO'^t$ under the Springer correspondence of
  type C.
\end{lem}
\begin{proof}
  %By the algorithm computing the Springer correspondence, it is clear that
  First note that $\ckcO'$ is a specail nilpotent orbit of type $D_n$,
  therefore $\ckcO'^t$ is a partition of type $C_n$ (see \cite{CM}*{Proposition~6.3.7}).

  By the explicitly formula of the Lusztig-Spaltenstein duality, we known that the D-collapsing
  $(\ckcO'^t)_D$ of $\ckcO'^t$ corresponds to the special representation
  $\sigma'$ of type $D$.

  The Springer correspondence algorithm for classical groups can be naturally extends to all partitions.
  Sommers showed that two partitions are mapped to the same Weyl group representations
  if and only if they has the same $D$-collapsing \cite{So}*{Lemma~9}.
  Note that the algorithms computing the Springer correspondence for type C and D are essentially the same
  (see \cite{Carter}*{Section~13.3} or \cite{So}*{Section~7}).
  By the injectivity of the Springer correspondence, we conclude that the type $C$ partition $\ckcO'^t$
  must corresponds to $\sigma$.
  \trivial[h]{
    Two partition $\lambda\sim_X \mu$ if they have the same $X$-collapsing.
  Sommers showed that two partition $\lambda$ and $\mu$ maps to the same module $E_\lambda = E_\mu$
  if they have the same X-collapse. But the Springer correspondence is injective,
  so we have the map $\sP/\sim_X \xrightarrow{1-1} \sP_X \hookrightarrow \widehat{W}$
  which gives the claim.
  }
\end{proof}



%As a corrollary, we can obtain the following simple formule for
The following lemma is a direct consequence of \Cref{lem:MD1}.
\begin{lem}
  Suppose $\ckcO$ has good parity of type $\tC$. Under the Springer
  correspondence of type $C$, the representation $\sigma(\ckcO)$ (see
  \eqref{eq:sigma.tC}) corresponds to the nilpotent orbit $\cO$ defined by
  \[
    (\bfcc_{2i-1}(\cO),\bfcc_{2i}(\cO)) =\begin{cases}
      (\bfrr_{2i-1}(\ckcO),\bfrr_{2i}(\ckcO)), & \text{if } \bfrr_{2i-1}(\ckcO),\bfrr_{2i}(\ckcO)\\
      (\bfrr_{2i-1}(\ckcO)-1,\bfrr_{2i}(\ckcO)+1), & \text{otherwise} \\
    \end{cases}
  \]
  for all $i\in \bN^+$.
\end{lem}
\begin{proof}
  Apply the algorithm of Springer correspondence of type D to the representation $\sigma'(\ckcO)$ gives the above formula.
\end{proof}
\trivial{
A technical point, the representation of $W'_n$ is given by symbol $\binom{\xi}{\mu}$ or
$\binom{\mu}{\xi}$. However, using the Springer correspondence formula, only one of the arrangement
can give a valid type D partition.

It is a interesting fact that a type D orbit $\cO$ is special if and only if $\cO^t$ is of type C.



}



\def\tdBV{\tdd_{\text{BV}}}
\begin{lem}
Suppose $\ckcO = \ckcO_{b}\cup \ckcO_{g}$.
Then $\cO = \cO_{b} \cup \cO_{g}$ where $\cO_{b} = \ckcO_{b}^{t}$ and
$\cO_{g}=\tdBV(\ckcO_{g})$.
\end{lem}

% Let $\ckcO'_{g} = (\ckcO_{g})_{D}$. Then $\sigma_{D}(\ckcO_{g}) = \sigma_{D}(\ckcO'_{g})$.
% Let $\sigma_{\ckcO_{g}} = \sigma_{\ckcO'_{g}}$
% Let $\ckcO' = \ckcO_{b}\cup \ckcO'_{g}$.

\trivial[]{
  We take the convention that $2\cO = [2r_{i}]$ if $\cO = [r_{i}]$.
  We also write $[r_{i}]\cup [r_{j}] = [r_{i},r_{j}]$.
  $\dagger \cO = [r_{i}+1]$.

We suppose
\[
\ckcO_{b} = [2r_{1}+1, 2r_{1}+1, \cdots, 2r_{k}+1,2r_{k}+1]
= (2c_{0},2c_{1},2c_{1}, \cdots, 2c_{l}, 2c_{l})
\]
where $l = r_{1}$.

Now
\[
\begin{split}
  W_{\ckcO_{b}} &= W_{c_{0}} \times S_{2c_{1}} \times S_{2c_{2}}\times \cdots \times S_{2c_{l}}\\
  \cksigma_{b} &:= j_{W_{\ckcO_{b}}}^{W_{b}} \sgn = ((c_{1},c_{2},\cdots, c_{k}),(c_{0},c_{1}, \cdots, c_{l}))\\
  & = ([r_{1},r_{2},\cdots, r_{k}],[r_{1}+1,r_{2}+1,\cdots,r_{k}+1])
\end{split}
\]
Therefore
\[
\sigma_{b} = \cksigma_{b}\otimes \sgn = ((r_{1}+1,r_{2}+1,\cdots,r_{k}+1),(r_{1},r_{2},\cdots, r_{k}))
\]
which corresponds to the orbit
\[
  \cO_{b} = (2r_{1}+1, 2r_{1}+1,2r_{2}+1, 2r_{2}+1,  \cdots,2r_{k}+1, 2r_{k}+1 ) = \ckcO_{b}^{t}.
\]
This implies
\[
  \sigma_{b} = j_{W_{L_{b}}}^{W_{b}}\sgn, \quad \text{where } W_{L,b} = \prod_{i=1}^{k} S_{2r_{i}+1}.
\]
(Note that $\cO'_{b} = (2r_{1}+1,2r_{2}+1, \cdots, 2r_{k}+1)$ which corresponds
to $j_{W_{L_{b}}}^{S_{b}}\sgn$ and $\ind_{L}^{G} \cO'_{b} = \cO_{b}$.
)

Now we deduce that
\[
  \begin{split}
    \sigma &:= j_{W_{b}\times W_{g}}^{W_{n}} \sigma_{b}\otimes \sigma_{g}\\
    & = j_{W_{L_{b}}\times W_{g}}^{W_{n}} \sgn \otimes \sigma_{g}
  \end{split}
\]
where $W_{L_{b}}\times W_{g}$ is a parabolic subgroup of $W_{n}$
corresponds to the Levi factor $L$ of type
\[
A_{2r_{1}+1}\times A_{2r_{2}+1}\times \cdots \times A_{2r_{k}+1} \times W_{g}.
\]
Therefore
\[
\cO = \ind_{L}^{G} \triv\times \cO_{g} = \cO_{b}\cup \cO_{g}.
\]


% Note that
% \[
%   j_{S_{b}}^{W_{b}} 1
%   = (j_{S_{b}}^{W_{b}}\sgn)\otimes \sgn
%   = (J_{W_{\ckcO_{b}}}^{W_{b}} \sgn)\otimes
%   \sgn
% \]

% It suffice to compute
% \[
%   (J_{W_{\ckcO_{b}}\times W_{\ckcO_{g}}}^{\WLam} \sgn)\otimes \sgn
%   = (J_{W_{b}\times W_{g}}^{W} \sigma_{b} \otimes \sigma_{g}) \otimes \sgn
% \]
%
\def\ckfll{\check\fll}
\def\ckfgg{\check\fgg}

We claim that the map $\tdd\colon \ckcO \mapsto \cO$ defined here
coincide with our Metaplectic BV duality paper.

Note that $\tdBV$ is compatible with parabolic induction.
Suppose $\ckfll$ is a Levi subgroup of $\ckfgg$ and
$\ckcO_{\ckfll}:=\ckcO\cap \ckfll\neq \emptyset$.
Then
\[
\tdBV(\ckcO) = \ind_{\fll}^{\fgg} \tdBV(\ckcO_{\ckfll}).
\]
(Since $\tdBV$ commute with the descent map, the claim follows from the
prosperity of $\dBV$)

The map $\tdd$ also compatible with parabolic induction.
It suffice to consider the case where $\ckfll$ is a maximal parabolic of type
$A_{l}\times C_{n}$ and
the orbit is trivial on the $A_{l}$ factor.
Suppose $l$ is the bad parity, the claim is clear by our computation for the bad
parity case.

Now suppose $l=2m$ has good parity. If there is a $i$ such that
$\bfrr_{2i+1}(\ckcO)=\bfrr_{2i+2}(\ckcO)=2m$. Then
$\bfcc_{2i+1}(\cO)=\bfcc_{2i+2}(\cO)=2m$ and the claim follows.

Otherwise, we can assume
\[
R_{2i-1}:=\bfrr_{2i-1}(\ckcO)> \bfrr_{2i}(\ckcO)=\bfrr_{2i+1}(\ckcO) > \bfrr_{2i+2}(\ckcO)=:R_{2i+2}.
\]
and then
\[
\cO = (\cdots, R_{2i-1}-1, 2m+1, 2m-1, R_{2i+2}, \cdots)
\]
One check again that $\cO = \ind_{\fll}^{\fgg}\cO_{\fll}$.
(Note that the induction operation is add two length $2m$ columns and then apply
C-collapsing.)

Now it suffice to check that
$\tdd(\ckcO) = \tdBV(\ckcO)$ for every orbits $\ckcO$ whose
rows are multiplicity free) (i.e. $\ckcO$ is distinguished).
This is clear by the explicit formula for the both sides.
}

\section{Matching specail shape and non-special shape painted bipartitions}
\label{app:comb}

\subsection{Proof of ...}
\begin{proof}
  Let $i$ be the minimal integer such that $(2i-1, 2i)\in \sP$.
  We define $\sP' = \sP - \set{(2i-1,2i)}$.
  We will establish a bijection
  \[
    \PBP_{\star}(\ckcO_{g}, \sP')\longrightarrow \PBP_{\star}(\ckcO_{g},\sP).
  \]
\end{proof}

% \subsection{Embedding Harish-Chandra cells to cells in $\sO$}
% Let $H$ be a Cartan subgroup of $G$ and $T$ is the maximal compact subgroup of
% $H$. %Let $(\fgg, T)$-module.
% Let $\cO_{H}$ be the category

\end{document}
